% -*- coding: utf-8 -*-
% !TEX program = lualatex
\documentclass[oneside]{book}

% -*- coding: utf-8 -*-
% !TEX program = lualatex

\usepackage[heading=true,scheme=chinese]{ctex}
\ctexset{
  figurename = 图,
  tablename  = 表,
}

\usepackage[a4paper,margin=2.5cm]{geometry}

\usepackage{codehigh} % https://ctan.org/pkg/codehigh
\usepackage{tabularray}
\usepackage{array,multirow,amsmath}
\usepackage{chemmacros,environ}

\UseTblrLibrary{booktabs,diagbox,siunitx}

\usepackage{hyperref}
\hypersetup{
  colorlinks=true,
  urlcolor=blue3,
  linkcolor=red3,
}

\usepackage{tcolorbox}
\tcbset{sharp corners, boxrule=0.5pt, colback=red9}

\usepackage{float}

\setcounter{tocdepth}{1}

\newcommand*{\K}[1]{\texttt{#1}}
\newcommand*{\V}[1]{\texttt{#1}}

\DefTblrTemplate{caption-tag}{default}{表\hspace{0.25em}\thetable}
% \DefTblrTemplate{caption-sep}{default}{:\enskip}
\DefTblrTemplate{contfoot-text}{default}{续下页}
\DefTblrTemplate{conthead-text}{default}{(接前页)}
% \DefTblrTemplate{caption-text}{default}{\InsertTblrText{caption}}

\NewTblrEnviron{newtblr}
\SetTblrOuter[newtblr]{long}
\SetTblrInner[newtblr]{
  hlines = {white}, column{1,2} = {co=1}, colsep = 5pt,
  row{odd} = {brown8}, row{even} = {gray8},
  row{1} = {fg=white, bg=purple2, font=\bfseries\sffamily},
}

\NewTblrEnviron{spectblr}
\SetTblrOuter[spectblr]{long}
\SetTblrInner[spectblr]{
  hlines = {white}, column{2} = {co=1}, colsep = 5pt,
  row{odd} = {brown8}, row{even} = {gray8},
  row{1} = {fg=white, bg=purple2, font=\bfseries\sffamily},
  rowhead = 1,
}

% \newcommand{\mywarning}[1]{%
%   \begin{tcolorbox}
%   The interfaces in this #1 should be seen as
%   \textcolor{red3}{\bfseries experimental}
%   and are likely to change in future releases, if necessary.
%   Don’t use them in important documents.
%   \end{tcolorbox}
% }
\newcommand{\mywarning}[1]{%
  \begin{tcolorbox}
   #1 中说明的接口是
  \textcolor{red3}{\bfseries 实验性的}
  并且在未来可能会发生变化,
  因此,请勿在重要的文档中使用它们。
  \end{tcolorbox}
}

\renewcommand*{\thefootnote}{*}

\newcommand*{\myversion}{2021L}
\newcommand*{\mydate}{Version \myversion\ (\the\year-\mylpad\month-\mylpad\day)\\\myrepo}
\newcommand*{\myrepo}{\url{https://github.com/lvjr/tabularray}}
\newcommand*{\mylpad}[1]{\ifnum#1<10 0\the#1\else\the#1\fi}

\colorlet{highback}{\ifodd\month azure9\else blue9\fi}
\CodeHigh{language=latex/table,style/main=highback,style/code=highback}
\NewCodeHighEnv{code}{style/main=gray9,style/code=gray9}
\NewCodeHighEnv{demo}{style/main=gray9,style/code=gray9,demo}

%\CodeHigh{lite}

\CodeHigh{lite}
\setcounter{chapter}{3}

\begin{document}

% \chapter{Use Long Tables}
\chapter{长表格}
\label{chap:long}

\mywarning{本章}

% \section{A Simple Example}
\section{简单示例}

% In fact, to make a decent long table with header and footer, it is better to separate header/footer as
% \underline{table head/foot} (which includes caption, footnotes, continuation text)
% and \underline{row head/foot} (which includes some rows of the table that should appear in every page).
% By this approach, alternating row colors should work as expected.
事实上,要制作一个有表头和表尾的长表格时,最好将表头/表尾分开设计为
\underline{题注/尾注}(包括标题、续表文本、表注、说明、续表文本)和
\underline{标题行/尾行}(每一页都重复出现的行)。
通过这种办法,交替使用有颜色的行就可以达到预期效果。

\NewTblrTheme{fancy}{
  \SetTblrStyle{firsthead}{font=\bfseries}
  \SetTblrStyle{firstfoot}{fg=blue2}
  \SetTblrStyle{middlefoot}{\itshape}
  \SetTblrStyle{caption-tag}{red2}
}
% \begin{longtblr}[
%   theme = fancy,
%   caption = {A Long Long Long Long Long Long Long Table},
%   entry = {Short Caption},
%   label = {tblr:test},
%   note{a} = {It is the first footnote.},
%   note{$\dag$} = {It is the second long long long long long long footnote.},
%   remark{Note} = {Some general note. Some general note. Some general note.},
%   remark{Source} = {Made up by myself. Made up by myself. Made up by myself.},
% ]{
\begin{longtblr}[
  theme = fancy,
  caption = {一个长长长长长长长长长的表格},
  entry = {短标题},
  label = {tblr:test},
  note{a} = {第一个表注。},
  note{$\dag$} = {每二个长长长长长长长的表注。},
  remark{注意} = {一些常规说明,一些常规说明,一些常规说明。},
  remark{来源} = {自力更生,自力更生,自力更生。},
]{
  colspec = {XXX}, width = 0.85\linewidth,
  rowhead = 2, rowfoot = 1,
  row{odd} = {gray9}, row{even} = {brown9},
  row{1-2} = {purple7}, row{Z} = {blue7},
}
\hline
 Head    & Head  & Head    \\
\hline
 Head    & Head  & Head    \\
\hline
 Alpha   & Beta  & Gamma   \\
\hline
 Epsilon & Zeta\TblrNote{a}       & Eta    \\
\hline
 Iota    & Kappa\TblrNote{$\dag$} & Lambda \\
\hline
 Nu      & Xi    & Omicron \\
\hline
 Rho     & Sigma & Tau     \\
\hline
 Phi     & Chi   & Psi     \\
\hline
 Alpha   & Beta  & Gamma   \\
\hline
 Epsilon & Zeta  & Eta     \\
\hline
 Iota    & Kappa & Lambda  \\
\hline
 Nu      & Xi    & Omicron \\
\hline
 Rho     & Sigma & Tau     \\
\hline
 Phi     & Chi   & Psi     \\
\hline
 Alpha   & Beta  & Gamma   \\
\hline
 Epsilon & Zeta  & Eta     \\
\hline
 Iota    & Kappa & Lambda  \\
\hline
 Nu      & Xi    & Omicron \\
\hline
 Rho     & Sigma & Tau     \\
\hline
 Phi     & Chi   & Psi     \\
\hline
 Alpha   & Beta  & Gamma   \\
\hline
 Epsilon & Zeta  & Eta     \\
\hline
 Iota    & Kappa & Lambda  \\
\hline
 Nu      & Xi    & Omicron \\
\hline
 Rho     & Sigma & Tau     \\
\hline
 Phi     & Chi   & Psi     \\
\hline
 Alpha   & Beta  & Gamma   \\
\hline
 Epsilon & Zeta  & Eta     \\
\hline
 Iota    & Kappa & Lambda  \\
\hline
 Nu      & Xi    & Omicron \\
\hline
 Rho     & Sigma & Tau     \\
\hline
 Phi     & Chi   & Psi     \\
\hline
 Alpha   & Beta  & Gamma   \\
\hline
 Epsilon & Zeta  & Eta     \\
\hline
 Iota    & Kappa & Lambda  \\
\hline
 Nu      & Xi    & Omicron \\
\hline
 Rho     & Sigma & Tau     \\
\hline
 Phi     & Chi   & Psi     \\
\hline
 Alpha   & Beta  & Gamma   \\
\hline
 Epsilon & Zeta  & Eta     \\
\hline
 Iota    & Kappa & Lambda  \\
\hline
 Nu      & Xi    & Omicron \\
\hline
 Rho     & Sigma & Tau     \\
\hline
 Phi     & Chi   & Psi     \\
\hline
 Alpha   & Beta  & Gamma   \\
\hline
 Epsilon & Zeta  & Eta     \\
\hline
 Iota    & Kappa & Lambda  \\
\hline
 Nu      & Xi    & Omicron \\
\hline
 Rho     & Sigma & Tau     \\
\hline
 Phi     & Chi   & Psi     \\
\hline
 Alpha   & Beta  & Gamma   \\
\hline
 Epsilon & Zeta  & Eta     \\
\hline
 Iota    & Kappa & Lambda  \\
\hline
 Nu      & Xi    & Omicron \\
\hline
 Rho     & Sigma & Tau     \\
\hline
 Phi     & Chi   & Psi     \\
\hline
 Alpha   & Beta  & Gamma   \\
\hline
 Epsilon & Zeta  & Eta     \\
\hline
 Iota    & Kappa & Lambda  \\
\hline
 Nu      & Xi    & Omicron \\
\hline
 Rho     & Sigma & Tau     \\
\hline
 Phi     & Chi   & Psi     \\
\hline
 Alpha   & Beta  & Gamma   \\
\hline
 Epsilon & Zeta  & Eta     \\
\hline
 Iota    & Kappa & Lambda  \\
\hline
 Nu      & Xi    & Omicron \\
\hline
 Rho     & Sigma & Tau     \\
\hline
 Phi     & Chi   & Psi     \\
\hline
Alpha   & Beta  & Gamma   \\
\hline
 Epsilon & Zeta  & Eta     \\
\hline
 Iota    & Kappa & Lambda  \\
\hline
 Nu      & Xi    & Omicron \\
\hline
 Rho     & Sigma & Tau     \\
\hline
 Phi     & Chi   & Psi     \\
\hline
 Alpha   & Beta  & Gamma   \\
\hline
 Epsilon & Zeta  & Eta     \\
\hline
 Iota    & Kappa & Lambda  \\
\hline
 Nu      & Xi    & Omicron \\
\hline
 Rho     & Sigma & Tau     \\
\hline
 Phi     & Chi   & Psi     \\
\hline
 Alpha   & Beta  & Gamma   \\
\hline
 Epsilon & Zeta  & Eta     \\
\hline
 Iota    & Kappa & Lambda  \\
\hline
 Nu      & Xi    & Omicron \\
\hline
 Rho     & Sigma & Tau     \\
\hline
 Phi     & Chi   & Psi     \\
%\hline
% Alpha   & Beta  & Gamma   \\
%\hline
% Epsilon & Zeta  & Eta     \\
%\hline
% Iota    & Kappa & Lambda  \\
%\hline
% Nu      & Xi    & Omicron \\
%\hline
% Rho     & Sigma & Tau     \\
%\hline
% Phi     & Chi   & Psi     \\
\hline
 Foot    & Foot  & Foot    \\
\hline
\end{longtblr}

% As you can see in the above example, the appearance of long tables of \verb!tabularray! package
% is similar to that of \verb!threeparttable! and \verb!threeparttablex! packages.
% We support table footnotes, but not page footnotes in \verb!tabularray! package.
正如在上面例子中看到的一样,\verb!Tabularray!宏包中的长表格
与\verb!threeparttable! 和 \verb!threeparttablex!宏包的表格类似,
\verb!Tabularray! 支持表格尾注,但表格尾注不在页脚排版。

\newpage

% The source code for the above long table is shown below. It is mainly self-explanatory.
上述表格的源码如下,该源码应该有足够的自明性。

\begin{codehigh}
\NewTblrTheme{fancy}{
  \SetTblrStyle{firsthead}{font=\bfseries}
  \SetTblrStyle{firstfoot}{fg=blue2}
  \SetTblrStyle{middlefoot}{\itshape}
  \SetTblrStyle{caption-tag}{red2}
}
\begin{longtblr}[
  theme = fancy,
  caption = {一个长长长长长长长长长的表格},
  entry = {短标题},
  label = {tblr:test},
  note{a} = {第一个表注。},
  note{$\dag$} = {每二个长长长长长长长的表注。},
  remark{注意} = {一些常规说明,一些常规说明,一些常规说明。},
  remark{来源} = {自力更生,自力更生,自力更生。},
]{
  colspec = {XXX}, width = 0.85\linewidth,
  rowhead = 2, rowfoot = 1,
  row{odd} = {gray9}, row{even} = {brown9},
  row{1-2} = {purple7}, row{Z} = {blue7},
}
\hline
 Head    & Head  & Head    \\
\hline
 Head    & Head  & Head    \\
\hline
 Alpha   & Beta  & Gamma   \\
\hline
 Epsilon & Zeta\TblrNote{a}       & Eta    \\
\hline
 Iota    & Kappa\TblrNote{$\dag$} & Lambda \\
\hline
 Nu      & Xi    & Omicron \\
\hline
 Rho     & Sigma & Tau     \\
\hline
......
\hline
 Nu      & Xi    & Omicron \\
\hline
 Rho     & Sigma & Tau     \\
\hline
 Phi     & Chi   & Psi     \\
\hline
 Foot    & Foot  & Foot    \\
\hline
\end{longtblr}
\end{codehigh}

% As you can see in the above code, we typeset long tables with \verb!longtblr! environemnt.
% And we can totally separate contents and styles of long tables with \verb!tabularray! package.
在\verb!Tabularray!宏包中,使用 \verb!longtblr! 环境排版长表格,
并完全实现了表格样式与内容的分离。

% Row head and row foot consist of some lines of the table and should appear in every page.
% Their options are inner specifications and should be put in the mandatory argument of the \verb!longtblr! environment.
% In the above example, We set \verb!rowhead=2! and \verb!rowfoot=1!.
标题行和尾行由表格的指定的行构成,并且标题行和尾行会出现在每页中。
它们由内部参数设置,应该置于 \verb!longtblr! 环境的必选参数中。
例如,在上面的代码中,设置了 \verb!rowhead=2! 和 \verb!rowfoot=1!。

% \begin{spectblr}[
%   caption = {Inner Specifications for Row Heads and Row Foots}
% ]{}
%   Key Nama    & Key Description & Initial Value \\
%   \V{rowhead} & number of the first rows of the table appear in every page & \V{0} \\
%   \V{rowfoot} & number of the last rows of the table appear in every page  & \V{0} \\
% \end{spectblr}
\begin{spectblr}[
  caption = {头行和尾行的内部参数}
]{}
  键    & 含义 & 初始值 \\
  \V{rowhead} & 每页要显示的标题行行数,从表格的起始行开始向后计数 & \V{0} \\
  \V{rowfoot} & 每页要显示的尾行行数,从表格最后一行开始向前计数  & \V{0} \\
\end{spectblr}

% Table head and table foot consist of the caption, continuation text, footnotes and remarks.
% Their options are outer specifications and should be put in the optional argument of the \verb!longtblr! environment.
表格题注和尾注由标题、表注、说明和续页文本构成。
它们由外部参数设置,应该置于 \verb!longtblr! 环境的可选参数中。


% \begin{spectblr}[
%   caption = {Outer Specifications for Table Heads and Table Foots}
% ]{verb}
%   Key Nama            & Key Description & Initial Value \\
%   \V{headsep}         & vertical space between table head and table body & \V{6pt} \\
%   \V{footsep}         & vertical space between table foot and table body & \V{6pt} \\
%   \V{presep}          & vertical space between table head and the above text & \verb!1.5\bigskipamount! \\
%   \V{postsep}         & vertical space between table foot and the below text & \verb!1.5\bigskipamount! \\
%   \V{theme}           & table theme (including settings for templates and styles) & None \\
%   \V{caption}         & table caption & None \\
%   \V{entry}           & short table caption to be put in List of Tables & None \\
%   \V{label}           & table label & None \\
%   \V{note\{<name>\}}   & table note with \V{<name>} as tag & None \\
%   \V{remark\{<name>\}} & table remark with \V{<name>} as tag & None \\
% \end{spectblr}
\begin{spectblr}[
  caption = {表格题注和尾注的外部参数}
]{verb}
  键            & 含义 & 初始值 \\
  \V{headsep}         & 表格题注与表格之间的垂直间距 & \V{6pt} \\
  \V{footsep}         & 表格尾注与表格之间的垂直间距 & \V{6pt} \\
  \V{presep}          & 表格题注与其之前文本之间的垂直间距 & \verb!1.5\bigskipamount! \\
  \V{postsep}         & 表格尾注与其之后文本之间的垂直间距 & \verb!1.5\bigskipamount! \\
  \V{theme}           & 表格主题 (包括模板及样式设置) & 无 \\
  \V{caption}         & 表格标题 & 无 \\
  \V{entry}           & 用于目录的表格短标题 & 无 \\
  \V{label}           & 表格标签 & 无 \\
  \V{note\{<name>\}}   & 表格表注,其中 \V{<name>} 是表注标签 & 无 \\
  \V{remark\{<name>\}} & 表格说明,其中 \V{<name>} 是说明标签 & 无 \\
\end{spectblr}

% \section{Customize Templates}
\section{个性化模板}

\subsection{模板概述}

% The template system for table heads and table foots in \verb!tabularray! is largely inspired by \verb!beamer!,
% \verb!caption! and \verb!longtable! packages. You can use \verb!\DefTblrTemplate! command to define and
% modify templates, and \verb!\SetTblrTemplate! command to choose default templates. In defining templates,
% you can include other templates with \verb!\UseTblrTemplate! and \verb!\ExpTblrTemplate! commands.
\verb!Tabularry!表格的题注和尾注模板系统的设计主要受\verb!beamer!、\verb!caption! 和 \verb!longtable!
宏包的启发。可以使用 \verb!\DefTblrTemplate! 命令定义或修改一个模板,
使用 \verb!\SetTblrTemplate! 命令选择默认模板。
在定义模板时,可以用 \verb!\UseTblrTemplate! 和 \verb!\ExpTblrTemplate! 命令引入其它模板。

% \begin{spectblr}[
%   caption = {Templates for Table Heads and Table Foots}
% ]{}
%   Template Name    & Template Description \\
%   \V{contfoot-text}& continuation text in the foot, normally ``Continued on next page'' \\
%   \V{contfoot}     & continuation paragraph in the foot, normally including \V{contfoot-text} template \\
%   \V{conthead-text}& continuation text in the head, normally ``(Continued)'' \\
%   \V{conthead}     & continuation paragraph in the head, normally including \V{conthead-text} template \\
%   \V{caption-tag}  & caption tag, normally like ``Table 4.2'' \\
%   \V{caption-sep}  & caption separator, normally like ``:\quad'' \\
%   \V{caption-text} & caption text, normally using user provided value \\
%   \V{caption}      & including \V{caption-tag} + \V{caption-sep} + \V{caption-text} \\
%   \V{note-tag}     & note tag, normally using user provided value \\
%   \V{note-sep}     & note separator, normally like ``\enskip'' \\
%   \V{note-text}    & note tag, normally using user provided value \\
%   \V{note}         & including \V{note-tag} + \V{note-sep} + \V{note-text} \\
%   \V{remark-tag}   & remark tag, normally using user provided value \\
%   \V{remark-sep}   & remark separator, normally like ``:\enskip'' \\
%   \V{remark-text}  & remark text, normally using user provided value\\
%   \V{remark}       & including \V{remark-tag} + \V{remark-sep} + \V{remark-text} \\
%   \V{firsthead}    & table head on the first page, normally including \V{caption} template \\
%   \V{middlehead}   & table head on middle pages, normally including \V{caption} and \V{conthead} templates \\
%   \V{lasthead}     & table head on the last page, normally including \V{caption} and \V{conthead} templates \\
%   \V{head}         & setting all of \V{firsthead}, \V{middlehead} and \V{lasthead} \\
%   \V{firstfoot}    & table foot on the first page, normally including \V{contfoot} template \\
%   \V{middlefoot}   & table foot on middle pages, normally including \V{contfoot} template \\
%   \V{lastfoot}     & table foot on the last page, normally including \V{note} and \V{remark} templates \\
%   \V{foot}         & setting all of \V{firstfoot}, \V{middlefoot} and \V{lastfoot} \\
% \end{spectblr}
\begin{spectblr}[
  caption = {题注和尾注模板}
]{}
  模板名称    & 模板说明 \\
  \V{contfoot-text}& 表格在每页尾部的续表文本,一般是 ``Continued on next page'' \\
  \V{contfoot}     & 表格在每页尾部的续表段落, 一般包括 \V{contfoot-text} 模板 \\
  \V{conthead-text}& 表格在每页标题中的续表文本, 一般是 ``(Continued)'' \\
  \V{conthead}     & 表格在每页标题中的续表段落, 一般包括 \V{conthead-text} 模板 \\
  \V{caption-tag}  & 标题标签, 一般类似 ``Table 4.2'' \\
  \V{caption-sep}  & 标题分隔符, 一般类似 ``:\quad'' \\
  \V{caption-text} & 标题文本, 一般由用户提供内容 \\
  \V{caption}      & 包括 \V{caption-tag} + \V{caption-sep} + \V{caption-text}的组合 \\
  \V{note-tag}     & 尾注标签, 一般由用户提供 \\
  \V{note-sep}     & 尾注分隔符, 一般类似 ``\enskip'' \\
  \V{note-text}    & 尾注内容,一般由用户提供 \\
  \V{note}         & 包括 \V{note-tag} + \V{note-sep} + \V{note-text}的组合 \\
  \V{remark-tag}   & 注释标签, 一般由用户提供 \\
  \V{remark-sep}   & 注释分隔符, 一般类似 ``:\enskip'' \\
  \V{remark-text}  & 注释文本, 一般由用户提供\\
  \V{remark}       & 包括 \V{remark-tag} + \V{remark-sep} + \V{remark-text}的组合 \\
  \V{firsthead}    & 第一页的表头, 一般包括 \V{caption} 模板 \\
  \V{middlehead}   & 中间页的表头,一般包括 \V{caption} 和 \V{conthead} 模板 \\
  \V{lasthead}     & 最后一页的表头,一般包括 \V{caption} 和 \V{conthead} 模板 \\
  \V{head}         & \V{firsthead}、 \V{middlehead} 和 \V{lasthead}的所有设置 \\
  \V{firstfoot}    & 第一页的表尾, 一般包括 \V{contfoot} 模板 \\
  \V{middlefoot}   & 中间页的表尾,一般包括 \V{contfoot} 模板 \\
  \V{lastfoot}     & 最后一页的表尾, 一般包括 \V{note} 和 \V{remark} 模板 \\
  \V{foot}         & \V{firstfoot}、\V{middlefoot} 和 \V{lastfoot}的所有设置 \\
\end{spectblr}

% A template which only includes short text is called a \underline{sub template}.
% Normally there is one \verb!-! in the name of a sub template.
% A template which includes one or more paragraphs is called a \underline{main template}.
% Normally there isn't any \verb!-! in the name of a main template.
仅包含短文本的模板称为\underline{子模板}。
一般在子模板的名称中有一个\verb!-!符号。
包含一个或多个段落的模板称为\underline{主模板}.
通常,在主模板的名称中不包含\verb!-!符号。

% For each of the above elements, two templates \verb!normal! and \verb!empty! are always defined.
% You can select one of them with \verb!\SetTblrTemplate! command.
除了上述模板,总是存在\verb!normal! 和 \verb!empty! 这两个模板。
可以使用 \verb!\SetTblrTemplate! 命令选择其中任意一个。

\subsection{续表模板}

% Let us have a look at the code for defining templates of continuation text first:
首先是续表文本的模板定义:

\begin{codehigh}
\DefTblrTemplate{contfoot-text}{normal}{Continued on next page}
\SetTblrTemplate{contfoot-text}{normal}
\DefTblrTemplate{conthead-text}{normal}{(Continued)}
\SetTblrTemplate{conthead-text}{normal}
\end{codehigh}

% In the above code, command \verb!\DefTblrTemplate! defines the templates with name \verb!normal!,
% and then command \verb!\SetTblrTemplate! sets the templates with name \verb!normal! as default.
% The \verb!normal! template is always defined and set as default for any template element in \verb!tabularray!.
% Therefore you had better use another name when defining new templates.
在以上代码中, \verb!\DefTblrTemplate! 命令定义了名为 \verb!normal! 的模板,
然后用 \verb!\SetTblrTemplate! 命令将名为 \verb!normal! 的模板设置为默认模板。
\verb!normal! 模板总是被定义的,并且会被 \verb!Tabularray! 设置为任何一个模板的默认模板。
因此,在定义一个新模板时,需要将其命名为其它名称。

% If you use \verb!default! as template name in \verb!\DefTblrTemplate!,
% you define and set it as default at the same time.
% Therefore the above code can be written in another way:
如果在\verb!\DefTblrTemplate!命令中使用 \verb!default! 作为模板名称,
则会在定义时同时将其设置为默认模板。
因此,上述代码也可以修改为:

\begin{codehigh}
\DefTblrTemplate{contfoot-text}{default}{Continued on next page}
\DefTblrTemplate{conthead-text}{default}{(Continued)}
\end{codehigh}

% You may modify the code to customize continuation text to fit your needs.
可以通过修改这些代码以适应自己的需求。

% The \verb!contfoot! and \verb!conthead! templates normally
% include their sub templates with \verb!\UseTblrTemplate! commands.
% But you can also handle user settings such as horizontal alignment here.
\verb!contfoot! 和 \verb!conthead! 模板使用 \verb!\UseTblrTemplate! 命令使用其子模板。
但是,可以使用诸如水平对齐等设置参数。

\begin{codehigh}
\DefTblrTemplate{contfoot}{default}{\UseTblrTemplate{contfoot-text}{default}}
\DefTblrTemplate{conthead}{default}{\UseTblrTemplate{conthead-text}{default}}
\end{codehigh}

\subsection{标题模板}

% Normally a caption consists of three parts, and their templates are defined with the follow code:
通常,标题由三部分构成,可以用如下代码定义其模板:

\begin{codehigh}
\DefTblrTemplate{caption-tag}{default}{Table\hspace{0.25em}\thetable}
\DefTblrTemplate{caption-sep}{default}{:\enskip}
\DefTblrTemplate{caption-text}{default}{\InsertTblrText{caption}}
\end{codehigh}

% The command \verb!\InsertTblrText{caption}! inserts the value of \verb!caption! key,
% which you could write in the optional argument of \verb!longtblr! environment.
\verb!\InsertTblrText{caption}! 命令用于为 \verb!caption! 键插入标题内容,
在\verb!longtblr!环境的可选参数中,也可以写入标题内容。

% The \verb!caption! template normally includes three sub templates with \verb!\UseTblrTemplate! commands:
% The \verb!caption! template will be used in \verb!firsthead! template.
\verb!caption! 模板通常用 \verb!\UseTblrTemplate! 插入三个子模板:
\verb!caption! 模板将被用于 \verb!firsthead! 模板。

\begin{codehigh}
\DefTblrTemplate{caption}{default}{
  \UseTblrTemplate{caption-tag}{default}
  \UseTblrTemplate{caption-sep}{default}
  \UseTblrTemplate{caption-text}{default}
}
\end{codehigh}

% Furthermore \verb!capcont! template includes \verb!conthead! template as well.
% The \verb!capcont! template will be used in \verb!middlehead! and \verb!lasthead! templates.
当然,\verb!capcont! 模板也包含 \verb!conthead! 模板。
\verb!capcont! 模板也被用于 \verb!middlehead! 和 \verb!lasthead! 模板。

\begin{codehigh}
\DefTblrTemplate{capcont}{default}{
  \UseTblrTemplate{caption-tag}{default}
  \UseTblrTemplate{caption-sep}{default}
  \UseTblrTemplate{caption-text}{default}
  \UseTblrTemplate{conthead}{default}
}
\end{codehigh}

% \subsection{Note and Remark Templates}
\subsection{表注和说明模板}

% The templates for table notes can be defined like this:
表注模板可以按如下方式定义:

\begin{codehigh}
\DefTblrTemplate{note-tag}{default}{\textsuperscript{\InsertTblrNoteTag}}
\DefTblrTemplate{note-sep}{default}{\space}
\DefTblrTemplate{note-text}{default}{\InsertTblrNoteText}
\end{codehigh}
\begin{codehigh}
\DefTblrTemplate{note}{default}{
  \MapTblrNotes{
    \noindent
    \UseTblrTemplate{note-tag}{default}
    \UseTblrTemplate{note-sep}{default}
    \UseTblrTemplate{note-text}{default}
    \par
  }
}
\end{codehigh}

% The \verb!\MapTblrNotes! command loops for all table notes,
% which are written in the optional argument of \verb!longtblr! environment.
% Inside the loop, you can use \verb!\InsertTblrNoteTag! and \verb!\InsertTblrNoteText!
% commands to insert current note tag and note text, respectively.
\verb!\MapTblrNotes! 命令用于遍历所有表格的表注,
这些表注在 \verb!longtblr! 环境的可选参数中设置。
在遍历中,可以分别通过 \verb!\InsertTblrNoteTag! 命令 \verb!\InsertTblrNoteText!
插入当前标签和表注文本。

% The definition of remark templates are similar to note templates.
注释模板的定义与表注模板的定义类似。
\nopagebreak
\begin{codehigh}
\DefTblrTemplate{remark-tag}{default}{\InsertTblrRemarkTag}
\DefTblrTemplate{remark-sep}{default}{:\space}
\DefTblrTemplate{remark-text}{default}{\InsertTblrRemarkText}
\end{codehigh}
\begin{codehigh}
\DefTblrTemplate{remark}{default}{
  \MapTblrRemarks{
    \noindent
    \UseTblrTemplate{remark-tag}{default}
    \UseTblrTemplate{remark-sep}{default}
    \UseTblrTemplate{remark-text}{default}
    \par
  }
}
\end{codehigh}

% \subsection{Head and Foot Templates}
\subsection{表头和表尾模板}

% The templates for table heads and foots are defined as including other templates:
表格的表头和表尾模板被定义为包含其它模板:

\begin{codehigh}
\DefTblrTemplate{firsthead}{default}{
  \UseTblrTemplate{caption}{default}
}
\DefTblrTemplate{middlehead,lasthead}{default}{
  \UseTblrTemplate{capcont}{default}
}
\DefTblrTemplate{firstfoot,middlefoot}{default}{
  \UseTblrTemplate{contfoot}{default}
}
\DefTblrTemplate{lastfoot}{default}{
  \UseTblrTemplate{note}{default}
  \UseTblrTemplate{remark}{default}
}
\end{codehigh}

% Note that you can define the same template for multiple elements in \verb!\DefTblrTemplate! command.
注意,可以在 \verb!\DefTblrTemplate! 命令中为多个元素定义同一个模板。

% \section{Change Styles}
\section{改变样式}

% You may change the styles of template elements.
池可以改变模板元素的样式。

\begin{codehigh}
\SetTblrStyle{firsthead}{font=\bfseries}
\SetTblrStyle{firstfoot}{fg=blue2}
\SetTblrStyle{middlefoot}{\itshape}
\SetTblrStyle{caption-tag}{red2}
\end{codehigh}

% All available keys for template elements are described in Table \ref{key:element}.
模板元素的有效设置详见表 \ref{key:element}.

% \begin{spectblr}[
%   caption = {Keys for the Styles of Template Elements},
%   label = {key:element},
%   remark{Note} = {In most cases, you can omit the underlined key names and write only their values.
%                   The keys \K{halign}, \K{indent} and \K{hang} are only for main templates.}
% ]{}
%   Key Name               & Key Description \\
%   \underline{\K{fg}}     & foreground color \\
%   \underline{\K{font}}   & font commands \\
%   \underline{\K{halign}} & horizontal alignment: \V{l} (left), \V{c} (center), or \V{r} (right) \\
%   \K{indent}             & parindent value \\
%   \K{hang}               & hangindent value \\
% \end{spectblr}
\begin{spectblr}[
  caption = {模板元素的样式键},
  label = {key:element},
  remark{注意} = {多数情况下,可以省略带下划线的键名而仅给出键值。
                  \K{halign}、 \K{indent} 和 \K{hang} 仅对主模板有效。}
]{}
  键名                   & 含义 \\
  \underline{\K{fg}}     & 前景颜色 \\
  \underline{\K{font}}   & 字体命令 \\
  \underline{\K{halign}} & 水平对齐方式: \V{l} (left)、 \V{c} (center)或 \V{r} (right) \\
  \K{indent}             & 段落缩进值 \\
  \K{hang}               & 悬挂缩进值 \\
\end{spectblr}

% When you write \verb!\UseTblrTemplate{element}{default}! in defining a template,
% beside including the code of template \verb!element!, the foreground color and font commands
% of the template \verb!element! will be set up automatically.
% In contrast, \verb!\ExpTblrTemplate{element}{default}! will only include template code.
在模板定义中,当使用 \verb!\UseTblrTemplate{element}{default}! 时,
除了包含模板\verb!element!代码外, 会自动设置模板\verb!element! 的前景颜色和字体命令。
相反, \verb!\ExpTblrTemplate{element}{default}! 将仅包含模板代码。

% \section{Define Themes}
\section{定义主题}

% You may define your own themes for table heads and foots with \verb!\NewTblrTheme! command.
% a theme consists of some template and style settings. For example:
可以使用 \verb!\NewTblrTheme! 命令定义表头与表尾的样式主题。
一个主题由模板和样式设置组成,例如:
\nopagebreak
\begin{codehigh}
\NewTblrTheme{fancy}{
  \DefTblrTemplate{conthead}{default}{[Continued]}
  \SetTblrStyle{firsthead}{font=\bfseries}
  \SetTblrStyle{firstfoot}{fg=blue2}
  \SetTblrStyle{middlefoot}{\itshape}
  \SetTblrStyle{caption-tag}{red2}
}
\end{codehigh}

% After defining the theme \verb!fancy!, you can use it
% by writing \verb!theme=fancy! in the optional argument of \verb!longtblr! environment.
定义了 \verb!fancy! 主题后,在以在 \verb!longtblr! 环境的可选参数中
使用 \verb!theme=fancy! 选项使用该主题。

% \section{Control Page Breaks}
\section{分页控制}

% Just like \verb!longtable! package, you can use \verb!\\*! or \verb!\nopagebreak!
% to prohibit a page break, and use \verb!\pagebreak! to force a page break.
像 \verb!longtable! 宏包一样,可以使用 \verb!\\*! 或 \verb!\nopagebreak!
禁用分页,用 \verb!\pagebreak! 实现强制分页。

\end{document}
