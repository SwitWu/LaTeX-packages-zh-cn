% -*- coding: utf-8 -*-
% !TEX program = lualatex
\documentclass[oneside]{book}

% -*- coding: utf-8 -*-
% !TEX program = lualatex

\usepackage[heading=true,scheme=chinese]{ctex}
\ctexset{
  figurename = 图,
  tablename  = 表,
}

\usepackage[a4paper,margin=2.5cm]{geometry}

\usepackage{codehigh} % https://ctan.org/pkg/codehigh
\usepackage{tabularray}
\usepackage{array,multirow,amsmath}
\usepackage{chemmacros,environ}

\UseTblrLibrary{booktabs,diagbox,siunitx}

\usepackage{hyperref}
\hypersetup{
  colorlinks=true,
  urlcolor=blue3,
  linkcolor=red3,
}

\usepackage{tcolorbox}
\tcbset{sharp corners, boxrule=0.5pt, colback=red9}

\usepackage{float}

\setcounter{tocdepth}{1}

\newcommand*{\K}[1]{\texttt{#1}}
\newcommand*{\V}[1]{\texttt{#1}}

\DefTblrTemplate{caption-tag}{default}{表\hspace{0.25em}\thetable}
% \DefTblrTemplate{caption-sep}{default}{:\enskip}
\DefTblrTemplate{contfoot-text}{default}{续下页}
\DefTblrTemplate{conthead-text}{default}{(接前页)}
% \DefTblrTemplate{caption-text}{default}{\InsertTblrText{caption}}

\NewTblrEnviron{newtblr}
\SetTblrOuter[newtblr]{long}
\SetTblrInner[newtblr]{
  hlines = {white}, column{1,2} = {co=1}, colsep = 5pt,
  row{odd} = {brown8}, row{even} = {gray8},
  row{1} = {fg=white, bg=purple2, font=\bfseries\sffamily},
}

\NewTblrEnviron{spectblr}
\SetTblrOuter[spectblr]{long}
\SetTblrInner[spectblr]{
  hlines = {white}, column{2} = {co=1}, colsep = 5pt,
  row{odd} = {brown8}, row{even} = {gray8},
  row{1} = {fg=white, bg=purple2, font=\bfseries\sffamily},
  rowhead = 1,
}

% \newcommand{\mywarning}[1]{%
%   \begin{tcolorbox}
%   The interfaces in this #1 should be seen as
%   \textcolor{red3}{\bfseries experimental}
%   and are likely to change in future releases, if necessary.
%   Don’t use them in important documents.
%   \end{tcolorbox}
% }
\newcommand{\mywarning}[1]{%
  \begin{tcolorbox}
   #1 中说明的接口是
  \textcolor{red3}{\bfseries 实验性的}
  并且在未来可能会发生变化,
  因此,请勿在重要的文档中使用它们。
  \end{tcolorbox}
}

\renewcommand*{\thefootnote}{*}

\newcommand*{\myversion}{2021L}
\newcommand*{\mydate}{Version \myversion\ (\the\year-\mylpad\month-\mylpad\day)\\\myrepo}
\newcommand*{\myrepo}{\url{https://github.com/lvjr/tabularray}}
\newcommand*{\mylpad}[1]{\ifnum#1<10 0\the#1\else\the#1\fi}

\colorlet{highback}{\ifodd\month azure9\else blue9\fi}
\CodeHigh{language=latex/table,style/main=highback,style/code=highback}
\NewCodeHighEnv{code}{style/main=gray9,style/code=gray9}
\NewCodeHighEnv{demo}{style/main=gray9,style/code=gray9,demo}

%\CodeHigh{lite}

\CodeHigh{lite}

\begin{document}

% \chapter{Overview of Features}
\chapter{概述}

% \section{Vertical Space}
\section{垂直间距}

% After loading \verb!tabularray! package in the preamble,
% we can use \verb!tblr! environments to typeset tabulars and arrays.
% The name \verb!tblr! is short for \verb!tabularray! or \verb!top-bottom-left-right!.
% The following is our first example:
在导言区加载\verb!tabularray!宏包后,
可以使用\verb!tblr!环境排版表格(tabulars)或阵列(arrays)。
环境\verb!tblr!的名称是\verb!tabularray!或\verb!top-bottom-left-right!的缩写。
以下示例展示了常规\verb!tabular!环境与\verb!tblr!环境排版表格的异同:

\begin{demo}
\begin{tabular}{lccr}
\hline
 Alpha   & Beta  & Gamma  & Delta \\
\hline
 Epsilon & Zeta  & Eta    & Theta \\
\hline
 Iota    & Kappa & Lambda & Mu    \\
\hline
\end{tabular}
\end{demo}

\begin{demohigh}
\begin{tblr}{lccr}
\hline
 Alpha   & Beta  & Gamma  & Delta \\
\hline
 Epsilon & Zeta  & Eta    & Theta \\
\hline
 Iota    & Kappa & Lambda & Mu    \\
\hline
\end{tblr}
\end{demohigh}

% You may notice that there is extra space above and below the table rows with \verb!tblr! envirenment.
% This space makes the table look better.
% If you don't like it, you could use \verb!\SetTblrInner! command:
显然,用\verb!tblr!环境排版表格时,在表格各行的上方和下方会增加额外垂直间距。
该间距使表格排版更为美观。当然,也可以使用 \verb!\SetTblrDefault!
命令根据需要进行调整该垂直间距,例如,可以用如下代码取消间距:

\begin{demohigh}
\SetTblrInner{rowsep=0pt}
\begin{tblr}{lccr}
\hline
 Alpha   & Beta  & Gamma  & Delta \\
\hline
 Epsilon & Zeta  & Eta    & Theta \\
\hline
 Iota    & Kappa & Lambda & Mu    \\
\hline
\end{tblr}
\end{demohigh} 

% But in many cases, this rowsep is useful:
但多数情况下,使用rowsep间距将使表格排版更为美观,例如如下带有大量分式的表格排版:

\begin{demo}
$\begin{array}{rrr}
\hline
 \dfrac{2}{3} &  \dfrac{2}{3} &  \dfrac{1}{3} \\
 \dfrac{2}{3} & -\dfrac{1}{3} & -\dfrac{2}{3} \\
 \dfrac{1}{3} & -\dfrac{2}{3} &  \dfrac{2}{3} \\
\hline
\end{array}$
\end{demo}

\begin{demohigh}
$\begin{tblr}{rrr}
\hline
 \dfrac{2}{3} &  \dfrac{2}{3} &  \dfrac{1}{3} \\
 \dfrac{2}{3} & -\dfrac{1}{3} & -\dfrac{2}{3} \\
 \dfrac{1}{3} & -\dfrac{2}{3} &  \dfrac{2}{3} \\
\hline
\end{tblr}$
\end{demohigh}

% Note that you can use \verb!tblr! in both text and math modes.
需要说明的是,\verb!tblr!环境既可以用于文本模式,也可以用于数学模式。

% \section{Multiline Cells}
\section{多行单元格}

% It's quite easy to write multiline cells without fixing the column width in \verb!tblr! environments:
% just enclose the cell text with braces and use \verb!\\! to break lines:
在使用\verb!tblr!环境排版表格时,不使用定宽列格式即可实现多行单元格排版。
此时,仅需用大括号将需要多行排版的单元格包围起来,
然后在大括号内的内容中使用\verb!\\!手动断行即可,如:

\begin{demohigh}
\begin{tblr}{|l|c|r|}
\hline
 Left & {Center \\ Cent \\ C} & {Right \\ R} \\
\hline
 {L \\ Left} & {C \\ Cent \\ Center} & R \\
\hline
\end{tblr}
\end{demohigh}

% \section{Cell Alignment}
\section{单元格对齐方式}

% From time to time,
% you may want to specify the horizontal and vertical alignment of cells at the same time.
% \verb!Tabularray! package provides a \verb!Q! column for this
% (In fact, \verb!Q! column is the only primitive column,
% other columns are defined as \verb!Q! columns with some options):
% 在排版时,可能需要同时指定单元格的水平与垂直对齐方式。
\verb!Tabularray!宏包提供了\verb!Q!列格式用于同时指定单元格水平和垂直对齐方式。
其实,\verb!Q!格式是\verb!Tabularray!宏包提供的唯一列格式的元格式,
其它的列格式都是通过为\verb!Q!元格式设置不同参数定义的。

\begin{demohigh}
\begin{tblr}{|Q[l,t]|Q[c,m]|Q[r,b]|}
\hline
 {Top Baseline \\ Left Left} & Middle Center & {Right Right \\ Bottom Baseline} \\
\hline
\end{tblr}
\end{demohigh} 

% Note that you can use more meaningful \verb!t! instead of \verb!p! for top baseline alignment.
% For some users who are familiar with word processors,
% these \verb!t! and \verb!b! columns are counter-intuitive.
% In \verb!tabularray! package, there are another two column types \verb!h! and \verb!f!,
% which will align cell text at the head and the foot, respectively:
注意,对于顶端基线对齐,可使用含义更为明确的\verb!t!格式替代\verb!p!格式。
对于熟悉字处理工具的用户而言,类似\verb!t!和\verb!b!这些格式是违背直觉的,
其中,\verb!t!是顶端与基线对齐,不是顶端对齐,而\verb!b!是底端与基线对齐,不是底端对齐。
在\verb!Tabularray!宏包中,还定义了\verb!h!和\verb!f!格式,
以分别实现常规意义上的顶端对齐和底端对齐:

\begin{demohigh}
\begin{tblr}{Q[h,4em]Q[t,4em]Q[m,4em]Q[b,4em]Q[f,4em]}
\hline
 {row\\head} & {top\\line} & {middle} & {line\\bottom} & {row\\foot} \\
\hline
 {row\\head} & {top\\line} & {11\\22\\mid\\44\\55} & {line\\bottom} & {row\\foot} \\
\hline
\end{tblr}
\end{demohigh}

% \section{Multirow Cells}
\section{单元格行合并}

% The above \verb!h! and \verb!f! alignments are necessary
% when we write multirow cells with \verb!\SetCell! command in \verb!tabularray!.
在\verb!Tabularray!中,当用\verb!\SetCell!命令进行单元格行合并时,
\verb!h!和\verb!f!对齐格式是非常必要的。

\begin{demo}
\begin{tabular}{|l|l|l|l|}
\hline
 \multirow[t]{4}{1.5cm}{Multirow Cell One} & Alpha &
 \multirow[b]{4}{1.5cm}{Multirow Cell Two} & Alpha \\
 & Beta  & & Beta \\
 & Gamma & & Gamma \\
 & Delta & & Delta \\
\hline
\end{tabular}
\end{demo}

\begin{demohigh}
\begin{tblr}{|l|l|l|l|}
\hline
 \SetCell[r=4]{h,1.5cm} Multirow Cell One & Alpha &
 \SetCell[r=4]{f,1.5cm} Multirow Cell Two & Alpha \\
 & Beta  & & Beta \\
 & Gamma & & Gamma \\
 & Delta & & Delta \\
\hline
\end{tblr}
\end{demohigh}

% Note that you don't need to load \verb!multirow! package first,
% since \verb!tabularray! doesn't depend on it.
% Furthermore, \verb!tabularray! will always typeset decent multirow cells.
% First, it will set correct vertical middle alignment,
% even though some rows have large height:
注意,由于\verb!Tabularray!宏包并不依赖于\verb!multirow!宏包,
因此,在行合并前并不需要载入\verb!multirow!宏包。
此外,\verb!Tabularray!宏包将自动处理单元格行合并后的垂直对齐,
这使用即便是在某些行的行高较大时,也能够正确地实现行合并后的垂直居中对齐。

\begin{demo}
\begin{tabular}{|l|m{4em}|}
\hline
 \multirow[c]{4}{1.5cm}{Multirow} & Alpha  \\
 & Beta  \\
 & Gamma \\
 & Delta Delta Delta \\
\hline
\end{tabular}
\end{demo}

\begin{demohigh}
\begin{tblr}{|l|m{4em}|}
\hline
 \SetCell[r=4]{m,1.5cm} Multirow & Alpha  \\
 & Beta  \\
 & Gamma \\
 & Delta Delta Delta \\
\hline
\end{tblr}
\end{demohigh}

% Second, it will enlarge row heights if the multirow cells have large height,
% therefore it always avoids vertical overflow:
同时,如果行合并后行高过大,则\verb!Tabularray!会自动扩展合并后单元格的行高,
从而确保不会发生文字垂直溢出现象。

\begin{demo}
\begin{tabular}{|l|m{4em}|}
\hline
 \multirow[c]{2}{1cm}{Line \\ Line \\ Line \\ Line} & Alpha \\
\cline{2-2}
 & Beta \\
\hline
\end{tabular}
\end{demo}

\begin{demohigh}
\begin{tblr}{|l|m{4em}|}
\hline
 \SetCell[r=2]{m,1cm} {Line \\ Line \\ Line \\ Line} & Alpha \\
\cline{2}
 & Beta \\
\hline
\end{tblr}
\end{demohigh}

% \section{Multi Rows and Columns}
\section{单元格行合并和列合并}

% It was a hard job to typeset cells with multiple rows and multiple columns. For example:
同时实现单元格行合并和列合并是比较困难的,例如:

\begin{demo}
\begin{tabular}{|c|c|c|c|c|}
\hline
 \multirow{2}{*}{2 Rows}
     & \multicolumn{2}{c|}{2 Columns}
                 & \multicolumn{2}{c|}{\multirow{2}{*}{2 Rows 2 Columns}} \\
\cline{2-3}
     & 2-2 & 2-3 & \multicolumn{2}{c|}{} \\
\hline
 3-1 & 3-2 & 3-3 & 3-4 & 3-5 \\
\hline
\end{tabular}
\end{demo}

% With \verb!tabularray! package, you can set spanned cells with \verb!\SetCell! command:
% within the optional argument of \verb!\SetCell! command,
% option \verb!r! is for rowspan number, and \verb!c! for colspan number;
% within the mandatory argument of it, horizontal and vertical alignment options are accepted.
% Therefore it's much simpler to typeset spanned cells:
在\verb!Tabularray!宏包中,可以使用\verb!\SetCell!命令实现单元格合并。
在\verb!\SetCell!命令的可选参数中,\verb!r!选项用于指定需要合并的行数,
\verb!c!选项用于指定需要合并的列数。
在\verb!\SetCell!命令的必选参数中,可以指定合并后的水平和垂直对齐方式。
因此,在\verb!Tabularray!中,单元格合并更为简捷:

\begin{demohigh}
\begin{tblr}{|c|c|c|c|c|}
\hline
 \SetCell[r=2]{c} 2 Rows
     & \SetCell[c=2]{c} 2 Columns
           &     & \SetCell[r=2,c=2]{c} 2 Rows 2 Columns & \\
\hline
     & 2-2 & 2-3 &     &     \\
\hline
 3-1 & 3-2 & 3-3 & 3-4 & 3-5 \\
\hline
\end{tblr}
\end{demohigh}

% Using \verb!\multicolumn! command, the omitted cells \textcolor{red3}{must} be removed.
% On the contrary,
使用\verb!\multicolumn!命令时,\textcolor{red3}{必须删除}需要合并的其它单元格。
相反,
% using \verb!\multirow! command, the omitted cells \textcolor{red3}{must not} be removed.
% \verb!\SetCell! command behaves the same as \verb!\multirow! command in this aspect.
在使用\verb!\multirow!命令时,则\textcolor{red3}{必须保留}需要合并的其它单元格。
\verb!\SetCell!命令则与\verb!\multirow!命令相同。

% With \verb!tblr! environment, any \verb!\hline! segments inside a spanned cell will be ignored,
% therefore we're free to use \verb!\hline! in the above example.
% Also, any omitted cell will definitely be ignored when typesetting,
% no matter it's empty or not.
% With this feature, we could put row and column numbers into the omitted cells,
% which will help us to locate cells when the tables are rather complex:
使用\verb!tblr!环境时,由于会自动忽略合并单元格的\verb!\hline!线段,
因此,在以上的示例中,可以直接使用\verb!\hline!命令绘制表格横线。
同时,无论单元格是否留空,任何合并时省略的单元格在排版时都会被忽略。
基于此,在排版表格时,可以将行号、列号或其它标识性内容写入需要省略的单元格,
这有助于在排版复杂表格时的单元格定位,如:

\begin{demohigh}
\begin{tblr}{|ll|c|rr|}
\hline
 \SetCell[r=3,c=2]{h} r=3 c=2 & 1-2 & \SetCell[r=2,c=3]{r} r=2 c=3 & 1-4 & 1-5 \\ 
 2-1 & 2-2 & 2-3 & 2-4 & 2-5 \\
\hline
 3-1 & 3-2 & MIDDLE & \SetCell[r=3,c=2]{f} r=3 c=2 & 3-5 \\
\hline
 \SetCell[r=2,c=3]{l} r=2 c=3 & 4-2 & 4-3 & 4-4 & 4-5 \\
 5-1 & 5-2 & 5-3 & 5-4 & 5-5 \\
\hline
\end{tblr}
\end{demohigh}

% \section{Column Types}
\section{列格式}

% \verb!Tabularray! package supports all normal column types, as well as
% the extendable \verb!X! column type,
% which first occurred in \verb!tabularx! package and was largely improved by \verb!tabu! package:
\verb!Tabularray!宏包支持所有常规的列格式,包括第一次出现在\verb!tabularx!宏包中,
并被\verb!tabu!宏包优化后的可扩展的\verb!X!列格式。

\begin{demohigh}
\begin{tblr}{|X[2,l]|X[3,l]|X[1,r]|X[r]|}
\hline
 Alpha & Beta & Gamma & Delta \\
\hline
\end{tblr}
\end{demohigh}

% Also, \verb!X! columns with negative coefficients are possible:
此外,还可以在\verb!X!列格式中使用负系数:

\begin{demohigh}
\begin{tblr}{|X[2,l]|X[3,l]|X[-1,r]|X[r]|}
\hline
 Alpha & Beta & Gamma & Delta \\
\hline
\end{tblr}
\end{demohigh}

% We need the width to typeset a table with \verb!X! columns.
% If unset, the default is \verb!\linewidth!.
% To change the width, we have to first put all column specifications into \verb!colspec={...}!:
在使用\verb!X!列格式排版表格时,需要指定表格宽度。
如未设置,则默认为 \verb!\linewidth!。
如需更改表格宽度,则必须将所有列格式置入\verb!colspec={...}!选项的参数中:

\begin{demohigh}
\begin{tblr}{width=0.8\linewidth,colspec={|X[2,l]|X[3,l]|X[-1,r]|X[r]|}}
\hline
 Alpha & Beta & Gamma & Delta \\
\hline
\end{tblr}
\end{demohigh}

% You can define new column types with \verb!\NewColumnType! command.
% For example, in \verb!tabularray! package,
% \verb!b! and \verb!X! columns are defined as special \verb!Q! columns:
可以使用\verb!\NewColumnType!命令定义新的列格式。
例如,在 \verb!Tabularray! 宏包中,\verb!b! 和\verb!X! 列格式
是用\verb!Q! 元格式通过指定必要参数实现的:
\begin{codehigh}
\NewColumnType{b}[1]{Q[b,wd=#1]}
\NewColumnType{X}[1][]{Q[co=1,#1]}
\end{codehigh}

% \section{Row Types}
\section{行格式}

% Now that we have column types and \verb!colspec! option,
% you may ask for row types and \verb!rowspec! option.
% Yes, they are here:
除了可以使用 \verb!colspec! 选项指定列格式外,
也可以通过 \verb!rowspec! 选项指定行格式,如:

\begin{demohigh}
\begin{tblr}{colspec={Q[l]Q[c]Q[r]},rowspec={|Q[t]|Q[m]|Q[b]|}}
 {Alpha \\ Alpha} & Beta               & Gamma \\
 Delta            & Epsilon            & {Zeta \\ Zeta}  \\
 Eta              & {Theta \\ Theta}   & Iota  \\
\end{tblr}
\end{demohigh}

% Same as column types, \verb!Q! is the only primitive row type,
% and other row types are defined as \verb!Q! types with different options.
% It's better to specify horizontal alignment in \verb!colspec!,
% and vertical alignment in \verb!rowspec!, respectively.
与列格式类似,\verb!Q! 是唯一的行格式的元格式,
其它行格式都是通过为 \verb!Q! 元格式指定不同参数实现的。
强烈建议在 \verb!colspec! 中指定水平对齐方式,在 \verb!rowspec! 中指定垂直对齐方式。

% Inside \verb!rowspec!, \verb!|! is the hline type.
% Therefore we need not to write \verb!\hline! commnad, which makes table code cleaner.
在 \verb!rowspec! 中,\verb!|! 用于指定表格横线格式。
因此,无需再在表格内容中使用 \verb!\hline! 命令,这会使表格代码更为清晰,代码可维护性更强。

% \section{Hlines and Vlines}
\section{表格横线与表格竖线}

% Hlines and vlines have been improved too. You can specify the widths and styles of them:
在\verb!Tabularray!宏包中,重新设计了表格横线和竖线命令,
可以通过参数指定其的宽度、线型、颜色等样式:

\begin{demohigh}
\begin{tblr}{|l|[dotted]|[2pt]c|r|[solid]|[dashed]|}
\hline
One   &  Two  & Three \\
\hline\hline[dotted]\hline
Four  & Five  &   Six \\
\hline[dashed]\hline[1pt]
Seven & Eight &  Nine \\
\hline
\end{tblr}
\end{demohigh}

% \section{Colorful Tables}
\section{彩色表格}

% To add colors to your tables, you need to load \verb!xcolor! package first.
% \verb!Tabularray! package will also load \verb!ninecolors! package for proper color contrast.
% First you can specify background option for \verb!Q! rows/columns inside \verb!rowspec!/\verb!colspec!:
如需排版彩色表格,则需加载 \verb!xcolor! 宏包,
\verb!Tabularray! 宏包一旦发现用户加载了\verb!xcolor!宏包,
则会自动加载 \verb!ninecolors! 宏包,以使用前景与背景对比度更为合理的颜色。
例如,可以在\verb!rowspec!/\verb!colspec! 选项中通过\verb!Q! 元格式
的背景参数为行/列指定背景色:

\begin{demohigh}
\begin{tblr}{colspec={lcr},rowspec={|Q[cyan7]|Q[azure7]|Q[blue7]|}}
 Alpha   & Beta  & Gamma  \\
 Epsilon & Zeta  & Eta    \\
 Iota    & Kappa & Lambda \\
\end{tblr}
\end{demohigh}

\begin{demohigh}
\begin{tblr}{colspec={Q[l,brown7]Q[c,yellow7]Q[r,olive7]},rowspec={|Q|Q|Q|}}
 Alpha   & Beta  & Gamma  \\
 Epsilon & Zeta  & Eta    \\
 Iota    & Kappa & Lambda \\
\end{tblr}
\end{demohigh}

% Also you can use \verb!\SetRow! or \verb!\SetColumn! command to specify row or column colors:
当然,也可以使用 \verb!\SetRow! 或 \verb!\SetColumn! 命令统一为指定的行或列设置颜色:

\begin{demohigh}
\begin{tblr}{colspec={lcr},rowspec={|Q|Q|Q|}}
 \SetRow{cyan7}  Alpha   & Beta  & Gamma  \\
 \SetRow{azure7} Epsilon & Zeta  & Eta    \\
 \SetRow{blue7}  Iota    & Kappa & Lambda \\
\end{tblr}
\end{demohigh}

\begin{demohigh}
\begin{tblr}{colspec={lcr},rowspec={|Q|Q|Q|}}
 \SetColumn{brown7}
 Alpha          & \SetColumn{yellow7}
                  Beta            & \SetColumn{olive7}
                                    Gamma  \\
 Epsilon        & Zeta            & Eta    \\
 Iota           & Kappa           & Lambda \\
\end{tblr}
\end{demohigh}

% Hlines and vlines can also have colors:
此外,还可以为表格横线和竖线指定颜色:

\begin{demohigh}
\begin{tblr}{colspec={lcr},rowspec={|[2pt,green7]Q|[teal7]Q|[green7]Q|[3pt,teal7]}}
 Alpha   & Beta  & Gamma  \\
 Epsilon & Zeta  & Eta    \\
 Iota    & Kappa & Lambda \\
\end{tblr}
\end{demohigh}

\begin{demohigh}
\begin{tblr}{colspec={|[2pt,violet5]l|[2pt,magenta5]c|[2pt,purple5]r|[2pt,red5]}}
 Alpha   & Beta  & Gamma  \\
 Epsilon & Zeta  & Eta    \\
 Iota    & Kappa & Lambda \\
\end{tblr}
\end{demohigh}

\end{document}
