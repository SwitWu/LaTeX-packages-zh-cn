% -*- coding: utf-8 -*-
% !TEX program = lualatex
\documentclass[oneside]{book}

% -*- coding: utf-8 -*-
% !TEX program = lualatex

\usepackage[heading=true,scheme=chinese]{ctex}
\ctexset{
  figurename = 图,
  tablename  = 表,
}

\usepackage[a4paper,margin=2.5cm]{geometry}

\usepackage{codehigh} % https://ctan.org/pkg/codehigh
\usepackage{tabularray}
\usepackage{array,multirow,amsmath}
\usepackage{chemmacros,environ}

\UseTblrLibrary{booktabs,diagbox,siunitx}

\usepackage{hyperref}
\hypersetup{
  colorlinks=true,
  urlcolor=blue3,
  linkcolor=red3,
}

\usepackage{tcolorbox}
\tcbset{sharp corners, boxrule=0.5pt, colback=red9}

\usepackage{float}

\setcounter{tocdepth}{1}

\newcommand*{\K}[1]{\texttt{#1}}
\newcommand*{\V}[1]{\texttt{#1}}

\DefTblrTemplate{caption-tag}{default}{表\hspace{0.25em}\thetable}
% \DefTblrTemplate{caption-sep}{default}{:\enskip}
\DefTblrTemplate{contfoot-text}{default}{续下页}
\DefTblrTemplate{conthead-text}{default}{(接前页)}
% \DefTblrTemplate{caption-text}{default}{\InsertTblrText{caption}}

\NewTblrEnviron{newtblr}
\SetTblrOuter[newtblr]{long}
\SetTblrInner[newtblr]{
  hlines = {white}, column{1,2} = {co=1}, colsep = 5pt,
  row{odd} = {brown8}, row{even} = {gray8},
  row{1} = {fg=white, bg=purple2, font=\bfseries\sffamily},
}

\NewTblrEnviron{spectblr}
\SetTblrOuter[spectblr]{long}
\SetTblrInner[spectblr]{
  hlines = {white}, column{2} = {co=1}, colsep = 5pt,
  row{odd} = {brown8}, row{even} = {gray8},
  row{1} = {fg=white, bg=purple2, font=\bfseries\sffamily},
  rowhead = 1,
}

% \newcommand{\mywarning}[1]{%
%   \begin{tcolorbox}
%   The interfaces in this #1 should be seen as
%   \textcolor{red3}{\bfseries experimental}
%   and are likely to change in future releases, if necessary.
%   Don’t use them in important documents.
%   \end{tcolorbox}
% }
\newcommand{\mywarning}[1]{%
  \begin{tcolorbox}
   #1 中说明的接口是
  \textcolor{red3}{\bfseries 实验性的}
  并且在未来可能会发生变化,
  因此,请勿在重要的文档中使用它们。
  \end{tcolorbox}
}

\renewcommand*{\thefootnote}{*}

\newcommand*{\myversion}{2021L}
\newcommand*{\mydate}{Version \myversion\ (\the\year-\mylpad\month-\mylpad\day)\\\myrepo}
\newcommand*{\myrepo}{\url{https://github.com/lvjr/tabularray}}
\newcommand*{\mylpad}[1]{\ifnum#1<10 0\the#1\else\the#1\fi}

\colorlet{highback}{\ifodd\month azure9\else blue9\fi}
\CodeHigh{language=latex/table,style/main=highback,style/code=highback}
\NewCodeHighEnv{code}{style/main=gray9,style/code=gray9}
\NewCodeHighEnv{demo}{style/main=gray9,style/code=gray9,demo}

%\CodeHigh{lite}

\CodeHigh{lite}
\setcounter{chapter}{1}

\begin{document}

% \chapter{Basic Interfaces}
\chapter{用户接口}

% \section{Old and New Interfaces}
\section{新旧用户接口}

% With \verb!tabularray! package, you can change the styles of tables via old interfaces or new interfaces.
在 \verb!Tabularray! 宏包中,可以通过用户接口实现对表格样式的控制。

% The old interfaces consist of some table commands inside the table contents.
% Same as \verb!tabular! and \verb!array! environments,
% all table commands \textcolor{red3}{must} be put at the beginning ot the cell text.
% Also, new table commands \textcolor{red3}{must} be defined with \verb!\NewTableCommand!.
旧的用户接口由置于表格内容中的一系列命令组成。
此时,与\verb!tabular!和\verb!array!环境类似,
所有的命令\textcolor{red3}{必须}用于单元格的文本内容之前。
另外,如果需要新的表格命令,
则\textcolor{red3}{必须}使用 \verb!\NewTableCommand! 命令进行定义。


% The new interfaces consist of some options inside the mandatory argument,
% hence totally separating the styles and the contents of tables.
新的用户接口由\verb!tblr!环境必选参数的选项实现,
因此,使用新用户接口可以实现表格内容与格式的完全分离。

\begin{newtblr}[
  % caption = {Old Interfaces and New Interfaces},
  caption = {新旧用户接口},
  label = {key:interface},
]{verb}
  % Old Interfaces                                 & New Interfaces          \\
  旧接口                                         & 新接口                  \\
  \verb!\SetHlines!                              & \K{hlines}              \\
  \verb!\SetHline!, \verb!\hline!, \verb!\cline! & \K{hline}, \K{rowspec}  \\
  \verb!\SetVlines!                              & \K{vlines}              \\
  \verb!\SetVline!, \verb!\vline!, \verb!\rline! & \K{vline}, \K{colspec}  \\
  \verb!\SetCells!                               & \K{cells}               \\
  \verb!\SetCell!                                & \K{cell}                \\
  \verb!\SetRows!                                & \K{rows}                \\
  \verb!\SetRow!                                 & \K{row}, \K{rowspec}    \\
  \verb!\SetColumns!                             & \K{columns}             \\
  \verb!\SetColumn!                              & \K{column}, \K{colspec} \\
\end{newtblr}

% \section{Hlines and Vlines}
\section{表格横线和竖线}

% All available keys for hlines and vlines are described in Table \ref{key:hvline}.
表格横线和竖线\verb!hlines!和\verb!vlines!选项的所有有效键值参见表\ref{key:hvline}。

% \begin{spectblr}[
%   caption = {Keys for Hlines and Vlines},
%   label = {key:hvline},
%   remark{Note} = {In most cases, you can omit the underlined key names and write only their values.}
% ]{}
%   Key & Description and Values & Initial Value \\
%   \underline{\K{dash}} & dash style: \V{solid}, \V{dashed} or \V{dotted} & \V{solid} \\
%   \K{text}             & replace hline/vline with text (like \V{!} specifier in \K{colspec}) & None \\
%   \underline{\K{wd}}   & rule width dimension & None \\
%   \underline{\K{fg}}   & rule color name & None \\
% \end{spectblr}
\begin{spectblr}[
  caption = {\V{hlines}和\V{vlines}的键值},
  label = {key:hvline},
  remark{注意} = {多数情况下,对于带有下划线的键,可以省略键名而只给出键值。}
]{}
  键 & 说明与可选键值 & 初始值 \\
  \underline{\K{dash}} & 线型:\V{solid}, \V{dashed} or \V{dotted} & \V{solid} \\
  \K{text}             & hline/vline的替换文本(如在\K{colspec}中将竖线替换为\V{!}) & 无 \\
  \underline{\K{wd}}   & 线宽 & 无 \\
  \underline{\K{fg}}   & 颜色 & 无 \\
\end{spectblr}

% \subsection{Hlines and Vlines in New Interfaces}
\subsection{新接口中的表格横线与竖线}

% Options \verb!hlines! and \verb!vlines! are for setting all hlines and vlines, respectively.
% With empty value, all hlines/vlines will be solid.
\verb!hlines! 和 \verb!vlines! 选项分别用于设置表格所有横线与竖线样式。
如果其键值留空,则会把表格所有横线/竖线都设置为实线。

\begin{demohigh}
\begin{tblr}{hlines,vlines}
 Alpha   & Beta  & Gamma   & Delta   \\
 Epsilon & Zeta  & Eta     & Theta   \\
 Iota    & Kappa & Lambda  & Mu      \\
\end{tblr}
\end{demohigh}

% With values inside one pair of braces, all hlines/vlines will be styled.
可以在一对大括号中通过参数指定横线/竖线的样式。

\begin{demohigh}
\begin{tblr}{
 hlines = {1pt,solid}, vlines = {red3,dashed},
}
 Alpha   & Beta  & Gamma   & Delta   \\
 Epsilon & Zeta  & Eta     & Theta   \\
 Iota    & Kappa & Lambda  & Mu      \\
\end{tblr}
\end{demohigh}

% Another pair of braces before will select segments in all hlines/vlines.
也可以在大括号前用另一对大括号选择需要设置的横线/竖线的单元格线段。

\begin{demohigh}
\begin{tblr}{
 vlines = {1,3,5}{dashed},
 vlines = {2,4,6}{solid},
}
 Alpha   & Beta  & Gamma   & Delta   \\
 Epsilon & Zeta  & Eta     & Theta   \\
 Iota    & Kappa & Lambda  & Mu      \\
 Nu      & Xi    & Omicron & Pi      \\
 Rho     & Sigma & Tau     & Upsilon \\
 Phi     & Chi   & Psi     & Omega   \\
\end{tblr}
\end{demohigh}

% The above example can be simplified with \verb!odd! and \verb!even! values.
% (More child selectors can be defined with \verb!\NewChildSelector! command.
% Advanced users could read the source code for this.)
以上的示例也可以简单地通过设置\verb!odd!和\verb!even!选择算子进行选择
(可以通过 \verb!\NewChildSelector! 命令定义选择算子,高级用户请阅读
\verb!Tabularray!源代码,通过模仿实现需要的选择算子的定义)。

\begin{demohigh}
\begin{tblr}{
 vlines = {odd}{dashed},
 vlines = {even}{solid},
}
 Alpha   & Beta  & Gamma   & Delta   \\
 Epsilon & Zeta  & Eta     & Theta   \\
 Iota    & Kappa & Lambda  & Mu      \\
 Nu      & Xi    & Omicron & Pi      \\
 Rho     & Sigma & Tau     & Upsilon \\
 Phi     & Chi   & Psi     & Omega   \\
\end{tblr}
\end{demohigh}

% Another pair of braces before will draw more hlines/vlines (in which \verb!-! stands for all line segments).
可以再增加一对大括号用于设置多重横线/竖线(其中,\verb!-! 表示选择所有单元格线段)。

\begin{demohigh}
\begin{tblr}{
 hlines = {1}{-}{dashed},
 hlines = {2}{-}{solid},
}
 Alpha   & Beta  & Gamma   & Delta   \\
 Epsilon & Zeta  & Eta     & Theta   \\
 Iota    & Kappa & Lambda  & Mu      \\
\end{tblr}
\end{demohigh}

% Options \verb!hline{i}! and \verb!vline{j}! are for setting some hlines and vlines, respectively.
% Their values are the same as options \verb!hliness! and \verb!vlines!:
\verb!hline{i}!和\verb!vline{j}!选项分别用于设置指定的横线/竖线,
\verb!i!或\verb!j!的值与 \verb!hlines! and \verb!vlines! 的参数含义相同:

\begin{demohigh}
\begin{tblr}{
 hline{1,7} = {1pt,solid},
 hline{3-5} = {blue3,dashed},
 vline{1,5} = {3-4}{dotted},
}
 Alpha   & Beta  & Gamma   & Delta   \\
 Epsilon & Zeta  & Eta     & Theta   \\
 Iota    & Kappa & Lambda  & Mu      \\
 Nu      & Xi    & Omicron & Pi      \\
 Rho     & Sigma & Tau     & Upsilon \\
 Phi     & Chi   & Psi     & Omega   \\
\end{tblr}
\end{demohigh}

% You can use \verb!X!, \verb!Y!, \verb!Z! to denote the last three childs, respectively.
% It is especially useful when you are writing long tables:
也可以使用\verb!X!、\verb!Y!和\verb!Z!分别指定三线样式,
这在排版长表格是特别有用。

\begin{demohigh}
\begin{tblr}{
 hline{1,Z} = {2pt},
 hline{2,Y} = {1pt},
 hline{3-X} = {dashed},
}
 Alpha   & Beta  & Gamma   & Delta   \\
 Epsilon & Zeta  & Eta     & Theta   \\
 Iota    & Kappa & Lambda  & Mu      \\
 Nu      & Xi    & Omicron & Pi      \\
 Rho     & Sigma & Tau     & Upsilon \\
 Phi     & Chi   & Psi     & Omega   \\
\end{tblr}
\end{demohigh}

% Now we show the usage of \verb!text! key by the following example%
% \footnote{Code from \url{https://tex.stackexchange.com/questions/603023/tabularray-and-tabularx-column-separator}.}:
下面的示例演示了 \verb!text! 键的使用方式%
\footnote{代码来自 \url{https://tex.stackexchange.com/questions/603023/tabularray-and-tabularx-column-separator}。}:

\begin{demohigh}
\begin{tblr}{
  vlines, hlines,
  colspec = {lX[c]X[c]X[c]X[c]},
  vline{2} = {1}{text=\clap{:}},
  vline{3} = {1}{text=\clap{\ch{+}}},
  vline{4} = {1}{text=\clap{\ch{->}}},
  vline{5} = {1}{text=\clap{\ch{+}}},
}
  Equation & \ch{CH4} & \ch{2 O2} & \ch{CO2} & \ch{2 H2O} \\
  Initial  & $n_1$    & $n_2$     & 0        & 0 \\
  Final    & $n_1-x$  & $n_2-2x$  & $x$      & $2x$ \\
\end{tblr}
\end{demohigh}

% You need to load \verb!chemmacros! package for the \verb!\ch! command.
注意,为使用 \verb!\ch! 命令,必须载入\verb!chemmacros!宏包。

% \subsection{Hlines and Vlines in Old Interfaces}
\subsection{旧接口中的表格横线与竖线}

% The \verb!\hline! command has an optional argument which accepts key-value options.
% The available keys are described in Table \ref{key:hvline}.
可以在\verb!\hline!命令中通过选项指定横线样式,
其有效的键值见表\ref{key:hvline}.

\begin{demohigh}
\begin{tblr}{llll}
\hline
 Alpha   & Beta  & Gamma  & Delta \\
\hline[dashed]
 Epsilon & Zeta  & Eta    & Theta \\
\hline[dotted]
 Iota    & Kappa & Lambda & Mu    \\
\hline[2pt,blue5]
\end{tblr}
\end{demohigh}

% The \verb!\cline! command also has an optional argument which is the same as \verb!\hline!.
\verb!\cline!命令的选项与\verb!\hline!相同。

\begin{demohigh}
\begin{tblr}{llll}
\cline{1-4}
 Alpha   & Beta  & Gamma  & Delta \\
\cline[dashed]{1,3}
 Epsilon & Zeta  & Eta    & Theta \\
\cline[dashed]{2,4}
 Iota    & Kappa & Lambda & Mu    \\
\cline[2pt,blue5]{-}
\end{tblr}
\end{demohigh}

% You can use child selectors in the mandatory argument of \verb!\cline!.
也可以在 \verb!\cline! 命令的必选参数中通过选择算子实现行/列的选择。

\begin{demohigh}
\begin{tblr}{llll}
\cline{1-4}
 Alpha   & Beta  & Gamma  & Delta \\
\cline[dashed]{odd}
 Epsilon & Zeta  & Eta    & Theta \\
\cline[dashed]{even}
 Iota    & Kappa & Lambda & Mu    \\
\cline[2pt,blue5]{-}
\end{tblr}
\end{demohigh}

% Commands \verb!\SetHline! combines the usages of \verb!\hline! and \verb!\cline!:
\verb!\SetHline! 命令组合了 \verb!\hline! 和 \verb!\cline!命令:

\begin{demohigh}
\begin{tblr}{llll}
\SetHline{1-3}{blue5,1pt}
 Alpha   & Beta  & Gamma  & Delta \\
 Epsilon & Zeta  & Eta    & Theta \\
 Iota    & Kappa & Lambda & Mu    \\
\SetHline{2-4}{teal5,1pt}
\end{tblr}
\end{demohigh}

\begin{demohigh}
\begin{tblr}{llll}
\SetHline[1]{1-3}{blue5,1pt}
\SetHline[2]{1-3}{azure5,1pt}
 Alpha   & Beta  & Gamma  & Delta \\
 Epsilon & Zeta  & Eta    & Theta \\
 Iota    & Kappa & Lambda & Mu    \\
\SetHline[1]{2-4}{teal5,1pt}
\SetHline[2]{2-4}{green5,1pt}
\end{tblr}
\end{demohigh}

% In fact, table command \verb!\SetHline[<index>]{<columns>}{<styles>}! at the beginning of row \verb!i!
% is the same as table option \verb!hline{i}={<index>}{<columns>}{<styles>}!.
本质上, 在第\verb!i!行前使用的表格命令 \verb!\SetHline[<index>]{<columns>}{<styles>}!
与表格选项 \verb!hline{i}={<index>}{<columns>}{<styles>}!的作用完全相同。

% Also, table command \verb!\SetHlines[<index>]{<columns>}{<styles>}! at the beginning of some row
% is the same as table option \verb!hlines{i}={<index>}{<columns>}{<styles>}!.
同样,在某些行前使用的表格命令 \verb!\SetHlines[<index>]{<columns>}{<styles>}!
与表格选项 \verb!hlines{i}={<index>}{<columns>}{<styles>}!的作用完全相同。

% The usages of table commands \verb!\vline!, \verb!\rline!, \verb!\SetVline!, \verb!\SetVlines!
% are similar to those of \verb!\hline!, \verb!\cline!, \verb!\SetHline!, \verb!\SetHlines!, respectively.
% But normally you don't need to use them.
\verb!\vline!、 \verb!\rline!、 \verb!\SetVline!和 \verb!\SetVlines!命令的使用方法
分别与\verb!\hline!, \verb!\cline!, \verb!\SetHline!, \verb!\SetHlines!的使用方法相同。
但通常情况下,一般不建议直接使用这些命令。

% \section{Cells and Spancells}
\section{单元格与单元格合并选项}

% All available keys for cells are described in Table \ref{key:cell} and Table \ref{key:cellspan}.
单元格选项的所有有效键值见表\ref{key:cell}和表 \ref{key:cellspan}.

% \begin{spectblr}[
%   caption = {Keys for the Content of Cells},
%   label = {key:cell},
%   remark{Note} = {In most cases, you can omit the underlined key names and write only their values.}
% ]{}
%   Key & Description and Values & Initial Value \\
%   \underline{\K{halign}}
%     & horizontal alignment: \V{l} (left), \V{c} (center), or \V{r} (right)
%     & \V{l} \\
%   \underline{\K{valign}}
%     & vertical alignment: \V{t} (top), \V{m} (middle), \V{b} (bottom),
%       \V{h} (head) or \V{f} (foot)
%     & \V{t} \\
%   \underline{\K{wd}} & width dimension & None \\
%   \underline{\K{bg}} & background color name & None \\
%   \K{fg}    & foreground color name & None \\
%   \K{font}  & font commands & None \\
%   \K{preto} & prepend text to the cell & None \\
%   \K{appto} & append text to the cell & None \\
%   \K{cmd}   & execute command for the cell text & None \\
% \end{spectblr}
\begin{spectblr}[
  caption = {\V{cells}键与键值},
  label = {key:cell},
  remark{注意} = {多数情况下,对于带有下划线的键,可以省略键名而只给出键值。}
]{}
  键 & 说明与可选键值 & 初始值 \\
  \underline{\K{halign}}
    & 水平对齐方式: \V{l} (left), \V{c} (center)或\V{r} (right)
    & \V{l} \\
  \underline{\K{valign}}
    & 垂直对齐方式: \V{t} (top), \V{m} (middle), \V{b} (bottom),
      \V{h} (head)或\V{f} (foot)
    & \V{t} \\
  \underline{\K{wd}} & 宽度 & 无 \\
  \underline{\K{bg}} & 背景颜色 & 无 \\
  \K{fg}    & 前景颜色 & 无 \\
  \K{font}  & 字体命令 & 无 \\
  \K{preto} & 单元格前导文本 & 无 \\
  \K{appto} & 单元格附加文本 & 无 \\
  \K{cmd}   & 对单元格文本要执行的命令 & 无 \\
\end{spectblr}
\vspace{-2em}
% \begin{spectblr}[
%   caption = {Keys for Multispan of Cells},
%   label = {key:cellspan},
% ]{}
%   Key & Description and Values & Initial Value \\
%   \K{r} & number of rows the cell spans    & 1 \\
%   \K{c} & number of columns the cell spans & 1 \\
% \end{spectblr}
\begin{spectblr}[
  caption = {单元格合并键与键值},
  label = {key:cellspan},
]{}
  键 & 说明与可选键值 & 初始值 \\
  \K{r} & 合并行数    & 1 \\
  \K{c} & 合并列数    & 1 \\
\end{spectblr}

% \subsection{Cells and Spancells in New Interfaces}
\subsection{单元格与单元格合并新接口}

% Option \verb!cells! is for setting all cells.
\verb!cells! 选项用于设置所有单元格样式。
\nopagebreak
\begin{demohigh}
\begin{tblr}{hlines={white},cells={c,blue7}}
 Alpha   & Beta  & Gamma   & Delta   \\
 Epsilon & Zeta  & Eta     & Theta   \\
 Iota    & Kappa & Lambda  & Mu      \\
 Nu      & Xi    & Omicron & Pi      \\
\end{tblr}
\end{demohigh}

% Option \verb!cell{i}{j}! is for setting some cells.
\verb!cell{i}{j}! 选项用于设置指定单元格样式。

\begin{demohigh}
\begin{tblr}{
 hlines = {white},
 vlines = {white},
 cell{1,6}{odd} = {teal7},
 cell{1,6}{even} = {green7},
 cell{2,4}{1,4} = {red7},
 cell{3,5}{1,4} = {purple7},
 cell{2}{2} = {r=4,c=2}{c,azure7},
}
 Alpha   & Beta  & Gamma   & Delta   \\
 Epsilon & Zeta  & Eta     & Theta   \\
 Iota    & Kappa & Lambda  & Mu      \\
 Nu      & Xi    & Omicron & Pi      \\
 Rho     & Sigma & Tau     & Upsilon \\
 Phi     & Chi   & Psi     & Omega   \\
\end{tblr}
\end{demohigh}

% \subsection{Cells and Spancells in Old Interfaces}
\subsection{单元格与单元格合并旧接口}

% The \verb!\SetCell! command has a mandatory argument for setting the styles of current cell.
% The available keys are described in Table \ref{key:cell}.
\verb!\SetCell!命令的必选参数用于设置当前单元格的样式。
其键与键值的有效值见表 \ref{key:cell}.

\begin{demohigh}
\begin{tblr}{llll}
\hline[1pt]
 Alpha   & \SetCell{bg=teal2,fg=white} Beta & Gamma \\
\hline
 Epsilon & Zeta & \SetCell{r,font=\scshape} Eta \\
\hline
 Iota    & Kappa & Lambda \\
\hline[1pt]
\end{tblr}
\end{demohigh}

% The \verb!\SetCell! command also has an optional argument for setting the multispan of current cell.
% The available keys are described in Table \ref{key:cellspan}.
 \verb!\SetCell!命令也可以使用可选参数设置当前单元格需要合并的列数和行数。
其键和键值的有效值见表 \ref{key:cellspan}.

\begin{demohigh}
\begin{tblr}{|X|X|X|X|X|X|}
\hline
 Alpha & Beta & Gamma & Delta & Epsilon & Zeta \\
\hline
 \SetCell[c=2]{c} Eta & 2-2
              & \SetCell[c=2]{c} Iota & 2-4
                              & \SetCell[c=2]{c} Lambda  & 2-6 \\
\hline
 \SetCell[c=3]{c} Nu & 3-2 & 3-3
                      & \SetCell[c=3]{c} Pi & 3-5 & 3-6   \\
\hline
 \SetCell[c=6]{c} Tau & 4-2 & 4-3 & 4-4 & 4-5 & 4-6 \\
\hline
\end{tblr}
\end{demohigh}

\begin{demohigh}
\begin{tblr}{|X|X|X|X|X|X|}
\hline
 Alpha & Beta    & Gamma   & Delta & Epsilon & Zeta \\
\hline
 \SetCell[r=2]{m} Eta
       & Theta   & Iota    & Kappa & Lambda  & \SetCell[r=2]{m} Mu  \\
\hline
 Nu    & Xi      & Omicron & Pi    & Rho     & Sigma \\
\hline
\end{tblr}
\end{demohigh}

% In fact, table command \verb!\SetCell[<span>]{<styles>}! at the beginning of cell at row \verb!i!
% and column \verb!j! is the same as table option \verb!cell{i}{j}={<span>}{<styles>}!.
本质上,在第\verb!i!行、第\verb!j!列的单元格前使用的表格命令\verb!\SetCell[<span>]{<styles>}!
与表格选项\verb!cell{i}{j}={<span>}{<styles>}!的作用完全相同。

% Also, table command \verb!\SetCells[<span>]{<styles>}! at the beginning of some cell
% is the same as table option \verb!cells={<span>}{<styles>}!.
同样,表格命令 \verb!\SetCells[<span>]{<styles>}!
与表格选项 \verb!cells={<span>}{<styles>}!的作用完全相同。

% \section{Rows and Columns}
\section{行(\V{rows})和列(\V{columns})选项}

% All available keys for rows and columns are described in Table \ref{key:row} and Table \ref{key:column}.
行(\V{rows})和列(\V{columns})选项的键及键值的有效值见表 \ref{key:row}和表 \ref{key:column}。

% \begin{spectblr}[
%   caption = {Keys for Rows},
%   label = {key:row},
%   remark{Note} = {In most cases, you can omit the underlined key names and write only their values.}
% ]{}
%   Key & Description and Values & Initial Value \\
%   \underline{\K{halign}}
%     & horizontal alignment: \V{l} (left), \V{c} (center), or \V{r} (right)
%     & \V{l} \\
%   \underline{\K{valign}}
%     & vertical alignment: \V{t} (top), \V{m} (middle), \V{b} (bottom),
%       \V{h} (head) or \V{f} (foot)
%     & \V{t} \\
%   \underline{\K{ht}} & height dimension & None \\
%   \underline{\K{bg}} & background color name & None \\
%   \K{fg} & foreground color name & None \\
%   \K{font} & font commands & None \\
%   \K{abovesep} & set vertical space above the row & \V{2pt} \\
%   \K{abovesep+} & increase vertical space above the row & None \\
%   \K{belowsep} & set vertical space below the row & \V{2pt} \\
%   \K{belowsep+} & increase vertical space below the row & None \\
%   \K{rowsep} & set vertical space above and below the row & \V{2pt} \\
%   \K{rowsep+} & increase vertical space above and below the row & None \\
%   \K{preto} & prepend text to every cell (like \V{>} specifier in \K{rowspec}) & None \\
%   \K{appto} & append text to every cell (like \V{<} specifier in \K{rowspec}) & None \\
%   \K{cmd}   & execute command for every cell text & None \\
% \end{spectblr}
\begin{spectblr}[
  caption = {\V{rows}选项的键和键值},
  label = {key:row},
  remark{注意} = {多数情况下,对于带有下划线的键,可以省略键名而只给出键值。}
]{}
  键 & 说明与可选键值 & 初始值 \\
  \underline{\K{halign}}
    & 水平对齐方式: \V{l} (left)、 \V{c} (center)或\V{r} (right)
    & \V{l} \\
  \underline{\K{valign}}
    & 垂直对齐方式: \V{t} (top)、 \V{m} (middle)、 \V{b} (bottom)、
      \V{h} (head) 或 \V{f} (foot)
    & \V{t} \\
  \underline{\K{ht}} & 行高 & 无 \\
  \underline{\K{bg}} & 背景颜色 & 无 \\
  \K{fg} & 前景颜色 & 无 \\
  \K{font} & 字体命令 & 无 \\
  \K{abovesep} & 行前垂直间距 & \V{2pt} \\
  \K{abovesep+} & 行前垂直间距增量 & 无 \\
  \K{belowsep} & 行后垂直间距 & \V{2pt} \\
  \K{belowsep+} & 行后垂直间距增量 & 无 \\
  \K{rowsep} & 行前行后垂直间距 & \V{2pt} \\
  \K{rowsep+} & 行前行后垂直间距增量 & 无 \\
  \K{preto} & 单元格前导文本(如\K{rowspec}选项中的\V{>}) & 无 \\
  \K{appto} & 单元格附加文本(如\K{rowspec}选项中的\V{<}) & 无 \\
  \K{cmd}   & 对单元格文本要执行的命令 & 无 \\
\end{spectblr}
\vspace{-2em}
% \begin{spectblr}[
%   caption = {Keys for Columns},
%   label = {key:column},
%   remark{Note} = {In most cases, you can omit the underlined key names and write only their values.}
% ]{}
%   Key & Description and Values & Initial Value \\
%   \underline{\K{halign}}
%     & horizontal alignment: \V{l} (left), \V{c} (center), or \V{r} (right)
%     & \V{l} \\
%   \underline{\K{valign}}
%     & vertical alignment: \V{t} (top), \V{m} (middle), \V{b} (bottom),
%       \V{h} (head) or \V{f} (foot)
%     & \V{t} \\
%   \underline{\K{wd}} & width dimension & None \\
%   \underline{\K{co}} & coefficient for the extendable column (\V{X} column) & None \\
%   \underline{\K{bg}} & background color name & None \\
%   \K{fg} & foreground color name & None \\
%   \K{font} & font commands & None \\
%   \K{leftsep} & set horizontal space to the left of the column & \V{6pt} \\
%   \K{leftsep+} & increase horizontal space to the left of the column & None \\
%   \K{rightsep} & set horizontal space to the right of the column & \V{6pt} \\
%   \K{rightsep+} & increase horizontal space to the right of the column & None \\
%   \K{colsep} & set horizontal space to both sides of the column & \V{6pt} \\
%   \K{colsep+} & increase horizontal space to both sides of the column & None \\
%   \K{preto} & prepend text to every cell (like \V{>} specifier in \K{colspec}) & None \\
%   \K{appto} & append text to every cell (like \V{<} specifier in \K{colspec}) & None \\
%   \K{cmd}   & execute command for every cell text & None \\
% \end{spectblr}
\begin{spectblr}[
  caption = {\V{columns}选项的键和键值},
  label = {key:column},
  remark{注意} = {多数情况下,对于带有下划线的键,可以省略键名而只给出键值。}
]{}
  键 & 说明与可选键值 & 初始值 \\
  \underline{\K{halign}}
    & 水平对齐方式: \V{l} (left)、 \V{c} (center)或 \V{r} (right)
    & \V{l} \\
  \underline{\K{valign}}
    & 垂直对齐方式: \V{t} (top)、 \V{m} (middle)、 \V{b} (bottom)、
      \V{h} (head)或 \V{f} (foot)
    & \V{t} \\
  \underline{\K{wd}} & 列宽 & 无 \\
  \underline{\K{co}} & 可扩展列的扩展系数(\V{X}列) & 无 \\
  \underline{\K{bg}} & 背景颜色 & 无 \\
  \K{fg} & 前景颜色 & 无 \\
  \K{font} & 字体命令 & 无 \\
  \K{leftsep} & 列左侧水平间距 & \V{6pt} \\
  \K{leftsep+} & 列左侧水平间距增量 & 无 \\
  \K{rightsep} & 列右侧水平间距 & \V{6pt} \\
  \K{rightsep+} & 列右侧水平间距增量 & 无 \\
  \K{rowsep} & 列左右水平间距 & \V{6pt} \\
  \K{rowsep+} & 列左右水平间距增量 & 无 \\
  \K{preto} & 单元格前导文本(如\K{colspec}选项中的\V{>}) & 无 \\
  \K{appto} & 单元格附加文本(如\K{colspec}选项中的\V{<}) & 无 \\
  \K{cmd}   & 对单元格文本要执行的命令 & 无 \\
\end{spectblr}

% \subsection{Rows and Columns in New Interfaces}
\subsection{行和列设置新接口}

% Options \verb!rows! and \verb!columns! are for setting all rows and columns, respectively.
\verb!rows! 和 \verb!columns! 选项分别用于设置表格的所有行/列格式。
\nopagebreak
\begin{demohigh}
\begin{tblr}{
 hlines, vlines,
 rows = {7mm}, columns = {15mm,c},
}
 Alpha   & Beta  & Gamma   & Delta \\
 Epsilon & Zeta  & Eta     & Theta \\
 Iota    & Kappa & Lambda  & Mu    \\
\end{tblr}
\end{demohigh}

% Options \verb!row{i}! and \verb!column{j}! are for setting some rows and columns, respectively.
 \verb!row{i}! 和 \verb!column{j}! 选项分别用于设置表格中指定行/列格式。

\begin{demohigh}
\begin{tblr}{
 hlines = {1pt,white},
 row{odd} = {blue7},
 row{even} = {azure7},
 column{1} = {purple7,c},
}
 Alpha   & Beta  & Gamma   & Delta   \\
 Epsilon & Zeta  & Eta     & Theta   \\
 Iota    & Kappa & Lambda  & Mu      \\
 Nu      & Xi    & Omicron & Pi      \\
 Rho     & Sigma & Tau     & Upsilon \\
 Phi     & Chi   & Psi     & Omega   \\
\end{tblr}
\end{demohigh}

% The following example demonstrates the usages of \verb!bg!, \verb!fg! and \verb!font! keys.
下面的示例演示了 \verb!bg!、 \verb!fg! 和 \verb!font! 键的基本用法。
\nopagebreak
\begin{demohigh}
\begin{tblr}{
 row{odd} = {bg=azure8},
 row{1}   = {bg=azure3, fg=white, font=\sffamily},
}
 Alpha & Beta    & Gamma \\
 Delta & Epsilon & Zeta  \\
 Eta   & Theta   & Iota  \\
 Kappa & Lambda  & Mu    \\
 Nu Xi Omicron & Pi Rho Sigma & Tau Upsilon Phi \\
\end{tblr}
\end{demohigh}

% The following example demonstrates the usages of
% \verb!abovesep!, \verb!belowsep!, \verb!leftsep!, \verb!rightsep! keys.
下面的示例演示了\verb!abovesep!、 \verb!belowsep!、 \verb!leftsep!、 \verb!rightsep! 键的用法。
%\nopagebreak
\begin{demohigh}
\begin{tblr}{
 hlines, vlines,
 rows = {abovesep=1pt,belowsep=5pt},
 columns = {leftsep=1pt,rightsep=5pt},
}
 Alpha   & Beta  & Gamma  & Delta \\
 Epsilon & Zeta  & Eta    & Theta \\
 Iota    & Kappa & Lambda & Mu    \\
\end{tblr}
\end{demohigh}

% The following example shows that we can replace \verb!\\[dimen]! with \verb!belowsep+! key.
下面的示例说明了可以用\verb!belowsep+! 键替代 \verb!\\[dimen]!命令。

\begin{demohigh}
\begin{tblr}{
 hlines, row{2} = {belowsep+=5pt},
}
 Alpha   & Beta  & Gamma  & Delta \\
 Epsilon & Zeta  & Eta    & Theta \\
 Iota    & Kappa & Lambda & Mu    \\
\end{tblr}
\end{demohigh}

% \subsection{Rows and Columns in Old Interfaces}
\subsection{行和列设置旧接口}

% The \verb!\SetRow! command has a mandatory argument for setting the styles of current row.
% The available keys are described in Table \ref{key:row}.
可以通过\verb!\SetRow! 命令的必选参数设置当前行的格式。
其有效参数见表 \ref{key:row}.

\begin{demohigh}
\begin{tblr}{llll}
\hline[1pt]
 \SetRow{azure8} Alpha & Beta & Gamma & Delta \\
\hline
 \SetRow{blue8,c} Epsilon & Zeta & Eta & Theta \\
\hline
 \SetRow{violet8} Iota & Kappa & Lambda & Mu \\
\hline[1pt]
\end{tblr}
\end{demohigh}

% In fact, table command \verb!\SetRow{<styles>}! at the beginning of row \verb!i!
% is the same as table option \verb!row{i}={<styles>}!.
本质上,在第\verb!i!行开始的表格命令\verb!\SetRow{<styles>}!与
表格选项\verb!row{i}={<styles>}!的作用相同。

% Also, table command \verb!\SetRows{<styles>}! at the beginning of some row
% is the same as table option \verb!rows={<styles>}!.
另外,在一些行开始的表格命令\verb!\SetRows{<styles>}!与
表格选项\verb!rows={<styles>}!的作用相同。

% The usages of table commands \verb!\SetColumn! and \verb!\SetColumns!
% are similar to those of \verb!\SetRow! and \verb!\SetRows!, respectively.
% But normally you don't need to use them.
\verb!\SetColumn! 和 \verb!\SetColumns!表格命令的用法分别与
\verb!\SetRow! 和 \verb!\SetRows!命令类似。
但一般不直接使用这两个命令。

% \section{Colspec and Rowspec}
\section{\V{colspec}和\V{rowspec}选项}

% Options \verb!colspec!/\verb!rowspec! are for setting column/row specifications
% with column/row type specifiers.
\verb!colspec!/\verb!rowspec! 选项用于使用column/row格式参数设置column/row的格式。

% \subsection{Colspec and Width}
\subsection{\V{colspec}选项和\V{width}选项}

% Option \verb!width! are for setting the width of the table with extendable columns.
% The following example demonstrates the usage of \verb!width! option.
\verb!width! 选项用于设置具备可扩展列的表格的宽度。
下面的示例演示了 \verb!width! 选项的用法。
\nopagebreak
\begin{demohigh}
\begin{tblr}{width=0.8\textwidth, colspec={|l|X[2]|X[3]|X[-1]|}}
 Alpha   & Beta  & Gamma  & Delta \\
 Epsilon & Zeta  & Eta    & Theta \\
 Iota    & Kappa & Lambda & Mu    \\
\end{tblr}
\end{demohigh}

% \subsection{Column Types}
\subsection{列格式}

% The \verb!tabularray! package has only one type of primitive column: the \verb!Q! column.
% Other types of columns are defined as \verb!Q! columns with some keys.
\verb!Tabularray! 宏包仅设计了一个\verb!Q! 列格式的元格式。
其它的列格式都是通过为\verb!Q! 元格式指定不同的参数实现定义的。

\begin{codehigh}
\NewColumnType{l}{Q[l]}
\NewColumnType{c}{Q[c]}
\NewColumnType{r}{Q[r]}
\NewColumnType{t}[1]{Q[t,wd=#1]}
\NewColumnType{m}[1]{Q[m,wd=#1]}
\NewColumnType{b}[1]{Q[b,wd=#1]}
\NewColumnType{h}[1]{Q[h,wd=#1]}
\NewColumnType{f}[1]{Q[f,wd=#1]}
\NewColumnType{X}[1][]{Q[co=1,#1]}
\end{codehigh}

\begin{demohigh}
\begin{tblr}{|t{15mm}|m{15mm}|b{20mm}|}
 Alpha   & Beta  & {Gamma\\Gamma} \\
 Epsilon & Zeta  & {Eta\\Eta} \\
 Iota    & Kappa & {Lambda\\Lambda} \\
\end{tblr}
\end{demohigh}

% Any new column type must be defined with \verb!\NewColumnType! command.
% It can have an optional argument when it's defined.
任何新的列格式都需要使用 \verb!\NewColumnType! 命令进行定义。
在定义时,可以使用可选参数。

% \subsection{Row Types}
\subsection{行格式}

% The \verb!tabularray! package has only one type of primitive row: the \verb!Q! row.
% Other types of rows are defined as \verb!Q! rows with some keys.
同样,\verb!Tabularray! 宏包仅设计了一个\verb!Q! 行格式的元格式。
其它的行格式都是通过为\verb!Q! 元格式指定不同的参数实现定义的。

\begin{codehigh}
\NewRowType{l}{Q[l]}
\NewRowType{c}{Q[c]}
\NewRowType{r}{Q[r]}
\NewRowType{t}[1]{Q[t,ht=#1]}
\NewRowType{m}[1]{Q[m,ht=#1]}
\NewRowType{b}[1]{Q[b,ht=#1]}
\NewRowType{h}[1]{Q[h,ht=#1]}
\NewRowType{f}[1]{Q[f,ht=#1]}
\end{codehigh}

\begin{demohigh}
\begin{tblr}{rowspec={|t{12mm}|m{10mm}|b{10mm}|}}
 Alpha   & Beta  & {Gamma\\Gamma} \\
 Epsilon & Zeta  & {Eta\\Eta} \\
 Iota    & Kappa & {Lambda\\Lambda} \\
\end{tblr}
\end{demohigh}

% Any new row type must be defined with \verb!\NewRowType! command.
% It can have an optional argument when it's defined.
任何新的行格式都需要使用 \verb!\NewRowType! 命令进行定义。
在定义时,可以使用可选参数。

\end{document}
