% -*- coding: utf-8 -*-
% !TEX program = lualatex
\documentclass[oneside]{book}

% -*- coding: utf-8 -*-
% !TEX program = lualatex

\usepackage[heading=true,scheme=chinese]{ctex}
\ctexset{
  figurename = 图,
  tablename  = 表,
}

\usepackage[a4paper,margin=2.5cm]{geometry}

\usepackage{codehigh} % https://ctan.org/pkg/codehigh
\usepackage{tabularray}
\usepackage{array,multirow,amsmath}
\usepackage{chemmacros,environ}

\UseTblrLibrary{booktabs,diagbox,siunitx}

\usepackage{hyperref}
\hypersetup{
  colorlinks=true,
  urlcolor=blue3,
  linkcolor=red3,
}

\usepackage{tcolorbox}
\tcbset{sharp corners, boxrule=0.5pt, colback=red9}

\usepackage{float}

\setcounter{tocdepth}{1}

\newcommand*{\K}[1]{\texttt{#1}}
\newcommand*{\V}[1]{\texttt{#1}}

\DefTblrTemplate{caption-tag}{default}{表\hspace{0.25em}\thetable}
% \DefTblrTemplate{caption-sep}{default}{:\enskip}
\DefTblrTemplate{contfoot-text}{default}{续下页}
\DefTblrTemplate{conthead-text}{default}{(接前页)}
% \DefTblrTemplate{caption-text}{default}{\InsertTblrText{caption}}

\NewTblrEnviron{newtblr}
\SetTblrOuter[newtblr]{long}
\SetTblrInner[newtblr]{
  hlines = {white}, column{1,2} = {co=1}, colsep = 5pt,
  row{odd} = {brown8}, row{even} = {gray8},
  row{1} = {fg=white, bg=purple2, font=\bfseries\sffamily},
}

\NewTblrEnviron{spectblr}
\SetTblrOuter[spectblr]{long}
\SetTblrInner[spectblr]{
  hlines = {white}, column{2} = {co=1}, colsep = 5pt,
  row{odd} = {brown8}, row{even} = {gray8},
  row{1} = {fg=white, bg=purple2, font=\bfseries\sffamily},
  rowhead = 1,
}

% \newcommand{\mywarning}[1]{%
%   \begin{tcolorbox}
%   The interfaces in this #1 should be seen as
%   \textcolor{red3}{\bfseries experimental}
%   and are likely to change in future releases, if necessary.
%   Don’t use them in important documents.
%   \end{tcolorbox}
% }
\newcommand{\mywarning}[1]{%
  \begin{tcolorbox}
   #1 中说明的接口是
  \textcolor{red3}{\bfseries 实验性的}
  并且在未来可能会发生变化,
  因此,请勿在重要的文档中使用它们。
  \end{tcolorbox}
}

\renewcommand*{\thefootnote}{*}

\newcommand*{\myversion}{2021L}
\newcommand*{\mydate}{Version \myversion\ (\the\year-\mylpad\month-\mylpad\day)\\\myrepo}
\newcommand*{\myrepo}{\url{https://github.com/lvjr/tabularray}}
\newcommand*{\mylpad}[1]{\ifnum#1<10 0\the#1\else\the#1\fi}

\colorlet{highback}{\ifodd\month azure9\else blue9\fi}
\CodeHigh{language=latex/table,style/main=highback,style/code=highback}
\NewCodeHighEnv{code}{style/main=gray9,style/code=gray9}
\NewCodeHighEnv{demo}{style/main=gray9,style/code=gray9,demo}

%\CodeHigh{lite}

\CodeHigh{lite}
\setcounter{chapter}{2}

\begin{document}

% \chapter{Extra Interfaces}
\chapter{附加接口}

% \section{Table Specifications}
\section{表格参数}

% All available keys for the whole table are described in Table \ref{key:table}.
表格的整体设置参数的键与键值见表\ref{key:table}。

% \begin{spectblr}[
%   caption = {Keys for the Whole Table},
%   label = {key:table},
% ]{}
%   Key & Description and Values & Initial Value \\
%   \K{rulesep} & space between two hlines or vlines & \V{2pt} \\
%   \K{stretch} & stretch ratio for struts added to cell text & \V{1} \\
%   \K{abovesep} & set vertical space above every row & \V{2pt} \\
%   \K{belowsep} & set vertical space below every row & \V{2pt} \\
%   \K{rowsep} & set vertical space above and below every row & \V{2pt} \\
%   \K{leftsep} & set horizontal space to the left of every column & \V{6pt} \\
%   \K{rightsep} & set horizontal space to the right of every column & \V{6pt} \\
%   \K{colsep} & set horizontal space to both sides of every column & \V{6pt} \\
%   \K{hspan} & horizontal span algorithm: \V{default}, \V{even}, or \V{minimal} & \V{default} \\
%   \K{vspan} & vertical span algorithm: \V{default} or \V{even} & \V{default} \\
% \end{spectblr}
\begin{spectblr}[
  caption = {表格整体设置的键与键值},
  label = {key:table},
]{}
  键 & 说明与可选键值 & 初始值 \\
  \K{rulesep} & 两条表格横线或竖线间的间距 & \V{2pt} \\
  \K{stretch} & 单元格文本行距拉伸系数 & \V{1} \\
  \K{abovesep} & 每行前的垂直间距 & \V{2pt} \\
  \K{belowsep} & 每行后的垂直间距 & \V{2pt} \\
  \K{rowsep} & 每行前后的垂直间距 & \V{2pt} \\
  \K{leftsep} & 每列左边的水平间距 & \V{6pt} \\
  \K{rightsep} & 每列右边的水平间距 & \V{6pt} \\
  \K{colsep} & 每列左右的水平间距 & \V{6pt} \\
  \K{hspan} & 单元格水平合并算法: \V{default}、 \V{even}或 \V{minimal} & \V{default} \\
  \K{vspan} & 单元格垂直合并算法: \V{default}或 \V{even} & \V{default} \\
\end{spectblr}

% The following example shows that we can replace \verb!\doublerulesep! parameter with \verb!rulesep! key.
下面的示例演示了使用 \verb!rulesep! 键替代\verb!\doublerulesep!参数的方法。
\nopagebreak
\begin{demohigh}
\begin{tblr}{
 colspec={||llll||},rowspec={|QQQ|},rulesep=4pt,
}
 Alpha   & Beta  & Gamma  & Delta \\
 Epsilon & Zeta  & Eta    & Theta \\
 Iota    & Kappa & Lambda & Mu    \\
\end{tblr}
\end{demohigh}

% The following example shows that we can replace \verb!\arraystretch! parameter with \verb!stretch! key.
下面的示例演示了使用 \verb!stretch! 键替代 \verb!\arraystretch! 参数的方法。

\begin{demohigh}
\begin{tblr}{hlines,stretch=1.5}
 Alpha   & Beta  & Gamma  & Delta \\
 Epsilon & Zeta  & Eta    & Theta \\
 Iota    & Kappa & Lambda & Mu    \\
\end{tblr}
\end{demohigh}

% The following example uses \verb!rowsep! and \verb!colsep! keys to set padding for all rows and columns.
下面的示例演示了使用 \verb!rowsep! 和\verb!colsep!键设置行和列间距设置。
\nopagebreak
\begin{demohigh}
\SetTblrInner{rowsep=2pt,colsep=2pt}
\begin{tblr}{hlines,vlines}
 Alpha   & Beta  & Gamma  & Delta \\
 Epsilon & Zeta  & Eta    & Theta \\
 Iota    & Kappa & Lambda & Mu    \\
\end{tblr}
\end{demohigh}

% With \verb!hspan=default! or \verb!hspan=even!,
% \verb!tabularray! package will compute column widths from span widths.
% But with \verb!hspan=minimal!, it will compute span widths from column widths.
% The following examples show the results from different \verb!hspan! values.
使用\verb!hspan=default! 或 \verb!hspan=even!时,
\verb!Tabularray! 宏包会使用合并宽度计算列宽度。
但是,使用\verb!hspan=minimal!时,则使用列宽度计算合并宽度。
下面的示例演示了使用 \verb!hspan! 键的设置结果。

\begin{demohigh}
\SetTblrInner{hlines, vlines, hspan=default}
\begin{tblr}{cell{2}{1}={c=2}{l},cell{3}{1}={c=3}{l},cell{4}{2}={c=2}{l}}
 111 111 & 222 222 & 333 333 \\
 12 Multi Columns Multi Columns 12 & & 333 \\
 13 Multi Columns Multi Columns Multi Columns 13 & & \\
 111 & 23 Multi Columns Multi Columns 23 & \\
\end{tblr}
\end{demohigh}

\begin{demohigh}
\SetTblrInner{hlines, vlines, hspan=even}
\begin{tblr}{cell{2}{1}={c=2}{l},cell{3}{1}={c=3}{l},cell{4}{2}={c=2}{l}}
 111 111 & 222 222 & 333 333 \\
 12 Multi Columns Multi Columns 12 & & 333 \\
 13 Multi Columns Multi Columns Multi Columns 13 & & \\
 111 & 23 Multi Columns Multi Columns 23 & \\
\end{tblr}
\end{demohigh}

\begin{demohigh}
\SetTblrInner{hlines, vlines, hspan=minimal}
\begin{tblr}{cell{2}{1}={c=2}{l},cell{3}{1}={c=3}{l},cell{4}{2}={c=2}{l}}
 111 111 & 222 222 & 333 333 \\
 12 Multi Columns Multi Columns 12 & & 333 \\
 13 Multi Columns Multi Columns Multi Columns 13 & & \\
 111 & 23 Multi Columns Multi Columns 23 & \\
\end{tblr}
\end{demohigh}

% The following examples show the results from different \verb!vspan! values.
下面的示例演示了使用 \verb!vspan! 键的设置结果。
\nopagebreak
\begin{demohigh}
\SetTblrInner{hlines, vlines, vspan=default}
\begin{tblr}{column{2}={3.25cm}, cell{2}{2}={r=3}{l}}
  Column1 & Column2 \\
  Row1 & Long text that needs multiple lines.
         Long text that needs multiple lines.
         Long text that needs multiple lines. \\
  Row2 & \\
  Row3 & \\
  Row4 & Short text \\
\end{tblr}
\end{demohigh}

\begin{demohigh}
\SetTblrInner{hlines, vlines, vspan=even}
\begin{tblr}{column{2}={3.25cm}, cell{2}{2}={r=3}{l}}
  Column1 & Column2 \\
  Row1 & Long text that needs multiple lines.
         Long text that needs multiple lines.
         Long text that needs multiple lines. \\
  Row2 & \\
  Row3 & \\
  Row4 & Short text \\
\end{tblr}
\end{demohigh}

% \section{Default Specifications}
\section{参数默认值}

% \verb!Tabularray! package provides \verb!\SetTblrInner! and \verb!\SetTblrOuter! commands
% for you to change the default inner and outer specifications of tables.
% Inner specifications are all specifications written in the mandatory argument of the \verb!tblr! environment,
% while outer specifications are all specifications written in the optional argument of the \verb!tblr! environment.
% At this time, most of the outer specifications are used for long tables (see Chapter \ref{chap:long}).
\verb!Tabularray! 宏包提供了 \verb!\SetTblrInner! 和 \verb!\SetTblrOuter! 两个命令,
这两个命令用于设置表格的内部和外部参数的默认值。
内部参数是指在 \verb!tblr! 环境的必选参数,
而外部参数是指在 \verb!tblr! 环境的可选参数。
目前,本宏包的外部参数主要用于长表格的参数设置(参见第\ref{chap:long}章).

% In the below example, the first line draws all hlines and vlines for all tables created afterwards,
% while the second line makes all tables created afterwards vertically align at bottom.
下面的示例中,第一行代码用于设置此后所有表格都绘制表格横线和竖线,
第二行代码用于设置此后所有表格的垂直对齐方式为底端基线对齐。

\begin{codehigh}
\SetTblrInner{hlines,vlines}
\SetTblrOuter{valign=b}
\end{codehigh}

% You can define new \verb!tabularray! environments using \verb!\NewTblrEnviron! command:
可以使用 \verb!\NewTblrEnviron! 命令定义新的 \verb!Tabularray! 环境:

\begin{demohigh}
\NewTblrEnviron{mytblr}
\SetTblrInner[mytblr]{hlines,vlines}
\SetTblrOuter[mytblr]{valign=b}
Text \begin{mytblr}{cccc}
 Alpha   & Beta  & Gamma  & Delta \\
 Epsilon & Zeta  & Eta    & Theta \\
 Iota    & Kappa & Lambda & Mu    \\
\end{mytblr} Text
\end{demohigh}

% If not giving the optional argument to \verb!\SetTblrInner! or \verb!\SetTblrOuter! command,
% we set the default specifications for \verb!tblr! environment.
% And different tabularray environments could have different default specifications.
如果在 \verb!\SetTblrInner! 或 \verb!\SetTblrOuter! 命令中未指定可选参数,
则是为 \verb!tblr! 环境指定默认参数。
因此,不同的 \verb!Tabularray! 环境可以具有不同的默认参数。

% \section{New Table Commands}
\section{定义表格新命令}

% All commands which change the specifications of tables \textcolor{red3}{must} be defined with \verb!\NewTableCommand!.
% The following example demonstrates how to define a new table command:
\textcolor{red3}{必须} 使用 \verb!\NewTableCommand! 命令定义所有用于改变表格参数的命令 。
下面的示例演示了如何定义一个新的表格命令:

\begin{demohigh}
\NewTableCommand\myhline{\hline[0.1em,red5]}
\begin{tblr}{llll}
\myhline
 Alpha   & Beta  & Gamma   & Delta \\
 Epsilon & Zeta  & Eta     & Theta \\
 Iota    & Kappa & Lambda  & Mu    \\
\myhline
\end{tblr}
\end{demohigh}

% \section{Expand Macros First}
\section{提前展开表格内容宏}

% \verb!Tabularray! need to see every \verb!&! and \verb!\\! when splitting the table body
% with \verb!l3regex!. And you can not put cell text inside any table command defined with
% \verb!\NewTableCommand!. But you could use outer specification \verb!expand! to make \verb!tabularray!
% expand every occurrence of a specified macro \underline{once} before splitting the table body:
\verb!Tabularray! 宏包使用\verb!l3regex!解析表格内容,因此,需要知道每一个 \verb!&! 和 \verb!\\! 符号。
所以,不能将表格内容放在任何由 \verb!\NewTableCommand! 命令定义的表格命令中。
但是,可以使用 \verb!expand! 选项为 \verb!Tabularray! 宏包指定在解析表格内容之前
\underline{一次}性展开包含表格内容的宏。

\begin{demohigh}
\def\tblrbody{
 \hline
  20 & 30 & 40 \\
  50 & 60 & 70 \\
 \hline
}
\begin{tblr}[expand=\tblrbody]{ccc}
 \hline
  AA & BB & CC \\
  \tblrbody
  DD & EE & FF \\
  \tblrbody
  GG & HH & II \\
 \hline
\end{tblr}
\end{demohigh}

% With this \verb!expand! option, you can also use \verb!environ! package
% to define a new environment based on \verb!tblr! environment:
通过 \verb!expand! 选项,可以利用 \verb!environ! 宏包
定义基于 \verb!tblr! 环境的新环境:


\begin{demohigh}
\NewEnviron{fancytblr}{
 Before Text
 \begin{tblr}[expand=\BODY]{hlines}
   \BODY
 \end{tblr}
 After Text
}
\begin{fancytblr}
  One   & Two   & Three \\
  Four  & Five  & Six   \\
  Seven & Eight & Nine  \\
\end{fancytblr}
\end{demohigh}

% \section{Use Verbatim Commands}
\section{原文排版命令}

% With inner specification \verb!verb!, you can write \verb!\verb! commands in the cell text:
通过 \verb!verb! 参数,可以在单元格文本中使用 \verb!\verb! 原文排版命令:

\begin{demohigh}
\begin{tblr}{hlines,verb}
  20 & 30 & \verb!\hello{world}!40 \\
  50 & \verb!\hello!60 & 70 \\
\end{tblr}
\end{demohigh}

% \section{Counters and Lengths}
\section{计数器和长度}

% Counters \verb!rownum!, \verb!colnum!, \verb!rowcount!, \verb!colcount! can be used in cell text:
可以在单元格文本中使用\verb!rownum!、 \verb!colnum!、 \verb!rowcount!和 \verb!colcount!  计数器:
\nopagebreak
\begin{demohigh}
\begin{tblr}{hlines}
 Cell[\arabic{rownum}][\arabic{colnum}] & Cell[\arabic{rownum}][\arabic{colnum}] &
 Cell[\arabic{rownum}][\arabic{colnum}] & Cell[\arabic{rownum}][\arabic{colnum}] \\
 Row=\arabic{rowcount}, Col=\arabic{colcount} &
 Row=\arabic{rowcount}, Col=\arabic{colcount} &
 Row=\arabic{rowcount}, Col=\arabic{colcount} &
 Row=\arabic{rowcount}, Col=\arabic{colcount} \\
 Cell[\arabic{rownum}][\arabic{colnum}] & Cell[\arabic{rownum}][\arabic{colnum}] &
 Cell[\arabic{rownum}][\arabic{colnum}] & Cell[\arabic{rownum}][\arabic{colnum}] \\
\end{tblr}
\end{demohigh}

% Also, lengths \verb!\leftsep!, \verb!\rightsep!, \verb!\abovesep!, \verb!\belowsep! can be used in cell text.
也可以在单元格文本中也可以使用
\verb!\leftsep!、 \verb!\rightsep!、 \verb!\abovesep!和 \verb!\belowsep!长度。

% \section{Tracing Tabularray}
\section{跟踪Tabularray}

% To trace internal data behind \verb!tblr! environment, you can use \verb!\SetTblrTracing! command.
% For example, \verb!\SetTblrTracing{all}! will turn on all tracings,
% and \verb!\SetTblrTracing{none}! will turn off all tracings.
% \verb!\SetTblrTracing{+row,+column}! will only tracing row and column data.
% All tracing messages will be written to the log files.
可以使用 \verb!\SetTblrTracing! 命令跟踪 \verb!tblr! 环境的内部数据流。
例如,\verb!\SetTblrTracing{all}! 用于打开所有跟踪,
\verb!\SetTblrTracing{none}! 用于关闭所有跟踪。
而\verb!\SetTblrTracing{+row,+column}! 仅跟踪指定的行和列的数据流。
所有的跟踪结果都将写入log文件。

\end{document}
