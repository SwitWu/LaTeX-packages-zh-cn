% -*- coding: utf-8 -*-
% !TEX program = lualatex
\documentclass[oneside]{book}

% -*- coding: utf-8 -*-
% !TEX program = lualatex

\usepackage[heading=true,scheme=chinese]{ctex}
\ctexset{
  figurename = 图,
  tablename  = 表,
}

\usepackage[a4paper,margin=2.5cm]{geometry}

\usepackage{codehigh} % https://ctan.org/pkg/codehigh
\usepackage{tabularray}
\usepackage{array,multirow,amsmath}
\usepackage{chemmacros,environ}

\UseTblrLibrary{booktabs,diagbox,siunitx}

\usepackage{hyperref}
\hypersetup{
  colorlinks=true,
  urlcolor=blue3,
  linkcolor=red3,
}

\usepackage{tcolorbox}
\tcbset{sharp corners, boxrule=0.5pt, colback=red9}

\usepackage{float}

\setcounter{tocdepth}{1}

\newcommand*{\K}[1]{\texttt{#1}}
\newcommand*{\V}[1]{\texttt{#1}}

\DefTblrTemplate{caption-tag}{default}{表\hspace{0.25em}\thetable}
% \DefTblrTemplate{caption-sep}{default}{:\enskip}
\DefTblrTemplate{contfoot-text}{default}{续下页}
\DefTblrTemplate{conthead-text}{default}{(接前页)}
% \DefTblrTemplate{caption-text}{default}{\InsertTblrText{caption}}

\NewTblrEnviron{newtblr}
\SetTblrOuter[newtblr]{long}
\SetTblrInner[newtblr]{
  hlines = {white}, column{1,2} = {co=1}, colsep = 5pt,
  row{odd} = {brown8}, row{even} = {gray8},
  row{1} = {fg=white, bg=purple2, font=\bfseries\sffamily},
}

\NewTblrEnviron{spectblr}
\SetTblrOuter[spectblr]{long}
\SetTblrInner[spectblr]{
  hlines = {white}, column{2} = {co=1}, colsep = 5pt,
  row{odd} = {brown8}, row{even} = {gray8},
  row{1} = {fg=white, bg=purple2, font=\bfseries\sffamily},
  rowhead = 1,
}

% \newcommand{\mywarning}[1]{%
%   \begin{tcolorbox}
%   The interfaces in this #1 should be seen as
%   \textcolor{red3}{\bfseries experimental}
%   and are likely to change in future releases, if necessary.
%   Don’t use them in important documents.
%   \end{tcolorbox}
% }
\newcommand{\mywarning}[1]{%
  \begin{tcolorbox}
   #1 中说明的接口是
  \textcolor{red3}{\bfseries 实验性的}
  并且在未来可能会发生变化,
  因此,请勿在重要的文档中使用它们。
  \end{tcolorbox}
}

\renewcommand*{\thefootnote}{*}

\newcommand*{\myversion}{2021L}
\newcommand*{\mydate}{Version \myversion\ (\the\year-\mylpad\month-\mylpad\day)\\\myrepo}
\newcommand*{\myrepo}{\url{https://github.com/lvjr/tabularray}}
\newcommand*{\mylpad}[1]{\ifnum#1<10 0\the#1\else\the#1\fi}

\colorlet{highback}{\ifodd\month azure9\else blue9\fi}
\CodeHigh{language=latex/table,style/main=highback,style/code=highback}
\NewCodeHighEnv{code}{style/main=gray9,style/code=gray9}
\NewCodeHighEnv{demo}{style/main=gray9,style/code=gray9,demo}

%\CodeHigh{lite}

\CodeHigh{lite}
\setcounter{chapter}{4}

\begin{document}

% \chapter{Use Some Libraries}
\chapter{使用扩展库}

\mywarning{本章}

% The \verb!tabularray! package emulates or fixes some commands in other packages.
% To avoid potential conflict, you need to enable them with \verb!\UseTblrLibrary! command.
 \verb!Tabularray! 宏包模仿或修改了其它宏包的一些命令,
为避免冲突,需要使用 \verb!\UseTblrLibrary! 载入这些扩展库。

% \section{Library \texttt{booktabs}}
\section{\texttt{booktabs}库}

% When you write \verb!\UseTblrLibrary{booktabs}!,
% \verb!tabularray! package will define commands \verb!\toprule!, \verb!\midrule!,
% \verb!\bottomrule! and \verb!\cmidrule! inside \verb!tblr! environment.
如果使用了 \verb!\UseTblrLibrary{booktabs}!,
则\verb!Tabularray! 宏包将定义 \verb!\toprule!、 \verb!\midrule!、
\verb!\bottomrule! 和 \verb!\cmidrule! 命令,
这些命令可以直接用于 \verb!tblr! 环境中。

\begin{demohigh}
\begin{tblr}{llll}
\toprule
 Alpha   & Beta  & Gamma   & Delta \\
\midrule
 Epsilon & Zeta  & Eta     & Theta \\
\cmidrule{1-3}
 Iota    & Kappa & Lambda  & Mu    \\
\cmidrule{2-4}
 Nu      & Xi    & Omicron & Pi    \\
\bottomrule
\end{tblr}
\end{demohigh}

% At this moment, \verb!trim! options for \verb!\cmidrule! command are not supported.
% (As a workaround, you may insert an empty column to separate two \verb!\cmidrule!'s.)
% But rule colors are possible just like \verb!\hline! and \verb!\cline! commands.
目前, \verb!\cmidrule! 命令还不支持\verb!trim! 选项
(一种变通方法是插入一个空列分隔两个 \verb!\cmidrule!)。
但是,可以用类似 \verb!\hline! 和 \verb!\cline! 命令的方式指定线条颜色。

\begin{demohigh}
\begin{tblr}{llll}
\toprule[purple3]
 Alpha   & Beta  & Gamma   & Delta \\
\midrule[blue3]
 Epsilon & Zeta  & Eta     & Theta \\
\cmidrule[azure3]{1-3}
 Iota    & Kappa & Lambda  & Mu    \\
\cmidrule[azure3]{2-4}
 Nu      & Xi    & Omicron & Pi    \\
\bottomrule[purple3]
\end{tblr}
\end{demohigh}

% \section{Library \texttt{diagbox}}
\section{ \texttt{diagbox}库}

% When writing \verb!\UseTblrLibrary{diagbox}! in the preamble of the document,
% \verb!tabularray! package loads \verb!diagbox! package,
% and you can use \verb!\diagbox! and \verb!\diagboxthree! commands inside \verb!tblr! environment.
当在导言区使用了\verb!\UseTblrLibrary{diagbox}!后,
\verb!Tabularray! 宏包会载入 \verb!diagbox! 宏包,
然后,就可以在\verb!tblr!环境中使用 \verb!\diagbox! 和\verb!\diagboxthree! 命令排版斜线表头。

\begin{demohigh}
\begin{tblr}{hlines,vlines}
 \diagbox{Aa}{Pp} & Beta & Gamma \\
 Epsilon & Zeta  & Eta \\
 Iota    & Kappa & Lambda \\
\end{tblr}
\end{demohigh}

\begin{demohigh}
\begin{tblr}{hlines,vlines}
 \diagboxthree{Aa}{Pp}{Hh} & Beta & Gamma \\
 Epsilon & Zeta  & Eta \\
 Iota    & Kappa & Lambda \\
\end{tblr}
\end{demohigh}

% You can also use \verb!\diagbox! and \verb!\diagboxthree! commands in math mode.
也可以在数学模式中使用 \verb!\diagbox!和 \verb!\diagboxthree! 命令。
\nopagebreak
\begin{demohigh}
$\begin{tblr}{|c|cc|}
\hline
 \diagbox{X_1}{X_2} & 0 & 1 \\
\hline
  0 & 0.1 & 0.2 \\
  1 & 0.3 & 0.4 \\
\hline
\end{tblr}$
\end{demohigh}

% \section{Library \texttt{siunitx}}
\section{ \texttt{siunitx}库}

% When writing \verb!\UseTblrLibrary{siunitx}! in the preamble of the document,
% \verb!tabularray! package loads \verb!siunitx! package,
% and defines \verb!S! column as \verb!Q! column with \verb!si! key.
当在导言区使用了\verb!\UseTblrLibrary{siunitx}!后,
\verb!Tabularray! 宏包会载入 \verb!siunitx! 宏包,
并定义了\verb!S!列格式,表示带有\verb!si!键的\verb!Q!列格式。

\begin{demohigh}
\begin{tblr}{
  hlines, vlines,
  colspec={
    S[table-format=3.2]
    S[table-format=3.2]
    S[table-format=3.2]
  }
}
 {{{Head}}} & {{{Head}}} & {{{Head}}} \\
   111      &   111      &   111      \\
     2.1    &     2.2    &     2.3    \\
    33.11   &    33.22   &    33.33   \\
\end{tblr}
\end{demohigh}

\begin{demohigh}
\begin{tblr}{
  hlines, vlines,
  colspec={
    Q[si={table-format=3.2},c]
    Q[si={table-format=3.2},c]
    Q[si={table-format=3.2},c]
  }
}
 {{{Head}}} & {{{Head}}} & {{{Head}}} \\
   111      &   111      &   111      \\
     2.1    &     2.2    &     2.3    \\
    33.11   &    33.22   &    33.33   \\
\end{tblr}
\end{demohigh}

% Note that you need to use \underline{triple} pairs of braces to guard non-numeric cells.
注意,需要使用 \underline{三重} 大括号对以确保单元格是非数字模式。

% Also you must use \verb!l!, \verb!c! or \verb!r! to set horizontal alignment for non-numeric cells:
另外,也必须使用 \verb!l!、 \verb!c! 或 \verb!r! 设置非数字单元格的水平对齐方式。

\begin{demohigh}
\begin{tblr}{
  hlines, vlines, columns={6em},
  colspec={
    Q[si={table-format=3.2,table-number-alignment=left},l,blue7]
    Q[si={table-format=3.2,table-number-alignment=center},c,teal7]
    Q[si={table-format=3.2,table-number-alignment=right},r,purple7]
  }
}
 {{{Head}}} & {{{Head}}} & {{{Head}}} \\
   111      &   111      &   111      \\
     2.1    &     2.2    &     2.3    \\
    33.11   &    33.22   &    33.33   \\
\end{tblr}
\end{demohigh}

% Both \verb!S! and \verb!s! columns are supported. In fact, These two columns are defined as follows:
此时,\verb!S! 和 \verb!s! 列格式都可用。
实质上,这两个列格式是按如下方式定义的:
\begin{codehigh}
\NewColumnType{S}[1][]{Q[si={##1},c]}
\NewColumnType{s}[1][]{Q[si={##1},c,cmd=\TblrUnit]}
\end{codehigh}

\end{document}
