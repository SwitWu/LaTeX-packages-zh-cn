%\documentclass{standalone}
\documentclass{article}
\usepackage{ctex}
\usepackage{nicematrix}
\usepackage{tikz}
\usetikzlibrary{arrows.meta,fit,positioning,quotes,calc,patterns}
\begin{document}

\newcommand{\myCellSpaceTopLimit}{1cm}
\newcommand{\myCellSpaceBottomLimit}{.8cm}
\setlength{\arraycolsep}{1.5cm}
\pagestyle{empty}
%\begin{figure}[htbp]
%  \centering
  \[
  \begin{NiceMatrix}[name=CellSpace,cell-space-top-limit = \myCellSpaceTopLimit,cell-space-bottom-limit = \myCellSpaceBottomLimit,create-medium-nodes,create-large-nodes,hvlines,rules/color=gray,baseline=c,
    code-after = {\begin{tikzpicture}
				\node [blend mode = multiply,fill = red!15,inner sep = 0 pt, fit = (1-1)] {} ;
\node [blend mode = multiply,fill = red!15,inner sep = 0 pt,fit = (1-2)] {} ;
\node [blend mode = multiply,fill = red!15,inner sep = 0 pt,fit = (2-1)] {} ;
\node [blend mode = multiply,fill = red!15,inner sep = 0 pt,fit = (2-2)] {} ;
%	蓝色行底线(第1行)
\draw[blue,dashed] ($(2-|1)+(0,\myCellSpaceBottomLimit)$) -- ($(2-|3)+(0,\myCellSpaceBottomLimit)$);
%	蓝色行底线(第2行)
\draw[blue,dashed] ($(3-|1)+(0,\myCellSpaceBottomLimit)$) -- ($(3-|3)+(0,\myCellSpaceBottomLimit)$);
%	红色行顶线(第1行)
\draw[red,dashed] ($(1-|1)+(0,-\myCellSpaceTopLimit)$) -- ($(1-|3)+(0,-\myCellSpaceTopLimit)$);
%	红色行顶线(第2行)
\draw[red,dashed] ($(2-|1)+(0,-\myCellSpaceTopLimit)$) -- ($(2-|3)+(0,-\myCellSpaceTopLimit)$);
%	红色箭头+标注(第1行),单元格顶部空白
\draw ($(1-|3)$) edge["单元格顶部空白" right,xshift=-.1cm,red,Latex-Latex]  ($(1-|3)+(0,-\myCellSpaceTopLimit)$);
%	红色箭头+标注(第2行),单元格顶部空白
\draw ($(2-|3)$) edge["单元格顶部空白" right,xshift=-.1cm,red,Latex-Latex]  ($(2-|3)+(0,-\myCellSpaceTopLimit)$);
%	蓝色箭头+标注(第1行),单元格底部空白
\draw ($(2-|3)$) edge["单元格底部空白" right,xshift=-.1cm,blue,Latex-Latex]  ($(2-|3)+(0,\myCellSpaceBottomLimit)$);
%	蓝色箭头+标注(第2行),单元格底部空白
\draw ($(3-|3)$) edge["单元格底部空白" right,xshift=-.1cm,blue,Latex-Latex]  ($(3-|3)+(0,\myCellSpaceBottomLimit)$);
    \end{tikzpicture}}
    ]
    A & a + b \\
    a & B
  \end{NiceMatrix}
  \]
%  \caption{阵列水平(列间)空白示意图}
%  \label{fig:阵列水平(列间)空白示意图}
%\end{figure}

\end{document}