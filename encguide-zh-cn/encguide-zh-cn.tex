% \iffalse meta-comment
%
% Copyright (C) 1993-2022
% The LaTeX Project and any individual authors listed elsewhere
% in this file.
%
% This file is part of the LaTeX base system.
% -------------------------------------------
%
% It may be distributed and/or modified under the
% conditions of the LaTeX Project Public License, either version 1.3c
% of this license or (at your option) any later version.
% The latest version of this license is in
%    http://www.latex-project.org/lppl.txt
% and version 1.3c or later is part of all distributions of LaTeX
% version 2008 or later.
%
% This file has the LPPL maintenance status "maintained".
%
% The list of all files belonging to the LaTeX base distribution is
% given in the file `manifest.txt'. See also `legal.txt' for additional
% information.
%
% The list of derived (unpacked) files belonging to the distribution
% and covered by LPPL is defined by the unpacking scripts (with
% extension .ins) which are part of the distribution.
%
% \fi
%


\NeedsTeXFormat{LaTeX2e}[1995/12/01]

\documentclass{ltxguide}[1994/11/20]
\usepackage{ctex}
\usepackage{indentfirst}
\setlength{\parindent}{2em}
\usepackage{longtable} %%%做长表格
\usepackage{booktabs} %%%提供了\toprule,\midrule 这样用来画表格横线的指令(感觉比\hline好用一点)
\usepackage[T1]{fontenc}
\IfFileExists{lmodern.sty}{\usepackage{lmodern}}{}
\usepackage{textcomp}
\usepackage{url}
\usepackage{mflogo}
\usepackage[colorlinks,linkcolor=red]{hyperref}

\addtolength\textheight{6\baselineskip}
\addtolength\topmargin{-2\baselineskip}


\newcommand\ttverb[1]{\texttt{\string#1}}


% for encodings
\providecommand{\Enc}[1]{\texttt{#1}}

% for packages
\providecommand{\Pkg}[1]{%
  \textsf{#1}}

% for files
\providecommand{\File}[1]{%
  \texttt{#1}}

% let's have meta values too
\providecommand{\meta}[1]{%
  \ensuremath{\langle}\emph{#1}\ensuremath{\rangle}}

\usepackage{tabularx}

% eine Umgebung zur Darstellung von Kodierungen
%
% Argumente:
%  #1: Name in LaTeX (z.B. OT1)
%  #2: Name der Kodierung (z.B. TeX text)
%  #3: Name des Autors (z.B. Don Knuth)
%  #4: Bereich der benützten Glyphindizes
%  #5: variable Positionen
%  #6: Beispielzeichensatz
%  #7: Referenz
%
% XXX add code to handle more than a single font example (e.g., larm1000,
% lbrm1000, and lcrm1000).
%
\newenvironment{encodinginfo}[7]%
  {\noindent
%   \begin{tabularx}{\textwidth}{l@{}l>{\raggedright\let\\\tabularnewline}X}%
   \begin{tabularx}{1.11\textwidth}{|rX|lX|}%
     \hline
     \LaTeX{}\ 名称(\LaTeX{} name):          & \texttt{#1}\\\hline%
     公共名称(Public name):          & #2\\\hline%
     作者(Author):                   & #3\\\hline%
     使用的字形槽(Glyph slots):& #4\\\hline%
     变量槽(Variable slots):& #5\\\hline%
     字体示例(Font example):& \def\@tempa{#6}\ifx\@tempa\@empty---%
                            \else\texttt{#6}\referenceftable{#6}\fi\\\hline%
     进一步参考(Further reference): & #7\\\hline%
   \end{tabularx}%
   \par\nobreak
   \vspace*{3pt}%
   \quote
  }%
  {\endquote
   \vspace{6pt}}

\makeatletter
\def\referenceftable#1{
  \@ifundefined{r@fonttable:#1}%
  \relax
  {;\space 编码表(encoding table)在第~\pageref{fonttable:#1}~页}%
}

% font table macros mainly lifted from manmac.tex
\def\oct#1{\hbox{\rm\'{}\kern-.2em\it#1\/\kern.05em}}
\def\hex#1{\hbox{\rm\H{}\tt#1}}

\def\oddline#1{\cr\noalign{\nointerlineskip}
  \multispan{19}\hrulefill&
  \setbox0=\hbox{\lower 2.3pt\hbox{\hex{#1x}}}\smash{\box0}\cr
  \noalign{\nointerlineskip}}
\def\evenline{\cr\noalign{\hrule}}
\def\chartstrut{\lower4.5pt\vbox to14pt{}}
\def\beginchart#1#2{$$\global\count@=0 #1
  \halign to\hsize\bgroup
    \chartstrut##\tabskip0pt plus10pt&
    &\hfil##\hfil&\vrule##\cr
    \lower6.5pt\null
  &#2&&\oct0&&\oct1&&\oct2&&\oct3&&\oct4&&\oct5&&\oct6&&\oct7&\evenline}
\def\endchart{\raise11.5pt\null&&&\hex 8&&\hex 9&&\hex A&&\hex B&
  &\hex C&&\hex D&&\hex E&&\hex F&\cr\egroup$$}
\def\:{\setbox0=\hbox{\noboundary\char\count@\noboundary}%
  \ifdim\ht0>7.5pt\reposition
  \else\ifdim\dp0>2.5pt\reposition\fi\fi
  \box0\global\advance\count@ by1 }
\def\reposition{\setbox0=\hbox{$\vcenter{\kern2pt\box0\kern2pt}$}}
\def\normalchart{%
  &\oct{00x}&&\:&&\:&&\:&&\:&&\:&&\:&&\:&&\:&&\oddline0
  &\oct{01x}&&\:&&\:&&\:&&\:&&\:&&\:&&\:&&\:&\evenline
  &\oct{02x}&&\:&&\:&&\:&&\:&&\:&&\:&&\:&&\:&&\oddline1
  &\oct{03x}&&\:&&\:&&\:&&\:&&\:&&\:&&\:&&\:&\evenline
  &\oct{04x}&&\:&&\:&&\:&&\:&&\:&&\:&&\:&&\:&&\oddline2
  &\oct{05x}&&\:&&\:&&\:&&\:&&\:&&\:&&\:&&\:&\evenline
  &\oct{06x}&&\:&&\:&&\:&&\:&&\:&&\:&&\:&&\:&&\oddline3
  &\oct{07x}&&\:&&\:&&\:&&\:&&\:&&\:&&\:&&\:&\evenline
  &\oct{10x}&&\:&&\:&&\:&&\:&&\:&&\:&&\:&&\:&&\oddline4
  &\oct{11x}&&\:&&\:&&\:&&\:&&\:&&\:&&\:&&\:&\evenline
  &\oct{12x}&&\:&&\:&&\:&&\:&&\:&&\:&&\:&&\:&&\oddline5
  &\oct{13x}&&\:&&\:&&\:&&\:&&\:&&\:&&\:&&\:&\evenline
  &\oct{14x}&&\:&&\:&&\:&&\:&&\:&&\:&&\:&&\:&&\oddline6
  &\oct{15x}&&\:&&\:&&\:&&\:&&\:&&\:&&\:&&\:&\evenline
  &\oct{16x}&&\:&&\:&&\:&&\:&&\:&&\:&&\:&&\:&&\oddline7
  &\oct{17x}&&\:&&\:&&\:&&\:&&\:&&\:&&\:&&\:&\evenline
  \top}

\def\notophalf{}
\def\tophalf{%
%\noalign{\vskip 5pt\hrule}
  &\oct{20x}&&\:&&\:&&\:&&\:&&\:&&\:&&\:&&\:&&\oddline8
  &\oct{21x}&&\:&&\:&&\:&&\:&&\:&&\:&&\:&&\:&\evenline
  &\oct{22x}&&\:&&\:&&\:&&\:&&\:&&\:&&\:&&\:&&\oddline9
  &\oct{23x}&&\:&&\:&&\:&&\:&&\:&&\:&&\:&&\:&\evenline
  &\oct{24x}&&\:&&\:&&\:&&\:&&\:&&\:&&\:&&\:&&\oddline A
  &\oct{25x}&&\:&&\:&&\:&&\:&&\:&&\:&&\:&&\:&\evenline
  &\oct{26x}&&\:&&\:&&\:&&\:&&\:&&\:&&\:&&\:&&\oddline B
  &\oct{27x}&&\:&&\:&&\:&&\:&&\:&&\:&&\:&&\:&\evenline
  &\oct{30x}&&\:&&\:&&\:&&\:&&\:&&\:&&\:&&\:&&\oddline C
  &\oct{31x}&&\:&&\:&&\:&&\:&&\:&&\:&&\:&&\:&\evenline
  &\oct{32x}&&\:&&\:&&\:&&\:&&\:&&\:&&\:&&\:&&\oddline D
  &\oct{33x}&&\:&&\:&&\:&&\:&&\:&&\:&&\:&&\:&\evenline
  &\oct{34x}&&\:&&\:&&\:&&\:&&\:&&\:&&\:&&\:&&\oddline E
  &\oct{35x}&&\:&&\:&&\:&&\:&&\:&&\:&&\:&&\:&\evenline
  &\oct{36x}&&\:&&\:&&\:&&\:&&\:&&\:&&\:&&\:&&\oddline F
  &\oct{37x}&&\:&&\:&&\:&&\:&&\:&&\:&&\:&&\:&\evenline}

\def\ftable#1#2{%
     \batchmode
     \font\X=#1%
     \errorstopmode
     \ifx\X\nullfont
       \@warning{Font #1 not found, table omitted}
     \else
       \count@="80
       \setbox0=\hbox{\X
        \loop\char\count@\advance\count@ by1 \ifnum\count@<"100
        \repeat}%
  \ifdim\wd0>0pt \let\top\tophalf\else\let\top\notophalf\fi
     \beginchart\X{\hfill\llap{\textbf{#1, \large#2}\label{fonttable:#1}}}\normalchart
     \endchart\par\vfill
    \fi}
\makeatother


\setcounter{tocdepth}{3}


\title{\heiti {\Huge \textbf{\LaTeX{}}\ 的字体编码}}

\author{Frank Mittelbach (弗兰克·米特巴赫) \and Robin Fairbairns (罗宾·费尔贝恩斯) \and Werner Lemberg (沃纳·莱姆伯格)\\[5pt] \LaTeX\ 项目团队\\[5pt] \ \ 赣医一附院神经科\ \ 黄旭华\ \ \ \ 译}

\date{\copyright~Copyright 1995--2016 \\[5pt] 2016 年 2 月 18 日}

\begin{document}

\maketitle

\tableofcontents

\section{介绍}

本文档解释了支持 \LaTeX{}\,字体编码(font encodings)的思想以及定义新编码时应用的约束(constraints);它还列出了已经定义的编码。

\subsection{\TeX{}\ 中的编码}

\TeX{}(程序)隐含地识别出三种编码方式,在某种意义上,它们都在 \TeX{}book~\cite{A-W:DKn86}\ 中讨论过:
\begin{itemize}
\item[1.] 输入编码(input encoding)。指定提交给 \TeX{}\ 处理的文件中字符(charac-ters)的含义(meanings)。\TeX{}book\ 建议“您的 \TeX{}\ 版本将识别您在键盘上键入的字符”(\TeX{}\ 程序提供输入字符的静态翻译[static translations])
\end{itemize}
这样直接使用 \TeX{}\ 的设施(facilities)并不是现代 \LaTeX{}\,(或任何其他 \TeX{}\ 宏包)处理输入编码的方式。本文件不涉及输入编码的主题;感兴趣的读者应该查看 \LaTeX{}\ 基础包(base package)\,\Pkg{inputenc}\ \cite[第~7.5.2~节,第~357~页]{A-W:MG2004}。
\begin{itemize}
\item[2.] \TeX{}\ 内部处理(processes internally)的令牌流(token stream)。该令牌流在 \TeX{}book\ 中进行了详细讨论。
\end{itemize}
同样,本文档没有涉及这个主题。\LaTeX{}\ 的内部字符表示(internal character representation)(\textsc{licr})在 \cite[第~7.11.2~节,第~442~页]{A-W:MG2004}\ 中进行了充分的讨论。
\begin{itemize}
\item[3.] 字体编码(font encoding)。即字符代码(character codes)到字体(fonts)中字形(glyphs)(用于 \TeX{}\ 排版输出)的映射(mapping)。同样,\TeX{}book\ 中列举了一组字体编码,但事实证明这组编码不足以满足现代多语言使用 \LaTeX\ 的需求。
\end{itemize}
本文档解释了\emph{为什么} Knuth(高纳德)的原始编码集(original set of encodings)不适合现代条件(modern conditions),并讨论了围绕新字体编码(new font encodings)的设计和定义的问题。

字体编码的\emph{重要性}不仅仅体现在映射用于排版的字体的字形上:它们的字形表(glyph tables)也是 \TeX{}\ 断字算法(hyphenation algorithm)运行的上下文(context)。\TeX{}\ 施加的限制(constraints)会影响在多语言环境中使用的新字体编码的结构化方式(详见第~\ref{sec:restrictions}\ 节)。

\subsection{\TeX{}\ 字体编码的历史}

在 \TeX{}\,3\ 到来之前,很少注意字体编码。在那个时候,人们使用 Donald Knuth(高纳德)的字体(计算机现代家族[Computer Modern family],使用我们现在称为 \Enc{OT1}\ 和 \Enc{OM}\ 系列的编码),或者使用自已的字体。

计算机现代文本编码(Computer Modern text encoding)在未修改的 \TeX{}\ 中产生问题,因为连字符(hyphenation)不能打断包含 \verb"\accent"\ (放置重音符号)命令的单词。即使在 \Enc{OT1}\ 编码具有必要的基于 \verb"\accent" 的符号的西欧语言中,这一缺陷也会破坏正在运行的文本的排版。

随着 \TeX{}\,3\ 的出现,它能在连字符模式集(hyphenation pattern sets)之间切换,很明显上述情况不会继续。因此,在爱尔兰(Ireland)科克(Cork)举行的 TUG\ 年度大会(Annual General Meeting)上,一个小组为 256 种字形字体(glyph fonts)指定了统一的编码(uniform encoding),其中包含重音字母(accented letters)和非 \textsc{ascii}\ 字母,这些字母是表达大多数西欧语言(以及一些东欧语言)所必需的,无需依靠 \verb"\accent"\ 命令。

这种“Cork(科克)”编码已经在一系列用 Metafont 设计的字体中实现,至少有一个字体系列(font series)可以在 Adobe Type 1 格式和 OpenType 格式中使用,并且在其他字体系列的许多虚拟字体映射(virtual-font mappings)中使用。

自从科克会议(Cork meeting)以来,人们一直致力于为 \TeX{}\ 使用的文本字体(text fonts)设计编码,而科克编码(Cork encoding)影响了许多此类编码的设计。

相比之下,自 Knuth 的贡献以来,数学字体(mathematical fonts)的编码几乎没有变化。在科克会议上成立了一个 TUG 技术工作组(Technical Working Group),其目的是定义一组 256 字形编码(glyph encodings),以规范和扩展 Knuth 的原始字体、使用自那时以来出现的其他几种字体的想法、以及数学和数学科研人员的已知需求。

独立地,贾斯汀·齐格勒(Justin Ziegler)与弗兰克·米特巴赫(Frank Mittelbach)和 \LaTeX\ 项目团队~\cite{ziegler}~的其他成员共同制定了第一个提案(pro-posal)(所谓的阿斯顿提案[\emph{Aston proposal}])。马蒂亚斯·克拉森(Matthias Clasen)和乌尔里克·维思(Ulrik Vieth)~\cite{clasen,clasen-vieth}~首次实现了该提案。

然而,这些数学编码(Mathematical encodings)的缓慢进展已经被(在过去十年左右)大量数学符号(mathematical symbols)添加到 Unicode~\cite{beeton}~中所超越;人们可以期待进一步的变化,因此新的公共数学字体编码(public mathematical font encodings)很可能会被进一步推迟。



\subsection{更多的信息}

对于 \LaTeX\ 的一般性的介绍(general introduction),包括 \LaTeXe\ 的新特性,您应该阅读 \emph{\LaTeXbook}, Leslie Lamport, Addison Wesley, 2nd~ed, 1994\ (《\emph{\LaTeX}:一个文档准备系统》,莱斯利·兰伯特,艾迪生·卫斯理,第2版,1994)

有关 \LaTeX\ 新功能的更详细描述,包括 200 多个宏包(packages)和近 1000 个现成运行示例(run examples)的概述,请参见 Frank Mittelbach (弗兰克·米特尔巴赫)和 Michel Goossens (米歇尔·古森斯)的~\cite{A-W:MG2004} \emph{\LaTeXcomp{} second edition}\ (《\LaTeX\ 指南》第二版)。

\LaTeX{}\ 项目赞助了一份关于数学字体编码的报告,该报告值得一读,因为它深入了解了定义数学使用方式的问题:参见~\cite{ziegler,clasen,clasen-vieth}。

\LaTeX{}\ 字体选择方案(font selection scheme)是基于 \TeX\ 的,其开发者 Donald E.~Knuth (高德纳) 和 Addison Wesley (艾迪生·卫斯理)于 1986 年在 \emph{The \TeX book}\ 中对其进行了描述,并于1991年修订,以包括 \TeX~3\ 的功能。

如欲了解更多关于 \TeX{}\ 和  \LaTeX\ 的信息,请联系您当地的 \TeX{} 用户组\,(Users Group)\,或国际 \TeX{}\ 用户组 (\url{http://www.tug.org})。



\section{现有的字体编码}

本节列出了当前分配的(currently assigned)编码;对于每种编码,我们列出了已注册的(\LaTeX{})名称、分配的编码目的(purpose)和作者。进一步的细节可能会列出编码中使用的代码位置(code positions)、变量槽(\emph{variable slots})(见下文)、示例字体(example font)(如果有相关字体,文档稍后将提供相关清单)以及供进一步参考的源代码(source)。

虽然一种编码的特点(feature)是,根据该编码编码的每种字体应该具有相同的字形集(glyph set),但也有一些编码(特别是 \Enc{OT1}\ 及其派生物),其中一些字形代码槽(glyph code slots)在不同字体的内容(contents)上有所不同。

\subsection{命名规范(Naming conventions)}

编码方案(encoding schemes)的名称是最多三个字母(均为大写)加数字的字符串(strings)。

\LaTeX\ 项目保留使用以下字母开头的编码名称(encoding names):|T|(标准 256 长文本编码[long text encodings]),|TS|(旨在扩展相应 |T| 编码的符号),|X|(不符合 |T| 编码严格要求的文本编码),|M|(标准 256 长数学编码[long mathematical encodings]),|S|(其他符号编码[symbol encodings]),|A|(其他特殊应用[special applications]),|OT|(标准 128 长文本编码)和 |OM|(标准 128 长数学编码)。

请不要将上述开头字母(starting letters)用于非便携式编码(non-portable encodings)。如果出现新的标准编码(standard encodings),我们将在 \LaTeX\ 的后续版本中添加它们。

网站(site)或系统(system)的局部编码方案(encoding schemes)应以 |L| 开头,用于广泛分发的实验性编码(experimental encodings)将以 |E| 开头,而 |U| 表示未知(Unknown)或未分类(Unclassified)编码。

\begin{quote}
  \itshape 我们建议不要引入新的编码名称,除非用户社区(user community)仔细考虑和讨论确认了编码的必要性。如果编码必须从一种字体更改为另一种字体,就会出现许多问题,因此最好开发能够与大量字体并行使用的编码。这允许文档使用不同的字体排版,而不会出现问题。

  \Enc{TS1}\ 编码是\emph{糟糕(bad)}编码的一个很好的例子(尽管它的开发出发点是好的),因为大量的字体只能实现其中的一部分。类似地,少数可用的数学字体集(mathematical fonts sets)(除了计算机现代数学[Computer Modern Math])几乎都实现了略微不同的编码,这是问题的一个巨大来源。如果可能的话,不要再添乱了!
\end{quote}


\subsection{128$^+$ 字形编码(文本)}

“OT”系列(series)字体编码始于 Donald Knuth 的原始文本编码(original text encoding),用于 \TeX{}\ 最早版本中的文本字体(text fonts)。编码指示符(encoding designator)的“O”可被视为表示“原始的(original)”或“旧的(old)”。

\begin{encodinginfo}{OT1}
        {\TeX{} text\ (\TeX{}\,文本)}
        {Donald Ervin Knuth}
        {0x00--0x7F}
        {0x0B--0x0F, 0x24, 0x3C, 0x3E, 0x5C, 0x7B--0x7D}
% {0X--'177}
% {'13--'17, '44, '74, '76, '134, '173--'175}
        {cmr10}
        {\cite[第 427 页]{A-W:DKn86}}

  Donald Knuth (高德纳)设计字体编码(以及字体)的环境与现在 \TeX{}\ 世界的环境完全不同:他的(主机)计算机内存很小,对多语言技术排版的经验(或需求)很少,因此牺牲一致性(appropriate)来提高效率(efficiency)是合适的。

  因此,Knuth 的原始字体(original fonts)在一些编码槽(encoded slots)中略有不同:例如,字形(glyphs) \texttt{\string<}、\texttt{\string>}、\verb=\=、\verb={= 和 \verb=}= 仅在打字机字体(typewriter fonts)中可用,而 \textdollar{} 和 \textsterling{} 符号共享相同的位置(在不同的字体形状[font shapes]中)。

  这意味着直接选择这些插槽(slots)可能会产生不可预测的结果,例如,在文档中键入 \texttt{\string<} 或 \verb=\symbol{'74}= 可能会生成“\textquestiondown”。
\end{encodinginfo}


\begin{encodinginfo}{OT2}
        {UW cyrillic encoding (UW cyrillic 编码)}
        {华盛顿大学(University of Washington)}
        {0x00--0x7F}
        {---}
        {wnr10}
        {\cite{Beeton:TB6-3-124}}
  尽管出于所有实际目的(practical purposes),最好使用 \Enc{T2}\ 编码之一,但是西里尔语包(Cyrillic bundle)中提供了对这种编码的支持。
\end{encodinginfo}


\begin{encodinginfo}{OT3}
        {UW IPA encoding (UW IPA 编码)}
        {华盛顿大学(University of Washington)}
        {0x00--0x7f}
        {---}
        {wsuipa10}
        {\cite[第 149 页]{CorkGW:91}}
  在引入 TIPA 系统(TIPA system,[译者注]一种处理语音符号的系统)后,\Enc{OT3}\ 编码从未真正用于 \LaTeXe{},TIPA 系统为 IPA 提供了更好的支持。特别是,还没有生成 \File{ot3enc.def} 文件。
\end{encodinginfo}


\begin{encodinginfo}{OT4}
        {Polish text encoding (波兰语文本编码)}
        {B.~Jackowski (B·杰科夫斯基) 和 M.~Ry\'cko (M·里科)} %% ?  Marcin Woli\'nski
  {0x00--0x7F, 0x81, 0x82, 0x86, 0x8A, 0x8B, 0x91, 0x99, 0x9B, 0xA1,
   0xA2,0xA6, 0xAA, 0xAB, 0xAE, 0xAF, 0xB1, 0xB9, 0xBB, 0xD3, 0xF3,
   0xFF}
  {0x0B--0x0F, 0x24, 0x3C, 0x3E, 0x5C, 0x7B--0x7D}
        {plr10}
        {---}

  虽然 Knuth 在其 \Enc{OT1}\ 编码中包括了排版“lost L”(\L)的方法,但他省略了 (\,\,\k{}),这是波兰语文本(Polish text)中也需要的变音符号(diacritic mark);因此,早在 \Enc{T1}\ 编码之前,就出现了使用这种编码的字体。
\end{encodinginfo}

\begin{encodinginfo}{OT5}
        {Not currently allocated (当前未分配)}
        {---}
        {---}
        {---}
        {}
        {---}

\end{encodinginfo}



\begin{encodinginfo}{OT6}
        {Armenian text encoding (亚美尼亚语文本编码)}
        {Serguei Dachian (塞尔吉·达奇安)}
        {0x03--0x0F, 0x13--0x7F}
        {---}
        {artmr10}
        {---}

  分配这种编码是为了允许在标准 \LaTeX{}\ 环境中使用Dachian(达奇安)的亚美尼亚字体(Armenian fonts)。

  由于许可证问题(license issues),\texttt{artmr}\ 字体不一定包含在分发的 \TeX{}\ 安装中(因此,下面未显示相应的编码表[encoding table])。但是,字体和支持宏(support macros)可以在 CTAN 档案中找到(请查找 \texttt{armtex})。

\end{encodinginfo}



\subsection{256 字形编码(文本)}

\begin{encodinginfo}{T1}
        {Cork encoding (科克编码)}
        {科克(Cork)欧洲 \TeX{}\ 会议}
        {0x00--0xFF}
        {---}
        {ecrm1000}
        {\cite[第 514 页]{tub:MFe90},\cite[第 99 页]{Knappen:TB17-2-96}}

  科克编码(Cork encoding)的开发是为了利用 \TeX{}\,3(当时)的新功能,允许在未修改版的(unmodified version)\TeX{}\ 中使用大多数西欧(和一些东欧)语言中的连字符(hyphenation)。

  除了使用 Metafont (“EC”字体)编写的实例(instances),以及最近可用的相同字体的 Adobe Type 1 实例外,编码(encoding)是在没有任何现有的字体设计(font design)的情况下开发的。

  还开发了大量(但不完整)使用虚拟字体(virtual fonts)的实例。这些后面的实例映射 Knuth 的原始(OT1 编码的)字体,或者包含 Adobe“标准(standard)”224 字形(glyphs)集的商业字体(com-mercial fonts)。
\end{encodinginfo}

\begin{encodinginfo}
  {T2A, T2B, T2C}
  {Cyrillic encodings (西里尔语编码)}
  {CyrTUG 字体团队(font team)}
  {0x00--0xFF}
  {--- (在每个编码)}
  {larm1000}
  {\cite{Berdnikov:eurotex-98}}

  在完整的西里尔语补码(Cyrillic complement of languages)中,有太多的字形(glyphs),以至于所有的字形不能被单一的 \LaTeX{}\ 兼容编码(compliant encoding)所覆盖(每个~\Enc{T2}~编码的下半部分与~\Enc{T1}~编码的下半部分相同,以便每个都应该是一个兼容的 \LaTeX{}\ 编码~--- 见第~\ref{sec:restrictions}~节)。因此,采取的方法是开发一个单一编码 \Enc{X2} (见~\ref{sec:extendedenc}),该单一编码包含全套语言(full set of languages)所需的所有字形,然后使用 \Enc{X2}\ 集和 \Enc{T1}\ 集导出(derive)三个 \LaTeX{}\ 兼容的 \Enc{T2}\ 族编码(family encodings)。

\end{encodinginfo}



\begin{encodinginfo}{T3}
        {IPA encoding (IPA 编码)}
        {FUKUI Rei (福井丽),东京大学}
        {0x00--0xFF}
        {---}
        {tipa10}
        {\cite[第 102 页]{Rei:TB17-2-102}}


  根据当前的国际语音协会(International Phonetic Association,IPA)建议 \cite{ipa},\Enc{T3}\ 编码(以及相关的宏)提供了语音描述(phonetic description)所需的字形(glyphs)。

  \Enc{T3}\ 编码\emph{不符合} \Enc{T}\ 编码的要求---名称是一个历史意外。正确的名称应该是 \Enc{X3},但是由于这个字体族(font family)已经在其当前编码名称(current encoding name)下使用了很长时间,因此名称不会因兼容性原因(compatibility reasons)而更改。

\end{encodinginfo}



\begin{encodinginfo}{T4}
        {African Latin (fc) (非洲拉丁语)}              % public name
        {J\"org Knappen (约尔格·克纳彭)}              % author name
        {0x00--0xFF}              % range(s) of slots used for glyphs
        {0x24}         % range(s) of slots with variable glyphs if any
        {fcr10}              % name of an example font
        {\cite{tub:JKn93}}

非洲拉丁字体(African Latin fonts)的下半部分(0--127)包含与欧洲拉丁字体(European Latin Fonts)(T1 编码)相同的字符,而上半部分(128--255)包含使用扩展拉丁字母表(extended Latin alphabets)的非洲语言的字母(letters)和符号(symbols)。由于空间(space)不足,J\"org(约尔格)不得不玩了一个不幸的把戏,将 \verb=\textdollar= 和 \verb=\textsterling= 分配到同一位置;如果需要,用户应该从文本配套字体(text companion font)中提取这些字符(characters)。与为单个字母(single letters)定义许多新的控制序列(control sequences)不同,有三个通用的类似重音的(accent-like)控制序列:\verb=\m= (Modified-1)、\verb=\M= (Modified-2) 和 \verb=\B= (Barred)。大多数标准的 \LaTeX{}\ 编码依赖命令(encoding-dependent commands)都可以工作。然而,冰岛特殊字母(Icelandic special letters)不可用,因此使用了“最佳替代(best replacements)”来代替 \verb=\Th=、 \verb=\th= 和 \verb=\dh= (分别禁止 T 和 d)。
\end{encodinginfo}


\begin{encodinginfo}{T5}
        {Vietnamese encoding (越南语编码)}
        {沃纳·莱姆伯格 (Werner Lemberg) 和弗拉基米尔·沃洛维奇 (Vladimir Volovich)}
        {0x00--0xFF}
        {---}
        {vnr10}
        {\cite{vnr}}

  \Enc{T5}\ 编码是为越南语(Vietnamese)开发的。同样,这种编码\emph{不}符合 \Enc{T}\ 编码的要求,因为其大量重音字母(accented letters)阻止了 \Enc{T}\ 编码的 \verb=\lccode= 和 \verb=\uccode= 映射要求(mapping requirements)。然而,由于越南语在排版中不使用分词(word division),因此这一要求对这种特定语言实际上并不重要。由于越南语文本中使用的每个字形(glyph)在内部都表示为 \textsc{licr}\ 宏,因此命令 \verb=\MakeUppercase= 和 \verb=\MakeLowercase= 仍按预期工作(因为它们更改了 \textsc{licr}\ 定义中 \textsc{ascii}\ 字符的大小写[case])。

\end{encodinginfo}

\begin{encodinginfo}
  {T6}
  {Armenian (亚美尼亚文)}
  {---}
  {---}
  {---}
  {}
  {---}

  保留这种编码,以便将来亚美尼亚 \TeX{}\ 的扩展能使用 256 字符(256-character)(连字符)字体。
\end{encodinginfo}

\begin{encodinginfo}{T7}
        {Greek encoding (希腊文编码)}
   {---}
   {---}
   {---}
   {}
   {---}

该名称已为 256 字形希腊文编码(greek encoding)而保留。编码本身至今尚未定义。

\end{encodinginfo}



\subsection{256$^-$ 字形编码 (文本符号)}

\begin{encodinginfo}{TS1}
        {Text Companion encoding (Cork) [文本配套编码(科克)]}
        {J\"org Knappen (约尔格·克纳彭)}
  {0x00--0x0D, 0x12, 0x15, 0x16, 0x18--0x1D, 0x20, 0x24, 0x27, 0x2A,
   0x2C--0x3A, 0x3C--0x3E, 0x4D, 0x4F, 0x57, 0x5B, 0x5D--0x60,
   0x62--0x64, 0x6C--0x6E, 0x7E--0xBF, 0xD6, 0xF6}
  {---}
        {tcrm1000}
        {\cite{Knappen:TB17-2-96}}

  文本符号编码(text symbol encoding)提供了对文本中常用的符号字形(symbolic glyphs)的访问(由于各种原因),符号字形的样式(style)随着它们周围的文本而变化。

  不幸的是,\Enc{TS1}\ 编码是在没有参考现有商业字体(commercial fonts)中可用的字形(glyphs)的情况下开发的。因此,只有为 \TeX{}\ 显式开发的(explicitly developed)字体族(font families)(即,通常源于 \MF{})实际上才包含 \Enc{TS1}\ 编码所需的所有字形。大多数其他字体族(无论是免费的还是商业的)通常只提供一半的字体。
%%
%% don't show the comment if the tables are not generated
%%
\expandafter\ifx\csname r@fonttable:tcrm1000\endcsname\relax
\else
  \expandafter\ifx\csname r@fonttable:ptmr8c\endcsname\relax
  \else
    \space (比较第~\pageref{fonttable:tcrm1000}~页和第~\pageref{fonttable:ptmr8c}~页 \Enc{TS1}\ 的两个表格)%
  \fi
\fi。
  为了在一定程度上改进这种情况,NFSS 提供了一种在 \Pkg{textcomp}\ 宏包(该宏包提供对 \Enc{TS1}\ 编码的支持)中基于每个字体族定义编码子集(define encoding subsets)的方法。
\end{encodinginfo}


\begin{encodinginfo}{TS3}
        {IPA symbol encoding (IPA 符号编码)}
        {FUKUI Rei (福井丽),东京大学}
        {0x00--0x0A, 0x20--0x49, 0x50--0x56, 0x70--0x7B}
        {---}
        {tipx10}
        {\cite{Rei:TB17-2-102}}

  \Enc{TS3}\ 编码(连同 \Enc{T3}\ 编码)根据国际语音协会(International Phonetic Association,IPA) \cite{ipa}\ 的指引提供字形(glyphs),以排版语音转录(phonetic transcriptions)。通过 \Pkg{tipa}\ 宏包提供支持。
\end{encodinginfo}




\subsection{256 字形编码 (文本扩展)}
\label{sec:extendedenc}

\begin{encodinginfo}
  {X2}
  {Cyrillic glyph container (西里尔文字形容器)}
  {The CyrTUG font team (CyrTUG 字体团队)}
  {0x00--0xFF}
  {---}
  {rxrm1000}
  {\cite{Berdnikov:eurotex-98}}

  此编码指定西里尔字符的字形容器(glyph container),用于指定 \Enc{T2A}、\Enc{T2B} 和 \Enc{T2C}\ 编码。
\end{encodinginfo}




\subsection{128$^+$ 字形编码 (数学)}


\begin{encodinginfo}{OML}
        {\TeX{} math italic (\TeX{}\ 数学斜体)}
        {Donald Ervin Knuth (高德纳)}
        {0x00--0x7F}
        {---}
        {cmmi10}
        {\cite[p.430]{A-W:DKn86}}

  \Enc{OML}\ 编码包含用于数学公式(mathematical formulas)(通常用于变量)的斜体拉丁字母(italic Latin letters)和希腊字母(Greek letters)以及一些符号(symbols)。

\end{encodinginfo}

\begin{encodinginfo}{OMS}
        {\TeX{} math symbol (\TeX{}\ 数学符号)}
        {Donald Ervin Knuth (高德纳)}
        {0x00--0x7F}
        {---}
        {cmsy10}
        {\cite[p.431]{A-W:DKn86}}

  \Enc{OMS}\ 编码包含基本的数学符号(basic mathematical symbols),以及大写的(uppercase)“书法(calligraphic)”拉丁字母(Latin alphabet)。
\end{encodinginfo}


\begin{encodinginfo}{OMX}
        {\TeX{} math extension (\TeX{}\ 数学扩展)}
        {Donald Ervin Knuth (高德纳)}
        {0x00--0x7F}
        {---}
        {cmex10}
        {\cite[p.432]{A-W:DKn86}}

  \Enc{OMS}\ 用可变尺寸(variable sizes)编码数学符号(mathematical symbols),比如 $\sum$ 符号(sign),如果显示在公式中,它的尺寸就会发生变化,还有括号(brackets)、大括号(braces)和根号(radicals)等的构造部分,它们可以伸展以适应它们所包含的东西。

\end{encodinginfo}




\subsection{256 字形编码 (数学)}

到目前为止,还没有 256 种字形数学编码(glyph mathematical encodings)。\cite{ziegler}\ 中给出了一个建议。

\subsection{其它编码}

\begin{encodinginfo}
  {C..}
  {CJK encodings (CJK 编码)}
  {Werner Lemberg (沃纳·莱姆伯格)}
  {0x00--0xFF}
  {---}
  {} % no font, of course
  {\cite{CJK}}

  \Pkg{CJK}\ 宏包定义了多种可访问中文(Chinese)、日文(Japanese)和韩文(Korean)字体的编码。

\end{encodinginfo}

\begin{encodinginfo}
  {E..}
  {Experimental encodings (实验性编码)}
  {---}
  {0x00--0xFF}
  {all (全部)}
  {}
  {\cite[第 416 页]{A-W:MG2004}}

  顾名思义,以字母 \Enc{E}\ 开头的编码用于实验性编码(experimental encodings),它有可能发生变化。
\end{encodinginfo}

\begin{encodinginfo}{L..}
        {Local encoding (site dependent) [局部编码(网站依赖)]}
        {---}
        {0x00--0xFF}
        {all (全部)}
        {}
        {\cite[p.416]{A-W:MG2004}}

        “局部(Local)”编码提供了开发适合特定 \TeX{}\ 环境的表示技术的手段。虽然开发人员可以随心所欲地指定自己的编码方式,但他们有强烈的动机(strong incentive)去遵守编码的 \LaTeX{}\ 规则(rules),因为不这样做,他们很难使用编码方式编写文本(compose text)。

        至少它的意图是 \Enc{L..}\ 编码是局部的(local)和网站相关的(site dependent)。然而,许多这样的编码在没有分配不同名称的情况下被广泛使用。

\end{encodinginfo}



\begin{encodinginfo}{LY1}
        {Y\&Y 256 glyph encoding (Y\&Y 256 字形编码)}
        {Berthold Horn (伯索尔德·霍恩)}
        {0x00--0x08, 0x0C, 0x10, 0x12--0xFF}
        {\emph{不相信}}
        {ptmr8y}
        {\cite[p.416]{A-W:MG2004}}

        这是 Y\&Y 开发的 \Enc{T1}\ 编码的一种替代方案,用于商业  \TeX{}\ 实现(implementation)。

\end{encodinginfo}


\begin{encodinginfo}{LV1}
        {MicroPress encoding (MicroPress 编码)}
        {Michael Vulis (迈克尔·瓦利斯)}
        {\emph{未知}}
        {\emph{未知}}
        {}
        {\cite[第 416 页]{A-W:MG2004}}

        这是 MicroPress 开发的一种编码,用于他们的一些字体。

\end{encodinginfo}


\begin{encodinginfo}{LGR}
        {Greek 256 glyph encoding (希腊 256 字形编码)}
        {\emph{未知}}
        {0x00--0xFF}
        {\emph{不相信}}
        {grmn1000}
        {\cite[第 575 页]{A-W:MG2004}}

        目前主要用于希腊语的编码。

        这种编码不符合第~\pageref{sec:restrictions}~页第~\ref{sec:restrictions}~节描述的 \Enc{T}\ 编码的限制(restrictions),因为它根本没有任何 \textsc{ascii}\ 字形。

\end{encodinginfo}


\begin{encodinginfo}
  {PD1}
  {PDF DocEncoding (PDF 文档编码)}
  {Adobe (奥多比)}
  {0x08--0x0A, 0x0C, 0x0D, 0x18--0x7E, 0x80--0x9E, 0xA0--0xAE, 0xB0--0xFF}
  {---}
  {}
  {\cite{Adobe:PDF-1.6},\cite{hyperref}}

  \Enc{PD1}\ 编码是一种虚拟编码(virtual encoding),具有 256 个字形,用于通过 pdf\LaTeX\ 生成 PDF 文档中的书签(bookmarks)和类似文本。编码是“虚拟的(virtual)”,因为根据设计,不存在覆盖 \Enc{PD1}\ 的 \TeX{}\ 字体。详情见~\cite{Adobe:PDF-1.6}~中的附录 D.1。
\end{encodinginfo}

\begin{encodinginfo}
  {PU}
  {PDF Unicode Encoding (PDF Unicode 编码)}
  {Adobe (奥多比)}
  {---}
  {---}
  {}
  {\cite{Adobe:PDF-1.6},\cite{hyperref}}

  PDF 文档中 Unicode 编码书签的另一种虚拟编码(virtual encoding)(超过 600 个字符)。
\end{encodinginfo}

\begin{encodinginfo}{U}
        {Unknown encoding (未知编码)}
        {---}
        {可能是 0x00-0xFF}
        {all (全部)}
        {wasy10}
        {\cite[第 416 页]{A-W:MG2004}}

  这种编码应用于抵抗分类(resist classification)的字体,例如,当很明显不会有多个字体使用相同编码时。

\end{encodinginfo}



\section{限制 (Restrictions)}
\label{sec:restrictions}


\subsection{通用文本编码所需的字形}

应该与 \LaTeX{}\ 一起用于“通用文本字体(general purpose text font)”的编码需要在特定的编码槽(encoding slots)中具有特定的固定字形(fixed glyphs)。“通用文本字体(general purpose text font)”是一种用于任意自然语言文本(natural language text)的字体,而不仅仅用于特殊环境(如语音字母表[phonetic alphabet])或用于排版单个符号(individual symbols)(例如,带有编码 \Enc{TS1}\ 的文本配套字体[text companion font])。

对于下面的字形(glyphs),它们必须处于通用文本编码的 \textsc{ascii}\ 位置:
\begin{center}
\begin{tabular}[t]{cc}
  字形 & 位置 \\ \hline
  !     & \number`\!    \\
  '     & \number`\'    \\
  (     & \number`\(    \\
  )     & \number`\)    \\
  \relax*       & \number`\*    \\
  +     & \number`\+    \\
  ,     & \number`\,    \\
  -     & \number`\-    \\
  .     & \number`\.    \\
  /     & \number`\/    \\
  0 \ldots\ 9   & \number`\0\ to \number`\9     \\
  \end{tabular}
  \quad
  \begin{tabular}[t]{cc}
  字形 & 位置 \\ \hline
  :     & \number`\:    \\
  ;     & \number`\;    \\
  =     & \number`\=    \\
  ?     & \number`\?    \\
  @     & \number`\@    \\
  A \ldots\ Z   & \number`\A\ to \number`\Z     \\
  \relax[       & \number`\[    \\
  ]     & \number`\]    \\
  `     & \number`\`    \\
  a \ldots\ z   & \number`\a\ to \number`\z     \\
\end{tabular}
\quad
\begin{tabular}[t]{cc}
字形\footnotemark      & 位置 \\ \hline
<       & \number`\<    \\
>       & \number`\>    \\
\string|        & \number`\|    \\
\end{tabular}\footnotetext{拉丁字母表(Latin alphabet) \Enc{OT}\ 编码违反了这三个字形的要求。}
\end{center}
此外,以下字形(glyphs)必须出现在编码中的某个位置\footnote{在这种情况下,位置并不重要,因为它们是由连字程序(ligature programs)生成的。},并与相应的连字程序(ligature programs)一起生成这些字形:
\begin{center}
\begin{tabular}[t]{cc}
字形   & 连字程序 \\ \hline
 ``     & \texttt{`\/`} \\
 ''     & \texttt{'\/'} \\
 --     & \texttt{-\/-} \\
 ---    & \texttt{-\/-\/-} \\
\end{tabular}
\end{center}

这是 $33 + 2 * 26 = 85$ 个“必需(required)”位置,其中 171 个位置空闲(free)。

如果有空闲的位置(free slots),那么添加全部或部分变音符(diacritics)将是填充它们的最佳方式。

如果没有足够的插槽(insufficient slots)容纳所需的字符(characters),一种可能的技术是创建一个子编码(subsidiary encoding),并将非字母字符(non-letter characters)移动到该子编码中。由于只有“字母(letters)”参与连字符算法(hyphenation algorithm),这种技术不影响排版结果的外观。

\subsection{大写表或小写表中的约束}

由于 \TeX{}\ 的一些与连字符(hyphenation)有关的技术限制(technical restrictions),在 \LaTeX{}\ 中不可能使用一个以上的 \verb=\lccode= 或 \verb=\uccode= 表。因此,所有编码都需要共享这两个表,这两个表被定义为 \Enc{T1}\ 编码的表。

\Enc{T1}\ 编码有一些令人讨厌的特性,如果要遵守这个限制,那么某些槽位(slot positions)或多或少不能用于其他编码。这是不幸的,但是因为 \Enc{T1}\ 已经很好地建立起来了,而且作为大量语言(languages)的基础,似乎更好地适应这种情况,而不是试图用一个稍微好一点的标准来取代 \Enc{T1} (结果是,在很长一段时间内,不同的 \LaTeX{}\ 安装[installations]因为不兼容的字体集[incompatible font sets]而无法相互通信)。

有问题的位置(positions)如下:
\begin{center}
\begin{tabular}{|rp{0.8\linewidth}X|}
\hline
25 (\char 25) & 奇怪的大写映射(uppercase maps)(和 105, \char 105\ 一样)\\\hline
26 (\char 26) & 奇怪的大写映射(uppercase maps)(和 106, \char 106\ 一样)\\\hline
27 (\char 27) & 小写字母(lowercase)映射到它自己,这使得这个槽受到连字符(用于支持 \Enc{OT1}\ 编码)的限制\\\hline
157 (\char 157) & 奇怪的大写映射(lowercase maps)(和 73, \char 73\ 一样)\\\hline
158 (\char 158) & 奇怪的大写映射(uppercase maps)(和 240, \char 240\ 一样)\\\hline
\end{tabular}
\end{center}
使用这种插槽(slots)的一种方法是用连字字形(ligature glyphs)填充它们,因为 \TeX{}\ 不会查阅(consult)通过连字程序(ligatures programs)构建的字形表(tables for glyphs),而是使用用于生成连字的单个字形(individual glyphs)的条目(entries)。

大/小写映射表(uppercase/lowercase mapping tables)的完整清单见第~\ref{sec:uclc-tab}~节(第~\pageref{sec:uclc-tab}~页)。

\newcount\temp \newcount\tempL \newcount\tempU

\def\nextstep{\global\tempL=\lccode\temp
              \global\tempU=\uccode\temp
              \lctablenumbersize\the\temp &
              \the\tempL&
              \the\tempU&\printlowerupper{\the\temp}{\the\tempL}{\the\tempU}\\
               \global\advance\temp by 1
               \stepprint}

\def\printlowerupper#1#2#3{\char#1\relax
   (\ifnum#2=0\relax--\else\char#2\fi
   /\ifnum#3=0\relax--\else\char#3\fi)}

\def\stepprint{\relax\ifnum\temp<\endval
                    \let\next=\nextstep
               \else
                     \let\next=\relax
               \fi
               \next}

\def\dolctable#1#2{{\temp=#1\relax
\def\endval{#2}%
\setlength\tabcolsep{1.5pt}%
\begin{tabular}[t]{@{}cccc@{}}
pos&lc&uc&glyphs\\\hline
\stepprint
\end{tabular}}}

\iffalse
\begin{center}
\tiny\let\lctablenumbersize\tiny
\mbox{\dolctable{0}{52}\vrule
\dolctable{52}{104}\vrule
\dolctable{104}{156}\vrule
\dolctable{156}{208}\vrule
\dolctable{208}{256}}
\end{center}
\fi

\iffalse
\begin{center}\tiny
\mbox{\dolctable{0}{65}\vrule
\dolctable{65}{128}\vrule
\dolctable{128}{193}\vrule
\dolctable{193}{256}}
\end{center}
\fi



\section{特定编码的(encoding specific)命令}

特定编码的命令(encoding specific command)能生成一个或多个字形(glyph),以产生可能在不同编码中实现不同的图形效果(graphic effect)。当编码在文档过程(course of the document)中发生变化时,特定编码的命令会自动更改实现(implementation)。特定编码的命令出现在 \LaTeX\ 的内部字符表示(internal character representation)(\textsc{licr})中,在\cite[第~7.11.2~节, 第~442~页]{A-W:MG2004}这里会进行讨论。

下表仅涵盖了 \Enc{OT1}\ 和 \Enc{T1}\ 编码的编码特定命令(encoding specific commands)。其他编码可以指定特定命令的附加编码(additional encoding)。在表中,前 15 个命令是“类似重音(accent-like)”的,并且需要将要重音字符作为参数。例如,|\v{c}| 是“\v{c}”的 \LaTeX{}\ 的内部字符表示(internal character representation)(\textsc{licr})。
%\vspace{-1em}
\begin{center}
\begin{longtable}{|l|l|c|l|}
\hline
\ttverb\`{}               &OT1,T1&   \a`{}& grave  (重音符、抑音符、沉音符)      \\ \hline
\ttverb\'{}               &OT1,T1&   \a'{}& acute  (尖重音、尖音符)      \\ \hline
\ttverb\^{}               &OT1,T1&   \^{}&  circumflex  (音调符号、长音符号) \\ \hline
\ttverb\~{}               &OT1,T1&   \~{}&  tilde  (波浪号)      \\ \hline
\ttverb\"{}               &OT1,T1&   \"{}&  umlaut  (曲音符号)     \\ \hline
\ttverb\H{}               &OT1,T1&   \H{}&  Hungarian umlaut  (匈牙利元音符号) \\ \hline
\ttverb\r{}               &OT1,T1&   \r{}&  ring  (环形记号)       \\ \hline
\ttverb\v{}               &OT1,T1&   \v{}&  ha\v{c}ek  (小钩)  \\ \hline
\ttverb\u{}               &OT1,T1&   \u{}&  breve  (短音符号、二全音符、弱读音节符号)      \\ \hline
\ttverb\t{}               &OT1,T1&   \t{}&  tie  (连结符号、连结线、连音符)        \\ \hline
\ttverb\={}               &OT1,T1&   \a={}& macron  (长音符、平调符)     \\ \hline
\ttverb\.{}               &OT1,T1&   \.{}&  dot  (附点、顿音记号、断奏符)        \\ \hline
\ttverb\b{}               &OT1,T1&   \b{}&  underbar  (下划线)   \\ \hline
\ttverb\c{}               &OT1,T1&   \c{}&  cedilla  (变音符号)    \\ \hline
\ttverb\d{}               &OT1,T1&   \d{}&  dot under  (下划点)  \\ \hline
\ttverb\k{}               &T1    &   \k{}&  ogonek  (鼻音化符号)     \\ \hline
\ttverb\AE                &OT1,T1&   \AE &               \\ \hline
\ttverb\DH                &T1    &   \DH &               \\ \hline
\ttverb\DJ                &T1    &   \DJ &               \\ \hline
\ttverb\L                 &OT1,T1&   \L  &               \\ \hline
\ttverb\NG                &T1    &   \NG &               \\ \hline
\ttverb\OE                &OT1,T1&   \OE &               \\ \hline
\ttverb\O                 &OT1,T1&   \O  &               \\ \hline
\ttverb\SS                &OT1,T1&   \SS &               \\ \hline
\ttverb\TH                &T1    &   \TH &               \\ \hline
% \ttverb\aa              &OT1,T1&   \aa &               \\ \hline  no-longer
\ttverb\ae                &OT1,T1&   \ae &               \\ \hline
\ttverb\dh                &T1    &   \dh &               \\ \hline
\ttverb\dj                &T1    &   \dj &               \\ \hline
\ttverb\guillemotleft     &T1    &   \guillemotleft  & guillemet  (吉约梅标记、书名号) \\ \hline
\ttverb\guillemotright    &T1    &   \guillemotright & guillemet  (吉约梅标记、书名号) \\ \hline
\ttverb\guilsinglleft     &T1    &   \guilsinglleft  & guillemet  (吉约梅标记、书名号) \\ \hline
\ttverb\guilsinglright    &T1    &   \guilsinglright & guillemet  (吉约梅标记、书名号) \\ \hline
\ttverb\i                 &OT1,T1&   \i  &               \\ \hline
\ttverb\j                 &OT1,T1&   \j  &               \\ \hline
\ttverb\l                 &OT1,T1&   \l  &               \\ \hline
\ttverb\ng                &T1    &   \ng &               \\ \hline
\ttverb\oe                &OT1,T1&   \oe &               \\ \hline
\ttverb\o                 &OT1,T1&   \o  &               \\ \hline
\ttverb\quotedblbase      &T1    &   \quotedblbase   &   \\ \hline
\ttverb\quotesinglbase    &T1    &   \quotesinglbase &   \\ \hline
\ttverb\ss                &OT1,T1&   \ss &               \\ \hline
\ttverb\textasciicircum   &OT1,T1&   \textasciicircum &  \\ \hline
\ttverb\textasciitilde    &OT1,T1&   \textasciitilde  &  \\ \hline
\ttverb\textbackslash     &OT1,T1&   \textbackslash   &  \\ \hline
\ttverb\textbar           &OT1,T1&   \textbar         &  \\ \hline
\ttverb\textbraceleft     &OT1,T1&   \textbraceleft   &  \\ \hline
\ttverb\textbraceright    &OT1,T1&   \textbraceright  &  \\ \hline
\ttverb\textcompwordmark  &OT1,T1&   \textcompwordmark& 不可见 \\ \hline
\ttverb\textdollar        &OT1,T1&   \textdollar      &  \\ \hline
\ttverb\textemdash        &OT1,T1&   \textemdash      &  \\ \hline
\ttverb\textendash        &OT1,T1&   \textendash      &  \\ \hline
\ttverb\textexclamdown    &OT1,T1&   \textexclamdown  &  \\ \hline
\ttverb\textgreater       &OT1,T1&   \textgreater     &  \\ \hline
\ttverb\textless          &OT1,T1&   \textless        &  \\ \hline
\ttverb\textquestiondown  &OT1,T1&   \textquestiondown&  \\ \hline
\ttverb\textquotedbl      &T1    &   \textquotedbl    &  \\ \hline
\ttverb\textquotedblleft  &OT1,T1&   \textquotedblleft&  \\ \hline
\ttverb\textquotedblright &OT1,T1&   \textquotedblright& \\ \hline
\ttverb\textquoteleft     &OT1,T1&   \textquoteleft   &  \\ \hline
\ttverb\textquoteright    &OT1,T1&   \textquoteright  &  \\ \hline
\ttverb\textregistered    &OT1,T1&   \textregistered  &  \\ \hline
\ttverb\textsection       &OT1,T1&   \textsection     &  \\ \hline
\ttverb\textsterling      &OT1,T1&   \textsterling    &  \\ \hline
\ttverb\texttrademark     &OT1,T1&   \texttrademark   &  \\ \hline
\ttverb\textunderscore    &OT1,T1&   \textunderscore  &  \\ \hline
\ttverb\textvisiblespace  &OT1,T1&   \textvisiblespace&  \\ \hline
\ttverb\th                &T1    &   \th              & \\ \hline
\end{longtable}
\end{center}


\section{基于 \TeX\ 系统的 Unicode 编码}
\label{sec:unicode}

前面的文本(text)假定了一个经典的 TeX 系统,该系统仅限于使用最多 256 个字符(characters)的字体。为了容纳不同语言(languages)和数学(mathematics)所需的所有字符(characters),有必要像上面描述的那样使用多种编码(multiple encodings),并且 \LaTeX\ 需要了解每种字体所使用的编码。

Unicode 的目标是提供一种单一的编码(single encoding),除了非常专业地使用非标准字符(non-standard characters)之外,这种编码可以消除大部分切换编码(switch encodings)的需要。尽管由于技术原因,并不是所有的插槽(slots)都可用于不同的字符,但是 0--256 (十六进制 FF)的 Unicode 代码在 0--1,114,111(十六进制 FFFF)范围内。Unicode 为所有文档提供了使用单一输入编码(single input encoding)(通常为 UTF-8)的可能性,并为所有字体提供了基本相同的 Unicode 编码,因此无需在不同上下文中切换编码。

Omega 可能是第一个广泛使用的支持 Unicode 的 \TeX\ 扩展。目前,在大多数现代 \TeX\ 发行版(distributions)中存在的两个积极支持的系统是 Xe\TeX\ 和 Lua\TeX。

当与这些扩展的 \TeX\ 引擎一起使用时,\LaTeX\ 的字体系统可以调用 Unicode 字体(通常是在操作系统上安装的 OpenType 字体,而不是专门为 \TeX\ 编码/安装的字体)。目前访问这些字体的常用方法是通过贡献的(contributed) \Pkg{fontspec}\ 宏包。这使用了编码 \Enc{TU}:“\TeX{} Unicode”(历史上使用了两种实验性编码 \Enc{EU1}\ 和 \Enc{EU2},具体取决于引擎,但不推荐使用)。Unicode 引擎的 \LaTeX\ 编码的确切规则(exact rules)尚未最终确定,因为(通常)要求每个槽(slot)都应定义。(这对于 Unicode 字体来说是不现实的,因为几乎所有字体都会处理整个范围的子集。)很少需要指定 \Enc{TU}\ 编码文档,因为 \Pkg{fontspec}\ 宏包在加载时设置了正确的编码。

第~\ref{sec:restrictions}~节描述的限制(restrictions)不适用,或者需要在基于 Unicode 的引擎中进行修改。显然,小写表(lowercase table)(和连字符模式[hyphenation patterns])不能局限于用于 \Enc{T1}\ 的值,并且只涉及前 256 个字符(characters)。

如果使用的是经典的 \TeX\ 或 pdf\TeX,则使用 \LaTeX\ 格式时, \LaTeX\ 会设置小写表(lowercase table)并根据 \Enc{T1}\ 将字符(characters)分类为字母或非字母。如果检测到基于 Unicode 的 \TeX,那么这些值将基于 Unicode 字符数据库(Unicode Character Database) \cite{ucd} 提供的分类(classification)和小写映射(lower-case mappings)。\LaTeX{}\ 团队已经编写了一个通用的加载程序包(generic loader bundle) 即 \Pkg{unicode-data},它提供了一种机制来直接从 Unicode 字符数据库(Unicode Character Database)数据文件(data files)加载这些信息,并在格式构建(format-building)过程中检测到一个与 Unicode 兼容的(Unicode-compliant)引擎时读取这些信息。

类似地,在现代 \TeX\ 发行版(distribution)使用的默认配置文件(default configuration files)中,每种受支持语言的连字符文件(hyphenation files)都是用 UTF-8 编码编写的,对所有字母都使用 Unicode 代码点(Unicode code points),然后,如果检测到一个经典的 \TeX\ 系统,就会加载一些额外的宏(additional macros),以便在可能的情况下将这些文件转换为 256 个字符的编码(256-character encodings),并假设为 \Enc{T1}\ 小写表(lowercase table)。对于 Unicode 引擎,不进行转换。(少数语言的连字符模式[hyphenation patterns]要求某些标点符号[punctuation characters]具有非零 c 值。这是在模式读取期间设置的,将来可能会在某个阶段使用 e-\TeX{} \verb=\savinghyphcodes= 机制,以避免操作文档中的 \verb=\lccode=。)


\begin{thebibliography}{99}
\addcontentsline{toc}{section}{\numberline{\relax}\refname}


\bibitem{Adobe:PDF-1.6} \emph{\textsc{PDF} reference}:
    Adobe portable document format version~1.6.  Adobe Systems
    Incorporated, 2005. % why \textsuperscript{3}?
  \url{http://partners.adobe.com/public/developer/en/pdf/PDFReference16.pdf}.

\bibitem{Beeton:TB6-3-124} Barbara Beeton:
  \emph{Mathematical symbols and cyrillic fonts ready for
      distribution}.  In: TUGBoat, 6\#3), 1985.
  \url{http://tug.org/TUGboat/Articles/tb06-3/tb13beetcyr.pdf}.

\bibitem{beeton} Barbara Beeton: \emph{Unicode
      and math, a combination whose time has come -- Finally!}.  In:
  TUGBoat, 21\#3, 2000.
  \url{http://www.tug.org/TUGboat/Articles/tb21-3/tb68beet.pdf}.


\bibitem{Berdnikov:eurotex-98} A.\@ Berdnikov, O.\@
  Lapko, M.\@ Kolodin, A.\@ Janishevsky and
  A.\@ Burykin: \emph{The Encoding Paradigm in
      \LaTeXe{} and the Projected X2 Encoding for Cyrillic Texts}.
  Euro\TeX~98.
  \url{http://www.gutenberg.eu.org/pub/GUTenberg/publicationsPDF/28-29-berdnikova.pdf}.

\bibitem{CJK} \emph{The \Pkg{CJK} package}:
  \url{http://cjk.ffii.org}.

\bibitem{clasen} Matthias Clasen: \emph{A new
      implementation of \LaTeX{} math}, 1997-98.
  \url{http://www.tug.org/twg/mfg/papers/current/newmath.ps.gz}.

\bibitem{clasen-vieth} Matthias Clasen and Ulrik
  Vieth: \emph{Towards a new Math Font Encoding
      for (La)\TeX}.  March 1998,
  \url{http://www.tug.org/twg/mfg/papers/current/mfg-euro-all.ps.gz}.

\bibitem{CorkGW:91}
Dean Guenther and Janene Winter.
\newblock An international phonetic alphabet.
\newblock In Guenther \cite{proc:MGu91}, pages 149--156.
\newblock Published as {TUG}boat 12\#1.

\bibitem{proc:MGu91}
Mary Guenther, editor.
\newblock {\em {\TeX} 90 Conference Proceedings}, March 1991.
\newblock Published as {TUG}boat 12\#1.

\bibitem{tub:MFe90}
Michael~J. Ferguson.
\newblock Report on multilingual activities.
\newblock {\em {TUG}boat}, 11(4):514--516, 1990.

\bibitem{fontinst} \emph{The \Pkg{fontinst} package}:
  \textlangle CTAN\textrangle\url{/fonts/utilities/fontinst}.

\bibitem{Rei:TB17-2-102} Fukui Rei:
  \emph{\textsl{TIPA}: A system for processing phonetic
      symbols in \LaTeX}.  In: TUGBoat, 17\#, 1996.
  \url{http://www.tug.org/TUGboat/Articles/tb17-2/tb51rei.pdf}.

\bibitem{hyperref} \emph{The \Pkg{hyperref} package}:
  \url{http://www.tug.org/applications/hyperref}.

\bibitem{tub:JKn93}
J\"org Knappen.
\newblock Fonts for Africa: The fc Fonts.
\newblock {\em {TUG}boat}, 14(2):104, 1993.

\bibitem{Knappen:TB17-2-96} J\"org Knappen:
  \emph{The \Pkg{dc} fonts~1.3: Move towards stability
      and completeness}.  In: TUGBoat 17\#2, 1996.
  \url{http://www.tug.org/TUGboat/Articles/tb17-2/tb51knap.pdf}.

\bibitem{A-W:DKn86}
Donald~E. Knuth.
\newblock {\em The {\TeX}book}.
\newblock Volume~A of {\em Computers \& {T}ypesetting\/},
  May 1989.
\newblock Eight printing.

\bibitem{vnr} \emph{The \Pkg{vnr} font family}, developed by
   the author of pdf\TeX, {H\`an Th\^e\protect\llap{\raise 0.5ex\hbox{\'{\relax}}} Th\`anh}.
   \url{http://vntex.org/download/vntex}.

 \bibitem{ipa} Home page of the International Phonetic Association.
   \url{http://www.arts.gla.ac.uk/IPA/ipa.html}

\bibitem{A-W:LLa94}
Leslie Lamport.
\newblock {\em {\LaTeX:} A Document Preparation System}.
\newblock Addison-Wesley, Reading, Massachusetts, second edition, 1994.

\bibitem{LH-Fonts} \emph{The \Pkg{lh}-Fonts for Cyrillic}:
  \textlangle CTAN\textrangle\url{/fonts/cyrillic/lh}.

\bibitem{A-W:MG2004}
Frank Mittelbach and Michel Goossens.
\newblock {\em The {\LaTeX} Companion second edition}.
\newblock With Johannes Braams, David Carlisle, and Chris Rowley.
\newblock Addison-Wesley, Reading, Massachusetts, 2004.

\bibitem{Unicode} \emph{The Unicode Standard}.
  \url{http://unicode.org}.

\bibitem{ucd} \emph{The Unicode Character Database}.
  \url{http://unicode.org/ucd}.

\bibitem{ziegler} Justin Ziegler, \emph{Technical
    Report on Math Font Encodings}, June 1994,
  \url{http://www.tug.org/twg/mfg/papers/ltx3pub/l3d007.ps.gz}.

\end{thebibliography}

\clearpage\appendix
\begin{center}
  \Large\bfseries 附录
\end{center}

\section{代码表(code tables)示例}

这个附录(appendix)包含上面提到的“示例(example)”字体的每种字体的表格(table),规定在使用 \LaTeX{}\ 处理文档时该字体是可用的。(\LaTeX{}\ 会为找不到的每种字体生成一条警告消息。)

\subsection{文本编码(Text encodings)}

\ftable{cmr10}{OT1}

\ftable{wnr10}{OT2}

\ftable{wsuipa10}{OT3}

\ftable{plr10}{OT4}

%\ftable{artmr10}{OT6}

\ftable{ecrm1000}{T1}

\ftable{larm1000}{T2A}

\ftable{lbrm1000}{T2B}

\ftable{lcrm1000}{T2C}

\ftable{tipa10}{T3}

\ftable{fcr10}{T4}

\ftable{vnr10}{T5}


\subsection{文本符号(text symbol)编码}

欧洲计算机现代家族(European Computer Modern family)提供的 \Enc{TS1}\ 完整表:
\ftable{tcrm1000}{TS1}

\pagebreak

相比之下,典型的 Postscript 字体通常不完全实现(incomplete implementations) \Enc{TS1},有时缺少一半以上的字形(glyphs):

\ftable{ptmr8c}{TS1}

\ftable{tipx10}{TS3}



\subsection{扩展文本(extended text)编码}

\ftable{rxrm1000}{X2}


\subsection{数学编码}

\ftable{cmmi10}{OML}

\ftable{cmsy10}{OMS}

\ftable{cmex10}{OMX}


\subsection{其他编码}

\ftable{ptmr8y}{LY1}

%%\ftable{????}{LV1}

\ftable{grmn1000}{LGR}

\ftable{wasy10}{U}
\ftable{logo10}{U}

\clearpage
\section{大小写表}
\label{sec:uclc-tab}

下面的两组表列出了 \LaTeX{}\ 标准256字符表(standard 256-character tables)中每个位置的 \verb"\uppercase" 和
\verb"\lowercase" 的值。

每个表的每一行列出:
\begin{quote}
\begin{center}
  \begin{tabular}{|r|p{0.8\textwidth}|}
    \hline
    pos & 在表(0-255)中的槽位(position) \\ \hline
    lc  & 槽位(position)处的 \verb"\lowercase" 表中的值 \\ 
        & (注意,此处的值 0 表示 \verb"\lowercase" 对该字符无效,并且连字符不适用于此字符)\\  \hline
    uc  & 槽位(position)处的 \verb"\uppercase" 表中的值 \\ 
        & (注意,此处的值 0 表示 \verb"\uppercase" 对该字符无效) \\ \hline
    glyphs & 为该槽位的 T1 编码指定的字形\\
           &布局为 \meta{字形}\textbf{(}\meta{小写字形}\textbf{/}\meta{大写字形}\textbf{)}\\ \hline
  \end{tabular}
\end{center}
\end{quote}

\begin{center}
  \let\lctablenumbersize\footnotesize
  \makebox[\textwidth]{\hss
    \dolctable{0}{32}\quad\dolctable{32}{64}\quad
    \dolctable{64}{96}\quad\dolctable{96}{128}%
  \hss}

  \makebox[\textwidth]{\hss
    \dolctable{128}{160}\quad\dolctable{160}{192}\quad
    \dolctable{192}{224}\quad\dolctable{224}{256}%
  \hss}
\end{center}
\end{document}


%%%%%%%%%%%%%%%%%%%%%%%%%%%%%%%%%%%%%%%%%%%%%%%%%%%%%%%%%%%%%%%
