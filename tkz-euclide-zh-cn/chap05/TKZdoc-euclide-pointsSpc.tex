\documentclass[../main.tex]{subfiles}
\begin{document}
% \section{Special points relating to a triangle}
\section{三角形特殊点}

% \subsection{Triangle center: \tkzcname{tkzDefTriangleCenter}}
\subsection{\tkzcname{tkzDefTriangleCenter}命令:定义三角形特殊点}

% This macro allows you to define the center of a triangle.
该命令用于定义与三角形相关的几个特殊点。

% \begin{NewMacroBox}{tkzDefTriangleCenter}{\oarg{local options}\parg{A,B,C}}%
% \tkzHandBomb{}Be careful, the arguments are lists of three points. This macro is
% used in conjunction with \tkzcname{tkzGetPoint} to get the center you are
% looking for. You can use \tkzname{tkzPointResult} if it is not necessary to keep
% the results.
%
% \medskip
% \begin{tabular}{lll}%
% \toprule
% arguments & default & definition \\
%
% \midrule
% \TAline{(pt1,pt2,pt3)}{no default}{three points}
% \midrule
% options             & default & definition                         \\
% \midrule
% \TOline{ortho}  {circum}{intersection of the altitudes of a triangle}
% \TOline{centroid} {circum}{centre of gravity. Intersection of the medians }
% \TOline{circum}{circum}{circle center circumscribed}
% \TOline{in}    {circum}{center of the circle inscribed in a triangle }
% \TOline{ex}    {circum}{center of a circle exinscribed to a triangle }
% \TOline{euler}{circum}{center of Euler's circle }
% \TOline{symmedian} {circum}{Lemoine's point or symmedian centre or Grebe's
% point}
% \TOline{spieker} {circum}{Spieker Circle Center}
% \TOline{nagel}{circum}{Nagel Center}
% \TOline{mittenpunkt} {circum}{also called the middlespoint}
% \TOline{feuerbach}{circum}{Feuerbach Point}
%
% \end{tabular}
% \end{NewMacroBox}
\begin{NewMacroBox}{tkzDefTriangleCenter}{\oarg{命令选项}\parg{A,B,C}}%
\tkzHandBomb{}注意:该命令的参数必须是一个三角形的3个顶点列表,
结合\tkzcname{tkzGetPoint}命令保存定义的点,并为其命名。
当然,仅临时使用,则可使用\tkzcname{tkzPointResult}命令使用该点。

\medskip
\begin{tabular}{lll}%
\toprule
参数 & 默认值 & 含义 \\

\midrule
\TAline{(pt1,pt2,pt3)}{无}{逗号分隔的三角形3个顶点列表}
\midrule
选项             & 默认值 & 含义                         \\
\midrule
\TOline{ortho}  {circum}{垂心,三条高的交点}
\TOline{centroid} {circum}{重心,三条中线的交点}
\TOline{circum}{circum}{外心,外接圆圆心}
\TOline{in}    {circum}{内心,内切圆圆心}
\TOline{ex}    {circum}{旁心,外切圆圆心}
\TOline{euler}{circum}{欧拉点,欧拉圆/费尔巴哈圆/九点圆圆心}
\TOline{symmedian} {circum}{陪位重心,Lemoine点或中间点或Grebe点}
\TOline{spieker} {circum}{Spieker点,中点三角形内切圆圆心}
\TOline{nagel}{circum}{Nagel点(界心),三个旁切圆切点与对应顶点连线的交点}
\TOline{mittenpunkt} {circum}{三个旁切圆圆心与对应边中点连线的交点}
\TOline{feuerbach}{circum}{Feuerbach点,内切圆与九点圆的公切点}

\end{tabular}
\end{NewMacroBox}

% \subsubsection{Option \tkzname{ortho} or \tkzname{orthic}}
\subsubsection{\tkzname{ortho}或\tkzname{orthic}选项:垂心}

% The intersection $H$ of the three altitudes  of a triangle is called the
% orthocenter.
三角形三条高的交点称为三角形的垂心。
\begin{tkzexample}[latex=6cm,small]
\begin{tikzpicture}
  \tkzDefPoint(0,0){A}
  \tkzDefPoint(5,1){B}
  \tkzDefPoint(1,4){C}
  \tkzClipPolygon(A,B,C)
  \tkzDefTriangleCenter[ortho](B,C,A)
    \tkzGetPoint{H}
  \tkzDefSpcTriangle[orthic,name=H](A,B,C){a,b,c}
  \tkzDrawPolygon[color=blue](A,B,C)
  \tkzDrawPoints(A,B,C,H)
  \tkzDrawLines[add=0 and 1](A,Ha B,Hb C,Hc)
  \tkzLabelPoint(H){$H$}
  \tkzAutoLabelPoints[center=H](A,B,C)
  \tkzMarkRightAngles(A,Ha,B B,Hb,C C,Hc,A)
\end{tikzpicture}
\end{tkzexample}

\newpage

% \subsubsection{Option \tkzname{centroid}}
\subsubsection{\tkzname{centroid}选项:重心}

三角形三条中线的交点称为重心。

\begin{tkzexample}[latex=6cm,small]
\begin{tikzpicture}[scale=.75]
  \tkzDefPoints{-1/1/A,5/1/B}
  \tkzDefEquilateral(A,B)
  \tkzGetPoint{C}
  \tkzDefTriangleCenter[centroid](A,B,C)
      \tkzGetPoint{G}
  \tkzDrawPolygon[color=brown](A,B,C)
  \tkzDrawPoints(A,B,C,G)
  \tkzDrawLines[add = 0 and 2/3](A,G B,G C,G)
\end{tikzpicture}
\end{tkzexample}

% \subsubsection{Option \tkzname{circum}}
\subsubsection{\tkzname{circum}选项:外心}

三角形外接圆圆心称为外心,也是三个边垂直平分线的交点。

\begin{tkzexample}[latex=6cm,small]
 \begin{tikzpicture}
  \tkzDefPoints{0/1/A,3/2/B,1/4/C}
  \tkzDefTriangleCenter[circum](A,B,C)
  \tkzGetPoint{G}
  \tkzDrawPolygon[color=brown](A,B,C)
  \tkzDrawCircle(G,A)
  \tkzDrawPoints(A,B,C,G)
 \end{tikzpicture}
\end{tkzexample}

\subsubsection{\tkzname{in}选项:内心}

% In geometry, the incircle or inscribed circle of a triangle is the largest
% circle contained in the triangle; it touches (is tangent to) the three sides.
% The center of the incircle is a triangle center called the triangle's incenter.
% The center of the incircle, called the incenter, can be found as the
% intersection of the three internal angle bisectors. The center of an excircle is
% the intersection of the internal bisector of one angle (at vertex $A$, for
% example) and the external bisectors of the other two. The center of this
% excircle is called the excenter relative to the vertex $A$, or the excenter of
% $A$. Because the internal bisector of an angle is perpendicular to its external
% bisector, it follows that the center of the incircle together with the three
% excircle centers form an orthocentric
% system.(\url{https://en.wikipedia.org/wiki/Incircle_and_excircles_of_a_triangle})
几何学中,三角形的内切圆是三角形中最大的圆,内切圆的圆心称为三角形的内心。
三角形的内也是三角形三个内角角平分线的交点。
(\url{https://en.wikipedia.org/wiki/Incircle_and_excircles_of_a_triangle})

\medskip

% We get the centre of the inscribed circle of the triangle. The result is of
% course in \tkzname{tkzPointResult}. We can retrieve it with \tkzcname{tkzGetPoint}.
% 得到的内心存储在\tkzname{tkzPointResult}命令中,可以通过\tkzcname{tkzGetPoint}检索该点。

\begin{tkzexample}[latex=6cm,small]
\begin{tikzpicture}[scale=1.25]
  \tkzDefPoints{0/1/A,3/2/B,1/4/C}
  \tkzDefTriangleCenter[in](A,B,C)\tkzGetPoint{I}
  \tkzDefPointBy[projection=onto A--C](I)
  \tkzGetPoint{Ib}
  \tkzDrawPolygon[color=blue](A,B,C)
  \tkzDrawPoints(A,B,C,I)
  \tkzDrawLines[add = 0 and 2/3](A,I B,I C,I)
  \tkzDrawCircle(I,Ib)
\end{tikzpicture}
\end{tkzexample}

\newpage

% \subsubsection{Option \tkzname{ex}}
\subsubsection{\tkzname{ex}选项:旁心}

% An excircle or escribed circle of the triangle is a circle lying outside the
% triangle, tangent to one of its sides and tangent to the extensions of the other
% two. Every triangle has three distinct excircles, each tangent to one of the
% triangle's sides.
% (\url{https://en.wikipedia.org/wiki/Incircle_and_excircles_of_a_triangle})
旁切圆圆心是一个顶点(例如顶点$A$)的内角角平分线与另外两个外角角平分线的交点。
该旁切圆圆心是相对于顶点$A$的一个旁心,或叫作$A$的旁心。
因为三角形内角角平分线与对应的外角角平分线垂直,所以,内心与3个旁心构成了一个正交系统。
旁切圆是位于三角形外部与某条边及另外两条边的延长线相切的圆,一个三角形有3个旁切圆。
(\url{https://en.wikipedia.org/wiki/Incircle_and_excircles_of_a_triangle})

% We get the centre of an inscribed circle of the triangle. The result is of
% course in \tkzname{tkzPointResult}. We can retrieve it with \tkzcname{tkzGetPoint}.
% 得到的旁心存储在\tkzname{tkzPointResult}命令中,可以通过\tkzcname{tkzGetPoint}检索该点。

\begin{tkzexample}[latex=7.5cm,small]
\begin{tikzpicture}[scale=0.70]
  \tkzDefPoints{0/1/A,3/2/B,1/4/C}
  \tkzDefTriangleCenter[ex](B,C,A)
  \tkzGetPoint{J_c}
  \tkzDefPointBy[projection=onto A--B](J_c)
  \tkzGetPoint{Tc}
  %or
  % \tkzDefCircle[ex](B,C,A)
  % \tkzGetFirstPoint{J_c}
  % \tkzGetSecondPoint{Tc}
  \tkzDrawPolygon[color=blue](A,B,C)
  \tkzDrawPoints(A,B,C,J_c)
  \tkzDrawCircle[red](J_c,Tc)
  \tkzDrawLines[add=1.5 and 0](A,C B,C)
  \tkzLabelPoints(J_c)
\end{tikzpicture}
\end{tkzexample}

\vspace*{-10pt}

% \subsubsection{Option \tkzname{euler}}
\subsubsection{\tkzname{euler}选项:欧拉点}

% This macro allows to obtain the center of the circle of the nine points or
% euler's circle or Feuerbach's circle.
% The nine-point circle, also called Euler's circle or the Feuerbach circle, is
% the circle that passes through the perpendicular feet $H_A$, $H_B$, and $H_C$
% dropped from the vertices of any reference triangle $ABC$ on the sides opposite
% them. Euler showed in 1765 that it also passes through the midpoints $M_A$,
% $M_B$, $M_C$ of the sides of $ABC$. By Feuerbach's theorem, the nine-point
% circle also passes through the midpoints $E_A$, $E_B$, and $E_C$ of the segments
% that join the vertices and the orthocenter $H$. These points are commonly
% referred to as the Euler points.
% (\url{https://mathworld.wolfram.com/Nine-PointCircle.html})
欧拉点是欧拉圆的圆心,欧拉圆又称九点圆或费尔巴哈圆的圆心。欧拉圆是通过三角形$ABC$三个顶点
向对边作垂线形成的三个垂脚$H_A$、$H_B$和$H_C$的圆,欧拉在1765年证明该圆同时通过三角形
$ABC$三个边的中点$M_A$、$M_B$和$M_C$。根据费尔巴哈定理,欧拉圆也通过三角形$ABC$三个顶点
与重心$H$连线线段的中点$E_A$、$E_B$和$E_C$。
(\url{https://mathworld.wolfram.com/Nine-PointCircle.html})


\begin{tkzexample}[latex=7.5cm,small]
\begin{tikzpicture}[scale=1]
  \tkzDefPoints{0/0/A,6/0/B,0.8/4/C}
  \tkzDefSpcTriangle[medial,
      name=M](A,B,C){_A,_B,_C}
  \tkzDefTriangleCenter[euler](A,B,C)
    \tkzGetPoint{N} % I= N nine points
  \tkzDefTriangleCenter[ortho](A,B,C)
    \tkzGetPoint{H}
  \tkzDefMidPoint(A,H) \tkzGetPoint{E_A}
  \tkzDefMidPoint(C,H) \tkzGetPoint{E_C}
  \tkzDefMidPoint(B,H) \tkzGetPoint{E_B}
  \tkzDefSpcTriangle[ortho,name=H](A,B,C){_A,_B,_C}
  \tkzDrawPolygon[color=blue](A,B,C)
  \tkzDrawCircle(N,E_A)
  \tkzDrawSegments[blue](A,H_A B,H_B C,H_C)
  \tkzDrawPoints(A,B,C,N,H)
  \tkzDrawPoints[red](M_A,M_B,M_C)
  \tkzDrawPoints[blue]( H_A,H_B,H_C)
  \tkzDrawPoints[green](E_A,E_B,E_C)
  \tkzAutoLabelPoints[center=N, font=\scriptsize]%
(A,B,C,M_A,M_B,M_C,H_A,H_B,H_C,E_A,E_B,E_C)
  \tkzLabelPoints[font=\scriptsize](H,N)
  \tkzMarkSegments[mark=s|,size=3pt,
     color=blue,line width=1pt](B,E_B E_B,H)
\end{tikzpicture}
\end{tkzexample}

% \subsubsection{Option \tkzname{symmedian}}
\subsubsection{\tkzname{symmedian}选项:陪位重心}

设$AN$、$BM$、$CE$是三角形$ABC$的三条中线,
$N'$、$M'$、$E'$分别是三条边$BC$、$CA$、$AB$上的点,
若$\widehat{BAN'}=\widehat{NAC}$、
$\widehat{CBM}=\widehat{M'BA}$、
$\widehat{ACE}=\widehat{E'CB}$,
则三条直线$AN'$、$BM'$、$CE'$交于一点$K$,
该点称为陪位重心。

\begin{tkzexample}[latex=6cm,small]
\begin{tikzpicture}
  \tkzDefPoint(0,0){A}
  \tkzDefPoint(5,0){B}
  \tkzDefPoint(1,4){C}
  \tkzDefTriangleCenter[symmedian](A,B,C) \tkzGetPoint{K}
  \tkzDefTriangleCenter[median](A,B,C)    \tkzGetPoint{G}
  \tkzDefTriangleCenter[in](A,B,C)        \tkzGetPoint{I}
  \tkzDefSpcTriangle[centroid,name=M](A,B,C){a,b,c}
  \tkzDefSpcTriangle[incentral,name=I](A,B,C){a,b,c}
  \tkzDrawPolygon[color=blue](A,B,C)
  \tkzDrawLines[add = 0 and 2/3,blue](A,K B,K C,K)
  \tkzDrawSegments[red,dashed](A,Ma B,Mb C,Mc)
  \tkzDrawSegments[orange,dashed](A,Ia B,Ib C,Ic)
  \tkzDrawLine[add=2 and 2](G,I)
  \tkzDrawPoints(A,B,C,K,G,I)
\end{tikzpicture}
\end{tkzexample}

% \subsubsection{Option \tkzname{nagel}}
\subsubsection{\tkzname{nagel}选项:界心(Nagel点)}

% Let $Ta$ be the point at which the excircle with center $Ja$ meets the side $BC$
% of a triangle $ABC$, and define $Tb$ and $Tc$ similarly. Then the lines $ATa$,
% $BTb$, and $CTc$ concur in the Nagel point $Na$.
% \href{https://mathworld.wolfram.com/NagelPoint.html}{Weisstein, Eric W. \enquote{Nagel
% point}. From MathWorld--A Wolfram Web Resource.}
令$Ta$、$Tb$和$Tc$分别为旁切圆与三角形三条边的切点,
连线$ATa$、$BTb$和$CTc$,其交点称为Nagel点,俗称三角形的\enquote{界心}。
\href{https://mathworld.wolfram.com/NagelPoint.html}{Weisstein, Eric W. \enquote{Nagel
point}. From MathWorld--A Wolfram Web Resource.}

\begin{tkzexample}[latex=7cm,small]
\begin{tikzpicture}[scale=0.45]
  \tkzDefPoints{0/0/A,6/0/B,4/6/C}
  \tkzDefSpcTriangle[ex](A,B,C){Ja,Jb,Jc}
  \tkzDefSpcTriangle[extouch](A,B,C){Ta,Tb,Tc}
  \tkzDrawPoints(Ja,Jb,Jc,Ta,Tb,Tc)
  \tkzLabelPoints(Ja,Jb,Jc,Ta,Tb,Tc)
  \tkzDrawPolygon[blue](A,B,C)
  \tkzDefTriangleCenter[nagel](A,B,C) \tkzGetPoint{Na}
  \tkzDrawPoints[blue](B,C,A)
  \tkzDrawPoints[red](Na)
  \tkzLabelPoints[blue](B,C,A)
  \tkzLabelPoints[red](Na)
  \tkzDrawLines[add=0 and 1](A,Ta B,Tb C,Tc)
  \tkzShowBB\tkzClipBB
  \tkzDrawLines[add=1 and 1,dashed](A,B B,C C,A)
  \tkzDrawCircles[ex,gray](A,B,C C,A,B B,C,A)
  \tkzDrawSegments[dashed](Ja,Ta Jb,Tb Jc,Tc)
  \tkzMarkRightAngles[fill=gray!20](Ja,Ta,C
              Jb,Tb,A Jc,Tc,B)
\end{tikzpicture}
\end{tkzexample}

\newpage

% \subsubsection{Option \tkzname{mittenpunkt}}
\subsubsection{\tkzname{mittenpunkt}选项}

三个旁切圆圆心与对应边中点连线的交点。

\begin{tkzexample}[latex=8cm,small]
\begin{tikzpicture}[scale=.5]
  \tkzDefPoints{0/0/A,6/0/B,4/6/C}
  \tkzDefSpcTriangle[centroid](A,B,C){Ma,Mb,Mc}
  \tkzDefSpcTriangle[ex](A,B,C){Ja,Jb,Jc}
  \tkzDefSpcTriangle[extouch](A,B,C){Ta,Tb,Tc}
  \tkzDefTriangleCenter[mittenpunkt](A,B,C)
  \tkzGetPoint{Mi}
  \tkzDrawPoints(Ma,Mb,Mc,Ja,Jb,Jc)
  \tkzClipBB
  \tkzDrawPolygon[blue](A,B,C)
  \tkzDrawLines[add=0 and 1](Ja,Ma
               Jb,Mb Jc,Mc)
  \tkzDrawLines[add=1 and 1](A,B A,C B,C)
  \tkzDrawCircles[gray](Ja,Ta Jb,Tb Jc,Tc)
  \tkzDrawPoints[blue](B,C,A)
  \tkzDrawPoints[red](Mi)
  \tkzLabelPoints[red](Mi)
  \tkzLabelPoints[left](Mb)
  \tkzLabelPoints(Ma,Mc,Jb,Jc)
  \tkzLabelPoints[above left](Ja,Jc)
  \tkzShowBB
\end{tikzpicture}
\end{tkzexample}
%<---------------------------------------------------------------------->
%<---------------------------------------------------------------------->
% \section{Draw a point}
\section{绘制点}

% \subsubsection{Drawing points \tkzcname{tkzDrawPoint}} \hypertarget{tdrp}{}
\subsection{\tkzcname{tkzDrawPoint}命令:绘制点} \hypertarget{tdrp}{}

% \begin{NewMacroBox}{tkzDrawPoint}{\oarg{local options}\parg{name}}%
% \begin{tabular}{lll}%
% arguments &  default & definition                 \\
% \midrule
% \TAline{name of point} {no default}  {Only one point name is accepted}
% \bottomrule
% \end{tabular}
%
% \medskip
% The argument is required. The disc takes the color of the circle, but  lighter.
% It is possible to change everything. The point is a node and therefore it is
% invariant if the drawing is modified by scaling.
%
% \medskip
% \begin{tabular}{lll}%
% \toprule
% options             & default & definition \\
% \midrule
% \TOline{shape}  {circle}{Possible \tkzname{cross} or \tkzname{cross out}}
% \TOline{size}  {6}{$6 \times$ \tkzcname{pgflinewidth}}
% \TOline{color}  {black}{the default color can be changed }
% \bottomrule
% \end{tabular}
% \medskip
% We can create other forms such as \tkzname{cross}
% \end{NewMacroBox}
\begin{NewMacroBox}{tkzDrawPoint}{\oarg{命令选项}\parg{名称}}%
\begin{tabular}{lll}%
参数 &  默认值 & 含义                 \\
\midrule
\TAline{点的名称} {无}  {只能有一个点的名称}
\bottomrule
\end{tabular}

\medskip
必选参数,点圆盘用填充色绘制,但颜色较浅,
可以通过选项实现更多效果。
由于采用node的方式绘制,
因此,缩放操作不影响点的尺寸。

\medskip
\begin{tabular}{lll}%
\toprule
选项             & 默认值 & 含义 \\
\midrule
\TOline{shape}  {circle}{可以使用\tkzname{cross}或\tkzname{cross out}}
\TOline{size}  {6}{$6 \times$ \tkzcname{pgflinewidth}}
\TOline{color}  {black}{默认颜色,可以修改}
\bottomrule
\end{tabular}
\medskip
能够创建如\tkzname{cross}的形式。
\end{NewMacroBox}

\vspace*{-30pt}

% \subsubsection{Example of point drawings}
\subsubsection{点绘制示例}

% Note that \tkzname{scale} does not affect the shape of the dots. Which is
% normal.  Most of the time, we are satisfied with a single point shape that we
% can define from the beginning, either with a macro or by modifying a
% configuration file.
注意,缩放不会影响点的形状,多数情况下,
无论采用宏还是通过修改配置文件从一开始就定义一个点的形状,
一般都可以得到令人满意的效果。

\begin{tkzexample}[latex=5cm,small]
\begin{tikzpicture}[scale=0.75]
  \tkzDefPoint(1,3){A}
  \tkzDefPoint(4,1){B}
  \tkzDefPoint(0,0){O}
  \tkzDrawPoint[color=red](A)
  \tkzDrawPoint[fill=blue!20,draw=blue](B)
  \tkzDrawPoint[color=green](O)
\end{tikzpicture}
\end{tkzexample}

\vfill

\subsection{\tkzcname{tkzDrawPoints}命令:绘制多个点}
% It is possible to draw several points at once but this macro is a little slower
% than the previous one. Moreover, we have to make do with the same options for
% all the points.
可以一次绘制多个点,但该命令比绘制单点慢。
另外,一次绘制多个点时,所有点使用相同选项。

\newpage

\hypertarget{tdrps}{}
% \begin{NewMacroBox}{tkzDrawPoints}{\oarg{local options}\parg{liste}}%
% \begin{tabular}{lll}%
% arguments &  default  & definition \\
% \midrule
% \TAline{points list}{no default}{example \tkzcname{tkzDrawPoints(A,B,C)}}
% \bottomrule
% \end{tabular}
%
% \medskip
% \begin{tabular}{lll}%
% options             & default & definition \\
% \midrule
% \TOline{shape}  {circle}{Possible \tkzname{cross} or \tkzname{cross out}}
% \TOline{size}  {6}{$6 \times$ \tkzcname{pgflinewidth}}
% \TOline{color}  {black}{the default color can be changed }
% \bottomrule
% \end{tabular}
% \medskip
% \tkzHandBomb{}Beware of the final \enquote{s}, an oversight leads to cascading errors if
% you try to draw multiple points. The options are the same as for the previous
% macro.
% \end{NewMacroBox}
\begin{NewMacroBox}{tkzDrawPoints}{\oarg{命令选项}\parg{点列表}}%
\begin{tabular}{lll}%
参数 &  默认值  & 含义 \\
\midrule
\TAline{点列表}{无}{例如:\tkzcname{tkzDrawPoints(A,B,C)},各点间用逗号分隔。}
\bottomrule
\end{tabular}

\medskip
\begin{tabular}{lll}%
选项             & 默认值 & 含义 \\
\midrule
\TOline{shape}  {circle}{可以是\tkzname{cross}或\tkzname{cross out}}
\TOline{size}  {6}{$6 \times$ \tkzcname{pgflinewidth}}
\TOline{color}  {black}{默认为黑色,可以被修改}
\bottomrule
\end{tabular}
\medskip
\tkzHandBomb{}命令最后有一个\enquote{s},如果没有\enquote{s}则会发生错误。
\end{NewMacroBox}

\vspace*{-20pt}

% \subsubsection{First example}
\subsubsection{示例1}

\begin{tkzexample}[latex=7cm,small]
\begin{tikzpicture}[scale=0.75]
  \tkzDefPoint(1,3){A}
  \tkzDefPoint(4,1){B}
  \tkzDefPoint(0,0){C}
  \tkzDrawPoints[size=6,color=red,
                 fill=red!50](A,B,C)
\end{tikzpicture}
\end{tkzexample}

% \subsubsection{Second example}
\subsubsection{示例2}

\begin{tkzexample}[latex=7cm,small]
\begin{tikzpicture}[scale=.5]
  \tkzDefPoint(2,3){A}  \tkzDefPoint(5,-1){B}
  \tkzDefPoint[label=below:$\mathcal{C}$,
               shift={(2,3)}](-30:5.5){E}
  \begin{scope}[shift=(A)]
    \tkzDefPoint(30:5){C}
  \end{scope}
  \tkzCalcLength[cm](A,B)\tkzGetLength{rAB}
  \tkzDrawCircle[R](A,\rAB cm)
  \tkzDrawSegment(A,B)
  \tkzDrawPoints(A,B,C)
  \tkzLabelPoints(B,C)
  \tkzLabelPoints[above](A)
\end{tikzpicture}
\end{tkzexample}

% \section{Point on line or circle}
\section{直线或圆上的点}

% \subsection{Point on a line}
\subsection{\tkzcname{tkzDefPointOnLine}命令:定义直线上的点}

% \begin{NewMacroBox}{tkzDefPointOnLine}{\oarg{local options}\parg{A,B}}%
% \begin{tabular}{lll}%
% arguments &  default & definition                 \\
% \midrule
% \TAline{pt1,pt2} {no default}  {Two points to define a line}
% \bottomrule
% \end{tabular}
%
% \medskip
% \begin{tabular}{lll}%
% options       & default & definition \\
% \midrule
% \TOline{pos=nb}  {}{nb is a decimal  }
% \end{tabular}
% \end{NewMacroBox}
\begin{NewMacroBox}{tkzDefPointOnLine}{\oarg{命令选项}\parg{A,B}}%
\begin{tabular}{lll}%
参数 &  默认值 & 含义                 \\
\midrule
\TAline{pt1,pt2} {无}  {定义直线的两个点}
\bottomrule
\end{tabular}

\medskip
\begin{tabular}{lll}%
选项       & 默认值 & 含义 \\
\midrule
\TOline{pos=nb}  {无}{距离起点A的比例}
\end{tabular}
\end{NewMacroBox}

\newpage

% \subsubsection{Use of option \tkzname{pos}}
\subsubsection{\tkzname{pos}选项}

\begin{tkzexample}[latex=9cm,small]
\begin{tikzpicture}
  \tkzDefPoints{0/0/A,4/0/B}
  \tkzDrawLine[red](A,B)
  \tkzDefPointOnLine[pos=1.2](A,B)
  \tkzGetPoint{P}
  \tkzDefPointOnLine[pos=-0.2](A,B)
  \tkzGetPoint{R}
  \tkzDefPointOnLine[pos=0.5](A,B)
  \tkzGetPoint{S}
  \tkzDrawPoints(A,B,P)
  \tkzLabelPoints(A,B)
  \tkzLabelPoint[above](P){pos=$1.2$}
  \tkzLabelPoint[above](R){pos=$-.2$}
  \tkzLabelPoint[above](S){pos=$.5$}
  \tkzDrawPoints(A,B,P,R,S)
  \tkzLabelPoints(A,B)
\end{tikzpicture}
\end{tkzexample}

% \subsection{Point on a circle}
\subsection{\tkzcname{tkzDefPointOnCircle}命令:定义圆上的点}

% \begin{NewMacroBox}{tkzDefPointOnCircle}{\oarg{local options}}%
% \begin{tabular}{lll}%
% options   & default & definition \\
% \midrule
% \TOline{angle}  {0}{angle formed with the abscissa axis}
% \TOline{center}  {|tkzPointResult|}{circle center required}
% \TOline{radius}  {|\BS tkzLengthResult|}{radius circle}
% \end{tabular}
% \end{NewMacroBox}
\begin{NewMacroBox}{tkzDefPointOnCircle}{\oarg{命令选项}}%
\begin{tabular}{lll}%
选项   & 默认值 & 含义 \\
\midrule
\TOline{angle}  {0}{与横轴夹角}
\TOline{center}  {|\BS tkzPointResult|}{圆心}
\TOline{radius}  {|\BS tkzLengthResult|}{半径}
\end{tabular}
\end{NewMacroBox}

\begin{tkzexample}[latex=6.5cm,small]
\begin{tikzpicture}[scale=1.15]
  \tkzDefPoints{0/0/A,4/0/B,0.8/3/C}
  \tkzDefPointOnCircle[angle=90,center=B,radius=1 cm]
  \tkzGetPoint{I}
  \tkzDefCircle[circum](A,B,C)
  \tkzGetPoint{G} \tkzGetLength{rG}
  \tkzDefPointOnCircle[angle=30,center=G,radius=\rG pt]
  \tkzGetPoint{J}
  \tkzDrawCircle[R,teal](B,1cm)
  \tkzDrawPoint[teal](I)
  \tkzDrawPoints(A,B,C)
  \tkzDrawCircle(G,J)
  \tkzDrawPoints(G,J)
  \tkzDrawPoint[red](J)
  \tkzLabelPoints(G,J)
\end{tikzpicture}
\end{tkzexample}

\end{document}
\endinput
