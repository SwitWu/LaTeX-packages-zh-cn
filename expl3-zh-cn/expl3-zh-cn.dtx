% \iffalse meta-comment
%
%% File: expl3.dtx
%
% Copyright (C) 1990-2022 The LaTeX Project
%
% It may be distributed and/or modified under the conditions of the
% LaTeX Project Public License (LPPL), either version 1.3c of this
% license or (at your option) any later version.  The latest version
% of this license is in the file
%
%    https://www.latex-project.org/lppl.txt
%
% This file is part of the "l3kernel bundle" (The Work in LPPL)
% and all files in that bundle must be distributed together.
%
% -----------------------------------------------------------------------
%
% The development version of the bundle can be found at
%
%    https://github.com/latex3/latex3
%
% for those people who are interested.
%
%<*driver|generic|package|2ekernel>
%</driver|generic|package|2ekernel>
\def\ExplFileDate{2022-08-23}%
%<*driver>
\documentclass[full]{l3doc}

\usepackage{graphicx}
\usepackage[T1]{fontenc}
\usepackage{tabularray}
\usepackage{caption}

%%%%%%%%%%%%% 以下设置中文字体 %%%%%%%%%%%%%%%%%%%%%%%%%%%%%%%%%%%%%%%%%
\usepackage{xeCJK}  %%

\setCJKfamilyfont{Heiti}{Source Han Sans Regular} %%%% 自定义\Heiti命令,显示思源黑体,用于标题页标题的中文部分
\newcommand{\Heiti}{\CJKfamily{Heiti}} %%%% 自定义\Heiti命令,显示思源黑体,用于标题页标题的中文部分

\setCJKfamilyfont{heiti}{Source Han Sans Light} %%自定义\heiti命令,显示思源黑体,用于正文的章节标题
\newcommand{\heiti}{\CJKfamily{heiti}} %%自定义\heiti命令,显示思源黑体,用于正文的章节标题

\setCJKfamilyfont{songti}{思源宋体 CN Light}  %%%% 自定义\songti命令,显示思源宋体,用于正文
\newcommand{\songti}{\CJKfamily{songti}} %%%% 自定义\songti命令,显示思源宋体,用于正文

\setCJKmainfont{思源宋体 CN Light} %%%% 设置中文的主字体为思源宋体 CN Light
%\setmainfont{Source Serif Pro} %%%% 设置英文的主字体为Source Serif Pro,也可以设为Times New Roman

\setCJKfamilyfont{kaiti}{KaiTi} %%设置中文字体楷体,用于强调
\newcommand{\kaiti}{\CJKfamily{kaiti}} %%设置中文字体楷体,用于强调
%%%%%%%%%%%%% 以上设置中文字体 %%%%%%%%%%%%%%%%%%%%%%%%%%%%%%%%%%%%%%%%%

%%%%%%%%%%%%% 以下设置中文版式 %%%%%%%%%%%%%%%%%%%%%%%%%%%%%%%%%%%%%%%%%
\usepackage{indentfirst} %%% 首行缩进
\setlength{\parindent}{2em} %%% 缩进2个字符(中文为2个字)
\linespread{1.5} %%% 设置行间距
%%%%%%%%%%%%% 以上设置中文版式 %%%%%%%%%%%%%%%%%%%%%%%%%%%%%%%%%%%%%%%%%
\usepackage{changepage} %%%用于整体缩进,\begin{adjustwidth}{2cm}{1cm}

%%%%%%%%%%%% 以下设置书签、目录 %%%%%%%%%%%%%%%%%%%%%%%%%%%%%%%%%%%%%%%%
\usepackage{xcolor}
\usepackage[colorlinks=true,linkcolor=red]{hyperref}
%%%%%%%%%%%% 以上设置书签、目录 %%%%%%%%%%%%%%%%%%%%%%%%%%%%%%%%%%%%%%%%%


%%%%%%% 以下在 tabular 表格中定制 横线如\hlinew{1.2pt} %%%%%%
\makeatletter
\def\hlinew#1{%
\noalign{\ifnum0=`}\fi\hrule \@height #1 \futurelet
\reserved@a\@xhline}
\makeatother%
%%%%%%% 以上在 tabular 表格中定制 横线如\hlinew{1.2pt} %%%%%%

%%%%%%% 以下自定义脚注 %%%%%%%%%%%%%%%%%%%%%%%%%%%%%%%%%%%%
\setlength{\footnotesep}{0.5cm} %%%设置几第脚注之间的距离
\setlength{\skip\footins}{3em} %%%设置脚注与正文之间的距离
%%\renewcommand\footnoterule{} %%%定义脚注线为空
\renewcommand\footnoterule{
     \kern -3pt                         % This -3 is negative
     \hrule width 0.6\textwidth height 0.6pt % of the sum of this 1
     \kern 2pt} %%%
%%%%%%% 以上自定义脚注 %%%%%%%%%%%%%%%%%%%%%%%%%%%%%%%%%%%%

%%%%%%% 以下制作索引 %%%%%%%%%%%%%%%%%%%%%%%%%%%%%%%%%%%%
\usepackage {makeidx}
\makeindex
%%%%%%% 以上制作索引 %%%%%%%%%%%%%%%%%%%%%%%%%%%%%%%%%%%%

\renewcommand{\contentsname}{\centerline{\heiti 目\ \ \ 录}}   %%% 在{document}后面加入该命令,将"contents"变成“目  录”
\renewcommand{\refname}{\heiti 参考文献}
\renewcommand{\tablename}{表}
\renewcommand{\abstractname}{\heiti 摘\ 要}

\begin{document}
  \DocInput{\jobname.dtx}
\end{document}
%</driver>
% \fi
%
% \providecommand\acro[1]{\textsc{\MakeLowercase{#1}}}
% \newenvironment{arg-description}{%
%   \begin{itemize}\def\makelabel##1{\hss\llap{\bfseries##1}}}{\end{itemize}}
%
% \title{^^A
%   {\Huge \textsf{expl3}}\ {\huge \Heiti 宏包和}{\huge \textsf{\LaTeX3}}\ {\huge \Heiti 编程}^^A
% }
% \author{^^A
%  \LaTeX{}\ 项目\ \thanks{E-mail:\href{mailto:latex-team@latex-project.org}{latex-team@latex-project.org}}
%  \\[6pt] 黄旭华\ \ 译
% }
%
% \date{2022-08-23\ 发行}
%
%
% \maketitle
% \vspace{6em}
% \begin{documentation}
%
% \begin{abstract}
%
% 本文档介绍了一组新的编程约定(programming conventions),这些约定旨在满足实施大规
% 模 \TeX{}\ 宏编程项目(如\LaTeX{})的要求。这些编程约定是 \LaTeX3\ 的底层。该系统
% 的主要特点如下:
% \begin{itemize}
%   \item  宏(macros)(或在 \LaTeX{}\ 术语中的命令)分为 \LaTeX{}\ 函数(functions)
%   和 \LaTeX{}\ 参数(parameters),以及包含相关命令的模块(modules);
%   \item  基于这些分类的系统命名方案(systematic naming scheme);
%   \item  控制函数参数展开(expansion)的简单机制。
% \end{itemize}
% 这个系统正在被用作 \LaTeX{}\ 项目中 \TeX{}\ 编程的基础。请注意,该语言(language)
% 既不适用于文档标记(document mark-up),也不适用于样式规范(style specification)。
% 相反,这些特性(features)将构建在这里描述的约定之上。
%
% \hspace{2em}文档介绍了 \pkg{expl3}\ 编程接口(programming interface)背后的思想。有关 \LaTeX{}\
% 项目提供的编程层(programming layer)的完整文档,请参阅附带的 \texttt{interface3}\ 文档。
%
% \end{abstract}
\newpage
%   \tableofcontents
\newpage
% \section{\heiti 介绍}
%
%开发一个 \LaTeXe{}\ 以外的 \LaTeX{}\ 内核的第一步是解决底层系统(underlying system)如何编程的问题。
%与目前混合使用 \LaTeX{}\ 和 \TeX{}\ 宏不同的是,\LaTeX3\ 系统提供了自己的一致接口(consistent interface),
%可以控制 \TeX{}\ 所需的所有函数(functions)。这项工作的一个关键部分是确保所有内容都是文档化的(documented),
%这样 \LaTeX{}\ 程序员和用户就可以高效地工作,而无需熟悉内核(kernel)的内部特性或 plain \TeX{}。
%
% \pkg{expl3}\ 宏包为 \LaTeX{}\ 提供了这个新的编程接口(programming interface)。
% 为了使编程系统化,\LaTeX3\ 使用了一些与 \LaTeXe{}\ 或 plain \TeX{}\ 完全不同的约定。
% 因此,以 \LaTeX3\ 开始的程序员需要熟悉新语言的语法。
%
% 下一节将介绍这种语言在基于 \TeX{}\ 的完整文档处理系统(document processing system)中
% 的应用。然后,我们描述了命令名(command names)的语法结构的主要特征,包括函数名(function names)
% 中使用的参数规范语法(argument specification syntax)。
%
% 本文将解释这种参数语法(argument syntax)背后的实际思想(practical ideas),以及展开
% 的控制机制(expansion control mechanism)和用于定义函数变体形式(variant forms of functions)的接口。
%
% 正如我们将要演示的那样,使用结构化命名方案(structured naming scheme)和函数变体形式
% (variant forms for functions)极大地提高了代码的可读性(readability),因此也提高了其可靠性(reliability)。
% 此外,经验表明,新语法产生的较长的命令名(longer command names)不会使{\kaiti 编写}(\emph{writing})
% 代码的过程变得非常困难。
%
% \newpage
% \section{\heiti 语言和接口}
%
% 可以识别与基于 \TeX{}\ 的文档处理系统(document processing system)中所需的各种接
% 口相关的几种不同语言(languages)。本节介绍我们认为对 \LaTeX3\ 系统最重要的功能。
%
% \begin{description}
%   \item[{\heiti 文档标记}(document mark-up)] 这包括那些要嵌入到文档(|.tex|文件)中的命令(通常称为标记[tags])。
%
%     人们普遍认为,这种标记(mark-up)本质上应该是{\kaiti 声明性的}(\emph{declarative})。
%     它可以是传统的基于 \TeX{}\ 的标记,如~\cite{A-W:LLa94}\ 和~\cite{A-W:GMS94}\ 中
%     所述的 \LaTeXe{},或者是通过 \acro{HTML}\ 或 \acro{XML}\ 定义的标记语言(mark-up language)。
%
%     更传统的 \TeX{}\ 编码约定(coding conventions)(如~\cite{A-W:K-TB}\ 中所述)的
%     一个问题是,\TeX{}\ 原语格式命令(primitive formatting commands)的名称和语法被巧妙地设计为当作者直接用作文
%     档标记或宏时是很“自然的(natural)”。具有讽刺意味的是,逻辑标记(logical mark-up)的普遍性
%     (以及公认的优越性)意味着在文档或作者定义的宏中几乎不需要这种显式格式化命令
%     (explicit formatting commands)。因此,它们几乎只被 \TeX{}\ 程序员用来定义更高级别的命令,
%     而且它们的独特的语法在这个社区中一点也不受欢迎。此外,如果它们不是作为原语
%     ({\kaiti 例如}~\tn{box}\ 或 \tn{special})被抢占的话,它们中的许多名称可以作为文档标记是非常有用的。
%
%   \item[{\heiti 设计师接口}(designer interface)] 这将(人类)排版设计师对文档的规范(specification)和
%     “格式化文档”的程序(program)联系起来。理想情况下,它应该使用声明性语言
%     (declarative language),这种语言可以方便地表达为各种文档元素(document elements)的布
%     局(layout)指定的关系(relationship)和间距(spacing)规则。
%
%     这种语言不嵌入在文档文本(document text)中,它在形式上与文档标记语言(document mark-up language)
%     有很大不同。对于 \LaTeX{}2.09\ 来说,这个级别几乎完全不存在,\LaTeXe{}\ 在这个领域做了
%     一些改进,但是仍然是这种情况,在 \LaTeX{}\ 中实现一个设计规范(design specification)需
%     要比可接受的更多的“底层(low-level)”编码。
%
%   \item[{\heiti 程序员接口}(programmer interface)]
%     该语言是在 \TeX{}(或后续程序)原语(primitives)的基础上实现基本排版功能的实现语言
%     (implementation language)。它还可以用于在 \TeX{}(如当前的 \LaTeX{}\ 系统)中实现
%     两种语言。
%
% \end{description}
%
% 最后一层(last layer)由本文档中描述的约定(conventions)所涵盖,本文档描述了一种旨在为 \LaTeX3\ 编
% 码提供适当基础的系统。其主要特性概述如下:
% \begin{itemize}
%   \item 所有命令的一致命名方案(naming scheme),包括 \TeX{}\ 原语。
%   \item 将命令分类为 \LaTeX{}\ 函数(functions)或 \LaTeX{}\ 参数(parameters),并根据其
%   功能(functionality)将其划分为模块(modules)。
%   \item 一种控制参数展开(controlling argument expansion)的简单机制。
%   \item 提供一组核心 \LaTeX{}\ 函数,足以处理队列(queues)、集合(stacks)、堆栈(stacks)和
%   属性列表(property lists)等编程结构(programming constructs)。
%   \item 一种 \TeX{}\ 编程环境(programming environment),例如,在该环境中忽略所有空格(space)。
% \end{itemize}
%
% \newpage
% \section{\heiti 命名方案}
%
% \LaTeX3\ 不使用 |@| 作为定义内部宏(internal macros)的“字母(letter)”。相反,
% 在内部宏名称中使用符号 |_| 和 |:| 来提供结构(structure)。与 plain \TeX{}\ 格式和 \LaTeXe{}\ 内
% 核不同,这些额外的字母(extra letters)仅用于宏名称(macro name)的部分(parts)之间(没有奇怪的元音替换[vowel replacement])。
%
% 虽然 \TeX{}\ 实际上是一个宏处理器(macro processor),但是通过对 \pkg{expl3}\ 编程语言的约定,
% 我们区分了{\kaiti 函数}(\emph{functions})和{\kaiti 变量}(\emph{variables})。函数可以有参数,它们可以被展开或被执行。
% 变量(variables)可以被赋值(assigned values),并在函数的参数中使用;变量不是直接使用的,而是
% 由函数(包括获取[getting]和设置函数[setting functions])操纵的。具有相关功能的函数和变量
% (例如,访问计数器或操纵令牌列表[token lists]等)一起收集到一个{\kaiti 模块}(\emph{module})中。
%
% \subsection[示例]{\heiti 示例}
%
% 在给出命名方案(naming scheme)的细节之前,这里有几个典型的例子来说明方案的特点:首先是一些变
% 量名(variable names)。
% \begin{quote}
%   \cs{\textbf{l_tmpa_box}}\ \ 是对应于 box 寄存器(box register)的局部变量(local variable)(因为~|l_|前缀)。\\
%   \cs{\textbf{g_tmpa_int}}\ \ 是对应于整数寄存器(integer register)(如一个 \TeX{}\ 计数寄存器)的全
%   局变量(global variable)(因为~|g_|前缀)。\\
%   \cs{\textbf{c_empty_tl}}\ \ 是始终为空(empty)的常量~(|c_|)令牌列表变量(token list variable)。
% \end{quote}
%
% 下面是一个典型函数名(typical function name)的示例:
%
% \cs{seq_push:Nn} 是一个函数(function),它将第二个参数所指定的令牌列表(token list)放在第
% 一个参数指定的堆栈(stack)上。两个参数的不同性质(natures)由~|:Nn| 后缀表示。第一个参数必须是
% “names”堆栈参数的单个令牌(single token):这样的单令牌参数
% (single-token arguments)表示为~|N|。第二个参数是一个普通的 \TeX{}“未限制参数(undelimited argument)”,
% 它可以是单令牌的或对称的(balanced)、用大括号分隔(brace-delimited)的令牌列表(这里我们将其
% 称为{\kaiti 大括号令牌列表}\textit{[braced token list]}):~|n| 表示这种“普通(normal)”
% 参数形式。函数名表明它属于 |seq| 模块(module)。
%
% \subsection[正式的命名语法]{\heiti 正式的命名语法}
%
% 现在我们将更详细地研究这些名称的语法。\LaTeX3\ 中的函数名(function name)由三部分组成:
%
% \begin{quote}
%   \color{red}{|\|\meta{module}|_|\meta{description}|:|\meta{arg-spec}}
% \end{quote}
% 而变量的名称有(多达)四个不同的部分:
% \begin{quote}
%  \color{blue}{|\|\meta{scope}|_|\meta{module}|_|\meta{description}|_|\meta{type}}
% \end{quote}
%
% 所有名称(names)的语法包含:
% \begin{quote}
%   \meta{module} 和 \meta{description}
% \end{quote}
% 它们都提供了关于命令的信息。
%
% \emph{module},即{\kaiti 模块},是密切相关的函数(functions)和变量(variables)的集合。典型的模块名(module names)
% 包括整数参数(integer parameters)和相关函数的~|int|、序列(sequences)的~|seq| 和盒子(box)的~|box|。
%
% 提供新编程功能的宏包将根据需要添加新模块;程序员可以为模块选择任何未使用的名称,仅由字母组成。
% 一般来说,模块名(module name)和模块前缀(module prefix)应该是相关的:例如,包含 \texttt{box}\ 函数
% 的内核模块(kernel module)称为 \texttt{l3box}。\pkg{l3prefixes.csv}\ 中列出了模块名称和程序员的联系方式。
%
% \emph{description},即{\kaiti 描述},提供了有关函数(function)或参数(parameter)的更详细信息,并为其
% 提供了唯一名称(unique name)。它应该由字母(letters)和~|_|~字符(characters)组成。一般来说,
% \emph{description}\ 应使用 |_| 分隔“单词(words)”或名称中其他易于跟踪(follow)的部分。
% 例如,\LaTeX3\ 内核提供了 \cs{if_cs_exist:N},正如预期的那样,它测试命令名(command name)是否存在。
%
% 当用于变量操作(variable manipulation)的函数可以局部或全局执行赋值时,后一种情况通过在函数名(function name)
% 的第二部分包含 |g| 来表示。因此 \cs{tl_set:Nn}\ 是一个局部函数(local function),但 \cs{tl_gset:Nn}\ 起全局作用。
% 这种类型的函数总是记录在一起(documented together),因此可以从 |g| 的存在与否在来推断作用范围(scope of action)。
% 有关变量范围(variable scope)的更多详细信息,请参见下一小节。
%
% \subsubsection[区分私有材料和公共材料]{\heiti 区分私有材料和公共材料}
%
% \TeX{}\ 语言的一个问题是,除了约定(convention)之外,它不支持名称空格(name spaces)和封装(encapsulation)。
% 因此,\LaTeXe{}\ 内核中的几乎所有内部命令最终都被扩展包(extension packages)用作修改(modifications)或
% 扩展(extensions)的入口点(entry point)。其结果是,现在几乎不可能在不破坏某些东西的情况下更改 \LaTeXe{}\ 内核
% 中的任何内容(即使它显然只是一个内部命令)。
%
% 在 \pkg{expl3}\ 中,我们希望通过明确区分公共接口(public interfaces)(扩展包可以使用和依赖)
% 和私有(private)函数和变量(不应出现在其模块之外),来大幅改善上述情况。在没有严重的计算开销(computing overhead)
% 的情况下,(几乎)无法实现这一点,因此我们只能通过命名约定(naming convention)和一些支持机制(support mechanisms)来实现。
% 然而,我们认为这种命名约定很容易理解(understand)和遵循(follow),因此我们相信这将被采用并能提供预期的结果。
%
% 由模块(module)创建的函数可以是“公共(public)”(用已定义的接口记录)或“私有(private)”
% (仅在该模块中使用,因此未正式记录)。重要的是仅使用记录的接口(documented interfaces);同时,
% 有必要在函数或变量的名称中显示它是公共的(public)还是私有的(private)。
%
% 为了明确区分这两种情况,使用以下约定(convention)。应在模块名(module name)的开头添加 |__| 来
% 定义私有函数(private functions)。因此
% \begin{verbatim}
%   \module_foo:nnn
% \end{verbatim}
% 是一个公共函数(public function)应该被记录,而
% \begin{verbatim}
%   \__module_foo:nnn
% \end{verbatim}
% 是该模块私有的,{\kaiti 不}应在该模块之外使用。
%
% 同样,私有变量(private variables)应该在模块名的开头使用"__" ,这样
% \begin{verbatim}
%   \l_module_foo_tl
% \end{verbatim}
% 是公共变量(public variable),而
% \begin{verbatim}
%   \l__module_foo_tl
% \end{verbatim}
% 是私有变量。
%
% \subsubsection[使用 \texttt{@@}\ 和 \pkg{l3docstrip}\ 标记私有代码]{{\heiti 使用} \texttt{@@}\ {\heiti 和} \pkg{l3docstrip}\ {\heiti 标记私有代码}}
%
% 内部函数(internal functions)的正式语法(formal syntax)允许明确区分公共(public)和私有(private)代码,
% 但包含冗余信息(redundant information)(每个内部函数或变量包含 \texttt{__\meta{module}})。
% 为了帮助程序员,\pkg{l3docstrip}\ 程序引入了以下语法
% \begin{quote}
%   \ttfamily
%   |%<@@=|\meta{module}|>|
% \end{quote}
% 然后,在代码中它允许 |@@| (和 |_@@| 在变量情况下)用作 \texttt{__\meta{module}}\ 的占位符(place holder)。举个例子
% \begin{verbatim}
%   %<@@=foo>
%   %    \begin{macrocode}
%   \cs_new:Npn \@@_function:n #1
%     ...
%   \tl_new:N \l_@@_my_tl
%   %    \end{macrocode}
% \end{verbatim}
% 提取时由 \pkg{l3docstrip}\ 转换为
% \begin{verbatim}
%   \cs_new:Npn \__foo_function:n #1
%     ...
%   \tl_new:N \l__foo_my_tl
% \end{verbatim}
% 正如您可以看到的, |_@@| 和 |@@| 都映射到 \texttt{__\meta{module}},因为我们认为这有助于
% 在使用 |@@| 约定时区分源代码(source)中的变量(variables)和函数(functions)。
%
% 请注意,您必须使用 \pkg{l3docstrip},而\textbf{不是} \texttt{.ins}\ 文件中的 \pkg{docstrip}\ 程序
% 才能使此工作正常 --- 原始 \LaTeXe{}\ \pkg{docstrip}\ 不理解 \texttt{@@},只会将其复制到您的代码中而不做任何修改!
%
%
% \subsubsection[变量:声明]{\heiti 变量:声明}
%
% 在格式良好的 \pkg{expl3}\ 代码中,应该始终在尝试赋值之前声明变量。即使对于基础 \TeX{}\ 实现(underlying \TeX{}\ implementation)
% 允许直接赋值(assignment)的变量类型(variable types)也是如此。这既适用于直接设置(setting directly)(\cs{tl_set:Nn}\ 等),
% 也适用于设置相等(setting equal)(\cs{tl_set_eq:NN}\ 等)。
%
% 为了帮助程序员坚持这种方法,可能会给出调试选项 |check-declarations|。
% \begin{verbatim}
%   \debug_on:n { check-declarations }
% \end{verbatim}
% 并且每当对未声明变量进行赋值时,将发出错误。这会影响性能,因此此选项只能用于测试。
%
% \subsubsection[变量:作用域和类型]{\heiti 变量:作用域和类型}
%
% 这个 \meta{scope}\ 即{\kaiti 作用域}的名称的一部分描述了如何访问(accessed)变量。变量分为局部(local)变量、
% 全局(global)变量或常量(constant)变量。此 \emph{scope} 类型(type)在名称开头显示为代码,使用的代码为:
% \begin{arg-description}
%   \item[c] 常量(constants)(值不应更改的全局变量);
%   \item[g] 其值只能全局设置的变量;
%   \item[l] 其值只能在局部设置的变量。
% \end{arg-description}
%
% 提供单独的函数(functions)来将数据赋值(assign)给局部和全局变量,例如,\cs{tl_set:Nn}\ 和 \cs{tl_gset:Nn}\ 分
% 别设置了局部或全局“令牌列表(token list)”变量的值。请注意,混合变量的局部和全局赋值是一种糟糕
% 的 \TeX{}\ 实践,这可能会耗尽存储堆栈(save stack)。\footnote{更详细的信息请参考 \emph{The \TeX{}book}, p.\,301}
%
% 这个 \meta{type}\ 即{\kaiti 类型}在可用 \emph{data-types}\ 即{\kaiti 数据类型}\ \footnote{当然,如果需
% 要一种全新的数据类型(data type),则情况并非如此。但是,希望只有内核团队(kernel team)才能创
% 建新的数据类型。}\ 列表中;这些包括原始 \TeX{}\ 数据类型,如各种寄存器(registers),但在 \LaTeX{}\ 编
% 程系统中添加了数据类型。
%
% \LaTeX3\ 的数据类型有:
% \begin{description}
%   \item[bool]   true\ 或 false (\LaTeX3\ 的实现不使用 \tn{iftrue}\ 或 \tn{iffalse});
%   \item[box]    盒子寄存器(box register);
%   \item[cctab]  类别代码表(category code table);
%   \item[clist]  逗号分隔列表(comma separated list);
%   \item[coffin] “带手柄的盒子(box with handles)” --- 进行 |box| (盒子)对齐操作(alignment operations)的较高级别的数据类型;
%   \item[dim]    “刚性(rigid)”长度;
%   \item[fp]     浮点值(floating-point values);
%   \item[ior]    输入流(input stream)(用于从文件读取);
%   \item[iow]    输出流(output stream)(用于写入文件);
%   \item[int]    整数值计数寄存器(integer-valued count register);
%   \item[muskip] “橡皮(rubber)”长度的数学模型(math mode);
%   \item[prop]   属性列表(property list);
%   \item[seq]    序列(sequence):用于实现列表(两端都有访问权限)和堆栈的数据类型;
%   \item[skip]   “橡皮(rubber)”长度;
%   \item[str]    \TeX{}\ 字符串:|tl| 的一种特殊情况,其中所有字符都有类别(category)“other”(类别~$12$),但类别 “space”(类别~$10$)的空格(spaces)除外;
%   \item[tl]     “令牌列表变量(token list variables)”:令牌列表的占位符。
% \end{description}
% 当 \meta{type}\ 和 \meta{module}\ 相同时(在更基本的模块中经常发生),出于美观,
% \meta{module}\ 部分经常被省略。
%
% “token list (令牌列表)”这个名称可能会引起混淆,因此一些背景知识是有用的。\TeX{}\ 处
% 理令牌(tokens)和令牌列表,而不是字符(characters)。它提供了两种存储这些令牌列表的方法:在宏(macros)
% 中和作为令牌寄存器(token registers)(|toks|)。\LaTeX3\ 中的实现(implementation)意味着不需要 |toks|,
% 并且存储令牌(storing tokens)的所有操作都可以使用 |tl| 变量类型。
%
% 有经验的 \TeX{}\ 程序员会注意到,列出的一些变量类型(variable types)是原生的(native) \TeX{}\ 寄存器,
% 而其他的则不是。一般来说,数据结构(data structure)的底层 \TeX{}\ 实现(implementation)可能有所不同,
% 但 \emph{documented interface}\ 即{\kaiti 记录的接口}是稳定的(stable)。例如,|prop| 数据类型最初实现为 |toks|,
% 但目前构建在 |tl| 数据结构之上。
%
% \subsubsection[变量:指导]{\heiti 变量:指导}
%
% 逗号列表(comma lists)和序列(sequences)都具有相似的特征。它们都使用特殊的分隔符(special delimiters)
% 将一个条目(entry)从下一个条目标记出来(mark out),并且两端都可以访问。一般来说,“手工(by hand)”创建逗号列表更容易,
% 因为它们可以直接键入。用户输入(user input)通常采用逗号分隔的列表形式,因此在许多情况下,这是可以使用的显而易见的数据类型。
% 另一方面,序列(sequences)使用特殊的内部令牌(internal tokens)来分隔条目。这意味着
% 它们可以用于包含逗号列表无法包含的材料(例如,项目[items]本身可能包含逗号!)。一般来说,在程序内部
% 创建固定列表(fixed lists)以及在不出现逗号的情况下处理用户输入时,逗号列表应该是首选的。另一方面,
% 序列(sequences)应该用来存储任意的数据列表。
%
% \pkg{expl3}\ 使用序列数据结构(sequence data structure)实现堆栈(stacks)。因此,创建堆栈
% 首先需要创建一个序列,然后使用以堆栈方式(stack manner)工作的序列函数(sequence functions)
% (\cs{seq_push:Nn},{\kaiti 等等})。
%
% 由于底层(underlying) \TeX{}\ 实现的性质,可以在不首先声明的情况下为令牌列表(token list)变量
% 和逗号列表(comma lists)赋值。但是,这是{\kaiti 不受支持的行为}。\LaTeX3\ 编码约定(coding convention)
% 是所有变量必须在使用前声明。
%
% \pkg{expl3}\ 包可以加载 \texttt{check-declarations}\ 选项,以验证在使用之前是否声明了所有变量。
% 这对性能有影响,因此用于开发期间的测试,而不是用于生成文档(production documents)。
%
% \subsubsection[函数:参数规范]{\heiti 函数:参数规范}
%
% 函数名在冒号(colon)后面以 \meta{arg-spec}\ 结尾。这表明了函数采用的参数类型,并提供了一种
% 方便的方法来命名仅在参数形式上不同的类似函数(参见下一节的示例)。
%
% 这个 \meta{arg-spec}\ 由字母(可能为空)列表组成,每个字母表示函数的一个参数。字母(letter),包括其大小写(case),
% 传达了所需参数类型的信息。
%
% 所有函数都有一个带参数的基形式(base form),使用一个下面的参数规范:
% \begin{arg-description}
%   \item[n]  未展开的令牌或大括号令牌列表(braced token list)。\\
%     这是一个标准的 \TeX{}\ 未限制(undelimited)宏参数(macro argument)。
%   \item[N]  单个令牌(与~|n| 不同,参数{\kaiti 不能}被大括号包围)。\\
%     采用~|N| 参数的命令的典型示例是~|\cs_set|,其中定义的命令必须没有括号(unbraced)。
%   \item[p]  原始 \TeX{}\ 参数规范(parameter specification)。\\
%     这可以很简单,比如~|#1#2#3|,但可以使用任意分隔的参数语法,比如:|#1,#2\q_stop#3|。这在定义函数时使用。
%   \item[T,F]
%     这些是~|n| 个参数的特殊情况,用于条件命令中的 true 和 false 代码。
% \end{arg-description}
% 还有另外两个更具普遍含义的规范(specifiers):
% \begin{arg-description}
%   \item[D] 代表\textbf{不要使用(do not use)}。这个特例用于 \TeX{}\ 原语(primitives)。
%   这些函数没有标准的语法(standardized syntax),它们依赖于引擎(engine),而且它们的名称可以在没有警告
%   的情况下更改,因此在宏包代码中{\kaiti 强烈建议不要使用}它们:程序员应该使
%   用 \href{interface3.pdf}{interface3.pdf}\ \footnote{如果原语(primitive)提供了内核中尚未提供的功能,
%   则鼓励程序员和用户向 \texttt{LaTeX-L}\ 邮件列表(\url{mailto:LATEX-L@listserv.uni-heidelberg.de})描
%   述它们的用例(use-case)和预期行为(intended behaviour),以便讨论可能的接口。目前,尽管未提供接口,
%   程序员可以使用 \href{l3styleguide.pdf}{l3styleguide.pdf} 中描述的过程(procedure)。}\ 中记录的接口。
%   \item[w] 这意味着参数语法(argument syntax)是“怪异的(weird)”,因为它不遵循任何
%   标准规则(standard rule)。它用于具有非标准形式参数的函数:例如 \TeX{}\ 级别的分隔参数(delimited arguments),
%   以及在某些原始 |\if|\ldots{}\ 命令之后所需的布尔测试(boolean tests)。
% \end{arg-description}
%
% 如果 |n| 参数由单个令牌(single token)组成,则几乎在所有情况下都可以省略周围的大括号---明确提
% 到了强制使用大括号的函数,即使对于单个令牌参数(token arguments)也是如此。然而,鼓励程序员总是
% 在 \texttt{n}\ 参数周围使用大括号,因为这使函数和参数之间的关系更清晰。
%
% 作为展开控制系统(expansion control system)的一部分,可以使用其他参数规范(argument specifiers)。
% 这些将在下一节中讨论。
%
% \newpage
% \section{\heiti 展开控制}
%
% 让我们看一下您可能希望执行的一些典型操作(typical operations)。假设我们维护一个打开文件的堆栈(stack),
% 并使用堆栈 |\g_ior_file_name_seq| 来跟踪它们(\texttt{ior}\ 是用于文件读取模块的前缀)。
% 这里的基本操作(basic operation)是将名称推送到这个堆栈上,这个堆栈可以由下面的操作(operation)完成:
% \begin{quote}
%   \cs{seq_gpush:Nn} |\g_ior_file_name_seq {#1}|
% \end{quote}
% 其中 |#1| 是文件名(filename)。换句话说,此操作将文件名按原样(as is)推送到堆栈上。
%
% 但是,我们可能会遇到文件名存储在某种变量中的情况,例如 |\l_ior_curr_file_tl|。在本例中,我们
% 希望检索(retrieve)变量的值。如果我们简单地使用
% \begin{quote}
%   \cs{seq_gpush:Nn} |\g_ior_file_name_seq| |\l_ior_curr_file_tl|
% \end{quote}
% 不获取推送到堆栈上的变量的值,只获取变量名本身。相反,需要适当数量的 \cs{exp_after:wN}
% (还有 额外的大括号)来改变展开顺序(order of expansion)\ \footnote{\cs{exp_after:wN}\ 是
% \TeX{} \tn{expandafter}\ 原语的 \LaTeX3\ 名称。}。例如:
% \begin{quote}
%   \cs{exp_after:wN}                              \\
%   |   |\cs{seq_gpush:Nn}                         \\
%   \cs{exp_after:wN}                              \\
%   |   \g_ior_file_name_seq|                      \\
%   \cs{exp_after:wN}                              \\
%   |   { \l_ior_curr_file_tl }|
% \end{quote}
%
% 上面的示例可能是最简单的情况,但是已经显示了代码如何更改为难以理解的内容。此外,
% 这里还有一个假设:存储容器(storage bin)在完成一次展开后会显示其内容。这意味着
% 您不能进行正确的检查(proper checking),而且您必须确切了解存储容器的工作方式,
% 以便获得正确的展开数(number of expansions)。因此,\LaTeX3\ 为程序员提供了一个通
% 用的方案(general scheme),使代码保持紧凑并易于理解。
%
% 为了表示函数的某些参数需要特殊处理,只需在函数的 arg-spec 部分使用不同的字母来
% 标记想要的行为(desired behavior)。在上面的例子中,人们会编写
% \begin{quote}
%   \cs{seq_gpush:NV} |\g_ior_file_name_seq \l_ior_curr_file_tl|
% \end{quote}
% 以达到预期效果。这里的 |V| (第二个参数)表示“检索变量的值”,然后将其传递给基函数(base function)。
%
% 以下字母可用于表示在将参数传递给基函数之前对参数的特殊处理:
% \begin{description}
%   \item[c] 用作命令名的字符串(character string)。\\ 参数(令牌或括号令牌列表){\kaiti 被完全展开};
%   结果必须是一个字符序列(sequence of characters),然后该字符序列用于构造命令
%   名({\kaiti 通过}~\tn{csname} \ldots \tn{endcsname})。
%   此命令名是作为参数传递给函数的单个令牌(single token)。因此
%     \begin{quote}
%       \cs{seq_gpush:cV} |{ g_file_name_seq }| \cs{l_tmpa_tl}
%     \end{quote}
%     相当于
%     \begin{quote}
%       \cs{seq_gpush:NV} |\g_file_name_seq| \cs{l_tmpa_tl}.
%     \end{quote}
%     完全展开(full expansion)意味着:(a)全部参数必须可展开,(b)任何变量都转换为其内容。
%     所以前面的例子也相当于
%     \begin{quote}
%       \cs{tl_new:N} |\g_file_seq_name_tl| \\
%       \cs{tl_gset:Nn} |\g_file_seq_name_tl { g_file_name_seq }| \\
%       \cs{seq_gpush:cV} |{| \cs{tl_use:N} |\g_file_seq_name_tl }| \cs{l_tmpa_tl}.
%     \end{quote}
%     令牌列表变量(token list variables)是可展开的,我们可以省略存取器函数(accessor function) \cs{tl_use:N}。
%     其他变量类型需要适当的 \cs{\meta{var}_use:N}\ 函数以在此上下文(context)中使用。
%   \item[V]  变量(variable)的值。\\
%     这意味着所讨论的寄存器(register)的内容被用作参数,可以是整数(integer)、长度
%     类型寄存器(length-type register)、令牌列表变量(token list variable)或类似变量。
%     该值作为大括号令牌列表(braced token list)传递给函数。可以应用于具
%     有 \cs{\meta{var}_use:N}\ 函数(浮点[floating points]和盒子[boxes]除外)的变量,
%     因此只传递(deliver)单个“值”。
%   \item[v] 寄存器(register)的值,由用作命令名的字符串构造。\\
%     这是 |c| 和 |V| 的组合,它首先从参数构造(constructs)控制序列(control sequence),
%     然后将结果寄存器(resulting register)的值传递给函数。可以应用于具
%     有 \cs{\meta{var}_use:N}\ 函数(浮点[floating points]和盒子[boxes]除外)的变量,
%     因此只传递(deliver)单个“值”。
%   \item[x]  完全展开的令牌或大括号令牌列表(braced token list)。\\
%     这意味着参数将像~\tn{edef}\ 的替换文本(replacement text)一样被展开,被展开后作
%     为大括号的令牌列表(braced token list)传递给函数。展开将一直进行,直到只剩下不可展开的令牌。
%     |x| 类型参数不能被嵌套(nested)。
%   \item[e]  完全扩展的令牌或大括号令牌列表(braced token list),不需要双重 |#| 令牌。
%   此扩展非常类似于 |x| 类型,但可能是嵌套的,不需要双重 |#| 令牌。
%   \item[f] 在大括号令牌列表(braced token list)中递归(recursively)展开第一个令牌。\\
%     与 |x| 类型几乎相同,但这里的令牌列表被完全展开,直到找到第一个不可展开的令牌,其余的保持不变。
%     请注意,如果此函数在参数的开头找到一个空格(space),它会将其吞并,而不会展开下一个令牌。
%   \item[o]  一级扩展令牌(One-level-expanded token)或大括号令牌列表(braced token list)。\\
%     这意味着将参数展开一级,就象 \tn{expandafter}\ 一样,并将展开为大括号令牌列
%     表(braced token list)传递给函数。请注意,如果原始参数(original argument)是一个大括号
%     令牌列表,则只会展开该列表中的第一个令牌。一般来说,对于简单的变量检索(simple variable retrieval),
%     应首选使用 \texttt{V}\ 而不是使用 \texttt{o}。
% \end{description}
%
% \subsection[简单意味着更好]{\heiti 简单意味着更好}
%
% 任何用 \TeX{}\ 编程的人都非常熟悉这样一个问题,即在调用函数之前,函数的参数要适当地扩展。
% 为了说明展开控制(expansion control)如何立即缓解这个问题,我们将考虑
% 从 \texttt{latex.ltx}\ 复制的两个示例。
%
% \begin{verbatim}
%        \global\expandafter\let
%              \csname\cf@encoding \string#1\expandafter\endcsname
%              \csname ?\string#1\endcsname
% \end{verbatim}
% 第一段代码本质上只是一个全局 \tn{let},在执行 \tn{let}\ 之前,必须首先构造它的两个参数。
% |#1| 是一个控制序列名称(control sequence name),如 |\textcurrency|。要定义的令牌是通过
% 连接存储在 |\cf@encoding| (必须完全展开)中的当前字体编码的(current font encoding)字符
% 和符号名称来获得的。第二个令牌相同,只是它使用默认编码 |?|。结果是所有 \TeX{}\ 程序员都喜爱
% 的 \tn{expandafter}\ 和 \tn{csname}\ 交织在一起,代码基本上是不可读的(unreadable)。
%
% 使用此处概述的约定(conventions)和功能(functionality),可以通过以下代码来完成任务:
% \begin{verbatim}
%   \cs_gset_eq:cc
%     { \cf@encoding \token_to_str:N  #1 } { ? \token_to_str:N #1 }
% \end{verbatim}
% 命令 \cs{cs_gset_eq:cc}\ 是一个全局~\tn{let},该全局~\tn{let}\ 在定义之前从两个参数中
% 生成命令名。这样生成的代码更具可读性,而且第一次更有可能是正确的。(\cs{token_to_str:N}\
% 是 \tn{string}\ 的 \LaTeX3\ 名称。)
%
% 这是第二个例子:
% \begin{verbatim}
%   \expandafter
%     \in@
%   \csname sym#3%
%     \expandafter
%       \endcsname
%     \expandafter
%       {%
%     \group@list}%
% \end{verbatim}
% 这段代码是另一个函数定义的一部分。它首先生成两个东西:一个令牌列表,通过展开 |\group@list| 一次
% 以及一个其名称来自~“|sym#3|”的令牌。然后调用~\cs{in@}\ 函数,并测试其第一个参数是否出现在第二个
% 参数的令牌列表中。
%
% 同样,我们可以极大地改进代码。首先,我们将重命名函数~\cs{in@},根据我们的约定,它测试它的第一个
% 参数是否出现在第二个参数中。这样的函数接受两个普通的(normal)“\enquote{\texttt{n}}”参数,
% 并对令牌列表进行操作:它可能被合理地命名为 |\tl_test_in:nn|。因此,我们需要的变量函
% 数(variant function)将用适当的参数类型(argument types)定义,其名称为 |\tl_test_in:cV|。
% 现在,这个代码片段很简单:
% \begin{verbatim}
%   \tl_test_in:cV { sym #3 } \group@list
% \end{verbatim}
% 通过使用 |\l_group_seq| 序列而不是 |\group@list| 裸令牌列表(bare token list),可以进一步
% 改进此代码。请注意,除了缺少 \tn{expandafter}\ 之外,|}| 后面的空格(space)也会自动忽略,
% 因为在此编程环境中会忽略所有空格。
%
% \subsection[旧功能中的新功能]{\heiti 旧功能中的新功能}
%
% 对于许多常见函数(common functions),\LaTeX3\ 内核提供了具有一系列参数形式的变体(variants),
% 类似地,提供新函数的扩展包(extension packages)将使它们以所有常见需要的形式(forms)可用。
%
% 然而,在某些情况下,有必要构建新的此类变体形式;因此,扩展模块(expansion module)提供了一种
% 直接的机制(straightforward mechanism)来创建具有任何所需参数类型的函数,从一个带
% 有“普通(normal)”\TeX{}\ 未分隔参数(undelimited arguments)的函数开始。
%
% 为了说明这一点,让我们假设您有一个“基函数(base function)”|\demo_cmd:Nnn|,它接受三个普通参数,
% 并且您需要构造变量 |\demo_cmd:cnx|,该变量的第一个参数用于构造命令的{\kaiti 名称},而第三个参数
% 必须在传递给 |\demo_cmd:Nnn| 之前被完全展开。要从基本形式生成变体形式(variant form),只需使用以下命令:
% \begin{verbatim}
%   \cs_generate_variant:Nn \demo_cmd:Nnn { cnx }
% \end{verbatim}
% 它定义了变体形式,这样你就可以编写,例如:
% \begin{verbatim}
%   \demo_cmd:cnx { abc } { pq } { \rst \xyz }
% \end{verbatim}
% 而不是 \ldots\ ,嗯,是这样的!
% \begin{verbatim}
%   \def \tempa {{pq}}%
%   \edef \tempb {\rst \xyz}%
%   \expandafter
%     \demo@cmd:nnn
%   \csname abc%
%     \expandafter
%       \expandafter
%     \expandafter
%         \endcsname
%     \expandafter
%       \tempa
%     \expandafter
%       {%
%     \tempb
%       }%
% \end{verbatim}
%
% 另一个示例:您可能希望声明一个函数 |\demo_cmd_b:xcxcx|,它是现有函数 |\demo_cmd_b:nnnnn|的一个变体,
% 它完全展开了参数 1、3 和 5,并使用~\tn{csname}\ 生成要作为参数 2 和 4 传递的命令。
% 您需要的定义很简单
% \begin{verbatim}
%   \cs_generate_variant:Nn \demo_cmd_b:nnnnn { xcxcx }
% \end{verbatim}
%
% 编写这种展开机制(extension mechanism)的目的是,如果某些现有命令的相同新形式由
% 两个扩展包(extension packages)实现,那么这两个定义是相同的,因此不会发生冲突。
%
% \newpage
% \section{\heiti 分发}
%
% \pkg{expl3}\ 模块(modules)被设计为在 \LaTeXe{}\ 上加载。
%
%   {\kaiti 核心 \pkg{expl3}\ 语言非常稳定,因此提供的语法约定(\emph{syntax conventions})和函数
%   现在可以广泛使用。某些函数可能仍有更改,但与 \pkg{expl3}\ 的范围(\emph{scope})相比,
%   这些更改将是微不足道的。对于此类弃用(\emph{deprecations}),有一个健壮的机制(\emph{robust mechanism})。}
%
% \pkg{expl3}\ 在 CTAN 上的分发分为三个包:\pkg{l3kernel}、\pkg{l3packages} 和 \pkg{l3experimental}。
% 出于历史原因,
% \begin{verbatim}
%   \RequirePackage{expl3}
% \end{verbatim}
% 加载现在作为 \pkg{l3kernel}\ 分发的代码。这个整体封装包(monolithic package)包含
% 团队认为稳定的所有模块(modules),并且此代码中的任何更改都非常有限。因此,这些材
% 料(material)适用于第三方宏包(third-party packages),而无需担心不受支持。所有这些代码都记录
% 在 \texttt{interface3.pdf}\ 中。使用最新的 \LaTeXe{}\ 内核,此代码内置于
% 格式文件(format file)中,因此无需任何进一步步骤即可使用。
%
% \pkg{l3packages}\ 包中的材料(material)也很稳定,但并不总是处于编程级别(programming level):
% 最值得注意的是,\pkg{xparse}\ 是稳定的,适合更广泛的使用。
%
% 最后,\pkg{l3experimental}\ 包含可供公共使用(public use)但尚未集成到 \pkg{l3kernel}\ 中的模块。
% 这些模块必须显式加载(loaded explicitly)。团队预计,随着时间的推移,所有这些模块都
% 将进入稳定状态(stable status),但在接口(interface)和功能细节(functionality detail)
% 方面可能会更加灵活。关于这些模块的反馈非常宝贵。
%
% \newpage
% \section{{\heiti 从} \LaTeXe{}\ {\heiti 迁移到} \pkg{expl3}}
%
% 为了帮助程序员在现有的 \LaTeXe{}\ 包中使用 \pkg{expl3}\ 代码,可能需要一些关于进行
% 更改的简短注释。欢迎在此提出建议!以下部分与代码有关,有些与编码风格(coding style)有关。
%
% \begin{itemize}
%   \item \pkg{expl3}\ 主要关注编程(programming)。这意味着某些区域仍然需要使
%   用 \LaTeXe{}\ 内部宏。例如,您可能需要 \tn{@ifpackageloaded},因为目前没有本
%   地 \pkg{expl3}\ 包加载模块(package loading module)。
%   \item 用户级宏(user level macros)应该使用 \pkg{xparse}\ 包中的机制生成,
%   \pkg{xparse}\ 是 \texttt{l3package} 包的一部分。
%   \item 在内部级别(internal level),大多数函数都应该生成 \tn{long}(使用 \cs{cs_new:Npn}),
%   而不是“short”(使用\cs{cs_new_nopar:Npn})。
%   \item 如果可能,请在使用之前声明所有变量和函数(使用 \cs{cs_new:Npn}, \cs{tl_new:N}\ 等)。
%   \item 在可能的情况下,优先选择“higher-level(高级)”功能而不是“lower-level(低级)”功能。
%   例如,使用 \cs{cs_if_exist:NTF}\ 而不是 \cs{if_cs_exist:N}。
%   \item 使用空格(space)使代码可读。通常,我们建议采用如下布局(layout):
%     \begin{verbatim}
%       \cs_new:Npn \foo_bar:Nn #1#2
%         {
%           \cs_if_exist:NTF #1
%             { \__foo_bar:n {#2} }
%             { \__foo_bar:nn {#2} { literal } }
%         }
%     \end{verbatim}
%     其中 |{| 和 |}| 周围使用空格,但孤立的(isolated)|#1|、|#2| 等除外。
%   \item 将不同的代码项(code items)放在不同的行上:可读性比紧凑性更重要。
%   \item 函数和变量使用长的描述性名称,辅助函数(auxiliary functions)使用父函数
%   (parent function)名称加上 |aux|、|auxi|、|auxii| 等。
%   \item 如果有疑问,请通过 LaTeX-L 列表询问团队:很快就会有人给您回复!
% \end{itemize}
%
%\newpage
% \section{\pkg{expl3}\ {\heiti 的加载时间选项}}
%
% 为了支持代码作者(code authors),\LaTeXe{}\ 的 \pkg{expl3}\ 包包含少量
% 加载时间选项(load-time options)。这些都以键值的方式工作,识别 \texttt{true}\
% 和 \texttt{false}\ 值。仅提供选项名称(option name)等同于使用具有 \texttt{true}\ 值的选项。
%
% \DescribeOption{check-declarations}
% 应声明 \pkg{expl3}\ 代码中使用的所有变量。对于基于 \TeX{}\ 寄存器(registers)的变量类型,
% \TeX{}\ 会强制执行此操作,但对于那些使用宏作为底层存储系统(underlying storage system)构建
% 的变量类型则不会强制执行。\texttt{check-declarations}\ 选项允许检查所有变量赋值,
% 如果未初始化任何变量,则会发出错误消息。另请参阅 \pkg{l3candidates}\ 中
% 的 \cs{debug_on:n} \texttt{\{check-declarations\}}\ 以获得更精细的控制(finer control)。
%
% \DescribeOption{log-functions}
% \texttt{log-functions}\ 选项用于在 \texttt{.log}\ 文件中记录每个新函数名。这对于
% 调试非常有用,因为这意味着有一个由每个加载的模块创建的所有函数的完整列表
% (引导代码[bootstrap code]所需的非常少的函数除外)。另请参阅 \pkg{l3candidates}\ 中
% 的\cs{debug_on:n} \texttt{\{log-functions\}}\ 以获得更精细的控制(finer control)。
%
% \DescribeOption{enable-debug}
% 为了允许比 \texttt{check-declarations} 和 \texttt{log-functions}\ 提供的更多的
% 本地化检查(localized checking)和日志记录(logging),\pkg{expl3}\ 提供了几
% 个 \cs[no-index]{debug_\ldots{}}\ 函数(在别处描述),这些函数可以打开组内
% (within a group)的相应检查(checks)。只有在使用 \texttt{enable-debug}\ 选项
% 加载 \pkg{expl3}\ 时,才能使用这些函数。
%
% \DescribeOption{backend}
% 选择要用于颜色(color)、图形(graphics)和相关操作(related operations)的后端(backend),
% 这些操作依赖于后端。可用选项包括:
% \begin{itemize}[font = \texttt]
%   \item[dvips] 使用 \texttt{dvips}\ 驱动程序。
%   \item[dvipdfmx] 使用 \texttt{dvipdfmx}\ 驱动程序。
%   \item[dvisvgm] 使用 \texttt{dvisvgm}\ 驱动程序。
%   \item[luatex] 使用 \LuaTeX{}\ 的直接 PDF 输出模式。
%   \item[pdftex] 使用 \pdfTeX{}\ 的直接 PDF 输出模式。
%   \item[xetex] 使用 \texttt{dvipdfmx}\ 驱动程序的 \XeTeX{}\ 版本
% \end{itemize}
% 出于历史原因,也有 \texttt{pdfmode}\ 等效于 \texttt{luatex} 或 \texttt{pdftex},
% \texttt{xdvipdfmx}\ 等效于 \texttt{xetex},但这些都不推荐使用。
%
% \DescribeOption{suppress-backend-headers}
% \texttt{suppress-backend-headers}\ 选项禁止加载后端特定的头文件(backend-specific header files);
% 目前,这只影响 \texttt{dvips}。此选项可用于支持基于 DVI 的路径(DVI-based routes),
% 该路径不支持 \texttt{dvips}\ 使用的标题行(|header| line)。
%
% 调试选项(debugging options)也可以使用 \verb|\keys_set:nn { sys } { ... }|;
% {\kaiti 只有}在尚未加载后端(backend)的情况下,才能以这种方式提供 \texttt{backend}\ 选项。
% 当 \pkg{expl3}\ 由 \LaTeXe{}\ 格式预加载时,这种设置选项的方法很有用。
%
% \newpage
% \section{{\heiti 使用} \pkg{expl3}\ {\heiti 和} \LaTeXe{}\ {\heiti 以外的格式}}
%
% 除了 \LaTeXe{}\ 包 \pkg{expl3}\ 之外,还有一个代码的“generic(通用)”加载程序
% (loader)\texttt{expl3-generic.tex}。这可以使用 plain \TeX{}\ 语法加载
% \begin{verbatim}
%   \input expl3-generic %
% \end{verbatim}
% 这使编程层(programming layer)能够使用其他格式。由于没有选项可以通过这种方式加载,
% 因此会自动使用“native(本地)”驱动程序。如果此“generic(通用)”加载程序
% 与 \LaTeXe{}\ 一起使用,代码会自动切换到适当的包路径(package route)。
%
% 使用通用接口(generic interface)加载编程层(programming layer)后,命
% 令 \cs{ExplSyntaxOn} 和 \cs{ExplSyntaxOff}\ 以及 \pkg{interface3}\ 详述的
% 代码级(code-level)函数和变量可用。请注意,{\kaiti 使用} \pkg{expl3}\ 的其
% 他 \LaTeXe{}\ 包不可加载:包加载(package loading)取决
% 于 \LaTeXe{}\ 包管理机制(package-management mechanism)。
%
% \newpage
% \section{\heiti 引擎/基本要求}
%
% 为了使用团队提供的 \pkg{expl3}\ 和更高级别的包,目前只有最小的基本需求
% 集(minimal set of primitive requirements)
% \begin{itemize}
%    \item 所有这些都来自 \TeX90。
%    \item 来自 \eTeX{}\ 的所有数据,{\kaiti 不包括}|\TeXXeTstate|、|\beginL|、
%    |\beginR|、|\endL| 和 |\endR|(即不包括 \TeX{}-\kern0pt-\reflectbox{\TeX{}})。
%    \item 功能等同于 \pdfTeX{}\ 原语 |\pdfstrcmp|。
% \end{itemize}
% 任何定义 |\pdfoutput|(即允许直接生成 PDF 文件而无需 DVI 中间文件)的引擎还必须
% 提供|\pdfcolorstack|、|\pdfliteral|、|\pdfmatrix|、|\pdfrestore| 和 |\pdfsave| 或
% 等效功能(equivalent functionality)。完全 Unicode 引擎必须提供一种以可扩展方
% 式(expandable manner)生成字符令牌(character tokens)的方法。
%
% 实际上,这些要求是由引擎(engines)来满足的
% \begin{itemize}
%    \item \pdfTeX{} v1.40 或更高版本。
%    \item \XeTeX{} v0.99992 或更高版本。
%    \item \LuaTeX{} v0.95 或更高版本。
%    \item e-(u)\pTeX{} 2012年中或更高版本。
% \end{itemize}
%
% \pkg{expl3}\ 核心之外的其他模块可能需要额外的原语(primitives)。特别是,
% 第三方作者(third-party authors)可能会大大扩展原始覆盖要求(primitive coverage requirements)。
%
% \newpage
% \section{\LaTeX{}\ {\heiti 项目}}
%
% \LaTeX3\ 的开发由 \LaTeX{}\ 项目(\url{https://www.latex-project.org/latex3/})执行。
%
% \begin{thebibliography}{1}
%
%   \bibitem{A-W:K-TB}
%     Donald E Knuth
%     \newblock \emph{The \TeX{}book}.
%     \newblock Addison-Wesley, Reading, Massachusetts, 1984.
%
%   \bibitem{A-W:GMS94}
%     Goossens, Mittelbach and Samarin.
%     \newblock \emph{ The \LaTeX{} Companion}.
%     \newblock Addison-Wesley, Reading, Massachusetts, 1994.
%
%   \bibitem{A-W:LLa94}
%     Leslie Lamport.
%     \newblock \emph{\LaTeX{}: A Document Preparation System}.
%     \newblock Addison-Wesley, Reading, Massachusetts, second edition, 1994.
%
%   \bibitem{tub:MR97-1}
%     Frank Mittelbach and Chris Rowley.
%     \newblock \enquote{The \LaTeX{} Project}.
%     \newblock \emph{TUGboat},
%     Vol.\,18, No.\,3, pp.\,195--198, 1997.
%
% \end{thebibliography}
%
% \end{documentation}
%
% \begin{implementation}
%
% \newpage
% \section{\pkg{expl3}\ {\heiti 实现}}
%
% 这里的实现(implementation)包括几个方面。有两个“加载程序(loaders)”需要定义:
% 代码中特定于 \LaTeXe{}\ 或非 \LaTeXe{}\ 格式的部分。它们必须以完全不同的方式涵盖
% 相同的概念:因此,大部分代码是以单独的块(separate blocks)的形式给出的。还有一小段
% 代码用于启动“有效负载(payload)”:这是为了确保加载始终以正确的方式进行。
%
% \subsection[加载程序联锁]{\heiti 加载程序联锁}
%
% 一个简短的设置,用于检查加载程序(loader)和“有效负载(payload)”版本是否匹配。
%
% \begin{macro}{\ExplLoaderFileDate}
%   由于 DocStrip 用于为来自同一源的所有文件生成 \cs{ExplFileDate},因此它必须匹配。
%   故加载程序只需使用新名称(new name)保存此信息。
%    \begin{macrocode}
%<*loader>
\let\ExplLoaderFileDate\ExplFileDate
%</loader>
%    \end{macrocode}
% \end{macro}
%
% \begin{macro}{\c__kernel_expl_date_tl}
%   \LaTeXe{}\ 加载程序(loader)存储 \cs{ExplFileDate}\ 的私有副本(private copy),该副本
%   在构建格式时是固定的,以后不能更改。而 \cs{ExplFileDate}\ 确保加载程序
%   (\texttt{2ekernel} 或 \texttt{package})版本与 \file{expl3-code.tex}\ 的版本匹配。
%   这可以确保最终的 |\usepackage{expl3}| 版本与格式中的版本匹配。在用户树((user tree))中
%   常常会有杂散的格式文件(stray format files),这些文件会因版本不匹配
%   ({\kaiti 例如}\url{https://github.com/latex3/latex3/issues/781})而引发错误。
%
%   这个代码块也发送给通用加载程序(generic loader),因为这里没有预加载 \pkg{expl3} ,所以
%   内核日期(kernel date)应该等于加载程序日期(loader date)。
%    \begin{macrocode}
%<*2ekernel|generic>
\begingroup
  \catcode`\_=11
  \expandafter
  \ifx\csname c__kernel_expl_date_tl\endcsname\relax
    \global\let\c__kernel_expl_date_tl\ExplFileDate
  \fi
\endgroup
%</2ekernel|generic>
%    \end{macrocode}
% \end{macro}
%
% 联锁测试(interlock test)本身很简单:必须定义 \cs{ExplLoaderFileDate}\ 并与 \cs{ExplFileDate}\ 相同。
% 因为这必须同时适用于 \LaTeXe{}\ 和其他格式,所以需要进行一些自动检测,这样做可以避免 \LaTeXe{}\ 和其他格
% 式有两个非常相似的块(blocks)。
%    \begin{macrocode}
%<*!loader>
\begingroup
  \def\next{\endgroup}%
  \expandafter\ifx\csname PackageError\endcsname\relax
    \begingroup
      \def\next{\endgroup\endgroup}%
      \def\PackageError#1#2#3%
        {%
          \endgroup
          \errhelp{#3}%
          \errmessage{#1 Error: #2!}%
        }%
  \fi
  \expandafter\ifx\csname ExplLoaderFileDate\endcsname\relax
    \def\next
      {%
        \PackageError{expl3}{No expl3 loader detected}
          {%
            您试图直接使用 expl3 代码,而不是使用正确的加载程序。expl3 加载将中止。
          }%
        \endgroup
        \endinput
      }
  \else
    \ifx\ExplLoaderFileDate\ExplFileDate
    \else
      \def\next
        {%
          \PackageError{expl3}{Mismatched expl3 files detected}
            {%
              您试图用不匹配的文件加载 expl3:可能有一个或多个“本地安装”的文件
              存在冲突。expl3 加载将中止。
            }%
          \endgroup
          \endinput
        }%
    \fi
\fi \next
%</!loader>
%    \end{macrocode}
%
% 有效负载(payload)的重新加载测试(reload test),以防万一。
%    \begin{macrocode}
%<*!loader>
\begingroup\expandafter\expandafter\expandafter\endgroup
\expandafter\ifx\csname ver@expl3-code.tex\endcsname\relax
  \expandafter\edef\csname ver@expl3-code.tex\endcsname
    {%
      \ExplFileDate\space
      L3 编程层
    }%
\else
  \expandafter\endinput
\fi
%</!loader>
%    \end{macrocode}
%
% 所有好处:记录所用代码的版本(为了记录完整性[log completeness])。由于这是或多或少
% 的 \cs{ProvidesPackage}\ 没有一个单独的文件(separate file),并且这也需要在没
% 有 \LaTeXe{}\ 的情况下工作,只需将信息直接写入日志(log)即可。
%    \begin{macrocode}
%<*!loader>
\immediate\write-1 %
  {%
    宏包:expl3
      \ExplFileDate\space
      L3 编程层 (代码)%
  }%
%</!loader>
%    \end{macrocode}
%
% \subsection[\LaTeXe{}\ 加载程序]{\LaTeXe{}\ {\heiti 加载程序}}
%
% 使用 \LaTeXe{}\ 加载(loading)可以作为格式的一部分(预加载)或作为一个包。我们必须考虑两种可能
% 的路径(paths),当然,要将包加载到预加载的(pre-load)。这意味着这里的代码必须能够安全地防止
% 重新加载(re-loading)。
%
%    \begin{macrocode}
%<*package&loader|2ekernel>
%    \end{macrocode}
%
% 标识(identify)包或添加到格式化消息(format message)中。
%    \begin{macrocode}
%<*2ekernel>
\everyjob\expandafter{\the\everyjob
  \message{L3 programming layer <\ExplFileDate>}%
}
%</2ekernel>
%<*!2ekernel>
\ProvidesPackage{expl3}
  [%
    \ExplFileDate\space
    L3 编程层 (加载程序)
  ]%
%</!2ekernel>
%    \end{macrocode}
%
% \begin{macro}{\ProvidesExplPackage, \ProvidesExplClass, \ProvidesExplFile}
%   对于在此基础上构建的其他包(packages)和类(classes),不需要每次都使
%   用 \cs{ExplSyntaxOn}\ 是很方便的。所有宏都使用相同的内部宏和适当的 \LaTeXe{}\ 命令。
%    \begin{macrocode}
\protected\def\ProvidesExplPackage
  {\@expl@provides@file@@Nnnnnn\ProvidesPackage{Package}}
\protected\def\ProvidesExplClass
  {\@expl@provides@file@@Nnnnnn\ProvidesClass{Document Class}}
\protected\def\ProvidesExplFile
  {\@expl@provides@file@@Nnnnnn\ProvidesFile{File}}
%    \end{macrocode}
% \end{macro}
%
% \begin{macro}{\@expl@provides@file@@Nnnnnn, \@expl@provides@generic@@wnnw}
%   我们需要检查 \cs[no-index]{Provides\meta{thing}}\ 是否存在,因为我们需要
%   在 \LaTeXe{}\ 内核中尽早加载。
%    \begin{macrocode}
\protected\long\def\@expl@provides@file@@Nnnnnn#1#2#3#4#5#6%
  {%
    \ifnum0%
        \ifdefined#11\fi
        \ifx\relax#1\else1\fi
        =11
      \expandafter#1%
    \else
      \@expl@provides@generic@@wnnw{#2}%
    \fi
      {#3}[{#4 \ifx\relax#5\relax\else v#5\space\fi #6}]%
    \ExplSyntaxOn
  }
\protected\long\def\@expl@provides@generic@@wnnw#1\fi#2[#3]%
  {%
    \immediate\write-1{#1: #2 #3}%
  }
%    \end{macrocode}
% \end{macro}
%
%  加载业务结束:这将保留 \cs{expl3}\ 语法。测试确保我们只加载一次,而不需要知道是否有
%  预加载步骤(preloading step)。
%    \begin{macrocode}
\begingroup\expandafter\expandafter\expandafter\endgroup
\expandafter\ifx\csname tex\string _let:D\endcsname\relax
  \expandafter\@firstofone
\else
  \expandafter\@gobble
\fi
  {\input expl3-code.tex }%
%    \end{macrocode}
%
% 检查引导程序代码(bootstrap code)是否中止了加载:如果中止了加载,则在这里静默(silently)退出。
%    \begin{macrocode}
\begingroup\expandafter\expandafter\expandafter\endgroup
\expandafter\ifx\csname tex\string _let:D\endcsname\relax
  \expandafter\endinput
\fi
%    \end{macrocode}
%
%    \begin{macrocode}
%<@@=expl>
%    \end{macrocode}
%
% 此时,如果定义了 \cs{c__kernel_expl_date_tl},只需调用 \cs{__kernel_dependency_version_check:Nn},
% 检查它是否与 \cs{ExplLoaderFileDate}\ 匹配。这里只在 \cs{c__kernel_expl_date_tl}\ 存在的情况下执行测试,
% 因为此文件可以以 \LaTeXe{}\ 格式加载,而不预加载 \pkg{expl3},其中令牌列表(token list)不存在。
%
% 这都是在 \texttt{package} \pkg{docstrip}\ 中完成的,因为它不适用于 \pkg{expl3.ltx}。
%    \begin{macrocode}
%<*package>
\ifcsname\detokenize{c__kernel_expl_date_tl}\endcsname
  \expandafter\@firstofone
\else
%    \end{macrocode}
% 如果 \cs{c__kernel_expl_date_tl}\ {\kaiti 不}存在,我们可能会以未预加
% 载 \pkg{expl3}\ 的格式加载,或者以出现上述错误的较早版本(尽管仍然兼容)加载。如果作为包加载,
% \file{expl3-code.tex}\ 已被读取,此时 \pkg{expl3}\ 语法已开启。否则,它已加载到稍旧的内核中,
% 因此我们将触发不兼容错误消息并中止加载。
%    \begin{macrocode}
  \ifodd\csname\detokenize{l__kernel_expl_bool}\endcsname
%    \end{macrocode}
%   在包模式(package mode)下,所有文件都会一次性加载,因此版本会匹配。我们只需设
%   置 \cs{c__kernel_expl_date_tl},以便进一步的依赖关系不会被中断:
%    \begin{macrocode}
    \global\expandafter\let\csname\detokenize
      {c__kernel_expl_date_tl}\endcsname\ExplLoaderFileDate
    \expandafter\expandafter
    \expandafter\@gobble
  \else
%    \end{macrocode}
%   在不兼容的版本中重新加载是一个错误:
%    \begin{macrocode}
    \expandafter\expandafter
    \expandafter\@firstofone
  \fi
\fi
  {\csname\detokenize{__kernel_dependency_version_check:Nn}\endcsname
     \ExplLoaderFileDate{expl3.sty}}%
%</package>
%    \end{macrocode}
%
% 在这里,我们还可以检测是否正在重新加载(reloading)。这段代码进入 \texttt{expl3.ltx} 和
% \texttt{expl3.sty},前者加载为 \LaTeXe{}\ 格式。第一次加载此代码时,\cs{g_@@_reload_bool}\ 布
% 尔值(boolean)不存在(下面 \cs{ifcsname}\ 的 \cs{else}\ 分支),因此我们创建了它。
% 如果 \cs{ifcsname}\ 为true,则执行 \cs{ExplSyntaxOn}(因为在重新加载时,不会再次读
% 取 \texttt{expl3-code.tex}),并将 \cs{g_@@_reload_bool}\ 设置为 true。
%    \begin{macrocode}
\ifcsname\detokenize{g_@@_reload_bool}\endcsname
  \ExplSyntaxOn
  \bool_gset_true:N \g_@@_reload_bool
\else
  \bool_new:N \g_@@_reload_bool
\fi
%    \end{macrocode}
%
% \begin{variable}{\c_@@_def_ext_tl}
%   \LaTeXe{}\ 所需,并避免重新加载问题。最好显式检查变量,而不是使用 \cs{g_@@_reload_bool},
%   因为有些变量只出现在一个代码文件中,所以 \cs{g_@@_reload_bool}\ 并不一定意味着变量已经声明。
%    \begin{macrocode}
\tl_if_exist:NF \c_@@_def_ext_tl
  { \tl_const:Nn \c_@@_def_ext_tl { def } }
%    \end{macrocode}
% \end{variable}
%
% \begin{macro}
%   {\__kernel_sys_configuration_load:n,\__kernel_sys_configuration_load_std:n}
%   为了加载配置,我们有以下几种情况:
%   \begin{itemize}
%     \item \pkg{expl3} \pkg{expl3}\ 是预加载的(pre-loading):在加载配置(configuration)时,我们已经有了完整的文件加载
%       堆栈(loading stack),这里只需要标准版本的代码。
%     \item 这个包是预加载的(pre-loading):我们再次使用标准版本,但我们还没有测试。
%     \item 该包在没有预加载代码(pre-loaded code)的情况下使用:我们需要手动管理 \pkg{expl3}\ 语法。
%   \end{itemize}
%    \begin{macrocode}
\cs_gset_protected:Npn \__kernel_sys_configuration_load:n #1
%<*!2ekernel>
  {
    \ExplSyntaxOff
    \cs_undefine:c { ver@ #1 .def }
    \@onefilewithoptions {#1} [ ] [ ]
      \c_@@_def_ext_tl
    \ExplSyntaxOn
  }
\cs_gset_protected:Npn \__kernel_sys_configuration_load_std:n #1
%</!2ekernel>
  {
    \cs_undefine:c { ver@ #1 .def }
    \@onefilewithoptions {#1} [ ] [ ]
      \c_@@_def_ext_tl
  }
%    \end{macrocode}
% \end{macro}
%
% \begin{variable}{\l_@@_options_clist}
%    \begin{macrocode}
%<*!2ekernel>
\clist_if_exist:NF \l_@@_options_clist
  { \clist_new:N \l_@@_options_clist }
\DeclareOption*
  { \clist_put_right:NV \l_@@_options_clist \CurrentOption }
\ProcessOptions \relax
%</!2ekernel>
%    \end{macrocode}
% \end{variable}
%
%   相当标准的设置创建(setting creation)。
%    \begin{macrocode}
\keys_define:nn { sys }
  {
    backend .choices:nn =
      { dvipdfmx , dvips , dvisvgm , luatex , pdftex , pdfmode , xdvipdfmx , xetex }
      { \sys_load_backend:n {#1} } ,
    check-declarations .code:n =
      {
        \sys_load_debug:
        \debug_on:n { check-declarations }
      } ,
    driver .meta:n = { backend = #1 } ,
    enable-debug .code:n =
      \sys_load_debug: ,
    log-functions .code:n =
      {
        \sys_load_debug:
        \debug_on:n { log-functions }
      } ,
    suppress-backend-headers .bool_gset_inverse:N
      = \g__kernel_backend_header_bool ,
    suppress-backend-headers .initial:n = false ,
    undo-recent-deprecations .code:n = {} % A stub
  }
%    \end{macrocode}
%
% \begin{macro}{\@expl@sys@load@backend@@@@}
%   后端(backend)必须在文档开始时就位(place):必须在检查全局选项(global options)以使用之前就位。
%
%   在本包中定义的 \cs[no-index]{@expl@...@@@@}\ 宏是 \LaTeXe{}\ 的接口。目前(这将随着
%   下一版本的 \LaTeXe{}\ 而改变)在代码的这一点上有两种可能的情况:
%   \cs{@expl@sys@load@backend@@@@}\ (以及其他)已经存在,因为它们是在 \texttt{ltexpl.ltx}\ 中定义的
%   (在 \texttt{2ekernel}\ 模式中)或在 \texttt{expl3.ltx}(在 \texttt{package}\ 模式中)\ 中定义的。
%
%   在 \texttt{2ekernel}\ 模式中,如果它们存在,我们将使用未来(2020-10-01)发布的 \LaTeXe{},
%   我们不需要(也不能)修补 \LaTeXe\ 的内部,因为这些命令已经存在。虽然它们不存
%   在于 \texttt{2ekernel}\ 模式中,但我们使用的是较旧版本的内核,所以我们{\kaiti 必须}进行修补。
%
%   在 \texttt{package}\ 模式中,如果存在这些命令,那么我们使用的是预加载
%   了 \textsf{expl3}\ 的 \LaTeXe{}\ 版本(任何版本),而且在任何情况下,补丁(patching)已经完成,
%   或者宏本身就是格式,因此无需执行任何操作。但如果在 \texttt{package}\ 模式中这
%   些宏不存在,我们有一个更旧版本的 \LaTeXe{},它甚至没有预加载 \textsf{expl3},
%   因此补丁是必要的。
%
%   这意味着在 \texttt{2ekernel}\ 和 \texttt{package}\ 模式中,我们必须检
%   查 \cs{@expl@sys@load@backend@@@@}\ 是否存在,如果不存在,则修补一些 \LaTeXe{}\ 内部构件。
%
%   在更新的 \LaTeXe{}\ 中,这些宏在 \texttt{ltexpl.dtx}\ 中有一个空定义(empty definition),
%   以防加载此文件(\texttt{expl3.ltx})时发生错误,因此可以在 \LaTeXe{}\ 内核中安全使用它们。
%
%   \cs{@expl@sys@load@backend@@@@}\ 在 \cs{document}\ 的开头插入,但在关闭后,
%   由 \cs{begin}\ 开始的组(group)。当使用 \cs{tl_put_left:Nn}\ 修补 \cs{document}\ 中
%   的后端加载(backend loading)时,我们需要确保它发生在组级别(group level)0,因此
%   出现了奇怪的 |\endgroup...\begingroup| 问题。
%
%   这段代码只应在未预加载 \pkg{expl3}\ 的情况下,在 \LaTeXe{}\ 中加
%   载 \texttt{expl3.sty}\ 时执行,因此我们检查 \cs{@expl@sys@load@backend@@@@}\ 是否存在。
%    \begin{macrocode}
\cs_if_exist:NF \@expl@sys@load@backend@@@@
  {
    \tl_put_left:Nn \document
      {
        \endgroup
        \@expl@sys@load@backend@@@@
        \begingroup
      }
  }
%    \end{macrocode}
%
%   现在我们无论如何都要定义它。
%    \begin{macrocode}
\cs_gset_protected:Npn \@expl@sys@load@backend@@@@
  {
    \str_if_exist:NF \c_sys_backend_str
      { \sys_load_backend:n { } }
  }
%    \end{macrocode}
% \end{macro}
%
%  处理宏包选项(package options)。
%    \begin{macrocode}
%<*!2ekernel>
\keys_set:nV { sys } \l_@@_options_clist \str_if_exist:NF \c_sys_backend_str
  { \sys_load_backend:n { } }
%</!2ekernel>
%    \end{macrocode}
%
%    \begin{macrocode}
%<*!2ekernel>
\bool_if:NT \g_@@_reload_bool
  {
    \cs_gset_eq:NN \__kernel_sys_configuration_load:n
      \__kernel_sys_configuration_load_std:n
    \ExplSyntaxOff
    \file_input_stop:
  }
%</!2ekernel>
%    \end{macrocode}
%
% 现在或在下一次运行期间加载代码的动态部分(dynamic part)。
%    \begin{macrocode}
\cs_if_free:cTF { ver@expl3.sty }
  {
    \tex_everyjob:D \exp_after:wN
      {
        \tex_the:D \tex_everyjob:D
        \sys_everyjob:
      }
  }
  { \sys_everyjob: }
%    \end{macrocode}
%
% \begin{variable}{\s_@@_stop}
%   内部扫描标记。此代码必须是重载安全的(reload-safe),因此必须使
%   用\cs{if_cs_exist:N} (\cs{cs_if_exist:NF})保护此代码,因为它对于等
%   于 \cs{scan_stop:}\ 的控制序列(control sequences)返回 false。
%    \begin{macrocode}
\reverse_if:N \if_cs_exist:N \s_@@_stop
  \scan_new:N \s_@@_stop
\fi:
%    \end{macrocode}
% \end{variable}
%
% \begin{macro}{\@pushfilename, \@popfilename}
% \begin{macro}{\@@_status_pop:w}
% \begin{macro}{\@expl@push@filename@@@@}
% \begin{macro}{\@expl@push@filename@aux@@@@}
% \begin{macro}{\@expl@pop@filename@@@@}
%   这里的想法是使用 \LaTeXe{}\ 的 \tn{@pushfilename}\ 和 \tn{@popfilename}\ 来跟
%   踪当前语法状态(syntax status)。这可以通过在每次推送到堆栈(stack)时保存当
%   前状态标志(status flag),然后在弹出阶段(pop stage)恢复它,并检查
%   代码环境(code environment)是否仍应处于活动状态来实现。
%
%   对于 \cs{@expl@sys@load@backend@@@@},这里的代码遵循与上面相同的修补逻辑。
%    \begin{macrocode}
\cs_if_exist:NF \@expl@push@filename@@@@
  {
    \tl_put_left:Nn  \@pushfilename { \@expl@push@filename@@@@ }
    \tl_put_right:Nn \@pushfilename { \@expl@push@filename@aux@@@@ }
  }
\cs_gset_protected:Npn \@expl@push@filename@@@@
  {
    \exp_args:Nx \__kernel_file_input_push:n
      {
        \tl_to_str:N \@currname .
        \tl_to_str:N \@currext
      }
    \tl_put_left:Nx \l_@@_status_stack_tl
      {
        \bool_if:NTF \l__kernel_expl_bool
          { 1 }
          { 0 }
      }
    \ExplSyntaxOff
  }
%    \end{macrocode}
%   需要使用这个小技巧来获取正在加载的文件的名称,以便我们可以记录(record)它。
%    \begin{macrocode}
\cs_gset_protected:Npn \@expl@push@filename@aux@@@@ #1#2#3
  {
    \str_gset:Nn \g_file_curr_name_str {#3}
    #1 #2 {#3}
  }
%    \end{macrocode}
%
%    \begin{macrocode}
\cs_if_exist:NF \@expl@pop@filename@@@@
  {
    \tl_put_right:Nn \@popfilename
      { \@expl@pop@filename@@@@ }
  }
\cs_gset_protected:Npn \@expl@pop@filename@@@@
  {
    \__kernel_file_input_pop:
    \tl_if_empty:NTF \l_@@_status_stack_tl
      { \ExplSyntaxOff }
      { \exp_after:wN \@@_status_pop:w \l_@@_status_stack_tl \s_@@_stop }
  }
%    \end{macrocode}
%   弹出辅助函数(pop auxiliary function)从堆栈(stack)中删除第一项(first item),
%   保存堆栈的其余部分,然后执行测试(test)。此处的标志(flag)不是正确的 \texttt{bool},
%   因此使用了低级测试(low-level test)。
%    \begin{macrocode}
\cs_gset_protected:Npn \@@_status_pop:w #1#2 \s_@@_stop
  {
    \tl_set:Nn \l_@@_status_stack_tl {#2}
    \int_if_odd:nTF {#1}
      { \ExplSyntaxOn }
      { \ExplSyntaxOff }
  }
%    \end{macrocode}
% \end{macro}
% \end{macro}
% \end{macro}
% \end{macro}
% \end{macro}
%
% \begin{variable}{\l_@@_status_stack_tl}
%   由于 \pkg{expl3}\ 本身无法加载已经处于活动状态的代码环境(code environment),
%   因此可以安全地调用包末尾的 \cs{ExplSyntaxOff}。
%    \begin{macrocode}
\tl_if_exist:NF \l_@@_status_stack_tl
  {
    \tl_new:N \l_@@_status_stack_tl
    \tl_set:Nn \l_@@_status_stack_tl { 0 }
  }
%    \end{macrocode}
% \end{variable}
%
%  按照承诺(promised)整理配置加载(configuration loading)。
%    \begin{macrocode}
%<*!2ekernel>
\cs_gset_eq:NN \__kernel_sys_configuration_load:n
  \__kernel_sys_configuration_load_std:n
%</!2ekernel>
%    \end{macrocode}
%
% 对于预加载,我们必须手动禁用语法(syntax)。
%    \begin{macrocode}
%<*2ekernel>
\ExplSyntaxOff
%</2ekernel>
%    \end{macrocode}
%
%    \begin{macrocode}
%</package&loader|2ekernel>
%    \end{macrocode}
%
% \subsection[通用加载程序]{\heiti 通用加载程序}
%
%    \begin{macrocode}
%<*generic>
%    \end{macrocode}
%
% 通用加载程序(generic loader)从测试开始,以确保当前格式不是 \LaTeXe{}!
%    \begin{macrocode}
\begingroup
  \def\tempa{LaTeX2e}%
  \def\next{}%
  \ifx\fmtname\tempa
    \def\next
      {%
        \PackageInfo{expl3}{Switching from generic to LaTeX2e loader}%
%    \end{macrocode}
% \cs{relax}\ 停止 \cs{RequirePackage}\ 扫描日期参数(date argument)。加载
% 包{\kaiti 之后}放入 \tn{endinput}\ 至关重要,否则 \tn{endinput}\ 将在文件第一行
% 末尾关闭文件 \file{expl3.sty}:实际上,只要 \file{expl3.sty}\ 处于打开状态,
% 就不可能关闭文件 \file{expl3-generic.tex}。
%    \begin{macrocode}
        \RequirePackage{expl3}\relax \endinput
      }%
  \fi
\expandafter\endgroup \next
%    \end{macrocode}
%
% 重新加载检查(reload check)并识别宏包(identify the package):没有 \LaTeXe{}\ 机制,
% 所以这一切非常基本(pretty basic)。
%    \begin{macrocode}
\begingroup\expandafter\expandafter\expandafter\endgroup
\expandafter\ifx\csname ver@expl3-generic.tex\endcsname\relax \else
  \immediate\write-1
    {%
      expl3 包信息:包已加载。%
    }%
  \expandafter\endinput
\fi \immediate\write-1
  {%
    Package: expl3
    \ExplFileDate\space
    L3 编程层 (加载程序)%
  }%
\expandafter\edef\csname ver@expl3-generic.tex\endcsname
  {\ExplFileDate\space L3 programming layer}%
%    \end{macrocode}
%
% \begin{variable}[int]{\l@expl@tidy@tl}
%   保存 |@| 的类别代码(category code),然后将其设置为“letter”。
%    \begin{macrocode}
\expandafter\edef\csname l@expl@tidy@tl\endcsname
  {%
    \catcode64=\the\catcode64\relax
    \let\expandafter\noexpand\csname l@expl@tidy@tl\endcsname
      \noexpand\undefined
  }%
\catcode64=11 %
%    \end{macrocode}
% \end{variable}
%
% \begin{macro}{\AtBeginDocument}
% \begin{macro}[int]{\expl@AtBeginDocument}
%   包代码中有一些 \cs{AtBeginDocument}\ 的用法:最简单的方法是“{\kaiti 现在}”编写代码。
%   由于诸如 \pkg{miniltx}\ 之类的捆绑包(bundles)可能已经定义了 \cs{AtBeginDocument},
%   因此任何现有定义都会保存起来,以便在有效负载(payload)之后进行恢复。
%    \begin{macrocode}
\let\expl@AtBeginDocument\AtBeginDocument
\def\AtBeginDocument#1{#1}%
\expandafter\def\expandafter\l@expl@tidy@tl\expandafter
  {%
    \l@expl@tidy@tl
    \let\AtBeginDocument\expl@AtBeginDocument
    \let\expl@AtBeginDocument\undefined
  }%
%    \end{macrocode}
% \end{macro}
% \end{macro}
%
%  加载业务结束:这将保留 \cs{expl3}\ 语法。
%    \begin{macrocode}
\input expl3-code.tex %
%    \end{macrocode}
%
% 检查引导程序代码(bootstrap code)是否中止了加载:如果中止了加载,则在这里静默(silently)退出。
%    \begin{macrocode}
\begingroup\expandafter\expandafter\expandafter\endgroup
\expandafter\ifx\csname tex\string _let:D\endcsname\relax
  \expandafter\endinput
\fi
%    \end{macrocode}
%
% \begin{macro}{\__kernel_sys_configuration_load:n}
%   非常基本。
%    \begin{macrocode}
\cs_gset_protected:Npn \__kernel_sys_configuration_load:n #1
  {
    \group_begin:
    \cs_set_protected:Npn \ProvidesExplFile
      {
        \char_set_catcode_space:n { `\  }
        \ProvidesExplFileAux
      }
    \cs_set_protected:Npn \ProvidesExplFileAux ##1##2##3##4
      {
        \group_end:
        \iow_log:x { File:~##1~##2~v##3~##4 }
      }
    \tex_input:D #1 .def \scan_stop:
  }
%    \end{macrocode}
% \end{macro}
%
% \begin{macro}{\g__kernel_backend_header_bool}
% 加载动态代码(dynamic code)和标准后端(standard back-end)。
%    \begin{macrocode}
\sys_everyjob: \bool_new:N \g__kernel_backend_header_bool \bool_gset_true:N
\g__kernel_backend_header_bool \sys_load_backend:n { }
%    \end{macrocode}
% \end{macro}
%
%  对于通用加载程序(generic loader),需要执行一些最后的步骤。
%  转换 \cs{expl3}\ 语法并整理少量的临时更改(temporary changes)。
%    \begin{macrocode}
\ExplSyntaxOff \l@expl@tidy@tl
%    \end{macrocode}
%
%    \begin{macrocode}
%</generic>
%    \end{macrocode}
%
% \end{implementation}
%
% \newpage
% \PrintIndex
