% !Mode:: "TeX:Hard:UTF-8"

\cprotect\chapter{\verb|\text| 命令}
\verb|\text| 命令的主要作用是用于一些行间公式的文字。 它和\LaTeX{} 的 \verb|\mbox| 命令的作用很像,  但是有更多优点。 如果您希望一段文字出现在下标,  可输入 \verb|..._{\textrm{word| \verb|or phrase}}|,  这要比 \verb|\mbox| 等价形式简单:
\verb|...{\mbox{\rmfamily\scriptsize word or|   \verb|phrase}}|。 注意标准的 \verb|\textrm| 命令会使用\pkg{amsmath}的 \verb|\text| 命令的定义,但保证 \verb|\rmfamily| 字体被使用。
\begin{tcblisting}{}
\begin{equation}
f_{[x_{i-1},x_i]} \text{ is monotonic,}\quad i=1,\dots,c+1
\end{equation}
\end{tcblisting}

\verb|\text| 的字体会和周围的环境字体保持一致,例如在定理环境中,\verb|\text| 的内容会变成斜体字体。

如果在 \verb|\text| 命令中包含数学表达式,一定要用 \verb|$...$| 指明。
\begin{tcblisting}{}
\[\partial_s f(x) = \frac{\partial}{\partial x_0} f(x) \quad \text{for $x=x_0+Ix_1$.}\]
\end{tcblisting}

函数名不应该用 \verb|\text| 输入,而应该用 \verb|\mathrm| 或 \verb|\DeclareMathOperator| 比较合适。这些是固定实体,不应根据外部内容进行更改(例如出现在用斜体设置的定理中),并且在声明的运算符的情况下,自动应用适当的间距。
