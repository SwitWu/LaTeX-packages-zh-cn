% !Mode:: "TeX:Hard:UTF-8"


\chapter{积分与求和}
\section{多行上下标}
\verb|\substack|命令可以用于生成多行下标或上标,例如

\begin{listing}
\[\sum_{\substack{0\leq i\leq m\\0<j<n}}
P(i,j)\]
\end{listing}

更一般的形式是 \verb|subarray| 环境,其可以指定多行公式居中(c)或者左对齐(l),例如
\begin{listing}
\[\sum_{\begin{subarray}{l}
        i\in\Lambda\\ 0<j<n
       \end{subarray}}
P(i,j)\]
\end{listing}

\cprotect\section{\verb|\sideset| 命令}\label{sec:sideset}
\verb|\sideset| 命令是为了一个特殊用途:将符号放在一个巨算符的上标或下标角落上,比如$\sum$或$\prod$。\emph{注意这个命令不能用于其他非求和符号类的符号}。 典型的例子是如果您想在一个求和符号上输入一撇(\verb|\prime|),如果没有求和上下限,您可以使用 \verb|\nolimits|,比如在行间模式中输入 \verb|\sum\nolimits' E_n|:
\begin{equation}
\sum\nolimits' E_n
\end{equation}

然而,如果您想要的不仅是一撇,还想要在求和符号的上方或下方再输入一些其他的东西,这就不容易了——事实上,要是没有 \verb|\sideset| 的话,这是非常困难的。但是利用 \verb|\sideset|,您可以输入
\begin{listing}
\[\sideset{}{'}
  \sum_{n<k,\;\text{$n$为奇数}} nE_n\]
\end{listing}

额外的一对空大括号可以通过这样一个事实来解释:\verb|\sideset| 具有在一个巨算符的每个角都放置一个或多个符号的能力。例如:
\begin{listing}
\[
\sideset{_{a}^{b}}%
        {_{c}^{d}}\prod
\]
\end{listing}

\section{上下标的放置与limits选项\label{sec7.3}}
默认的上下标位置取决于主算符。求和类符号默认为displaylimits放置:当求和类符号出现在行间公式时,上标或下标出现在求和类符号的正上方或正下方,但是在行内公式中,上下标被放置在旁边,来避免周围的文本行不悦目及浪费地伸展。而积分类符号则默认将上下标放在旁边,即使在行间公式中也是一样。 (参看第二章中的 \verb|intlimits| 选项)。

诸如$\sin$或$\lim$之类的算符名,可能既有displaylimits,又有limits放置模式,这取决于它们的定义。标准的算符名是根据其数学用途来定义的。

\verb|\limits| 和 \verb|\nolimits| 命令可以用来修改主算符的正常上下标位置:
\[\sum\nolimits_X,\qquad\iint\limits_X,\qquad\varliminf\nolimits_{n\to\infty}\]

要定义一个命令使得其上标与 \verb|\sum| 的displaylimits行为一致,在其定义的末尾加入 \verb|\displaylimits|。 当多种 \verb|\limits|,  \verb|\nolimits| 或 \verb|\displaylimits| 选项连续出现时,  最后一个选项优先。
\section{多重积分符号}
\verb|\iint|、\verb|\iiint|、\verb|\iiiint| 输出多重积分符号,且给出自适应的间距,在文本和行间模式都是一样。 \verb|\idotsint| 是这种积分号的扩展,只给出两个积分号,但在中间用点来填充。
\begin{equation}
\iint\limits_Af(x,y)\,\text dx\text dy\qquad\iiint\limits_Af(x,y,z)\,\text dx\text dy\text d z
\end{equation}
\begin{equation}
\iiiint\limits_Af(w,x,y,z)\,\text dw\text dx\text dy\text dz\qquad\idotsint\limits_Af(x_1,\cdots,x_k)
\end{equation}
