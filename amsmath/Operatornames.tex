% !Mode:: "TeX:Hard:UTF-8"

\chapter{算符名称}
\section{定义新算符名称}
数学函数,例如$\log$、$\sin$和$\lim$用罗马形式输出使得它们与一般的单个字母的斜体数学变量相区别。比较常用的函数都有预定义的名称,例如 \verb|\log|、\verb|\sin|、\verb|\lim| 等,但是在数学文档中经常会出现一些新的命令,所以\pkg{amsmath}包提供了一个定义新算符名称的命令。要定义一个类似于 \verb|\sin| 的命令 \verb|\xxx|,您可以在导言区输入
\begin{verbatim}
\DeclareMathOperator{\xxx}{xxx}
\end{verbatim}

于是在输入 \verb|\xxx| 时就会以适当的字体输出xxx,并且如果有必要的话会在两边自动加上适当的间距,所以您得到的是$A\xxx B$而不是$A\text{xxx}B$。在 \verb|\DeclareMathOperator| 的第二个参数(名称部分)中,  第二个连字符是文本连字符而不是负号,  星号*将会以抬升的文本输出而不是以一个居中的数学星号(比较$a\text{-}b\text{*}c$和$a-b*c$),  否则名称文本将以数学模式打印,  以便您可以使用,  例如那里的上标和下标。

如果新的算符需要将上标或下标置于‘limits’位置,例如$\lim$、$\sup$和$\max$,则可以使用带 \verb+*+ 形式的 \verb|\DeclareMathOperator| 命令:
\begin{verbatim}
\DeclareMathOperator*{\Lim}{Lim}
\end{verbatim}
也可以参看\S \ref{sec7.3} 的关于上下标的设置。

下面的算符名称是预定义的:

\begin{center}
\begin{tabular}{rlrlrlrl}
\verb|\arccos|&$\arccos$&\verb|\deg|&$\deg$&\verb|\lg|&$\lg$&\verb|\projlim|&$\projlim$\\
\verb|\arcsin|&$\arcsin$&\verb|\det|&$\det$&\verb|\lim|&$\lim$&\verb|\sec|&$\sec$\\
\verb|\arctan|&$\arctan$&\verb|\dim|&$\dim$&\verb|\liminf|&$\liminf$&\verb|\sin|&$\sin$\\
\verb|\arg|&$\arg$&\verb|\exp|&$\exp$&\verb|\limsup|&$\limsup$&\verb|\sinh|&$\sinh$\\
\verb|\cos|&$\cos$&\verb|\gcd|&$\gcd$&\verb|\ln|&$\ln$&\verb|\sup|&$\sup$\\
\verb|\cosh|&$\cosh$&\verb|\hom|&$\hom$&\verb|\log|&$\log$&\verb|\tan|&$\tan$\\
\verb|\cot|&$\cot$&\verb|\inf|&$\inf$&\verb|\max|&$\max$&\verb|\tanh|&$\tanh$\\
\verb|\coth|&$\coth$&\verb|\injlim|&$\injlim$&\verb|\min|&$\min$&&\\
\verb|\csc|&$\csc$&\verb|\ker|&$\ker$&\verb|\Pr|&$\Pr$&&\\
\verb|\varinjlim|&$\varinjlim$&\verb|\varliminf|&$\varliminf$&&&&\\
\verb|\varprojlim|&$\varprojlim$&\verb|\varlimsup|&$\varlimsup$&&&&
\end{tabular}
\end{center}

还有一个命令叫做 \verb|\operatorname|,在数学公式中使用 \verb|\operatorname{abc}| 与使用由\\ \verb|\DeclareMathOperator| 定义的 \verb|\abc| 是等价的,这对于构造更加复杂的符号或者其它的目的可能有时会有用。变体 \verb|\operatorname*| 将上下标置于‘limits’位置。
\cprotect\section{\verb|\mod| 及其相关符号}
命令 \verb|\mod|、\verb|\bmod|、\verb|\pmod|、\verb|\pod| 用于处理“mod”符号的特殊间距约定。\verb|\bmod| 和 \verb|\pmod| 可以在\LaTeX{} 中使用,但使用\pkg{amsmath}包时,\verb|\pmod| 用于非行间模式时,它的间距将调整为较小的值。 \verb|\mod| 和 \verb|\pod| 是某些作者首选的 \verb|\pmod| 变体:\verb|\mod| 省略了括号,而 \verb|\pod| 省略“mod”并保留括号。
\begin{tcblisting}{}
\begin{equation}
\gcd(n,m\bmod n);\quad x\equiv  y\pmod b;
\quad x\equiv y\mod c;\quad x\equiv y\pod d
\end{equation}
\end{tcblisting}
