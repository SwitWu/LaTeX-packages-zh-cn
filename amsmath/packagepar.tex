% !Mode:: "TeX:Hard:UTF-8"

\chapter{\pkg{amsmath}宏包的可选项}

\pkg{amsmath}宏包包含如下可选项:

\verb|centertags|(默认)对于包含 \verb|split| 环境的公式,  将公式编号垂直放置在公式总高度的中间。

\verb|tbtags| 顶部或底部标签:对于包含 \verb|split| 环境的公式,  如果编号在右边(或左边),  则将公式编号与最后一个(或第一行)放在同一水平。

\verb|sumlimits|(默认)将行间公式的求和符号的上标和下标分别放在上方和下方。 这个选项也影响了其他同种类型的符号——$\displaystyle{\prod,\coprod,\bigotimes,\bigoplus}$等,但是不包含积分符号(见下)。

\verb|nosumlimits| 总是将求和符号的上标和下标放在求和符号的旁边,  包括行间公式。

\verb|intlimits| 类似于 \verb|sumlimits|,  但是针对积分符号的。

\verb|nointlimits|(默认)是 \verb|intlimits| 的反义。

\verb|namelimits|(默认)类似于 \verb|sumlimits|,  但是对某些运算符例如$\det,\inf,\max,\min$等出现在行间公式中时,  通常在下面放置下标。

\verb|nonamelimits| 是 \verb|namelimits| 的反义。

\verb|alignedleftspaceyes|

\verb|alignedleftspaceno|

\verb|alignedleftspaceyesifneg|

要使用这些宏包选项,将选项名称放在 \verb|\usepackage| 的可选参数中——例如

\verb|\usepackage[intlimits]{amsmath}|

对于 \hologo{AmS} 文档类和任何其它预加载\pkg{amsmath}宏包的文档类,想要的选项都必须通过 \verb|\documentclass| 来指定——例如

\verb|\documentclass[intlimits,tbtags,reqno]{amsart}|。

\pkg{amsmath}包还识别通常通过 \verb|\documentclass| 命令选择(隐式或显式)的以下选项,  因此不需要在 \verb|\usepackage{amsmath}| 的选项列表中重复声明。

\verb|leqno| 将公式编号放在左边。

\verb|reqno| 将公式编号放在右边。

\verb|fleqn| 将公式放在距离左边缘固定缩进的位置而不是放在中间。

有三个选项被添加来控制 \verb|aligned| 和 \verb|gathered| 环境左侧的空格。 在2017发布之前,  这些结构的左侧而不是右侧添加了一个较小的空格。 这似乎是操作的一个意外特征,  一般是通过在环境前面加前缀 \verb|\!| 来修正。

新的缺省行为旨在确保在大多数情况下,  环境中没有添加很小的空格,  并且使用 \verb|\!\begin{aligned}| 的现有文档继续像以前一样有效。

\verb|alignedleftspaceyes| 通常在 \verb|aligned| 和 \verb|gathered| 的左侧添加 \verb|\,|。

\verb|alignedleftspaceno| 通常在 \verb|aligned| 和 \verb|gathered| 的左侧不加 \verb|\,|。

\verb|alignedleftspaceyesifneg| 只在环境前面添加了负距离以后才添加 \verb|\,| (新的缺省行为)。