% !Mode:: "TeX:Hard:UTF-8"

\chapter{行间公式}
\section{简介}
\pkg{amsmath}提供了很多除\LaTeX{} 自带之外的行间公式结构。 增加的集合包括:
\begin{center}\ttfamily
\begin{tabular}{*{4}{p{5em}}}
equation&equation*&align&align*\\
gather&gather*&alignat&alignat*\\
multline&multline*&flalign&flalign*\\
split&&&
\end{tabular}
\end{center}
(虽然标准的 \verb|eqnarray| 环境仍然可用,但是最好还是用 \verb|align| 或者 \verb|equation+split|。在 \verb|eqnarray| 环境中,关系符号周围的间距不是首选的数学间距,并且与其他环境中出现的间距不一致。此环境中的长行可能会导致错误放置或套印的公式编号。此环境也不支持使用定理宏包提供的 \verb|\qed| 和 \verb|\qedhere|。 )

除了 \verb|split| 外,每个环境都至少有加星号和不加星号两种形式,其中未加星号的使用 \LaTeX 的公式计数器自动编号,您可以将 \verb|\notag| 放在任一行结束之前的位置来抑制该编号。\verb|\notag| 不应该出现在行间公式外,因为它会打乱编号。您还可以使用您自己的 \verb|\tag{label}| 来覆盖数字,这里的 \verb|label| 表示任意文本,比如 \verb|$*$| 或者 \verb|ii| 来给公式“编号”。一个 tag 可以通过 \verb|\tag{\ref{<label>}<modifier>}| 来引用一个不同的已经标签过的行间公式,其中 \verb| <modifier>| 为可选项。 如果您使用了 \pkg{hyperref} 包,则使用 \verb|\ref*|;使用带星号的形式 \verb|\ref| 可防止引用包含嵌套链接的已修改标签,使其无法链接到行间公式。

还有一个 \verb|\tag*| 命令,它可以使您提供的文本按字面排版,而不在其周围添加括号。在所有 \pkg{amsmath} 对齐结构的未编号版本中,也可以使用 \verb|\tag| 和 \verb|\tag*|。 在示例文件 {\ttfamily testmath.tex} 和 {\ttfamily subeqn.tex} 中可以找到使用 \verb|\tag| 标记的一些示例。

\verb|split| 环境是一种特殊的从属形式,只在其它某个环境内部使用,但不能在 \verb|multline| 内使用。\verb|split| 仅支持一个对齐(\&)列;如果需要更多,则应使用 \verb|aligned| 或 \verb|alignedat|。\verb|split| 结构的宽度是整行宽度。

在进行对齐的结构中(\verb|split|、\verb|align| 及其变体),关系符号前面有一个\&,但后面没有,不同于 \verb|eqnarray|。 将\& 放在关系符号后面会干扰正常间距;它必须放在前面。

在所有的多行环境中,行之间用 \verb|\\| 分割,   \verb|\\| 不应该用于结束最后一行,否则在最后一行下方会导致不想要的额外的垂直间距。

在\emph{所有}的数学环境中(行内或行间),\emph{不}允许使用空行(相当于 \verb|\par|),
否则会导致错误。

不同行间公式的比较(垂直线表示标准边线)

\begin{listing}
\begin{equation*}
a=b
\end{equation*}
\end{listing}

\begin{listing}
\begin{equation}
a=b
\end{equation}
\end{listing}

\begin{listing}
\begin{equation}\label{xx}
\begin{split}
a={}&b+c-d\\
&+e-f\\
={}&g+h\\
={}&i
\end{split}
\end{equation}
\end{listing}





\begin{listing}
\begin{multline}
a+b+c+d+e+f\\
+i+j+k+l+m+n
\end{multline}
\end{listing}

\begin{listing}
\begin{gather}
a_1=b_1+c_1\\
a_2=b_2+c_2-d_2+e_2
\end{gather}

\end{listing}



\begin{listing}
\begin{align}
a_1=b_1+c_1\\
a_2=b_2+c_2-d_2+e_2
\end{align}
\end{listing}



\begin{listing}
\begin{align}
a_{11}&=b_{11}&a_{12}&=b_{12}\\
a_{21}&=b_{21}&a_{22}&=b_{22}+c_{22}
\end{align}

\end{listing}



\begin{listing}
\begin{flalign*}
a_{11}+b_{11}&=c_{11}
&a_{12}&=b_{12}\\
b_{21}&=c_{21}
&a_{22}&=b_{22}+c_{22}
\end{flalign*}

\end{listing}

\begin{listing}
\begin{flalign}
a_{11}+b_{11}&=c_{11}&a_{12}&=b_{12}\\
b_{21}&=c_{21}&a_{22}&=b_{22}+c_{22}
\end{flalign}
\end{listing}

\verb|multline| 环境是 \verb|equation| 环境的变体,  它适用于一行无法放下的公式。  \verb|multline| 的第一行在最左边,  最后一行在最右边,  除了在两边有缩进量 \verb|\multlinegap|,  中间的任何其他行都会在行间公式宽度内独立居中(除非 \verb|fleqn| 选项有效)。

类似于 \verb|equation|,\verb|multline| 只有一个公式编号(因此,不应在单个行中标记 \verb|\notag|)。公式编号放在最后一行(\verb|reqno| 选项)或第一行(\verb|legno| 选项)上;\verb|multline| 不支持 \verb|split| 的垂直居中。

可以用 \verb|\shoveleft| 和 \verb|\shoveright| 使得中间行强制居左或居右。 这两个命令以整行作为参数,  直到最后的 \verb|\\| 但不包含 \verb|\\|。 例如
\begin{tcblisting}{}
\begin{multline}
\framebox[.65\columnwidth]{A}\\
\framebox[.5\columnwidth]{B}\\
\shoveright{\framebox[.55\columnwidth]{C}}\\
\framebox[.65\columnwidth]{D}
\end{multline}
\end{tcblisting}

\verb|\multlinegap| 的值可以通过通常的\LaTeX 命令 \verb|\setlength| 或 \verb|\addtolength| 修改。

\section{对齐的换行公式}
类似于 \verb|multline|,\verb|split| 环境是针对单个长公式,超出一行而不得不分成多行公式。然而,与 \verb|multline| 不同的是,\verb|split| 环境提供了行之间的对齐,使用\&标记对齐点。与其他\pkg{amsmath}的公式结构不同,\verb|split| 环境不提供编号,因为它只能在其他行间公式结构中使用,一般是 \verb|equation|、\verb|align| 或者 \verb|gather| 等提供编号的环境。 例如
\begin{tcblisting}{}
\begin{equation}\label{e:barwq}\begin{split}
H_c&=\frac{1}{2n} \sum^n_{k=0}(-1)^{k}(n-k)^{p-2}
  \sum_{k_1+\dots+k_p=1}\prod^p_{i=1} \binom{n_i}{k _i}\\
  &\quad\cdot[(n-1)-(n_i-k_i)]^{n_i-k_i}\cdot
   \left[(n-k)^2-\sum^p_{i=1}(n_i-k_i)^2\right].
\end{split}\end{equation}
\end{tcblisting}


除了像 \verb|\label|这样生成不可见实体的命令,  \verb|split| 结构应该构成整个封闭结构的整个主体。
\section{不带对齐的公式组}
\verb|gather| 环境用于不需要对齐的连续公式组,每一个公式分别在整行宽度内居中放置。 \verb|gather| 中的公式被命令 \verb|\\| 分开,\verb|gather| 中的每一个公式都可以包含一个\\ \verb|\begin{split}...\end{split}| 结构,例如
\begin{tcblisting}{listing only}
\begin{gather}
first equation\\
\begin{split}
second & equation\\
       & on two lines
\end{split}\\
third equation
\end{gather}
\end{tcblisting}

\section{带有多列对齐的公式组}
\verb|align| 环境用于两个或更多需要垂直对齐的公式,一般的像等于号之类的二元算符是被对齐的。

要使多个公式列并排,请使用额外的 \verb|&| 分隔各列。
\begin{tcblisting}{}
\begin{align}
x&=y          & X&=Y         & a&=b+c\label{eq:C}\\
x'&=y'        & X'&=Y'       & a'&=b\label{eq:D}\\
x+x'&=y+y'    & X+X'&=Y+Y'   & a'b&=c'b
\end{align}
\end{tcblisting}

公式上的逐行注释可以通过在 \verb|align| 环境中机智地应用 \verb|\text| 来完成。
\begin{tcblisting}{}
\begin{align}
x& = y_1-y_2+y_3-y_5+y_8-\dots
                     && \text{by \eqref{eq:C}}\\
 & = y'\circ y^*     && \text{by \eqref{eq:D}}\\
 & = y(0) y'         && \text {by Axiom 1.}
\end{align}
\end{tcblisting}

变体环境 \verb|alignat| 允许显式指定等式之间的水平空间。 此环境接受一个参数,  即“公式列数”(右左对齐列对的数目:参数是对齐的数目):计算任意行中的最大数目的\&,  加1并除以2。
\begin{tcblisting}{}
\begin{alignat}{2}
x& = y_1-y_2+y_3-y_5+y_8-\dots
                     &\quad& \text{by \eqref{eq:C}}\\
 & = y'\circ y^*     && \text{by \eqref{eq:D}}\\
 & = y(0) y'         && \text {by Axiom 1.}
\end{alignat}
\end{tcblisting}

\verb|flalign| 环境(“full length alignment”)将公式列之间的空间拉伸到可能的最大宽度,  只在公式编号的空白处留下足够的空间。
\begin{tcblisting}{}
\begin{flalign}
x&=y                & X&=Y\\
x'&=y'              & X'&=Y'\\
x+x'&=y+y'          & X+X'&=Y+Y'
\end{flalign}
\end{tcblisting}

\begin{tcblisting}{}
  \begin{flalign*}
x&=y                & X&=Y\\
x'&=y'              & X'&=Y'\\
x+x'&=y+y'          & X+X'&=Y+Y'
\end{flalign*}
\end{tcblisting}

\section{对齐构建区块}
类似于 \verb|equation|,  多公式环境 \verb|gather|,  \verb|align| 和 \verb|alignat| 旨在构建一个全长的结构。 这意味着,  例如,  我们无法在整个结构旁边添加括号。 但是变体环境 \verb|gathered|,  \verb|aligned| 和 \verb|alignedat| 使得内容的长度就是它的实际长度,  因此可以用作包含表达式的组件。 例如
\begin{tcblisting}{}
\begin{equation*}
\left.\begin{aligned}
B'&=-\partial\times E,\\
E'&=\partial\times B-4\pi j,
\end{aligned}
\right\}
\qquad \text{Maxwell 方程组}
\end{equation*}
\end{tcblisting}

类似于 \verb|array| 环境,这些带有 \verb|-ed| 的变体也接受可选参数 \verb|[t]|,\verb|[b]| 或者默认的 \verb|[c]| 来指定垂直对齐方式。 为协调起见,不要在可选项前添加空格或断行。

像下面的``Cases"结构在数学中时很常见的:
\begin{equation}
P_{r-j}=\begin{cases}
0,&\text{如果}r-j\text{为奇数}\\
r!(-1)^{(r-j)/2},&\text{如果}r-j\text{为偶数}
\end{cases}
\end{equation}
而在\pkg{amsmath}包中有一个 \verb|cases| 环境可以很简单地实现:
\begin{tcblisting}{listing only}
P_{r-j}=\begin{cases}
0,&\text{如果}r-j\text{为奇数}\\
r!(-1)^{(r-j)/2},&\text{如果}r-j\text{为偶数}
\end{cases}
\end{tcblisting}
\noindent 这里注意 \verb|\text{blabla}| 的使用以及嵌套的数学公式。  \verb|cases| 环境的字体设置为 \verb|\textstyle|,   如果需要 \verb|\displaystyle| ,  一定要直接声明;在 \pkg{mathtools} 包中提供的 \verb|dcases| 环境可以直接实现。

\verb|-ed| 和 \verb|cases| 环境只能放在一个封闭的数学环境中,  可以是行内的 \verb|$...$|,  也可以是任意的行间环境。

\section{调整标签的位置}
在多行公式中放公式编号可能是一个相当复杂的问题。\pkg{amsmath}宏包的环境尽量避免在公式内容上套印公式编号,如有必要,可将该编号向下或向上移动到单独的行。

精确计算一个公式的轮廓的困难有时会导致数字的移动看起来不正确。\verb|\raisetag| 命令来调整当前公式编号的垂直位置,如果它已从其正常位置移开,要将特定编号向上移动6pt,使用 \verb|\raisetag{6pt}|。(在行间公式结束时,也向上移动此公式后的文本。)这种调整就像换行符和分页符一样是很好的调整,因此在您的文档接近定稿之前,应该先撤消。或者,您最终可能会重复多次微调以跟上文档内容的变化。
\section{垂直间距与多行公式的换行}
您可以使用 \verb|\\[距离]| 命令在所有\pkg{amsmath}的行间公式环境中获取行与行之间的额外垂直空间,这在\LaTeX 中很常见。不要在 \verb|\\| 和 \verb|[| 之间键入空格;仅对于\pkg{amsmath}定义的行间环境,该空格被解释为表示带括号的内容是文档内容的一部分。

当使用\pkg{amsmath}宏包时,公式行之间的分页符通常是不允许的;其原理是,此类内容中的分页符应引起作者的个人注意。要在特定的行间公式中获取单独的分页符,可以使用\verb|\displaybreak|。最好将 \verb|\displaybreak| 命令放在要生效的位置 \verb|\\| 之前。像\LaTeX{} 的 \verb|\pagebreak| 一样,\verb|\displaybreak| 采用0到4之间的可选参数来表示分页的可取性。\verb|\displaybreak[0]| 表示“允许在不鼓励中断的情况下中断此处”:不带可选参数的 \verb|\displaybreak| 与 \verb|\displaybreak[4]| 一样,表示强制分页。

如果想随处使用分页符,即使在多行公式中间,可以将 \verb|\allowdisplaybreaks[1]| 放在导言区。可选参数1--4可用于更精细的控制:[1]表示允许分页符,但尽可能避免换页。参数2、3、4表示允许性增加。 当使 \verb|\allowdisplaybreaks| 启用行间公式分页时,可以像往常一样,使用 \verb|\\*| 命令禁止给定行后的分页中断。

\textbf{注}:一些公式环境把内容包装在了一个不可断行的盒子中,那么此时 \verb|\displaybreak| 和 \verb|\allowdisplaybreaks| 对它们都不起作用,这些包括 \verb|split|、\verb|aligned|、\verb|gathered| 和 \verb|alignedat|。
\section{中断行间公式}
\verb|\intertext| 命令用于在多行公式中间插入一或两行文字(参见第六章的 \verb|\text| 命令)。它的显著特点是保持对齐,如果您只是简单地结束行间公式然后再重新使用行间公式,那么这种对齐就不会出现。\verb|\intertext| 只能出现在 \verb|\\| 或 \verb|\\*| 命令之后。注意下面例子中的“和”的位置。

\begin{tcblisting}{}
\begin{align}
A_1&=N_0(\lambda;\Omega')-\phi(\lambda;\Omega),\\
A_2&=\phi(\lambda;\Omega')-\phi(\lambda;\Omega),\\
\intertext{和}
A_3&=\mathcal{N}(\lambda;\omega).
\end{align}
\end{tcblisting}

\pkg{mathtools} 包提供了 \verb|\shortintertext| 命令用于插入少量的文字;它比 \verb|\intertext| 使用更少的垂直距离,当公式编号在右边时尤其有效。

\section{公式编号}
\subsection{编号秩序}
在\LaTeX{} 中,如果您想让公式随着 \verb|section| 进行编号——即在第一节和第二节中分别让公式编号形式为(1.1)、(1.2)、\ldots、(2.1)、(2.2)、\ldots 等等——重新定义 \verb|\theequation| 即可:
\begin{verbatim}
\renewcommand{\theequation}{\thesection.\arabic{equaiton}}
\end{verbatim}
这是非常有效的方法,  但是公式编号计数器不会随着新的章节的开始而归零,  除非您使用 \verb|\setcounter| 进行重置。 此时可以在导言区加入
\begin{tcblisting}{listing only}
\makeatletter
\@addtoreset{equation}{section}
\makeatother
\end{tcblisting}

但是 \pkg{amsmath} 包提供了更好的命令 \verb|\numberwithin|,  使得公式编号与章节编号绑定,  且随着新一节的开始,  公式编号自动归零,  输入
\begin{verbatim}
\numberwithin{equation}{section}
\end{verbatim}
正如此命令的名字所示,  \verb|\numberwithin| 可以适用于任何计数器,  不仅仅是 \verb|equation| 计数器。
\subsection{编号公式的交叉引用}
\verb|\eqref|使得公式的交叉引用更加简单,  它还在公式旁边自动增加了括号。 即如果 \verb|\ref{abc}| 得到3.2,  则 \verb|\eqref{abc}| 会得到(3.2)。 由 \verb|eqref| 引用的公式不论内容如何,  它旁边的括号都是直立形状的。
\subsection{附属编号序列}
\pkg{amsmath}还提供了一个包装环境,  \verb|subequations|,  以便在特定的对齐环境或类似环境中使用从属编号方案对公式进行编号。 例如
\begin{verbatim}
\begin{subequations}
...
\end{subequations}
\end{verbatim}
使得在此部分的所有公式编号形式变为(4.9a)(4.9b)(4.9c)...,  如果它的上一个公式编号为(4.8)。 在 \verb|\begin{subequations}| 后紧跟的 \verb|\label| 命令会生成一个带括号的4.9的引用标签,  而不是(4.9a)。 \verb|subequations| 使用的计数器组是 \verb|parentequation| 和  \verb|equation|,  而 \verb|\addtocounter|,  \verb|\setcounter|, \verb|\value| 等,  可以像往常一样应用于这些计数器。 要获得任何除小写字母之外的从属编号,  使用标准的\LaTeX 改变编号形式\cite{3},  \S6.3,  \S C.8.4。 例如像下面一样重定义\verb|\theequation|来得到罗马数字。
\begin{verbatim}
\begin{subequations}
\renewcommand{\theequation}{\theparentequation \roman{equation}}
...
\end{subequations}
\end{verbatim}

\subsection{编号风格}
默认的公式编号设置为\verb|\normalfont|,  这意味着在粗体部分标题中,  粗体是被抑制的。 解决方法是使用 \verb|(\ref{...})| 而不用 \verb|\eqref{...}|。

如果行间公式编号被指定了较小的类型,  则公式编号的大小也会变小。 默认的大小可以通过在导言区加入以下补丁得到确保:
\begin{verbatim}
\makeatletter
\renewcommand{\maketag@@@}[1]{\hbox{\m@th\normalsize\normalfont#1}}%
\makeatother
\end{verbatim}
(这个修正或许会放在\pkg{amsmath}以后的版本中。 )