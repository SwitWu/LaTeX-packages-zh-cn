% !Mode:: "TeX:Hard:UTF-8"

\chapter{使用数学字体}
\section{简介}
对于\LaTeX 的更综合性的信息,参考\LaTeX{} 的字体说明(\verb|fntguide.tex|)或者\LaTeX 指南\cite{4}。\LaTeX{} 的基本数学字体命令有 \verb|\mathbf|、\verb|\mathrm|、\verb|\mathcal|、\verb|\mathsf|、\verb|\mathtt| 和 \verb|\mathit|。额外的数学命令,比如\verb|\mathbb| 用于黑板字体,  \verb|\mathfrak| 用于Fraktur,\verb|\mathscr| 用于Euler手记,这些可以分别通过加载宏包\pkg{amsfonts}和\pkg{euscript}得到。
\section{推荐使用的数学字体命令}
如果您需要在您的文档中频繁使用数学字体命令,您可能希望它们能有简写名称,比如 \verb|\mb| 而不是 \verb|\mathbf|。当然,这并不是难事,只要您使用适当的 \verb|\newcommand| 命令就可以定义简写了。但是在\LaTeX{} 中,提供短名称实际上对作者来说是帮倒忙,因为这会隐藏一个更好的选择:使用来自基本数学对象的名称而不是来自用于区分对象的字体名称来自定义您的命令名。例如,如果您使用粗体表示向量,那么如果定义一个 \verb|vector| 命令而不是一个 \verb|math-bold| 命令,从长远来看您会得到更好的服务。
\begin{verbatim}
\newcommand{\vect}[1]{\mathbf{#1}}
\end{verbatim}
您可以输入\verb|\vect{a}+\vert{b}| 来得到$\mathbf a+\mathbf b$。如果您决定在接下来的几个月内将粗体用作某些其它目的,而向量使用一个小箭头来表示,您只需要修改 \verb|\vect| 的命令即可,否则您不得不通过文档来替换所有出现的 \verb|\mathbf|,甚至需要检查每一个命令看它是否确实是一个向量的例子。

用于为特定字体的不同字母分配不同的命令名也是有用的。
\begin{verbatim}
\DeclareSymbolFont{AMSb}{U}{msb}{m}{n}% 或者使用amsfonts包
\DeclareMathSymbol{\C}{\mathalpha}{AMSb}{''43}
\DeclareMathSymbol{\R}{\mathalpha}{AMSb}{''52}
\end{verbatim}

这些声明定义了命令\verb|\C| 和 \verb|\R|,  是来自\verb|AMSb| 数学符号字体的黑板字体。 如果您在文档中更喜欢实数或复数,  您会发现此方法比定义一个\verb|\field| 命令,  然后输入\verb|\field{C}|,  \verb|\field{R}| 更方便。 但是为了获得最大的灵活性和控制权,  可以定义这样一个 \verb|\field| 命令,  然后根据该命令定义\verb|\C| 和 \verb|\R|:
\begin{verbatim}
\usepackage{amsfonts}%获取\mathbb 字母表
\newcommand{\field}[1}{\mathbb{#1}}
\newcommand{\C}{\field{C}}
\newcommand{\R}{\field{R}}
\end{verbatim}
\section{粗体数学符号}
\verb|\mathbf| 命令通常用于在数学中输出粗体拉丁字母,但是对于绝大多数其它类型的数学符号都是无效的。 或者它的效果依赖于正在使用的一组数学字体。例如,输入
\begin{verbatim}
\Delta \mathbf{\Delta}\mathbf{+}\delta \mathbf{\delta}
\end{verbatim}
将得到$\Delta \mathbf{\Delta}+\delta\delta$,  \verb|\mathbf| 对加号和小写的delta无效。 因此\pkg{amsmath}提供了两个额外的命令:\verb|\boldsymbol| 和 \verb|\pmb|,可以用来对其他的数学符号生效。\verb|\boldsymbol| 可以用于\verb|\mathbf| 不起作用的数学符号(当且仅当您当前的数学字体集包括该符号的粗体版本)。对于没有由您的数学字体集提供的真正粗体版本的任何数学符号,可以使用 \verb|\pmb| 作为最后的手段:“pmb”代表“穷人的粗体”,该命令通过用轻度平板印刷打印符号的多个副本来工作。输出的质量较差,尤其是对于包含细线笔画的符号。当使用\LaTeX{} 数学字体的标准默认集(Computer Modern)时,可能需要 \verb|\pmb| 的符号只有大型运算符符号,扩展定界符符号或\pkg{amssymb}包\cite{8}提供的额外数学符号。

下面的公式展示了一些可能的结果:
\begin{tcblisting}{}
\[
A_\infty+ \pi A_0
\sim \mathbf{A}_{\boldsymbol{\infty}} \boldsymbol{+}
  \boldsymbol{\pi} \mathbf{A}_{\boldsymbol{0}}
\sim\pmb{A}_{\pmb{\infty}} \pmb{+}\pmb{\pi} \pmb{A}_{\pmb{0}}
\]
\end{tcblisting}

如果您只想使用 \verb|\boldsymbol| 命令而不加载完整的\pkg{amsmath}包,那么推荐\pkg{bm}宏包(这是一个标准的\LaTeX{} 宏包,而不是一个 \hologo{AmS} 宏包,如果您有1997或者更新的\LaTeX{} 版本,这个包已经包含在内了。)。
\section{斜体希腊字母}
对于斜体的大写希腊字母,提供了以下命令
\begin{center}
\begin{tabular}{rlrl}
\verb|\varGamma|&$\varGamma$&\verb|\varSigma|&$\varSigma$\\
\verb|\varDelta|&$\varDelta$&\verb|\varUpsilon|&$\varUpsilon$\\
\verb|\varTheta|&$\varTheta$&\verb|\varPhi|&$\varPhi$\\
\verb|\varLambda|&$\varLambda$&\verb|\varPsi|&$\varPsi$\\
\verb|\varXi|&$\varXi$&\verb|\varOmega|&$\varOmega$\\
\verb|\varPi|&$\varPi$&&
\end{tabular}
\end{center}