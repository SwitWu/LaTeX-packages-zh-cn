% \iffalse meta-comment
%
% Copyright 1996-2010 by Hideo Umeki <latexgeometry@gmail.com>
% Copyright 2018  Hideo Umeki and David Carlisle
%
% LaTeX Package: Geometry
% -----------------------
%
% This work may be distributed and/or modified under the conditions of
% the LaTeX Project Public License, either version 1.3c of this license
% or (at your option) any later version. The latest version of this
% license is in
%    http://www.latex-project.org/lppl.txt
% and version 1.3c or later is part of all distributions of LaTeX
% version 2005/12/01 or later.
%
% This work is "maintained" (as per the LPPL maintenance status)
% by David Carlisle
%
% This work consists of the files geometry.dtx and
% the derived files: geometry.{sty,ins,drv}, geometry-samples.tex.
%
% * Distribution
%    CTAN: macros/latex/contrib/geometry/README.md
%    CTAN: macros/latex/contrib/geometry/changes.txt
%    CTAN: macros/latex/contrib/geometry/geometry.ins
%    CTAN: macros/latex/contrib/geometry/geometry.dtx
%    CTAN: macros/latex/contrib/geometry/geometry.pdf
%    CTAN: macros/latex/contrib/geometry/geometry-de.dtx
%    CTAN: macros/latex/contrib/geometry/geometry-de.pdf
%
% * See README for installation.
%
%<*ignore>
\begingroup
  \def\x{LaTeX2e}
\expandafter\endgroup
\ifcase 0\ifx\install y1\fi\expandafter
         \ifx\csname processbatchFile\endcsname\relax\else1\fi
         \ifx\fmtname\x\else 1\fi\relax
\else\csname fi\endcsname
%</ignore>
%<package|driver>\NeedsTeXFormat{LaTeX2e}
%<package>\ProvidesPackage{geometry}
%<package>  [2020/01/02 v5.9 Page Geometry]
%<*install>
\input docstrip.tex
\Msg{************************************************************************}
\Msg{* Installation}
\Msg{* Package: geometry 2020/01/02 v5.9 Page Geometry}
\Msg{************************************************************************}

\keepsilent
\askforoverwritefalse
\preamble

Copyright (C) 1996-2010
by Hideo Umeki <latexgeometry@gmail.com>
Copyright (C) 2018-2020
by Hideo Umeki and David Carlisle https://github.com/davidcarlisle/geometry

This work may be distributed and/or modified under the conditions of
the LaTeX Project Public License, either version 1.3c of this license
or (at your option) any later version. The latest version of this
license is in
   http://www.latex-project.org/lppl.txt
and version 1.3c or later is part of all distributions of LaTeX
version 2005/12/01 or later.

This work is "maintained" (as per the LPPL maintenance status)
by David Carlisle.

This work consists of the files geometry.dtx and
the derived files: geometry.{sty,ins,drv}, geometry-samples.tex.

\endpreamble

\generate{%
  \file{geometry.ins}{\from{geometry.dtx}{install}}%
  \file{geometry.drv}{\from{geometry.dtx}{driver}}%
  \usedir{tex/latex/geometry}%
  \file{geometry.sty}{\from{geometry.dtx}{package}}%
  \file{geometry.cfg}{\from{geometry.dtx}{config}}%
  \file{geometry-samples.tex}{\from{geometry.dtx}{samples}}%
}

\obeyspaces
\Msg{************************************************************************}
\Msg{*}
\Msg{* To finish the installation you have to move the following}
\Msg{* file into a directory searched by LaTeX:}
\Msg{*}
\Msg{* \space\space geometry.sty}
\Msg{*}
\Msg{* To produce the documentation run the file `geometry.drv'}
\Msg{* through LaTeX.}
\Msg{*}
\Msg{* Happy TeXing!}
\Msg{*}
\Msg{************************************************************************}
\endbatchfile
%</install>
%<*ignore>
\fi
%</ignore>
%<*driver>
\ProvidesFile{geometry.drv}
\documentclass{ltxdoc}
\usepackage{graphicx}
\usepackage[a4paper, hmargin=2.5cm, vmargin=1cm,
        includeheadfoot]{geometry}
\DeclareRobustCommand\XeTeX{%
      X\lower.5ex\hbox{\kern-.07em\reflectbox{E}}%
      \kern-.15em\TeX}
\DeclareRobustCommand\XeLaTeX{%
      X\lower.5ex\hbox{\kern-.07em\reflectbox{E}}%
      \kern-.15em\LaTeX}


\usepackage{float} %%%% 防止表格等浮动
%%%%%%%%%%%%% 以下设置中文字体 %%%%%%%%%%%%%%%%%%%%%%%%%%%%%%%%%%%%%%%%%
\usepackage{xeCJK}  %%

\setCJKfamilyfont{Heiti}{Source Han Sans Regular} %%%% 自定义\Heiti命令,显示思源黑体,用于标题页标题的中文部分
\newcommand{\Heiti}{\CJKfamily{Heiti}} %%%% 自定义\Heiti命令,显示思源黑体,用于标题页标题的中文部分

\setCJKfamilyfont{heiti}{Source Han Sans Light} %%自定义\heiti命令,显示思源黑体,用于正文的章节标题
\newcommand{\heiti}{\CJKfamily{heiti}} %%自定义\heiti命令,显示思源黑体,用于正文的章节标题

\setCJKfamilyfont{songti}{思源宋体 CN Light}  %%%% 自定义\songti命令,显示思源宋体,用于正文
\newcommand{\songti}{\CJKfamily{songti}} %%%% 自定义\songti命令,显示思源宋体,用于正文

\setCJKfamilyfont{heitixt}{思源黑体_CN_LightItalic.otf}  %%%% 自定义\heitixt命令,显示思源黑体斜体
\newcommand{\heitixt}{\CJKfamily{heitixt}} %%%% 自定义\heitixt命令,显示思源黑体斜体

\setCJKmainfont{思源宋体 CN Light} %%%% 设置中文的主字体为思源宋体 CN Light
\setmainfont{Source Serif Pro} %%%% 设置英文的主字体为Source Serif Pro,最好看
%%%%\setmainfont{Source Han Serif SC} %%%% 设置英文的主字体为Source Han Serif SC
%%%%%%\setmainfont{Times New Roman} %%%% 设置英文的主字体为Times New Roman

\setCJKfamilyfont{kaiti}{KaiTi} %%设置中文字体楷体,用于强调
\newcommand{\kaiti}{\CJKfamily{kaiti}} %%设置中文字体楷体,用于强调
%%%%%%%%%%%%% 以上设置中文字体 %%%%%%%%%%%%%%%%%%%%%%%%%%%%%%%%%%%%%%%%%
\usepackage{fontspec}

%%% 以下输入带圈的数字,调用时的命令:如 \char"2469 生成 ⑩ %%%%
%%详参目录中的“latex 如何添加圆圈数字? - Tsingke - 博客园.mhtml”%%%%
\xeCJKDeclareCharClass{CJK}{%
  "24EA,        % ⓪
  "2460->"2473, % ①–⑳
  "3251->"32BF, % ㉑–㊿
  "24FF,        % ⓿
  "2776->"277F, % ❶–❿
  "24EB->"24F4  % ⓫–⓴
}
%%% 以上输入带圈的数字,调用时的命令:如 \char"2469 生成 ⑩ %%%%

%%%%%%%%%%%%% 以下设置中文版式 %%%%%%%%%%%%%%%%%%%%%%%%%%%%%%%%%%%%%%%%%
\usepackage{indentfirst} %%% 首行缩进
\setlength{\parindent}{2em} %%% 缩进2个字符(中文为2个字)
\linespread{1.5} %%% 设置行间距
%%%%%%%%%%%%% 以上设置中文版式 %%%%%%%%%%%%%%%%%%%%%%%%%%%%%%%%%%%%%%%%%
\usepackage{changepage} %%%用于整体缩进,\begin{adjustwidth}{2cm}{1cm}

%%%%%%% 以下在 tabular 表格中定制 横线如\hlinew{1.2pt} %%%%%%
\makeatletter
\def\hlinew#1{%
\noalign{\ifnum0=`}\fi\hrule \@height #1 \futurelet
\reserved@a\@xhline}
\makeatother%
%%%%%%% 以上在 tabular 表格中定制 横线如\hlinew{1.2pt} %%%%%%

%%%%%%% 以下自定义脚注 %%%%%%%%%%%%%%%%%%%%%%%%%%%%%%%%%%%%
\setlength{\footnotesep}{0.5cm} %%%设置几第脚注之间的距离
\setlength{\skip\footins}{2.0em} %%%设置脚注与正文之间的距离
%%\renewcommand\footnoterule{} %%%定义脚注线为空
\renewcommand\footnoterule{
     \kern -3pt                         % This -3 is negative
     \hrule width 0.6\textwidth height 0.6pt % of the sum of this 1
     \kern 2pt} %%%
%%%%%%% 以上自定义脚注 %%%%%%%%%%%%%%%%%%%%%%%%%%%%%%%%%%%%

\renewcommand{\contentsname}{\centerline{\heiti {\Large 目\ \ \ 录}}}   %%% 在{document}后面加入该命令,将"contents"变成“目  录”
%%%\renewcommand{\thepart}{第{\Roman{part}}部分}
\renewcommand{\refname}{\heiti 参考文献}
\renewcommand{\figurename}{\Heiti 图}
\renewcommand{\tablename}{\Heiti 表}
\renewcommand{\abstractname}{\heiti {\Large 摘\ 要}}
\renewcommand{\listfigurename}{\centerline{\heiti {\large 图形目录}}}
\renewcommand{\listtablename}{\centerline{\heiti {\large 表格目录}}}
\renewcommand{\indexname}{\heiti 索引} %%%%让最后生成的PDF文件的书签的索引显示“索引”而不是“Index”

%%%%%%%%%% 以下将正文中的“Part Ⅰ ”中文化成“第Ⅰ部分” %%%%%%%%%%%%%%%%%%%%%%%%%%%%%%%%%%%%%%%%%
%%%\usepackage[center]{titlesec}  %%标题居中
\usepackage{titlesec}
\titleformat{\part}{\centering\Large\bfseries}{{\heiti 第} \thepart {\heiti 部分}}{1.2em}{}
%%%%%%%%%% 以上将正文中的“Part Ⅰ ”中文化成“第Ⅰ部分” %%%%%%%%%%%%%%%%%%%%%%%%%%%%%%%%%%%%%%%%%

%%%%%%% 以下调整目录条目之间的间距 %%%%%%%%%%%%%%%%%%%%%%%%%%%%%%%%%%%%
\usepackage{tocloft}
\setlength{\cftbeforetoctitleskip}{20pt} %%% “目录”二字的段前间距为20pt
\setlength{\cftaftertoctitleskip}{50pt}  %%% “目录”二字的段后间距为50pt
\setlength{\cftbeforepartskip}{25pt} %%% 部分(part)之前的空白为25pt
\renewcommand{\cftpartafterpnum}{\vspace{6pt}} %%% 部分(part)之后的空白为6pt
\setlength{\cftbeforesecskip}{14pt} %%% 节(sec)之前的空白为14pt
\renewcommand{\cftsecafterpnum}{\vspace{3pt}} %%% 节(sec)之后的空白为3pt
\renewcommand{\cftsubsecafterpnum}{\vspace{3pt}} %%% 小节(subsec)之后的空白为3pt
%%%%%%% 以上调整目录条目之间的间距 %%%%%%%%%%%%%%%%%%%%%%%%%%%%%%%%%%%%

%%%%%%%%%%%% 以下设置书签和目录的颜色、链接%%%%%%%%%%%%%%%%%%%%%%%%%%%%%%
\usepackage[svgnames]{xcolor}
\definecolor{myurlcolor}{rgb}{0,0,0.7}
%%%%\definecolor{mylinkcolor}{rgb}{0.7,0,0}
\definecolor{mylinkcolor}{RGB}{178,0,0}
\definecolor{codecolor}{rgb}{0,0.4,0.2}
\definecolor{overviewcolor}{rgb}{0,0.2,0.4}
\definecolor{Mylightgreen}{RGB}{216,233,236} %定义名为Mylightgreen的颜色(RGB分别为216,233,236)
\usepackage[xetex,bookmarks=true,hidelinks,%
colorlinks,linkcolor=mylinkcolor,urlcolor=myurlcolor,%
pageanchor=true,hyperindex=true,
]{hyperref}
%%%%%%%%%%%% 以上设置书签和目录的颜色、链接%%%%%%%%%%%%%%%%%%%%%%%%%%%%%%

%%%%%%%%%%%% 以下设置网址太长自动换行 %%%%%%%%%%%%%%%%%%%%%%%%%%%%%%
\usepackage{url}
\def\UrlBreaks{\do\A\do\B\do\C\do\D\do\E\do\F\do\G\do\H\do\I\do\J
\do\K\do\L\do\M\do\N\do\O\do\P\do\Q\do\R\do\S\do\T\do\U\do\V
\do\W\do\X\do\Y\do\Z\do\[\do\\\do\]\do\^\do\_\do\`\do\a\do\b
\do\c\do\d\do\e\do\f\do\g\do\h\do\i\do\j\do\k\do\l\do\m\do\n
\do\o\do\p\do\q\do\r\do\s\do\t\do\u\do\v\do\w\do\x\do\y\do\z
\do\.\do\@\do\\\do\/\do\!\do\_\do\|\do\;\do\>\do\]\do\)\do\,
\do\?\do\'\do+\do\=\do\#}
%%%%%%%%%%%% 以上设置网址太长自动换行 %%%%%%%%%%%%%%%%%%%%%%%%%%%%%%
\usepackage{tcolorbox}


\begin{document}
 \DocInput{\jobname.dtx}
\end{document}
%</driver>
% \fi
%
%
% \GetFileInfo{geometry.sty}
%
% \title{{\Huge\textsf{geometry}}\ {\huge \Heiti 宏包}}
% \date{\filedate\ \ \ \ \ \fileversion}
% \author{Hideo Umeki\ \thanks{Hideo Umeki:梅木秀夫} \hspace{0.1em}【著】 \\ \url{https://github.com/davidcarlisle/geometry}\\[5pt] 黄\ 旭\ 华\ \thanks{一名业余 \LaTeX\ 爱好者。} \hspace{1.0em}【译】}
%
% \MakeShortVerb{|}
%
% \def\OpenB{{\ttfamily\char`\{}}
% \def\Comma{{\ttfamily\char`,}}
% \def\CloseB{{\ttfamily\char`\}}}
% \def\Gm{\textsf{geometry}}
% \newcommand\argii[2]{\OpenB\meta{#1}\Comma\meta{#2}\CloseB}
% \newcommand\argiii[3]{\OpenB\meta{#1}\Comma\meta{#2}\Comma\meta{#3}\CloseB}
% \newcommand\vargii[2]{\OpenB#1\Comma#2\CloseB}
% \newcommand\vargiii[3]{\OpenB#1\Comma#2\Comma#3\CloseB}
% \newcommand\OR{\ \strut\vrule width .4pt\ }
% \newcommand\gpart[1]{\textsf{\textsl{\color[rgb]{.0,.45,.7}#1}}}%
% \newcommand\glen[1]{\textsf{#1}}
% \newenvironment{key}[2]{\expandafter\macro\expandafter{`#2'}}{\endmacro}
% \newenvironment{Options}%
%  {\begin{list}{}{%
%   \renewcommand{\makelabel}[1]{\texttt{##1}\hfil}%
%   \setlength{\itemsep}{-.5\parsep}
%   \settowidth{\labelwidth}{\texttt{xxxxxxxxxxx\space}}%
%   \setlength{\leftmargin}{\labelwidth}%
%   \addtolength{\leftmargin}{\labelsep}}%
%   \raggedright}
%  {\end{list}}
% \newenvironment{Spec}%
%  {\begin{list}{}{%
%   \renewcommand{\makelabel}[1]{\fbox{##1}\hfil}%
%   \setlength{\itemsep}{-.5\parsep}
%   \settowidth{\labelwidth}{\texttt{S(x,x)}}%
%   \setlength{\leftmargin}{\labelwidth}%
%   \addtolength{\leftmargin}{\labelsep}%
%   \addtolength{\leftmargin}{2em}%
%   \setlength{\rightmargin}{2em}}%
%   \raggedright}
%  {\end{list}}
% \def\Ss(#1,#2){\textsf{S(#1,#2)}}%
% \def\Sp(#1,#2,#3){\mbox{|(#1,#2,#3)|}}%
% \def\onlypre{\llap{$^{\dagger\:}$}}%
%
% \maketitle
% \vspace{10.3cm}
% \begin{abstract}
% \vspace{1.5em}
% 该宏包提供了一个灵活、方便的页面尺寸接口(interface to page dimensions)。
% 您可以使用直观的参数更改页面布局(page layout)。例如,如果要将版口(margin)设置为
% 距离纸张的每个边缘为2cm,只需键入 |\usepackage[margin=2cm]{geometry}|。可以
% 使用 |\newgeometry| 命令更改文档中间的(middle of the document)页面布局。
% \end{abstract}
%
% \phantomsection  ^^A 将“摘要”添加到目录
% \addcontentsline{toc}{section}{\large \heiti \textbf{摘要}} ^^A 将“摘要”添加到目录
%
% \clearpage
% \thispagestyle{empty}
% \tableofcontents
% \markright{\heiti 目录}  ^^A 页眉中的“目录”
%
% \clearpage
% \thispagestyle{empty}
%
% \section{{\heiti 版本}5{\heiti 前言}}
%
% \begin{itemize}
%  \item \textbf{更改文档中间的页面布局。}\par
%    新命令 \cs{newgeometry}\{$\cdots$\}\ 和 \cs{restoregeometry}\ 允许用户更改文档中间的页面尺寸。
%    \cs{newgeometry}\ 与 \cs{geometry}\ 几乎相似,只是 \cs{newgeometry}\ 禁用了前言中指定的所有选项,
%    并跳过了与纸张尺寸相关的选项:|landscape|、|portrait| 和
%    纸张尺寸选项(如 |papersize|、|paper=a4paper| 等)。
%  \item \textbf{用于指定布局区域(layout area)的一组新选项。}\par
%   添加了为计算页面尺寸的区域指定的选项:\textsf{layout}、\textsf{layoutsize}、
%   \textsf{layoutwidth}、\textsf{layoutheight}\ 等。这些选项将有助于将指定的布局
%   打印到不同尺寸的纸张上。例如,对于 |a4paper| 和 |layout=a5paper|,\Gm\ 宏包使
%   用“A5”布局来计算版口(margins),而纸张尺寸仍然为“A4”。
%  \item \textbf{新的驱动程序选项 |xetex|。}\par
%   添加了新的驱动程序(driver)选项 |xetex|。修改了驱动程序自动检测例程(auto-detection routine),
%   以避免未定义控制序列的错误。注意 \TeX{} Live\ 中的“geometry.cfg”禁用了自动检测例程并设置了 |pdftex|,
%   它不再是必需的,即使它仍然存在也没有问题。强烈建议使用 \XeLaTeX{}\ 设置 |xetex|。
%  \item \textbf{为\  \ JIS B-series\ \ 和\ \ ISO C-series\ \ 预设了新的纸张尺寸。}\par
%   添加了以下纸张尺寸预设(papersize presets):JIS\ \footnote{JIS:Japanese Industrial Standards,日本工业标准}\ B-系列的纸张尺寸预设为 |b0j|$\sim$|b6j|,
%   ISO\ \footnote{ISO:International Organization for Standardization,国际标准化组织}\ C-系列(v5.4$\sim$)的
%   纸张尺寸预设为 |c0paper|$\sim$|c6paper|。
%  \item \textbf{更改了没有被设置的版口的默认值}\par
%   在以前的版本中,如果只指定了一个版口(例如,|bottom=1cm|),则 \Gm\ 将通过一个比例设置
%   另一边的版口(margin)(在垂直排版中,该比例默认为 1:1),即在这种情况下,|top=1cm|。版本5使用
%   默认 |scale|($=0.7$)设置文本正文尺寸(text-body size),并以此来确定没有被设置的版口。
%   (见第~\ref{sec:rules}~节)
%  \item \textbf{|showframe| 和 |showcrop| 选项适用于每一页。}\par
%   使用 |showframe| 选项,页面框架(page frames)将显示在每个页面上。此外,一个新的选项 |showcrop| 在
%   每页布局区域(layout area)的每个角落(corner)打印裁剪标记(crop marks)。请注意,
%   如果指定的布局尺寸(layout size)不小于纸张尺寸(paper size),标记(marks)将不可见。
%   5.4版引入了一个使用 \textsf{atbegshi}\ 宏包的新的 |\shipout| 重载过程(overloading process),
%   因此当指定 showframe 或 showcrop 选项时,需要 \textsf{atbegshi}\ 宏包。
%  \item \textbf{在处理类选项之前加载 geometry.cfg}\par
%    以前的版本在处理文档类选项(document class options)之后加载 \textsf{geometry.cfg}。
%    现在,在处理类选项之前已加载子配置文件(config file),您可以更改在 \textsf{geometry.cfg}\ 中
%    指定的行为(behavior),方法是将选项添加到 |\documentclass| 以及 |\usepackage| 和 |\geometry| 中。
%  \item \textbf{已删除的选项:|compat2| 和 |twosideshift|。}\par
%    版本5不再与以前的版本兼容。为了简单起见,已删除了 |compat2| 和 |twosideshift| 选项。
% \end{itemize}
%
% \clearpage
% \section{\heiti 介绍}
%
% 在 \LaTeX\ 中为页面布局(page layout)设置尺寸(dimensions)并不简单。您需要调整几个 \LaTeX{}\ 原
% 始尺寸(native dimensions),以便将文本区域(text area)放置在所需的位置。例如,
% 如果您想在所使用的纸张中居中显示文本区域,则必须指定原始尺寸(native dimensions)如下:
% \begin{quote}
%    |\usepackage{calc}|\\
%    |\setlength\textwidth{7in}|\\
%    |\setlength\textheight{10in}|\\
%    |\setlength\oddsidemargin{(\paperwidth-\textwidth)/2 - 1in}|\\
%    |\setlength\topmargin{(\paperheight-\textheight|\\
%    |                      -\headheight-\headsep-\footskip)/2 - 1in}|.
% \end{quote}
% 如果没有 \textsl{calc}\ 宏包,上面的示例将需要更多繁琐的设置。\Gm\ 宏包提供了一种
% 设置页面布局参数(page layout parameters)的简单方法。在这种情况下,您要做的就是:
% \begin{quote}
%    |\usepackage[text={7in,10in},centering]{geometry}|.
% \end{quote}
% 除了居中问题(centering problem),设置纸张每边的版口(margins)也很麻烦。但 \Gm\ 也使其变得容易。
% 如果要将每个版口设置为1.5in,可以键入:
% \begin{quote}
%    |\usepackage[margin=1.5in]{geometry}|
% \end{quote}
% 因此, \Gm\ 具有自动完成机制(auto-completion mechanism),其中未指定的尺寸(unspecified dimensions)将自动确定。
% 当您必须按照以下严格说明(strict instructions)设置页面布局时,\Gm\ 宏包也会很有用,例如:
% \begin{quote}\slshape
%   The total allowable width of the text area is $6.5$ inches wide by $8.75$
%   inches high. The top margin on each page should be $1.2$ inches from
%   the top edge of the page. The left margin should be $0.9$ inch from
%   the left edge. The footer with page number should be at the bottom
%   of the text area.
% \end{quote}
% 在这种情况下,可以使用 \Gm\ 键入:
% \begin{quote}
% |\usepackage[total={6.5in,8.75in},|\\
% |            top=1.2in, left=0.9in, includefoot]{geometry}|.
% \end{quote}
%
% 在文档准备系统(document preparation system)中的纸张上设置文本区域(text area)与
% 在视窗操作系统(window system)中的背景上放置窗口(placing a window)有一定的相似性。
% 名称“geometry”来自用于在 X window 系统中指定窗口大小和位置的 |-geometry| 选项。
%
% \clearpage
% \section{{\heiti 页面} geometry}
%
% 图~\ref{fig:layout}~显示了 \Gm\ 宏包中定义的页面布局尺寸(page layout dimensions)。
% 页面布局包含 \gpart{total body}\ (版心,或可打印区)和 \gpart{margins}\ (版口)。
% \gpart{total body}\ (版心,或可打印区)由 \gpart{body}\ (正文区[text area])和可选的 \gpart{header}\ (页眉)、
% \gpart{footer}\ (页脚)和 marginal notes (旁注,marginpar)组成。有四个 \gpart{margins}\ (版口):
% \gpart{left}\ (订口、里口)、\gpart{right}\ (切口、外口、翻口)、\gpart{top}\ (天头、上边空、上空)、
% \gpart{bottom}\ (地脚、下边空、下空)。
% 对于双开面文档(twosided documents),水平版口(horizontal margins)应称
% 为 \gpart{inner}\ (订口、里口)和 \gpart{outer}\ (切口、外口、翻口)。
%
%
% \vspace{1em}
%
% {\centering
% \begin{tabular*}{1.0\linewidth}{@{\hspace{0pt}}rl}
%  \hlinew{1.2pt}
%  {\Heiti 术语}&{\Heiti 含义}\\ \hlinew{0.7pt}
%   \gpart{papers}\ (\gpart{纸张})& \gpart{total body}\ (可打印区、版心)\ + \gpart{margins}\ (版口)\\
%   \gpart{total body}\ (\gpart{可打印区、版心})&\gpart{body}\ (文本区、正文)\ + 可选的 \gpart{head}\ (页眉)\\
                                              &\hspace{9em} + 可选的 \gpart{foot}\ (页脚)\\
                                              &\hspace{9em} + 可选的 \gpart{marginpar}\ (旁注与正文之间的间距)\\
%   \gpart{margins}\ (\gpart{版口})& \gpart{top}\ (天头),\,\gpart{bottom}\ (地脚),\gpart{left} (订口,\,\gpart{inner}\ [里口]),\,\gpart{right} (切口,\,\gpart{outer}\ [外口])\\ \hlinew{1.2pt}
%   \end{tabular*}}
%
%   \vspace{1em}
%
% 每个版口(margin)都是从纸张的相应边缘开始测量的。例如,订口(left margin)即内口(inner margin)是指
% 纸张的左(内)边缘与版心(total body,可打印区)之间的水平距离(horizontal distance)。
% 因此, \Gm\ 中定义的订口(left margins,里口)和天头(top margins) 与原始
% 尺寸(native dimensions)\ \cs{leftmargin}\ 和 \cs{topmargin}\ 是不同。
% 正文(body,文本区域)的尺寸可以通过 \cs{textwidth}\ 和 \cs{textheight}\ 修改。
% 纸张(paper)、版心(total body,可打印区)和版口(margin)的尺寸具有以下关系:
% \begin{eqnarray}
%  \label{eq:paperwidth}
%  |paperwidth (纸张宽度)| &=& |left (订口)|+|width (版心宽度)|+|right (切口)| \\
%  |paperheight (纸张高度)| &=& |top (天头)|+|height (版心高度)|+|bottom (地脚)|
%  \label{eq:paperheight}
% \end{eqnarray}
%
% \begin{figure}[H]
%  \centering\small
%  {\unitlength=.65pt
%  \begin{picture}(450,250)(0,-10)
%  \put(20,0){\framebox(170,230){}}
%  \put(20,235){\makebox(170,230)[br]{\gpart{paper}\ \scriptsize(\gpart{纸张})}}
%  \begingroup\thicklines
%  \put(40,30){\framebox(120,170){}}\endgroup
%  \put(40,30){\makebox(120,165)[tr]{\gpart{total body}~}}
%  \put(40,15){\makebox(120,165)[tr]{\scriptsize(\gpart{可打印区,版心})~}}
%  \put(45,30){\makebox(0,170)[l]{|height|}}
%  \put(45,15){\makebox(0,170)[l]{\scriptsize |(版心高度)|}}
%  \put(40,35){\makebox(120,0)[bc]{|width|\ \scriptsize (|版心宽度|)}}
%  \put(50,-20){\makebox(120,0)[bc]{|paperwidth|\ \scriptsize (|纸张宽度|)}}
%  \put(10,45){\makebox(0,170)[r]{|paperheight|}}
%  \put(10,30){\makebox(0,170)[r]{\scriptsize |(纸张高度)|}}
%  \put(90,200){\makebox(0,30)[lc]{|top|\ \tiny (|天头|)}}
%  \put(90,0){\makebox(0,30)[lc]{|bottom|\ \tiny (|地脚|)}}
%  \put(10,70){\makebox(0,0)[r]{|left|\ \scriptsize (|订口|)}}
%  \put(10,55){\makebox(0,0)[r]{\tiny (|inner|,里口)}}
%  \put(200,70){\makebox(0,0)[l]{|right|\scriptsize (|切口|)}}
%  \put(200,55){\makebox(0,0)[l]{\tiny (|outer|,外口)}}
%  \put(80,230){\vector(0,-1){30}}\put(80,30){\vector(0,-1){30}}
%  \put(80,200){\vector(0,1){30}}\put(80,0){\vector(0,1){30}}
%  \put(20,70){\vector(1,0){20}}\put(40,70){\vector(-1,0){20}}
%  \put(160,70){\vector(1,0){30}}\put(190,70){\vector(-1,0){30}}
%  \multiput(160,30)(5,0){24}{\line(1,0){2}}
%  \multiput(160,200)(5,0){24}{\line(1,0){2}}
%  \begingroup\thicklines
%  \put(280,30){\framebox(120,170){}}\endgroup
%  \put(283,133){\makebox(0,12)[l]{|textheight|\ \tiny (|正文高度|)}}
%  \put(295,130){\vector(0,-1){100}}\put(295,150){\vector(0,1){50}}
%  \multiput(280,220)(5,0){24}{\line(1,0){3}}
%  \put(280,208){\makebox(120,20)[bc]{\gpart{head}\ \scriptsize(\gpart{页眉})}}
%  \multiput(280,207)(5,0){24}{\line(1,0){3}}
%  \put(420,225){\makebox(0,0)[l]{|headheight|\ \scriptsize (|页眉高度|)}}
%  \put(418,225){\line(-2,-1){20}}
%  \put(420,213){\makebox(0,0)[l]{|headsep|\ \scriptsize (|页眉底与正文顶的间距|)}}
%  \put(418,213){\line(-2,-1){20}}
%  \put(420,12){\makebox(0,0)[l]{|footskip|\ \scriptsize (|正文底与页脚底的间距|)}}
%  \put(418,12){\line(-2,1){20}}
%  \put(280,40){\makebox(120,140)[c]{\gpart{body}\ \scriptsize(\gpart{正文})}}
%  \put(305,45){\vector(-1,0){25}}\put(375,45){\vector(1,0){25}}
%  \put(80,230){\vector(0,-1){30}}\put(80,30){\vector(0,-1){30}}
%  \put(280,48){\makebox(120,0)[c]{|textwidth|}}
%  \put(280,37){\makebox(120,0)[c]{\tiny |(正文宽度)|}}
%  \put(280,15){\makebox(120,10)[c]{\gpart{foot}\ \scriptsize(\gpart{页脚})}}
%  \multiput(280,14)(5,0){24}{\line(1,0){2}}
%  \put(410,30){\dashbox{3}(30,170){}}
%  \put(415,30){\makebox(30,170)[l]{\gpart{marginal note}\ \scriptsize(\gpart{旁注})}}
%  \put(425,45){\vector(-1,0){15}}\put(425,45){\vector(1,0){15}}
%  \put(450,85){\makebox(0,0)[l]{|marginparsep|}}
%  \put(450,70){\makebox(0,0)[l]{\scriptsize (|旁注与正文的间距|)}}
%  \put(448,85){\line(-3,-1){43}}
%  \put(450,45){\makebox(0,0)[l]{|marginparwidth|\ \scriptsize (|旁注宽度|)}}
%  \end{picture}}
%  \vspace{1em}
%  \caption[Dimension names for \Gm]{%
%  \begin{minipage}[t]{.8\textwidth}\raggedright\small
%  \Gm\ 宏包中使用的尺寸名称(dimension names)。\\
%  {\kaiti |width(版心宽度)| $=$ |textwidth(正文宽度)|,
%  |height(版心高度)| $=$ |textheight(正文高度)|,这是默认情况。|left(订口)|、|right(切口)|、|top(天头)|、|bottom(地脚)|等为版口(margins)。
%  如果通过 |twoside| (双开面)选项交换了(swapped)反面页(verso pages)的版口,则由 |left| (订口)和 |right| (切口)选项指定的版口
%  分别用于内侧(inside)和外侧(outside)版口。|inner| (里口)和 |outer| (外口)分别是 |left| (订口)和 |right| (切口)的别名。}
%  \end{minipage}}
%  \label{fig:layout}
% \end{figure}
%
% 版心或可打印区(total body)的宽度(|width|)和高度(|height|)的定义如下:
% \begin{eqnarray}
%  \label{eq:width}
%  |width(版心宽度)| &:=& |textwidth(正文宽度)| \quad( +\>  |marginparsep| + |marginparwidth| )\\
%  |height(版心高度)| &:=& |textheight(正文高度)| \quad(+\> |headheight| + |headsep| + |footskip| )
%  \label{eq:height}
% \end{eqnarray}
% 当 |marginparsep(旁注与正文的间距)| 和 |marginparwidth(旁注宽度)|在 |width(版心宽度)| 的范围之内,
% 即 |includemp| 这个选项被指定为 |true| 时,等式\ (\ref{eq:width})\ |width:=textwidth| 是在水平方向的默认设置。
% 而等式\ (\ref{eq:height})\ |height:=textheight| 是在垂直方向的默认设置。如果 |includehead| 这个选项被指定为 |true|,
% |headheight(页眉高度)| 和 |headsep(页眉底与正文顶的间距)| 则被包含在 |height(版心高度)| 中。
% 同样,如果 |includefoot| 这个选项被指定为 |true|,|footskip(正文底与页脚底的间距)| 则被包含在 |height(版心高度)| 中。
% 图~\ref{fig:includes}\ 显示了这些选项在垂直方向(vertical direction)上的工作方式。
%
% \vspace{1em}
%
% \begin{figure}[H]
%  \centering\small
%  {\unitlength=.65pt
%  \begin{picture}(490,280)(0,-10)
%  \put(60,250){\makebox(120,0)[bl]{\textbf{(a)}\ \ 默认(\textit{default})}}%
%  \put(20,0){\framebox(170,230){}}
%  \put(20,230){\makebox(170,15)[r]{\gpart{paper}\ \scriptsize(\gpart{纸张})}}
%  \begingroup\thicklines
%  \put(40,30){\framebox(120,165){}}\endgroup
%  \put(70,165){\vector(0,1){30}}
%  \put(55,145){\makebox(0,20)[lc]{|textheight|}}
%  \put(75,130){\makebox(0,20)[lc]{\scriptsize (|正文高度|)}}
%  \put(70,145){\vector(0,-1){115}}
%  \multiput(40,203)(5,0){24}{\line(1,0){3}}
%  \multiput(40,213)(5,0){24}{\line(1,0){3}}
%  \multiput(40,10)(5,0){24}{\line(1,0){3}}
%  \put(40,203){\makebox(120,20)[bc]{\gpart{head}\ \tiny(\gpart{页眉})}}
%  \put(45,20){\makebox(120,140)[c]{\gpart{body}\ \scriptsize(\gpart{正文})}}
%  \put(40,11){\makebox(120,10)[c]{\gpart{foot}\ \tiny(\gpart{页脚})}}
%  \put(150,230){\vector(0,-1){35}}\put(150,30){\vector(0,-1){30}}
%  \put(150,195){\vector(0,1){35}}\put(150,0){\vector(0,1){30}}
%  \put(160,197){\makebox(0,30)[lc]{|top|}}
%  \put(161,186){\makebox(0,30)[lc]{\tiny (|天头|)}}
%  \put(160,0){\makebox(0,30)[lc]{|bottom|}}
%  \put(161,-10){\makebox(0,30)[lc]{\tiny (|地脚|)}}
%  \multiput(160,30)(5,0){24}{\line(1,0){2}}
%  \multiput(160,195)(5,0){24}{\line(1,0){2}}
%  \put(255,250){\makebox(120,0)[bl]
%      {\textbf{(b)}\ \ |includehead| 和 |includefoot|}}%
%  \put(260,0){\framebox(170,230){}}
%  \put(260,230){\makebox(170,15)[r]{\gpart{paper}\ \scriptsize(\gpart{纸张})}}
%  \begingroup\thicklines
%  \put(280,30){\framebox(120,165){}}\endgroup
%  \put(310,152){\vector(0,1){25}}
%  \put(295,130){\makebox(0,20)[lc]{|textheight|}}
%  \put(315,115){\makebox(0,20)[lc]{\scriptsize (|正文高度|)}}
%  \put(310,130){\vector(0,-1){80}}
%  \multiput(280,184)(5,0){24}{\line(1,0){3}}
%  \multiput(280,177)(5,0){24}{\line(1,0){3}}
%  \multiput(280,50)(5,0){24}{\line(1,0){3}}
%  \put(280,184){\makebox(120,10)[c]{\gpart{head}\ \tiny(\gpart{页眉})}}
%  \put(285,20){\makebox(120,140)[c]{\gpart{body}\ \scriptsize(\gpart{正文})}}
%  \put(400,140){\line(1,1){45}}
%  \put(447,187){\makebox(50,10)[l]{\gpart{total body}}}
%  \put(447,172){\makebox(50,10)[l]{\scriptsize(\gpart{版心、可打印区})}}
%  \put(280,31){\makebox(120,10)[c]{\gpart{foot}\ \tiny(\gpart{页脚})}}
%  \put(370,230){\vector(0,-1){35}}\put(370,30){\vector(0,-1){30}}
%  \put(370,195){\vector(0,1){35}}\put(370,0){\vector(0,1){30}}
%  \put(380,197){\makebox(0,30)[lc]{|top|\tiny (|天头|)}}
%  \put(380,0){\makebox(0,30)[lc]{|bottom|}}
%  \put(380,0){\makebox(0,10)[lc]{\tiny (|地脚|)}}
%  \end{picture}}
%  \caption[An effect of \texttt{includehead} and \texttt{includefoot}.]{%
%  \begin{minipage}[t]{.8\textwidth}\raggedright\small
%    |includehead| 和 |includefoot| 分别将页眉和页脚包含在版心(即可打印区,\gpart{total body})中。\\
%    {\kaiti \textbf{(a)} |height| $=$ |textheight| (默认)。
%    如果 |includehead| 和 |includefoot| 均为 |true|,则 \textbf{(b)} |height| $=$ |textheight| $+$ |headheight| $+$ |headsep| $+$ |footskip|。
%    如果指定了天头(top margins)和地脚(bottom margins),则 |includehead| 和 |includefoot| 会导致更短的 |textheight|。}
%  \end{minipage}}
%  \label{fig:includes}
% \end{figure}
%
% \vspace{1em}
%
% 因此,页面布局由每个方向的三个部分(长度)组成:一个正文(body)和两个版口(margins)。
% 如果明确指定了其中两个,则另一个长度是确定的而无需指定。图~\ref{fig:Labc}~显示了
% 页面尺寸(page dimensions)的简单模型。当给定长度 |L| 并将其划分为正文(body) |b|、
% 版口(margins) |a| 和 |c| 时,很明显:
% \begin{equation}
%   |L|=|a|+|b|+|c|  \label{eq:Labc}
% \end{equation}
% 该等式表明,当三个(|a|、|b| 和 |c|)中的两个确定之后,另外一个就可以解出来。
% 如果有两个或两个以上没有被确定,则等式~(\ref{eq:Labc})~不可能在它们之间没有任何
% 其他关系的情况下被解出来。如果它们都是确定的,那么需要检查它们是否能满足等式~(\ref{eq:Labc}),
% 这是确定过多或“过度确定(overspecified)”。
% \begin{figure}[H]
%  \centering
%  {\unitlength=0.8pt
%  \begin{picture}(300,60)(0,-5)
%  \begingroup\linethickness{5pt}
%  \put(0,5){\textcolor{green}{\line(1,0){60}}}
%  \put(60,5){\textcolor{black}{\line(1,0){160}}}
%  \put(220,5){\textcolor{green}{\line(1,0){80}}}
%  \endgroup
%  \put(0,15){\makebox(60,10)[b]{|a|}}
%  \put(60,0){\line(0,1){20}}
%  \put(60,15){\makebox(160,10)[b]{|b|}}
%  \put(220,0){\line(0,1){20}}
%  \put(220,15){\makebox(80,10)[b]{|c|}}
%  \put(0,0){\line(0,1){50}}
%  \put(150,35){\vector(-1,0){150}}
%  \put(0,40){\makebox(300,10){|L|}}
%  \put(150,35){\vector(1,0){150}}
%  \put(300,0){\line(0,1){50}}
%  \end{picture}}
%  \caption{页面尺寸(page dimensions)的简单模型。}
%  \label{fig:Labc}
% \end{figure}
%
% \Gm\ 宏包的自动完成机制(auto-completion mechanism)省去了确定页面布局尺寸的麻烦。例如,
% 您可以为 A4 纸设置如下:
% \begin{quote}
%  |\usepackage[width=14cm, left=3cm]{geometry}|
% \end{quote}
% 在这种情况下,您不必设置切口(即外口,right margin)。有关自动完成(auto-completion)的
% 详细信息请参阅第~\ref{sec:rules}~节。
%
% \clearpage
% \section{\heiti 用户接口}
%
% \subsection[命令]{\heiti 命令}
%
% \Gm\ 宏包提供了以下命令:
% \begin{itemize}\setlength{\itemsep}{-.5\parsep}
%  \item {\large \color{blue}{|\geometry{|\meta{options}|}|}}
%  \item {\large \color{blue}{|\newgeometry{|\meta{options}|}|}}\ 和\ {\large \color{blue}{|\restoregeometry|}}
%  \item {\large \color{blue}{|\savegeometry{|\meta{name}|}|}}\ 和\ {\large \color{blue}{|\loadgeometry{|\meta{name}|}|}}
% \end{itemize}
%
% |\geometry{|\meta{options}|}| 根据参数中指定的选项更改页面布局。
% 此命令(如果有的话)应该只放在前言中(在 |\begin{document}| 之前)。
%
% \Gm\ 宏包可以作为类(class)的一部分来使用,也可以作为文档中使用的另一个包的一部分来使用。
% \cs{geometry}\ 命令可以覆盖前言部分中的一些设置。允许多次使用 \cs{geometry},
% 然后相关的设置(options concatenated)都会起作用。如果尚未加载 \Gm\ 宏包,
% 则只能使用 |\usepackage[|\meta{options}|]{geometry}| 来代替 \cs{geometry}。
%
% \medskip
% |\newgeometry{|\meta{options}|}| 更改文档中间的(mid-document)页面布局。
% \cs{newgeometry}\ 与 \cs{geometry}\ 几乎相似,除了 \cs{newgeometry}\ 禁用了
% 前言中 \cs{usepackage}\ 和 \cs{geometry}\ 指定的所有选项,并跳过(skips)与
% 纸张尺寸相关的选项(papersize-related options)。\cs{restoregeometry}\ 恢复前言中指定的页面布局。
% 此命令没有参数。详见第~\ref{sec:midchange}~节。
%
% \medskip
% |\savegeometry{|\meta{name}|}| 将页面尺寸(page dimensions)保存为 \meta{name},
% 您可以将此命令放在其中。|\loadgeometry{|\meta{name}|}| 加载保存为 \meta{name}\ 的页面尺寸。
% 有关详细信息,请参阅第~\ref{sec:midchange}~节。
%
% \subsection[可选参数]{\heiti 可选参数}
%
% \Gm\ 宏包采用 \textsf{keyval}\ 接口“\meta{key}=\meta{value}”,该接口可以用作 \cs{usepackage}、
% \cs{geometry}\ 和 \cs{newgeometry}\ 的可选参数。
%
% 该参数包含一个逗号分隔的(comma-separated) \textsf{keyval}\ 选项列表,其基本规则如下:
% \begin{itemize}\setlength{\itemsep}{-.5\parsep}
% \item 允许多行,但不允许空行。
% \item 单词之间的空格将被忽略。
% \item 选项基本上与顺序无关(order-independent)。(有一些例外情况,详见第~\ref{sec:optionorder}~节。)
% \end{itemize}
%  例如:
% \begin{quote}
% |\usepackage[ a5paper ,  hmargin = { 3cm,|\\
% |                .8in } , height|\\
% |         =  10in ]{geometry}|
% \end{quote}
% 相当于:
% \begin{quote}
%   |\usepackage[height=10in,a5paper,hmargin={3cm,0.8in}]{geometry}|
% \end{quote}
% 某些选项允许有子列表(sub-list),例如 |{3cm,0.8in}|。请注意,子列表中的值顺序很重要。
% 上述设置也等同于以下设置:
% \begin{quote}
%   |\usepackage{geometry}|\\
%   |\geometry{height=10in,a5paper,hmargin={3cm,0.8in}}|
% \end{quote}
% 或
% \begin{quote}
%   |\usepackage[a5paper]{geometry}|\\
%   |\geometry{hmargin={3cm,0.8in},height=8in}|\\
%   |\geometry{height=10in}|.
% \end{quote}
% 因此,多次使用 \cs{geometry}\ 命令只是用来定义附加的选项(appends options)。
%
% \Gm\ 支持 \textsl{calc}\ 宏包\ \footnote{CTAN:~\texttt{macros/latex/required/tools}}。
% 例如:
% \begin{quote}
%   |\usepackage{calc}|\\
%   |\usepackage[textheight=20\baselineskip+10pt]{geometry}|
% \end{quote}
%
% \subsection[选项类型]{\heiti 选项类型}
% \Gm\ 的选项分为四种类型:
%
% \begin{enumerate}\itemsep=0pt
% \item \textbf{布尔型(Boolean type)}
%
%    采用布尔值(|true| 或 |false|)。如果没有赋值,则默认设置为 |true|。
%    \begin{quote}
%       \meta{key}|=true|\OR|false|.\\
%       \meta{key}\ 如果没有赋值则相当于:\meta{key}|=true|。
%    \end{quote}
%    {\kaiti 例如:}|verbose=true|, |includehead|,
%    |twoside=false|.\\
%    纸张名称(paper name)是例外。首选纸张名称应设置为没有赋值(no values)。
%    无论给出什么值,它都会被忽略。例如,|a4paper=XXX| 相当于 |a4paper|。
%
% \item \textbf{单值型(single-valued type)}
%
%    取一个强制值(mandatory value),即该参数必须赋值。
%    \begin{quote}
%    \meta{key}|=|\meta{value}.
%    \end{quote}
%    {\kaiti 例如:}|width=7in|, |left=1.25in|,
%    |footskip=1cm|, |height=.86\paperheight|。
%
% \item \textbf{双值型(double-valued type)}
%
%    在大括号(braces)中使用一对逗号分隔的值。如果两个值相同,则可以将其缩写为一个值。
%    \begin{quote}
%    \meta{key}|=|\argii{value1}{value2}.\\
%    \meta{key}|=|\meta{value} 等同于
%              \meta{key}|=|\argii{value}{value}。
%    \end{quote}
%    {\kaiti 例如:}|hmargin={1.5in,1in}|, |scale=0.8|,
%    |body={7in,10in}|.
%
% \item \textbf{三值型(triple-valued type)}
%
%    在大括号中采用三个逗号分隔的强制值(即必须赋三个值)。
%    \begin{quote}
%    \meta{key}|=|\argiii{value1}{value2}{value3}
%    \end{quote}
%    每个值都必须是一个尺寸(dimension)或空(null)。当您给出一个空值(empty value)或“|*|”时,
%    它表示空(null),并将适当的值留给自动完成机制(auto-completion mechanism)。
%    您需要指定至少一个尺寸,通常是两个尺寸。您可以将所有的值设置为空(null),但这毫无意义。\\
%    {\kaiti 例如:}
%    |hdivide={2cm,*,1cm}|, |vdivide={3cm,19cm, }|,
%                   |divide={1in,*,1in}|.
% \end{enumerate}
%
% \clearpage
% \section{\heiti 详解选项}
%
% 本节介绍 \Gm\ 中的所有可用选项。带有匕首(dagger) $^\dagger$ 的选项
% 不可用作 \cs{newgeometry}\ 的参数(见第~\ref{sec:midchange}~节)。
%
% \subsection[纸张尺寸]{\heiti 纸张尺寸}\label{sec:paper}
%
% 下面的选项设置纸张/介质(paper/media)的尺寸(size)和方向(orientation)。
% \begin{Options}
% \item[\onlypre paper\OR papername] ~\\
%  按名称指定纸张尺寸。|paper=|\meta{paper-name}。为了方便起见,可以不使用 |paper=| 来
%  指定纸张名称(paper name)。例如,|a4paper| 等同于 |paper=a4paper|。
% \item[\onlypre \vtop{
%  \hbox{a0paper, a1paper, a2paper, a3paper, a4paper, a5paper, a6paper,}
%  \hbox{b0paper, b1paper, b2paper, b3paper, b4paper, b5paper, b6paper,}
%  \hbox{c0paper, c1paper, c2paper, c3paper, c4paper, c5paper, c6paper,}
%  \hbox{b0j, b1j, b2j, b3j, b4j, b5j, b6j,}
%  \hbox{ansiapaper, ansibpaper, ansicpaper, ansidpaper, ansiepaper,}
%  \hbox{letterpaper, executivepaper, legalpaper}}]~\\[1ex]
%    指定纸张名称。值部分(value part)即使有也会被忽略。例如,以下几种纸张具有相同的效果:
%    |a5paper|、|a5paper=true|、|a5paper=false| 等等。|a[0-6]paper|、|b[0-6]paper|、
%    |c[0-6]paper| 分别是 ISO A、 B 和 C 系列(series)纸张尺寸。
%    JIS\ (Japanese Industrial Standards,日本工业标准) A 系列与 ISO A 系列相同,
%    但 JIS B 系列与 ISO B 系列不同。|b[0-6]j| 应用于 JIS B 系列。
% \item[\onlypre screen] 特殊纸张尺寸(special paper size)\ (W,H) = (225mm,180mm)。
%    对于使用个人电脑(Personal Computer,PC)和
%    视频投影仪(video projector)的演示(presentation),将“slide”文档类(documentclass)
%    设置为“|screen,centering|”将非常有用。
% \item[\onlypre paperwidth] 纸张宽度。|paperwidth=|\meta{length}。
% \item[\onlypre paperheight] 纸张高度。|paperheight=|\meta{length}。
% \item[\onlypre papersize] 纸张的宽度和高度。|papersize=|\argii{width}{height}\ \ 或\ \ |papersize=|\meta{length}。
% \item[\onlypre landscape] 将纸张方向切换为横向模式(landscape mode)。
% \item[\onlypre portrait] 将纸张方向切换为纵向模式(portrait mode)。这相当于 |landscape=false|。
% \end{Options}
%
% 用于纸张名称(例如 |a4paper|)与方向即 |portrait| (纵向)和 |landscape| (横向)的选项可以
% 设置为文档类选项(document class options)。例如,您可以设置 |\documentclass[a4paper,landscape]{article}|,
% 然后在 \Gm\ 中会处理 |a4paper| 和 |landscape|。这也是 |twoside| (双开面)和 |twocolumn| (双栏)的情况
% (另见第~\ref{sec:dimension}~节)。
%
% \subsection[布局尺寸]{\heiti 布局尺寸}
%
%  无论纸张尺寸(paper size)如何,都可以使用本节中描述的选项指定布局区域(layout area)。
%  这些选项将有助于将指定的布局(specified layout)打印到不同尺寸的纸张上。例如,对
%  于 |a4paper| 和 |layout=a5paper|,宏包使用“A5”布局(layout)来计算“A4”纸张(paper)上的
%  版口(margins)。布局尺寸(layout size)默认为与纸张(paper)相同。布局尺寸的选项
%  在 \cs{newgeometry}\ 中可用,因此您可以更改文档中间的(middle)布局尺寸。纸张尺寸本身无法更改。
%  图~\ref{fig:layoutandpaper}~显示了布局(|layout|)和纸张(|paper|)之间的区别。
% \begin{Options}
% \item[layout:] 即布局(layout),按纸张名称(paper name)指定布局尺寸(layout size)。
% |layout=|\meta{paper-name}。\Gm\ 中定义的所有纸张名称都可用。详见第~\ref{sec:paper}~节。
% \item[layoutwidth:] 布局(layout)的宽度。|layoutwidth=|\meta{length}。
% \item[layoutheight:] 布局(layout)的高度。|layoutheight=|\meta{length}。
% \item[layoutsize:] 即布局的尺寸(layoutsize)。布局的宽度和高度。|layoutsize=|\argii{width}{height}\ 或 |layoutsize=|\meta{length}。
% \item[layouthoffset:] 即布局水平偏移量(layouthoffset),指定距纸张左边缘(left edge)的水平偏移量(horizontal offset)。|layouthoffset=|\meta{length}。
% \item[layoutvoffset:] 即布局垂直偏移量(layoutvoffset),指定距纸张上边缘(top edge)的垂直偏移量(vertical offset)。|layoutvoffset=|\meta{length}。
% \item[layoutoffset:] 同时指定水平偏移量和垂直偏移量。|layoutoffset=|\argii{hoffset}{voffset}\ 或\ |layoutsize=|\meta{length}。
% \end{Options}
%
% \vspace{1em}
%
% \begin{figure}[H]
%  \centering\small
%  {\unitlength=.6pt
%  \begin{picture}(450,250)(0,-10)
%  \put(20,1){\makebox(168,12)[r]{\gpart{paper}\ {\scriptsize (\gpart{纸张})}}}
%  \put(20,0){\framebox(170,230){}}
%  \put(21,40){\dashbox{3}(140,189){}}
%  \put(21,26){\makebox(140,12)[r]{\gpart{layout}\ {\scriptsize (\gpart{布局})}}}
%  \put(40,51){\makebox(100,10){\gpart{foot}\ {\tiny (\gpart{页脚})}}}
%  \put(40,50){\line(1,0){100}}
%  \put(40,65){\framebox(100,125){\gpart{body}\ {\scriptsize (\gpart{正文})}}}
%  \put(40,200){\framebox(100,10){\gpart{head}\ {\tiny (\gpart{页眉})}}}
%  \put(20,245){\makebox(140,20){|layoutwidth|}}
%  \put(20,230){\makebox(140,20){\scriptsize |(布局宽度)|}}
%  \put(40,240){\vector(-1,0){20}}
%  \put(140,240){\vector(1,0){20}}
%  \put(10,145){\vector(0,1){85}}
%  \put(15,125){\makebox(0,20)[r]{|layoutheight|}}
%  \put(8,110){\makebox(0,20)[r]{\scriptsize |(布局高度)|}}
%  \put(10,125){\vector(0,-1){85}}
%  \put(280,1){\makebox(168,12)[r]{\gpart{paper}\ {\scriptsize (\gpart{纸张})}}}
%  \put(280,0){\framebox(170,230){}}
%  \put(293,35){\dashbox{3}(140,189){}}
%  \put(293,21){\makebox(140,12)[r]{\gpart{layout}\ {\scriptsize (\gpart{布局})}}}
%  \put(312,46){\makebox(100,10){\gpart{foot}\ {\tiny (\gpart{页脚})}}}
%  \put(312,45){\line(1,0){100}}
%  \put(312,60){\framebox(100,125){\gpart{body}\ {\scriptsize (\gpart{正文})}}}
%  \put(312,195){\framebox(100,10){\gpart{head}\ {\tiny (\gpart{页眉})}}}
%  \put(235,245){\makebox(80,20)[l]{|layouthoffset|}}
%  \put(235,230){\makebox(80,20)[l]{\scriptsize |(布局水平偏移量)|}}
%  \put(260,210){\vector(1,0){20}}
%  \put(308,210){\vector(-1,0){15}}
%  \put(260,210){\line(-1,2){10}}
%  \put(345,245){\makebox(100,20){|layoutvoffset|}}
%  \put(350,230){\makebox(100,20){\scriptsize |(布局垂直偏移量)|}}
%  \put(350,250){\vector(0,-1){20}}
%  \put(350,209){\vector(0,1){15}}
%  \end{picture}}
%  \caption[layout and paper]{%
%  \begin{minipage}[t]{.7\textwidth}\raggedright\small
%  与布局尺寸(layout size)相关的尺寸。\\
%  {\kaiti 请注意:布局尺寸默认为与纸张尺寸( paper size)相同,
%  因此在大多数情况下不必明确指定布局相关选项(layout-related options)。}
%  \end{minipage}}
%  \label{fig:layoutandpaper}
% \end{figure}
%
% \subsection[正文尺寸]{\heiti 正文尺寸}\label{sec:body}
%
% 本节介绍了设定 \gpart{total body}\ \ (版心、可打印区)尺寸的选项。
%
% \begin{Options}
% \item[hscale]
%    版心(\gpart{total body})宽度与纸张宽度(\cs{paperwidth})之比。|hscale=|\meta{h-scale},
%    例如,|hscale=0.8| 等同于 |width=0.8|\cs{paperwidth}。(默认为0.7)
% \item[vscale]
%    版心(\gpart{total body})高度与纸张高度(\cs{paperheight})之比,例如,|vscale=|\meta{v-scale}。(默认为0.7)。
%    |vscale=0.9| 等同于 |height=0.9|\cs{paperheight}。
% \item[scale] 版心(\gpart{total body})与纸张的比例。|scale=|\argii{h-scale}{v-scale}\ 或\ |scale=|\meta{scale}。(默认为0.7)。
% \item[width\OR totalwidth] ~\\
%    版心(\gpart{total body})的宽度。|width=|\meta{length}\ 或\ |totalwidth=|\meta{length}。
%    该尺寸默认为 |textwidth| (正文宽度),但如果 |includemp| 设置为 |true|,则 |width| $\ge$ |textwidth|,
%    因为 |width| (版心宽度)包含了旁注(marginal notes)的宽度。如果同时指定了 |textwidth| (正文宽度) 和 |width|  (版心宽度),
%    则 |textwidth| (正文宽度)的值优先于 |width| (版心宽度)的值。
% \item[height\OR totalheight] ~\\
%    版心(\gpart{total body})的高度,默认情况下包含了页眉(header)和页脚(footer)。
%    如果设置了 |includehead| 或 |includefoot|,则 |height| (版心高度)包含了页眉或页脚以及 |textheight|(正文高度)。
%    |height=|\meta{length}\ 或\ |totalheight=|\meta{length}。如果同时指定了 |textheight| (正文高度)和 |height| (版心高度),
%    则 |height| (版心高度)将被忽略。
% \item[total] 版心(\gpart{total body})的宽度和高度。|total=|\argii{width}{height}\ 或\ |total=|\meta{length}。
% \item[textwidth] 指定 \cs{textwidth}\ (正文宽度),即 \gpart{body}\ (正文、文本区域)的宽度。|textwidth=|\meta{length}。
% \item[textheight] 指定 \cs{textheight}\ (正文高度),即 \gpart{body}\ (正文、文本区域)的高度。|textheight=|\meta{length}。
% \item[text\OR body] 同时指定页面正文(body)的 \cs{textwidth}\ (正文宽度)和 \cs{textheight}\ (正文高度)。
%    |body=|\argii{width}{height}\ \ 或\ \ |text=|\meta{length}。
% \item[lines] 允许用户通过行数(number of lines)来指定 \cs{textheight}\ (正文高度)。|lines|=\meta{integer}。
% \item[includehead] 将页面的页眉(head),即 \cs{headheight}\ (页眉高度)和 \cs{headsep}\ (页眉底与正文顶的间距)
%    都包含进了 \gpart{total body}\ (版心、可打印区)。默认设置为 |false|。作用和 |ignorehead| 的作用相反。
%    参见图~\ref{fig:includes}~和图~\ref{fig:modes}。
% \item[includefoot] 将页面的页脚(foot),即 \cs{footskip}\ (正文底与页脚底的间距)包含进
%    了 \gpart{total body}\ (版心、可打印区),作用和 |ignorefoot| 的作用相反。
%    默认情况下为 |false|。参见图~\ref{fig:includes}~和图~\ref{fig:modes}。
% \item[includeheadfoot]~\\
%    将 |includehead| 和 |includefoot| 设置为 |true|,作用与 |ignoreheadfoot| 的作用相反。
%    参见图~\ref{fig:includes}~和图~\ref{fig:modes}。
% \item[includemp] 在计算水平参数(horizontal calculation)时,将旁注(margin notes),
%    即 \cs{marginparwidth}\ (旁注宽度)和 \cs{marginparsep}\ (旁注与正文的间距)均包含
%    在 \gpart{body}\ (正文)中。
% \item[includeall] 将 |includeheadfoot| 和 |includemp| 均设置为 |true|。见图~\ref{fig:modes}。
% \item[ignorehead] 在确定垂直布局(vertical layout)时,忽略页眉(head),即 |headheight| (页眉高度)
%    和 |headsep| (页眉底与正文顶的间距),但不改变这些长度。它等同于 |includehead=false|。
%    默认设置为 |true|。另请参考 |includehead|。
% \item[ignorefoot] 在确定垂直布局(vertical layout)时,忽略页脚(foot),
%    即 |footskip| (正文底与页脚底的间距),但不改变这个长度。默认设置为 |true|。
%    另请参考 |includefoot|。
% \item[ignoreheadfoot]~\\ 将|ignorehead| 和 |ignorefoot| 均设置为 |true|。
%    另请参见 |includeheadfoot|。
% \item[ignoremp]
%    在确定水平版口(horizontal margins)时忽略旁注(marginal notes)(默认为 |true|)。
%    如果旁注超出页面(page),则 |verbose=true| 时将显示警告消息。另请参见 |includemp| 和图~\ref{fig:modes}。
% \item[ignoreall]
%    将 |ignoreheadfoot| 和 |ignoremp| 均设置为 |true|。另请参见 |includeall|。
% \item[heightrounded]~\\
%    该选项使得 \cs{textheight}\ (正文高度)成为 \textit{n}\ 倍(\textit{n}\ 为整数)
%    的 \cs{baselineskip}\ 加上 \cs{topskip}\ (页眉顶与正文顶的间距),以避免在某些情况下
%    出现“underfull vbox (未满 vbox)”。例如,如果 \cs{textheight}\ 为486pt,
%    而 \cs{baselineskip}\ 为12pt,\cs{topskip}\ 为10pt,则:
%    \begin{quote}
%      $(39\times12\textrm{pt}+10\textrm{pt}=)\: 478\textrm{pt}
%       < 486\textrm{pt} <
%      490\textrm{pt} \:(=40\times12\textrm{pt}+10\textrm{pt})$,
%    \end{quote}
%    因此,\cs{textheight}\ 四舍五入为 490pt。在默认情况下 |heightrounded=false|。
% \end{Options}
%
% 图~\ref{fig:modes}~显示了不同布局模式(layout modes)的各种布局。页眉和页脚的尺寸
% 可以通过 |nohead| 或 |nofoot| 模型控制,该模型将每个长度直接设置为 0pt。
% 另一方面,前缀为“|ignore| (忽略)”的选项{\kaiti 不会}更改相应的原始尺寸(native dimensions)。
%
% \vspace{1em}
%
% \begin{figure}[H]
%  \centering\small
%  {\unitlength=.65pt
%  \begin{picture}(460,525)(0,0)
%  \put( 20,310){\framebox(120,170){}}
%  \put( 20,507){\makebox(120,0)[bl]%
%  {\textbf{(a)}\ \ |includeheadfoot|}}
%  \put( 20,460){\line(1,0){120}}\put( 20,450){\line(1,0){120}}
%  \put( 20,330){\line(1,0){120}}
%  \put( 20,485){\makebox(120,0)[br]{\gpart{total body}\ {\scriptsize (\gpart{版心})}}}
%  \put( 20,335){\makebox(120,0)[bc]{|textwidth| {\tiny |(正文的宽度)|}}}
%  \put(143,470){\makebox(0,0)[l]{|headheight|{\tiny |(页眉高)|}}}
%  \put(143,450){\makebox(0,0)[l]{|headsep|}}
%  \put(143,439){\makebox(0,0)[l]{\tiny |(页眉底与正文顶的间距)|}}
%  \put(143,390){\makebox(0,0)[l]{|textheight|}}
%  \put(143,380){\makebox(0,0)[l]{{\tiny |(正文的高度)|}}}
%  \put(143,320){\makebox(0,0)[l]{|footskip|}}
%  \put(143,309){\makebox(0,0)[l]{\tiny |(正文底与页脚底的间距)|}}
%  \put( 10,461){\makebox(120,20)[bc]{\gpart{head}\ {\scriptsize (\gpart{页眉})}}}
%  \put( 10,320){\makebox(120,140)[c]{\gpart{body}\ {\scriptsize (\gpart{正文})}}}
%  \put( 10,311){\makebox(120,10)[c]{\gpart{foot}\ {\scriptsize (\gpart{页脚})}}}
%  \put(250,310){\framebox(120,170){}}
%  \put(250,507){\makebox(120,0)[bl]%
%  {\textbf{(b)}\ \ |includeall|}}
%  \put(250,460){\line(1,0){95}}\put(250,450){\line(1,0){95}}
%  \put(250,330){\line(1,0){95}}\put(345,330){\line(0,1){120}}
%  \put(350,330){\line(0,1){120}}\put(350,450){\line(1,0){20}}
%  \put(350,330){\line(1,0){20}}
%  \put(250,485){\makebox(120,0)[br]{\gpart{total body}\ {\scriptsize (\gpart{版心})}}}
%  \put(250,461){\makebox(95,20)[bc]{\gpart{head}\ {\scriptsize (\gpart{页眉})}}}
%  \put(250,320){\makebox(95,140)[c]{\gpart{body}\ {\scriptsize (\gpart{正文})}}}
%  \put(385,390){\makebox(95,0)[cl]%
%  {\gpart{\shortstack[l]{marginal note}}\ {\scriptsize (\gpart{旁注})}}}
%  \put(250,311){\makebox(95,10)[c]{\gpart{foot}\ {\scriptsize (\gpart{页脚})}}}
%  \put(250,335){\makebox(95,0)[bc]{|textwidth|{\tiny |(正文宽)|}}}
%  \multiput(360, 390)(4,0){6}{\line(1,0){2}}
%  \multiput(348,333)(0,-4){12}{\line(0,1){2}}
%  \multiput(360,333)(0,-4){8}{\line(0,1){2}}
%  \put(355,292){\makebox(0,0)[bl]{|marginparwidth| {\tiny |(旁注的宽度)|}}}
%  \put(345,275){\makebox(0,0)[bl]{|marginparsep| {\tiny |(旁注与正文的间距)|}}}
%  \put( 20, 40){\framebox(120,170){}}
%  \put( 20,237){\makebox(120,0)[bl]%
%  {\textbf{(c)}\ \ |includefoot|}}
%  \put( 20, 60){\line(1,0){120}}
%  \put( 20,215){\makebox(120,0)[br]{\gpart{total body}\ {\scriptsize (\gpart{版心})}}}
%  \put(143,130){\makebox(0,0)[l]{|textheight|}}
%  \put(143,120){\makebox(0,0)[l]{{\tiny |(正文的高度)|}}}
%  \put(143, 50){\makebox(0,0)[l]{|footskip|}}
%  \put(143, 39){\makebox(0,0)[l]{\tiny |(正文底与页脚底的间距)|}}
%  \put( 20, 50){\makebox(120,160)[c]{\gpart{body}\ {\scriptsize (\gpart{正文})}}}
%  \put( 20, 41){\makebox(120,10)[c]{\gpart{foot}\ {\scriptsize (\gpart{页脚})}}}
%  \put( 20, 65){\makebox(120,10)[c]{|textwidth| {\tiny |(正文的宽度)|}}}
%  \put(250, 40){\framebox(120,170){}}
%  \put(250,237){\makebox(120,0)[bl]%
%  {\textbf{(d)}\ \ |includefoot,includemp|}}
%  \put(250, 60){\line(1,0){95}}\put(350, 60){\line(1,0){20}}
%  \put(250,215){\makebox(120,0)[br]{\gpart{total body}\ {\scriptsize (\gpart{版心})}}}
%  \put(250, 50){\makebox(95,160)[c]{\gpart{body}\ {\scriptsize (\gpart{正文})}}}
%  \put(385,130){\makebox(95,0)[cl]%
%  {\gpart{\shortstack[l]{marginal note}}\ {\scriptsize (\gpart{旁注})}}}
%  \put(250, 41){\makebox(95,10)[c]{\gpart{foot}\ {\scriptsize (\gpart{页脚})}}}
%  \put(250, 65){\makebox(95,0)[bc]{|textwidth|{\tiny |(正文宽)|}}}
%  \put(345, 60){\line(0,1){150}}\put(350, 60){\line(0,1){150}}
%  \multiput(360, 130)(4,0){6}{\line(1,0){2}}
%  \multiput(348, 63)(0,-4){12}{\line(0,1){2}}
%  \multiput(360, 63)(0,-4){8}{\line(0,1){2}}
%  \put(355,22){\makebox(0,0)[bl]{|marginparwidth| {\tiny |(旁注的宽度)|}}}
%  \put(345, 5){\makebox(0,0)[bl]{|marginparsep| {\tiny |(旁注与正文的间距)|}}}
%  \end{picture}}
%  \caption[具有不同布局模式的 \gpart{total body}\ 的布局示例]{%
%  \begin{minipage}[t]{.8\textwidth}\small
%    具有不同选项(switches)的 \gpart{total body}\ \ (版心、可打印区)的布局示例。\\
%    {\kaiti (a) |includeheadfoot|,(b) |includeall|,(c) |includefoot|,(d) |includefoot,includemp|。
%    如果将 |reversemp| 设置为 |true|,则在每页上都会交换旁注(marginal notes)的位置。
%    |twoside| 选项交换版口(margins)和旁注在反面页(verso pages)上的位置。请注意,
%    如果有旁注,不管是有 |ignoremp| 或 |includemp=false|,还是在某些情况下超出了页面的范围,
%    旁注都会被打印出来。}
%  \end{minipage}}
%  \label{fig:modes}
% \end{figure}
%
% \vspace{1em}
%
% 以下选项可以使用大括号中的三个逗号分隔的值同时设定正文(body)和版口(margins)。
%
% \begin{Options}
% \item[hdivide]
%   水平方向的参数(left,width,right)。|hdivide=|\argiii{left margin}{width}{right margin}。
%   请注意,不应指定所有三个参数。使用此选项的最佳方法是指定三个参数中的两个,
%   并将剩下的一个设置为 null (nothing)或“|*|”。例如,当您设置 |hdivide={2cm,15cm, }| 时,
%   页面右边缘的(right-side edge of page)版口会被自动设置成 |paperwidth-2cm-15cm|。
% \item[vdivide]
%   垂直方向的参数(top,height,bottom)。\\ |vdivide=|\argiii{top margin}{height}{bottom margin}。
% \item[divide]
%   |divide=|\vargiii{$A$}{$B$}{$C$}\ 被解释为 |hdivide=|\vargiii{$A$}{$B$}{$C$}\ 和\ |vdivide=|\vargiii{$A$}{$B$}{$C$}。
% \end{Options}
%
% \subsection[版口尺寸]{\heiti 版口尺寸}\label{sec:margin}
%
% 下面列出了指定版口尺寸(size of the margins)的选项:
%
% \begin{Options}
% \item[left\OR lmargin\OR inner]~\\
%    \gpart{total body}\ (版心、可打印区)的订口(left margin)(对于单开面[oneside])或
%    里口(inner margin)(对于双开面[twoside])。换言之,也就是纸张左(内)边缘与
%    版心(\gpart{total body})最左边之间的距离。|left=|\meta{length}。|inner| 没有特殊的含义,
%    只是 |left| 和 |lmargin| 的别名。
% \item[right\OR rmargin\OR outer]~\\
%    \gpart{total body}\ (版心、可打印区)的切口(right)或外口(outer margin)。|right=|\meta{length}。
% \item[top\OR tmargin]
%    页面的天头(top margin)。|top=|\meta{length}。请注意,此选项与原始尺寸(native dimension)\ \cs{topmargin}\ 无关。
% \item[bottom\OR bmargin]~\\
%    页面的地脚(bottom margin)。|bottom=|\meta{length}。
% \item[hmargin] 订口(left margin)和切口(right margin)。|hmargin=|\argii{left margin}{right margin}\ 或 |hmargin=|\meta{length}。
% \item[vmargin]
%   天头(top margin)和地脚(bottom margin)。|vmargin=|\argii{top margin}{bottom margin} 或 |vmargin=|\meta{length}。
% \item[margin]
%   |margin=|\vargii{$A$}{$B$}\ 等同于 |hmargin=|\vargii{$A$}{$B$}\ 和 |vmargin=|\vargii{$A$}{$B$}。|margin=|$A$ 自动展开为 |hmargin=|$A$ 和 |vmargin=|$A$。
% \item[hmarginratio]
%    |left| (inner) 与 |right| (outer)即订口(里口)与切口(外口)的水平版口比例(horizontal margin ratio)。
%    应使用冒号分隔的两个值指定 \meta{ratio}\ 的值。每个值都应该是小于 100 的正整数,
%    以防止算术溢出(arithmetic overflow),例如,|2:3| 而不是 |1:1.5|。
%    默认比例为单开面 |1:1|,双开面 |2:3|。
% \item[vmarginratio]
%    |top| 与 |bottom| 即天头与地脚的水平版口比例(vertical margin ratio)。默认比例为 |2:3|。
% \item[marginratio\OR ratio]~\\
%   水平版口和垂直版口的比例。|marginratio=|\argii{horizontal ratio}{vertical ratio}\ 或 |marginratio=|\meta{ratio}。
% \item[hcentering]
%   设置水平方向自动居中(auto-centering),相当于 |hmarginratio=1:1|。单开面默认设置为 |true|。另请参见 |hmarginratio|。
% \item[vcentering]
%   设置垂直方向自动居中,相当于 |vmarginratio=1:1|。默认值为 |false|。另请参见 |vmarginratio|。
% \item[centering]
%   设置自动居中,相当于 |marginratio=1:1|。另请参见 |marginratio|。默认值为 |false|。另请参见 |marginratio|。
% \item[twoside]
%   切换到双开面模式(twoside mode),这时订口(left margins)和切口(right margins)的设置会翻页互换(swapped on verso pages)。
%   该选项设置 \cs{@twoside}\ 和 \cs{@mparswitch}\ 的切换。另请参见 |asymmetric|。
% \item[asymmetric]
%   实现了一种双开面布局(twosided layout),在这种布局中,订口和切口不会因翻页而互换位置
%   (通过将 \cs{oddsidemargin}\ 设置为 \cs{evensidemargin} |+| |bindingoffset|),
%   并且旁注(marginal notes)始终保持在同一侧。此选项可用作双开面选项(twoside option)的替代方案。
%   另请参见 |twoside|。
% \item[bindingoffset]~\\
%   从页面的左手侧(lefthand-side of the page)删除指定的空白(对于单开面),或从内侧删除指定的空白(对于双开面)。
%   |bindingoffset=|\meta{length}。如果页面通过压装装订(press binding)(胶粘、缝合、扒钉\ldots),
%   这很有用。见图~\ref{fig:bindingoffset}。
% \item[hdivide] 参见第~\ref{sec:body}~节中的说明。
% \item[vdivide] 参见第~\ref{sec:body}~节中的说明。
% \item[divide] 参见第~\ref{sec:body}~节中的说明。
%
% \vspace{1em}
%
% \end{Options}
% \begin{figure}[H]
%  \centering\small
%  {\unitlength=.65pt
%  \begin{picture}(500,270)(0,0)
%  \put(20,0){\framebox(170,230){}}
%  \put(20,265){\makebox(80,20)[l]{\textbf{a)}\ \ \ 单开面(oneside)的每页或}}
%  \put(20,245){\makebox(80,20)[l]{\hspace{3.5ex}双开面(twoside)的奇数页}}
%  \put(110,227){\makebox(80,20)[r]{\gpart{paper}\ {\scriptsize (\gpart{纸张})}}}
%  \put(55,36){\framebox(110,170)[tc]{\gpart{total body}}}
%  \put(55,21){\makebox(110,170)[tc]{\scriptsize (\gpart{版心、可打印区})}}
%  \multiput(38,0)(0,7){33}{\line(0,1){4}}
%  \put(38,100){\vector(1,0){17}}\put(55,100){\vector(-1,0){17}}
%  \put(60,95){\makebox(80,10)[l]{|left| {\scriptsize |(订口)|}}}
%  \put(60,80){\makebox(80,10)[l]{(|inner|,{\scriptsize |里口|})}}
%  \put(165,100){\vector(1,0){25}}\put(190,100){\vector(-1,0){25}}
%  \put(195,95){\makebox(80,10)[l]{|right|{\scriptsize |(切口)|}}}
%  \put(195,80){\makebox(80,10)[l]{(|outer|,{\scriptsize |外口|})}}
%  \put(20,16){\vector(1,0){18}}
%  \put(45,10){\makebox(80,10)[bl]{|bindingoffset|{\tiny |(装订偏移量)|}}}
%  \put(280,265){\makebox(80,20)[l]{\textbf{b)}\ \ \ 双开面(twoside)的}}
%  \put(280,245){\makebox(80,20)[l]{\hspace{3.5ex}偶数页(back, 背面)}}
%  \put(280,0){\framebox(170,230){}}
%  \put(370,227){\makebox(80,20)[r]{\gpart{paper}\ {\scriptsize (\gpart{纸张})}}}
%  \put(305,36){\framebox(110,170)[tc]{\gpart{total body}}}
%  \put(305,21){\makebox(110,170)[tc]{\scriptsize (\gpart{版心、可打印区})}}
%  \multiput(432,0)(0,7){33}{\line(0,1){4}}
%  \put(280,100){\vector(1,0){25}}\put(305,100){\vector(-1,0){25}}
%  \put(310,110){\makebox(80,10)[l]{\scriptsize|(外口)|}}
%  \put(310,95){\makebox(80,10)[l]{|outer|}}
%  \put(310,80){\makebox(80,10)[l]{(|right|)}}
%  \put(310,65){\makebox(80,10)[l]{\scriptsize|(切口)|}}
%  \put(415,100){\vector(1,0){17}}\put(432,100){\vector(-1,0){17}}
%  \put(373,110){\makebox(80,10)[l]{\scriptsize|(里口)|}}
%  \put(373,95){\makebox(80,10)[l]{|inner|}}
%  \put(373,80){\makebox(80,10)[l]{(|left|)}}
%  \put(373,65){\makebox(80,10)[l]{\scriptsize|(订口)|}}
%  \put(450,16){\vector(-1,0){18}}
%  \put(289,10){\makebox(80,10)[bl]{{\tiny |(装订偏移量)|}|bindingoffset|}}
%  \end{picture}}
%  \vspace{1em}
%  \caption[\texttt{bindingoffset} option]{%
%   \begin{minipage}[t]{.8\textwidth}\raggedright\small
%   |bindingoffset| (装订偏移量)选项将指定的长度添加到订口(里口,inner margin)。\\ {\kaiti 注意:
%   |twoside| (双开面)选项在偶数页(even pages)会交换水平版口(horizontal margins)和
%   旁注(marginal notes)(连同 |bindingoffset|)的位置(参见 \textbf{b)})。
%   但是 |asymmetric| (不对称)这个选项不会交换版口和旁注的位置,但仍然会交换 |bindingoffset| 的位置。}
%   \end{minipage}}
%  \label{fig:bindingoffset}
% \end{figure}
%
% \subsection[原始尺寸]{\heiti 原始尺寸}\label{sec:dimension}
%
% 下面的选项会覆盖 \LaTeX\ 的原始尺寸(native dimensions)来改变(switches for)
% 页面布局(page layout)(请参见图~\ref{fig:layout}~中的右侧部分)。
%
% \begin{Options}
% \item[headheight\OR head]~\\
%    修改 \cs{headheight},即页眉的高度(height of header)。|headheight=|\meta{length}\ 或 |head=|\meta{length}。
% \item[headsep]
%    修改 \cs{headsep},即页眉底和正文(body)顶的间距。|headsep=|\meta{length}。
% \item[footskip\OR foot]~\\
%    修改 \cs{footskip},即正文最后一行文本的基线(baseline)与页脚基线的间距[译者注:正文底与页脚底的间距]。
%    |footskip=|\meta{length}\ 或 |foot=|\meta{length}。
% \item[nohead]
%    消除(eliminates)页眉(head),这等同于 \cs{headheight}|=0pt| 和 \cs{headsep}|=0pt|。
% \item[nofoot] eliminates spaces for the foot of the page, which is
%    equivalent to \cs{footskip}|=0pt|.
%    消除(eliminates)页脚(foot),这等同于 \cs{footskip}|=0pt|。
% \item[noheadfoot]
%    等同于 |nohead| 和 |nofoot|,这意味着将 \cs{headheight}、\cs{headsep}\ 和 \cs{footskip}\ 都设置为 |0pt|。
% \item[footnotesep]
%    更改 \cs{skip}\cs{footins},即正文文本底部(bottom of text body)和脚注文本顶部(top of footnote text)之间的间距。
% \item[marginparwidth\OR marginpar]~\\
%    更改 \cs{marginparwidth},即旁注(marginal notes)的宽度。|marginparwidth=|\meta{length}。
% \item[marginparsep]
%    修改 \cs{marginparsep},即旁注(marginal notes)与正文(body)的间距。|marginparsep=|\meta{length}。
% \item[nomarginpar]
%    将旁注的空间(spaces for marginal notes)缩小为 0pt,这等同于 \cs{marginparwidth}|=0pt| 和 \cs{marginparsep}|=0pt|。
% \item[columnsep]
%    修改 \cs{columnsep},即在双栏模式(|twocolumn| mode)中两栏的间距。
% \item[hoffset]  修改 \cs{hoffset},即水平偏移量(horizontal offset)。|hoffset=|\meta{length}。
% \item[voffset]  修改 \cs{voffset},即垂直偏移量(vertical offset)。|voffset=|\meta{length}。
% \item[offset] 水平偏移量(horizontal offset)和垂直偏移量(vertical offset)。\\
%    |offset=|\argii{hoffset}{voffset}\ 或 |offset=|\meta{length}.
% \item[twocolumn]
%   使用 \cs{@twocolumntrue}\ 设置双栏模式(|twocolumn| mode)。|twocolumn=false| 表示
%   具有 \cs{@twocolumnfalse}\ 的单栏模式(|onecolumn| mode)。您可以指定 |onecolumn| (默认为 |true|),
%   而不是 |twocolumn=false|。
% \item[onecolumn]
%   和 |twocolumn=false| 作用相同。另一方面,|onecolumn=false| 和 |twocolumn| 作用相同。
% \item[twoside]
%   同时设置 \cs{@twosidetrue}\ 和 \cs{@mparswitchtrue}。见第~\ref{sec:margin}~节。
% \item[textwidth] 直接设置 \cs{textwidth},即正文宽度。见第~\ref{sec:body}~节。
% \item[textheight] 直接设置 \cs{textheight},即正文高度。见第~\ref{sec:body}~节。
% \item[reversemp\OR reversemarginpar]~\\
%   使用 \cs{@reversemargintrue}\ 使旁注(marginal notes)显示在左(内)版口即订口(里口)位置。
%   该选项不会更改 |includemp| 模式(|includemp| mode)。默认设置为 |false|。
% \end{Options}
%
% \subsection[驱动程序]{\heiti 驱动程序}\label{sec:drivers}
%
% 该宏包支持的驱动程序(drivers)有:|dvips|、|dvipdfm|、|pdftex|、|luatex|、|xetex| 和 |vtex|。
% 您可以通过设置 |dvipdfm| 来使用 \textsf{dvipdfmx}\ 和 \textsf{xdvipdfmx},
% |dvipdfmx| 和 |xdvipdfmx| 是\ |dvipdfm|\ 选项的两个别名当然也是受支持的。
% 您还可以通过设置 |pdftex| 来使用 \textsf{pdflatex},设置 |vtex| 来使用 V\TeX{}\ 环境。
% 驱动程序选项是专用的(exclusive)。驱动程序可以通过 |driver=|\meta{driver name}\ 来设置,
% 也可以通过像 |pdftex| 这样的驱动程序的名称来直接设置。\Gm\ 宏包默认支持这些驱动程序。
% 因此,在大多数情况下您无需设置驱动程序。但是,如果要使用 |dvipdfm|,则应显式指定(specify it explicitly)它。
%
% \begin{Options}
% \item[\onlypre driver]
% 用 |driver=|\meta{driver name}\ 来指定驱动程序(driver)。|dvips|、|dvipdfm|、|pdftex|、
% |luatex|、|vtex|、|xetex|、|auto| 和 |none| 均可用作驱动程序的名称(driver name)。
% 上述名称 |auto| 和 |none| 必须通过 |driver=| 来指定,剩余的其它名称都可以不通过 |driver=| 来直接指定。
% |driver=auto| 让系统自动决定(auto-detection)当前使用的驱动程序,无论之前的设置是什么。
% |driver=none| 禁用自动决定(auto-detection)并设置为无驱动程序(no driver),
% 从而不让系统自动决定使用的驱动程序,当您希望让其他宏包不在当前的驱动程序设定(driver setting)
% 下工作时,这可能很有用。例如,如果要将 \textsf{crop}\ 宏包与 \Gm\ 一起使用,
% 应在 \textsf{crop}\ 宏包之前调用 |\usepackage[driver=none]{geometry}|。
% \item[\onlypre dvips]
%     使用 \cs{special}\ 宏(macro)在 dvi 输出(dvi output)中写入纸张尺寸(paper size)。
%     例如,如果您使用 \textsl{dvips}\ 作为 DVI 转 PS 驱动程序(DVI-to-PS driver),
%     通过调用 |\geometry{a3paper,landscape}| 在 A3 纸张上横向(landscape orientation)
%     打印一个文档,则不需要在 \textsl{dvips}\ 中设置“|-t a3 -t landscape|”这个选项。
% \item[\onlypre dvipdfm]
%      除了水平方向校正(landscape correction)外,它的工作方式与 |dvips| 类似。
%      使用 \textsf{dvipdfmx}\ 和 \textsf{xdvipdfmx}\ 来处理 dvi 输出(dvi output)时,可以设置此选项。
% \item[\onlypre pdftex]
%      在内部自动设置 \cs{pdfpagewidth}\ 和 \cs{pdfpageheight}。
% \item[\onlypre luatex]
%      在内部自动设置 \cs{pagewidth}\ 和 \cs{pageheight}。
% \item[\onlypre xetex]
%     与 |pdftex| 相同,只是忽略了 \XeLaTeX{}\ 中未定义的 |\pdf{h,v}origin|。
%     在版本5中引入了此选项。请注意在 \TeX{} Live\ 中已不再需要“geometry.cfg”,
%     在 \TeX{} Live\ 中禁用自动检测例程(auto-detection routine)和设置 |pdftex|,
%     但即使未删除它也不会产生问题。如果您想使用在 \XeTeX{}\ 中 dvipdfm 的一些特殊功能,
%     除了使用 |xetex|,也可以在 \XeLaTeX{}\ 中使用 |dvipdfm| 选项。
% \item[\onlypre vtex]
%     在 V\TeX 中设置 \cs{mediawidth}\ 和 \cs{mediaheight}\ 的值。选择此驱动程序(显式或自动)后,
%     \Gm\ 将自动检测(auto-detect) V\TeX 中选择的输出模式(output mode)(DVI、PDF 或 PS),
%     并对其进行适当设置。
% \end{Options}
% 如果显式驱动程序设置(explicit driver setting)与正在使用的排版程序(typesetting program)不匹配,
% 将选择默认驱动程序 |dvips|。
%
% \subsection[其他选项]{\heiti 其他选项}
%
%  这里介绍了其他有用的选项。
%
% \begin{Options}
% \item[\onlypre verbose]
%   在终端(terminal)上显示参数结果(parameter results)。|verbose=false| (默认值)
%   仍然将它们放入日志文件(log file)中。
% \item[\onlypre reset]
%   设置回布局尺寸(layout dimensions)并切换到加载 \Gm\ 宏包之前的设置。
%   |geometry.cfg| 中给出的选项也被清除。请注意,这不能使用 |truedimen| 重置 |pass| 和 |mag|。
%   |reset=false| 无效,并且无法取消先前的 |reset|(|=true|)(如果有)。例如,当您输入:
%   \begin{quote}
%     |\documentclass[landscape]{article}|\\
%     |\usepackage[twoside,reset,left=2cm]{geometry}|
%   \end{quote}
%   在 |geometry.cfg| 使用 |\ExecuteOptions{scale=0.9}|,那么结果是,
%   |landscape| 和 |left=2cm| 仍然有效,而 |scale=0.9| 和 |twoside| 则无效。
% \item[\onlypre mag]
%   设置放大倍数(magnification value)\ \cs{mag},并根据放大倍数自动修改 \cs{hoffset}\ 和 \cs{voffset}。
%   |mag=|\meta{value}。注意,\meta{value}\ 应该是一个整数值,大小通常为 1000。例如,
%   在 |a4paper| 中设置 |mag=1414|,会提供了一个放大的打印(enlarged print)以适合 |a3paper|,|a3paper| 就
%   是$1.414$(=$\sqrt{2}$)倍的 |a4paper|。字体放大(font enlargement)需要额外的磁盘空间。
%   {\kaiti 请注意,在任何其他具有“true”尺寸设置之前设置 |mag|,例如 |1.5truein|、|2truecm| 等。}
%   另请参见 |truedimen| 选项。
% \item[\onlypre truedimen]
%   将所有内部声明的尺寸值(internal explicit dimension values)更改为 \textit{true}\ 尺寸,
%   例如,将 |1in| 更改为 |1truein|。通常,此选项将与 |mag| 选项一起使用。请注意,
%   这对于外部指定的尺寸(externally specified dimensions)无效。例如,
%   当您设置“\texttt{mag=1440, margin=10pt, truedimen}”时,版口(margins)不是“true”
%   而是放大的(magnified)。如果要设置精确的版口,
%   则应设置为“\texttt{mag=1440, margin=10truept, truedimen}”。
% \item[\onlypre pass]
%   让所有除 |verbose| 和 |showframe| 之外的 geometry 选项和计算结果(calculations)无效。
%   它与顺序无关,可用于检查文档类(documentclass)、其他包宏、不用 \Gm\ 的手动设置等的
%   的页面布局(page layout)。
% \item[\onlypre showframe]
%   显示文本区域(text area)和页面(page)等的可见框架(visible frames),以及第一页中的
%   页眉线(lines for the head)和页脚线(lines for the foot)。
% \item[\onlypre showcrop]
%   在用户指定的布局区域(layout area)的每个角落(corner)打印裁剪标记(crop marks)。
% \end{Options}
%
% \clearpage
%
% \section{\heiti 进程选项}\label{sec:process}
%
% \subsection[加载顺序]{\heiti 加载顺序}\label{sec:loadorder}
%
% \Gm\ 将会首先加载 \TeX{}\ 能够找到的 |geometry.cfg| 文件。例如,在 |geometry.cfg| 中,
% 您可以写入 |\ExecuteOptions{a4paper}|,以将 A4 纸张指定为默认纸张(default paper)。
% 基本上,您可以把 \Gm\ 中定义的所有选项和 |\ExecuteOptions{}| 一起使用。
%
% 在文档前言中,参数的加载顺序如下:
% \begin{enumerate}
%  \item |geometry.cfg|,如果存在。
%  \item 使用 |\documentclass[|\meta{options}|]{...}| 指定的选项。
%  \item 使用 |\usepackage[|\meta{options}|]{geometry}| 指定的选项。
%  \item
%  使用 |\geometry{|\meta{options}|}| 指定的选项,可以被多次调用。\\
%  (|reset| 选项将取消 |\usepackage{geometry}| 或 |\geometry| 中给出的指定选项。)
% \end{enumerate}
%
% \subsection[选项顺序]{\heiti 选项顺序}\label{sec:optionorder}
%
%  \Gm\ 选项的规则(specification)与顺序无关(order-independent),
%  后来的选项会覆盖之前的或者相同的设置,例如:
% \begin{center}
% |[left=2cm, right=3cm]| 等同于 |[right=3cm, left=2cm]|.
% \end{center}
% 多次调用的选项将覆盖以前的设置。例如:
% \begin{center}
%  |[verbose=true, verbose=false]| 结果为 |verbose=false|.
% \end{center}
% |[hmargin={3cm,2cm}, left=1cm]| 与 |hmargin={1cm,2cm}| 相同,
% 其中 left (或 inner)片口被 |left=1cm| 覆盖。
%
% |reset| 和 |mag| 是两个例外。|reset| 选项将使它之前所有的 geometry 选项(|pass| 除外)均失效。
% 如果您设置了:
% \begin{quote}
% |\documentclass[landscape]{article}|\\
% |\usepackage[margin=1cm,twoside]{geometry}|\\
% |\geometry{a5paper, reset, left=2cm}|
% \end{quote}
% 然后 |margin=1cm|、|twoside| 和 |a5paper| 均失效,最终的效果等于:
% \begin{quote}
% |\documentclass[landscape]{article}|\\
% |\usepackage[left=2cm]{geometry}|
% \end{quote}
%
% |mag| 选项应设置在具有“true”长度的任何其他设置之前,例如 |left=1.5truecm|、|width=5truein| 等。
% 可以在调用此宏包之前最早(primitive)设置 |\mag|。
%
% \subsection[优先级别]{\heiti 优先级别}\label{sec:priority}
%
% 有几种方法可以设置正文(\gpart{body})的尺寸(dimensions):|scale| (比例)、|total| (总计)、
% |text| (文本)和 |lines| (行数)。\Gm\ 宏包为更具体的规范(more concrete specification)
% 提供了更高的优先级别(higher priority)。这是 \gpart{body}\ 的优先规则(priority rule):
% \[\begin{array}{c}
% \textrm{优先级(priority):}\qquad\textrm{低(low)}\quad
%    \longrightarrow\quad\textrm{高(high)}\\[1em]
% \left\{\begin{array}{l}|hscale|\ {\scriptsize |(版心与纸张的宽度比)|}\\|vscale|\  {\scriptsize |(版心与纸张的高度比)|}\\|scale|\  {\scriptsize |(版心与纸张的比例)|}
%        \end{array}\right\} <
% \left\{\begin{array}{l}|width|\  {\scriptsize |(版心的宽度)|}\\|height|\  {\scriptsize |(版心的高度)|}\\|total|\  {\scriptsize |(版心的宽和高)|}
%        \end{array}\right\} <
% \left\{\begin{array}{l}|textwidth|\  {\scriptsize |(正文的宽度)|}\\|textheight|\  {\scriptsize |(正文的高度)|}
%         \\|text|\ {\scriptsize |(正文的宽和高)|}\end{array}\right\} < |lines|\ {\scriptsize |(行数)|}
% \end{array}\]
% 例如:
% \begin{quote}
%  |\usepackage[hscale=0.8, textwidth=7in, width=18cm]{geometry}|\ \ 的排版效果与
% \end{quote}
% |\usepackage[textwidth=7in]{geometry}|\ \ 的排版效果相同。\\[20pt] 另一个例子:
% \begin{quote}
%  |\usepackage[lines=30, scale=0.8, text=7in]{geometry}|\ \ 的排版效果是
% \end{quote}
% \texttt{[lines=30, textwidth=7in]}。
%
% \subsection[默认设置]{\heiti 默认设置}\label{sec:defaults}
%
% 本节总结了稍后描述的自动完成(auto-completion)的默认设置(default settings)。
%
% 垂直版口比例(vertical margin ratio)的默认值为 $2/3$,即:
% \begin{equation}
%  |top|\ {\scriptsize |(天头)|} : |bottom|\ {\scriptsize |(地脚)|}  = 2 : 3 \qquad\textit{default}\ {\footnotesize |(默认值)|}
% \end{equation}
% 水平版口比例(horizontal margin ratio)的默认值取决于文档是单开面的(onesided)还是双开面的(twosided):
% \begin{equation}
%  |left|\;(|inner|) : |right|\;(|outer|)
%       = \left\{ \begin{array}{ll}
%              1 : 1 \qquad {\small |(单开面的默认值)|}\\
%              2 : 3 \qquad {\small |(双开面的默认值)|}
%         \end{array}\right.
% \end{equation}
% 显然,对于单开面的文档,水平版口比例(horizontal margin ratio)默认是“居中(centering)”的。
%
% \Gm\ 宏包对于单开面文档(\textit{onesided} documents)具有以下默认设置:
% \begin{itemize}\setlength{\itemsep}{-.5\parsep}
%   \item |scale=0.7|,即:正文(\gpart{body})是纸张(\gpart{paper})的$0.7$倍。
%   \item |marginratio={1:1, 2:3}|,即:水平版口比例是 1:1,垂直版口比例是 2:3。
%   \item |ignoreall|,即:在计算正文(\gpart{body})尺寸时,不包括页眉、页脚及旁注(marginal notes)。
% \end{itemize}
% 对于带有 |twoside| 选项的双开面文档(\textit{twosided} document),默认设置与单开面相同,
% 只是水平版口比例(horizontal margin ratio)也设置为 |2:3|。
%
% 其他选项(additional options)将覆盖以前指定的尺寸(dimensions)。
%
% \subsection[自动完成]{\heiti 自动完成} \label{sec:rules}
%
% 图~\ref{fig:specrule}~示意性地显示了存在多少定义模式(specification patterns)以及
% 如何解决定义的模糊性(ambiguity of the specifications)。每个轴表示了明确地
% 给正文(body)和版口(margins)定义长度的次数。\Ss($m$,$b$) 代表一种定义(specification),
% 它用一组数字 $(\gpart{margin},\gpart{body})=(m,b)$ 来表示。
%
% 例如,定义 |width=14cm,left=3cm| 被归到 \Ss(1,1) 类中,这是一个完整的定义(adequate specification)。
% 如果您加上 |right=4cm|,它将被归到 \Ss(2,1) 类中,也就会产生指定过多(overspecified)的情况。
% 如果只给出 |width=14cm|,则被归到 \Ss(0,1) 类中,也就是未完全指定(underspecified)。
%
% \Gm\ 宏包具有自动完成机制(auto-completion mechanism),在该机制中,
% 如果布局参数(layout parameters)未完全指定(underspecified)或指定过多(overspecified),
% \Gm\ 将使用默认值和其他关系来解决歧义(ambiguity)。以下是定义(specifications)和
% 完成(completion)规则:
% \vspace{-1em}
% \begin{figure}[H]
%  \centering
%  {\unitlength=1pt
%  \begin{picture}(400,150)(40,0)
%  \put(1,49){\makebox(90,49)[r]{\large 0}}
%  \put(1,1){\makebox(70,99)[r]{\gpart{正文} \large (\gpart{body})}}
%  \put(1,1){\makebox(90,49)[r]{\large 1}}
%  \put(100,100){\makebox(99,20){\large 0}}
%  \put(100,50){\framebox(99,49){}}
%  \put(100,80){\makebox(99,15){\Ss(0,0)}}
%  \put(100,50){\makebox(99,49){{\scriptsize 使用} |scale|}}
%  {\linethickness{1pt}%
%  \put(150,35){\line(0,1){30}}
%  \put(150,35){\line(-1,3){4}}
%  \put(150,35){\line(1,3){4}}}
%  \put(100,0){\framebox(99,49){}}
%  \put(100,2){\makebox(99,15){\Ss(0,1)}}
%  \put(100,0){\makebox(99,49){{\scriptsize 使用} |marginratio|}}
%  \put(200,120){\makebox(99,12){\gpart{版口} \large (\gpart{margin})}}
%  \put(200,100){\makebox(99,20){\large 1}}
%  \put(200,50){\framebox(99,49){{\scriptsize 使用} |scale|}}
%  \put(200,55){\makebox(89,10)[r]{\scriptsize\shortstack[l]{\tiny 如果指定了\\\tiny 比例(|ratio|)}}}
%  \put(200,2){\makebox(99,15){\Ss(1,1)}}
%  {\linethickness{1pt}%
%  \put(250,35){\line(0,1){30}}
%  \put(250,35){\line(-1,3){4}}
%  \put(250,35){\line(1,3){4}}}
%  {\linethickness{1pt}%
%  \put(225,25){\line(-1,0){35}}
%  \put(225,25){\line(-3,-1){12}}
%  \put(225,25){\line(-3,1){12}}}
%  \put(200,80){\makebox(99,15){\Ss(1,0)}}
%  \put(200,0){\framebox(99,49){\textcolor{red}{\ \ \ \ \ \ \ \ \ \ \ \footnotesize 可以求解(\textit{solvable})}}}
%  \put(300,100){\makebox(99,20){\large 2}}
%  \put(300,80){\makebox(99,15){\Ss(2,0)}}
%  \put(300,50){\framebox(99,49){\textcolor{red}{\ \ \ \ \ \footnotesize 可以求解(\textit{solvable})}}}
%  \put(300,0){\framebox(99,49){forget |body|}}
%  \put(300,2){\makebox(99,15){\Ss(2,1)}}
%  {\linethickness{1pt}%
%  \multiput(290,65)(5,0){6}{\line(1,0){3}}
%  \put(320,65){\line(-3,-1){12}}
%  \put(320,65){\line(-3,1){12}}}
%  {\linethickness{1pt}%
%  \put(350,65){\line(0,-1){30}}
%  \put(350,65){\line(-1,-3){4}}
%  \put(350,65){\line(1,-3){4}}}
%  \end{picture}}
%  \caption[Specifications and completion rules]{%
%  \begin{minipage}[t]{.7\textwidth}\raggedright\small
%  定义(specifications)\ \ \Ss(0,0) 到 \Ss(2,1)和完成规则(箭头)。\\
%  {\kaiti 列号(column numbers)和行号(row numbers)分别表示版口(margin)和正文(body)的明确指定(explicitly specified)长度(lengths)的次数。
%  \Ss($m$,$b$) 代表一种定义(specification),它用一组数字 $(\gpart{margin},\gpart{body})=(m,b)$ 来表示。}
%  \end{minipage}}
%  \label{fig:specrule}
% \end{figure}
%
% \begin{Spec}
% \item[\Ss(0,0)]
% 未指定任何内容。\Gm\ 宏包将正文(\gpart{body})设置为默认 |scale| ($=0.7$)。例如,
% |width| 设置为 $|0.7|\times|layoutwidth|$。请注意,默认情况下,
% |layoutwidth| 和 |layoutheight| 将分别等于 |\paperwidth| 和 |\paperheight|。
% 因此 \Ss(0,0) 变为 \Ss(0,1)。参见 \Ss(0,1)。\\[3pt]
% {\footnotesize 译者注:|scale| 为正文(\gpart{body})和纸张(\gpart{paper})的比例;|width| 为版心宽度;|height| 为版心高度;
% |layoutwidth| \\[-5pt] \hspace{3.6em} 为布局宽度;|layoutheight| 为布局高度;|\paperwidth| 为纸张宽度;
% |\paperheight| 为纸张高度。}
% \bigskip
%
% \item[\Ss(0,1)]
% 指定正文(\gpart{body}),例如 |width=7in|、|lines=20|、|body={20cm,24cm}|、|scale=0.9| 等等。
% 然后,\Gm\ 使用版口比例(margin ratio)设置版口。如果未指定版口比例,则使用默认值。
% 垂直版口比例(default vertical margin ratio)默认值的定义为:
% \begin{equation}
%  |top|\ {\scriptsize |(天头)|} : |bottom|\ {\scriptsize |(地脚)|}  = 2 : 3 \qquad\textit{default}\ {\footnotesize |(默认值)|}
% \end{equation}
% 水平版口比例(horizontal margin ratio)的默认值取决于文档是单开面的(onesided)还是双开面的(twosided):
% \begin{equation}
%  |left|\;(|inner|) : |right|\;(|outer|)
%       = \left\{ \begin{array}{ll}
%              1 : 1 \qquad {\small |(单开面的默认值)|}\\
%              2 : 3 \qquad {\small |(双开面的默认值)|}
%         \end{array}\right.
% \end{equation}
% 例如,如果在 A4 纸上指定 |height=22cm|,\Gm\ 将按如下方式计算天头(|top|):
% \begin{equation}
%   \begin{array}{ll}
%   |top| &= ( |layoutheight| - |height| ) \times 2/5 \\
%         &= (29.7-22)\times2/5 = 3.08\textrm{(cm)}
%   \end{array}
% \end{equation}
% 因此,已经确定了天头(|top| margin)和正文高度(body |height|),
% 垂直版口定义(specification for the vertical)将变为 \Ss(1,1),所有参数都可以求解(solved)。
% \bigskip
%
% \item[\Ss(1,0)]
% 只指定了一个版口(margin),例如 |bottom=2cm|、|left=1in|、|top=3cm| 等等。
%
% \begin{itemize}
% \item
% \textbf{如果{\kaiti 没有}指定版口比例(margin ratio)}:\Gm\ 将使用默认 |scale| ($=0.7$) 来设置正文(\gpart{body})。
% 例如,如果指定了 |top=2.4cm|,则 \Gm\ 设置:
% \begin{center}
%     $|height|= |0.7|\times|layoutheight|$
%            ~~($=|0.7\paperheight|$ by default),
% \end{center}
% 然后 \Ss(1,0) 变为 \Ss(1,1),其中地脚(|bottom|)用 $|layoutheight|-(|height|+|top|)$ 来计算,
% 如果布局尺寸(layout size)等于纸张尺寸(paper size),则在 A4 纸上的结果为 6.51cm。
% \medskip
% \item
% \textbf{如果指定了版口比例(margin ratio)}:例如 |hmarginratio={1:2}|, |vratio={3:4}| 等等,
% 则 \Gm\ 将使用指定的版口比例设置其他版口。例如,如果指定了一组选项“|top=2.4cm,vratio={3:4}|”,
% 则 \Gm\ 将地脚(|bottom|)设置为 3.2cm:
% \begin{center}
%     $|bottom|= |top|/3\times4 = 3.2\textrm{cm}$
% \end{center}
% 因此 \Ss(1,0) 变为 \Ss(2,0)。
% \end{itemize}
%
% 请注意,版本 4 或更早版本通过版口比例(margin ratio)来设置其它版口(margin)。因此,
% 在版本5中,使用相同的定义(specification),排版结果将与版本4中的结果不同。例如,
% 如果只指定了 |top=2.4cm|,在版本4或更早版本中会得到 |bottom=2.4cm| 的结果,
% 但在版本5中将得到 |bottom=6.51cm| 的结果。
% \bigskip
%
% \item[\Ss(2,1)]
%  正文(\gpart{body})和两个版口(\gpart{margins})都已指定,例如 |vdivide={1in,8in,1.5in}|,
%  “|left=3cm,width=13cm,right=4cm|”等等。由于 \Gm\ 在尺寸指定过多(dimensions are overspecified)
%  时基本上优先考虑版口(\gpart{margins}),所以 \Gm\ 会忽略(forgets)并重置(resets)正文(\gpart{body})。
%  例如,如果您指定了
% \begin{center}
%    |\usepackage[a4paper,left=3cm,width=13cm,right=4cm]{geometry}|,
% \end{center}
% 则 |width| 将被重置为默认值14厘米,因为 A4纸的宽度是21厘米。
% \end{Spec}
%
% \clearpage
% \section{\heiti 在文档中间改变布局}\label{sec:midchange}
%
% 版本 5 提供了新的命令 \cs{newgeometry\{$\cdots$\}}\ 和 \cs{restoregeometry},
% 这两个命令允许您更改文档中间的页面尺寸(page dimensions)。与前言中的 \cs{geometry}\ 不同,
% \cs{newgeometry}\ 仅在 |\begin{document}| 之后才可用,它重置(resets)了所有指定的选项,
% 但以下与纸张尺寸相关的选项(papersize-related options)除外:|landscape|、|portrait| 和
% 纸张大小选项(如 |papersize|、|paper=a4paper| 等),这些选项不能用 \cs{newgeometry}\ 更改。
%
% \cs{restoregeometry}\ 命令将会恢复(restores)在前言中(|\begin{document}| 之前)
% 用 |\usepackage{geometry}| 和 \cs{geometry}\ 定义的页面布局(page layout)。
%
% 请注意,\cs{newgeometry}\ 和 \cs{restoregeometry}\ 都在调用它们的位置插入 |\clearpage|。
%
% 以下是更改文档中间(mid-document)布局的示例。|hmargin=3cm|\ (|left| 和 |right| 版口为 |3cm|)
% 指定的布局 L1 更改为布局L2\ (|left=3cm|, |right=1cm|,|bottom=0.1cm|)。
% 使用 \cs{restoregeometry}\ 恢复为布局 L1。
%
% \vspace{2em}
% \begin{center}
% \begin{minipage}{.8\textwidth}
% |\usepackage[hmargin=3cm]{geometry}|\\
% |\begin{document}|\\
%  \medskip
%  \hspace{1cm}\fbox{Layout L1}\\
% \medskip
% |\newgeometry{left=3cm,right=1cm,bottom=0.1cm}|\\
% \medskip
%  \hspace{1cm}\fbox{Layout L2 (新的)}\\
% \medskip
% |\restoregeometry|\\
% \medskip
%  \hspace{1cm}\fbox{Layout L1 (已恢复的)}\\
% \medskip
% |\newgeometry{margin=1cm,includefoot}|\\
% \medskip
%  \hspace{1cm}\fbox{Layout L3 (新的)}\\
% \medskip
% |\end{document}|
% \end{minipage}%
% \end{center}
% \vspace{2em}
% \begin{center}
%  \centering\small
%  {\unitlength=.8pt
%  \begin{picture}(450,180)(0,0)
%  \put(0,165){\makebox(95,12){现存的(saved)}}
%  \put(15,135){\framebox(65,10){\gpart{head}\ {\scriptsize (\gpart{页眉})}}}
%  \put(15,60){\framebox(65,70){\gpart{body}\ {\scriptsize (\gpart{正文})}}}
%  \put(15,45){\makebox(65,12){\gpart{foot}\ {\scriptsize (\gpart{页脚})}}}
%  \put(15,45){\line(1,0){65}}
%  \put(0,20){\framebox(95,140){}}
%  \put(0,0){\makebox(95,20){L1}}
%  \put(104,90){\circle*{4}}
%  \put(110,90){\circle*{4}}
%  \put(116,90){\circle*{4}}
%  \put(125,165){\makebox(95,12){\cs{newgeometry}}}
%  \put(140,135){\framebox(71,10){\gpart{head}\ {\scriptsize (\gpart{页眉})}}}
%  \put(140,33){\framebox(71,97){\gpart{body}\ {\scriptsize (\gpart{正文})}}}
%  \put(140,21){\makebox(71,12){\gpart{foot}\ {\scriptsize (\gpart{页脚})}}}
%  \put(125,20){\framebox(95,140){}}
%  \put(125,0){\makebox(95,20){L2 (新的)}}
%  \put(229,90){\circle*{4}}
%  \put(235,90){\circle*{4}}
%  \put(241,90){\circle*{4}}
%  \put(250,165){\makebox(95,12){\cs{restoregeometry}}}
%  \put(265,135){\framebox(65,10){\gpart{head}\ {\scriptsize (\gpart{页眉})}}}
%  \put(265,60){\framebox(65,70){\gpart{body}\ {\scriptsize (\gpart{正文})}}}
%  \put(265,45){\makebox(65,12){\gpart{foot}\ {\scriptsize (\gpart{页脚})}}}
%  \put(265,45){\line(1,0){65}}
%  \put(250,20){\framebox(95,140){}}
%  \put(250,0){\makebox(95,20){L1 (已恢复的)}}
%  \put(354,90){\circle*{4}}
%  \put(360,90){\circle*{4}}
%  \put(366,90){\circle*{4}}
%  \put(375,165){\makebox(95,12){\cs{newgeometry}}}
%  \put(383,41){\framebox(80,111){\gpart{body}\ {\scriptsize (\gpart{正文})}}}
%  \put(383,29){\makebox(80,12){\gpart{foot}\ {\scriptsize (\gpart{页脚})}}}
%  \put(383,29){\line(1,0){80}}
%  \put(375,20){\framebox(95,140){}}
%  \put(375,0){\makebox(95,20){L3 (新的)}}
%  \end{picture}}
% \end{center}
%
% \clearpage
% 如果您想在文档中重复使用(reuse)更多不同的布局,
% 一组命令 |\savegeometry{|\meta{name}|}| 和 |\loadgeometry{|\meta{name}|}| 非常方便。例如:
% \begin{center}
\begin{verbatim}
     \usepackage[hmargin=3cm]{geometry}
     \begin{document}
           L1
     \newgeometry{left=3cm,right=1cm,bottom=0.1cm}
     \savegeometry{L2}
           L2 (new, saved)
     \restoregeometry
           L1 (restored)
     \newgeometry{margin=1cm,includefoot}
           L3 (new)
     \loadgeometry{L2}
           L2 (loaded)
     \end{document}
\end{verbatim}
% \end{center}
%
% \clearpage
% \section{\heiti 示例}
%
% \begin{enumerate}
% \item
% 文本区(text area)位于纸张中心的单开面页面布局(onesided page layout)。
% 以下示例具有相同的排版结果,因为默认情况下,单开面的水平版口比例(horizontal margin ratio)设置为 |1:1|。
% \begin{itemize}
%   \item |centering|
%   \item |marginratio=1:1|
%   \item |vcentering|
% \end{itemize}
%
% \item 装订内侧偏移量(inside offset for binding)设置为 |1cm| 的双开面页面布局(twosided page layout)。
% \begin{itemize}
%   \item |twoside, bindingoffset=1cm|
% \end{itemize}
% 在这种情况下,|textwidth| (正文宽度)比默认的双开面文档短 $0.7\times|1cm|$ ($=|0.7cm|$),
% 因为默认的正文(\gpart{body})宽度设置为 |scale=0.7|,
% 这意味着 $|width|=|0.7|\times|layoutwidth|\ $(默认为$=|0.7\paperwidth|$)。
%
% \item
% 一种布局,版口 |left| (订口)、|right| (切口)、|top| (天头)分别为 |3cm|、|2cm|、|2.5in|,|textheight| (正文高度)为 40 行(lines),
% 页眉和页脚包含在版心(\gpart{total body})中。下面的两个示例具有相同的排版结果。
% \begin{itemize}
%   \item |left=3cm, right=2cm, lines=40, top=2.5in, includeheadfoot|
%   \item |hmargin={3cm,2cm}, tmargin=2.5in, lines=40, includeheadfoot|
% \end{itemize}
%
% \item
%  版心(\gpart{total body})高度为 |10in|,地脚(|bottom|)为 |2cm|,|width| (版心宽度)为默认值的布局。
%  将自动计算天头(|top|)。下面的每个解决方案都会产生相同的页面布局(page layout)。
% \begin{itemize}
%     \item |vdivide={*, 10in, 2cm}|
%     \item |bmargin=2cm, height=10in|
%     \item |bottom=2cm, textheight=10in|
% \end{itemize}
% 注意,页眉(\gpart{head})和页脚(\gpart{foot})的尺寸不包含在版心(\gpart{total body})高度(|height|)内。
% 额外的 |includefoot| 使 \cs{footskip}\ (正文底与页脚底的间距)包含在 |totalheight| (版心高度)中。因此,在以下两种情况中,
% 前一种布局中的 |textheight| (正文高度)比后一种布局(精确到10in)短 \cs{footskip}。换句话说,
% 在这种情况下,当 |includefoot=true| 时,|height| = |textheight| + |footskip|。
% \begin{itemize}
%     \item |bmargin=2cm, height=10in, includefoot|
%     \item |bottom=2cm, textheight=10in, includefoot|
% \end{itemize}
%\item
% \glen{textwidth}\ (正文宽度)和 \glen{textheight}\ (正文高度)占 \gpart{paper}\ (纸张)的90\%,
% 且 \gpart{body}\ (正文)居中的布局。只要没有从默认值修改为 |layoutwidth| (布局宽度)
% 和 |layoutheight| (布局高度),下面的每个解决方案都会产生相同的页面布局。
% \begin{itemize}
%   \item |scale=0.9, centering|
%   \item |text={.9\paperwidth,.9\paperheight}, ratio=1:1|
%   \item |width=.9\paperwidth, vmargin=.05\paperheight, marginratio=1:1|
%   \item |hdivide={*,0.9\paperwidth,*}, vdivide={*,0.9\paperheight,*}|
%   (as for onesided documents)
%   \item |margin={0.05\paperwidth,0.05\paperheight}|
% \end{itemize}
% 您可以添加 |heightrounded| 四舍五入,以避免出现像下面这的“underfull vbox warning (未满 vbox 警告)”:
% \begin{quote}\small
%  |Underfull \vbox (badness 10000) has occurred while \output is active|.\\
%  {\scriptsize |译:当 \output 处于活动状态时,\vbox (坏值 10000)未满。|}
% \end{quote}
% 关于 |heightrounded| 的详细描述请参阅第~\ref{sec:body}~节。
%
% \item
% 旁注(marginal notes)的宽度设置为 |3cm| 并且包含在 \gpart{total body}\ (版心)宽度内的一种布局。
% 下面的两个解决方案排版效果相同。
% \begin{itemize}
%   \item |marginparwidth=3cm, includemp|
%   \item |marginpar=3cm, ignoremp=false|
% \end{itemize}
%
% \item
% 一种布局,其中 \gpart{body}\ (正文)占据整个A5纸张,A5纸张为横向(landscape)。
% 下面的两个解决方案排版效果相同。
% \begin{itemize}
%   \item |a5paper, landscape, scale=1.0|
%   \item |landscape=TRUE, paper=a5paper, margin=0pt|
% \end{itemize}
%
% \item  一个屏幕尺寸的布局(screen size layout),它适合于个人电脑(PC)和视频投影仪(video projector)演示。
\begin{verbatim}
   \documentclass{slide}
   \usepackage[screen,margin=0.8in]{geometry}
    ...
   \begin{slide}
      ...
   \end{slide}
\end{verbatim}
% \item 字体(fonts)和空格(spaces)都从A4放大到A3的布局。在下面的例子中,得到的纸张尺寸是A3。
% \begin{itemize}
%     \item |a4paper, mag=1414|.
% \end{itemize}
% 如果您想使用两倍大的字体(two times bigger fonts),但不改变纸张尺寸(paper size),则可以键入:
% \begin{itemize}
%   \item |letterpaper, mag=2000, truedimen|.
% \end{itemize}
%  您可以添加 |dvips| 选项,这对于通过 |dviout| 或 |xdvi| 以适当的纸张尺寸预览它非常有用。
%
% \item  在加载 \Gm\ 之前,更改第一页的布局并将其他页面保留为默认布局。
% 使用 |pass| 选项、|\newgeometry| 和| \restoregeometry|。
\begin{verbatim}
   \documentclass{book}
   \usepackage[pass]{geometry}
      % 'pass' 忽略宏包布局(package layout),因此,原始的'book'布局被存储在这里。
   \begin{document}
   \newgeometry{margin=1cm} % 更改第一页的尺寸。
      Page 1
   \restoregeometry  % 恢复原来的'book'布局。
      Page 2 and more
   \end{document}
\end{verbatim}
%
% \item 一个复杂的页面布局。
\begin{verbatim}
  \usepackage[a5paper, landscape, twocolumn, twoside,
      left=2cm, hmarginratio=2:1, includemp, marginparwidth=43pt,
      bottom=1cm, foot=.7cm, includefoot, textheight=11cm, heightrounded,
      columnsep=1cm, dvips,  verbose]{geometry}
\end{verbatim}
% 试着自己排版并检查排版效果。|:-)|
% \end{enumerate}
%
% \clearpage
% \section{\heiti 已知的问题}
% \begin{itemize}
%  \item 在 |mag| $\neq 1000$ 和 |truedimen| 的情况下,以冗余模式(verbose mode)显示
%  的 |paperwidth| (纸张宽度)和 |paperheight| (纸张高度)与生成的 PDF 的实际尺寸不同。
%  无论如何,PDF 本身是正确的。
%
%  \item 在 |mag| $\neq 1000$,{\kaiti 无}|truedimen| 和 \textsf{hyperref}\ 的情况下,
%  \textsf{hyperref}\ 应在 \Gm\ 之前加载。否则,生成的 PDF 尺寸将出错。
%
%  \item 在 \textsf{crop}\ 宏包和 |mag| $\neq 1000$ 的情况下,
%  \textsf{crop}\ 宏包的 |center| 选项不起作用。
%
% \end{itemize}
%
%
% \clearpage
% \section{\heiti 致谢}
%  感谢许多好心人提出的有益建议和意见,这些好心人(按字母或姓氏的先后顺序排列)包括:
%  \vspace{1em}
%
% \begin{center}
%  \begin{tabular*}{0.7\linewidth}{@{\hspace{40pt}}rl}
%  \hlinew{1.2pt}
%  {\Heiti 人名} &  {\Heiti 中译名}  \\ \hlinew{0.7pt}
%  Jean-Bernard Addor & 让·伯纳德·阿多尔 \\
%  Frank Bennett & 弗兰克·贝内特 \\
%  Alexis Dimitriadis & 亚历克西斯·迪米特里亚迪斯 \\
%  Friedrich Flender & 弗里德里希·弗伦德 \\
%  Adrian Heathcote & 阿德里安·希斯科特 \\
%  Stephan Hennig & 斯蒂芬·亨尼格 \\
%  Morten H\o{}gholm & 莫滕·霍霍姆 \\
%  Jonathan Kew & 乔纳森·邱 \\
%  James Kilfiger & 詹姆斯·基尔费格 \\
%  Yusuke Kuroki & 黑木愈介 \\
%  Jean-Marc Lasgouttes & 让·马克·拉斯高茨 \\
%  Wlodzimierz Macewicz & 沃齐米日·麦克维奇 \\
%  Frank Mittelbach & 弗兰克·米特尔巴赫 \\
%  Eckhard Neber & 埃克哈德·奈伯 \\
%  Rolf Niepraschk & 罗尔夫·尼普拉施克 \\
%  Hans Fr.~Nordhaug &  汉斯·弗罗斯特·诺德豪格 \\
%  Heiko Oberdiek & 海科·奥伯迪克 \\
%  Keith Reckdahl &  基思·雷克达尔 \\
%  Peter Riocreux &  彼得·里奥克鲁克斯 \\
%  Will Robertson & 威尔·罗伯逊 \\
%  Pablo Rodriguez &  巴勃罗·罗德里格斯 \\
%  Nico Schl\"{o}emer & 尼科·施洛默 \\
%  Perry C.~Stearns &  佩里·C·斯特恩斯 \\
%  Frank Stengel &  弗兰克·斯坦格尔 \\
%  Plamen Tanovski &  普拉门·塔诺夫斯基 \\
%  Petr Uher &  皮特·乌赫尔 \\
%  Piet van Oostrum &  彼得·范·奥斯特罗姆 \\
%  Vladimir Volovich &  弗拉基米尔·沃洛维奇 \\
%  Michael Vulis &  迈克尔·武利斯 \\
%  \hlinew{1.2pt}
% \end{tabular*}
\end{center}
% \StopEventually{%
% }
%
% \newgeometry{hmargin={4.2cm,1.5cm},vmargin={1cm,1cm},
%              includeheadfoot, marginpar=3.8cm}
%
% \clearpage
% \section{\heiti 实现}
%    \begin{macrocode}
%<*package>
%    \end{macrocode}
%    此宏包需要以下宏包:\textsf{keyval}、\textsf{ifvtex}。
%    \begin{macrocode}
\RequirePackage{keyval}%
\RequirePackage{ifvtex}%
%    \end{macrocode}
%
%    此处声明了内部开关(internal switches)。
%    \begin{macrocode}
\newif\ifGm@verbose
\newif\ifGm@landscape
\newif\ifGm@swap@papersize
\newif\ifGm@includehead
\newif\ifGm@includefoot
\newif\ifGm@includemp
\newif\ifGm@hbody
\newif\ifGm@vbody
\newif\ifGm@heightrounded
\newif\ifGm@showframe
\newif\ifGm@showcrop
\newif\ifGm@pass
\newif\ifGm@resetpaper
\newif\ifGm@layout
\newif\ifGm@newgm
%    \end{macrocode}
%    \begin{macro}{\Gm@cnth}
%    \begin{macro}{\Gm@cntv}
%    水平(horizontal)和垂直(vertical)分区模式(partitioning patterns)的计数器(counters)。
%    \begin{macrocode}
\newcount\Gm@cnth
\newcount\Gm@cntv
%    \end{macrocode}
%    \end{macro}\end{macro}
%    \begin{macro}{\c@Gm@tempcnt}
%    计数器(counter)用于设置带有 \textsf{calc}\ 的数字(number)。
%    \begin{macrocode}
\newcount\c@Gm@tempcnt
%    \end{macrocode}
%    \end{macro}
%    \begin{macro}{\Gm@bindingoffset}
%    订口(inner margin,里口)的装订偏移量(binding offset)。
%    \begin{macrocode}
\newdimen\Gm@bindingoffset
%    \end{macrocode}
%    \end{macro}
%    \begin{macro}{\Gm@wd@mp}
%    \begin{macro}{\Gm@odd@mp}
%    \begin{macro}{\Gm@even@mp}
%    |includemp| 模式中 \cs{textwidth} (正文宽度)、\cs{oddsidemargin}\ 和 \cs{evensidemargin}\ 的校正\\ 长度(correction lengths)。
%    \begin{macrocode}
\newdimen\Gm@wd@mp
\newdimen\Gm@odd@mp
\newdimen\Gm@even@mp
%    \end{macrocode}
%    \end{macro}\end{macro}\end{macro}
%    \begin{macro}{\Gm@layoutwidth}
%    \begin{macro}{\Gm@layoutheight}
%    \begin{macro}{\Gm@layouthoffset}
%    \begin{macro}{\Gm@layoutvoffset}
%    布局区域(layout area)的尺寸。
%    \begin{macrocode}
\newdimen\Gm@layoutwidth
\newdimen\Gm@layoutheight
\newdimen\Gm@layouthoffset
\newdimen\Gm@layoutvoffset
%    \end{macrocode}
%    \end{macro}\end{macro}\end{macro}\end{macro}
%    \begin{macro}{\Gm@dimlist}
%    可以存储 \LaTeX{}\ 原始尺寸(native dimensions)的令牌(token)。
%    \begin{macrocode}
\newtoks\Gm@dimlist
%    \end{macrocode}
%    \end{macro}
%    \begin{macro}{\Gm@warning}
%    用于打印警告消息(warning messages)的宏。
%    \begin{macrocode}
\def\Gm@warning#1{\PackageWarningNoLine{geometry}{#1}}%
%    \end{macrocode}
%    \end{macro}
%    \begin{macro}{\ifGm@preamble}
%    宏(macro)只有在前言中指定为 |\usepackage| 的选项(option)和/或 |\geometry| 的参数(argument)时,
%    才执行作为参数给出的选项。否则,宏将打印警告消息(warning message)并忽略选项设置(option setting)。
%    \begin{macrocode}
\def\ifGm@preamble#1{%
  \ifGm@newgm
   \Gm@warning{`#1': not available in `\string\newgeometry'; skipped}%
  \else
    \expandafter\@firstofone
  \fi}%
%    \end{macrocode}
%    \end{macro}
%    \begin{macro}{\Gm@Dhratio}
%    \begin{macro}{\Gm@Dhratiotwo}
%    \begin{macro}{\Gm@Dvratio}
%    定义了水平和垂直版口比例(\textsl{marginalratio})的默认值。\cs{Gm@Dhratiotwo}\ 表示
%    双开面页面布局(twoside page layout)的水平版口比例(horizontal \textsl{marginratio})的默认值,
%    并且在反面页(verso pages)上交换订口(left margins,里口)和切口(right margins,外口),
%    这是由 |twoside| 设置的。
%    \begin{macrocode}
\def\Gm@Dhratio{1:1}% = left:right default for oneside
\def\Gm@Dhratiotwo{2:3}% = inner:outer default for twoside.
\def\Gm@Dvratio{2:3}% = top:bottom default
%    \end{macrocode}
%    \end{macro}\end{macro}\end{macro}
%    \begin{macro}{\Gm@Dhscale}
%    \begin{macro}{\Gm@Dvscale}
%    水平和垂直 \textsl{scale}\ 的默认值定义为 $0.7$。
%    \begin{macrocode}
\def\Gm@Dhscale{0.7}%
\def\Gm@Dvscale{0.7}%
%    \end{macrocode}
%    \end{macro}\end{macro}
%    \begin{macro}{\Gm@dvips}%
%    \begin{macro}{\Gm@dvipdfm}%
%    \begin{macro}{\Gm@pdftex}%
%    \begin{macro}{\Gm@luatex}%
%    \begin{macro}{\Gm@xetex}%
%    \begin{macro}{\Gm@vtex}%
%    驱动程序的名称(driver names)。
%    \begin{macrocode}
\def\Gm@dvips{dvips}%
\def\Gm@dvipdfm{dvipdfm}%
\def\Gm@pdftex{pdftex}%
\def\Gm@luatex{luatex}%
\def\Gm@xetex{xetex}%
\def\Gm@vtex{vtex}%
%    \end{macrocode}
%    \end{macro}\end{macro}\end{macro}\end{macro}\end{macro}\end{macro}
%    \begin{macro}{\Gm@true}%
%    \begin{macro}{\Gm@false}%
%    用于 |true| 和 |false| 的宏(macros)。
%    \begin{macrocode}
\def\Gm@true{true}%
\def\Gm@false{false}%
%    \end{macrocode}
%    \end{macro}\end{macro}
%    \begin{macro}{\Gm@orgpw}
%    \begin{macro}{\Gm@orgph}
%    这些宏保持原始(original)纸张(媒体)尺寸不变。
%    \begin{macrocode}
\edef\Gm@orgpw{\the\paperwidth}%
\edef\Gm@orgph{\the\paperheight}%
%    \end{macrocode}
%    \end{macro}\end{macro}
%    \begin{macro}{\Gm@savelength}
%    该宏将指定的长度(length)保存到 |\Gm@restore|。
%    \begin{macrocode}
\def\Gm@savelength#1{%
  \g@addto@macro\Gm@restore{\expandafter\noexpand\expandafter\csname
  #1\endcsname\expandafter=\expandafter\the\csname #1\endcsname\relax}}%
%    \end{macrocode}
%    \end{macro}
%    \begin{macro}{\Gm@saveboolean}
%    该宏将指定的布尔值(boolean)保存到 |\Gm@restore|。
%    \begin{macrocode}
\def\Gm@saveboolean#1{%
  \csname if#1\endcsname
    \g@addto@macro\Gm@restore{\expandafter\noexpand\csname #1true\endcsname}%
  \else
    \g@addto@macro\Gm@restore{\expandafter\noexpand\csname #1false\endcsname}%
  \fi}%
%    \end{macrocode}
%    \end{macro}
%    \begin{macro}{\Gm@restore}
%    |\Gm@restore| 宏的初始化。
%    \begin{macrocode}
\def\Gm@restore{}%
%    \end{macrocode}
%    \end{macro}
%    \begin{macro}{\Gm@save}
%    宏的定义(definition of the macro)保存实际长度的(real lengths)\ \ \LaTeX\ 选项。
%    \begin{macrocode}
\def\Gm@save{%
  \Gm@savelength{paperwidth}%
  \Gm@savelength{paperheight}%
  \Gm@savelength{textwidth}%
  \Gm@savelength{textheight}%
  \Gm@savelength{evensidemargin}%
  \Gm@savelength{oddsidemargin}%
  \Gm@savelength{topmargin}%
  \Gm@savelength{headheight}%
  \Gm@savelength{headsep}%
  \Gm@savelength{topskip}%
  \Gm@savelength{footskip}%
  \Gm@savelength{baselineskip}%
  \Gm@savelength{marginparwidth}%
  \Gm@savelength{marginparsep}%
  \Gm@savelength{columnsep}%
  \Gm@savelength{hoffset}%
  \Gm@savelength{voffset}
  \Gm@savelength{Gm@layoutwidth}%
  \Gm@savelength{Gm@layoutheight}%
  \Gm@savelength{Gm@layouthoffset}%
  \Gm@savelength{Gm@layoutvoffset}%
  \Gm@saveboolean{@twocolumn}%
  \Gm@saveboolean{@twoside}%
  \Gm@saveboolean{@mparswitch}%
  \Gm@saveboolean{@reversemargin}}%
%    \end{macrocode}
%    \end{macro}
%    \begin{macro}{\Gm@initnewgm}
%    宏初始化 |\newgeometry| 中的布局的参数。
%    \begin{macrocode}
\def\Gm@initnewgm{%
  \Gm@passfalse
  \Gm@swap@papersizefalse
  \Gm@dimlist={}
  \Gm@hbodyfalse
  \Gm@vbodyfalse
  \Gm@heightroundedfalse
  \Gm@includeheadfalse
  \Gm@includefootfalse
  \Gm@includempfalse
  \let\Gm@width\@undefined
  \let\Gm@height\@undefined
  \let\Gm@textwidth\@undefined
  \let\Gm@textheight\@undefined
  \let\Gm@lines\@undefined
  \let\Gm@hscale\@undefined
  \let\Gm@vscale\@undefined
  \let\Gm@hmarginratio\@undefined
  \let\Gm@vmarginratio\@undefined
  \let\Gm@lmargin\@undefined
  \let\Gm@rmargin\@undefined
  \let\Gm@tmargin\@undefined
  \let\Gm@bmargin\@undefined
  \Gm@layoutfalse
  \Gm@layouthoffset\z@
  \Gm@layoutvoffset\z@
  \Gm@bindingoffset\z@}%
%    \end{macrocode}
%    \end{macro}
%    \begin{macro}{\Gm@initall}
%    一旦加载该宏,就会调用此初始化(initialization)。一旦指定了 |reset| 选项,也会立即调用该初始化。
%    \begin{macrocode}
\def\Gm@initall{%
  \let\Gm@driver\@empty
  \let\Gm@truedimen\@empty
  \let\Gm@paper\@undefined
  \Gm@resetpaperfalse
  \Gm@landscapefalse
  \Gm@verbosefalse
  \Gm@showframefalse
  \Gm@showcropfalse
  \Gm@newgmfalse
  \Gm@initnewgm}%
%    \end{macrocode}
%    \end{macro}
%    \begin{macro}{\Gm@setdriver}
%    该宏将设置指定的驱动程序(driver)。
%    \begin{macrocode}
\def\Gm@setdriver#1{%
  \expandafter\let\expandafter\Gm@driver\csname Gm@#1\endcsname}%
%    \end{macrocode}
%    \end{macro}
%    \begin{macro}{\Gm@unsetdriver}
%    如果已经设置了指定的驱动程序(driver),则该宏将取消该驱动程序的设置。
%    \begin{macrocode}
\def\Gm@unsetdriver#1{%
  \expandafter\ifx\csname Gm@#1\endcsname\Gm@driver\let\Gm@driver\@empty\fi}%
%    \end{macrocode}
%    \end{macro}
%    \begin{macro}{\Gm@setbool}
%    \begin{macro}{\Gm@setboolrev}
%    该宏用于布尔选项的处理(boolean option processing)。
%    \begin{macrocode}
\def\Gm@setbool{\@dblarg\Gm@@setbool}%
\def\Gm@setboolrev{\@dblarg\Gm@@setboolrev}%
\def\Gm@@setbool[#1]#2#3{\Gm@doif{#1}{#3}{\csname Gm@#2\Gm@bool\endcsname}}%
\def\Gm@@setboolrev[#1]#2#3{\Gm@doifelse{#1}{#3}%
  {\csname Gm@#2\Gm@false\endcsname}{\csname Gm@#2\Gm@true\endcsname}}%
%    \end{macrocode}
%    \end{macro}\end{macro}
%    \begin{macro}{\Gm@doif}
%    \begin{macro}{\Gm@doifelse}
%    \cs{Gm@doif}\ 使用选项 |#1| 的布尔值 |#2| 执行(excutes)第三个参数 |#3|。
%    如果布尔选项 |#1| 的值 |#2| 为 |true|,则 \cs{Gm@doifelse}\ 执行第三个参数 |#3|,
%    如果布尔选项 |#1| 的值 |#2| 为 |false|,则 \cs{Gm@doifelse}\ 执行第四个参数 |#4|。
%    \begin{macrocode}
\def\Gm@doif#1#2#3{%
  \lowercase{\def\Gm@bool{#2}}%
  \ifx\Gm@bool\@empty
    \let\Gm@bool\Gm@true
  \fi
  \ifx\Gm@bool\Gm@true
  \else
    \ifx\Gm@bool\Gm@false
    \else
      \let\Gm@bool\relax
    \fi
  \fi
  \ifx\Gm@bool\relax
    \Gm@warning{`#1' should be set to `true' or `false'}%
  \else
    #3
  \fi}%
\def\Gm@doifelse#1#2#3#4{%
  \Gm@doif{#1}{#2}{\ifx\Gm@bool\Gm@true #3\else #4\fi}}%
%    \end{macrocode}
%    \end{macro}\end{macro}
%    \begin{macro}{\Gm@reverse}
%    宏反转布尔值(reverses a bool value)。
%    \begin{macrocode}
\def\Gm@reverse#1{%
  \csname ifGm@#1\endcsname
  \csname Gm@#1false\endcsname\else\csname Gm@#1true\endcsname\fi}%
%    \end{macrocode}
%    \end{macro}
%    \begin{macro}{\Gm@defbylen}
%    \begin{macro}{\Gm@defbycnt}
%    宏 \cs{Gm@defbylen}\ 和 \cs{Gm@defbycnt}\ 分别通过长度(length)和计数器(counter)
%    与 \textsf{calc}\ 宏包定义 \cs{Gm@xxxx}\ 变量(variables)。
%    \begin{macrocode}
\def\Gm@defbylen#1#2{%
  \begingroup\setlength\@tempdima{#2}%
  \expandafter\xdef\csname Gm@#1\endcsname{\the\@tempdima}\endgroup}%
\def\Gm@defbycnt#1#2{%
  \begingroup\setcounter{Gm@tempcnt}{#2}%
  \expandafter\xdef\csname Gm@#1\endcsname{\the\value{Gm@tempcnt}}\endgroup}%
%    \end{macrocode}
%    \end{macro}\end{macro}
%    \begin{macro}{\Gm@set@ratio}
%    该宏解析(parses)指定版口比例(marginal ratios)的选项的值,
%    该版口比例在 \cs{Gm@setbyratio}\ 宏中使用。
%    \begin{macrocode}
\def\Gm@sep@ratio#1:#2{\@tempcnta=#1\@tempcntb=#2}%
%    \end{macrocode}
%    \end{macro}
%    \begin{macro}{\Gm@setbyratio}
%    该宏确定由 |#4| 计算 |#3|$\times a / b$ 指定的尺寸,其中 $a$ 和 $b$ 由具有 $a:b$ 值的 \cs{Gm@mratio}\ 给出。
%    如果括号中的 |#1| 是 |b|,则 $a$ 和 $b$ 将被交换。第二个参数 |h| 或 |v| 表示
%    水平(horizontal)或垂直(vertical),在这个宏中不使用。
%    \begin{macrocode}
\def\Gm@setbyratio[#1]#2#3#4{% determine #4 by ratio
  \expandafter\Gm@sep@ratio\Gm@mratio\relax
  \if#1b
    \edef\@@tempa{\the\@tempcnta}%
    \@tempcnta=\@tempcntb
    \@tempcntb=\@@tempa\relax
  \fi
  \expandafter\setlength\expandafter\@tempdimb\expandafter
    {\csname Gm@#3\endcsname}%
  \ifnum\@tempcntb>\z@
    \multiply\@tempdimb\@tempcnta
    \divide\@tempdimb\@tempcntb
  \fi
  \expandafter\edef\csname Gm@#4\endcsname{\the\@tempdimb}}%
%    \end{macrocode}
%    \end{macro}
%    \begin{macro}{\Gm@detiv}
%    此宏从 |#1|(\glen{layoutwidth}\ 或 \glen{layoutheight})、|#2| 和 |#3| 中确定第四个长度(|#4|)。
%    它在 \cs{Gm@detall}\ 宏中使用。
%    \begin{macrocode}
\def\Gm@detiv#1#2#3#4{% determine #4.
  \expandafter\setlength\expandafter\@tempdima\expandafter
    {\csname Gm@layout#1\endcsname}%
  \expandafter\setlength\expandafter\@tempdimb\expandafter
    {\csname Gm@#2\endcsname}%
  \addtolength\@tempdima{-\@tempdimb}%
  \expandafter\setlength\expandafter\@tempdimb\expandafter
    {\csname Gm@#3\endcsname}%
  \addtolength\@tempdima{-\@tempdimb}%
  \ifdim\@tempdima<\z@
    \Gm@warning{`#4' results in NEGATIVE (\the\@tempdima).%
    ^^J\@spaces `#2' or `#3' should be shortened in length}%
  \fi
  \expandafter\edef\csname Gm@#4\endcsname{\the\@tempdima}}%
%    \end{macrocode}
%    \end{macro}
%    \begin{macro}{\Gm@detiiandiii}
%    此宏使用第一个参数(|#1|)从 |#1| 确定 |#2| 和 |#3|。第一个参数可以是 |width| 或 |height|,
%    它可以扩展为纸张的尺寸和正文的尺寸。它用于 \cs{Gm@detall}\ 宏。
%    \begin{macrocode}
\def\Gm@detiiandiii#1#2#3{% determine #2 and #3.
  \expandafter\setlength\expandafter\@tempdima\expandafter
    {\csname Gm@layout#1\endcsname}%
  \expandafter\setlength\expandafter\@tempdimb\expandafter
    {\csname Gm@#1\endcsname}%
  \addtolength\@tempdima{-\@tempdimb}%
  \ifdim\@tempdima<\z@
    \Gm@warning{`#2' and `#3' result in NEGATIVE (\the\@tempdima).%
                  ^^J\@spaces `#1' should be shortened in length}%
  \fi
  \ifx\Gm@mratio\@undefined
    \expandafter\Gm@sep@ratio\Gm@Dmratio\relax
  \else
    \expandafter\Gm@sep@ratio\Gm@mratio\relax
    \ifnum\@tempcntb>\z@\else
      \Gm@warning{margin ratio a:b should be non-zero; default used}%
      \expandafter\Gm@sep@ratio\Gm@Dmratio\relax
    \fi
  \fi
  \@tempdimb=\@tempdima
  \advance\@tempcntb\@tempcnta
  \divide\@tempdima\@tempcntb
  \multiply\@tempdima\@tempcnta
  \advance\@tempdimb-\@tempdima
  \expandafter\edef\csname Gm@#2\endcsname{\the\@tempdima}%
  \expandafter\edef\csname Gm@#3\endcsname{\the\@tempdimb}}%
%    \end{macrocode}
%    \end{macro}
%
%    \begin{macro}{\Gm@detall}
%    此宏确定每个方向(each direction)的分区(partition)。第一个参数(|#1|)应该是 |h| 或 |v|,
%    第二个参数(|#2|)应该是 |width| 或 |height|,第三个参数(|#3|)应该是 |lmargin| 或 |top|,
%    最后一个参数 (|#4|)应该是 |rmargin| 或 |bottom|。
%    \begin{macrocode}
\def\Gm@detall#1#2#3#4{%
  \@tempcnta\z@
  \if#1h
    \let\Gm@mratio\Gm@hmarginratio
    \edef\Gm@Dmratio{\if@twoside\Gm@Dhratiotwo\else\Gm@Dhratio\fi}%
  \else
    \let\Gm@mratio\Gm@vmarginratio
    \edef\Gm@Dmratio{\Gm@Dvratio}%
  \fi
%    \end{macrocode}
%    \cs{@tempcnta}\ 被视为一个三位二进制值(three-digit binary value),顶部(top)、
%    中间(middle)和底部(bottom)分别表示为用户指定的 |left|(|top|)、|width|(|height|)和 |right|(|bottom|) 版口(margins)。
%    \begin{macrocode}
  \if#1h
    \ifx\Gm@lmargin\@undefined\else\advance\@tempcnta4\relax\fi
    \ifGm@hbody\advance\@tempcnta2\relax\fi
    \ifx\Gm@rmargin\@undefined\else\advance\@tempcnta1\relax\fi
    \Gm@cnth\@tempcnta
  \else
    \ifx\Gm@tmargin\@undefined\else\advance\@tempcnta4\relax\fi
    \ifGm@vbody\advance\@tempcnta2\relax\fi
    \ifx\Gm@bmargin\@undefined\else\advance\@tempcnta1\relax\fi
    \Gm@cntv\@tempcnta
  \fi
%    \end{macrocode}
%    如果值为 |000| (=0),且没有任何固定值(fixed)(默认):
%    \begin{macrocode}
  \ifcase\@tempcnta
    \if#1h
      \Gm@defbylen{width}{\Gm@Dhscale\Gm@layoutwidth}%
    \else
      \Gm@defbylen{height}{\Gm@Dvscale\Gm@layoutheight}%
    \fi
    \Gm@detiiandiii{#2}{#3}{#4}%
%    \end{macrocode}
%    如果值为 |001| (=1),且有固定值 |right|(|bottom|):
%    \begin{macrocode}
  \or
    \ifx\Gm@mratio\@undefined
      \if#1h
        \Gm@defbylen{width}{\Gm@Dhscale\Gm@layoutwidth}%
      \else
        \Gm@defbylen{height}{\Gm@Dvscale\Gm@layoutheight}%
      \fi
      \setlength\@tempdimc{\@nameuse{Gm@#4}}%
      \Gm@detiiandiii{#2}{#3}{#4}%
      \expandafter\let\csname Gm@#2\endcsname\@undefined
      \Gm@defbylen{#4}{\@tempdimc}%
    \else
      \Gm@setbyratio[f]{#1}{#4}{#3}%
    \fi
    \Gm@detiv{#2}{#3}{#4}{#2}%
%    \end{macrocode}
%    如果值为 |010| (=2),且有固定值 |width|(|height|):
%    \begin{macrocode}
  \or\Gm@detiiandiii{#2}{#3}{#4}%
%    \end{macrocode}
%    如果值为 |011| (=3),且有固定值 |width|(|height|) 和 |right|(|bottom|):
%    \begin{macrocode}
  \or\Gm@detiv{#2}{#2}{#4}{#3}%
%    \end{macrocode}
%    如果值为 |100| (=4),且有固定值 |left|(|top|):
%    \begin{macrocode}
  \or
    \ifx\Gm@mratio\@undefined
      \if#1h
        \Gm@defbylen{width}{\Gm@Dhscale\Gm@layoutwidth}%
      \else
        \Gm@defbylen{height}{\Gm@Dvscale\Gm@layoutheight}%
      \fi
      \setlength\@tempdimc{\@nameuse{Gm@#3}}%
      \Gm@detiiandiii{#2}{#4}{#3}%
      \expandafter\let\csname Gm@#2\endcsname\@undefined
      \Gm@defbylen{#3}{\@tempdimc}%
    \else
      \Gm@setbyratio[b]{#1}{#3}{#4}%
    \fi
    \Gm@detiv{#2}{#3}{#4}{#2}%
%    \end{macrocode}
%    如果值为 |101| (=5),且有固定值 |left|(|top|) 和 |right|(|bottom|):
%    \begin{macrocode}
  \or\Gm@detiv{#2}{#3}{#4}{#2}%
%    \end{macrocode}
%    如果值为 |110| (=6),且有固定值 |left|(|top|) 和 |width|(|height|):
%    \begin{macrocode}
  \or\Gm@detiv{#2}{#2}{#3}{#4}%
%    \end{macrocode}
%    如果值为 |111| (=7),且所有值均固定(fixed),尽管它被过度确定(over-specified):
%    \begin{macrocode}
  \or\Gm@warning{Over-specification in `#1'-direction.%
                  ^^J\@spaces `#2' (\@nameuse{Gm@#2}) is ignored}%
    \Gm@detiv{#2}{#3}{#4}{#2}%
  \else\fi}%
%    \end{macrocode}
%    \end{macro}
%
%    \begin{macro}{\Gm@clean}
%    用于设置未指定尺寸的(unspecified dimensions)宏为 \cs{@undefined}。
%    这是由 \cs{geometry}\ 宏使用的。
%    \begin{macrocode}
\def\Gm@clean{%
  \ifnum\Gm@cnth<4\let\Gm@lmargin\@undefined\fi
  \ifodd\Gm@cnth\else\let\Gm@rmargin\@undefined\fi
  \ifnum\Gm@cntv<4\let\Gm@tmargin\@undefined\fi
  \ifodd\Gm@cntv\else\let\Gm@bmargin\@undefined\fi
  \ifGm@hbody\else
    \let\Gm@hscale\@undefined
    \let\Gm@width\@undefined
    \let\Gm@textwidth\@undefined
  \fi
  \ifGm@vbody\else
    \let\Gm@vscale\@undefined
    \let\Gm@height\@undefined
    \let\Gm@textheight\@undefined
  \fi
  }%
%    \end{macrocode}
%    \end{macro}
%
%    \begin{macro}{\Gm@parse@divide}
%    宏解析 (|h|,|v|)|divide| 选项。
%    \begin{macrocode}
\def\Gm@parse@divide#1#2#3#4{%
  \def\Gm@star{*}%
  \@tempcnta\z@
  \@for\Gm@tmp:=#1\do{%
    \expandafter\KV@@sp@def\expandafter\Gm@frag\expandafter{\Gm@tmp}%
    \edef\Gm@value{\Gm@frag}%
    \ifcase\@tempcnta\relax\edef\Gm@key{#2}%
      \or\edef\Gm@key{#3}%
      \else\edef\Gm@key{#4}%
    \fi
    \@nameuse{Gm@set\Gm@key false}%
    \ifx\empty\Gm@value\else
    \ifx\Gm@star\Gm@value\else
      \setkeys{Gm}{\Gm@key=\Gm@value}%
    \fi\fi
    \advance\@tempcnta\@ne}%
  \let\Gm@star\relax}%
%    \end{macrocode}
%    \end{macro}
%
%    \begin{macro}{\Gm@branch}
%    该宏将一个值分割为相同的两个值。
%    \begin{macrocode}
\def\Gm@branch#1#2#3{%
  \@tempcnta\z@
  \@for\Gm@tmp:=#1\do{%
    \KV@@sp@def\Gm@frag{\Gm@tmp}%
    \edef\Gm@value{\Gm@frag}%
    \ifcase\@tempcnta\relax% cnta == 0
      \setkeys{Gm}{#2=\Gm@value}%
    \or% cnta == 1
      \setkeys{Gm}{#3=\Gm@value}%
    \else\fi
    \advance\@tempcnta\@ne}%
  \ifnum\@tempcnta=\@ne
    \setkeys{Gm}{#3=\Gm@value}%
  \fi}%
%    \end{macrocode}
%    \end{macro}
%
%    \begin{macro}{\Gm@magtooffset}
%    此宏用于按 \cs{mag}\ 调整偏移量(offsets)。
%    \begin{macrocode}
\def\Gm@magtooffset{%
  \@tempdima=\mag\Gm@truedimen sp%
  \@tempdimb=1\Gm@truedimen in%
  \divide\@tempdimb\@tempdima
  \multiply\@tempdimb\@m
  \addtolength{\hoffset}{1\Gm@truedimen in}%
  \addtolength{\voffset}{1\Gm@truedimen in}%
  \addtolength{\hoffset}{-\the\@tempdimb}%
  \addtolength{\voffset}{-\the\@tempdimb}}%
%    \end{macrocode}
%    \end{macro}
%
%    \begin{macro}{\Gm@setlength}
%    此宏存储 \LaTeX{}\ 原始尺寸(native dimensions),这些尺寸将在以后存储(stored)和设置(afterwards)。
%    \begin{macrocode}
\def\Gm@setlength#1#2{%
  \let\Gm@len=\relax\let\Gm@td=\relax
  \edef\addtolist{\noexpand\Gm@dimlist=%
  {\the\Gm@dimlist \Gm@len{#1}{#2}}}\addtolist}%
%    \end{macrocode}
%    \end{macro}
%    \begin{macro}{\Gm@expandlengths}
%    这个宏处理 \cs{Gm@dimlist}。
%    \begin{macrocode}
\def\Gm@expandlengths{%
  \def\Gm@td{\Gm@truedimen}%
  \def\Gm@len##1##2{\setlength{##1}{##2}}%
  \the\Gm@dimlist}%
%    \end{macrocode}
%    \end{macro}
%
%    \begin{macro}{\Gm@setsize}
%    该宏使用 \cs{Gm@setlength}\ 宏设置 |paperwidth| (纸张宽度)和 |paperheight| (纸张高度)。
%    \begin{macrocode}
\def\Gm@setsize#1(#2,#3)#4{%
  \let\Gm@td\relax
  \expandafter\Gm@setlength\csname #1width\endcsname{#2\Gm@td #4}%
  \expandafter\Gm@setlength\csname #1height\endcsname{#3\Gm@td #4}%
  \ifGm@landscape\Gm@swap@papersizetrue\else\Gm@swap@papersizefalse\fi}%
%    \end{macrocode}
%    \end{macro}
%    \begin{macro}{\Gm@setpaper@ifpre}
%    宏将更改纸张尺寸(paper size)。
%    \begin{macrocode}
\def\Gm@setpaper@ifpre#1{%
  \ifGm@preamble{#1}{\def\Gm@paper{#1}\@nameuse{Gm@#1}{paper}}}%
%    \end{macrocode}
%    \end{macro}
%    此处定义了各种纸张尺寸。
%    \begin{macrocode}
\@namedef{Gm@a0paper}#1{\Gm@setsize{#1}(841,1189){mm}}% ISO A0
\@namedef{Gm@a1paper}#1{\Gm@setsize{#1}(594,841){mm}}% ISO A1
\@namedef{Gm@a2paper}#1{\Gm@setsize{#1}(420,594){mm}}% ISO A2
\@namedef{Gm@a3paper}#1{\Gm@setsize{#1}(297,420){mm}}% ISO A3
\@namedef{Gm@a4paper}#1{\Gm@setsize{#1}(210,297){mm}}% ISO A4
\@namedef{Gm@a5paper}#1{\Gm@setsize{#1}(148,210){mm}}% ISO A5
\@namedef{Gm@a6paper}#1{\Gm@setsize{#1}(105,148){mm}}% ISO A6
\@namedef{Gm@b0paper}#1{\Gm@setsize{#1}(1000,1414){mm}}% ISO B0
\@namedef{Gm@b1paper}#1{\Gm@setsize{#1}(707,1000){mm}}% ISO B1
\@namedef{Gm@b2paper}#1{\Gm@setsize{#1}(500,707){mm}}% ISO B2
\@namedef{Gm@b3paper}#1{\Gm@setsize{#1}(353,500){mm}}% ISO B3
\@namedef{Gm@b4paper}#1{\Gm@setsize{#1}(250,353){mm}}% ISO B4
\@namedef{Gm@b5paper}#1{\Gm@setsize{#1}(176,250){mm}}% ISO B5
\@namedef{Gm@b6paper}#1{\Gm@setsize{#1}(125,176){mm}}% ISO B6
\@namedef{Gm@c0paper}#1{\Gm@setsize{#1}(917,1297){mm}}% ISO C0
\@namedef{Gm@c1paper}#1{\Gm@setsize{#1}(648,917){mm}}% ISO C1
\@namedef{Gm@c2paper}#1{\Gm@setsize{#1}(458,648){mm}}% ISO C2
\@namedef{Gm@c3paper}#1{\Gm@setsize{#1}(324,458){mm}}% ISO C3
\@namedef{Gm@c4paper}#1{\Gm@setsize{#1}(229,324){mm}}% ISO C4
\@namedef{Gm@c5paper}#1{\Gm@setsize{#1}(162,229){mm}}% ISO C5
\@namedef{Gm@c6paper}#1{\Gm@setsize{#1}(114,162){mm}}% ISO C6
\@namedef{Gm@b0j}#1{\Gm@setsize{#1}(1030,1456){mm}}% JIS B0
\@namedef{Gm@b1j}#1{\Gm@setsize{#1}(728,1030){mm}}% JIS B1
\@namedef{Gm@b2j}#1{\Gm@setsize{#1}(515,728){mm}}% JIS B2
\@namedef{Gm@b3j}#1{\Gm@setsize{#1}(364,515){mm}}% JIS B3
\@namedef{Gm@b4j}#1{\Gm@setsize{#1}(257,364){mm}}% JIS B4
\@namedef{Gm@b5j}#1{\Gm@setsize{#1}(182,257){mm}}% JIS B5
\@namedef{Gm@b6j}#1{\Gm@setsize{#1}(128,182){mm}}% JIS B6
\@namedef{Gm@ansiapaper}#1{\Gm@setsize{#1}(8.5,11){in}}%
\@namedef{Gm@ansibpaper}#1{\Gm@setsize{#1}(11,17){in}}%
\@namedef{Gm@ansicpaper}#1{\Gm@setsize{#1}(17,22){in}}%
\@namedef{Gm@ansidpaper}#1{\Gm@setsize{#1}(22,34){in}}%
\@namedef{Gm@ansiepaper}#1{\Gm@setsize{#1}(34,44){in}}%
\@namedef{Gm@letterpaper}#1{\Gm@setsize{#1}(8.5,11){in}}%
\@namedef{Gm@legalpaper}#1{\Gm@setsize{#1}(8.5,14){in}}%
\@namedef{Gm@executivepaper}#1{\Gm@setsize{#1}(7.25,10.5){in}}%
\@namedef{Gm@screen}#1{\Gm@setsize{#1}(225,180){mm}}%
%    \end{macrocode}
%
%  \begin{key}{Gm}{paper}
%    |paper| 以纸张名称(paper name)作为其值。
%    \begin{macrocode}
\define@key{Gm}{paper}{\setkeys{Gm}{#1}}%
\let\KV@Gm@papername\KV@Gm@paper
%    \end{macrocode}
%  \end{key}
%  \begin{key}{Gm}{a[0-6]paper}
%  \begin{key}{Gm}{b[0-6]paper}
%  \begin{key}{Gm}{b[0-6]j}
%  \begin{key}{Gm}{ansi[a-e]paper}
%  \begin{key}{Gm}{letterpaper}
%  \begin{key}{Gm}{legalpaper}
%  \begin{key}{Gm}{executivepaper}
%  \begin{key}{Gm}{screen}
%    以下是可用的纸张名称(paper name)。
%    \begin{macrocode}
\define@key{Gm}{a0paper}[true]{\Gm@setpaper@ifpre{a0paper}}%
\define@key{Gm}{a1paper}[true]{\Gm@setpaper@ifpre{a1paper}}%
\define@key{Gm}{a2paper}[true]{\Gm@setpaper@ifpre{a2paper}}%
\define@key{Gm}{a3paper}[true]{\Gm@setpaper@ifpre{a3paper}}%
\define@key{Gm}{a4paper}[true]{\Gm@setpaper@ifpre{a4paper}}%
\define@key{Gm}{a5paper}[true]{\Gm@setpaper@ifpre{a5paper}}%
\define@key{Gm}{a6paper}[true]{\Gm@setpaper@ifpre{a6paper}}%
\define@key{Gm}{b0paper}[true]{\Gm@setpaper@ifpre{b0paper}}%
\define@key{Gm}{b1paper}[true]{\Gm@setpaper@ifpre{b1paper}}%
\define@key{Gm}{b2paper}[true]{\Gm@setpaper@ifpre{b2paper}}%
\define@key{Gm}{b3paper}[true]{\Gm@setpaper@ifpre{b3paper}}%
\define@key{Gm}{b4paper}[true]{\Gm@setpaper@ifpre{b4paper}}%
\define@key{Gm}{b5paper}[true]{\Gm@setpaper@ifpre{b5paper}}%
\define@key{Gm}{b6paper}[true]{\Gm@setpaper@ifpre{b6paper}}%
\define@key{Gm}{c0paper}[true]{\Gm@setpaper@ifpre{c0paper}}%
\define@key{Gm}{c1paper}[true]{\Gm@setpaper@ifpre{c1paper}}%
\define@key{Gm}{c2paper}[true]{\Gm@setpaper@ifpre{c2paper}}%
\define@key{Gm}{c3paper}[true]{\Gm@setpaper@ifpre{c3paper}}%
\define@key{Gm}{c4paper}[true]{\Gm@setpaper@ifpre{c4paper}}%
\define@key{Gm}{c5paper}[true]{\Gm@setpaper@ifpre{c5paper}}%
\define@key{Gm}{c6paper}[true]{\Gm@setpaper@ifpre{c6paper}}%
\define@key{Gm}{b0j}[true]{\Gm@setpaper@ifpre{b0j}}%
\define@key{Gm}{b1j}[true]{\Gm@setpaper@ifpre{b1j}}%
\define@key{Gm}{b2j}[true]{\Gm@setpaper@ifpre{b2j}}%
\define@key{Gm}{b3j}[true]{\Gm@setpaper@ifpre{b3j}}%
\define@key{Gm}{b4j}[true]{\Gm@setpaper@ifpre{b4j}}%
\define@key{Gm}{b5j}[true]{\Gm@setpaper@ifpre{b5j}}%
\define@key{Gm}{b6j}[true]{\Gm@setpaper@ifpre{b6j}}%
\define@key{Gm}{ansiapaper}[true]{\Gm@setpaper@ifpre{ansiapaper}}%安西亚纸
\define@key{Gm}{ansibpaper}[true]{\Gm@setpaper@ifpre{ansibpaper}}%
\define@key{Gm}{ansicpaper}[true]{\Gm@setpaper@ifpre{ansicpaper}}%
\define@key{Gm}{ansidpaper}[true]{\Gm@setpaper@ifpre{ansidpaper}}%
\define@key{Gm}{ansiepaper}[true]{\Gm@setpaper@ifpre{ansiepaper}}%
\define@key{Gm}{letterpaper}[true]{\Gm@setpaper@ifpre{letterpaper}}%信纸
\define@key{Gm}{legalpaper}[true]{\Gm@setpaper@ifpre{legalpaper}}%
\define@key{Gm}{executivepaper}[true]{\Gm@setpaper@ifpre{executivepaper}}%
\define@key{Gm}{screen}[true]{\Gm@setpaper@ifpre{screen}}%
%    \end{macrocode}
%  \end{key}\end{key}\end{key}\end{key}\end{key}
%  \end{key}\end{key}\end{key}
%  \begin{key}{Gm}{paperwidth}
%  \begin{key}{Gm}{paperheight}
%  \begin{key}{Gm}{papersize}
%    也可以直接指定纸张尺寸。
%    \begin{macrocode}
\define@key{Gm}{paperwidth}{\ifGm@preamble{paperwidth}{%
  \def\Gm@paper{custom}\Gm@setlength\paperwidth{#1}}}%
\define@key{Gm}{paperheight}{\ifGm@preamble{paperheight}{%
  \def\Gm@paper{custom}\Gm@setlength\paperheight{#1}}}%
\define@key{Gm}{papersize}{\ifGm@preamble{papersize}{%
  \def\Gm@paper{custom}\Gm@branch{#1}{paperwidth}{paperheight}}}%
%    \end{macrocode}
%  \end{key}\end{key}\end{key}
%  \begin{key}{Gm}{layout}
%  \begin{key}{Gm}{layoutwidth}
%  \begin{key}{Gm}{layoutheight}
%  \begin{key}{Gm}{layoutsize}
%    也可以直接指定布局尺寸(layout size)。
%    \begin{macrocode}
\define@key{Gm}{layout}{\Gm@layouttrue\@nameuse{Gm@#1}{Gm@layout}}%
\let\KV@Gm@layoutname\KV@Gm@layout
\define@key{Gm}{layoutwidth}{\Gm@layouttrue\Gm@setlength\Gm@layoutwidth{#1}}%
\define@key{Gm}{layoutheight}{\Gm@layouttrue\Gm@setlength\Gm@layoutheight{#1}}%
\define@key{Gm}{layoutsize}{\Gm@branch{#1}{layoutwidth}{layoutheight}}%
%    \end{macrocode}
%  \end{key}\end{key}\end{key}\end{key}
%  \begin{key}{Gm}{landscape}
%  \begin{key}{Gm}{portrait}
%    设置纸张方向(orientation)。
%    \begin{macrocode}
\define@key{Gm}{landscape}[true]{\ifGm@preamble{landscape}{%
  \Gm@doifelse{landscape}{#1}%
  {\ifGm@landscape\else\Gm@landscapetrue\Gm@reverse{swap@papersize}\fi}%
  {\ifGm@landscape\Gm@landscapefalse\Gm@reverse{swap@papersize}\fi}}}%
\define@key{Gm}{portrait}[true]{\ifGm@preamble{portrait}{%
  \Gm@doifelse{portrait}{#1}%
  {\ifGm@landscape\Gm@landscapefalse\Gm@reverse{swap@papersize}\fi}%
  {\ifGm@landscape\else\Gm@landscapetrue\Gm@reverse{swap@papersize}\fi}}}%
%    \end{macrocode}
%  \end{key}\end{key}
%  \begin{key}{Gm}{hscale}
%  \begin{key}{Gm}{vscale}
%  \begin{key}{Gm}{scale}
%    这些选项可以确定正文(\gpart{total body})的长度,给出与纸张尺寸(paper size)的比例(\textit{scale(s)})。
%    \begin{macrocode}
\define@key{Gm}{hscale}{\Gm@hbodytrue\edef\Gm@hscale{#1}}%
\define@key{Gm}{vscale}{\Gm@vbodytrue\edef\Gm@vscale{#1}}%
\define@key{Gm}{scale}{\Gm@branch{#1}{hscale}{vscale}}%
%    \end{macrocode}
%  \clearpage
%  \end{key}\end{key}\end{key}
%  \begin{key}{Gm}{width}
%  \begin{key}{Gm}{height}
%  \begin{key}{Gm}{total}
%  \begin{key}{Gm}{totalwidth}
%  \begin{key}{Gm}{totalheight}
%    这些选项给出了正文(\gpart{total body})的具体尺寸。
%    |totalwidth| 和 |totalheight| 分别是 |width| 和 |height|的别名。
%    \begin{macrocode}
\define@key{Gm}{width}{\Gm@hbodytrue\Gm@defbylen{width}{#1}}%
\define@key{Gm}{height}{\Gm@vbodytrue\Gm@defbylen{height}{#1}}%
\define@key{Gm}{total}{\Gm@branch{#1}{width}{height}}%
\let\KV@Gm@totalwidth\KV@Gm@width
\let\KV@Gm@totalheight\KV@Gm@height
%    \end{macrocode}
%  \end{key}\end{key}\end{key}\end{key}\end{key}
%  \begin{key}{Gm}{textwidth}
%  \begin{key}{Gm}{textheight}
%  \begin{key}{Gm}{text}
%  \begin{key}{Gm}{body}
%    这些选项直接设置尺寸 \cs{textwidth}\ 和 \cs{textheight}。|body| 是 |text| 的别名。
%    \begin{macrocode}
\define@key{Gm}{textwidth}{\Gm@hbodytrue\Gm@defbylen{textwidth}{#1}}%
\define@key{Gm}{textheight}{\Gm@vbodytrue\Gm@defbylen{textheight}{#1}}%
\define@key{Gm}{text}{\Gm@branch{#1}{textwidth}{textheight}}%
\let\KV@Gm@body\KV@Gm@text
%    \end{macrocode}
%  \end{key}\end{key}\end{key}\end{key}
%  \begin{key}{Gm}{lines}
%    该选项用行数(number of lines)设置 \cs{textheight}。
%    \begin{macrocode}
\define@key{Gm}{lines}{\Gm@vbodytrue\Gm@defbycnt{lines}{#1}}%
%    \end{macrocode}
%  \end{key}
%  \begin{key}{Gm}{includehead}
%  \begin{key}{Gm}{includefoot}
%  \begin{key}{Gm}{includeheadfoot}
%  \begin{key}{Gm}{includemp}
%  \begin{key}{Gm}{includeall}
%    这些选项将相应的尺寸(corresponding dimensions)作为正文(\gpart{body})的一部分。
%    \begin{macrocode}
\define@key{Gm}{includehead}[true]{\Gm@setbool{includehead}{#1}}%
\define@key{Gm}{includefoot}[true]{\Gm@setbool{includefoot}{#1}}%
\define@key{Gm}{includeheadfoot}[true]{\Gm@doifelse{includeheadfoot}{#1}%
  {\Gm@includeheadtrue\Gm@includefoottrue}%
  {\Gm@includeheadfalse\Gm@includefootfalse}}%
\define@key{Gm}{includemp}[true]{\Gm@setbool{includemp}{#1}}%
\define@key{Gm}{includeall}[true]{\Gm@doifelse{includeall}{#1}%
  {\Gm@includeheadtrue\Gm@includefoottrue\Gm@includemptrue}%
  {\Gm@includeheadfalse\Gm@includefootfalse\Gm@includempfalse}}%
%    \end{macrocode}
%  \end{key}\end{key}\end{key}\end{key}\end{key}
%  \begin{key}{Gm}{ignorehead}
%  \begin{key}{Gm}{ignorefoot}
%  \begin{key}{Gm}{ignoreheadfoot}
%  \begin{key}{Gm}{ignoremp}
%  \begin{key}{Gm}{ignoreall}
%  这些选项在确定 \gpart{body}\ 时不包含 \gpart{head}、\gpart{foot}\ 和 \gpart{marginpars}。
%    \begin{macrocode}
\define@key{Gm}{ignorehead}[true]{%
  \Gm@setboolrev[ignorehead]{includehead}{#1}}%
\define@key{Gm}{ignorefoot}[true]{%
  \Gm@setboolrev[ignorefoot]{includefoot}{#1}}%
\define@key{Gm}{ignoreheadfoot}[true]{\Gm@doifelse{ignoreheadfoot}{#1}%
  {\Gm@includeheadfalse\Gm@includefootfalse}%
  {\Gm@includeheadtrue\Gm@includefoottrue}}%
\define@key{Gm}{ignoremp}[true]{%
  \Gm@setboolrev[ignoremp]{includemp}{#1}}%
\define@key{Gm}{ignoreall}[true]{\Gm@doifelse{ignoreall}{#1}%
  {\Gm@includeheadfalse\Gm@includefootfalse\Gm@includempfalse}%
  {\Gm@includeheadtrue\Gm@includefoottrue\Gm@includemptrue}}%
%    \end{macrocode}
%  \end{key}\end{key}\end{key}\end{key}\end{key}
%  \begin{key}{Gm}{heightrounded}
%    该选项将 \cs{textheight}\ 四舍五入为 \cs{baselineskip}\ 的n倍加上 \cs{topskip}。
%    \begin{macrocode}
\define@key{Gm}{heightrounded}[true]{\Gm@setbool{heightrounded}{#1}}%
%    \end{macrocode}
%  \end{key}
%  \begin{key}{Gm}{hdivide}
%  \begin{key}{Gm}{vdivide}
%  \begin{key}{Gm}{divide}
%    这些选项对于指定纸张每个方向的分区(partitioning)非常有用。
%    \begin{macrocode}
\define@key{Gm}{hdivide}{\Gm@parse@divide{#1}{lmargin}{width}{rmargin}}%
\define@key{Gm}{vdivide}{\Gm@parse@divide{#1}{tmargin}{height}{bmargin}}%
\define@key{Gm}{divide}{\Gm@parse@divide{#1}{lmargin}{width}{rmargin}%
  \Gm@parse@divide{#1}{tmargin}{height}{bmargin}}%
%    \end{macrocode}
%  \end{key}\end{key}\end{key}
%
%  \begin{key}{Gm}{lmargin}
%  \begin{key}{Gm}{rmargin}
%  \begin{key}{Gm}{tmargin}
%  \begin{key}{Gm}{bmargin}
%  \begin{key}{Gm}{left}
%  \begin{key}{Gm}{inner}
%  \begin{key}{Gm}{innermargin}
%  \begin{key}{Gm}{right}
%  \begin{key}{Gm}{outer}
%  \begin{key}{Gm}{outermargin}
%  \begin{key}{Gm}{top}
%  \begin{key}{Gm}{bottom}
%    这些选项设置版口(\gpart{margins})。|left|、|inner|、|innermargin| 是 |lmargin| 的别名。
%    |right|、|outer|、|outermargin| 是 |rmargin| 的别名。
%    |top| 和 |bottom| 分别是 |tmargin| 和 |bmargin| 的别名。
%    \begin{macrocode}
\define@key{Gm}{lmargin}{\Gm@defbylen{lmargin}{#1}}%
\define@key{Gm}{rmargin}{\Gm@defbylen{rmargin}{#1}}%
\let\KV@Gm@left\KV@Gm@lmargin
\let\KV@Gm@inner\KV@Gm@lmargin
\let\KV@Gm@innermargin\KV@Gm@lmargin
\let\KV@Gm@right\KV@Gm@rmargin
\let\KV@Gm@outer\KV@Gm@rmargin
\let\KV@Gm@outermargin\KV@Gm@rmargin
\define@key{Gm}{tmargin}{\Gm@defbylen{tmargin}{#1}}%
\define@key{Gm}{bmargin}{\Gm@defbylen{bmargin}{#1}}%
\let\KV@Gm@top\KV@Gm@tmargin
\let\KV@Gm@bottom\KV@Gm@bmargin
%    \end{macrocode}
%  \end{key}\end{key}\end{key}\end{key}\end{key}
%  \end{key}\end{key}\end{key}\end{key}\end{key}
%  \end{key}\end{key}
%  \begin{key}{Gm}{hmargin}
%  \begin{key}{Gm}{vmargin}
%  \begin{key}{Gm}{margin}
%  这些选项是设置版口(\gpart{margins})的简写(shorthands)。
%    \begin{macrocode}
\define@key{Gm}{hmargin}{\Gm@branch{#1}{lmargin}{rmargin}}%
\define@key{Gm}{vmargin}{\Gm@branch{#1}{tmargin}{bmargin}}%
\define@key{Gm}{margin}{\Gm@branch{#1}{lmargin}{tmargin}%
  \Gm@branch{#1}{rmargin}{bmargin}}%
%    \end{macrocode}
%  \end{key}\end{key}\end{key}
%  \begin{key}{Gm}{hmarginratio}
%  \begin{key}{Gm}{vmarginratio}
%  \begin{key}{Gm}{marginratio}
%  \begin{key}{Gm}{hratio}
%  \begin{key}{Gm}{vratio}
%  \begin{key}{Gm}{ratio}
%  这些选项用于指定版口比例(margin ratios)。
%    \begin{macrocode}
\define@key{Gm}{hmarginratio}{\edef\Gm@hmarginratio{#1}}%
\define@key{Gm}{vmarginratio}{\edef\Gm@vmarginratio{#1}}%
\define@key{Gm}{marginratio}{\Gm@branch{#1}{hmarginratio}{vmarginratio}}%
\let\KV@Gm@hratio\KV@Gm@hmarginratio
\let\KV@Gm@vratio\KV@Gm@vmarginratio
\let\KV@Gm@ratio\KV@Gm@marginratio
%    \end{macrocode}
%  \end{key}\end{key}\end{key}
%  \end{key}\end{key}\end{key}
%  \begin{key}{Gm}{hcentering}
%  \begin{key}{Gm}{vcentering}
%  \begin{key}{Gm}{centering}
%    居中放置正文(\gpart{body})的有用简写(shorthands)。
%    \begin{macrocode}
\define@key{Gm}{hcentering}[true]{\Gm@doifelse{hcentering}{#1}%
  {\def\Gm@hmarginratio{1:1}}{}}%
\define@key{Gm}{vcentering}[true]{\Gm@doifelse{vcentering}{#1}%
  {\def\Gm@vmarginratio{1:1}}{}}%
\define@key{Gm}{centering}[true]{\Gm@doifelse{centering}{#1}%
  {\def\Gm@hmarginratio{1:1}\def\Gm@vmarginratio{1:1}}{}}%
%    \end{macrocode}
%  \end{key}\end{key}\end{key}
%  \begin{key}{Gm}{twoside}
%    如果 |twoside=true|,则 \cs{@twoside}\ 和 \cs{@mparswitch}\ 设置为 |true|。
%    \begin{macrocode}
\define@key{Gm}{twoside}[true]{\Gm@doifelse{twoside}{#1}%
  {\@twosidetrue\@mparswitchtrue}{\@twosidefalse\@mparswitchfalse}}%
%    \end{macrocode}
%  \end{key}
%  \begin{key}{Gm}{asymmetric}
%    |asymmetric| 设置 \cs{@mparswitchfalse}\ 和 \cs{@twosidetrue}。|asymmetric=false| 无效。
%    \begin{macrocode}
\define@key{Gm}{asymmetric}[true]{\Gm@doifelse{asymmetric}{#1}%
  {\@twosidetrue\@mparswitchfalse}{}}%
%    \end{macrocode}
%  \end{key}
%  \begin{key}{Gm}{bindingoffset}
%    该宏将指定的空间(specified space)添加到里口(inner margin,订口)。
%    \begin{macrocode}
\define@key{Gm}{bindingoffset}{\Gm@setlength\Gm@bindingoffset{#1}}%
%    \end{macrocode}
%  \end{key}
%  \begin{key}{Gm}{headheight}
%  \begin{key}{Gm}{headsep}
%  \begin{key}{Gm}{footskip}
%  \begin{key}{Gm}{head}
%  \begin{key}{Gm}{foot}
%    直接设置 \gpart{head}\ 和/或 \gpart{foot}\ 的尺寸。
%    \begin{macrocode}
\define@key{Gm}{headheight}{\Gm@setlength\headheight{#1}}%
\define@key{Gm}{headsep}{\Gm@setlength\headsep{#1}}%
\define@key{Gm}{footskip}{\Gm@setlength\footskip{#1}}%
\let\KV@Gm@head\KV@Gm@headheight
\let\KV@Gm@foot\KV@Gm@footskip
%    \end{macrocode}
%  \end{key}\end{key}\end{key}\end{key}\end{key}
%  \begin{key}{Gm}{nohead}
%  \begin{key}{Gm}{nofoot}
%  \begin{key}{Gm}{noheadfoot}
%    它们是将 \gpart{head}\ 和/或 \gpart{foot}\ 设为 |0pt| 的缩写。
%    \begin{macrocode}
\define@key{Gm}{nohead}[true]{\Gm@doifelse{nohead}{#1}%
  {\Gm@setlength\headheight\z@\Gm@setlength\headsep\z@}{}}%
\define@key{Gm}{nofoot}[true]{\Gm@doifelse{nofoot}{#1}%
  {\Gm@setlength\footskip\z@}{}}%
\define@key{Gm}{noheadfoot}[true]{\Gm@doifelse{noheadfoot}{#1}%
  {\Gm@setlength\headheight\z@\Gm@setlength\headsep
  \z@\Gm@setlength\footskip\z@}{}}%
%    \end{macrocode}
%  \end{key}\end{key}\end{key}
%  \begin{key}{Gm}{footnotesep}
%    该选项直接设置原始尺寸(native dimension)\ \ \cs{footnotesep}。
%    \begin{macrocode}
\define@key{Gm}{footnotesep}{\Gm@setlength{\skip\footins}{#1}}%
%    \end{macrocode}
%  \end{key}
%  \begin{key}{Gm}{marginparwidth}
%  \begin{key}{Gm}{marginpar}
%  \begin{key}{Gm}{marginparsep}
%   它们直接设置原始尺寸(native dimension)\ \ \cs{marginparwidth}\ 和 \cs{marginparsep}。
%    \begin{macrocode}
\define@key{Gm}{marginparwidth}{\Gm@setlength\marginparwidth{#1}}%
\let\KV@Gm@marginpar\KV@Gm@marginparwidth
\define@key{Gm}{marginparsep}{\Gm@setlength\marginparsep{#1}}%
%    \end{macrocode}
%  \end{key}\end{key}\end{key}
%  \begin{key}{Gm}{nomarginpar}
%   该宏是 \cs{marginparwidth}|=0pt| 和 \cs{marginparsep}|=0pt| 的缩写。
%    \begin{macrocode}
\define@key{Gm}{nomarginpar}[true]{\Gm@doifelse{nomarginpar}{#1}%
  {\Gm@setlength\marginparwidth\z@\Gm@setlength\marginparsep\z@}{}}%
%    \end{macrocode}
%  \end{key}
%  \begin{key}{Gm}{columnsep}
%    该选项设置原始尺寸(native dimension)\ \ \cs{columnsep}。
%    \begin{macrocode}
\define@key{Gm}{columnsep}{\Gm@setlength\columnsep{#1}}%
%    \end{macrocode}
%  \end{key}
%  \begin{key}{Gm}{hoffset}
%  \begin{key}{Gm}{voffset}
%  \begin{key}{Gm}{offset}
%    前两个选项设置原始尺寸(native dimension)\ \ \cs{hoffset}\ 和 \cs{voffset}。
%    |offset| 可以将两者设置为相同的值。
%    \begin{macrocode}
\define@key{Gm}{hoffset}{\Gm@setlength\hoffset{#1}}%
\define@key{Gm}{voffset}{\Gm@setlength\voffset{#1}}%
\define@key{Gm}{offset}{\Gm@branch{#1}{hoffset}{voffset}}%
%    \end{macrocode}
%  \end{key}\end{key}\end{key}
%  \begin{key}{Gm}{layouthoffset}
%  \begin{key}{Gm}{layoutvoffset}
%  \begin{key}{Gm}{layoutoffset}
%    \begin{macrocode}
\define@key{Gm}{layouthoffset}{\Gm@setlength\Gm@layouthoffset{#1}}%
\define@key{Gm}{layoutvoffset}{\Gm@setlength\Gm@layoutvoffset{#1}}%
\define@key{Gm}{layoutoffset}{\Gm@branch{#1}{layouthoffset}{layoutvoffset}}%
%    \end{macrocode}
%  \end{key}\end{key}\end{key}
%  \begin{key}{Gm}{twocolumn}
%    该选项设置 \cs{twocolumn}\ 开关(switch)。
%    \begin{macrocode}
\define@key{Gm}{twocolumn}[true]{%
  \Gm@doif{twocolumn}{#1}{\csname @twocolumn\Gm@bool\endcsname}}%
%    \end{macrocode}
%  \end{key}
%  \begin{key}{Gm}{onecolumn}
%    此选项与 |twocolumn| 选项的效果相反。
%    \begin{macrocode}
\define@key{Gm}{onecolumn}[true]{%
  \Gm@doifelse{onecolumn}{#1}{\@twocolumnfalse}{\@twocolumntrue}}%
%    \end{macrocode}
%  \end{key}
%  \begin{key}{Gm}{reversemp}
%  \begin{key}{Gm}{reversemarginpar}
%    这两个选项都设置了 \cs{reversemargin}\ (反向版口)。
%    \begin{macrocode}
\define@key{Gm}{reversemp}[true]{%
  \Gm@doif{reversemp}{#1}{\csname @reversemargin\Gm@bool\endcsname}}%
\define@key{Gm}{reversemarginpar}[true]{%
  \Gm@doif{reversemarginpar}{#1}{\csname @reversemargin\Gm@bool\endcsname}}%
%    \end{macrocode}
%  \end{key}\end{key}
%  \begin{key}{Gm}{dviver}
%    \begin{macrocode}
\define@key{Gm}{driver}{\ifGm@preamble{driver}{%
  \edef\@@tempa{#1}\edef\@@auto{auto}\edef\@@none{none}%
  \ifx\@@tempa\@empty\let\Gm@driver\relax\else
  \ifx\@@tempa\@@none\let\Gm@driver\relax\else
  \ifx\@@tempa\@@auto\let\Gm@driver\@empty\else
  \setkeys{Gm}{#1}\fi\fi\fi\let\@@auto\relax\let\@@none\relax}}%
%    \end{macrocode}
%  \end{key}
%  \begin{key}{Gm}{dvips}
%  \begin{key}{Gm}{dvipdfm}\begin{key}{Gm}{dvipdfmx}\begin{key}{Gm}{xdvipdfmx}
%  \begin{key}{Gm}{pdftex}
%  \begin{key}{Gm}{luatex}
%  \begin{key}{Gm}{xetex}
%  \begin{key}{Gm}{vtex}
%   \Gm\ 宏包支持 |dvips|、|dvipdfm|、|pdflatex|、|luatex| 和 |vtex|。|dvipdfm| 工
%   作方式和 |dvips| 相同。
%    \begin{macrocode}
\define@key{Gm}{dvips}[true]{\ifGm@preamble{dvips}{%
  \Gm@doifelse{dvips}{#1}{\Gm@setdriver{dvips}}{\Gm@unsetdriver{dvips}}}}%
\define@key{Gm}{dvipdfm}[true]{\ifGm@preamble{dvipdfm}{%
  \Gm@doifelse{dvipdfm}{#1}{\Gm@setdriver{dvipdfm}}{\Gm@unsetdriver{dvipdfm}}}}%
\define@key{Gm}{dvipdfmx}[true]{\ifGm@preamble{dvipdfm}{%
  \Gm@doifelse{dvipdfm}{#1}{\Gm@setdriver{dvipdfm}}{\Gm@unsetdriver{dvipdfm}}}}%
\define@key{Gm}{xdvipdfmx}[true]{\ifGm@preamble{dvipdfm}{%
  \Gm@doifelse{dvipdfm}{#1}{\Gm@setdriver{dvipdfm}}{\Gm@unsetdriver{dvipdfm}}}}%
\define@key{Gm}{pdftex}[true]{\ifGm@preamble{pdftex}{%
  \Gm@doifelse{pdftex}{#1}{\Gm@setdriver{pdftex}}{\Gm@unsetdriver{pdftex}}}}%
\define@key{Gm}{luatex}[true]{\ifGm@preamble{luatex}{%
  \Gm@doifelse{luatex}{#1}{\Gm@setdriver{luatex}}{\Gm@unsetdriver{luatex}}}}%
\define@key{Gm}{xetex}[true]{\ifGm@preamble{xetex}{%
  \Gm@doifelse{xetex}{#1}{\Gm@setdriver{xetex}}{\Gm@unsetdriver{xetex}}}}%
\define@key{Gm}{vtex}[true]{\ifGm@preamble{vtex}{%
  \Gm@doifelse{vtex}{#1}{\Gm@setdriver{vtex}}{\Gm@unsetdriver{vtex}}}}%
%    \end{macrocode}
%  \end{key}\end{key}\end{key}\end{key}\end{key}\end{key}\end{key}\end{key}
%  \begin{key}{Gm}{verbose}
%    冗余模式(verbose mode)。
%    \begin{macrocode}
\define@key{Gm}{verbose}[true]{\ifGm@preamble{verbose}{\Gm@setbool{verbose}{#1}}}%
%    \end{macrocode}
%  \end{key}
%  \begin{key}{Gm}{reset}
%    该选项取消了 |reset| 之前指定的所有选项(除外 |pass|)。除了无法重置带有 |truedimen| 的 |mag| ($\neq1000$)。
%    \begin{macrocode}
\define@key{Gm}{reset}[true]{\ifGm@preamble{reset}{%
  \Gm@doifelse{reset}{#1}{\Gm@restore@org\Gm@initall
  \ProcessOptionsKV[c]{Gm}\Gm@setdefaultpaper}{}}}%
%    \end{macrocode}
%  \end{key}
%  \begin{key}{Gm}{resetpaper}
%    如果 |resetpaper| 设置为 |true|,则丢弃宏包装中重新定义的纸张尺寸,并恢复原始纸张尺寸。
%    此选项对于使用普通(normal)打印机(printers)和纸张(papers)打印非标准尺寸
%    的(nonstandard sized)文件可能很有用。
%    \begin{macrocode}
\define@key{Gm}{resetpaper}[true]{\ifGm@preamble{resetpaper}{%
  \Gm@setbool{resetpaper}{#1}}}%
%    \end{macrocode}
%  \end{key}
%  \begin{key}{Gm}{mag}
%    |mag| 在指定时立即展开,所以当 |reset| 被设置为 |truedimen| 时 |reset| 不能重置 |mag|。
%    \begin{macrocode}
\define@key{Gm}{mag}{\ifGm@preamble{mag}{\mag=#1}}%
%    \end{macrocode}
%  \end{key}
%  \begin{key}{Gm}{truedimen}
%    如果 |truedimen| 设置为 |true|,则所有内部显式尺寸(internal explicit dimensions)
%    都将更改为 \textit{true}\ 尺寸,例如,|1in| 更改为 |1truein|。
%    \begin{macrocode}
\define@key{Gm}{truedimen}[true]{\ifGm@preamble{truedimen}{%
  \Gm@doifelse{truedimen}{#1}{\let\Gm@truedimen\Gm@true}%
  {\let\Gm@truedimen\@empty}}}%
%    \end{macrocode}
%  \end{key}
%  \begin{key}{Gm}{pass}
%    该选项使指定的所有选项无效,但冗余开关(verbose switch)除外。
%    \begin{macrocode}
\define@key{Gm}{pass}[true]{\ifGm@preamble{pass}{\Gm@setbool{pass}{#1}}}%
%    \end{macrocode}
%  \end{key}
%  \begin{key}{Gm}{showframe}
%    |Showframe| 选项打印页面框架(page frames),以帮助您了解最终的布局(resulting layout)是什么样的。
%    \begin{macrocode}
\define@key{Gm}{showframe}[true]{\Gm@setbool{showframe}{#1}}%
%    \end{macrocode}
%  \end{key}
%  \begin{key}{Gm}{showcrop}
%    |showcrop| 选项在布局区域(layout area)的每个角落打印裁剪标记(crop marks)。
%    \begin{macrocode}
\define@key{Gm}{showcrop}[true]{\Gm@setbool{showcrop}{#1}}%
%    \end{macrocode}
%  \end{key}
%    \begin{macro}{\Gm@setdefaultpaper}
%    该宏存储纸张尺寸(paper dimensions)。此宏应在 |\ProcessOptionsKV[c]{Gm}| 之后被调用。
%    如果在 |\documentclass| 中指定了 |landscape| 选项,该类(class)将立即交换纸张尺寸。
%    \begin{macrocode}
\def\Gm@setdefaultpaper{%
  \ifx\Gm@paper\@undefined
    \Gm@setsize{paper}(\strip@pt\paperwidth,\strip@pt\paperheight){pt}%
    \Gm@setsize{Gm@layout}(\strip@pt\paperwidth,\strip@pt\paperheight){pt}%
    \Gm@swap@papersizefalse
  \fi}%
%    \end{macrocode}
%    \end{macro}
%    \begin{macro}{\Gm@adjustpaper}
%    该宏检查纸张宽度/纸张高度(paperwidth/paperwidthheight)是否大于0pt,这在 \cs{Gm@process}\ 中使用。
%    当纸张方向(paper orientation)被 |landscape| 和 |portrait| 选项更改时,可以交换纸张尺寸。
%    \begin{macrocode}
\def\Gm@adjustpaper{%
  \ifdim\paperwidth>\p@\else
    \PackageError{geometry}{%
    \string\paperwidth\space(\the\paperwidth) too short}{%
    Set a paper type (e.g., `a4paper').}%
  \fi
  \ifdim\paperheight>\p@\else
    \PackageError{geometry}{%
    \string\paperheight\space(\the\paperheight) too short}{%
    Set a paper type (e.g., `a4paper').}%
  \fi
  \ifGm@swap@papersize
    \setlength\@tempdima{\paperwidth}%
    \setlength\paperwidth{\paperheight}%
    \setlength\paperheight{\@tempdima}%
  \fi
  \ifGm@layout\else
    \setlength\Gm@layoutwidth{\paperwidth}%
    \setlength\Gm@layoutheight{\paperheight}%
  \fi}%
%    \end{macrocode}
%    \end{macro}
%    \begin{macro}{\Gm@checkmp}
%    宏检查 |marginpars| 是否超出页面。
%    \begin{macrocode}
\def\Gm@checkmp{%
  \ifGm@includemp\else
    \@tempcnta\z@\@tempcntb\@ne
    \if@twocolumn
      \@tempcnta\@ne
    \else
      \if@reversemargin
        \@tempcnta\@ne\@tempcntb\z@
      \fi
    \fi
    \@tempdima\marginparwidth
    \advance\@tempdima\marginparsep
    \ifnum\@tempcnta=\@ne
      \@tempdimc\@tempdima
      \setlength\@tempdimb{\Gm@lmargin}%
      \advance\@tempdimc-\@tempdimb
      \ifdim\@tempdimc>\z@
        \Gm@warning{The marginal notes overrun the paper edge.^^J
        \@spaces Add \the\@tempdimc\space and more to the left margin}%
      \fi
    \fi
    \ifnum\@tempcntb=\@ne
      \@tempdimc\@tempdima
      \setlength\@tempdimb{\Gm@rmargin}%
      \advance\@tempdimc-\@tempdimb
      \ifdim\@tempdimc>\z@
        \Gm@warning{The marginal notes overrun the paper.^^J
        \@spaces Add \the\@tempdimc\space and more to the right margin}%
      \fi
    \fi
  \fi}%
%    \end{macrocode}
%    \end{macro}
%    \begin{macro}{\Gm@adjustmp}
%    当设置 |includemp| 时,该宏设置 marginpar 校正(correction),这在 \cs{Gm@process}\ 中使用。
%    在此处设置变量 \cs{Gm@wd@mp}、\cs{Gm@odd@mp}\ 和 \cs{Gm@even@mp}。
%    请注意 \cs{Gm@even@mp}\ 只能用于双开面布局(twoside layout)。
%    \begin{macrocode}
\def\Gm@adjustmp{%
  \ifGm@includemp
    \@tempdimb\marginparwidth
    \advance\@tempdimb\marginparsep
    \Gm@wd@mp\@tempdimb
    \Gm@odd@mp\z@
    \Gm@even@mp\z@
    \if@twocolumn
      \Gm@wd@mp2\@tempdimb
      \Gm@odd@mp\@tempdimb
      \Gm@even@mp\@tempdimb
    \else
      \if@reversemargin
        \Gm@odd@mp\@tempdimb
        \if@mparswitch\else
          \Gm@even@mp\@tempdimb
        \fi
      \else
        \if@mparswitch
          \Gm@even@mp\@tempdimb
        \fi
      \fi
    \fi
  \fi}%
%    \end{macrocode}
%    \end{macro}
%    \begin{macro}{\Gm@adjustbody}
%    如果用户指定了 \gpart{body}\ 的水平尺寸(horizontal dimension),则在这里正确设置 \cs{Gm@width}。
%    \begin{macrocode}
\def\Gm@adjustbody{
  \ifGm@hbody
    \ifx\Gm@width\@undefined
      \ifx\Gm@hscale\@undefined
        \Gm@defbylen{width}{\Gm@Dhscale\Gm@layoutwidth}%
      \else
        \Gm@defbylen{width}{\Gm@hscale\Gm@layoutwidth}%
      \fi
    \fi
    \ifx\Gm@textwidth\@undefined\else
      \setlength\@tempdima{\Gm@textwidth}%
      \ifGm@includemp
        \advance\@tempdima\Gm@wd@mp
      \fi
      \edef\Gm@width{\the\@tempdima}%
    \fi
  \fi
%    \end{macrocode}
%    如果用户指定了 \gpart{body}\ 的垂直尺寸(vertical dimension),则在这里正确设置 \cs{Gm@height}。
%    \begin{macrocode}
  \ifGm@vbody
    \ifx\Gm@height\@undefined
      \ifx\Gm@vscale\@undefined
        \Gm@defbylen{height}{\Gm@Dvscale\Gm@layoutheight}%
      \else
        \Gm@defbylen{height}{\Gm@vscale\Gm@layoutheight}%
      \fi
    \fi
    \ifx\Gm@lines\@undefined\else
%    \end{macrocode}
%    必须调整 \cs{topskip},以便公式“$\cs{textheight} = (lines - 1) \times \cs{baselineskip} + \cs{topskip}$”
%    即使用户指定了较大的字体尺寸,也是正确的。如果 \cs{topskip}\ 小于 \cs{ht}\cs{strutbox},
%    则将 \cs{topskip}\ 设置为 \cs{ht}\cs{strutbox}。
%    \begin{macrocode}
      \ifdim\topskip<\ht\strutbox
        \setlength\@tempdima{\topskip}%
        \setlength\topskip{\ht\strutbox}%
        \Gm@warning{\noexpand\topskip was changed from \the\@tempdima\space
        to \the\topskip}%
      \fi
      \setlength\@tempdima{\baselineskip}%
      \multiply\@tempdima\Gm@lines
      \addtolength\@tempdima{\topskip}%
      \addtolength\@tempdima{-\baselineskip}%
      \edef\Gm@textheight{\the\@tempdima}%
    \fi
    \ifx\Gm@textheight\@undefined\else
      \setlength\@tempdima{\Gm@textheight}%
      \ifGm@includehead
        \addtolength\@tempdima{\headheight}%
        \addtolength\@tempdima{\headsep}%
      \fi
      \ifGm@includefoot
        \addtolength\@tempdima{\footskip}%
      \fi
      \edef\Gm@height{\the\@tempdima}%
    \fi
  \fi}%
%    \end{macrocode}
%    \end{macro}
%    \begin{macro}{\Gm@process}
%    定义了处理指定尺寸(processing the specified dimensions)的主要宏(main macro)。
%    \begin{macrocode}
\def\Gm@process{%
%    \end{macrocode}
%    如果设置了 |pass|,则恢复原始尺寸(original dimensions)和开关(switches),并在此结束过程(process)。
%    \begin{macrocode}
  \ifGm@pass
    \Gm@restore@org
  \else
    \Gm@@process
  \fi}%
%    \end{macrocode}
%    主处理宏(main processing macro)。
%    \begin{macrocode}
\def\Gm@@process{%
  \Gm@expandlengths
  \Gm@adjustpaper
  \addtolength\Gm@layoutwidth{-\Gm@bindingoffset}%
  \Gm@adjustmp
  \Gm@adjustbody
  \Gm@detall{h}{width}{lmargin}{rmargin}%
  \Gm@detall{v}{height}{tmargin}{bmargin}%
%    \end{macrocode}
%    根据自动完成计算(auto-completion calculation)的结果正确设置实际尺寸(real dimensions)。
%    \begin{macrocode}
  \setlength\textwidth{\Gm@width}%
  \setlength\textheight{\Gm@height}%
  \setlength\topmargin{\Gm@tmargin}%
  \setlength\oddsidemargin{\Gm@lmargin}%
  \addtolength\oddsidemargin{-1\Gm@truedimen in}%
%    \end{macrocode}
%    如果 |includemp| 设置为 |true|,则调整 \cs{textwidth}\ 和 \cs{oddsidemargin}。
%    \begin{macrocode}
  \ifGm@includemp
    \advance\textwidth-\Gm@wd@mp
    \advance\oddsidemargin\Gm@odd@mp
  \fi
%    \end{macrocode}
%    确定 \cs{evensidemargin}。在双开面页面布局(twoside page layout)中,
%    使用外口(right margin)的值 \cs{Gm@rmargin}。如果包含了旁注宽度(marginal note width),
%    则应用 \cs{Gm@even@mp}\ 校正 \cs{evensidemargin}。
%    \begin{macrocode}
  \if@mparswitch
    \setlength\evensidemargin{\Gm@rmargin}%
    \addtolength\evensidemargin{-1\Gm@truedimen in}%
    \ifGm@includemp
      \advance\evensidemargin\Gm@even@mp
    \fi
  \else
    \evensidemargin\oddsidemargin
  \fi
%    \end{macrocode}
%    \cs{oddsidemargin}\ 的装订偏移量校正(bindingoffset correction)。
%    \begin{macrocode}
  \advance\oddsidemargin\Gm@bindingoffset
  \addtolength\topmargin{-1\Gm@truedimen in}%
%    \end{macrocode}
%    如果版心(\gpart{total body},可打印区)包含了页眉,将从 \cs{textheight}\ 中删
%    除 \cs{headheight}\ 和 \cs{headsep},否则从 \cs{topmargin}\ 中删除。
%    \begin{macrocode}
  \ifGm@includehead
    \addtolength\textheight{-\headheight}%
    \addtolength\textheight{-\headsep}%
  \else
    \addtolength\topmargin{-\headheight}%
    \addtolength\topmargin{-\headsep}%
  \fi
%    \end{macrocode}
%    如果版心(\gpart{total body},可打印区)包含了页脚,则会从 \cs{textheight}\ 中删除 \cs{footskip}。
%    \begin{macrocode}
  \ifGm@includefoot
    \addtolength\textheight{-\footskip}%
  \fi
%    \end{macrocode}
%    如果设置了 |heightrounded|,\cs{textheight}\ 将四舍五入。
%    \begin{macrocode}
  \ifGm@heightrounded
    \setlength\@tempdima{\textheight}%
    \addtolength\@tempdima{-\topskip}%
    \@tempcnta\@tempdima
    \@tempcntb\baselineskip
    \divide\@tempcnta\@tempcntb
    \setlength\@tempdimb{\baselineskip}%
    \multiply\@tempdimb\@tempcnta
    \advance\@tempdima-\@tempdimb
    \multiply\@tempdima\tw@
    \ifdim\@tempdima>\baselineskip
      \addtolength\@tempdimb{\baselineskip}%
    \fi
    \addtolength\@tempdimb{\topskip}%
    \textheight\@tempdimb
  \fi
%    \end{macrocode}
%    通过添加 \cs{Gm@bindingoffset}\ 来设回原来的纸张宽度(paper width)。
%    \begin{macrocode}
  \advance\oddsidemargin\Gm@layouthoffset%
  \advance\evensidemargin\Gm@layouthoffset%
  \advance\topmargin\Gm@layoutvoffset%
  \addtolength\Gm@layoutwidth{\Gm@bindingoffset}%
  }% end of \Gm@@process
%    \end{macrocode}
%    \end{macro}
%
%    \begin{macro}{\Gm@detectdriver}
%    该宏检查排版环境(typeset environment),并在必要时更改驱动程序选项(driver option)。
%    为了使引擎检测(engine detection)更加健壮,
%    该宏用 \textsf{ifpdf}、\textsf{ifvtex}\ 和 \textsf{ifxetex}\ 宏包重写。
%    \begin{macrocode}
\def\Gm@detectdriver{%
%    \end{macrocode}
%    如果未明确指定(specified explicitly)驱动程序选项(driver option),
%    则驱动程序自动检测(driver auto-detection)将起作用(works)。
%    \begin{macrocode}
  \ifx\Gm@driver\@empty
    \typeout{*geometry* driver: auto-detecting}%
%    \end{macrocode}
%    \cs{ifpdf}\ 是在“oberdiek”捆绑包(bundle)中的 \textsf{ifpdf}\ 宏包中定义的。
%    \begin{macrocode}
    \ifpdf
      \ifx\pdfextension\@undefined
        \Gm@setdriver{pdftex}%
      \else
        \Gm@setdriver{luatex}%
      \fi
    \else
      \Gm@setdriver{dvips}%
    \fi
%    \end{macrocode}
%    \cs{ifvtex}\ 是在“oberdiek”捆绑包(bundle)中的 \textsf{ifvtex}\ 宏包中定义的。
%    \begin{macrocode}
    \ifvtex
      \Gm@setdriver{vtex}%
    \fi
%    \end{macrocode}
%    \cs{ifxetex}\ 在 Will Robertson (威尔·罗伯逊)编写的 \textsf{ifxetex}\ 宏包中定义的。
%    \begin{macrocode}
    \ifxetex
      \Gm@setdriver{xetex}
    \fi
%    \end{macrocode}
%    当用户设置驱动程序选项(driver option)时,检查其是否有效。
%    \begin{macrocode}
  \else
    \ifx\Gm@driver\Gm@xetex %%
      \ifxetex\else
        \Gm@warning{Wrong driver setting: `xetex'; trying `pdftex' driver}%
        \Gm@setdriver{pdftex}
      \fi
    \fi
    \ifx\Gm@driver\Gm@vtex
      \ifvtex\else
        \Gm@warning{Wrong driver setting: `vtex'; trying `dvips' driver}%
        \Gm@setdriver{dvips}%
      \fi
    \fi
  \fi
  \ifx\Gm@driver\relax
    \typeout{*geometry* detected driver: <none>}%
  \else
    \typeout{*geometry* detected driver: \Gm@driver}%
  \fi}%
%    \end{macrocode}
%    \end{macro}
%    \begin{macro}{\Gm@showparams}
%    如果 |verbose| 为 |true|,则将结果参数(resulted parammeters)和尺寸(dimensions)打
%    印到 STDOUT。将 |\Gm@width| 和 |\Gm@height| 展开以获得实际尺寸(real size)。
%    \begin{macrocode}
\def\Gm@showparams#1{%
  \ifGm@verbose\expandafter\typeout\else\expandafter\wlog\fi
  {\Gm@logcontent{#1}}}%
\def\Gm@showdim#1{* \string#1=\the#1^^J}%
\def\Gm@showbool#1{\@nameuse{ifGm@#1}#1\space\fi}%
%    \end{macrocode}
%    \end{macro}
%    \begin{macro}{\Gm@logcontent}
%    geometry 参数的内容和页面布局的原始尺寸(native dimensions)。
%    \begin{macrocode}
\def\Gm@logcontent#1{%
  *geometry* verbose mode - [ #1 ] result:^^J%
  \ifGm@pass * pass: disregarded the geometry package!^^J%
  \else
  * driver: \if\Gm@driver<none>\else\Gm@driver\fi^^J%
  * paper: \ifx\Gm@paper\@undefined<default>\else\Gm@paper\fi^^J%
  * layout: \ifGm@layout<custom>\else<same size as paper>\fi^^J%
  \ifGm@layout
  * layout(width,height): (\the\Gm@layoutwidth,\the\Gm@layoutheight)^^J%
  \fi
  * layoutoffset:(h,v)=(\the\Gm@layouthoffset,\the\Gm@layoutvoffset)^^J%
  \@ifundefined{Gm@lines}{}{* lines: \Gm@lines^^J}%
  \@ifundefined{Gm@hmarginratio}{}{* hratio: \Gm@hmarginratio^^J}%
  \@ifundefined{Gm@vmarginratio}{}{* vratio: \Gm@vmarginratio^^J}%
  \ifdim\Gm@bindingoffset=\z@\else
  * bindingoffset: \the\Gm@bindingoffset^^J\fi
  * modes: %
   \Gm@showbool{landscape}%
   \Gm@showbool{includehead}%
   \Gm@showbool{includefoot}%
   \Gm@showbool{includemp}%
   \if@twoside twoside\space\fi%
   \if@mparswitch\else\if@twoside asymmetric\space\fi\fi%
   \Gm@showbool{heightrounded}%
   \ifx\Gm@truedimen\@empty\else truedimen\space\fi%
   \Gm@showbool{showframe}%
   \Gm@showbool{showcrop}%
  ^^J%
  * h-part:(L,W,R)=(\Gm@lmargin, \Gm@width, \Gm@rmargin)^^J%
  * v-part:(T,H,B)=(\Gm@tmargin, \Gm@height, \Gm@bmargin)^^J%
  \fi
  \Gm@showdim{\paperwidth}%
  \Gm@showdim{\paperheight}%
  \Gm@showdim{\textwidth}%
  \Gm@showdim{\textheight}%
  \Gm@showdim{\oddsidemargin}%
  \Gm@showdim{\evensidemargin}%
  \Gm@showdim{\topmargin}%
  \Gm@showdim{\headheight}%
  \Gm@showdim{\headsep}%
  \Gm@showdim{\topskip}%
  \Gm@showdim{\footskip}%
  \Gm@showdim{\marginparwidth}%
  \Gm@showdim{\marginparsep}%
  \Gm@showdim{\columnsep}%
  * \string\skip\string\footins=\the\skip\footins^^J%
  \Gm@showdim{\hoffset}%
  \Gm@showdim{\voffset}%
  \Gm@showdim{\mag}%
  * \string\@twocolumn\if@twocolumn true\else false\fi^^J%
  * \string\@twoside\if@twoside true\else false\fi^^J%
  * \string\@mparswitch\if@mparswitch true\else false\fi^^J%
  * \string\@reversemargin\if@reversemargin true\else false\fi^^J%
  * (1in=72.27pt=25.4mm, 1cm=28.453pt)^^J}%
%    \end{macrocode}
%    \end{macro}
%
%    用于页面框架(page frames)和裁剪标记(cropmarks)的宏。
%    \begin{macrocode}
\def\Gm@cropmark(#1,#2,#3,#4){%
  \begin{picture}(0,0)
    \setlength\unitlength{1truemm}%
    \linethickness{0.25pt}%
    \put(#3,0){\line(#1,0){17}}%
    \put(0,#4){\line(0,#2){17}}%
  \end{picture}}%
\providecommand*\vb@xt@{\vbox to}%
\def\Gm@vrule{\vrule width 0.2pt height\textheight depth\z@}%
\def\Gm@hrule{\hrule height 0.2pt depth\z@ width\textwidth}%
\def\Gm@hruled{\hrule height\z@ depth0.2pt width\textwidth}%
\newcommand*{\Gm@vrules@mpi}{%
  \hb@xt@\@tempdima{\llap{\Gm@vrule}\ignorespaces
  \hskip \textwidth\Gm@vrule\hskip \marginparsep
  \llap{\Gm@vrule}\hfil\Gm@vrule}}%
\newcommand*{\Gm@vrules@mpii}{%
  \hb@xt@\@tempdima{\hskip-\marginparwidth\hskip-\marginparsep
  \llap{\Gm@vrule}\ignorespaces
  \hskip \marginparwidth\rlap{\Gm@vrule}\hskip \marginparsep
  \llap{\Gm@vrule}\hskip\textwidth\rlap{\Gm@vrule}\hss}}%
\newcommand*{\Gm@pageframes}{%
  \vb@xt@\z@{%
   \ifGm@showcrop
    \vb@xt@\z@{\vskip-1\Gm@truedimen in\vskip\Gm@layoutvoffset%
     \hb@xt@\z@{\hskip-1\Gm@truedimen in\hskip\Gm@layouthoffset%
      \vb@xt@\Gm@layoutheight{%
       \let\protect\relax
       \hb@xt@\Gm@layoutwidth{\Gm@cropmark(-1,1,-3,3)\hfil\Gm@cropmark(1,1,3,3)}%
       \vfil
       \hb@xt@\Gm@layoutwidth{\Gm@cropmark(-1,-1,-3,-3)\hfil\Gm@cropmark(1,-1,3,-3)}}%
     \hss}%
    \vss}%
   \fi%
   \ifGm@showframe
    \if@twoside
     \ifodd\count\z@
       \let\@themargin\oddsidemargin
     \else
       \let\@themargin\evensidemargin
     \fi
    \fi
    \moveright\@themargin%
    \vb@xt@\z@{%
     \vskip\topmargin\vb@xt@\z@{\vss\Gm@hrule}%
     \vskip\headheight\vb@xt@\z@{\vss\Gm@hruled}%
     \vskip\headsep\vb@xt@\z@{\vss\Gm@hrule}%
     \@tempdima\textwidth
     \advance\@tempdima by \marginparsep
     \advance\@tempdima by \marginparwidth
     \if@mparswitch
      \ifodd\count\z@
       \Gm@vrules@mpi
      \else
       \Gm@vrules@mpii
      \fi
     \else
      \Gm@vrules@mpi
     \fi
     \vb@xt@\z@{\vss\Gm@hrule}%
     \vskip\footskip\vb@xt@\z@{\vss\Gm@hruled}%
     \vss}%
    \fi%
  }}%
%    \end{macrocode}
%
%    \begin{macro}{\ProcessOptionsKV}
%    此宏可以使用“key=value”方案处理类和包选项。默认情况下,只有类选项(class options)
%    使用可选参数(optional argument)“|c|”进行处理,包选项(package options)使用“|p|”进行处理,
%    默认情况下两者都使用。
%    \begin{macrocode}
\def\ProcessOptionsKV{\@ifnextchar[%]
  {\@ProcessOptionsKV}{\@ProcessOptionsKV[]}}%
\def\@ProcessOptionsKV[#1]#2{%
  \let\@tempa\@empty
  \@tempcnta\z@
  \if#1p\@tempcnta\@ne\else\if#1c\@tempcnta\tw@\fi\fi
  \ifodd\@tempcnta
   \edef\@tempa{\@ptionlist{\@currname.\@currext}}%
  \else
    \@for\CurrentOption:=\@classoptionslist\do{%
      \@ifundefined{KV@#2@\CurrentOption}%
      {}{\edef\@tempa{\@tempa,\CurrentOption,}}}%
    \ifnum\@tempcnta=\z@
      \edef\@tempa{\@tempa,\@ptionlist{\@currname.\@currext}}%
    \fi
  \fi
  \edef\@tempa{\noexpand\setkeys{#2}{\@tempa}}%
  \@tempa
  \AtEndOfPackage{\let\@unprocessedoptions\relax}}%
%    \end{macrocode}
%    \end{macro}
%    \begin{macrocode}
\def\Gm@setkeys{\setkeys{Gm}}%
%    \end{macrocode}
%    \begin{macro}{\Gm@processconf}
%    将 \cs{ExecuteOptions}\ 替换为 \cs{Gm@setkey},以便可以将“\meta{key}=\meta{value}”作为参数处理。
%    \begin{macrocode}
\def\Gm@processconfig{%
  \let\Gm@origExecuteOptions\ExecuteOptions
  \let\ExecuteOptions\Gm@setkeys
  \InputIfFileExists{geometry.cfg}{}{}
  \let\ExecuteOptions\Gm@origExecuteOptions}%
%    \end{macrocode}
%    \end{macro}
%
%  加载 \Gm\ 之前的原始页面布局(original page layout)保存在此处。
%  |\Gm@restore@org| 在此处定义为 |reset| 选项。
%    \begin{macrocode}
\Gm@save
\edef\Gm@restore@org{\Gm@restore}%
\Gm@initall
%    \end{macrocode}
%    处理配置文件(config file)。
%    \begin{macrocode}
\Gm@processconfig
%    \end{macrocode}
%    在此处处理 \cs{documentclass}\ 的可选参数。
%    \begin{macrocode}
\ProcessOptionsKV[c]{Gm}%
%    \end{macrocode}
%    存储类(class)默认给定的纸张尺寸。
%    \begin{macrocode}
\Gm@setdefaultpaper
%    \end{macrocode}
%    在此处处理 \cs{usepackage}\ 的可选参数。
%    \begin{macrocode}
\ProcessOptionsKV[p]{Gm}%
%    \end{macrocode}
%    处理布局尺寸(layout dimensions)的实际设置(actual settings)和计算(calculation)。
%    \begin{macrocode}
\Gm@process
%    \end{macrocode}
%
%    \begin{macro}{\AtBeginDocument}
%   |verbose|、|showframe|和驱动程序(drivers)的进程(processes)将添加到 |\AtBeginDocument| 中。
%   |\Gm@restore@org| 在这里被重新定义,其纸张尺寸在前言中指定,以便 |\newgeometry| 使用。
%   这应该在使用 |\mag| 放大(magnifying)纸张尺寸之前完成,
%   因为更改纸张尺寸会影响布局计算(layout calculation)。
%    \begin{macrocode}
\AtBeginDocument{%
  \Gm@savelength{paperwidth}%
  \Gm@savelength{paperheight}%
  \edef\Gm@restore@org{\Gm@restore}%
%    \end{macrocode}
%    如果 |resetpaper|,则使用原始纸张尺寸(original paper size)。
%    \begin{macrocode}
  \ifGm@resetpaper
    \edef\Gm@pw{\Gm@orgpw}%
    \edef\Gm@ph{\Gm@orgph}%
  \else
    \edef\Gm@pw{\the\paperwidth}%
    \edef\Gm@ph{\the\paperheight}%
  \fi
%    \end{macrocode}
%    如果未设置 |pass|,将根据指定的 |mag| 乘以纸张尺寸。
%    \begin{macrocode}
  \ifGm@pass\else
    \ifnum\mag=\@m\else
      \Gm@magtooffset
      \divide\paperwidth\@m
      \multiply\paperwidth\the\mag
      \divide\paperheight\@m
      \multiply\paperheight\the\mag
    \fi
  \fi
%    \end{macrocode}
%    检查驱动程序选项(driver options)。
%    \begin{macrocode}
  \Gm@detectdriver
%    \end{macrocode}
%    如果定义了 |xetex| 和 |\pdfpagewidth|,则将设置 |\pdfpagewidth| 和 |\pdfpageheight|。
%    \begin{macrocode}
  \ifx\Gm@driver\Gm@xetex
    \@ifundefined{pdfpagewidth}{}{%
      \setlength\pdfpagewidth{\Gm@pw}%
      \setlength\pdfpageheight{\Gm@ph}}%
    \ifnum\mag=\@m\else
      \ifx\Gm@truedimen\Gm@true
        \setlength\paperwidth{\Gm@pw}%
        \setlength\paperheight{\Gm@ph}%
      \fi
    \fi
  \fi
%    \end{macrocode}
%    如果 |pdftex| 设置为 |true|,则 pdf 命令(pdf-commands)设置正确。为避免 |pdftex| 放大问题,
%    将 \cs{pdfhorigin}\ 和 \cs{pdfvorigin}\ 调整为 \cs{mag}。
%    \begin{macrocode}
  \ifx\Gm@driver\Gm@pdftex
    \@ifundefined{pdfpagewidth}{}{%
      \setlength\pdfpagewidth{\Gm@pw}%
      \setlength\pdfpageheight{\Gm@ph}}%
    \ifnum\mag=\@m\else
      \@tempdima=\mag sp%
      \@ifundefined{pdfhorigin}{}{%
        \divide\pdfhorigin\@tempdima
        \multiply\pdfhorigin\@m
        \divide\pdfvorigin\@tempdima
        \multiply\pdfvorigin\@m}%
      \ifx\Gm@truedimen\Gm@true
        \setlength\paperwidth{\Gm@pw}%
        \setlength\paperheight{\Gm@ph}%
      \fi
    \fi
  \fi
%    \end{macrocode}
%    如果将 |luatex| 设置为 |true|,则 pdf 命令(pdf-commands)将正确设置。为避免 |luatex| 放大问题,
%    将 \cs{horigin}\ 和 \cs{vorigin}\ 调整为 \cs{mag}。
%    \begin{macrocode}
  \ifx\Gm@driver\Gm@luatex
    \setlength\pagewidth{\Gm@pw}%
    \setlength\pageheight{\Gm@ph}%
    \ifnum\mag=\@m\else
      \@tempdima=\mag sp
        \edef\Gm@horigin{\pdfvariable horigin}%
        \edef\Gm@vorigin{\pdfvariable vorigin}%
        \divide\Gm@horigin\@tempdima
        \multiply\Gm@horigin\@m
        \divide\Gm@vorigin\@tempdima
        \multiply\Gm@vorigin\@m
      \ifx\Gm@truedimen\Gm@true
        \setlength\paperwidth{\Gm@pw}%
        \setlength\paperheight{\Gm@ph}%
      \fi
    \fi
  \fi
%    \end{macrocode}
%    对于 V\TeX{}\ 环境,此处设置 V\TeX{}\ 变量。
%    \begin{macrocode}
  \ifx\Gm@driver\Gm@vtex
    \@ifundefined{mediawidth}{}{%
      \mediawidth=\paperwidth
      \mediaheight=\paperheight}%
    \ifvtexdvi
      \AtBeginDvi{\special{papersize=\the\paperwidth,\the\paperheight}}%
    \fi
  \fi
%    \end{macrocode}
%    如果指定了 |dvips| 或 |dvipdfm|,则纸张尺寸(paper size)将使用 \cs{special}\ 嵌入 dvi 文件中。
%    对于 dvips,添加了横向校正(landscape correction),因为 dvips 转换的横向文档(landscape document)
%    在 PostScript 查看器(viewers)中是颠倒的。
%    \begin{macrocode}
  \ifx\Gm@driver\Gm@dvips
    \AtBeginDvi{\special{papersize=\the\paperwidth,\the\paperheight}}%
    \ifx\Gm@driver\Gm@dvips\ifGm@landscape
      \AtBeginDvi{\special{! /landplus90 true store}}%
    \fi\fi
%    \end{macrocode}
%    如果指定了 |dvipdfm|,并且加载了“oberdiek”捆绑包(bundle)中的 \textsf{atbegshi}\ 宏包,
%    则使用 \cs{AtBeginShipoutFirst}\ 而不是 \cs{AtBeginDvi},以
%    与 \textsf{hyperref}\ 和 |dvipdfm| 程序兼容。
%    \begin{macrocode}
  \else\ifx\Gm@driver\Gm@dvipdfm
    \ifcase\ifx\AtBeginShipoutFirst\relax\@ne\else
        \ifx\AtBeginShipoutFirst\@undefined\@ne\else\z@\fi\fi
      \AtBeginShipoutFirst{\special{papersize=\the\paperwidth,\the\paperheight}}%
    \or
      \AtBeginDvi{\special{papersize=\the\paperwidth,\the\paperheight}}%
    \fi
  \fi\fi
%    \end{macrocode}
%    当 |showframe=true| 时,每页上的页面框架(page frames)被发送出去(shipped out);
%    当 |showcrop=true| 时,每页上的裁切标记(cropmarks)被发送出去(shipped out)。
%    \textsf{atbegshi}\ 宏包用于重载(overloading) |\shipout|。
%    【译者注:这句翻译可能有误,这句的原文是:
%    Page frames are shipped out when |showframe=true|, cropmarks for |showcrop=true|
%    on each page. The \textsf{atbegshi} package is used for overloading |\shipout|.】
%    \begin{macrocode}
  \@tempswafalse
  \ifGm@showframe
    \@tempswatrue
  \else\ifGm@showcrop
    \@tempswatrue
  \fi\fi
  \if@tempswa
    \RequirePackage{atbegshi}%
      \AtBeginShipout{\setbox\AtBeginShipoutBox=\vbox{%
        \baselineskip\z@skip\lineskip\z@skip\lineskiplimit\z@
        \Gm@pageframes\box\AtBeginShipoutBox}}%
  \fi
%    \end{macrocode}
%    \cs{restoregeometry}\ 的布局尺寸(layout dimensions)保存在 \cs{AtBeginDocument}\ 的末尾。
%    \begin{macrocode}
  \Gm@save
  \edef\Gm@restore@pkg{\Gm@restore}%
%    \end{macrocode}
%    该宏包检查 |marginpars| 是否超出页面(overrun the page),如果是 |verbose|,除非是 |pass|。
%    \begin{macrocode}
  \ifGm@verbose\ifGm@pass\else\Gm@checkmp\fi\fi
%    \end{macrocode}
%   |\Gm@showparams| 将生成的参数(resulting parameters)和尺寸(dimensions)放入日志文件(log file)。
%   通过 |verbose|,它们也会显示在终端(terminal)上。
%    \begin{macrocode}
  \Gm@showparams{preamble}%
%    \end{macrocode}
%   以下几行释放了(free)不再需要的内存。
%    \begin{macrocode}
  \let\Gm@pw\relax
  \let\Gm@ph\relax
  }% end of \AtBeginDocument
%    \end{macrocode}
%    \end{macro}
%
%    \begin{macro}{\geometry}
%    可以在前言中(在 |\begin{document}| 之前)多次调用\cs{geometry}。
%    \begin{macrocode}
\newcommand{\geometry}[1]{%
  \Gm@clean
  \setkeys{Gm}{#1}%
  \Gm@process}%
\@onlypreamble\geometry
%    \end{macrocode}
%    \end{macro}
%    \begin{macro}{\Gm@changelayout}
%   可以从 |\newgeometry|、|\restoregeometry| 和 |\loadgeometry| 调用该宏,它会更改文档中间的布局。
%    \begin{macrocode}
\DeclareRobustCommand\Gm@changelayout{%
  \setlength{\@colht}{\textheight}
  \setlength{\@colroom}{\textheight}%
  \setlength{\vsize}{\textheight}
  \setlength{\columnwidth}{\textwidth}%
  \if@twocolumn%
    \advance\columnwidth-\columnsep
    \divide\columnwidth\tw@%
    \@firstcolumntrue%
  \fi%
  \setlength{\hsize}{\columnwidth}%
  \setlength{\linewidth}{\hsize}}%
%    \end{macrocode}
%    \end{macro}
%    \begin{macro}{\newgeometry}
%    更改布局的 |\newgeometry| 宏只能在文档中使用。它将重置前言中指定的选项,
%    但纸张尺寸选项和 |\mag| 除外。
%    \begin{macrocode}
\newcommand{\newgeometry}[1]{%
  \clearpage
  \Gm@restore@org
  \Gm@initnewgm
  \Gm@newgmtrue
  \setkeys{Gm}{#1}%
  \Gm@newgmfalse
  \Gm@process
  \ifnum\mag=\@m\else\Gm@magtooffset\fi
  \Gm@changelayout
  \Gm@showparams{newgeometry}}%
%    \end{macrocode}
%    \end{macro}
%    \begin{macro}{\restoregeometry}
%    该宏将恢复(restores)前言中指定的结果布局(resulting layout),即 |\begin{document}| 之
%    后的第一页布局(first-page layout)。
%    \begin{macrocode}
\newcommand{\restoregeometry}{%
  \clearpage
  \Gm@restore@pkg
  \Gm@changelayout}%
%    \end{macrocode}
%    \end{macro}
%    \begin{macro}{\savegeometry}
%    宏使用参数指定的名称保存布局。可以使用 |\loadgeometry{|\meta{name}|}| 加载保存的布局。
%    \begin{macrocode}
\newcommand*{\savegeometry}[1]{%
  \Gm@save
  \expandafter\edef\csname Gm@restore@@#1\endcsname{\Gm@restore}}%
%    \end{macrocode}
%    \end{macro}
%    \begin{macro}{\loadgeometry}
%    该宏加载使用 |\savegeometry{|\meta{name}|}| 保存的布局。如果找不到该名称(name),
%    该宏将对其发出警告,对布局不做任何操作。
%    \begin{macrocode}
\newcommand*{\loadgeometry}[1]{%
  \clearpage
  \@ifundefined{Gm@restore@@#1}{%
    \PackageError{geometry}{%
    \string\loadgeometry : name `#1' undefined}{%
    The name `#1' should be predefined with \string\savegeometry}%
  }{\@nameuse{Gm@restore@@#1}%
  \Gm@changelayout}}%
%</package>
%    \end{macrocode}
%    \end{macro}
%
% \clearpage
% \section{\heiti 配置文件}
%    在 |geometry.cfg| 配置文件中,可以使用 \cs{ExecuteOptions}\ 设置站点(site)或用户默认设置。
%    \begin{macrocode}
%<*config>
%<<SAVE_INTACT
%  取消注释(uncomment)并编辑下面的行以设置默认选项。
%\ExecuteOptions{a4paper}

%SAVE_INTACT
%</config>
%    \end{macrocode}
%
% \clearpage
% \section{\heiti 示例文件}
%    下面是 \Gm\ 宏包的示例文件:
%    \begin{macrocode}
%<*samples>
%<<SAVE_INTACT
\documentclass[12pt]{article}% 默认使用信纸(letterpaper)
% \documentclass[12pt,a4paper]{article}% 用于 A4 纸
%---------------------------------------------------------------
% 编辑和取消注释(uncomment)下面的设置之一
%---------------------------------------------------------------
% \usepackage{geometry}
% \usepackage[centering]{geometry}
% \usepackage[width=10cm,vscale=.7]{geometry}
% \usepackage[margin=1cm, papersize={12cm,19cm}, resetpaper]{geometry}
% \usepackage[margin=1cm,includeheadfoot]{geometry}
\usepackage[margin=1cm,includeheadfoot,includemp]{geometry}
% \usepackage[margin=1cm,bindingoffset=1cm,twoside]{geometry}
% \usepackage[hmarginratio=2:1, vmargin=2cm]{geometry}
% \usepackage[hscale=0.5,twoside]{geometry}
% \usepackage[hscale=0.5,asymmetric]{geometry}
% \usepackage[hscale=0.5,heightrounded]{geometry}
% \usepackage[left=1cm,right=4cm,top=2cm,includefoot]{geometry}
% \usepackage[lines=20,left=2cm,right=6cm,top=2cm,twoside]{geometry}
% \usepackage[width=15cm, marginparwidth=3cm, includemp]{geometry}
% \usepackage[hdivide={1cm,,2cm}, vdivide={3cm,8in,}, nohead]{geometry}
% \usepackage[headsep=20pt, head=40pt,foot=20pt,includeheadfoot]{geometry}
% \usepackage[text={6in,8in}, top=2cm, left=2cm]{geometry}
% \usepackage[centering,includemp,twoside,landscape]{geometry}
% \usepackage[mag=1414,margin=2cm]{geometry}
% \usepackage[mag=1414,margin=2truecm,truedimen]{geometry}
% \usepackage[a5paper, landscape, twocolumn, twoside,
%    left=2cm, hmarginratio=2:1, includemp, marginparwidth=43pt,
%    bottom=1cm, foot=.7cm, includefoot, textheight=11cm, heightrounded,
%    columnsep=1cm,verbose]{geometry}
%---------------------------------------------------------------
% 无需更改以下内容
%---------------------------------------------------------------
\geometry{verbose,showframe}% the options appended.
\usepackage{lipsum}% for dummy text of 150 paragraphs
\newcommand\mynote{\marginpar[\raggedright
A sample margin note in the left side.]%
{\raggedright A sample margin note.}}%
\newcommand\myfootnote{\footnote{This is a sample footnote text.}}
\begin{document}
\lipsum[1-2]\mynote\lipsum[3-4]\mynote
\lipsum[5-11]\mynote\lipsum[12]\myfootnote
\lipsum[13-22]\mynote\lipsum[23-32]
\end{document}
%SAVE_INTACT
%</samples>
%    \end{macrocode}
%
% \Finale
%
\endinput
