% \iffalse meta-comment
%
% File: bookmark.dtx
% Version: 2020-11-06 v1.29
% Info: PDF bookmarks
%
% Copyright (C)
%    2007-2011 Heiko Oberdiek
%    2016-2020 Oberdiek Package Support Group
%    https://github.com/ho-tex/bookmark/issues
%
% This work may be distributed and/or modified under the
% conditions of the LaTeX Project Public License, either
% version 1.3c of this license or (at your option) any later
% version. This version of this license is in
%    https://www.latex-project.org/lppl/lppl-1-3c.txt
% and the latest version of this license is in
%    https://www.latex-project.org/lppl.txt
% and version 1.3 or later is part of all distributions of
% LaTeX version 2005/12/01 or later.
%
% This work has the LPPL maintenance status "maintained".
%
% The Current Maintainers of this work are
% Heiko Oberdiek and the Oberdiek Package Support Group
% https://github.com/ho-tex/bookmark/issues
%
% This work consists of the main source file bookmark.dtx
% and the derived files
%    bookmark.sty, bookmark.pdf, bookmark.ins, bookmark.drv,
%    bkm-dvipdfm.def, bkm-dvips.def,
%    bkm-pdftex.def, bkm-vtex.def,
%    bkm-dvipdfm-2019-12-03.def, bkm-dvips-2019-12-03.def,
%    bkm-pdftex-2019-12-03.def, bkm-vtex-2019-12-03.def,
%    bookmark-example.tex.
%
% Distribution:
%    CTAN:macros/latex/contrib/bookmark/bookmark.dtx
%    CTAN:macros/latex/contrib/bookmark/bookmark-frozen.dtx
%    CTAN:macros/latex/contrib/bookmark/bookmark.pdf
%
% Unpacking:
%    (a) If bookmark.ins is present:
%           tex bookmark.ins
%    (b) Without bookmark.ins:
%           tex bookmark.dtx
%    (c) If you insist on using LaTeX
%           latex \let\install=y% \iffalse meta-comment
%
% File: bookmark.dtx
% Version: 2020-11-06 v1.29
% Info: PDF bookmarks
%
% Copyright (C)
%    2007-2011 Heiko Oberdiek
%    2016-2020 Oberdiek Package Support Group
%    https://github.com/ho-tex/bookmark/issues
%
% This work may be distributed and/or modified under the
% conditions of the LaTeX Project Public License, either
% version 1.3c of this license or (at your option) any later
% version. This version of this license is in
%    https://www.latex-project.org/lppl/lppl-1-3c.txt
% and the latest version of this license is in
%    https://www.latex-project.org/lppl.txt
% and version 1.3 or later is part of all distributions of
% LaTeX version 2005/12/01 or later.
%
% This work has the LPPL maintenance status "maintained".
%
% The Current Maintainers of this work are
% Heiko Oberdiek and the Oberdiek Package Support Group
% https://github.com/ho-tex/bookmark/issues
%
% This work consists of the main source file bookmark.dtx
% and the derived files
%    bookmark.sty, bookmark.pdf, bookmark.ins, bookmark.drv,
%    bkm-dvipdfm.def, bkm-dvips.def,
%    bkm-pdftex.def, bkm-vtex.def,
%    bkm-dvipdfm-2019-12-03.def, bkm-dvips-2019-12-03.def,
%    bkm-pdftex-2019-12-03.def, bkm-vtex-2019-12-03.def,
%    bookmark-example.tex.
%
% Distribution:
%    CTAN:macros/latex/contrib/bookmark/bookmark.dtx
%    CTAN:macros/latex/contrib/bookmark/bookmark-frozen.dtx
%    CTAN:macros/latex/contrib/bookmark/bookmark.pdf
%
% Unpacking:
%    (a) If bookmark.ins is present:
%           tex bookmark.ins
%    (b) Without bookmark.ins:
%           tex bookmark.dtx
%    (c) If you insist on using LaTeX
%           latex \let\install=y% \iffalse meta-comment
%
% File: bookmark.dtx
% Version: 2020-11-06 v1.29
% Info: PDF bookmarks
%
% Copyright (C)
%    2007-2011 Heiko Oberdiek
%    2016-2020 Oberdiek Package Support Group
%    https://github.com/ho-tex/bookmark/issues
%
% This work may be distributed and/or modified under the
% conditions of the LaTeX Project Public License, either
% version 1.3c of this license or (at your option) any later
% version. This version of this license is in
%    https://www.latex-project.org/lppl/lppl-1-3c.txt
% and the latest version of this license is in
%    https://www.latex-project.org/lppl.txt
% and version 1.3 or later is part of all distributions of
% LaTeX version 2005/12/01 or later.
%
% This work has the LPPL maintenance status "maintained".
%
% The Current Maintainers of this work are
% Heiko Oberdiek and the Oberdiek Package Support Group
% https://github.com/ho-tex/bookmark/issues
%
% This work consists of the main source file bookmark.dtx
% and the derived files
%    bookmark.sty, bookmark.pdf, bookmark.ins, bookmark.drv,
%    bkm-dvipdfm.def, bkm-dvips.def,
%    bkm-pdftex.def, bkm-vtex.def,
%    bkm-dvipdfm-2019-12-03.def, bkm-dvips-2019-12-03.def,
%    bkm-pdftex-2019-12-03.def, bkm-vtex-2019-12-03.def,
%    bookmark-example.tex.
%
% Distribution:
%    CTAN:macros/latex/contrib/bookmark/bookmark.dtx
%    CTAN:macros/latex/contrib/bookmark/bookmark-frozen.dtx
%    CTAN:macros/latex/contrib/bookmark/bookmark.pdf
%
% Unpacking:
%    (a) If bookmark.ins is present:
%           tex bookmark.ins
%    (b) Without bookmark.ins:
%           tex bookmark.dtx
%    (c) If you insist on using LaTeX
%           latex \let\install=y% \iffalse meta-comment
%
% File: bookmark.dtx
% Version: 2020-11-06 v1.29
% Info: PDF bookmarks
%
% Copyright (C)
%    2007-2011 Heiko Oberdiek
%    2016-2020 Oberdiek Package Support Group
%    https://github.com/ho-tex/bookmark/issues
%
% This work may be distributed and/or modified under the
% conditions of the LaTeX Project Public License, either
% version 1.3c of this license or (at your option) any later
% version. This version of this license is in
%    https://www.latex-project.org/lppl/lppl-1-3c.txt
% and the latest version of this license is in
%    https://www.latex-project.org/lppl.txt
% and version 1.3 or later is part of all distributions of
% LaTeX version 2005/12/01 or later.
%
% This work has the LPPL maintenance status "maintained".
%
% The Current Maintainers of this work are
% Heiko Oberdiek and the Oberdiek Package Support Group
% https://github.com/ho-tex/bookmark/issues
%
% This work consists of the main source file bookmark.dtx
% and the derived files
%    bookmark.sty, bookmark.pdf, bookmark.ins, bookmark.drv,
%    bkm-dvipdfm.def, bkm-dvips.def,
%    bkm-pdftex.def, bkm-vtex.def,
%    bkm-dvipdfm-2019-12-03.def, bkm-dvips-2019-12-03.def,
%    bkm-pdftex-2019-12-03.def, bkm-vtex-2019-12-03.def,
%    bookmark-example.tex.
%
% Distribution:
%    CTAN:macros/latex/contrib/bookmark/bookmark.dtx
%    CTAN:macros/latex/contrib/bookmark/bookmark-frozen.dtx
%    CTAN:macros/latex/contrib/bookmark/bookmark.pdf
%
% Unpacking:
%    (a) If bookmark.ins is present:
%           tex bookmark.ins
%    (b) Without bookmark.ins:
%           tex bookmark.dtx
%    (c) If you insist on using LaTeX
%           latex \let\install=y\input{bookmark.dtx}
%        (quote the arguments according to the demands of your shell)
%
% Documentation:
%    (a) If bookmark.drv is present:
%           latex bookmark.drv
%    (b) Without bookmark.drv:
%           latex bookmark.dtx; ...
%    The class ltxdoc loads the configuration file ltxdoc.cfg
%    if available. Here you can specify further options, e.g.
%    use A4 as paper format:
%       \PassOptionsToClass{a4paper}{article}
%
%    Programm calls to get the documentation (example):
%       pdflatex bookmark.dtx
%       makeindex -s gind.ist bookmark.idx
%       pdflatex bookmark.dtx
%       makeindex -s gind.ist bookmark.idx
%       pdflatex bookmark.dtx
%
% Installation:
%    TDS:tex/latex/bookmark/bookmark.sty
%    TDS:tex/latex/bookmark/bkm-dvipdfm.def
%    TDS:tex/latex/bookmark/bkm-dvips.def
%    TDS:tex/latex/bookmark/bkm-pdftex.def
%    TDS:tex/latex/bookmark/bkm-vtex.def
%    TDS:tex/latex/bookmark/bkm-dvipdfm-2019-12-03.def
%    TDS:tex/latex/bookmark/bkm-dvips-2019-12-03.def
%    TDS:tex/latex/bookmark/bkm-pdftex-2019-12-03.def
%    TDS:tex/latex/bookmark/bkm-vtex-2019-12-03.def%
%    TDS:doc/latex/bookmark/bookmark.pdf
%    TDS:doc/latex/bookmark/bookmark-example.tex
%    TDS:source/latex/bookmark/bookmark.dtx
%    TDS:source/latex/bookmark/bookmark-frozen.dtx
%
%<*ignore>
\begingroup
  \catcode123=1 %
  \catcode125=2 %
  \def\x{LaTeX2e}%
\expandafter\endgroup
\ifcase 0\ifx\install y1\fi\expandafter
         \ifx\csname processbatchFile\endcsname\relax\else1\fi
         \ifx\fmtname\x\else 1\fi\relax
\else\csname fi\endcsname
%</ignore>
%<*install>
\input docstrip.tex
\Msg{************************************************************************}
\Msg{* Installation}
\Msg{* Package: bookmark 2020-11-06 v1.29 PDF bookmarks (HO)}
\Msg{************************************************************************}

\keepsilent
\askforoverwritefalse

\let\MetaPrefix\relax
\preamble

This is a generated file.

Project: bookmark
Version: 2020-11-06 v1.29

Copyright (C)
   2007-2011 Heiko Oberdiek
   2016-2020 Oberdiek Package Support Group

This work may be distributed and/or modified under the
conditions of the LaTeX Project Public License, either
version 1.3c of this license or (at your option) any later
version. This version of this license is in
   https://www.latex-project.org/lppl/lppl-1-3c.txt
and the latest version of this license is in
   https://www.latex-project.org/lppl.txt
and version 1.3 or later is part of all distributions of
LaTeX version 2005/12/01 or later.

This work has the LPPL maintenance status "maintained".

The Current Maintainers of this work are
Heiko Oberdiek and the Oberdiek Package Support Group
https://github.com/ho-tex/bookmark/issues


This work consists of the main source file bookmark.dtx and bookmark-frozen.dtx
and the derived files
   bookmark.sty, bookmark.pdf, bookmark.ins, bookmark.drv,
   bkm-dvipdfm.def, bkm-dvips.def, bkm-pdftex.def, bkm-vtex.def,
   bkm-dvipdfm-2019-12-03.def, bkm-dvips-2019-12-03.def,
   bkm-pdftex-2019-12-03.def, bkm-vtex-2019-12-03.def,
   bookmark-example.tex.

\endpreamble
\let\MetaPrefix\DoubleperCent

\generate{%
  \file{bookmark.ins}{\from{bookmark.dtx}{install}}%
  \file{bookmark.drv}{\from{bookmark.dtx}{driver}}%
  \usedir{tex/latex/bookmark}%
  \file{bookmark.sty}{\from{bookmark.dtx}{package}}%
  \file{bkm-dvipdfm.def}{\from{bookmark.dtx}{dvipdfm}}%
  \file{bkm-dvips.def}{\from{bookmark.dtx}{dvips,pdfmark}}%
  \file{bkm-pdftex.def}{\from{bookmark.dtx}{pdftex}}%
  \file{bkm-vtex.def}{\from{bookmark.dtx}{vtex}}%
  \usedir{doc/latex/bookmark}%
  \file{bookmark-example.tex}{\from{bookmark.dtx}{example}}%
  \file{bkm-pdftex-2019-12-03.def}{\from{bookmark-frozen.dtx}{pdftexfrozen}}%
  \file{bkm-dvips-2019-12-03.def}{\from{bookmark-frozen.dtx}{dvipsfrozen}}%
  \file{bkm-vtex-2019-12-03.def}{\from{bookmark-frozen.dtx}{vtexfrozen}}%
  \file{bkm-dvipdfm-2019-12-03.def}{\from{bookmark-frozen.dtx}{dvipdfmfrozen}}%
}

\catcode32=13\relax% active space
\let =\space%
\Msg{************************************************************************}
\Msg{*}
\Msg{* To finish the installation you have to move the following}
\Msg{* files into a directory searched by TeX:}
\Msg{*}
\Msg{*     bookmark.sty, bkm-dvipdfm.def, bkm-dvips.def,}
\Msg{*     bkm-pdftex.def, bkm-vtex.def, bkm-dvipdfm-2019-12-03.def,}
\Msg{*     bkm-dvips-2019-12-03.def, bkm-pdftex-2019-12-03.def,}
\Msg{*     and bkm-vtex-2019-12-03.def}
\Msg{*}
\Msg{* To produce the documentation run the file `bookmark.drv'}
\Msg{* through LaTeX.}
\Msg{*}
\Msg{* Happy TeXing!}
\Msg{*}
\Msg{************************************************************************}

\endbatchfile
%</install>
%<*ignore>
\fi
%</ignore>
%<*driver>
\NeedsTeXFormat{LaTeX2e}
\ProvidesFile{bookmark.drv}%
  [2020-11-06 v1.29 PDF bookmarks (HO)]%
\documentclass{ltxdoc}
\usepackage{ctex}
\usepackage{indentfirst}
\setlength{\parindent}{2em}
\usepackage{holtxdoc}[2011/11/22]
\usepackage{xcolor}
\usepackage{hyperref}
\usepackage[open,openlevel=3,atend]{bookmark}[2020/11/06] %%%打开书签,显示的深度为3级,即显示part、section、subsection。
\bookmarksetup{color=red}
\begin{document}

  \renewcommand{\contentsname}{目\quad 录}
  \renewcommand{\abstractname}{摘\quad 要}
  \renewcommand{\historyname}{历史}
  \DocInput{bookmark.dtx}%
\end{document}
%</driver>
% \fi
%
%
%
% \GetFileInfo{bookmark.drv}
%
%% \title{\xpackage{bookmark} 宏包}
% \title{\heiti {\Huge \textbf{\xpackage{bookmark}\ 宏包}}}
% \date{2020-11-06\ \ \ v1.29}
% \author{Heiko Oberdiek \thanks
% {如有问题请点击:\url{https://github.com/ho-tex/bookmark/issues}}\\[5pt]赣医一附院神经科\ \ 黄旭华\ \ \ \ 译}
%
% \maketitle
%
% \begin{abstract}
% 这个宏包为 \xpackage{hyperref}\ 宏包实现了一个新的书签(bookmark)(大纲[outline])组织。现在
% 可以设置样式(style)和颜色(color)等书签属性(bookmark properties)。其他动作类型(action types)可用
% (URI、GoToR、Named)。书签是在第一次编译运行(compile run)中生成的。\xpackage{hyperref}\
% 宏包必需运行两次。
% \end{abstract}
%
% \tableofcontents
%
% \section{文档(Documentation)}
%
% \subsection{介绍}
%
% 这个 \xpackage{bookmark}\ 宏包试图为书签(bookmarks)提供一个更现代的管理:
% \begin{itemize}
% \item 书签已经在第一次 \hologo{TeX}\ 编译运行(compile run)中生成。
% \item 可以更改书签的字体样式(font style)和颜色(color)。
% \item 可以执行比简单的 GoTo 操作(actions)更多的操作。
% \end{itemize}
%
% 与 \xpackage{hyperref} \cite{hyperref} 一样,书签(bookmarks)也是按照书签生成宏
% (bookmark generating macros)(\cs{bookmark})的顺序生成的。级别号(level number)用于
% 定义书签的树结构(tree structure)。限制没有那么严格:
% \begin{itemize}
% \item 级别值(level values)可以跳变(jump)和省略(omit)。\cs{subsubsection}\ 可以跟在
%       \cs{chapter}\ 之后。这种情况如在 \xpackage{hyperref}\ 中则产生错误,它将显示一个警告(warning)
%       并尝试修复此错误。
% \item 多个书签可能指向同一目标(destination)。在 \xpackage{hyperref}\ 中,这会完全弄乱
%       书签树(bookmark tree),因为算法假设(algorithm assumes)目标名称(destination names)
%       是键(keys)(唯一的)。
% \end{itemize}
%
% 注意,这个宏包是作为书签管理(bookmark management)的实验平台(experimentation platform)。
% 欢迎反馈。此外,在未来的版本中,接口(interfaces)也可能发生变化。
%
% \subsection{选项(Options)}
%
% 可在以下四个地方放置选项(options):
% \begin{enumerate}
% \item \cs{usepackage}|[|\meta{options}|]{bookmark}|\\
%       这是放置驱动程序选项(driver options)和 \xoption{atend}\ 选项的唯一位置。
% \item \cs{bookmarksetup}|{|\meta{options}|}|\\
%       此命令仅用于设置选项(setting options)。
% \item \cs{bookmarksetupnext}|{|\meta{options}|}|\\
%       这些选项在下一个 \cs{bookmark}\ 命令的选项之后存储(stored)和调用(called)。
% \item \cs{bookmark}|[|\meta{options}|]{|\meta{title}|}|\\
%       此命令设置书签。选项设置(option settings)仅限于此书签。
% \end{enumerate}
% 异常(Exception):加载该宏包后,无法更改驱动程序选项(Driver options)、\xoption{atend}\ 选项
% 、\xoption{draft}\slash\xoption{final}选项。
%
% \subsubsection{\xoption{draft} 和 \xoption{final}\ 选项}
%
% 如果一个\LaTeX\ 文件要被编译了多次,那么可以使用 \xoption{draft}\ 选项来禁用该宏包的书签内
% 容(bookmark stuff),这样可以节省一点时间。默认 \xoption{final}\ 选项。两个选项都是
% 布尔选项(boolean options),如果没有值,则使用值 |true|。|draft=true| 与 |final=false| 相同。
%
% 除了驱动程序选项(driver options)之外,\xpackage{bookmark}\ 宏包选项都是局部选项(local options)。
% \xoption{draft}\ 选项和 \xoption{final}\ 选项均属于文档类选项(class option)(译者注:文档类选项为全局选项),
% 因此,在 \xpackage{bookmark}\ 宏包中未能看到这两个选项。如果您想使用全局的(global) \xoption{draft}选项
% 来优化第一次 \LaTeX\ 运行(runs),可以在导言(preamble)中引入 \xpackage{ifdraft}\ 宏包并设置 \LaTeX\ 的
% \cs{PassOptionsToPackage},例如:
%\begin{quote}
%\begin{verbatim}
%\documentclass[draft]{article}
%\usepackage{ifdraft}
%\ifdraft{%
%   \PassOptionsToPackage{draft}{bookmark}%
%}{}
%\end{verbatim}
%\end{quote}
%
% \subsubsection{驱动程序选项(Driver options)}
%
% 支持的驱动程序( drivers)包括 \xoption{pdftex}、\xoption{dvips}、\xoption{dvipdfm} (\xoption{xetex})、
% \xoption{vtex}。\hologo{TeX}\ 引擎 \hologo{pdfTeX}、\hologo{XeTeX}、\hologo{VTeX}\ 能被自动检测到。
% 默认的 DVI 驱动程序是 \xoption{dvips}。这可以通过 \cs{BookmarkDriverDefault}\ 在配置
% 文件 \xfile{bookmark.cfg}\ 中进行更改,例如:
% \begin{quote}
% |\def\BookmarkDriverDefault{dvipdfm}|
% \end{quote}
% 当前版本的(current versions)驱动程序使用新的 \LaTeX\ 钩子(\LaTeX-hooks)。如果检测到比
% 2020-10-01 更旧的格式,则将以前驱动程序的冻结版本(frozen versions)作为备份(fallback)。
%
% \paragraph{用 dvipdfmx 打开书签(bookmarks)。}旧版本的宏包有一个 \xoption{dvipdfmx-outline-open}\ 选项
% 可以激活代码,而该代码可以指定一个大纲条目(outline entry)是否打开。该宏包现在假设所有使用的 dvipdfmx 版本都是
% 最新版本,足以理解该代码,因此始终激活该代码。选项本身将被忽略。
%
%
% \subsubsection{布局选项(Layout options)}
%
% \paragraph{字体(Font)选项:}
%
% \begin{description}
% \item[\xoption{bold}:] 如果受 PDF 浏览器(PDF viewer)支持,书签将以粗体字体(bold font)显示(自 PDF 1.4起)。
% \item[\xoption{italic}:] 使用斜体字体(italic font)(自 PDF 1.4起)。
% \end{description}
% \xoption{bold}(粗体) 和 \xoption{italic}(斜体)可以同时使用。而 |false| 值(value)禁用字体选项。
%
% \paragraph{颜色(Color)选项:}
%
% 彩色书签(Colored bookmarks)是 PDF 1.4 的一个特性(feature),并非所有的 PDF 浏览器(PDF viewers)都支持彩色书签。
% \begin{description}
% \item[\xoption{color}:] 这里 color(颜色)可以作为 \xpackage{color}\ 宏包或 \xpackage{xcolor}\ 宏包的
% 颜色规范(color specification)给出。空值(empty value)表示未设置颜色属性。如果未加载 \xpackage{xcolor}\ 宏包,
% 能识别的值(recognized values)只有:
%   \begin{itemize}
%   \item 空值(empty value)表示未设置颜色属性,\\
%         例如:|color={}|
%   \item 颜色模型(color model) rgb 的显式颜色规范(explicit color specification),\\
%         例如,红色(red):|color=[rgb]{1,0,0}|
%   \item 颜色模型(color model)灰(gray)的显式颜色规范(explicit color specification),\\
%         例如,深灰色(dark gray):|color=[gray]{0.25}|
%   \end{itemize}
%   请注意,如果加载了 \xpackage{color}\ 宏包,此限制(restriction)也适用。然而,如果加载了 \xpackage{xcolor}\ 宏包,
%   则可以使用所有颜色规范(color specifications)。
% \end{description}
%
% \subsubsection{动作选项(Action options)}
%
% \begin{description}
% \item[\xoption{dest}:] 目的地名称(destination name)。
% \item[\xoption{page}:] 页码(page number),第一页(first page)为 1。
% \item[\xoption{view}:] 浏览规范(view specification),示例如下:\\
%   |view={FitB}|, |view={FitH 842}|, |view={XYZ 0 100 null}|\ \  一些浏览规范参数(view specification parameters)
%   将数字(numbers)视为具有单位 bp 的参数。它们可以作为普通数字(plain numbers)或在 \cs{calc}\ 内部以
%   长度表达式(length expressions)给出。如果加载了 \xpackage{calc}\ 宏包,则支持该宏包的表达式(expressions)。否则,
%   使用 \hologo{eTeX}\ 的 \cs{dimexpr}。例如:\\
%   |view={FitH \calc{\paperheight-\topmargin-1in}}|\\
%   |view={XYZ 0 \calc{\paperheight} null}|\\
%   注意 \cs{calc}\ 不能用于 |XYZ| 的第三个参数,因为该参数是缩放值(zoom value),而不是长度(length)。

% \item[\xoption{named}:] 已命名的动作(Named action)的名称:\\
%   |FirstPage|(第一页),|LastPage|(最后一页),|NextPage|(下一页),|PrevPage|(前一页)
% \item[\xoption{gotor}:] 外部(external) PDF 文件的名称。
% \item[\xoption{uri}:] URI 规范(URI specification)。
% \item[\xoption{rawaction}:] 原始动作规范(raw action specification)。由于这些规范取决于驱动程序(driver),因此不应使用此选项。
% \end{description}
% 通过分析指定的选项来选择书签的适当动作。动作由不同的选项集(sets of options)区分:
% \begin{quote}
 \begin{tabular}{|@{}r|l@{}|}
%   \hline
%   \ \textbf{动作(Action)}\  & \ \textbf{选项(Options)}\ \\ \hline
%   \ \textsf{GoTo}\  &\  \xoption{dest}\ \\ \hline
%   \ \textsf{GoTo}\  & \ \xoption{page} + \xoption{view}\ \\ \hline
%   \ \textsf{GoToR}\  & \ \xoption{gotor} + \xoption{dest}\ \\ \hline
%   \ \textsf{GoToR}\  & \ \xoption{gotor} + \xoption{page} + \xoption{view}\ \ \ \\ \hline
%   \ \textsf{Named}\  &\  \xoption{named}\ \\ \hline
%   \ \textsf{URI}\  & \ \xoption{uri}\ \\ \hline
% \end{tabular}
% \end{quote}
%
% \paragraph{缺少动作(Missing actions)。}
% 如果动作缺少 \xpackage{bookmark}\ 宏包,则抛出错误消息(error message)。根据驱动程序(driver)
% (\xoption{pdftex}、\xoption{dvips}\ 和好友[friends]),宏包在文档末尾很晚才检测到它。
% 自 2011/04/21 v1.21 版本以后,该宏包尝试打印 \cs{bookmark}\ 的相应出现的行号(line number)和文件名(file name)。
% 然而,\hologo{TeX}\ 确实提供了行号,但不幸的是,文件名是一个秘密(secret)。但该宏包有如下获取文件名的方法:
% \begin{itemize}
% \item 如果 \hologo{LuaTeX} (独立于 DVI 或 PDF 模式)正在运行,则自动使用其 |status.filename|。
% \item 宏包的 \cs{currfile} \cite{currfile}\ 重新定义了 \hologo{LaTeX}\ 的内部结构,以跟踪文件名(file name)。
% 如果加载了该宏包,那么它的 \cs{currfilepath}\ 将被检测到并由 \xpackage{bookmark}\ 自动使用。
% \item 可以通过 \cs{bookmarksetup}\ 或 \cs{bookmark}\ 中的 \xoption{scrfile}\ 选项手动设置(set manually)文件名。
% 但是要小心,手动设置会禁用以前的文件名检测方法。错误的(wrong)或丢失的(missed)文件名设置(file name setting)可能会在错误消息中
% 为您提供错误的源位置(source location)。
% \end{itemize}
%
% \subsubsection{级别选项(Level options)}
%
% 书签条目(bookmark entries)的顺序由 \cs{bookmark}\ 命令的的出现顺序(appearance order)定义。
% 树结构(tree structure)由书签节点(bookmark nodes)的属性 \xoption{level}(级别)构建。
% \xoption{level}\ 的值是整数(integers)。如果书签条目级别的值高于前一个节点,则该条目将成为
% 前一个节点的子(child)节点。差值的绝对值并不重要。
%
% \xpackage{bookmark}\ 宏包能记住全局属性(global property)“current level(当前级别)”中上
% 一个书签条目(previous bookmark entry)的级别。
%
% 级别系统的(level system)行为(behaviour)可以通过以下选项进行配置:
% \begin{description}
% \item[\xoption{level}:]
%    设置级别(level),请参阅上面的说明。如果给出的选项 \xoption{level}\ 没有值,那么将恢复默
%    认行为,即将“当前级别(current level)”用作级别值(level value)。自 2010/10/19 v1.16 版本以来,
%    如果宏 \cs{toclevel@part}、\cs{toclevel@section}\ 被定义过(通过 \xpackage{hyperref}\ 宏包完成,
%    请参阅它的 \xoption{bookmarkdepth}\ 选项),则 \xpackage{bookmark}\ 宏包还支持 |part|、|section| 等名称。
%
% \item[\xoption{rellevel}:]
%    设置相对于前一级别的(previous level)级别。正值表示书签条目成为前一个书签条目的子条目。
% \item[\xoption{keeplevel}:]
%    使用由\xoption{level}\ 或 \xoption{rellevel}\ 设置的级别,但不要更改全局属性“current level(当前级别)”。
%    可以通过设置为 |false| 来禁用该选项。
% \item[\xoption{startatroot}:]
%    此时,书签树(bookmark tree)再次从顶层(top level)开始。下一个书签条目不会作为上一个条目的子条目进行排序。
%    示例场景:文档使用 part。但是,最后几章(last chapters)不应放在最后一部分(last part)下面:
%    \begin{quote}
%\begin{verbatim}
%\documentclass{book}
%[...]
%\begin{document}
%  \part{第一部分}
%    \chapter{第一部分的第1章}
%    [...]
%  \part{第二部分(Second part)}
%    \chapter{第二部分的第1章}
%    [...]
%  \bookmarksetup{startatroot}
%  \chapter{Index}% 不属于第二部分
%\end{document}
%\end{verbatim}
%    \end{quote}
% \end{description}
%
% \subsubsection{样式定义(Style definitions)}
%
% 样式(style)是一组选项设置(option settings)。它可以由宏 \cs{bookmarkdefinestyle}\ 定义,
% 并由它的 \xoption{style}\ 选项使用。
% \begin{declcs}{bookmarkdefinestyle} \M{name} \M{key value list}
% \end{declcs}
% 选项设置(option settings)的 \meta{key value list}(键值列表)被指定为样式名(style \meta{name})。
%
% \begin{description}
% \item[\xoption{style}:]
%   \xoption{style}\ 选项的值是以前定义的样式的名称(name)。现在执行其选项设置(option settings)。
%   选项可以包括 \xoption{style}\ 选项。通过递归调用相同样式的无限递归(endless recursion)被阻止并抛出一个错误。
% \end{description}
%
% \subsubsection{钩子支持(Hook support)}
%
% 处理宏\cs{bookmark}\ 的可选选项(optional options)后,就会调用钩子(hook)。
% \begin{description}
% \item[\xoption{addtohook}:]
%   代码(code)作为该选项的值添加到钩子中。
% \end{description}
%
% \begin{declcs}{bookmarkget} \M{option}
% \end{declcs}
% \cs{bookmarkget}\ 宏提取 \meta{option}\ 选项的最新选项设置(latest option setting)的值。
% 对于布尔选项(boolean option),如果启用布尔选项,则返回 1,否则结果为零。结果数字(resulting numbers)
% 可以直接用于 \cs{ifnum}\ 或 \cs{ifcase}。如果您想要数字 \texttt{0}\ 和 \texttt{1},
% 请在 \cs{bookmarkget}\ 前面加上 \cs{number}\ 作为前缀。\cs{bookmarkget}\ 宏是可展开的(expandable)。
% 如果选项不受支持,则返回空字符串(empty string)。受支持的布尔选项有:
% \begin{quote}
%   \xoption{bold}、
%   \xoption{italic}、
%   \xoption{open}
% \end{quote}
% 其他受支持的选项有:
% \begin{quote}
%   \xoption{depth}、
%   \xoption{dest}、
%   \xoption{color}、
%   \xoption{gotor}、
%   \xoption{level}、
%   \xoption{named}、
%   \xoption{openlevel}、
%   \xoption{page}、
%   \xoption{rawaction}、
%   \xoption{uri}、
%   \xoption{view}、
% \end{quote}
% 另外,以下键(key)是可用的:
% \begin{quote}
%   \xoption{text}
% \end{quote}
% 它返回大纲条目(outline entry)的文本(text)。
%
% \paragraph{选项设置(Option setting)。}
% 在钩子(hook)内部可以使用 \cs{bookmarksetup}\ 设置选项。
%
% \subsection{与 \xpackage{hyperref}\ 的兼容性}
%
% \xpackage{bookmark}\ 宏包自动禁用 \xpackage{hyperref}\ 宏包的书签(bookmarks)。但是,
% \xpackage{bookmark}\ 宏包使用了 \xpackage{hyperref}\ 宏包的一些代码。例如,
% \xpackage{bookmark}\ 宏包重新定义了 \xpackage{hyperref}\ 宏包在 \cs{addcontentsline}\ 命令
% 和其他命令中插入的\cs{Hy@writebookmark}\ 钩子。因此,不应禁用 \xpackage{hyperref}\ 宏包的书签。
%
% \xpackage{bookmark}\ 宏包使用 \xpackage{hyperref}\ 宏包的 \cs{pdfstringdef},且不提供替换(replacement)。
%
% \xpackage{hyperref}\ 宏包的一些选项也能在 \xpackage{bookmark}\ 宏包中实现(implemented):
% \begin{quote}
% \begin{tabular}{|l@{}|l@{}|}
%   \hline
%   \xpackage{hyperref}\ 宏包的选项\  &\ \xpackage{bookmark}\ 宏包的选项\ \ \\ \hline
%   \xoption{bookmarksdepth} &\ \xoption{depth}\\ \hline
%   \xoption{bookmarksopen} & \ \xoption{open}\\ \hline
%   \xoption{bookmarksopenlevel}\ \ \  &\ \xoption{openlevel}\\ \hline
%   \xoption{bookmarksnumbered} \ \ \ &\ \xoption{numbered}\\ \hline
% \end{tabular}
% \end{quote}
%
% 还可以使用以下命令:
% \begin{quote}
%   \cs{pdfbookmark}\\
%   \cs{currentpdfbookmark}\\
%   \cs{subpdfbookmark}\\
%   \cs{belowpdfbookmark}
% \end{quote}
%
% \subsection{在末尾添加书签}
%
% 宏包选项 \xoption{atend}\ 启用以下宏(macro):
% \begin{declcs}{BookmarkAtEnd}
%   \M{stuff}
% \end{declcs}
% \cs{BookmarkAtEnd}\ 宏将 \meta{stuff}\ 放在文档末尾。\meta{stuff}\ 表示书签命令(bookmark commands)。举例:
% \begin{quote}
%\begin{verbatim}
%\usepackage[atend]{bookmark}
%\BookmarkAtEnd{%
%  \bookmarksetup{startatroot}%
%  \bookmark[named=LastPage, level=0]{Last page}%
%}
%\end{verbatim}
% \end{quote}
%
% 或者,可以在 \cs{bookmark}\ 中给出 \xoption{startatroot}\ 选项:
% \begin{quote}
%\begin{verbatim}
%\BookmarkAtEnd{%
%  \bookmark[
%    startatroot,
%    named=LastPage,
%    level=0,
%  ]{Last page}%
%}
%\end{verbatim}
% \end{quote}
%
% \paragraph{备注(Remarks):}
% \begin{itemize}
% \item
%   \cs{BookmarkAtEnd} 隐藏了这样一个事实,即在文档末尾添加书签的方法取决于驱动程序(driver)。
%
%   为此,驱动程序 \xoption{pdftex}\ 使用 \xpackage{atveryend}\ 宏包。如果 \cs{AtEndDocument}\ 太早,
%   最后一个页面(last page)可能不会被发送出去(shipped out)。由于需要 \xext{aux}\ 文件,此驱动程序使
%   用 \cs{AfterLastShipout}。
%
%   其他驱动程序(\xoption{dvipdfm}、\xoption{xetex}、\xoption{vtex})的实现(implementation)
%   取决于 \cs{special},\cs{special}\ 在最后一页之后没有效果。在这种情况下,\xpackage{atenddvi}\ 宏包的
%   \cs{AtEndDvi}\ 有帮助。它将其参数(argument)放在文档的最后一页(last page)。至少需要运行 \hologo{LaTeX}\ 两次,
%   因为最后一页是由引用(reference)检测到的。
%
%   \xoption{dvips}\ 现在使用新的 LaTeX 钩子 \texttt{shipout/lastpage}。
% \item
%   未指定 \cs{BookmarkAtEnd}\ 参数的扩展时间(time of expansion)。这可以立即发生,也可以在文档末尾发生。
% \end{itemize}
%
% \subsection{限制/行动计划}
%
% \begin{itemize}
% \item 支持缺失动作(missing actions)(启动,\dots)。
% \item 对 \xpackage{hyperref}\ 的 \xoption{bookmarkstype}\ 选项进行了更好的设计(design)。
% \end{itemize}
%
% \section{示例(Example)}
%
%    \begin{macrocode}
%<*example>
%    \end{macrocode}
%    \begin{macrocode}
\documentclass{article}
\usepackage{xcolor}[2007/01/21]
\usepackage{hyperref}
\usepackage[
  open,
  openlevel=2,
  atend
]{bookmark}[2019/12/03]

\bookmarksetup{color=blue}

\BookmarkAtEnd{%
  \bookmarksetup{startatroot}%
  \bookmark[named=LastPage, level=0]{End/Last page}%
  \bookmark[named=FirstPage, level=1]{First page}%
}

\begin{document}
\section{First section}
\subsection{Subsection A}
\begin{figure}
  \hypertarget{fig}{}%
  A figure.
\end{figure}
\bookmark[
  rellevel=1,
  keeplevel,
  dest=fig
]{A figure}
\subsection{Subsection B}
\subsubsection{Subsubsection C}
\subsection{Umlauts: \"A\"O\"U\"a\"o\"u\ss}
\newpage
\bookmarksetup{
  bold,
  color=[rgb]{1,0,0}
}
\section{Very important section}
\bookmarksetup{
  italic,
  bold=false,
  color=blue
}
\subsection{Italic section}
\bookmarksetup{
  italic=false
}
\part{Misc}
\section{Diverse}
\subsubsection{Subsubsection, omitting subsection}
\bookmarksetup{
  startatroot
}
\section{Last section outside part}
\subsection{Subsection}
\bookmarksetup{
  color={}
}
\begingroup
  \bookmarksetup{level=0, color=green!80!black}
  \bookmark[named=FirstPage]{First page}
  \bookmark[named=LastPage]{Last page}
  \bookmark[named=PrevPage]{Previous page}
  \bookmark[named=NextPage]{Next page}
\endgroup
\bookmark[
  page=2,
  view=FitH 800
]{Page 2, FitH 800}
\bookmark[
  page=2,
  view=FitBH \calc{\paperheight-\topmargin-1in-\headheight-\headsep}
]{Page 2, FitBH top of text body}
\bookmark[
  uri={http://www.dante.de/},
  color=magenta
]{Dante homepage}
\bookmark[
  gotor={t.pdf},
  page=1,
  view={XYZ 0 1000 null},
  color=cyan!75!black
]{File t.pdf}
\bookmark[named=FirstPage]{First page}
\bookmark[rellevel=1, named=LastPage]{Last page (rellevel=1)}
\bookmark[named=PrevPage]{Previous page}
\bookmark[level=0, named=FirstPage]{First page (level=0)}
\bookmark[
  rellevel=1,
  keeplevel,
  named=LastPage
]{Last page (rellevel=1, keeplevel)}
\bookmark[named=PrevPage]{Previous page}
\end{document}
%    \end{macrocode}
%    \begin{macrocode}
%</example>
%    \end{macrocode}
%
% \StopEventually{
% }
%
% \section{实现(Implementation)}
%
% \subsection{宏包(Package)}
%
%    \begin{macrocode}
%<*package>
\NeedsTeXFormat{LaTeX2e}
\ProvidesPackage{bookmark}%
  [2020-11-06 v1.29 PDF bookmarks (HO)]%
%    \end{macrocode}
%
% \subsubsection{要求(Requirements)}
%
% \paragraph{\hologo{eTeX}.}
%
%    \begin{macro}{\BKM@CalcExpr}
%    \begin{macrocode}
\begingroup\expandafter\expandafter\expandafter\endgroup
\expandafter\ifx\csname numexpr\endcsname\relax
  \def\BKM@CalcExpr#1#2#3#4{%
    \begingroup
      \count@=#2\relax
      \advance\count@ by#3#4\relax
      \edef\x{\endgroup
        \def\noexpand#1{\the\count@}%
      }%
    \x
  }%
\else
  \def\BKM@CalcExpr#1#2#3#4{%
    \edef#1{%
      \the\numexpr#2#3#4\relax
    }%
  }%
\fi
%    \end{macrocode}
%    \end{macro}
%
% \paragraph{\hologo{pdfTeX}\ 的转义功能(escape features)}
%
%    \begin{macro}{\BKM@EscapeName}
%    \begin{macrocode}
\def\BKM@EscapeName#1{%
  \ifx#1\@empty
  \else
    \EdefEscapeName#1#1%
  \fi
}%
%    \end{macrocode}
%    \end{macro}
%    \begin{macro}{\BKM@EscapeString}
%    \begin{macrocode}
\def\BKM@EscapeString#1{%
  \ifx#1\@empty
  \else
    \EdefEscapeString#1#1%
  \fi
}%
%    \end{macrocode}
%    \end{macro}
%    \begin{macro}{\BKM@EscapeHex}
%    \begin{macrocode}
\def\BKM@EscapeHex#1{%
  \ifx#1\@empty
  \else
    \EdefEscapeHex#1#1%
  \fi
}%
%    \end{macrocode}
%    \end{macro}
%    \begin{macro}{\BKM@UnescapeHex}
%    \begin{macrocode}
\def\BKM@UnescapeHex#1{%
  \EdefUnescapeHex#1#1%
}%
%    \end{macrocode}
%    \end{macro}
%
% \paragraph{宏包(Packages)。}
%
% 不要加载由 \xpackage{hyperref}\ 加载的宏包
%    \begin{macrocode}
\RequirePackage{hyperref}[2010/06/18]
%    \end{macrocode}
%
% \subsubsection{宏包选项(Package options)}
%
%    \begin{macrocode}
\SetupKeyvalOptions{family=BKM,prefix=BKM@}
\DeclareLocalOptions{%
  atend,%
  bold,%
  color,%
  depth,%
  dest,%
  draft,%
  final,%
  gotor,%
  italic,%
  keeplevel,%
  level,%
  named,%
  numbered,%
  open,%
  openlevel,%
  page,%
  rawaction,%
  rellevel,%
  srcfile,%
  srcline,%
  startatroot,%
  uri,%
  view,%
}
%    \end{macrocode}
%    \begin{macro}{\bookmarksetup}
%    \begin{macrocode}
\newcommand*{\bookmarksetup}{\kvsetkeys{BKM}}
%    \end{macrocode}
%    \end{macro}
%    \begin{macro}{\BKM@setup}
%    \begin{macrocode}
\def\BKM@setup#1{%
  \bookmarksetup{#1}%
  \ifx\BKM@HookNext\ltx@empty
  \else
    \expandafter\bookmarksetup\expandafter{\BKM@HookNext}%
    \BKM@HookNextClear
  \fi
  \BKM@hook
  \ifBKM@keeplevel
  \else
    \xdef\BKM@currentlevel{\BKM@level}%
  \fi
}
%    \end{macrocode}
%    \end{macro}
%
%    \begin{macro}{\bookmarksetupnext}
%    \begin{macrocode}
\newcommand*{\bookmarksetupnext}[1]{%
  \ltx@GlobalAppendToMacro\BKM@HookNext{,#1}%
}
%    \end{macrocode}
%    \end{macro}
%    \begin{macro}{\BKM@setupnext}
%    \begin{macrocode}
%    \end{macrocode}
%    \end{macro}
%    \begin{macro}{\BKM@HookNextClear}
%    \begin{macrocode}
\def\BKM@HookNextClear{%
  \global\let\BKM@HookNext\ltx@empty
}
%    \end{macrocode}
%    \end{macro}
%    \begin{macro}{\BKM@HookNext}
%    \begin{macrocode}
\BKM@HookNextClear
%    \end{macrocode}
%    \end{macro}
%
%    \begin{macrocode}
\DeclareBoolOption{draft}
\DeclareComplementaryOption{final}{draft}
%    \end{macrocode}
%    \begin{macro}{\BKM@DisableOptions}
%    \begin{macrocode}
\def\BKM@DisableOptions{%
  \DisableKeyvalOption[action=warning,package=bookmark]%
      {BKM}{draft}%
  \DisableKeyvalOption[action=warning,package=bookmark]%
      {BKM}{final}%
}
%    \end{macrocode}
%    \end{macro}
%    \begin{macrocode}
\DeclareBoolOption[\ifHy@bookmarksopen true\else false\fi]{open}
%    \end{macrocode}
%    \begin{macro}{\bookmark@open}
%    \begin{macrocode}
\def\bookmark@open{%
  \ifBKM@open\ltx@one\else\ltx@zero\fi
}
%    \end{macrocode}
%    \end{macro}
%    \begin{macrocode}
\DeclareStringOption[\maxdimen]{openlevel}
%    \end{macrocode}
%    \begin{macro}{\BKM@openlevel}
%    \begin{macrocode}
\edef\BKM@openlevel{\number\@bookmarksopenlevel}
%    \end{macrocode}
%    \end{macro}
%    \begin{macrocode}
%\DeclareStringOption[\c@tocdepth]{depth}
\ltx@IfUndefined{Hy@bookmarksdepth}{%
  \def\BKM@depth{\c@tocdepth}%
}{%
  \let\BKM@depth\Hy@bookmarksdepth
}
\define@key{BKM}{depth}[]{%
  \edef\BKM@param{#1}%
  \ifx\BKM@param\@empty
    \def\BKM@depth{\c@tocdepth}%
  \else
    \ltx@IfUndefined{toclevel@\BKM@param}{%
      \@onelevel@sanitize\BKM@param
      \edef\BKM@temp{\expandafter\@car\BKM@param\@nil}%
      \ifcase 0\expandafter\ifx\BKM@temp-1\fi
              \expandafter\ifnum\expandafter`\BKM@temp>47 %
                \expandafter\ifnum\expandafter`\BKM@temp<58 %
                  1%
                \fi
              \fi
              \relax
        \PackageWarning{bookmark}{%
          Unknown document division name (\BKM@param)\MessageBreak
          for option `depth'%
        }%
      \else
        \BKM@SetDepthOrLevel\BKM@depth\BKM@param
      \fi
    }{%
      \BKM@SetDepthOrLevel\BKM@depth{%
        \csname toclevel@\BKM@param\endcsname
      }%
    }%
  \fi
}
%    \end{macrocode}
%    \begin{macro}{\bookmark@depth}
%    \begin{macrocode}
\def\bookmark@depth{\BKM@depth}
%    \end{macrocode}
%    \end{macro}
%    \begin{macro}{\BKM@SetDepthOrLevel}
%    \begin{macrocode}
\def\BKM@SetDepthOrLevel#1#2{%
  \begingroup
    \setbox\z@=\hbox{%
      \count@=#2\relax
      \expandafter
    }%
  \expandafter\endgroup
  \expandafter\def\expandafter#1\expandafter{\the\count@}%
}
%    \end{macrocode}
%    \end{macro}
%    \begin{macrocode}
\DeclareStringOption[\BKM@currentlevel]{level}[\BKM@currentlevel]
\define@key{BKM}{level}{%
  \edef\BKM@param{#1}%
  \ifx\BKM@param\BKM@MacroCurrentLevel
    \let\BKM@level\BKM@param
  \else
    \ltx@IfUndefined{toclevel@\BKM@param}{%
      \@onelevel@sanitize\BKM@param
      \edef\BKM@temp{\expandafter\@car\BKM@param\@nil}%
      \ifcase 0\expandafter\ifx\BKM@temp-1\fi
              \expandafter\ifnum\expandafter`\BKM@temp>47 %
                \expandafter\ifnum\expandafter`\BKM@temp<58 %
                  1%
                \fi
              \fi
              \relax
        \PackageWarning{bookmark}{%
          Unknown document division name (\BKM@param)\MessageBreak
          for option `level'%
        }%
      \else
        \BKM@SetDepthOrLevel\BKM@level\BKM@param
      \fi
    }{%
      \BKM@SetDepthOrLevel\BKM@level{%
        \csname toclevel@\BKM@param\endcsname
      }%
    }%
  \fi
}
%    \end{macrocode}
%    \begin{macro}{\BKM@MacroCurrentLevel}
%    \begin{macrocode}
\def\BKM@MacroCurrentLevel{\BKM@currentlevel}
%    \end{macrocode}
%    \end{macro}
%    \begin{macrocode}
\DeclareBoolOption{keeplevel}
\DeclareBoolOption{startatroot}
%    \end{macrocode}
%    \begin{macro}{\BKM@startatrootfalse}
%    \begin{macrocode}
\def\BKM@startatrootfalse{%
  \global\let\ifBKM@startatroot\iffalse
}
%    \end{macrocode}
%    \end{macro}
%    \begin{macro}{\BKM@startatroottrue}
%    \begin{macrocode}
\def\BKM@startatroottrue{%
  \global\let\ifBKM@startatroot\iftrue
}
%    \end{macrocode}
%    \end{macro}
%    \begin{macrocode}
\define@key{BKM}{rellevel}{%
  \BKM@CalcExpr\BKM@level{#1}+\BKM@currentlevel
}
%    \end{macrocode}
%    \begin{macro}{\bookmark@level}
%    \begin{macrocode}
\def\bookmark@level{\BKM@level}
%    \end{macrocode}
%    \end{macro}
%    \begin{macro}{\BKM@currentlevel}
%    \begin{macrocode}
\def\BKM@currentlevel{0}
%    \end{macrocode}
%    \end{macro}
%    Make \xpackage{bookmark}'s option \xoption{numbered} an alias
%    for \xpackage{hyperref}'s \xoption{bookmarksnumbered}.
%    \begin{macrocode}
\DeclareBoolOption[%
  \ifHy@bookmarksnumbered true\else false\fi
]{numbered}
\g@addto@macro\BKM@numberedtrue{%
  \let\ifHy@bookmarksnumbered\iftrue
}
\g@addto@macro\BKM@numberedfalse{%
  \let\ifHy@bookmarksnumbered\iffalse
}
\g@addto@macro\Hy@bookmarksnumberedtrue{%
  \let\ifBKM@numbered\iftrue
}
\g@addto@macro\Hy@bookmarksnumberedfalse{%
  \let\ifBKM@numbered\iffalse
}
%    \end{macrocode}
%    \begin{macro}{\bookmark@numbered}
%    \begin{macrocode}
\def\bookmark@numbered{%
  \ifBKM@numbered\ltx@one\else\ltx@zero\fi
}
%    \end{macrocode}
%    \end{macro}
%
% \paragraph{重定义 \xpackage{hyperref}\ 宏包的选项}
%
%    \begin{macro}{\BKM@PatchHyperrefOption}
%    \begin{macrocode}
\def\BKM@PatchHyperrefOption#1{%
  \expandafter\BKM@@PatchHyperrefOption\csname KV@Hyp@#1\endcsname%
}
%    \end{macrocode}
%    \end{macro}
%    \begin{macro}{\BKM@@PatchHyperrefOption}
%    \begin{macrocode}
\def\BKM@@PatchHyperrefOption#1{%
  \expandafter\BKM@@@PatchHyperrefOption#1{##1}\BKM@nil#1%
}
%    \end{macrocode}
%    \end{macro}
%    \begin{macro}{\BKM@@@PatchHyperrefOption}
%    \begin{macrocode}
\def\BKM@@@PatchHyperrefOption#1\BKM@nil#2#3{%
  \def#2##1{%
    #1%
    \bookmarksetup{#3={##1}}%
  }%
}
%    \end{macrocode}
%    \end{macro}
%    \begin{macrocode}
\BKM@PatchHyperrefOption{bookmarksopen}{open}
\BKM@PatchHyperrefOption{bookmarksopenlevel}{openlevel}
\BKM@PatchHyperrefOption{bookmarksdepth}{depth}
%    \end{macrocode}
%
% \paragraph{字体样式(font style)选项。}
%
%    注意:\xpackage{bitset}\ 宏是基于零的,PDF 规范(PDF specifications)以1开头。
%    \begin{macrocode}
\bitsetReset{BKM@FontStyle}%
\define@key{BKM}{italic}[true]{%
  \expandafter\ifx\csname if#1\endcsname\iftrue
    \bitsetSet{BKM@FontStyle}{0}%
  \else
    \bitsetClear{BKM@FontStyle}{0}%
  \fi
}%
\define@key{BKM}{bold}[true]{%
  \expandafter\ifx\csname if#1\endcsname\iftrue
    \bitsetSet{BKM@FontStyle}{1}%
  \else
    \bitsetClear{BKM@FontStyle}{1}%
  \fi
}%
%    \end{macrocode}
%    \begin{macro}{\bookmark@italic}
%    \begin{macrocode}
\def\bookmark@italic{%
  \ifnum\bitsetGet{BKM@FontStyle}{0}=1 \ltx@one\else\ltx@zero\fi
}
%    \end{macrocode}
%    \end{macro}
%    \begin{macro}{\bookmark@bold}
%    \begin{macrocode}
\def\bookmark@bold{%
  \ifnum\bitsetGet{BKM@FontStyle}{1}=1 \ltx@one\else\ltx@zero\fi
}
%    \end{macrocode}
%    \end{macro}
%    \begin{macro}{\BKM@PrintStyle}
%    \begin{macrocode}
\def\BKM@PrintStyle{%
  \bitsetGetDec{BKM@FontStyle}%
}%
%    \end{macrocode}
%    \end{macro}
%
% \paragraph{颜色(color)选项。}
%
%    \begin{macrocode}
\define@key{BKM}{color}{%
  \HyColor@BookmarkColor{#1}\BKM@color{bookmark}{color}%
}
%    \end{macrocode}
%    \begin{macro}{\BKM@color}
%    \begin{macrocode}
\let\BKM@color\@empty
%    \end{macrocode}
%    \end{macro}
%    \begin{macro}{\bookmark@color}
%    \begin{macrocode}
\def\bookmark@color{\BKM@color}
%    \end{macrocode}
%    \end{macro}
%
% \subsubsection{动作(action)选项}
%
%    \begin{macrocode}
\def\BKM@temp#1{%
  \DeclareStringOption{#1}%
  \expandafter\edef\csname bookmark@#1\endcsname{%
    \expandafter\noexpand\csname BKM@#1\endcsname
  }%
}
%    \end{macrocode}
%    \begin{macro}{\bookmark@dest}
%    \begin{macrocode}
\BKM@temp{dest}
%    \end{macrocode}
%    \end{macro}
%    \begin{macro}{\bookmark@named}
%    \begin{macrocode}
\BKM@temp{named}
%    \end{macrocode}
%    \end{macro}
%    \begin{macro}{\bookmark@uri}
%    \begin{macrocode}
\BKM@temp{uri}
%    \end{macrocode}
%    \end{macro}
%    \begin{macro}{\bookmark@gotor}
%    \begin{macrocode}
\BKM@temp{gotor}
%    \end{macrocode}
%    \end{macro}
%    \begin{macro}{\bookmark@rawaction}
%    \begin{macrocode}
\BKM@temp{rawaction}
%    \end{macrocode}
%    \end{macro}
%
%    \begin{macrocode}
\define@key{BKM}{page}{%
  \def\BKM@page{#1}%
  \ifx\BKM@page\@empty
  \else
    \edef\BKM@page{\number\BKM@page}%
    \ifnum\BKM@page>\z@
    \else
      \PackageError{bookmark}{Page must be positive}\@ehc
      \def\BKM@page{1}%
    \fi
  \fi
}
%    \end{macrocode}
%    \begin{macro}{\BKM@page}
%    \begin{macrocode}
\let\BKM@page\@empty
%    \end{macrocode}
%    \end{macro}
%    \begin{macro}{\bookmark@page}
%    \begin{macrocode}
\def\bookmark@page{\BKM@@page}
%    \end{macrocode}
%    \end{macro}
%
%    \begin{macrocode}
\define@key{BKM}{view}{%
  \BKM@CheckView{#1}%
}
%    \end{macrocode}
%    \begin{macro}{\BKM@view}
%    \begin{macrocode}
\let\BKM@view\@empty
%    \end{macrocode}
%    \end{macro}
%    \begin{macro}{\bookmark@view}
%    \begin{macrocode}
\def\bookmark@view{\BKM@view}
%    \end{macrocode}
%    \end{macro}
%    \begin{macro}{BKM@CheckView}
%    \begin{macrocode}
\def\BKM@CheckView#1{%
  \BKM@CheckViewType#1 \@nil
}
%    \end{macrocode}
%    \end{macro}
%    \begin{macro}{\BKM@CheckViewType}
%    \begin{macrocode}
\def\BKM@CheckViewType#1 #2\@nil{%
  \def\BKM@type{#1}%
  \@onelevel@sanitize\BKM@type
  \BKM@TestViewType{Fit}{}%
  \BKM@TestViewType{FitB}{}%
  \BKM@TestViewType{FitH}{%
    \BKM@CheckParam#2 \@nil{top}%
  }%
  \BKM@TestViewType{FitBH}{%
    \BKM@CheckParam#2 \@nil{top}%
  }%
  \BKM@TestViewType{FitV}{%
    \BKM@CheckParam#2 \@nil{bottom}%
  }%
  \BKM@TestViewType{FitBV}{%
    \BKM@CheckParam#2 \@nil{bottom}%
  }%
  \BKM@TestViewType{FitR}{%
    \BKM@CheckRect{#2}{ }%
  }%
  \BKM@TestViewType{XYZ}{%
    \BKM@CheckXYZ{#2}{ }%
  }%
  \@car{%
    \PackageError{bookmark}{%
      Unknown view type `\BKM@type',\MessageBreak
      using `FitH' instead%
    }\@ehc
    \def\BKM@view{FitH}%
  }%
  \@nil
}
%    \end{macrocode}
%    \end{macro}
%    \begin{macro}{\BKM@TestViewType}
%    \begin{macrocode}
\def\BKM@TestViewType#1{%
  \def\BKM@temp{#1}%
  \@onelevel@sanitize\BKM@temp
  \ifx\BKM@type\BKM@temp
    \let\BKM@view\BKM@temp
    \expandafter\@car
  \else
    \expandafter\@gobble
  \fi
}
%    \end{macrocode}
%    \end{macro}
%    \begin{macro}{BKM@CheckParam}
%    \begin{macrocode}
\def\BKM@CheckParam#1 #2\@nil#3{%
  \def\BKM@param{#1}%
  \ifx\BKM@param\@empty
    \PackageWarning{bookmark}{%
      Missing parameter (#3) for `\BKM@type',\MessageBreak
      using 0%
    }%
    \def\BKM@param{0}%
  \else
    \BKM@CalcParam
  \fi
  \edef\BKM@view{\BKM@view\space\BKM@param}%
}
%    \end{macrocode}
%    \end{macro}
%    \begin{macro}{BKM@CheckRect}
%    \begin{macrocode}
\def\BKM@CheckRect#1#2{%
  \BKM@@CheckRect#1#2#2#2#2\@nil
}
%    \end{macrocode}
%    \end{macro}
%    \begin{macro}{\BKM@@CheckRect}
%    \begin{macrocode}
\def\BKM@@CheckRect#1 #2 #3 #4 #5\@nil{%
  \def\BKM@temp{0}%
  \def\BKM@param{#1}%
  \ifx\BKM@param\@empty
    \def\BKM@param{0}%
    \def\BKM@temp{1}%
  \else
    \BKM@CalcParam
  \fi
  \edef\BKM@view{\BKM@view\space\BKM@param}%
  \def\BKM@param{#2}%
  \ifx\BKM@param\@empty
    \def\BKM@param{0}%
    \def\BKM@temp{1}%
  \else
    \BKM@CalcParam
  \fi
  \edef\BKM@view{\BKM@view\space\BKM@param}%
  \def\BKM@param{#3}%
  \ifx\BKM@param\@empty
    \def\BKM@param{0}%
    \def\BKM@temp{1}%
  \else
    \BKM@CalcParam
  \fi
  \edef\BKM@view{\BKM@view\space\BKM@param}%
  \def\BKM@param{#4}%
  \ifx\BKM@param\@empty
    \def\BKM@param{0}%
    \def\BKM@temp{1}%
  \else
    \BKM@CalcParam
  \fi
  \edef\BKM@view{\BKM@view\space\BKM@param}%
  \ifnum\BKM@temp>\z@
    \PackageWarning{bookmark}{Missing parameters for `\BKM@type'}%
  \fi
}
%    \end{macrocode}
%    \end{macro}
%    \begin{macro}{\BKM@CheckXYZ}
%    \begin{macrocode}
\def\BKM@CheckXYZ#1#2{%
  \BKM@@CheckXYZ#1#2#2#2\@nil
}
%    \end{macrocode}
%    \end{macro}
%    \begin{macro}{\BKM@@CheckXYZ}
%    \begin{macrocode}
\def\BKM@@CheckXYZ#1 #2 #3 #4\@nil{%
  \def\BKM@param{#1}%
  \let\BKM@temp\BKM@param
  \@onelevel@sanitize\BKM@temp
  \ifx\BKM@param\@empty
    \let\BKM@param\BKM@null
  \else
    \ifx\BKM@temp\BKM@null
    \else
      \BKM@CalcParam
    \fi
  \fi
  \edef\BKM@view{\BKM@view\space\BKM@param}%
  \def\BKM@param{#2}%
  \let\BKM@temp\BKM@param
  \@onelevel@sanitize\BKM@temp
  \ifx\BKM@param\@empty
    \let\BKM@param\BKM@null
  \else
    \ifx\BKM@temp\BKM@null
    \else
      \BKM@CalcParam
    \fi
  \fi
  \edef\BKM@view{\BKM@view\space\BKM@param}%
  \def\BKM@param{#3}%
  \ifx\BKM@param\@empty
    \let\BKM@param\BKM@null
  \fi
  \edef\BKM@view{\BKM@view\space\BKM@param}%
}
%    \end{macrocode}
%    \end{macro}
%    \begin{macro}{\BKM@null}
%    \begin{macrocode}
\def\BKM@null{null}
\@onelevel@sanitize\BKM@null
%    \end{macrocode}
%    \end{macro}
%
%    \begin{macro}{\BKM@CalcParam}
%    \begin{macrocode}
\def\BKM@CalcParam{%
  \begingroup
  \let\calc\@firstofone
  \expandafter\BKM@@CalcParam\BKM@param\@empty\@empty\@nil
}
%    \end{macrocode}
%    \end{macro}
%    \begin{macro}{\BKM@@CalcParam}
%    \begin{macrocode}
\def\BKM@@CalcParam#1#2#3\@nil{%
  \ifx\calc#1%
    \@ifundefined{calc@assign@dimen}{%
      \@ifundefined{dimexpr}{%
        \setlength{\dimen@}{#2}%
      }{%
        \setlength{\dimen@}{\dimexpr#2\relax}%
      }%
    }{%
      \setlength{\dimen@}{#2}%
    }%
    \dimen@.99626\dimen@
    \edef\BKM@param{\strip@pt\dimen@}%
    \expandafter\endgroup
    \expandafter\def\expandafter\BKM@param\expandafter{\BKM@param}%
  \else
    \endgroup
  \fi
}
%    \end{macrocode}
%    \end{macro}
%
% \subsubsection{\xoption{atend}\ 选项}
%
%    \begin{macrocode}
\DeclareBoolOption{atend}
\g@addto@macro\BKM@DisableOptions{%
  \DisableKeyvalOption[action=warning,package=bookmark]%
      {BKM}{atend}%
}
%    \end{macrocode}
%
% \subsubsection{\xoption{style}\ 选项}
%
%    \begin{macro}{\bookmarkdefinestyle}
%    \begin{macrocode}
\newcommand*{\bookmarkdefinestyle}[2]{%
  \@ifundefined{BKM@style@#1}{%
  }{%
    \PackageInfo{bookmark}{Redefining style `#1'}%
  }%
  \@namedef{BKM@style@#1}{#2}%
}
%    \end{macrocode}
%    \end{macro}
%    \begin{macrocode}
\define@key{BKM}{style}{%
  \BKM@StyleCall{#1}%
}
\newif\ifBKM@ok
%    \end{macrocode}
%    \begin{macro}{\BKM@StyleCall}
%    \begin{macrocode}
\def\BKM@StyleCall#1{%
  \@ifundefined{BKM@style@#1}{%
    \PackageWarning{bookmark}{%
      Ignoring unknown style `#1'%
    }%
  }{%
%    \end{macrocode}
%    检查样式堆栈(style stack)。
%    \begin{macrocode}
    \BKM@oktrue
    \edef\BKM@StyleCurrent{#1}%
    \@onelevel@sanitize\BKM@StyleCurrent
    \let\BKM@StyleEntry\BKM@StyleEntryCheck
    \BKM@StyleStack
    \ifBKM@ok
      \expandafter\@firstofone
    \else
      \PackageError{bookmark}{%
        Ignoring recursive call of style `\BKM@StyleCurrent'%
      }\@ehc
      \expandafter\@gobble
    \fi
    {%
%    \end{macrocode}
%    在堆栈上推送当前样式(Push current style on stack)。
%    \begin{macrocode}
      \let\BKM@StyleEntry\relax
      \edef\BKM@StyleStack{%
        \BKM@StyleEntry{\BKM@StyleCurrent}%
        \BKM@StyleStack
      }%
%    \end{macrocode}
%   调用样式(Call style)。
%    \begin{macrocode}
      \expandafter\expandafter\expandafter\bookmarksetup
      \expandafter\expandafter\expandafter{%
        \csname BKM@style@\BKM@StyleCurrent\endcsname
      }%
%    \end{macrocode}
%    从堆栈中弹出当前样式(Pop current style from stack)。
%    \begin{macrocode}
      \BKM@StyleStackPop
    }%
  }%
}
%    \end{macrocode}
%    \end{macro}
%    \begin{macro}{\BKM@StyleStackPop}
%    \begin{macrocode}
\def\BKM@StyleStackPop{%
  \let\BKM@StyleEntry\relax
  \edef\BKM@StyleStack{%
    \expandafter\@gobbletwo\BKM@StyleStack
  }%
}
%    \end{macrocode}
%    \end{macro}
%    \begin{macro}{\BKM@StyleEntryCheck}
%    \begin{macrocode}
\def\BKM@StyleEntryCheck#1{%
  \def\BKM@temp{#1}%
  \ifx\BKM@temp\BKM@StyleCurrent
    \BKM@okfalse
  \fi
}
%    \end{macrocode}
%    \end{macro}
%    \begin{macro}{\BKM@StyleStack}
%    \begin{macrocode}
\def\BKM@StyleStack{}
%    \end{macrocode}
%    \end{macro}
%
% \subsubsection{源文件位置(source file location)选项}
%
%    \begin{macrocode}
\DeclareStringOption{srcline}
\DeclareStringOption{srcfile}
%    \end{macrocode}
%
% \subsubsection{钩子支持(Hook support)}
%
%    \begin{macro}{\BKM@hook}
%    \begin{macrocode}
\def\BKM@hook{}
%    \end{macrocode}
%    \end{macro}
%    \begin{macrocode}
\define@key{BKM}{addtohook}{%
  \ltx@LocalAppendToMacro\BKM@hook{#1}%
}
%    \end{macrocode}
%
%    \begin{macro}{bookmarkget}
%    \begin{macrocode}
\newcommand*{\bookmarkget}[1]{%
  \romannumeral0%
  \ltx@ifundefined{bookmark@#1}{%
    \ltx@space
  }{%
    \expandafter\expandafter\expandafter\ltx@space
    \csname bookmark@#1\endcsname
  }%
}
%    \end{macrocode}
%    \end{macro}
%
% \subsubsection{设置和加载驱动程序}
%
% \paragraph{检测驱动程序。}
%
%    \begin{macro}{\BKM@DefineDriverKey}
%    \begin{macrocode}
\def\BKM@DefineDriverKey#1{%
  \define@key{BKM}{#1}[]{%
    \def\BKM@driver{#1}%
  }%
  \g@addto@macro\BKM@DisableOptions{%
    \DisableKeyvalOption[action=warning,package=bookmark]%
        {BKM}{#1}%
  }%
}
%    \end{macrocode}
%    \end{macro}
%    \begin{macrocode}
\BKM@DefineDriverKey{pdftex}
\BKM@DefineDriverKey{dvips}
\BKM@DefineDriverKey{dvipdfm}
\BKM@DefineDriverKey{dvipdfmx}
\BKM@DefineDriverKey{xetex}
\BKM@DefineDriverKey{vtex}
\define@key{BKM}{dvipdfmx-outline-open}[true]{%
 \PackageWarning{bookmark}{Option 'dvipdfmx-outline-open' is obsolete
   and ignored}{}}
%    \end{macrocode}
%    \begin{macro}{\bookmark@driver}
%    \begin{macrocode}
\def\bookmark@driver{\BKM@driver}
%    \end{macrocode}
%    \end{macro}
%    \begin{macrocode}
\InputIfFileExists{bookmark.cfg}{}{}
%    \end{macrocode}
%    \begin{macro}{\BookmarkDriverDefault}
%    \begin{macrocode}
\providecommand*{\BookmarkDriverDefault}{dvips}
%    \end{macrocode}
%    \end{macro}
%    \begin{macro}{\BKM@driver}
% Lua\TeX\ 和 pdf\TeX\ 共享驱动程序。
%    \begin{macrocode}
\ifpdf
  \def\BKM@driver{pdftex}%
  \ifx\pdfoutline\@undefined
    \ifx\pdfextension\@undefined\else
      \protected\def\pdfoutline{\pdfextension outline }
    \fi
  \fi
\else
  \ifxetex
    \def\BKM@driver{dvipdfm}%
  \else
    \ifvtex
      \def\BKM@driver{vtex}%
    \else
      \edef\BKM@driver{\BookmarkDriverDefault}%
    \fi
  \fi
\fi
%    \end{macrocode}
%    \end{macro}
%
% \paragraph{过程选项(Process options)。}
%
%    \begin{macrocode}
\ProcessKeyvalOptions*
\BKM@DisableOptions
%    \end{macrocode}
%
% \paragraph{\xoption{draft}\ 选项}
%
%    \begin{macrocode}
\ifBKM@draft
  \PackageWarningNoLine{bookmark}{Draft mode on}%
  \let\bookmarksetup\ltx@gobble
  \let\BookmarkAtEnd\ltx@gobble
  \let\bookmarkdefinestyle\ltx@gobbletwo
  \let\bookmarkget\ltx@gobble
  \let\pdfbookmark\ltx@undefined
  \newcommand*{\pdfbookmark}[3][]{}%
  \let\currentpdfbookmark\ltx@gobbletwo
  \let\subpdfbookmark\ltx@gobbletwo
  \let\belowpdfbookmark\ltx@gobbletwo
  \newcommand*{\bookmark}[2][]{}%
  \renewcommand*{\Hy@writebookmark}[5]{}%
  \let\ReadBookmarks\relax
  \let\BKM@DefGotoNameAction\ltx@gobbletwo % package `hypdestopt'
  \expandafter\endinput
\fi
%    \end{macrocode}
%
% \paragraph{验证和加载驱动程序。}
%
%    \begin{macrocode}
\def\BKM@temp{dvipdfmx}%
\ifx\BKM@temp\BKM@driver
  \def\BKM@driver{dvipdfm}%
\fi
\def\BKM@temp{pdftex}%
\ifpdf
  \ifx\BKM@temp\BKM@driver
  \else
    \PackageWarningNoLine{bookmark}{%
      Wrong driver `\BKM@driver', using `pdftex' instead%
    }%
    \let\BKM@driver\BKM@temp
  \fi
\else
  \ifx\BKM@temp\BKM@driver
    \PackageError{bookmark}{%
      Wrong driver, pdfTeX is not running in PDF mode.\MessageBreak
      Package loading is aborted%
    }\@ehc
    \expandafter\expandafter\expandafter\endinput
  \fi
  \def\BKM@temp{dvipdfm}%
  \ifxetex
    \ifx\BKM@temp\BKM@driver
    \else
      \PackageWarningNoLine{bookmark}{%
        Wrong driver `\BKM@driver',\MessageBreak
        using `dvipdfm' for XeTeX instead%
      }%
      \let\BKM@driver\BKM@temp
    \fi
  \else
    \def\BKM@temp{vtex}%
    \ifvtex
      \ifx\BKM@temp\BKM@driver
      \else
        \PackageWarningNoLine{bookmark}{%
          Wrong driver `\BKM@driver',\MessageBreak
          using `vtex' for VTeX instead%
        }%
        \let\BKM@driver\BKM@temp
      \fi
    \else
      \ifx\BKM@temp\BKM@driver
        \PackageError{bookmark}{%
          Wrong driver, VTeX is not running in PDF mode.\MessageBreak
          Package loading is aborted%
        }\@ehc
        \expandafter\expandafter\expandafter\endinput
      \fi
    \fi
  \fi
\fi
\providecommand\IfFormatAtLeastTF{\@ifl@t@r\fmtversion}
\IfFormatAtLeastTF{2020/10/01}{}{\edef\BKM@driver{\BKM@driver-2019-12-03}}
\InputIfFileExists{bkm-\BKM@driver.def}{}{%
  \PackageError{bookmark}{%
    Unsupported driver `\BKM@driver'.\MessageBreak
    Package loading is aborted%
  }\@ehc
  \endinput
}
%    \end{macrocode}
%
% \subsubsection{与 \xpackage{hyperref}\ 的兼容性}
%
%    \begin{macro}{\pdfbookmark}
%    \begin{macrocode}
\let\pdfbookmark\ltx@undefined
\newcommand*{\pdfbookmark}[3][0]{%
  \bookmark[level=#1,dest={#3.#1}]{#2}%
  \hyper@anchorstart{#3.#1}\hyper@anchorend
}
%    \end{macrocode}
%    \end{macro}
%    \begin{macro}{\currentpdfbookmark}
%    \begin{macrocode}
\def\currentpdfbookmark{%
  \pdfbookmark[\BKM@currentlevel]%
}
%    \end{macrocode}
%    \end{macro}
%    \begin{macro}{\subpdfbookmark}
%    \begin{macrocode}
\def\subpdfbookmark{%
  \BKM@CalcExpr\BKM@CalcResult\BKM@currentlevel+1%
  \expandafter\pdfbookmark\expandafter[\BKM@CalcResult]%
}
%    \end{macrocode}
%    \end{macro}
%    \begin{macro}{\belowpdfbookmark}
%    \begin{macrocode}
\def\belowpdfbookmark#1#2{%
  \xdef\BKM@gtemp{\number\BKM@currentlevel}%
  \subpdfbookmark{#1}{#2}%
  \global\let\BKM@currentlevel\BKM@gtemp
}
%    \end{macrocode}
%    \end{macro}
%
%    节号(section number)、文本(text)、标签(label)、级别(level)、文件(file)
%    \begin{macro}{\Hy@writebookmark}
%    \begin{macrocode}
\def\Hy@writebookmark#1#2#3#4#5{%
  \ifnum#4>\BKM@depth\relax
  \else
    \def\BKM@type{#5}%
    \ifx\BKM@type\Hy@bookmarkstype
      \begingroup
        \ifBKM@numbered
          \let\numberline\Hy@numberline
          \let\booknumberline\Hy@numberline
          \let\partnumberline\Hy@numberline
          \let\chapternumberline\Hy@numberline
        \else
          \let\numberline\@gobble
          \let\booknumberline\@gobble
          \let\partnumberline\@gobble
          \let\chapternumberline\@gobble
        \fi
        \bookmark[level=#4,dest={\HyperDestNameFilter{#3}}]{#2}%
      \endgroup
    \fi
  \fi
}
%    \end{macrocode}
%    \end{macro}
%
%    \begin{macro}{\ReadBookmarks}
%    \begin{macrocode}
\let\ReadBookmarks\relax
%    \end{macrocode}
%    \end{macro}
%
%    \begin{macrocode}
%</package>
%    \end{macrocode}
%
% \subsection{dvipdfm 的驱动程序}
%
%    \begin{macrocode}
%<*dvipdfm>
\NeedsTeXFormat{LaTeX2e}
\ProvidesFile{bkm-dvipdfm.def}%
  [2020-11-06 v1.29 bookmark driver for dvipdfm (HO)]%
%    \end{macrocode}
%
%    \begin{macro}{\BKM@id}
%    \begin{macrocode}
\newcount\BKM@id
\BKM@id=\z@
%    \end{macrocode}
%    \end{macro}
%
%    \begin{macro}{\BKM@0}
%    \begin{macrocode}
\@namedef{BKM@0}{000}
%    \end{macrocode}
%    \end{macro}
%    \begin{macro}{\ifBKM@sw}
%    \begin{macrocode}
\newif\ifBKM@sw
%    \end{macrocode}
%    \end{macro}
%
%    \begin{macro}{\bookmark}
%    \begin{macrocode}
\newcommand*{\bookmark}[2][]{%
  \if@filesw
    \begingroup
      \def\bookmark@text{#2}%
      \BKM@setup{#1}%
      \edef\BKM@prev{\the\BKM@id}%
      \global\advance\BKM@id\@ne
      \BKM@swtrue
      \@whilesw\ifBKM@sw\fi{%
        \def\BKM@abslevel{1}%
        \ifnum\ifBKM@startatroot\z@\else\BKM@prev\fi=\z@
          \BKM@startatrootfalse
          \expandafter\xdef\csname BKM@\the\BKM@id\endcsname{%
            0{\BKM@level}\BKM@abslevel
          }%
          \BKM@swfalse
        \else
          \expandafter\expandafter\expandafter\BKM@getx
              \csname BKM@\BKM@prev\endcsname
          \ifnum\BKM@level>\BKM@x@level\relax
            \BKM@CalcExpr\BKM@abslevel\BKM@x@abslevel+1%
            \expandafter\xdef\csname BKM@\the\BKM@id\endcsname{%
              {\BKM@prev}{\BKM@level}\BKM@abslevel
            }%
            \BKM@swfalse
          \else
            \let\BKM@prev\BKM@x@parent
          \fi
        \fi
      }%
      \csname HyPsd@XeTeXBigCharstrue\endcsname
      \pdfstringdef\BKM@title{\bookmark@text}%
      \edef\BKM@FLAGS{\BKM@PrintStyle}%
      \let\BKM@action\@empty
      \ifx\BKM@gotor\@empty
        \ifx\BKM@dest\@empty
          \ifx\BKM@named\@empty
            \ifx\BKM@rawaction\@empty
              \ifx\BKM@uri\@empty
                \ifx\BKM@page\@empty
                  \PackageError{bookmark}{Missing action}\@ehc
                  \edef\BKM@action{/Dest[@page1/Fit]}%
                \else
                  \ifx\BKM@view\@empty
                    \def\BKM@view{Fit}%
                  \fi
                  \edef\BKM@action{/Dest[@page\BKM@page/\BKM@view]}%
                \fi
              \else
                \BKM@EscapeString\BKM@uri
                \edef\BKM@action{%
                  /A<<%
                    /S/URI%
                    /URI(\BKM@uri)%
                  >>%
                }%
              \fi
            \else
              \edef\BKM@action{/A<<\BKM@rawaction>>}%
            \fi
          \else
            \BKM@EscapeName\BKM@named
            \edef\BKM@action{%
              /A<</S/Named/N/\BKM@named>>%
            }%
          \fi
        \else
          \BKM@EscapeString\BKM@dest
          \edef\BKM@action{%
            /A<<%
              /S/GoTo%
              /D(\BKM@dest)%
            >>%
          }%
        \fi
      \else
        \ifx\BKM@dest\@empty
          \ifx\BKM@page\@empty
            \def\BKM@page{0}%
          \else
            \BKM@CalcExpr\BKM@page\BKM@page-1%
          \fi
          \ifx\BKM@view\@empty
            \def\BKM@view{Fit}%
          \fi
          \edef\BKM@action{/D[\BKM@page/\BKM@view]}%
        \else
          \BKM@EscapeString\BKM@dest
          \edef\BKM@action{/D(\BKM@dest)}%
        \fi
        \BKM@EscapeString\BKM@gotor
        \edef\BKM@action{%
          /A<<%
            /S/GoToR%
            /F(\BKM@gotor)%
            \BKM@action
          >>%
        }%
      \fi
      \special{pdf:%
        out
              [%
              \ifBKM@open
                \ifnum\BKM@level<%
                    \expandafter\ltx@firstofone\expandafter
                    {\number\BKM@openlevel} %
                \else
                  -%
                \fi
              \else
                -%
              \fi
              ] %
            \BKM@abslevel
        <<%
          /Title(\BKM@title)%
          \ifx\BKM@color\@empty
          \else
            /C[\BKM@color]%
          \fi
          \ifnum\BKM@FLAGS>\z@
            /F \BKM@FLAGS
          \fi
          \BKM@action
        >>%
      }%
    \endgroup
  \fi
}
%    \end{macrocode}
%    \end{macro}
%    \begin{macro}{\BKM@getx}
%    \begin{macrocode}
\def\BKM@getx#1#2#3{%
  \def\BKM@x@parent{#1}%
  \def\BKM@x@level{#2}%
  \def\BKM@x@abslevel{#3}%
}
%    \end{macrocode}
%    \end{macro}
%
%    \begin{macrocode}
%</dvipdfm>
%    \end{macrocode}
%
% \subsection{\hologo{VTeX}\ 的驱动程序}
%
%    \begin{macrocode}
%<*vtex>
\NeedsTeXFormat{LaTeX2e}
\ProvidesFile{bkm-vtex.def}%
  [2020-11-06 v1.29 bookmark driver for VTeX (HO)]%
%    \end{macrocode}
%
%    \begin{macrocode}
\ifvtexpdf
\else
  \PackageWarningNoLine{bookmark}{%
    The VTeX driver only supports PDF mode%
  }%
\fi
%    \end{macrocode}
%
%    \begin{macro}{\BKM@id}
%    \begin{macrocode}
\newcount\BKM@id
\BKM@id=\z@
%    \end{macrocode}
%    \end{macro}
%
%    \begin{macro}{\BKM@0}
%    \begin{macrocode}
\@namedef{BKM@0}{00}
%    \end{macrocode}
%    \end{macro}
%    \begin{macro}{\ifBKM@sw}
%    \begin{macrocode}
\newif\ifBKM@sw
%    \end{macrocode}
%    \end{macro}
%
%    \begin{macro}{\bookmark}
%    \begin{macrocode}
\newcommand*{\bookmark}[2][]{%
  \if@filesw
    \begingroup
      \def\bookmark@text{#2}%
      \BKM@setup{#1}%
      \edef\BKM@prev{\the\BKM@id}%
      \global\advance\BKM@id\@ne
      \BKM@swtrue
      \@whilesw\ifBKM@sw\fi{%
        \ifnum\ifBKM@startatroot\z@\else\BKM@prev\fi=\z@
          \BKM@startatrootfalse
          \def\BKM@parent{0}%
          \expandafter\xdef\csname BKM@\the\BKM@id\endcsname{%
            0{\BKM@level}%
          }%
          \BKM@swfalse
        \else
          \expandafter\expandafter\expandafter\BKM@getx
              \csname BKM@\BKM@prev\endcsname
          \ifnum\BKM@level>\BKM@x@level\relax
            \let\BKM@parent\BKM@prev
            \expandafter\xdef\csname BKM@\the\BKM@id\endcsname{%
              {\BKM@prev}{\BKM@level}%
            }%
            \BKM@swfalse
          \else
            \let\BKM@prev\BKM@x@parent
          \fi
        \fi
      }%
      \pdfstringdef\BKM@title{\bookmark@text}%
      \BKM@vtex@title
      \edef\BKM@FLAGS{\BKM@PrintStyle}%
      \let\BKM@action\@empty
      \ifx\BKM@gotor\@empty
        \ifx\BKM@dest\@empty
          \ifx\BKM@named\@empty
            \ifx\BKM@rawaction\@empty
              \ifx\BKM@uri\@empty
                \ifx\BKM@page\@empty
                  \PackageError{bookmark}{Missing action}\@ehc
                  \def\BKM@action{!1}%
                \else
                  \edef\BKM@action{!\BKM@page}%
                \fi
              \else
                \BKM@EscapeString\BKM@uri
                \edef\BKM@action{%
                  <u=%
                    /S/URI%
                    /URI(\BKM@uri)%
                  >%
                }%
              \fi
            \else
              \edef\BKM@action{<u=\BKM@rawaction>}%
            \fi
          \else
            \BKM@EscapeName\BKM@named
            \edef\BKM@action{%
              <u=%
                /S/Named%
                /N/\BKM@named
              >%
            }%
          \fi
        \else
          \BKM@EscapeString\BKM@dest
          \edef\BKM@action{\BKM@dest}%
        \fi
      \else
        \ifx\BKM@dest\@empty
          \ifx\BKM@page\@empty
            \def\BKM@page{1}%
          \fi
          \ifx\BKM@view\@empty
            \def\BKM@view{Fit}%
          \fi
          \edef\BKM@action{/D[\BKM@page/\BKM@view]}%
        \else
          \BKM@EscapeString\BKM@dest
          \edef\BKM@action{/D(\BKM@dest)}%
        \fi
        \BKM@EscapeString\BKM@gotor
        \edef\BKM@action{%
          <u=%
            /S/GoToR%
            /F(\BKM@gotor)%
            \BKM@action
          >>%
        }%
      \fi
      \ifx\BKM@color\@empty
        \let\BKM@RGBcolor\@empty
      \else
        \expandafter\BKM@toRGB\BKM@color\@nil
      \fi
      \special{%
        !outline \BKM@action;%
        p=\BKM@parent,%
        i=\number\BKM@id,%
        s=%
          \ifBKM@open
            \ifnum\BKM@level<\BKM@openlevel
              o%
            \else
              c%
            \fi
          \else
            c%
          \fi,%
        \ifx\BKM@RGBcolor\@empty
        \else
          c=\BKM@RGBcolor,%
        \fi
        \ifnum\BKM@FLAGS>\z@
          f=\BKM@FLAGS,%
        \fi
        t=\BKM@title
      }%
    \endgroup
  \fi
}
%    \end{macrocode}
%    \end{macro}
%    \begin{macro}{\BKM@getx}
%    \begin{macrocode}
\def\BKM@getx#1#2{%
  \def\BKM@x@parent{#1}%
  \def\BKM@x@level{#2}%
}
%    \end{macrocode}
%    \end{macro}
%    \begin{macro}{\BKM@toRGB}
%    \begin{macrocode}
\def\BKM@toRGB#1 #2 #3\@nil{%
  \let\BKM@RGBcolor\@empty
  \BKM@toRGBComponent{#1}%
  \BKM@toRGBComponent{#2}%
  \BKM@toRGBComponent{#3}%
}
%    \end{macrocode}
%    \end{macro}
%    \begin{macro}{\BKM@toRGBComponent}
%    \begin{macrocode}
\def\BKM@toRGBComponent#1{%
  \dimen@=#1pt\relax
  \ifdim\dimen@>\z@
    \ifdim\dimen@<\p@
      \dimen@=255\dimen@
      \advance\dimen@ by 32768sp\relax
      \divide\dimen@ by 65536\relax
      \dimen@ii=\dimen@
      \divide\dimen@ii by 16\relax
      \edef\BKM@RGBcolor{%
        \BKM@RGBcolor
        \BKM@toHexDigit\dimen@ii
      }%
      \dimen@ii=16\dimen@ii
      \advance\dimen@-\dimen@ii
      \edef\BKM@RGBcolor{%
        \BKM@RGBcolor
        \BKM@toHexDigit\dimen@
      }%
    \else
      \edef\BKM@RGBcolor{\BKM@RGBcolor FF}%
    \fi
  \else
    \edef\BKM@RGBcolor{\BKM@RGBcolor00}%
  \fi
}
%    \end{macrocode}
%    \end{macro}
%    \begin{macro}{\BKM@toHexDigit}
%    \begin{macrocode}
\def\BKM@toHexDigit#1{%
  \ifcase\expandafter\@firstofone\expandafter{\number#1} %
    0\or 1\or 2\or 3\or 4\or 5\or 6\or 7\or
    8\or 9\or A\or B\or C\or D\or E\or F%
  \fi
}
%    \end{macrocode}
%    \end{macro}
%    \begin{macrocode}
\begingroup
  \catcode`\|=0 %
  \catcode`\\=12 %
%    \end{macrocode}
%    \begin{macro}{\BKM@vtex@title}
%    \begin{macrocode}
  |gdef|BKM@vtex@title{%
    |@onelevel@sanitize|BKM@title
    |edef|BKM@title{|expandafter|BKM@vtex@leftparen|BKM@title\(|@nil}%
    |edef|BKM@title{|expandafter|BKM@vtex@rightparen|BKM@title\)|@nil}%
    |edef|BKM@title{|expandafter|BKM@vtex@zero|BKM@title\0|@nil}%
    |edef|BKM@title{|expandafter|BKM@vtex@one|BKM@title\1|@nil}%
    |edef|BKM@title{|expandafter|BKM@vtex@two|BKM@title\2|@nil}%
    |edef|BKM@title{|expandafter|BKM@vtex@three|BKM@title\3|@nil}%
  }%
%    \end{macrocode}
%    \end{macro}
%    \begin{macro}{\BKM@vtex@leftparen}
%    \begin{macrocode}
  |gdef|BKM@vtex@leftparen#1\(#2|@nil{%
    #1%
    |ifx||#2||%
    |else
      (%
      |ltx@ReturnAfterFi{%
        |BKM@vtex@leftparen#2|@nil
      }%
    |fi
  }%
%    \end{macrocode}
%    \end{macro}
%    \begin{macro}{\BKM@vtex@rightparen}
%    \begin{macrocode}
  |gdef|BKM@vtex@rightparen#1\)#2|@nil{%
    #1%
    |ifx||#2||%
    |else
      )%
      |ltx@ReturnAfterFi{%
        |BKM@vtex@rightparen#2|@nil
      }%
    |fi
  }%
%    \end{macrocode}
%    \end{macro}
%    \begin{macro}{\BKM@vtex@zero}
%    \begin{macrocode}
  |gdef|BKM@vtex@zero#1\0#2|@nil{%
    #1%
    |ifx||#2||%
    |else
      |noexpand|hv@pdf@char0%
      |ltx@ReturnAfterFi{%
        |BKM@vtex@zero#2|@nil
      }%
    |fi
  }%
%    \end{macrocode}
%    \end{macro}
%    \begin{macro}{\BKM@vtex@one}
%    \begin{macrocode}
  |gdef|BKM@vtex@one#1\1#2|@nil{%
    #1%
    |ifx||#2||%
    |else
      |noexpand|hv@pdf@char1%
      |ltx@ReturnAfterFi{%
        |BKM@vtex@one#2|@nil
      }%
    |fi
  }%
%    \end{macrocode}
%    \end{macro}
%    \begin{macro}{\BKM@vtex@two}
%    \begin{macrocode}
  |gdef|BKM@vtex@two#1\2#2|@nil{%
    #1%
    |ifx||#2||%
    |else
      |noexpand|hv@pdf@char2%
      |ltx@ReturnAfterFi{%
        |BKM@vtex@two#2|@nil
      }%
    |fi
  }%
%    \end{macrocode}
%    \end{macro}
%    \begin{macro}{\BKM@vtex@three}
%    \begin{macrocode}
  |gdef|BKM@vtex@three#1\3#2|@nil{%
    #1%
    |ifx||#2||%
    |else
      |noexpand|hv@pdf@char3%
      |ltx@ReturnAfterFi{%
        |BKM@vtex@three#2|@nil
      }%
    |fi
  }%
%    \end{macrocode}
%    \end{macro}
%    \begin{macrocode}
|endgroup
%    \end{macrocode}
%
%    \begin{macrocode}
%</vtex>
%    \end{macrocode}
%
% \subsection{\hologo{pdfTeX}\ 的驱动程序}
%
%    \begin{macrocode}
%<*pdftex>
\NeedsTeXFormat{LaTeX2e}
\ProvidesFile{bkm-pdftex.def}%
  [2020-11-06 v1.29 bookmark driver for pdfTeX (HO)]%
%    \end{macrocode}
%
%    \begin{macro}{\BKM@DO@entry}
%    \begin{macrocode}
\def\BKM@DO@entry#1#2{%
  \begingroup
    \kvsetkeys{BKM@DO}{#1}%
    \def\BKM@DO@title{#2}%
    \ifx\BKM@DO@srcfile\@empty
    \else
      \BKM@UnescapeHex\BKM@DO@srcfile
    \fi
    \BKM@UnescapeHex\BKM@DO@title
    \expandafter\expandafter\expandafter\BKM@getx
        \csname BKM@\BKM@DO@id\endcsname\@empty\@empty
    \let\BKM@attr\@empty
    \ifx\BKM@DO@flags\@empty
    \else
      \edef\BKM@attr{\BKM@attr/F \BKM@DO@flags}%
    \fi
    \ifx\BKM@DO@color\@empty
    \else
      \edef\BKM@attr{\BKM@attr/C[\BKM@DO@color]}%
    \fi
    \ifx\BKM@attr\@empty
    \else
      \edef\BKM@attr{attr{\BKM@attr}}%
    \fi
    \let\BKM@action\@empty
    \ifx\BKM@DO@gotor\@empty
      \ifx\BKM@DO@dest\@empty
        \ifx\BKM@DO@named\@empty
          \ifx\BKM@DO@rawaction\@empty
            \ifx\BKM@DO@uri\@empty
              \ifx\BKM@DO@page\@empty
                \PackageError{bookmark}{%
                  Missing action\BKM@SourceLocation
                }\@ehc
                \edef\BKM@action{goto page1{/Fit}}%
              \else
                \ifx\BKM@DO@view\@empty
                  \def\BKM@DO@view{Fit}%
                \fi
                \edef\BKM@action{goto page\BKM@DO@page{/\BKM@DO@view}}%
              \fi
            \else
              \BKM@UnescapeHex\BKM@DO@uri
              \BKM@EscapeString\BKM@DO@uri
              \edef\BKM@action{user{<</S/URI/URI(\BKM@DO@uri)>>}}%
            \fi
          \else
            \BKM@UnescapeHex\BKM@DO@rawaction
            \edef\BKM@action{%
              user{%
                <<%
                  \BKM@DO@rawaction
                >>%
              }%
            }%
          \fi
        \else
          \BKM@EscapeName\BKM@DO@named
          \edef\BKM@action{%
            user{<</S/Named/N/\BKM@DO@named>>}%
          }%
        \fi
      \else
        \BKM@UnescapeHex\BKM@DO@dest
        \BKM@DefGotoNameAction\BKM@action\BKM@DO@dest
      \fi
    \else
      \ifx\BKM@DO@dest\@empty
        \ifx\BKM@DO@page\@empty
          \def\BKM@DO@page{0}%
        \else
          \BKM@CalcExpr\BKM@DO@page\BKM@DO@page-1%
        \fi
        \ifx\BKM@DO@view\@empty
          \def\BKM@DO@view{Fit}%
        \fi
        \edef\BKM@action{/D[\BKM@DO@page/\BKM@DO@view]}%
      \else
        \BKM@UnescapeHex\BKM@DO@dest
        \BKM@EscapeString\BKM@DO@dest
        \edef\BKM@action{/D(\BKM@DO@dest)}%
      \fi
      \BKM@UnescapeHex\BKM@DO@gotor
      \BKM@EscapeString\BKM@DO@gotor
      \edef\BKM@action{%
        user{%
          <<%
            /S/GoToR%
            /F(\BKM@DO@gotor)%
            \BKM@action
          >>%
        }%
      }%
    \fi
    \pdfoutline\BKM@attr\BKM@action
                count\ifBKM@DO@open\else-\fi\BKM@x@childs
                {\BKM@DO@title}%
  \endgroup
}
%    \end{macrocode}
%    \end{macro}
%    \begin{macro}{\BKM@DefGotoNameAction}
%    \cs{BKM@DefGotoNameAction}\ 宏是一个用于 \xpackage{hypdestopt}\ 宏包的钩子(hook)。
%    \begin{macrocode}
\def\BKM@DefGotoNameAction#1#2{%
  \BKM@EscapeString\BKM@DO@dest
  \edef#1{goto name{#2}}%
}
%    \end{macrocode}
%    \end{macro}
%    \begin{macrocode}
%</pdftex>
%    \end{macrocode}
%
%    \begin{macrocode}
%<*pdftex|pdfmark>
%    \end{macrocode}
%    \begin{macro}{\BKM@SourceLocation}
%    \begin{macrocode}
\def\BKM@SourceLocation{%
  \ifx\BKM@DO@srcfile\@empty
    \ifx\BKM@DO@srcline\@empty
    \else
      .\MessageBreak
      Source: line \BKM@DO@srcline
    \fi
  \else
    \ifx\BKM@DO@srcline\@empty
      .\MessageBreak
      Source: file `\BKM@DO@srcfile'%
    \else
      .\MessageBreak
      Source: file `\BKM@DO@srcfile', line \BKM@DO@srcline
    \fi
  \fi
}
%    \end{macrocode}
%    \end{macro}
%    \begin{macrocode}
%</pdftex|pdfmark>
%    \end{macrocode}
%
% \subsection{具有 pdfmark 特色(specials)的驱动程序}
%
% \subsubsection{dvips 驱动程序}
%
%    \begin{macrocode}
%<*dvips>
\NeedsTeXFormat{LaTeX2e}
\ProvidesFile{bkm-dvips.def}%
  [2020-11-06 v1.29 bookmark driver for dvips (HO)]%
%    \end{macrocode}
%    \begin{macro}{\BKM@PSHeaderFile}
%    \begin{macrocode}
\def\BKM@PSHeaderFile#1{%
  \special{PSfile=#1}%
}
%    \end{macrocode}
%    \begin{macro}{\BKM@filename}
%    \begin{macrocode}
\def\BKM@filename{\jobname.out.ps}
%    \end{macrocode}
%    \end{macro}
%    \begin{macrocode}
\AddToHook{shipout/lastpage}{%
  \BKM@pdfmark@out
  \BKM@PSHeaderFile\BKM@filename
  }
%    \end{macrocode}
%    \end{macro}
%    \begin{macrocode}
%</dvips>
%    \end{macrocode}
%
% \subsubsection{公共部分(Common part)}
%
%    \begin{macrocode}
%<*pdfmark>
%    \end{macrocode}
%
%    \begin{macro}{\BKM@pdfmark@out}
%    不要在这里使用 \xpackage{rerunfilecheck}\ 宏包,因为在 \hologo{TeX}\ 运行期间不会
%    读取 \cs{BKM@filename}\ 文件。
%    \begin{macrocode}
\def\BKM@pdfmark@out{%
  \if@filesw
    \newwrite\BKM@file
    \immediate\openout\BKM@file=\BKM@filename\relax
    \BKM@write{\@percentchar!}%
    \BKM@write{/pdfmark where{pop}}%
    \BKM@write{%
      {%
        /globaldict where{pop globaldict}{userdict}ifelse%
        /pdfmark/cleartomark load put%
      }%
    }%
    \BKM@write{ifelse}%
  \else
    \let\BKM@write\@gobble
    \let\BKM@DO@entry\@gobbletwo
  \fi
}
%    \end{macrocode}
%    \end{macro}
%    \begin{macro}{\BKM@write}
%    \begin{macrocode}
\def\BKM@write#{%
  \immediate\write\BKM@file
}
%    \end{macrocode}
%    \end{macro}
%
%    \begin{macro}{\BKM@DO@entry}
%    Pdfmark 的规范(specification)说明 |/Color| 是颜色(color)的键名(key name),
%    但是 ghostscript 只将键(key)传递到 PDF 文件中,因此键名必须是 |/C|。
%    \begin{macrocode}
\def\BKM@DO@entry#1#2{%
  \begingroup
    \kvsetkeys{BKM@DO}{#1}%
    \ifx\BKM@DO@srcfile\@empty
    \else
      \BKM@UnescapeHex\BKM@DO@srcfile
    \fi
    \def\BKM@DO@title{#2}%
    \BKM@UnescapeHex\BKM@DO@title
    \expandafter\expandafter\expandafter\BKM@getx
        \csname BKM@\BKM@DO@id\endcsname\@empty\@empty
    \let\BKM@attr\@empty
    \ifx\BKM@DO@flags\@empty
    \else
      \edef\BKM@attr{\BKM@attr/F \BKM@DO@flags}%
    \fi
    \ifx\BKM@DO@color\@empty
    \else
      \edef\BKM@attr{\BKM@attr/C[\BKM@DO@color]}%
    \fi
    \let\BKM@action\@empty
    \ifx\BKM@DO@gotor\@empty
      \ifx\BKM@DO@dest\@empty
        \ifx\BKM@DO@named\@empty
          \ifx\BKM@DO@rawaction\@empty
            \ifx\BKM@DO@uri\@empty
              \ifx\BKM@DO@page\@empty
                \PackageError{bookmark}{%
                  Missing action\BKM@SourceLocation
                }\@ehc
                \edef\BKM@action{%
                  /Action/GoTo%
                  /Page 1%
                  /View[/Fit]%
                }%
              \else
                \ifx\BKM@DO@view\@empty
                  \def\BKM@DO@view{Fit}%
                \fi
                \edef\BKM@action{%
                  /Action/GoTo%
                  /Page \BKM@DO@page
                  /View[/\BKM@DO@view]%
                }%
              \fi
            \else
              \BKM@UnescapeHex\BKM@DO@uri
              \BKM@EscapeString\BKM@DO@uri
              \edef\BKM@action{%
                /Action<<%
                  /Subtype/URI%
                  /URI(\BKM@DO@uri)%
                >>%
              }%
            \fi
          \else
            \BKM@UnescapeHex\BKM@DO@rawaction
            \edef\BKM@action{%
              /Action<<%
                \BKM@DO@rawaction
              >>%
            }%
          \fi
        \else
          \BKM@EscapeName\BKM@DO@named
          \edef\BKM@action{%
            /Action<<%
              /Subtype/Named%
              /N/\BKM@DO@named
            >>%
          }%
        \fi
      \else
        \BKM@UnescapeHex\BKM@DO@dest
        \BKM@EscapeString\BKM@DO@dest
        \edef\BKM@action{%
          /Action/GoTo%
          /Dest(\BKM@DO@dest)cvn%
        }%
      \fi
    \else
      \ifx\BKM@DO@dest\@empty
        \ifx\BKM@DO@page\@empty
          \def\BKM@DO@page{1}%
        \fi
        \ifx\BKM@DO@view\@empty
          \def\BKM@DO@view{Fit}%
        \fi
        \edef\BKM@action{%
          /Page \BKM@DO@page
          /View[/\BKM@DO@view]%
        }%
      \else
        \BKM@UnescapeHex\BKM@DO@dest
        \BKM@EscapeString\BKM@DO@dest
        \edef\BKM@action{%
          /Dest(\BKM@DO@dest)cvn%
        }%
      \fi
      \BKM@UnescapeHex\BKM@DO@gotor
      \BKM@EscapeString\BKM@DO@gotor
      \edef\BKM@action{%
        /Action/GoToR%
        /File(\BKM@DO@gotor)%
        \BKM@action
      }%
    \fi
    \BKM@write{[}%
    \BKM@write{/Title(\BKM@DO@title)}%
    \ifnum\BKM@x@childs>\z@
      \BKM@write{/Count \ifBKM@DO@open\else-\fi\BKM@x@childs}%
    \fi
    \ifx\BKM@attr\@empty
    \else
      \BKM@write{\BKM@attr}%
    \fi
    \BKM@write{\BKM@action}%
    \BKM@write{/OUT pdfmark}%
  \endgroup
}
%    \end{macrocode}
%    \end{macro}
%    \begin{macrocode}
%</pdfmark>
%    \end{macrocode}
%
% \subsection{\xoption{pdftex}\ 和 \xoption{pdfmark}\ 的公共部分}
%
%    \begin{macrocode}
%<*pdftex|pdfmark>
%    \end{macrocode}
%
% \subsubsection{写入辅助文件(auxiliary file)}
%
%    \begin{macrocode}
\AddToHook{begindocument}{%
 \immediate\write\@mainaux{\string\providecommand\string\BKM@entry[2]{}}}
%    \end{macrocode}
%
%    \begin{macro}{\BKM@id}
%    \begin{macrocode}
\newcount\BKM@id
\BKM@id=\z@
%    \end{macrocode}
%    \end{macro}
%
%    \begin{macro}{\BKM@0}
%    \begin{macrocode}
\@namedef{BKM@0}{000}
%    \end{macrocode}
%    \end{macro}
%    \begin{macro}{\ifBKM@sw}
%    \begin{macrocode}
\newif\ifBKM@sw
%    \end{macrocode}
%    \end{macro}
%
%    \begin{macro}{\bookmark}
%    \begin{macrocode}
\newcommand*{\bookmark}[2][]{%
  \if@filesw
    \begingroup
      \BKM@InitSourceLocation
      \def\bookmark@text{#2}%
      \BKM@setup{#1}%
      \ifx\BKM@srcfile\@empty
      \else
        \BKM@EscapeHex\BKM@srcfile
      \fi
      \edef\BKM@prev{\the\BKM@id}%
      \global\advance\BKM@id\@ne
      \BKM@swtrue
      \@whilesw\ifBKM@sw\fi{%
        \ifnum\ifBKM@startatroot\z@\else\BKM@prev\fi=\z@
          \BKM@startatrootfalse
          \expandafter\xdef\csname BKM@\the\BKM@id\endcsname{%
            0{\BKM@level}0%
          }%
          \BKM@swfalse
        \else
          \expandafter\expandafter\expandafter\BKM@getx
              \csname BKM@\BKM@prev\endcsname
          \ifnum\BKM@level>\BKM@x@level\relax
            \expandafter\xdef\csname BKM@\the\BKM@id\endcsname{%
              {\BKM@prev}{\BKM@level}0%
            }%
            \ifnum\BKM@prev>\z@
              \BKM@CalcExpr\BKM@CalcResult\BKM@x@childs+1%
              \expandafter\xdef\csname BKM@\BKM@prev\endcsname{%
                {\BKM@x@parent}{\BKM@x@level}{\BKM@CalcResult}%
              }%
            \fi
            \BKM@swfalse
          \else
            \let\BKM@prev\BKM@x@parent
          \fi
        \fi
      }%
      \pdfstringdef\BKM@title{\bookmark@text}%
      \edef\BKM@FLAGS{\BKM@PrintStyle}%
      \csname BKM@HypDestOptHook\endcsname
      \BKM@EscapeHex\BKM@dest
      \BKM@EscapeHex\BKM@uri
      \BKM@EscapeHex\BKM@gotor
      \BKM@EscapeHex\BKM@rawaction
      \BKM@EscapeHex\BKM@title
      \immediate\write\@mainaux{%
        \string\BKM@entry{%
          id=\number\BKM@id
          \ifBKM@open
            \ifnum\BKM@level<\BKM@openlevel
              ,open%
            \fi
          \fi
          \BKM@auxentry{dest}%
          \BKM@auxentry{named}%
          \BKM@auxentry{uri}%
          \BKM@auxentry{gotor}%
          \BKM@auxentry{page}%
          \BKM@auxentry{view}%
          \BKM@auxentry{rawaction}%
          \BKM@auxentry{color}%
          \ifnum\BKM@FLAGS>\z@
            ,flags=\BKM@FLAGS
          \fi
          \BKM@auxentry{srcline}%
          \BKM@auxentry{srcfile}%
        }{\BKM@title}%
      }%
    \endgroup
  \fi
}
%    \end{macrocode}
%    \end{macro}
%    \begin{macro}{\BKM@getx}
%    \begin{macrocode}
\def\BKM@getx#1#2#3{%
  \def\BKM@x@parent{#1}%
  \def\BKM@x@level{#2}%
  \def\BKM@x@childs{#3}%
}
%    \end{macrocode}
%    \end{macro}
%    \begin{macro}{\BKM@auxentry}
%    \begin{macrocode}
\def\BKM@auxentry#1{%
  \expandafter\ifx\csname BKM@#1\endcsname\@empty
  \else
    ,#1={\csname BKM@#1\endcsname}%
  \fi
}
%    \end{macrocode}
%    \end{macro}
%
%    \begin{macro}{\BKM@InitSourceLocation}
%    \begin{macrocode}
\def\BKM@InitSourceLocation{%
  \edef\BKM@srcline{\the\inputlineno}%
  \BKM@LuaTeX@InitFile
  \ifx\BKM@srcfile\@empty
    \ltx@IfUndefined{currfilepath}{}{%
      \edef\BKM@srcfile{\currfilepath}%
    }%
  \fi
}
%    \end{macrocode}
%    \end{macro}
%    \begin{macro}{\BKM@LuaTeX@InitFile}
%    \begin{macrocode}
\ifluatex
  \ifnum\luatexversion>36 %
    \def\BKM@LuaTeX@InitFile{%
      \begingroup
        \ltx@LocToksA={}%
      \edef\x{\endgroup
        \def\noexpand\BKM@srcfile{%
          \the\expandafter\ltx@LocToksA
          \directlua{%
             if status and status.filename then %
               tex.settoks('ltx@LocToksA', status.filename)%
             end%
          }%
        }%
      }\x
    }%
  \else
    \let\BKM@LuaTeX@InitFile\relax
  \fi
\else
  \let\BKM@LuaTeX@InitFile\relax
\fi
%    \end{macrocode}
%    \end{macro}
%
% \subsubsection{读取辅助数据(auxiliary data)}
%
%    \begin{macrocode}
\SetupKeyvalOptions{family=BKM@DO,prefix=BKM@DO@}
\DeclareStringOption[0]{id}
\DeclareBoolOption{open}
\DeclareStringOption{flags}
\DeclareStringOption{color}
\DeclareStringOption{dest}
\DeclareStringOption{named}
\DeclareStringOption{uri}
\DeclareStringOption{gotor}
\DeclareStringOption{page}
\DeclareStringOption{view}
\DeclareStringOption{rawaction}
\DeclareStringOption{srcline}
\DeclareStringOption{srcfile}
%    \end{macrocode}
%
%    \begin{macrocode}
\AtBeginDocument{%
  \let\BKM@entry\BKM@DO@entry
}
%    \end{macrocode}
%
%    \begin{macrocode}
%</pdftex|pdfmark>
%    \end{macrocode}
%
% \subsection{\xoption{atend}\ 选项}
%
% \subsubsection{钩子(Hook)}
%
%    \begin{macrocode}
%<*package>
%    \end{macrocode}
%    \begin{macrocode}
\ifBKM@atend
\else
%    \end{macrocode}
%    \begin{macro}{\BookmarkAtEnd}
%    这是一个虚拟定义(dummy definition),如果没有给出 \xoption{atend}\ 选项,它将生成一个警告。
%    \begin{macrocode}
  \newcommand{\BookmarkAtEnd}[1]{%
    \PackageWarning{bookmark}{%
      Ignored, because option `atend' is missing%
    }%
  }%
%    \end{macrocode}
%    \end{macro}
%    \begin{macrocode}
  \expandafter\endinput
\fi
%    \end{macrocode}
%    \begin{macro}{\BookmarkAtEnd}
%    \begin{macrocode}
\newcommand*{\BookmarkAtEnd}{%
  \g@addto@macro\BKM@EndHook
}
%    \end{macrocode}
%    \end{macro}
%    \begin{macrocode}
\let\BKM@EndHook\@empty
%    \end{macrocode}
%    \begin{macrocode}
%</package>
%    \end{macrocode}
%
% \subsubsection{在文档末尾使用钩子的驱动程序}
%
%    驱动程序 \xoption{pdftex}\ 使用 LaTeX 钩子 \xoption{enddocument/afterlastpage}
%    (相当于以前使用的 \xpackage{atveryend}\ 的 \cs{AfterLastShipout}),因为它仍然需要 \xext{aux}\ 文件。
%    它使用 \cs{pdfoutline}\ 作为最后一页之后可以使用的书签(bookmakrs)。
%    \begin{itemize}
%    \item
%      驱动程序 \xoption{pdftex}\ 使用 \cs{pdfoutline}, \cs{pdfoutline}\ 可以在最后一页之后使用。
%    \end{itemize}
%    \begin{macrocode}
%<*pdftex>
\ifBKM@atend
  \AddToHook{enddocument/afterlastpage}{%
    \BKM@EndHook
  }%
\fi
%</pdftex>
%    \end{macrocode}
%
% \subsubsection{使用 \xoption{shipout/lastpage}\ 的驱动程序}
%
%    其他驱动程序使用 \cs{special}\ 命令实现 \cs{bookmark}。因此,最后的书签(last bookmarks)
%    必须放在最后一页(last page),而不是之后。不能使用 \cs{AtEndDocument},因为为时已晚,
%    最后一页已经输出了。因此,我们使用 LaTeX 钩子 \xoption{shipout/lastpage}。至少需要运行
%    两次 \hologo{LaTeX}。PostScript 驱动程序 \xoption{dvips}\ 使用外部 PostScript 文件作为书签。
%    为了避免与 pgf 发生冲突,文件写入(file writing)也被移到了最后一个输出页面(shipout page)。
%    \begin{macrocode}
%<*dvipdfm|vtex|pdfmark>
\ifBKM@atend
  \AddToHook{shipout/lastpage}{\BKM@EndHook}%
\fi
%</dvipdfm|vtex|pdfmark>
%    \end{macrocode}
%
% \section{安装(Installation)}
%
% \subsection{下载(Download)}
%
% \paragraph{宏包(Package)。} 在 CTAN\footnote{\CTANpkg{bookmark}}上提供此宏包:
% \begin{description}
% \item[\CTAN{macros/latex/contrib/bookmark/bookmark.dtx}] 源文件(source file)。
% \item[\CTAN{macros/latex/contrib/bookmark/bookmark.pdf}] 文档(documentation)。
% \end{description}
%
%
% \paragraph{捆绑包(Bundle)。} “bookmark”捆绑包(bundle)的所有宏包(packages)都可以在兼
% 容 TDS 的 ZIP 归档文件中找到。在那里,宏包已经被解包,文档文件(documentation files)已经生成。
% 文件(files)和目录(directories)遵循 TDS 标准。
% \begin{description}
% \item[\CTANinstall{install/macros/latex/contrib/bookmark.tds.zip}]
% \end{description}
% \emph{TDS}\ 是指标准的“用于 \TeX\ 文件的目录结构(Directory Structure)”(\CTANpkg{tds})。
% 名称中带有 \xfile{texmf}\ 的目录(directories)通常以这种方式组织。
%
% \subsection{捆绑包(Bundle)的安装}
%
% \paragraph{解压(Unpacking)。} 在您选择的 TDS 树(也称为 \xfile{texmf}\ 树)中解
% 压 \xfile{bookmark.tds.zip},例如(在 linux 中):
% \begin{quote}
%   |unzip bookmark.tds.zip -d ~/texmf|
% \end{quote}
%
% \subsection{宏包(Package)的安装}
%
% \paragraph{解压(Unpacking)。} \xfile{.dtx}\ 文件是一个自解压 \docstrip\ 归档文件(archive)。
% 这些文件是通过 \plainTeX\ 运行 \xfile{.dtx}\ 来提取的:
% \begin{quote}
%   \verb|tex bookmark.dtx|
% \end{quote}
%
% \paragraph{TDS.} 现在,不同的文件必须移动到安装 TDS 树(installation TDS tree)
% (也称为 \xfile{texmf}\ 树)中的不同目录中:
% \begin{quote}
% \def\t{^^A
% \begin{tabular}{@{}>{\ttfamily}l@{ $\rightarrow$ }>{\ttfamily}l@{}}
%   bookmark.sty & tex/latex/bookmark/bookmark.sty\\
%   bkm-dvipdfm.def & tex/latex/bookmark/bkm-dvipdfm.def\\
%   bkm-dvips.def & tex/latex/bookmark/bkm-dvips.def\\
%   bkm-pdftex.def & tex/latex/bookmark/bkm-pdftex.def\\
%   bkm-vtex.def & tex/latex/bookmark/bkm-vtex.def\\
%   bookmark.pdf & doc/latex/bookmark/bookmark.pdf\\
%   bookmark-example.tex & doc/latex/bookmark/bookmark-example.tex\\
%   bookmark.dtx & source/latex/bookmark/bookmark.dtx\\
% \end{tabular}^^A
% }^^A
% \sbox0{\t}^^A
% \ifdim\wd0>\linewidth
%   \begingroup
%     \advance\linewidth by\leftmargin
%     \advance\linewidth by\rightmargin
%   \edef\x{\endgroup
%     \def\noexpand\lw{\the\linewidth}^^A
%   }\x
%   \def\lwbox{^^A
%     \leavevmode
%     \hbox to \linewidth{^^A
%       \kern-\leftmargin\relax
%       \hss
%       \usebox0
%       \hss
%       \kern-\rightmargin\relax
%     }^^A
%   }^^A
%   \ifdim\wd0>\lw
%     \sbox0{\small\t}^^A
%     \ifdim\wd0>\linewidth
%       \ifdim\wd0>\lw
%         \sbox0{\footnotesize\t}^^A
%         \ifdim\wd0>\linewidth
%           \ifdim\wd0>\lw
%             \sbox0{\scriptsize\t}^^A
%             \ifdim\wd0>\linewidth
%               \ifdim\wd0>\lw
%                 \sbox0{\tiny\t}^^A
%                 \ifdim\wd0>\linewidth
%                   \lwbox
%                 \else
%                   \usebox0
%                 \fi
%               \else
%                 \lwbox
%               \fi
%             \else
%               \usebox0
%             \fi
%           \else
%             \lwbox
%           \fi
%         \else
%           \usebox0
%         \fi
%       \else
%         \lwbox
%       \fi
%     \else
%       \usebox0
%     \fi
%   \else
%     \lwbox
%   \fi
% \else
%   \usebox0
% \fi
% \end{quote}
% 如果你有一个 \xfile{docstrip.cfg}\ 文件,该文件能配置并启用 \docstrip\ 的 TDS 安装功能,
% 则一些文件可能已经在正确的位置了,请参阅 \docstrip\ 的文档(documentation)。
%
% \subsection{刷新文件名数据库}
%
% 如果您的 \TeX~发行版(\TeX\,Live、\mikTeX、\dots)依赖于文件名数据库(file name databases),
% 则必须刷新这些文件名数据库。例如,\TeX\,Live\ 用户运行 \verb|texhash| 或 \verb|mktexlsr|。
%
% \subsection{一些感兴趣的细节}
%
% \paragraph{用 \LaTeX\ 解压。}
% \xfile{.dtx}\ 根据格式(format)选择其操作(action):
% \begin{description}
% \item[\plainTeX:] 运行 \docstrip\ 并解压文件。
% \item[\LaTeX:] 生成文档。
% \end{description}
% 如果您坚持通过 \LaTeX\ 使用\docstrip (实际上 \docstrip\ 并不需要 \LaTeX),那么请您的意图告知自动检测程序:
% \begin{quote}
%   \verb|latex \let\install=y\input{bookmark.dtx}|
% \end{quote}
% 不要忘记根据 shell 的要求引用这个参数(argument)。
%
% \paragraph{知生成文档。}
% 您可以同时使用 \xfile{.dtx}\ 或 \xfile{.drv}\ 来生成文档。可以通过配置文件 \xfile{ltxdoc.cfg}\ 配置该进程。
% 例如,如果您希望 A4 作为纸张格式,请将下面这行写入此文件中:
% \begin{quote}
%   \verb|\PassOptionsToClass{a4paper}{article}|
% \end{quote}
% 下面是一个如何使用 pdf\LaTeX\ 生成文档的示例:
% \begin{quote}
%\begin{verbatim}
%pdflatex bookmark.dtx
%makeindex -s gind.ist bookmark.idx
%pdflatex bookmark.dtx
%makeindex -s gind.ist bookmark.idx
%pdflatex bookmark.dtx
%\end{verbatim}
% \end{quote}
%
% \begin{thebibliography}{9}
%
% \bibitem{hyperref}
%   Sebastian Rahtz, Heiko Oberdiek:
%   \textit{The \xpackage{hyperref} package};
%   2011/04/17 v6.82g;
%   \CTANpkg{hyperref}
%
% \bibitem{currfile}
%   Martin Scharrer:
%   \textit{The \xpackage{currfile} package};
%   2011/01/09 v0.4.
%   \CTANpkg{currfile}
%
% \end{thebibliography}
%
% \begin{History}
%   \begin{Version}{2007/02/19 v0.1}
%   \item
%     First experimental version.
%   \end{Version}
%   \begin{Version}{2007/02/20 v0.2}
%   \item
%     Option \xoption{startatroot} added.
%   \item
%     Dummies for \cs{pdf(un)escape...} commands added to get
%     the package basically work for non-\hologo{pdfTeX} users.
%   \end{Version}
%   \begin{Version}{2007/02/21 v0.3}
%   \item
%     Dependency from \hologo{pdfTeX} 1.30 removed by using package
%     \xpackage{pdfescape}.
%   \end{Version}
%   \begin{Version}{2007/02/22 v0.4}
%   \item
%     \xpackage{hyperref}'s \xoption{bookmarkstype} respected.
%   \end{Version}
%   \begin{Version}{2007/03/02 v0.5}
%   \item
%     Driver options \xoption{vtex} (PDF mode), \xoption{dvipsone},
%     and \xoption{textures} added.
%   \item
%     Implementation of option \xoption{depth} completed. Division names
%     are supported, see \xpackage{hyperref}'s
%     option \xoption{bookmarksdepth}.
%   \item
%     \xpackage{hyperref}'s options \xoption{bookmarksopen},
%     \xoption{bookmarksopenlevel}, and \xoption{bookmarksdepth} respected.
%   \end{Version}
%   \begin{Version}{2007/03/03 v0.6}
%   \item
%     Option \xoption{numbered} as alias for \xpackage{hyperref}'s
%     \xoption{bookmarksnumbered}.
%   \end{Version}
%   \begin{Version}{2007/03/07 v0.7}
%   \item
%     Dependency from \hologo{eTeX} removed.
%   \end{Version}
%   \begin{Version}{2007/04/09 v0.8}
%   \item
%     Option \xoption{atend} added.
%   \item
%     Option \xoption{rgbcolor} removed.
%     \verb|rgbcolor=<r> <g> <b>| can be replaced by
%     \verb|color=[rgb]{<r>,<g>,<b>}|.
%   \item
%     Support of recent cvs version (2007-03-29) of dvipdfmx
%     that extends the \cs{special} for bookmarks to specify
%     open outline entries. Option \xoption{dvipdfmx-outline-open}
%     or \cs{SpecialDvipdfmxOutlineOpen} notify the package.
%   \end{Version}
%   \begin{Version}{2007/04/25 v0.9}
%   \item
%     The syntax of \cs{special} of dvipdfmx, if feature
%     \xoption{dvipdfmx-outline-open} is enabled, has changed.
%     Now cvs version 2007-04-25 is needed.
%   \end{Version}
%   \begin{Version}{2007/05/29 v1.0}
%   \item
%     Bug fix in code for second parameter of XYZ.
%   \end{Version}
%   \begin{Version}{2007/07/13 v1.1}
%   \item
%     Fix for pdfmark with GoToR action.
%   \end{Version}
%   \begin{Version}{2007/09/25 v1.2}
%   \item
%     pdfmark driver respects \cs{nofiles}.
%   \end{Version}
%   \begin{Version}{2008/08/08 v1.3}
%   \item
%     Package \xpackage{flags} replaced by package \xpackage{bitset}.
%     Now flags are also supported without \hologo{eTeX}.
%   \item
%     Hook for package \xpackage{hypdestopt} added.
%   \end{Version}
%   \begin{Version}{2008/09/13 v1.4}
%   \item
%     Fix for bug introduced in v1.3, package \xpackage{flags} is one-based,
%     but package \xpackage{bitset} is zero-based. Thus options \xoption{bold}
%     and \xoption{italic} are wrong in v1.3. (Daniel M\"ullner)
%   \end{Version}
%   \begin{Version}{2009/08/13 v1.5}
%   \item
%     Except for driver options the other options are now local options.
%     This resolves a problem with KOMA-Script v3.00 and its option \xoption{open}.
%   \end{Version}
%   \begin{Version}{2009/12/06 v1.6}
%   \item
%     Use of package \xpackage{atveryend} for drivers \xoption{pdftex}
%     and \xoption{pdfmark}.
%   \end{Version}
%   \begin{Version}{2009/12/07 v1.7}
%   \item
%     Use of package \xpackage{atveryend} fixed.
%   \end{Version}
%   \begin{Version}{2009/12/17 v1.8}
%   \item
%     Support of \xpackage{hyperref} 2009/12/17 v6.79v for \hologo{XeTeX}.
%   \end{Version}
%   \begin{Version}{2010/03/30 v1.9}
%   \item
%     Package name in an error message fixed.
%   \end{Version}
%   \begin{Version}{2010/04/03 v1.10}
%   \item
%     Option \xoption{style} and macro \cs{bookmarkdefinestyle} added.
%   \item
%     Hook support with option \xoption{addtohook} added.
%   \item
%     \cs{bookmarkget} added.
%   \end{Version}
%   \begin{Version}{2010/04/04 v1.11}
%   \item
%     Bug fix (introduced in v1.10).
%   \end{Version}
%   \begin{Version}{2010/04/08 v1.12}
%   \item
%     Requires \xpackage{ltxcmds} 2010/04/08.
%   \end{Version}
%   \begin{Version}{2010/07/23 v1.13}
%   \item
%     Support for \xclass{memoir}'s \cs{booknumberline} added.
%   \end{Version}
%   \begin{Version}{2010/09/02 v1.14}
%   \item
%     (Local) options \xoption{draft} and \xoption{final} added.
%   \end{Version}
%   \begin{Version}{2010/09/25 v1.15}
%   \item
%     Fix for option \xoption{dvipdfmx-outline-open}.
%   \item
%     Option \xoption{dvipdfmx-outline-open} is set automatically,
%     if XeTeX $\geq$ 0.9995 is detected.
%   \end{Version}
%   \begin{Version}{2010/10/19 v1.16}
%   \item
%     Option `startatroot' now acts globally.
%   \item
%     Option `level' also accepts names the same way as option `depth'.
%   \end{Version}
%   \begin{Version}{2010/10/25 v1.17}
%   \item
%     \cs{bookmarksetupnext} added.
%   \item
%     Using \cs{kvsetkeys} of package \xpackage{kvsetkeys}, because
%     \cs{setkeys} of package \xpackage{keyval} is not reentrant.
%     This can cause problems (unknown keys) with older versions of
%     hyperref that also uses \cs{setkeys} (found by GL).
%   \end{Version}
%   \begin{Version}{2010/11/05 v1.18}
%   \item
%     Use of \cs{pdf@ifdraftmode} of package \xpackage{pdftexcmds} for
%     the default of option \xoption{draft}.
%   \end{Version}
%   \begin{Version}{2011/03/20 v1.19}
%   \item
%     Use of \cs{dimexpr} fixed, if \hologo{eTeX} is not used.
%     (Bug found by Martin M\"unch.)
%   \item
%     Fix in documentation. Also layout options work without \hologo{eTeX}.
%   \end{Version}
%   \begin{Version}{2011/04/13 v1.20}
%   \item
%     Bug fix: \cs{BKM@SetDepth} renamed to \cs{BKM@SetDepthOrLevel}.
%   \end{Version}
%   \begin{Version}{2011/04/21 v1.21}
%   \item
%     Some support for file name and line number in error messages
%     at end of document (pdfTeX and pdfmark based drivers).
%   \end{Version}
%   \begin{Version}{2011/05/13 v1.22}
%   \item
%     Change of version 2010/11/05 v1.18 reverted, because otherwise
%     draftmode disables some \xext{aux} file entries.
%   \end{Version}
%   \begin{Version}{2011/09/19 v1.23}
%   \item
%     Some \cs{renewcommand}s changed to \cs{def} to avoid trouble
%     if the commands are not defined, because hyperref stopped early.
%   \end{Version}
%   \begin{Version}{2011/12/02 v1.24}
%   \item
%     Small optimization in \cs{BKM@toHexDigit}.
%   \end{Version}
%   \begin{Version}{2016/05/16 v1.25}
%   \item
%     Documentation updates.
%   \end{Version}
%   \begin{Version}{2016/05/17 v1.26}
%   \item
%     define \cs{pdfoutline} to allow pdftex driver to be used with Lua\TeX.
%   \end{Version}
%   \begin{Version}{2019/06/04 v1.27}
%   \item
%     unknown style options are ignored (issue 67)
%   \end{Version}

%   \begin{Version}{2019/12/03 v1.28}
%   \item
%     Documentation updates.
%   \item adjust package loading (all required packages already loaded
%     by \xpackage{hyperref}).
%   \end{Version}
%   \begin{Version}{2020-11-06 v1.29}
%   \item Adapted the dvips to avoid a clash with pgf.
%         https://github.com/pgf-tikz/pgf/issues/944
%   \item All drivers now use the new LaTeX hooks
%         and so require a format 2020-10-01 or newer. The older
%         drivers are provided as frozen versions and are used if an older
%         format is detected.
%   \item Added support for destlabel option of hyperref, https://github.com/ho-tex/bookmark/issues/1
%   \item Removed the \xoption{dvipsone} and \xoption{textures} driver.
%   \item Removed the code for option \xoption{dvipdfmx-outline-open}
%     and \cs{SpecialDvipdfmxOutlineOpen}. All dvipdfmx version should now support
%     this out-of-the-box.
%   \end{Version}
% \end{History}
%
% \PrintIndex
%
% \Finale
\endinput

%        (quote the arguments according to the demands of your shell)
%
% Documentation:
%    (a) If bookmark.drv is present:
%           latex bookmark.drv
%    (b) Without bookmark.drv:
%           latex bookmark.dtx; ...
%    The class ltxdoc loads the configuration file ltxdoc.cfg
%    if available. Here you can specify further options, e.g.
%    use A4 as paper format:
%       \PassOptionsToClass{a4paper}{article}
%
%    Programm calls to get the documentation (example):
%       pdflatex bookmark.dtx
%       makeindex -s gind.ist bookmark.idx
%       pdflatex bookmark.dtx
%       makeindex -s gind.ist bookmark.idx
%       pdflatex bookmark.dtx
%
% Installation:
%    TDS:tex/latex/bookmark/bookmark.sty
%    TDS:tex/latex/bookmark/bkm-dvipdfm.def
%    TDS:tex/latex/bookmark/bkm-dvips.def
%    TDS:tex/latex/bookmark/bkm-pdftex.def
%    TDS:tex/latex/bookmark/bkm-vtex.def
%    TDS:tex/latex/bookmark/bkm-dvipdfm-2019-12-03.def
%    TDS:tex/latex/bookmark/bkm-dvips-2019-12-03.def
%    TDS:tex/latex/bookmark/bkm-pdftex-2019-12-03.def
%    TDS:tex/latex/bookmark/bkm-vtex-2019-12-03.def%
%    TDS:doc/latex/bookmark/bookmark.pdf
%    TDS:doc/latex/bookmark/bookmark-example.tex
%    TDS:source/latex/bookmark/bookmark.dtx
%    TDS:source/latex/bookmark/bookmark-frozen.dtx
%
%<*ignore>
\begingroup
  \catcode123=1 %
  \catcode125=2 %
  \def\x{LaTeX2e}%
\expandafter\endgroup
\ifcase 0\ifx\install y1\fi\expandafter
         \ifx\csname processbatchFile\endcsname\relax\else1\fi
         \ifx\fmtname\x\else 1\fi\relax
\else\csname fi\endcsname
%</ignore>
%<*install>
\input docstrip.tex
\Msg{************************************************************************}
\Msg{* Installation}
\Msg{* Package: bookmark 2020-11-06 v1.29 PDF bookmarks (HO)}
\Msg{************************************************************************}

\keepsilent
\askforoverwritefalse

\let\MetaPrefix\relax
\preamble

This is a generated file.

Project: bookmark
Version: 2020-11-06 v1.29

Copyright (C)
   2007-2011 Heiko Oberdiek
   2016-2020 Oberdiek Package Support Group

This work may be distributed and/or modified under the
conditions of the LaTeX Project Public License, either
version 1.3c of this license or (at your option) any later
version. This version of this license is in
   https://www.latex-project.org/lppl/lppl-1-3c.txt
and the latest version of this license is in
   https://www.latex-project.org/lppl.txt
and version 1.3 or later is part of all distributions of
LaTeX version 2005/12/01 or later.

This work has the LPPL maintenance status "maintained".

The Current Maintainers of this work are
Heiko Oberdiek and the Oberdiek Package Support Group
https://github.com/ho-tex/bookmark/issues


This work consists of the main source file bookmark.dtx and bookmark-frozen.dtx
and the derived files
   bookmark.sty, bookmark.pdf, bookmark.ins, bookmark.drv,
   bkm-dvipdfm.def, bkm-dvips.def, bkm-pdftex.def, bkm-vtex.def,
   bkm-dvipdfm-2019-12-03.def, bkm-dvips-2019-12-03.def,
   bkm-pdftex-2019-12-03.def, bkm-vtex-2019-12-03.def,
   bookmark-example.tex.

\endpreamble
\let\MetaPrefix\DoubleperCent

\generate{%
  \file{bookmark.ins}{\from{bookmark.dtx}{install}}%
  \file{bookmark.drv}{\from{bookmark.dtx}{driver}}%
  \usedir{tex/latex/bookmark}%
  \file{bookmark.sty}{\from{bookmark.dtx}{package}}%
  \file{bkm-dvipdfm.def}{\from{bookmark.dtx}{dvipdfm}}%
  \file{bkm-dvips.def}{\from{bookmark.dtx}{dvips,pdfmark}}%
  \file{bkm-pdftex.def}{\from{bookmark.dtx}{pdftex}}%
  \file{bkm-vtex.def}{\from{bookmark.dtx}{vtex}}%
  \usedir{doc/latex/bookmark}%
  \file{bookmark-example.tex}{\from{bookmark.dtx}{example}}%
  \file{bkm-pdftex-2019-12-03.def}{\from{bookmark-frozen.dtx}{pdftexfrozen}}%
  \file{bkm-dvips-2019-12-03.def}{\from{bookmark-frozen.dtx}{dvipsfrozen}}%
  \file{bkm-vtex-2019-12-03.def}{\from{bookmark-frozen.dtx}{vtexfrozen}}%
  \file{bkm-dvipdfm-2019-12-03.def}{\from{bookmark-frozen.dtx}{dvipdfmfrozen}}%
}

\catcode32=13\relax% active space
\let =\space%
\Msg{************************************************************************}
\Msg{*}
\Msg{* To finish the installation you have to move the following}
\Msg{* files into a directory searched by TeX:}
\Msg{*}
\Msg{*     bookmark.sty, bkm-dvipdfm.def, bkm-dvips.def,}
\Msg{*     bkm-pdftex.def, bkm-vtex.def, bkm-dvipdfm-2019-12-03.def,}
\Msg{*     bkm-dvips-2019-12-03.def, bkm-pdftex-2019-12-03.def,}
\Msg{*     and bkm-vtex-2019-12-03.def}
\Msg{*}
\Msg{* To produce the documentation run the file `bookmark.drv'}
\Msg{* through LaTeX.}
\Msg{*}
\Msg{* Happy TeXing!}
\Msg{*}
\Msg{************************************************************************}

\endbatchfile
%</install>
%<*ignore>
\fi
%</ignore>
%<*driver>
\NeedsTeXFormat{LaTeX2e}
\ProvidesFile{bookmark.drv}%
  [2020-11-06 v1.29 PDF bookmarks (HO)]%
\documentclass{ltxdoc}
\usepackage{ctex}
\usepackage{indentfirst}
\setlength{\parindent}{2em}
\usepackage{holtxdoc}[2011/11/22]
\usepackage{xcolor}
\usepackage{hyperref}
\usepackage[open,openlevel=3,atend]{bookmark}[2020/11/06] %%%打开书签,显示的深度为3级,即显示part、section、subsection。
\bookmarksetup{color=red}
\begin{document}

  \renewcommand{\contentsname}{目\quad 录}
  \renewcommand{\abstractname}{摘\quad 要}
  \renewcommand{\historyname}{历史}
  \DocInput{bookmark.dtx}%
\end{document}
%</driver>
% \fi
%
%
%
% \GetFileInfo{bookmark.drv}
%
%% \title{\xpackage{bookmark} 宏包}
% \title{\heiti {\Huge \textbf{\xpackage{bookmark}\ 宏包}}}
% \date{2020-11-06\ \ \ v1.29}
% \author{Heiko Oberdiek \thanks
% {如有问题请点击:\url{https://github.com/ho-tex/bookmark/issues}}\\[5pt]赣医一附院神经科\ \ 黄旭华\ \ \ \ 译}
%
% \maketitle
%
% \begin{abstract}
% 这个宏包为 \xpackage{hyperref}\ 宏包实现了一个新的书签(bookmark)(大纲[outline])组织。现在
% 可以设置样式(style)和颜色(color)等书签属性(bookmark properties)。其他动作类型(action types)可用
% (URI、GoToR、Named)。书签是在第一次编译运行(compile run)中生成的。\xpackage{hyperref}\
% 宏包必需运行两次。
% \end{abstract}
%
% \tableofcontents
%
% \section{文档(Documentation)}
%
% \subsection{介绍}
%
% 这个 \xpackage{bookmark}\ 宏包试图为书签(bookmarks)提供一个更现代的管理:
% \begin{itemize}
% \item 书签已经在第一次 \hologo{TeX}\ 编译运行(compile run)中生成。
% \item 可以更改书签的字体样式(font style)和颜色(color)。
% \item 可以执行比简单的 GoTo 操作(actions)更多的操作。
% \end{itemize}
%
% 与 \xpackage{hyperref} \cite{hyperref} 一样,书签(bookmarks)也是按照书签生成宏
% (bookmark generating macros)(\cs{bookmark})的顺序生成的。级别号(level number)用于
% 定义书签的树结构(tree structure)。限制没有那么严格:
% \begin{itemize}
% \item 级别值(level values)可以跳变(jump)和省略(omit)。\cs{subsubsection}\ 可以跟在
%       \cs{chapter}\ 之后。这种情况如在 \xpackage{hyperref}\ 中则产生错误,它将显示一个警告(warning)
%       并尝试修复此错误。
% \item 多个书签可能指向同一目标(destination)。在 \xpackage{hyperref}\ 中,这会完全弄乱
%       书签树(bookmark tree),因为算法假设(algorithm assumes)目标名称(destination names)
%       是键(keys)(唯一的)。
% \end{itemize}
%
% 注意,这个宏包是作为书签管理(bookmark management)的实验平台(experimentation platform)。
% 欢迎反馈。此外,在未来的版本中,接口(interfaces)也可能发生变化。
%
% \subsection{选项(Options)}
%
% 可在以下四个地方放置选项(options):
% \begin{enumerate}
% \item \cs{usepackage}|[|\meta{options}|]{bookmark}|\\
%       这是放置驱动程序选项(driver options)和 \xoption{atend}\ 选项的唯一位置。
% \item \cs{bookmarksetup}|{|\meta{options}|}|\\
%       此命令仅用于设置选项(setting options)。
% \item \cs{bookmarksetupnext}|{|\meta{options}|}|\\
%       这些选项在下一个 \cs{bookmark}\ 命令的选项之后存储(stored)和调用(called)。
% \item \cs{bookmark}|[|\meta{options}|]{|\meta{title}|}|\\
%       此命令设置书签。选项设置(option settings)仅限于此书签。
% \end{enumerate}
% 异常(Exception):加载该宏包后,无法更改驱动程序选项(Driver options)、\xoption{atend}\ 选项
% 、\xoption{draft}\slash\xoption{final}选项。
%
% \subsubsection{\xoption{draft} 和 \xoption{final}\ 选项}
%
% 如果一个\LaTeX\ 文件要被编译了多次,那么可以使用 \xoption{draft}\ 选项来禁用该宏包的书签内
% 容(bookmark stuff),这样可以节省一点时间。默认 \xoption{final}\ 选项。两个选项都是
% 布尔选项(boolean options),如果没有值,则使用值 |true|。|draft=true| 与 |final=false| 相同。
%
% 除了驱动程序选项(driver options)之外,\xpackage{bookmark}\ 宏包选项都是局部选项(local options)。
% \xoption{draft}\ 选项和 \xoption{final}\ 选项均属于文档类选项(class option)(译者注:文档类选项为全局选项),
% 因此,在 \xpackage{bookmark}\ 宏包中未能看到这两个选项。如果您想使用全局的(global) \xoption{draft}选项
% 来优化第一次 \LaTeX\ 运行(runs),可以在导言(preamble)中引入 \xpackage{ifdraft}\ 宏包并设置 \LaTeX\ 的
% \cs{PassOptionsToPackage},例如:
%\begin{quote}
%\begin{verbatim}
%\documentclass[draft]{article}
%\usepackage{ifdraft}
%\ifdraft{%
%   \PassOptionsToPackage{draft}{bookmark}%
%}{}
%\end{verbatim}
%\end{quote}
%
% \subsubsection{驱动程序选项(Driver options)}
%
% 支持的驱动程序( drivers)包括 \xoption{pdftex}、\xoption{dvips}、\xoption{dvipdfm} (\xoption{xetex})、
% \xoption{vtex}。\hologo{TeX}\ 引擎 \hologo{pdfTeX}、\hologo{XeTeX}、\hologo{VTeX}\ 能被自动检测到。
% 默认的 DVI 驱动程序是 \xoption{dvips}。这可以通过 \cs{BookmarkDriverDefault}\ 在配置
% 文件 \xfile{bookmark.cfg}\ 中进行更改,例如:
% \begin{quote}
% |\def\BookmarkDriverDefault{dvipdfm}|
% \end{quote}
% 当前版本的(current versions)驱动程序使用新的 \LaTeX\ 钩子(\LaTeX-hooks)。如果检测到比
% 2020-10-01 更旧的格式,则将以前驱动程序的冻结版本(frozen versions)作为备份(fallback)。
%
% \paragraph{用 dvipdfmx 打开书签(bookmarks)。}旧版本的宏包有一个 \xoption{dvipdfmx-outline-open}\ 选项
% 可以激活代码,而该代码可以指定一个大纲条目(outline entry)是否打开。该宏包现在假设所有使用的 dvipdfmx 版本都是
% 最新版本,足以理解该代码,因此始终激活该代码。选项本身将被忽略。
%
%
% \subsubsection{布局选项(Layout options)}
%
% \paragraph{字体(Font)选项:}
%
% \begin{description}
% \item[\xoption{bold}:] 如果受 PDF 浏览器(PDF viewer)支持,书签将以粗体字体(bold font)显示(自 PDF 1.4起)。
% \item[\xoption{italic}:] 使用斜体字体(italic font)(自 PDF 1.4起)。
% \end{description}
% \xoption{bold}(粗体) 和 \xoption{italic}(斜体)可以同时使用。而 |false| 值(value)禁用字体选项。
%
% \paragraph{颜色(Color)选项:}
%
% 彩色书签(Colored bookmarks)是 PDF 1.4 的一个特性(feature),并非所有的 PDF 浏览器(PDF viewers)都支持彩色书签。
% \begin{description}
% \item[\xoption{color}:] 这里 color(颜色)可以作为 \xpackage{color}\ 宏包或 \xpackage{xcolor}\ 宏包的
% 颜色规范(color specification)给出。空值(empty value)表示未设置颜色属性。如果未加载 \xpackage{xcolor}\ 宏包,
% 能识别的值(recognized values)只有:
%   \begin{itemize}
%   \item 空值(empty value)表示未设置颜色属性,\\
%         例如:|color={}|
%   \item 颜色模型(color model) rgb 的显式颜色规范(explicit color specification),\\
%         例如,红色(red):|color=[rgb]{1,0,0}|
%   \item 颜色模型(color model)灰(gray)的显式颜色规范(explicit color specification),\\
%         例如,深灰色(dark gray):|color=[gray]{0.25}|
%   \end{itemize}
%   请注意,如果加载了 \xpackage{color}\ 宏包,此限制(restriction)也适用。然而,如果加载了 \xpackage{xcolor}\ 宏包,
%   则可以使用所有颜色规范(color specifications)。
% \end{description}
%
% \subsubsection{动作选项(Action options)}
%
% \begin{description}
% \item[\xoption{dest}:] 目的地名称(destination name)。
% \item[\xoption{page}:] 页码(page number),第一页(first page)为 1。
% \item[\xoption{view}:] 浏览规范(view specification),示例如下:\\
%   |view={FitB}|, |view={FitH 842}|, |view={XYZ 0 100 null}|\ \  一些浏览规范参数(view specification parameters)
%   将数字(numbers)视为具有单位 bp 的参数。它们可以作为普通数字(plain numbers)或在 \cs{calc}\ 内部以
%   长度表达式(length expressions)给出。如果加载了 \xpackage{calc}\ 宏包,则支持该宏包的表达式(expressions)。否则,
%   使用 \hologo{eTeX}\ 的 \cs{dimexpr}。例如:\\
%   |view={FitH \calc{\paperheight-\topmargin-1in}}|\\
%   |view={XYZ 0 \calc{\paperheight} null}|\\
%   注意 \cs{calc}\ 不能用于 |XYZ| 的第三个参数,因为该参数是缩放值(zoom value),而不是长度(length)。

% \item[\xoption{named}:] 已命名的动作(Named action)的名称:\\
%   |FirstPage|(第一页),|LastPage|(最后一页),|NextPage|(下一页),|PrevPage|(前一页)
% \item[\xoption{gotor}:] 外部(external) PDF 文件的名称。
% \item[\xoption{uri}:] URI 规范(URI specification)。
% \item[\xoption{rawaction}:] 原始动作规范(raw action specification)。由于这些规范取决于驱动程序(driver),因此不应使用此选项。
% \end{description}
% 通过分析指定的选项来选择书签的适当动作。动作由不同的选项集(sets of options)区分:
% \begin{quote}
 \begin{tabular}{|@{}r|l@{}|}
%   \hline
%   \ \textbf{动作(Action)}\  & \ \textbf{选项(Options)}\ \\ \hline
%   \ \textsf{GoTo}\  &\  \xoption{dest}\ \\ \hline
%   \ \textsf{GoTo}\  & \ \xoption{page} + \xoption{view}\ \\ \hline
%   \ \textsf{GoToR}\  & \ \xoption{gotor} + \xoption{dest}\ \\ \hline
%   \ \textsf{GoToR}\  & \ \xoption{gotor} + \xoption{page} + \xoption{view}\ \ \ \\ \hline
%   \ \textsf{Named}\  &\  \xoption{named}\ \\ \hline
%   \ \textsf{URI}\  & \ \xoption{uri}\ \\ \hline
% \end{tabular}
% \end{quote}
%
% \paragraph{缺少动作(Missing actions)。}
% 如果动作缺少 \xpackage{bookmark}\ 宏包,则抛出错误消息(error message)。根据驱动程序(driver)
% (\xoption{pdftex}、\xoption{dvips}\ 和好友[friends]),宏包在文档末尾很晚才检测到它。
% 自 2011/04/21 v1.21 版本以后,该宏包尝试打印 \cs{bookmark}\ 的相应出现的行号(line number)和文件名(file name)。
% 然而,\hologo{TeX}\ 确实提供了行号,但不幸的是,文件名是一个秘密(secret)。但该宏包有如下获取文件名的方法:
% \begin{itemize}
% \item 如果 \hologo{LuaTeX} (独立于 DVI 或 PDF 模式)正在运行,则自动使用其 |status.filename|。
% \item 宏包的 \cs{currfile} \cite{currfile}\ 重新定义了 \hologo{LaTeX}\ 的内部结构,以跟踪文件名(file name)。
% 如果加载了该宏包,那么它的 \cs{currfilepath}\ 将被检测到并由 \xpackage{bookmark}\ 自动使用。
% \item 可以通过 \cs{bookmarksetup}\ 或 \cs{bookmark}\ 中的 \xoption{scrfile}\ 选项手动设置(set manually)文件名。
% 但是要小心,手动设置会禁用以前的文件名检测方法。错误的(wrong)或丢失的(missed)文件名设置(file name setting)可能会在错误消息中
% 为您提供错误的源位置(source location)。
% \end{itemize}
%
% \subsubsection{级别选项(Level options)}
%
% 书签条目(bookmark entries)的顺序由 \cs{bookmark}\ 命令的的出现顺序(appearance order)定义。
% 树结构(tree structure)由书签节点(bookmark nodes)的属性 \xoption{level}(级别)构建。
% \xoption{level}\ 的值是整数(integers)。如果书签条目级别的值高于前一个节点,则该条目将成为
% 前一个节点的子(child)节点。差值的绝对值并不重要。
%
% \xpackage{bookmark}\ 宏包能记住全局属性(global property)“current level(当前级别)”中上
% 一个书签条目(previous bookmark entry)的级别。
%
% 级别系统的(level system)行为(behaviour)可以通过以下选项进行配置:
% \begin{description}
% \item[\xoption{level}:]
%    设置级别(level),请参阅上面的说明。如果给出的选项 \xoption{level}\ 没有值,那么将恢复默
%    认行为,即将“当前级别(current level)”用作级别值(level value)。自 2010/10/19 v1.16 版本以来,
%    如果宏 \cs{toclevel@part}、\cs{toclevel@section}\ 被定义过(通过 \xpackage{hyperref}\ 宏包完成,
%    请参阅它的 \xoption{bookmarkdepth}\ 选项),则 \xpackage{bookmark}\ 宏包还支持 |part|、|section| 等名称。
%
% \item[\xoption{rellevel}:]
%    设置相对于前一级别的(previous level)级别。正值表示书签条目成为前一个书签条目的子条目。
% \item[\xoption{keeplevel}:]
%    使用由\xoption{level}\ 或 \xoption{rellevel}\ 设置的级别,但不要更改全局属性“current level(当前级别)”。
%    可以通过设置为 |false| 来禁用该选项。
% \item[\xoption{startatroot}:]
%    此时,书签树(bookmark tree)再次从顶层(top level)开始。下一个书签条目不会作为上一个条目的子条目进行排序。
%    示例场景:文档使用 part。但是,最后几章(last chapters)不应放在最后一部分(last part)下面:
%    \begin{quote}
%\begin{verbatim}
%\documentclass{book}
%[...]
%\begin{document}
%  \part{第一部分}
%    \chapter{第一部分的第1章}
%    [...]
%  \part{第二部分(Second part)}
%    \chapter{第二部分的第1章}
%    [...]
%  \bookmarksetup{startatroot}
%  \chapter{Index}% 不属于第二部分
%\end{document}
%\end{verbatim}
%    \end{quote}
% \end{description}
%
% \subsubsection{样式定义(Style definitions)}
%
% 样式(style)是一组选项设置(option settings)。它可以由宏 \cs{bookmarkdefinestyle}\ 定义,
% 并由它的 \xoption{style}\ 选项使用。
% \begin{declcs}{bookmarkdefinestyle} \M{name} \M{key value list}
% \end{declcs}
% 选项设置(option settings)的 \meta{key value list}(键值列表)被指定为样式名(style \meta{name})。
%
% \begin{description}
% \item[\xoption{style}:]
%   \xoption{style}\ 选项的值是以前定义的样式的名称(name)。现在执行其选项设置(option settings)。
%   选项可以包括 \xoption{style}\ 选项。通过递归调用相同样式的无限递归(endless recursion)被阻止并抛出一个错误。
% \end{description}
%
% \subsubsection{钩子支持(Hook support)}
%
% 处理宏\cs{bookmark}\ 的可选选项(optional options)后,就会调用钩子(hook)。
% \begin{description}
% \item[\xoption{addtohook}:]
%   代码(code)作为该选项的值添加到钩子中。
% \end{description}
%
% \begin{declcs}{bookmarkget} \M{option}
% \end{declcs}
% \cs{bookmarkget}\ 宏提取 \meta{option}\ 选项的最新选项设置(latest option setting)的值。
% 对于布尔选项(boolean option),如果启用布尔选项,则返回 1,否则结果为零。结果数字(resulting numbers)
% 可以直接用于 \cs{ifnum}\ 或 \cs{ifcase}。如果您想要数字 \texttt{0}\ 和 \texttt{1},
% 请在 \cs{bookmarkget}\ 前面加上 \cs{number}\ 作为前缀。\cs{bookmarkget}\ 宏是可展开的(expandable)。
% 如果选项不受支持,则返回空字符串(empty string)。受支持的布尔选项有:
% \begin{quote}
%   \xoption{bold}、
%   \xoption{italic}、
%   \xoption{open}
% \end{quote}
% 其他受支持的选项有:
% \begin{quote}
%   \xoption{depth}、
%   \xoption{dest}、
%   \xoption{color}、
%   \xoption{gotor}、
%   \xoption{level}、
%   \xoption{named}、
%   \xoption{openlevel}、
%   \xoption{page}、
%   \xoption{rawaction}、
%   \xoption{uri}、
%   \xoption{view}、
% \end{quote}
% 另外,以下键(key)是可用的:
% \begin{quote}
%   \xoption{text}
% \end{quote}
% 它返回大纲条目(outline entry)的文本(text)。
%
% \paragraph{选项设置(Option setting)。}
% 在钩子(hook)内部可以使用 \cs{bookmarksetup}\ 设置选项。
%
% \subsection{与 \xpackage{hyperref}\ 的兼容性}
%
% \xpackage{bookmark}\ 宏包自动禁用 \xpackage{hyperref}\ 宏包的书签(bookmarks)。但是,
% \xpackage{bookmark}\ 宏包使用了 \xpackage{hyperref}\ 宏包的一些代码。例如,
% \xpackage{bookmark}\ 宏包重新定义了 \xpackage{hyperref}\ 宏包在 \cs{addcontentsline}\ 命令
% 和其他命令中插入的\cs{Hy@writebookmark}\ 钩子。因此,不应禁用 \xpackage{hyperref}\ 宏包的书签。
%
% \xpackage{bookmark}\ 宏包使用 \xpackage{hyperref}\ 宏包的 \cs{pdfstringdef},且不提供替换(replacement)。
%
% \xpackage{hyperref}\ 宏包的一些选项也能在 \xpackage{bookmark}\ 宏包中实现(implemented):
% \begin{quote}
% \begin{tabular}{|l@{}|l@{}|}
%   \hline
%   \xpackage{hyperref}\ 宏包的选项\  &\ \xpackage{bookmark}\ 宏包的选项\ \ \\ \hline
%   \xoption{bookmarksdepth} &\ \xoption{depth}\\ \hline
%   \xoption{bookmarksopen} & \ \xoption{open}\\ \hline
%   \xoption{bookmarksopenlevel}\ \ \  &\ \xoption{openlevel}\\ \hline
%   \xoption{bookmarksnumbered} \ \ \ &\ \xoption{numbered}\\ \hline
% \end{tabular}
% \end{quote}
%
% 还可以使用以下命令:
% \begin{quote}
%   \cs{pdfbookmark}\\
%   \cs{currentpdfbookmark}\\
%   \cs{subpdfbookmark}\\
%   \cs{belowpdfbookmark}
% \end{quote}
%
% \subsection{在末尾添加书签}
%
% 宏包选项 \xoption{atend}\ 启用以下宏(macro):
% \begin{declcs}{BookmarkAtEnd}
%   \M{stuff}
% \end{declcs}
% \cs{BookmarkAtEnd}\ 宏将 \meta{stuff}\ 放在文档末尾。\meta{stuff}\ 表示书签命令(bookmark commands)。举例:
% \begin{quote}
%\begin{verbatim}
%\usepackage[atend]{bookmark}
%\BookmarkAtEnd{%
%  \bookmarksetup{startatroot}%
%  \bookmark[named=LastPage, level=0]{Last page}%
%}
%\end{verbatim}
% \end{quote}
%
% 或者,可以在 \cs{bookmark}\ 中给出 \xoption{startatroot}\ 选项:
% \begin{quote}
%\begin{verbatim}
%\BookmarkAtEnd{%
%  \bookmark[
%    startatroot,
%    named=LastPage,
%    level=0,
%  ]{Last page}%
%}
%\end{verbatim}
% \end{quote}
%
% \paragraph{备注(Remarks):}
% \begin{itemize}
% \item
%   \cs{BookmarkAtEnd} 隐藏了这样一个事实,即在文档末尾添加书签的方法取决于驱动程序(driver)。
%
%   为此,驱动程序 \xoption{pdftex}\ 使用 \xpackage{atveryend}\ 宏包。如果 \cs{AtEndDocument}\ 太早,
%   最后一个页面(last page)可能不会被发送出去(shipped out)。由于需要 \xext{aux}\ 文件,此驱动程序使
%   用 \cs{AfterLastShipout}。
%
%   其他驱动程序(\xoption{dvipdfm}、\xoption{xetex}、\xoption{vtex})的实现(implementation)
%   取决于 \cs{special},\cs{special}\ 在最后一页之后没有效果。在这种情况下,\xpackage{atenddvi}\ 宏包的
%   \cs{AtEndDvi}\ 有帮助。它将其参数(argument)放在文档的最后一页(last page)。至少需要运行 \hologo{LaTeX}\ 两次,
%   因为最后一页是由引用(reference)检测到的。
%
%   \xoption{dvips}\ 现在使用新的 LaTeX 钩子 \texttt{shipout/lastpage}。
% \item
%   未指定 \cs{BookmarkAtEnd}\ 参数的扩展时间(time of expansion)。这可以立即发生,也可以在文档末尾发生。
% \end{itemize}
%
% \subsection{限制/行动计划}
%
% \begin{itemize}
% \item 支持缺失动作(missing actions)(启动,\dots)。
% \item 对 \xpackage{hyperref}\ 的 \xoption{bookmarkstype}\ 选项进行了更好的设计(design)。
% \end{itemize}
%
% \section{示例(Example)}
%
%    \begin{macrocode}
%<*example>
%    \end{macrocode}
%    \begin{macrocode}
\documentclass{article}
\usepackage{xcolor}[2007/01/21]
\usepackage{hyperref}
\usepackage[
  open,
  openlevel=2,
  atend
]{bookmark}[2019/12/03]

\bookmarksetup{color=blue}

\BookmarkAtEnd{%
  \bookmarksetup{startatroot}%
  \bookmark[named=LastPage, level=0]{End/Last page}%
  \bookmark[named=FirstPage, level=1]{First page}%
}

\begin{document}
\section{First section}
\subsection{Subsection A}
\begin{figure}
  \hypertarget{fig}{}%
  A figure.
\end{figure}
\bookmark[
  rellevel=1,
  keeplevel,
  dest=fig
]{A figure}
\subsection{Subsection B}
\subsubsection{Subsubsection C}
\subsection{Umlauts: \"A\"O\"U\"a\"o\"u\ss}
\newpage
\bookmarksetup{
  bold,
  color=[rgb]{1,0,0}
}
\section{Very important section}
\bookmarksetup{
  italic,
  bold=false,
  color=blue
}
\subsection{Italic section}
\bookmarksetup{
  italic=false
}
\part{Misc}
\section{Diverse}
\subsubsection{Subsubsection, omitting subsection}
\bookmarksetup{
  startatroot
}
\section{Last section outside part}
\subsection{Subsection}
\bookmarksetup{
  color={}
}
\begingroup
  \bookmarksetup{level=0, color=green!80!black}
  \bookmark[named=FirstPage]{First page}
  \bookmark[named=LastPage]{Last page}
  \bookmark[named=PrevPage]{Previous page}
  \bookmark[named=NextPage]{Next page}
\endgroup
\bookmark[
  page=2,
  view=FitH 800
]{Page 2, FitH 800}
\bookmark[
  page=2,
  view=FitBH \calc{\paperheight-\topmargin-1in-\headheight-\headsep}
]{Page 2, FitBH top of text body}
\bookmark[
  uri={http://www.dante.de/},
  color=magenta
]{Dante homepage}
\bookmark[
  gotor={t.pdf},
  page=1,
  view={XYZ 0 1000 null},
  color=cyan!75!black
]{File t.pdf}
\bookmark[named=FirstPage]{First page}
\bookmark[rellevel=1, named=LastPage]{Last page (rellevel=1)}
\bookmark[named=PrevPage]{Previous page}
\bookmark[level=0, named=FirstPage]{First page (level=0)}
\bookmark[
  rellevel=1,
  keeplevel,
  named=LastPage
]{Last page (rellevel=1, keeplevel)}
\bookmark[named=PrevPage]{Previous page}
\end{document}
%    \end{macrocode}
%    \begin{macrocode}
%</example>
%    \end{macrocode}
%
% \StopEventually{
% }
%
% \section{实现(Implementation)}
%
% \subsection{宏包(Package)}
%
%    \begin{macrocode}
%<*package>
\NeedsTeXFormat{LaTeX2e}
\ProvidesPackage{bookmark}%
  [2020-11-06 v1.29 PDF bookmarks (HO)]%
%    \end{macrocode}
%
% \subsubsection{要求(Requirements)}
%
% \paragraph{\hologo{eTeX}.}
%
%    \begin{macro}{\BKM@CalcExpr}
%    \begin{macrocode}
\begingroup\expandafter\expandafter\expandafter\endgroup
\expandafter\ifx\csname numexpr\endcsname\relax
  \def\BKM@CalcExpr#1#2#3#4{%
    \begingroup
      \count@=#2\relax
      \advance\count@ by#3#4\relax
      \edef\x{\endgroup
        \def\noexpand#1{\the\count@}%
      }%
    \x
  }%
\else
  \def\BKM@CalcExpr#1#2#3#4{%
    \edef#1{%
      \the\numexpr#2#3#4\relax
    }%
  }%
\fi
%    \end{macrocode}
%    \end{macro}
%
% \paragraph{\hologo{pdfTeX}\ 的转义功能(escape features)}
%
%    \begin{macro}{\BKM@EscapeName}
%    \begin{macrocode}
\def\BKM@EscapeName#1{%
  \ifx#1\@empty
  \else
    \EdefEscapeName#1#1%
  \fi
}%
%    \end{macrocode}
%    \end{macro}
%    \begin{macro}{\BKM@EscapeString}
%    \begin{macrocode}
\def\BKM@EscapeString#1{%
  \ifx#1\@empty
  \else
    \EdefEscapeString#1#1%
  \fi
}%
%    \end{macrocode}
%    \end{macro}
%    \begin{macro}{\BKM@EscapeHex}
%    \begin{macrocode}
\def\BKM@EscapeHex#1{%
  \ifx#1\@empty
  \else
    \EdefEscapeHex#1#1%
  \fi
}%
%    \end{macrocode}
%    \end{macro}
%    \begin{macro}{\BKM@UnescapeHex}
%    \begin{macrocode}
\def\BKM@UnescapeHex#1{%
  \EdefUnescapeHex#1#1%
}%
%    \end{macrocode}
%    \end{macro}
%
% \paragraph{宏包(Packages)。}
%
% 不要加载由 \xpackage{hyperref}\ 加载的宏包
%    \begin{macrocode}
\RequirePackage{hyperref}[2010/06/18]
%    \end{macrocode}
%
% \subsubsection{宏包选项(Package options)}
%
%    \begin{macrocode}
\SetupKeyvalOptions{family=BKM,prefix=BKM@}
\DeclareLocalOptions{%
  atend,%
  bold,%
  color,%
  depth,%
  dest,%
  draft,%
  final,%
  gotor,%
  italic,%
  keeplevel,%
  level,%
  named,%
  numbered,%
  open,%
  openlevel,%
  page,%
  rawaction,%
  rellevel,%
  srcfile,%
  srcline,%
  startatroot,%
  uri,%
  view,%
}
%    \end{macrocode}
%    \begin{macro}{\bookmarksetup}
%    \begin{macrocode}
\newcommand*{\bookmarksetup}{\kvsetkeys{BKM}}
%    \end{macrocode}
%    \end{macro}
%    \begin{macro}{\BKM@setup}
%    \begin{macrocode}
\def\BKM@setup#1{%
  \bookmarksetup{#1}%
  \ifx\BKM@HookNext\ltx@empty
  \else
    \expandafter\bookmarksetup\expandafter{\BKM@HookNext}%
    \BKM@HookNextClear
  \fi
  \BKM@hook
  \ifBKM@keeplevel
  \else
    \xdef\BKM@currentlevel{\BKM@level}%
  \fi
}
%    \end{macrocode}
%    \end{macro}
%
%    \begin{macro}{\bookmarksetupnext}
%    \begin{macrocode}
\newcommand*{\bookmarksetupnext}[1]{%
  \ltx@GlobalAppendToMacro\BKM@HookNext{,#1}%
}
%    \end{macrocode}
%    \end{macro}
%    \begin{macro}{\BKM@setupnext}
%    \begin{macrocode}
%    \end{macrocode}
%    \end{macro}
%    \begin{macro}{\BKM@HookNextClear}
%    \begin{macrocode}
\def\BKM@HookNextClear{%
  \global\let\BKM@HookNext\ltx@empty
}
%    \end{macrocode}
%    \end{macro}
%    \begin{macro}{\BKM@HookNext}
%    \begin{macrocode}
\BKM@HookNextClear
%    \end{macrocode}
%    \end{macro}
%
%    \begin{macrocode}
\DeclareBoolOption{draft}
\DeclareComplementaryOption{final}{draft}
%    \end{macrocode}
%    \begin{macro}{\BKM@DisableOptions}
%    \begin{macrocode}
\def\BKM@DisableOptions{%
  \DisableKeyvalOption[action=warning,package=bookmark]%
      {BKM}{draft}%
  \DisableKeyvalOption[action=warning,package=bookmark]%
      {BKM}{final}%
}
%    \end{macrocode}
%    \end{macro}
%    \begin{macrocode}
\DeclareBoolOption[\ifHy@bookmarksopen true\else false\fi]{open}
%    \end{macrocode}
%    \begin{macro}{\bookmark@open}
%    \begin{macrocode}
\def\bookmark@open{%
  \ifBKM@open\ltx@one\else\ltx@zero\fi
}
%    \end{macrocode}
%    \end{macro}
%    \begin{macrocode}
\DeclareStringOption[\maxdimen]{openlevel}
%    \end{macrocode}
%    \begin{macro}{\BKM@openlevel}
%    \begin{macrocode}
\edef\BKM@openlevel{\number\@bookmarksopenlevel}
%    \end{macrocode}
%    \end{macro}
%    \begin{macrocode}
%\DeclareStringOption[\c@tocdepth]{depth}
\ltx@IfUndefined{Hy@bookmarksdepth}{%
  \def\BKM@depth{\c@tocdepth}%
}{%
  \let\BKM@depth\Hy@bookmarksdepth
}
\define@key{BKM}{depth}[]{%
  \edef\BKM@param{#1}%
  \ifx\BKM@param\@empty
    \def\BKM@depth{\c@tocdepth}%
  \else
    \ltx@IfUndefined{toclevel@\BKM@param}{%
      \@onelevel@sanitize\BKM@param
      \edef\BKM@temp{\expandafter\@car\BKM@param\@nil}%
      \ifcase 0\expandafter\ifx\BKM@temp-1\fi
              \expandafter\ifnum\expandafter`\BKM@temp>47 %
                \expandafter\ifnum\expandafter`\BKM@temp<58 %
                  1%
                \fi
              \fi
              \relax
        \PackageWarning{bookmark}{%
          Unknown document division name (\BKM@param)\MessageBreak
          for option `depth'%
        }%
      \else
        \BKM@SetDepthOrLevel\BKM@depth\BKM@param
      \fi
    }{%
      \BKM@SetDepthOrLevel\BKM@depth{%
        \csname toclevel@\BKM@param\endcsname
      }%
    }%
  \fi
}
%    \end{macrocode}
%    \begin{macro}{\bookmark@depth}
%    \begin{macrocode}
\def\bookmark@depth{\BKM@depth}
%    \end{macrocode}
%    \end{macro}
%    \begin{macro}{\BKM@SetDepthOrLevel}
%    \begin{macrocode}
\def\BKM@SetDepthOrLevel#1#2{%
  \begingroup
    \setbox\z@=\hbox{%
      \count@=#2\relax
      \expandafter
    }%
  \expandafter\endgroup
  \expandafter\def\expandafter#1\expandafter{\the\count@}%
}
%    \end{macrocode}
%    \end{macro}
%    \begin{macrocode}
\DeclareStringOption[\BKM@currentlevel]{level}[\BKM@currentlevel]
\define@key{BKM}{level}{%
  \edef\BKM@param{#1}%
  \ifx\BKM@param\BKM@MacroCurrentLevel
    \let\BKM@level\BKM@param
  \else
    \ltx@IfUndefined{toclevel@\BKM@param}{%
      \@onelevel@sanitize\BKM@param
      \edef\BKM@temp{\expandafter\@car\BKM@param\@nil}%
      \ifcase 0\expandafter\ifx\BKM@temp-1\fi
              \expandafter\ifnum\expandafter`\BKM@temp>47 %
                \expandafter\ifnum\expandafter`\BKM@temp<58 %
                  1%
                \fi
              \fi
              \relax
        \PackageWarning{bookmark}{%
          Unknown document division name (\BKM@param)\MessageBreak
          for option `level'%
        }%
      \else
        \BKM@SetDepthOrLevel\BKM@level\BKM@param
      \fi
    }{%
      \BKM@SetDepthOrLevel\BKM@level{%
        \csname toclevel@\BKM@param\endcsname
      }%
    }%
  \fi
}
%    \end{macrocode}
%    \begin{macro}{\BKM@MacroCurrentLevel}
%    \begin{macrocode}
\def\BKM@MacroCurrentLevel{\BKM@currentlevel}
%    \end{macrocode}
%    \end{macro}
%    \begin{macrocode}
\DeclareBoolOption{keeplevel}
\DeclareBoolOption{startatroot}
%    \end{macrocode}
%    \begin{macro}{\BKM@startatrootfalse}
%    \begin{macrocode}
\def\BKM@startatrootfalse{%
  \global\let\ifBKM@startatroot\iffalse
}
%    \end{macrocode}
%    \end{macro}
%    \begin{macro}{\BKM@startatroottrue}
%    \begin{macrocode}
\def\BKM@startatroottrue{%
  \global\let\ifBKM@startatroot\iftrue
}
%    \end{macrocode}
%    \end{macro}
%    \begin{macrocode}
\define@key{BKM}{rellevel}{%
  \BKM@CalcExpr\BKM@level{#1}+\BKM@currentlevel
}
%    \end{macrocode}
%    \begin{macro}{\bookmark@level}
%    \begin{macrocode}
\def\bookmark@level{\BKM@level}
%    \end{macrocode}
%    \end{macro}
%    \begin{macro}{\BKM@currentlevel}
%    \begin{macrocode}
\def\BKM@currentlevel{0}
%    \end{macrocode}
%    \end{macro}
%    Make \xpackage{bookmark}'s option \xoption{numbered} an alias
%    for \xpackage{hyperref}'s \xoption{bookmarksnumbered}.
%    \begin{macrocode}
\DeclareBoolOption[%
  \ifHy@bookmarksnumbered true\else false\fi
]{numbered}
\g@addto@macro\BKM@numberedtrue{%
  \let\ifHy@bookmarksnumbered\iftrue
}
\g@addto@macro\BKM@numberedfalse{%
  \let\ifHy@bookmarksnumbered\iffalse
}
\g@addto@macro\Hy@bookmarksnumberedtrue{%
  \let\ifBKM@numbered\iftrue
}
\g@addto@macro\Hy@bookmarksnumberedfalse{%
  \let\ifBKM@numbered\iffalse
}
%    \end{macrocode}
%    \begin{macro}{\bookmark@numbered}
%    \begin{macrocode}
\def\bookmark@numbered{%
  \ifBKM@numbered\ltx@one\else\ltx@zero\fi
}
%    \end{macrocode}
%    \end{macro}
%
% \paragraph{重定义 \xpackage{hyperref}\ 宏包的选项}
%
%    \begin{macro}{\BKM@PatchHyperrefOption}
%    \begin{macrocode}
\def\BKM@PatchHyperrefOption#1{%
  \expandafter\BKM@@PatchHyperrefOption\csname KV@Hyp@#1\endcsname%
}
%    \end{macrocode}
%    \end{macro}
%    \begin{macro}{\BKM@@PatchHyperrefOption}
%    \begin{macrocode}
\def\BKM@@PatchHyperrefOption#1{%
  \expandafter\BKM@@@PatchHyperrefOption#1{##1}\BKM@nil#1%
}
%    \end{macrocode}
%    \end{macro}
%    \begin{macro}{\BKM@@@PatchHyperrefOption}
%    \begin{macrocode}
\def\BKM@@@PatchHyperrefOption#1\BKM@nil#2#3{%
  \def#2##1{%
    #1%
    \bookmarksetup{#3={##1}}%
  }%
}
%    \end{macrocode}
%    \end{macro}
%    \begin{macrocode}
\BKM@PatchHyperrefOption{bookmarksopen}{open}
\BKM@PatchHyperrefOption{bookmarksopenlevel}{openlevel}
\BKM@PatchHyperrefOption{bookmarksdepth}{depth}
%    \end{macrocode}
%
% \paragraph{字体样式(font style)选项。}
%
%    注意:\xpackage{bitset}\ 宏是基于零的,PDF 规范(PDF specifications)以1开头。
%    \begin{macrocode}
\bitsetReset{BKM@FontStyle}%
\define@key{BKM}{italic}[true]{%
  \expandafter\ifx\csname if#1\endcsname\iftrue
    \bitsetSet{BKM@FontStyle}{0}%
  \else
    \bitsetClear{BKM@FontStyle}{0}%
  \fi
}%
\define@key{BKM}{bold}[true]{%
  \expandafter\ifx\csname if#1\endcsname\iftrue
    \bitsetSet{BKM@FontStyle}{1}%
  \else
    \bitsetClear{BKM@FontStyle}{1}%
  \fi
}%
%    \end{macrocode}
%    \begin{macro}{\bookmark@italic}
%    \begin{macrocode}
\def\bookmark@italic{%
  \ifnum\bitsetGet{BKM@FontStyle}{0}=1 \ltx@one\else\ltx@zero\fi
}
%    \end{macrocode}
%    \end{macro}
%    \begin{macro}{\bookmark@bold}
%    \begin{macrocode}
\def\bookmark@bold{%
  \ifnum\bitsetGet{BKM@FontStyle}{1}=1 \ltx@one\else\ltx@zero\fi
}
%    \end{macrocode}
%    \end{macro}
%    \begin{macro}{\BKM@PrintStyle}
%    \begin{macrocode}
\def\BKM@PrintStyle{%
  \bitsetGetDec{BKM@FontStyle}%
}%
%    \end{macrocode}
%    \end{macro}
%
% \paragraph{颜色(color)选项。}
%
%    \begin{macrocode}
\define@key{BKM}{color}{%
  \HyColor@BookmarkColor{#1}\BKM@color{bookmark}{color}%
}
%    \end{macrocode}
%    \begin{macro}{\BKM@color}
%    \begin{macrocode}
\let\BKM@color\@empty
%    \end{macrocode}
%    \end{macro}
%    \begin{macro}{\bookmark@color}
%    \begin{macrocode}
\def\bookmark@color{\BKM@color}
%    \end{macrocode}
%    \end{macro}
%
% \subsubsection{动作(action)选项}
%
%    \begin{macrocode}
\def\BKM@temp#1{%
  \DeclareStringOption{#1}%
  \expandafter\edef\csname bookmark@#1\endcsname{%
    \expandafter\noexpand\csname BKM@#1\endcsname
  }%
}
%    \end{macrocode}
%    \begin{macro}{\bookmark@dest}
%    \begin{macrocode}
\BKM@temp{dest}
%    \end{macrocode}
%    \end{macro}
%    \begin{macro}{\bookmark@named}
%    \begin{macrocode}
\BKM@temp{named}
%    \end{macrocode}
%    \end{macro}
%    \begin{macro}{\bookmark@uri}
%    \begin{macrocode}
\BKM@temp{uri}
%    \end{macrocode}
%    \end{macro}
%    \begin{macro}{\bookmark@gotor}
%    \begin{macrocode}
\BKM@temp{gotor}
%    \end{macrocode}
%    \end{macro}
%    \begin{macro}{\bookmark@rawaction}
%    \begin{macrocode}
\BKM@temp{rawaction}
%    \end{macrocode}
%    \end{macro}
%
%    \begin{macrocode}
\define@key{BKM}{page}{%
  \def\BKM@page{#1}%
  \ifx\BKM@page\@empty
  \else
    \edef\BKM@page{\number\BKM@page}%
    \ifnum\BKM@page>\z@
    \else
      \PackageError{bookmark}{Page must be positive}\@ehc
      \def\BKM@page{1}%
    \fi
  \fi
}
%    \end{macrocode}
%    \begin{macro}{\BKM@page}
%    \begin{macrocode}
\let\BKM@page\@empty
%    \end{macrocode}
%    \end{macro}
%    \begin{macro}{\bookmark@page}
%    \begin{macrocode}
\def\bookmark@page{\BKM@@page}
%    \end{macrocode}
%    \end{macro}
%
%    \begin{macrocode}
\define@key{BKM}{view}{%
  \BKM@CheckView{#1}%
}
%    \end{macrocode}
%    \begin{macro}{\BKM@view}
%    \begin{macrocode}
\let\BKM@view\@empty
%    \end{macrocode}
%    \end{macro}
%    \begin{macro}{\bookmark@view}
%    \begin{macrocode}
\def\bookmark@view{\BKM@view}
%    \end{macrocode}
%    \end{macro}
%    \begin{macro}{BKM@CheckView}
%    \begin{macrocode}
\def\BKM@CheckView#1{%
  \BKM@CheckViewType#1 \@nil
}
%    \end{macrocode}
%    \end{macro}
%    \begin{macro}{\BKM@CheckViewType}
%    \begin{macrocode}
\def\BKM@CheckViewType#1 #2\@nil{%
  \def\BKM@type{#1}%
  \@onelevel@sanitize\BKM@type
  \BKM@TestViewType{Fit}{}%
  \BKM@TestViewType{FitB}{}%
  \BKM@TestViewType{FitH}{%
    \BKM@CheckParam#2 \@nil{top}%
  }%
  \BKM@TestViewType{FitBH}{%
    \BKM@CheckParam#2 \@nil{top}%
  }%
  \BKM@TestViewType{FitV}{%
    \BKM@CheckParam#2 \@nil{bottom}%
  }%
  \BKM@TestViewType{FitBV}{%
    \BKM@CheckParam#2 \@nil{bottom}%
  }%
  \BKM@TestViewType{FitR}{%
    \BKM@CheckRect{#2}{ }%
  }%
  \BKM@TestViewType{XYZ}{%
    \BKM@CheckXYZ{#2}{ }%
  }%
  \@car{%
    \PackageError{bookmark}{%
      Unknown view type `\BKM@type',\MessageBreak
      using `FitH' instead%
    }\@ehc
    \def\BKM@view{FitH}%
  }%
  \@nil
}
%    \end{macrocode}
%    \end{macro}
%    \begin{macro}{\BKM@TestViewType}
%    \begin{macrocode}
\def\BKM@TestViewType#1{%
  \def\BKM@temp{#1}%
  \@onelevel@sanitize\BKM@temp
  \ifx\BKM@type\BKM@temp
    \let\BKM@view\BKM@temp
    \expandafter\@car
  \else
    \expandafter\@gobble
  \fi
}
%    \end{macrocode}
%    \end{macro}
%    \begin{macro}{BKM@CheckParam}
%    \begin{macrocode}
\def\BKM@CheckParam#1 #2\@nil#3{%
  \def\BKM@param{#1}%
  \ifx\BKM@param\@empty
    \PackageWarning{bookmark}{%
      Missing parameter (#3) for `\BKM@type',\MessageBreak
      using 0%
    }%
    \def\BKM@param{0}%
  \else
    \BKM@CalcParam
  \fi
  \edef\BKM@view{\BKM@view\space\BKM@param}%
}
%    \end{macrocode}
%    \end{macro}
%    \begin{macro}{BKM@CheckRect}
%    \begin{macrocode}
\def\BKM@CheckRect#1#2{%
  \BKM@@CheckRect#1#2#2#2#2\@nil
}
%    \end{macrocode}
%    \end{macro}
%    \begin{macro}{\BKM@@CheckRect}
%    \begin{macrocode}
\def\BKM@@CheckRect#1 #2 #3 #4 #5\@nil{%
  \def\BKM@temp{0}%
  \def\BKM@param{#1}%
  \ifx\BKM@param\@empty
    \def\BKM@param{0}%
    \def\BKM@temp{1}%
  \else
    \BKM@CalcParam
  \fi
  \edef\BKM@view{\BKM@view\space\BKM@param}%
  \def\BKM@param{#2}%
  \ifx\BKM@param\@empty
    \def\BKM@param{0}%
    \def\BKM@temp{1}%
  \else
    \BKM@CalcParam
  \fi
  \edef\BKM@view{\BKM@view\space\BKM@param}%
  \def\BKM@param{#3}%
  \ifx\BKM@param\@empty
    \def\BKM@param{0}%
    \def\BKM@temp{1}%
  \else
    \BKM@CalcParam
  \fi
  \edef\BKM@view{\BKM@view\space\BKM@param}%
  \def\BKM@param{#4}%
  \ifx\BKM@param\@empty
    \def\BKM@param{0}%
    \def\BKM@temp{1}%
  \else
    \BKM@CalcParam
  \fi
  \edef\BKM@view{\BKM@view\space\BKM@param}%
  \ifnum\BKM@temp>\z@
    \PackageWarning{bookmark}{Missing parameters for `\BKM@type'}%
  \fi
}
%    \end{macrocode}
%    \end{macro}
%    \begin{macro}{\BKM@CheckXYZ}
%    \begin{macrocode}
\def\BKM@CheckXYZ#1#2{%
  \BKM@@CheckXYZ#1#2#2#2\@nil
}
%    \end{macrocode}
%    \end{macro}
%    \begin{macro}{\BKM@@CheckXYZ}
%    \begin{macrocode}
\def\BKM@@CheckXYZ#1 #2 #3 #4\@nil{%
  \def\BKM@param{#1}%
  \let\BKM@temp\BKM@param
  \@onelevel@sanitize\BKM@temp
  \ifx\BKM@param\@empty
    \let\BKM@param\BKM@null
  \else
    \ifx\BKM@temp\BKM@null
    \else
      \BKM@CalcParam
    \fi
  \fi
  \edef\BKM@view{\BKM@view\space\BKM@param}%
  \def\BKM@param{#2}%
  \let\BKM@temp\BKM@param
  \@onelevel@sanitize\BKM@temp
  \ifx\BKM@param\@empty
    \let\BKM@param\BKM@null
  \else
    \ifx\BKM@temp\BKM@null
    \else
      \BKM@CalcParam
    \fi
  \fi
  \edef\BKM@view{\BKM@view\space\BKM@param}%
  \def\BKM@param{#3}%
  \ifx\BKM@param\@empty
    \let\BKM@param\BKM@null
  \fi
  \edef\BKM@view{\BKM@view\space\BKM@param}%
}
%    \end{macrocode}
%    \end{macro}
%    \begin{macro}{\BKM@null}
%    \begin{macrocode}
\def\BKM@null{null}
\@onelevel@sanitize\BKM@null
%    \end{macrocode}
%    \end{macro}
%
%    \begin{macro}{\BKM@CalcParam}
%    \begin{macrocode}
\def\BKM@CalcParam{%
  \begingroup
  \let\calc\@firstofone
  \expandafter\BKM@@CalcParam\BKM@param\@empty\@empty\@nil
}
%    \end{macrocode}
%    \end{macro}
%    \begin{macro}{\BKM@@CalcParam}
%    \begin{macrocode}
\def\BKM@@CalcParam#1#2#3\@nil{%
  \ifx\calc#1%
    \@ifundefined{calc@assign@dimen}{%
      \@ifundefined{dimexpr}{%
        \setlength{\dimen@}{#2}%
      }{%
        \setlength{\dimen@}{\dimexpr#2\relax}%
      }%
    }{%
      \setlength{\dimen@}{#2}%
    }%
    \dimen@.99626\dimen@
    \edef\BKM@param{\strip@pt\dimen@}%
    \expandafter\endgroup
    \expandafter\def\expandafter\BKM@param\expandafter{\BKM@param}%
  \else
    \endgroup
  \fi
}
%    \end{macrocode}
%    \end{macro}
%
% \subsubsection{\xoption{atend}\ 选项}
%
%    \begin{macrocode}
\DeclareBoolOption{atend}
\g@addto@macro\BKM@DisableOptions{%
  \DisableKeyvalOption[action=warning,package=bookmark]%
      {BKM}{atend}%
}
%    \end{macrocode}
%
% \subsubsection{\xoption{style}\ 选项}
%
%    \begin{macro}{\bookmarkdefinestyle}
%    \begin{macrocode}
\newcommand*{\bookmarkdefinestyle}[2]{%
  \@ifundefined{BKM@style@#1}{%
  }{%
    \PackageInfo{bookmark}{Redefining style `#1'}%
  }%
  \@namedef{BKM@style@#1}{#2}%
}
%    \end{macrocode}
%    \end{macro}
%    \begin{macrocode}
\define@key{BKM}{style}{%
  \BKM@StyleCall{#1}%
}
\newif\ifBKM@ok
%    \end{macrocode}
%    \begin{macro}{\BKM@StyleCall}
%    \begin{macrocode}
\def\BKM@StyleCall#1{%
  \@ifundefined{BKM@style@#1}{%
    \PackageWarning{bookmark}{%
      Ignoring unknown style `#1'%
    }%
  }{%
%    \end{macrocode}
%    检查样式堆栈(style stack)。
%    \begin{macrocode}
    \BKM@oktrue
    \edef\BKM@StyleCurrent{#1}%
    \@onelevel@sanitize\BKM@StyleCurrent
    \let\BKM@StyleEntry\BKM@StyleEntryCheck
    \BKM@StyleStack
    \ifBKM@ok
      \expandafter\@firstofone
    \else
      \PackageError{bookmark}{%
        Ignoring recursive call of style `\BKM@StyleCurrent'%
      }\@ehc
      \expandafter\@gobble
    \fi
    {%
%    \end{macrocode}
%    在堆栈上推送当前样式(Push current style on stack)。
%    \begin{macrocode}
      \let\BKM@StyleEntry\relax
      \edef\BKM@StyleStack{%
        \BKM@StyleEntry{\BKM@StyleCurrent}%
        \BKM@StyleStack
      }%
%    \end{macrocode}
%   调用样式(Call style)。
%    \begin{macrocode}
      \expandafter\expandafter\expandafter\bookmarksetup
      \expandafter\expandafter\expandafter{%
        \csname BKM@style@\BKM@StyleCurrent\endcsname
      }%
%    \end{macrocode}
%    从堆栈中弹出当前样式(Pop current style from stack)。
%    \begin{macrocode}
      \BKM@StyleStackPop
    }%
  }%
}
%    \end{macrocode}
%    \end{macro}
%    \begin{macro}{\BKM@StyleStackPop}
%    \begin{macrocode}
\def\BKM@StyleStackPop{%
  \let\BKM@StyleEntry\relax
  \edef\BKM@StyleStack{%
    \expandafter\@gobbletwo\BKM@StyleStack
  }%
}
%    \end{macrocode}
%    \end{macro}
%    \begin{macro}{\BKM@StyleEntryCheck}
%    \begin{macrocode}
\def\BKM@StyleEntryCheck#1{%
  \def\BKM@temp{#1}%
  \ifx\BKM@temp\BKM@StyleCurrent
    \BKM@okfalse
  \fi
}
%    \end{macrocode}
%    \end{macro}
%    \begin{macro}{\BKM@StyleStack}
%    \begin{macrocode}
\def\BKM@StyleStack{}
%    \end{macrocode}
%    \end{macro}
%
% \subsubsection{源文件位置(source file location)选项}
%
%    \begin{macrocode}
\DeclareStringOption{srcline}
\DeclareStringOption{srcfile}
%    \end{macrocode}
%
% \subsubsection{钩子支持(Hook support)}
%
%    \begin{macro}{\BKM@hook}
%    \begin{macrocode}
\def\BKM@hook{}
%    \end{macrocode}
%    \end{macro}
%    \begin{macrocode}
\define@key{BKM}{addtohook}{%
  \ltx@LocalAppendToMacro\BKM@hook{#1}%
}
%    \end{macrocode}
%
%    \begin{macro}{bookmarkget}
%    \begin{macrocode}
\newcommand*{\bookmarkget}[1]{%
  \romannumeral0%
  \ltx@ifundefined{bookmark@#1}{%
    \ltx@space
  }{%
    \expandafter\expandafter\expandafter\ltx@space
    \csname bookmark@#1\endcsname
  }%
}
%    \end{macrocode}
%    \end{macro}
%
% \subsubsection{设置和加载驱动程序}
%
% \paragraph{检测驱动程序。}
%
%    \begin{macro}{\BKM@DefineDriverKey}
%    \begin{macrocode}
\def\BKM@DefineDriverKey#1{%
  \define@key{BKM}{#1}[]{%
    \def\BKM@driver{#1}%
  }%
  \g@addto@macro\BKM@DisableOptions{%
    \DisableKeyvalOption[action=warning,package=bookmark]%
        {BKM}{#1}%
  }%
}
%    \end{macrocode}
%    \end{macro}
%    \begin{macrocode}
\BKM@DefineDriverKey{pdftex}
\BKM@DefineDriverKey{dvips}
\BKM@DefineDriverKey{dvipdfm}
\BKM@DefineDriverKey{dvipdfmx}
\BKM@DefineDriverKey{xetex}
\BKM@DefineDriverKey{vtex}
\define@key{BKM}{dvipdfmx-outline-open}[true]{%
 \PackageWarning{bookmark}{Option 'dvipdfmx-outline-open' is obsolete
   and ignored}{}}
%    \end{macrocode}
%    \begin{macro}{\bookmark@driver}
%    \begin{macrocode}
\def\bookmark@driver{\BKM@driver}
%    \end{macrocode}
%    \end{macro}
%    \begin{macrocode}
\InputIfFileExists{bookmark.cfg}{}{}
%    \end{macrocode}
%    \begin{macro}{\BookmarkDriverDefault}
%    \begin{macrocode}
\providecommand*{\BookmarkDriverDefault}{dvips}
%    \end{macrocode}
%    \end{macro}
%    \begin{macro}{\BKM@driver}
% Lua\TeX\ 和 pdf\TeX\ 共享驱动程序。
%    \begin{macrocode}
\ifpdf
  \def\BKM@driver{pdftex}%
  \ifx\pdfoutline\@undefined
    \ifx\pdfextension\@undefined\else
      \protected\def\pdfoutline{\pdfextension outline }
    \fi
  \fi
\else
  \ifxetex
    \def\BKM@driver{dvipdfm}%
  \else
    \ifvtex
      \def\BKM@driver{vtex}%
    \else
      \edef\BKM@driver{\BookmarkDriverDefault}%
    \fi
  \fi
\fi
%    \end{macrocode}
%    \end{macro}
%
% \paragraph{过程选项(Process options)。}
%
%    \begin{macrocode}
\ProcessKeyvalOptions*
\BKM@DisableOptions
%    \end{macrocode}
%
% \paragraph{\xoption{draft}\ 选项}
%
%    \begin{macrocode}
\ifBKM@draft
  \PackageWarningNoLine{bookmark}{Draft mode on}%
  \let\bookmarksetup\ltx@gobble
  \let\BookmarkAtEnd\ltx@gobble
  \let\bookmarkdefinestyle\ltx@gobbletwo
  \let\bookmarkget\ltx@gobble
  \let\pdfbookmark\ltx@undefined
  \newcommand*{\pdfbookmark}[3][]{}%
  \let\currentpdfbookmark\ltx@gobbletwo
  \let\subpdfbookmark\ltx@gobbletwo
  \let\belowpdfbookmark\ltx@gobbletwo
  \newcommand*{\bookmark}[2][]{}%
  \renewcommand*{\Hy@writebookmark}[5]{}%
  \let\ReadBookmarks\relax
  \let\BKM@DefGotoNameAction\ltx@gobbletwo % package `hypdestopt'
  \expandafter\endinput
\fi
%    \end{macrocode}
%
% \paragraph{验证和加载驱动程序。}
%
%    \begin{macrocode}
\def\BKM@temp{dvipdfmx}%
\ifx\BKM@temp\BKM@driver
  \def\BKM@driver{dvipdfm}%
\fi
\def\BKM@temp{pdftex}%
\ifpdf
  \ifx\BKM@temp\BKM@driver
  \else
    \PackageWarningNoLine{bookmark}{%
      Wrong driver `\BKM@driver', using `pdftex' instead%
    }%
    \let\BKM@driver\BKM@temp
  \fi
\else
  \ifx\BKM@temp\BKM@driver
    \PackageError{bookmark}{%
      Wrong driver, pdfTeX is not running in PDF mode.\MessageBreak
      Package loading is aborted%
    }\@ehc
    \expandafter\expandafter\expandafter\endinput
  \fi
  \def\BKM@temp{dvipdfm}%
  \ifxetex
    \ifx\BKM@temp\BKM@driver
    \else
      \PackageWarningNoLine{bookmark}{%
        Wrong driver `\BKM@driver',\MessageBreak
        using `dvipdfm' for XeTeX instead%
      }%
      \let\BKM@driver\BKM@temp
    \fi
  \else
    \def\BKM@temp{vtex}%
    \ifvtex
      \ifx\BKM@temp\BKM@driver
      \else
        \PackageWarningNoLine{bookmark}{%
          Wrong driver `\BKM@driver',\MessageBreak
          using `vtex' for VTeX instead%
        }%
        \let\BKM@driver\BKM@temp
      \fi
    \else
      \ifx\BKM@temp\BKM@driver
        \PackageError{bookmark}{%
          Wrong driver, VTeX is not running in PDF mode.\MessageBreak
          Package loading is aborted%
        }\@ehc
        \expandafter\expandafter\expandafter\endinput
      \fi
    \fi
  \fi
\fi
\providecommand\IfFormatAtLeastTF{\@ifl@t@r\fmtversion}
\IfFormatAtLeastTF{2020/10/01}{}{\edef\BKM@driver{\BKM@driver-2019-12-03}}
\InputIfFileExists{bkm-\BKM@driver.def}{}{%
  \PackageError{bookmark}{%
    Unsupported driver `\BKM@driver'.\MessageBreak
    Package loading is aborted%
  }\@ehc
  \endinput
}
%    \end{macrocode}
%
% \subsubsection{与 \xpackage{hyperref}\ 的兼容性}
%
%    \begin{macro}{\pdfbookmark}
%    \begin{macrocode}
\let\pdfbookmark\ltx@undefined
\newcommand*{\pdfbookmark}[3][0]{%
  \bookmark[level=#1,dest={#3.#1}]{#2}%
  \hyper@anchorstart{#3.#1}\hyper@anchorend
}
%    \end{macrocode}
%    \end{macro}
%    \begin{macro}{\currentpdfbookmark}
%    \begin{macrocode}
\def\currentpdfbookmark{%
  \pdfbookmark[\BKM@currentlevel]%
}
%    \end{macrocode}
%    \end{macro}
%    \begin{macro}{\subpdfbookmark}
%    \begin{macrocode}
\def\subpdfbookmark{%
  \BKM@CalcExpr\BKM@CalcResult\BKM@currentlevel+1%
  \expandafter\pdfbookmark\expandafter[\BKM@CalcResult]%
}
%    \end{macrocode}
%    \end{macro}
%    \begin{macro}{\belowpdfbookmark}
%    \begin{macrocode}
\def\belowpdfbookmark#1#2{%
  \xdef\BKM@gtemp{\number\BKM@currentlevel}%
  \subpdfbookmark{#1}{#2}%
  \global\let\BKM@currentlevel\BKM@gtemp
}
%    \end{macrocode}
%    \end{macro}
%
%    节号(section number)、文本(text)、标签(label)、级别(level)、文件(file)
%    \begin{macro}{\Hy@writebookmark}
%    \begin{macrocode}
\def\Hy@writebookmark#1#2#3#4#5{%
  \ifnum#4>\BKM@depth\relax
  \else
    \def\BKM@type{#5}%
    \ifx\BKM@type\Hy@bookmarkstype
      \begingroup
        \ifBKM@numbered
          \let\numberline\Hy@numberline
          \let\booknumberline\Hy@numberline
          \let\partnumberline\Hy@numberline
          \let\chapternumberline\Hy@numberline
        \else
          \let\numberline\@gobble
          \let\booknumberline\@gobble
          \let\partnumberline\@gobble
          \let\chapternumberline\@gobble
        \fi
        \bookmark[level=#4,dest={\HyperDestNameFilter{#3}}]{#2}%
      \endgroup
    \fi
  \fi
}
%    \end{macrocode}
%    \end{macro}
%
%    \begin{macro}{\ReadBookmarks}
%    \begin{macrocode}
\let\ReadBookmarks\relax
%    \end{macrocode}
%    \end{macro}
%
%    \begin{macrocode}
%</package>
%    \end{macrocode}
%
% \subsection{dvipdfm 的驱动程序}
%
%    \begin{macrocode}
%<*dvipdfm>
\NeedsTeXFormat{LaTeX2e}
\ProvidesFile{bkm-dvipdfm.def}%
  [2020-11-06 v1.29 bookmark driver for dvipdfm (HO)]%
%    \end{macrocode}
%
%    \begin{macro}{\BKM@id}
%    \begin{macrocode}
\newcount\BKM@id
\BKM@id=\z@
%    \end{macrocode}
%    \end{macro}
%
%    \begin{macro}{\BKM@0}
%    \begin{macrocode}
\@namedef{BKM@0}{000}
%    \end{macrocode}
%    \end{macro}
%    \begin{macro}{\ifBKM@sw}
%    \begin{macrocode}
\newif\ifBKM@sw
%    \end{macrocode}
%    \end{macro}
%
%    \begin{macro}{\bookmark}
%    \begin{macrocode}
\newcommand*{\bookmark}[2][]{%
  \if@filesw
    \begingroup
      \def\bookmark@text{#2}%
      \BKM@setup{#1}%
      \edef\BKM@prev{\the\BKM@id}%
      \global\advance\BKM@id\@ne
      \BKM@swtrue
      \@whilesw\ifBKM@sw\fi{%
        \def\BKM@abslevel{1}%
        \ifnum\ifBKM@startatroot\z@\else\BKM@prev\fi=\z@
          \BKM@startatrootfalse
          \expandafter\xdef\csname BKM@\the\BKM@id\endcsname{%
            0{\BKM@level}\BKM@abslevel
          }%
          \BKM@swfalse
        \else
          \expandafter\expandafter\expandafter\BKM@getx
              \csname BKM@\BKM@prev\endcsname
          \ifnum\BKM@level>\BKM@x@level\relax
            \BKM@CalcExpr\BKM@abslevel\BKM@x@abslevel+1%
            \expandafter\xdef\csname BKM@\the\BKM@id\endcsname{%
              {\BKM@prev}{\BKM@level}\BKM@abslevel
            }%
            \BKM@swfalse
          \else
            \let\BKM@prev\BKM@x@parent
          \fi
        \fi
      }%
      \csname HyPsd@XeTeXBigCharstrue\endcsname
      \pdfstringdef\BKM@title{\bookmark@text}%
      \edef\BKM@FLAGS{\BKM@PrintStyle}%
      \let\BKM@action\@empty
      \ifx\BKM@gotor\@empty
        \ifx\BKM@dest\@empty
          \ifx\BKM@named\@empty
            \ifx\BKM@rawaction\@empty
              \ifx\BKM@uri\@empty
                \ifx\BKM@page\@empty
                  \PackageError{bookmark}{Missing action}\@ehc
                  \edef\BKM@action{/Dest[@page1/Fit]}%
                \else
                  \ifx\BKM@view\@empty
                    \def\BKM@view{Fit}%
                  \fi
                  \edef\BKM@action{/Dest[@page\BKM@page/\BKM@view]}%
                \fi
              \else
                \BKM@EscapeString\BKM@uri
                \edef\BKM@action{%
                  /A<<%
                    /S/URI%
                    /URI(\BKM@uri)%
                  >>%
                }%
              \fi
            \else
              \edef\BKM@action{/A<<\BKM@rawaction>>}%
            \fi
          \else
            \BKM@EscapeName\BKM@named
            \edef\BKM@action{%
              /A<</S/Named/N/\BKM@named>>%
            }%
          \fi
        \else
          \BKM@EscapeString\BKM@dest
          \edef\BKM@action{%
            /A<<%
              /S/GoTo%
              /D(\BKM@dest)%
            >>%
          }%
        \fi
      \else
        \ifx\BKM@dest\@empty
          \ifx\BKM@page\@empty
            \def\BKM@page{0}%
          \else
            \BKM@CalcExpr\BKM@page\BKM@page-1%
          \fi
          \ifx\BKM@view\@empty
            \def\BKM@view{Fit}%
          \fi
          \edef\BKM@action{/D[\BKM@page/\BKM@view]}%
        \else
          \BKM@EscapeString\BKM@dest
          \edef\BKM@action{/D(\BKM@dest)}%
        \fi
        \BKM@EscapeString\BKM@gotor
        \edef\BKM@action{%
          /A<<%
            /S/GoToR%
            /F(\BKM@gotor)%
            \BKM@action
          >>%
        }%
      \fi
      \special{pdf:%
        out
              [%
              \ifBKM@open
                \ifnum\BKM@level<%
                    \expandafter\ltx@firstofone\expandafter
                    {\number\BKM@openlevel} %
                \else
                  -%
                \fi
              \else
                -%
              \fi
              ] %
            \BKM@abslevel
        <<%
          /Title(\BKM@title)%
          \ifx\BKM@color\@empty
          \else
            /C[\BKM@color]%
          \fi
          \ifnum\BKM@FLAGS>\z@
            /F \BKM@FLAGS
          \fi
          \BKM@action
        >>%
      }%
    \endgroup
  \fi
}
%    \end{macrocode}
%    \end{macro}
%    \begin{macro}{\BKM@getx}
%    \begin{macrocode}
\def\BKM@getx#1#2#3{%
  \def\BKM@x@parent{#1}%
  \def\BKM@x@level{#2}%
  \def\BKM@x@abslevel{#3}%
}
%    \end{macrocode}
%    \end{macro}
%
%    \begin{macrocode}
%</dvipdfm>
%    \end{macrocode}
%
% \subsection{\hologo{VTeX}\ 的驱动程序}
%
%    \begin{macrocode}
%<*vtex>
\NeedsTeXFormat{LaTeX2e}
\ProvidesFile{bkm-vtex.def}%
  [2020-11-06 v1.29 bookmark driver for VTeX (HO)]%
%    \end{macrocode}
%
%    \begin{macrocode}
\ifvtexpdf
\else
  \PackageWarningNoLine{bookmark}{%
    The VTeX driver only supports PDF mode%
  }%
\fi
%    \end{macrocode}
%
%    \begin{macro}{\BKM@id}
%    \begin{macrocode}
\newcount\BKM@id
\BKM@id=\z@
%    \end{macrocode}
%    \end{macro}
%
%    \begin{macro}{\BKM@0}
%    \begin{macrocode}
\@namedef{BKM@0}{00}
%    \end{macrocode}
%    \end{macro}
%    \begin{macro}{\ifBKM@sw}
%    \begin{macrocode}
\newif\ifBKM@sw
%    \end{macrocode}
%    \end{macro}
%
%    \begin{macro}{\bookmark}
%    \begin{macrocode}
\newcommand*{\bookmark}[2][]{%
  \if@filesw
    \begingroup
      \def\bookmark@text{#2}%
      \BKM@setup{#1}%
      \edef\BKM@prev{\the\BKM@id}%
      \global\advance\BKM@id\@ne
      \BKM@swtrue
      \@whilesw\ifBKM@sw\fi{%
        \ifnum\ifBKM@startatroot\z@\else\BKM@prev\fi=\z@
          \BKM@startatrootfalse
          \def\BKM@parent{0}%
          \expandafter\xdef\csname BKM@\the\BKM@id\endcsname{%
            0{\BKM@level}%
          }%
          \BKM@swfalse
        \else
          \expandafter\expandafter\expandafter\BKM@getx
              \csname BKM@\BKM@prev\endcsname
          \ifnum\BKM@level>\BKM@x@level\relax
            \let\BKM@parent\BKM@prev
            \expandafter\xdef\csname BKM@\the\BKM@id\endcsname{%
              {\BKM@prev}{\BKM@level}%
            }%
            \BKM@swfalse
          \else
            \let\BKM@prev\BKM@x@parent
          \fi
        \fi
      }%
      \pdfstringdef\BKM@title{\bookmark@text}%
      \BKM@vtex@title
      \edef\BKM@FLAGS{\BKM@PrintStyle}%
      \let\BKM@action\@empty
      \ifx\BKM@gotor\@empty
        \ifx\BKM@dest\@empty
          \ifx\BKM@named\@empty
            \ifx\BKM@rawaction\@empty
              \ifx\BKM@uri\@empty
                \ifx\BKM@page\@empty
                  \PackageError{bookmark}{Missing action}\@ehc
                  \def\BKM@action{!1}%
                \else
                  \edef\BKM@action{!\BKM@page}%
                \fi
              \else
                \BKM@EscapeString\BKM@uri
                \edef\BKM@action{%
                  <u=%
                    /S/URI%
                    /URI(\BKM@uri)%
                  >%
                }%
              \fi
            \else
              \edef\BKM@action{<u=\BKM@rawaction>}%
            \fi
          \else
            \BKM@EscapeName\BKM@named
            \edef\BKM@action{%
              <u=%
                /S/Named%
                /N/\BKM@named
              >%
            }%
          \fi
        \else
          \BKM@EscapeString\BKM@dest
          \edef\BKM@action{\BKM@dest}%
        \fi
      \else
        \ifx\BKM@dest\@empty
          \ifx\BKM@page\@empty
            \def\BKM@page{1}%
          \fi
          \ifx\BKM@view\@empty
            \def\BKM@view{Fit}%
          \fi
          \edef\BKM@action{/D[\BKM@page/\BKM@view]}%
        \else
          \BKM@EscapeString\BKM@dest
          \edef\BKM@action{/D(\BKM@dest)}%
        \fi
        \BKM@EscapeString\BKM@gotor
        \edef\BKM@action{%
          <u=%
            /S/GoToR%
            /F(\BKM@gotor)%
            \BKM@action
          >>%
        }%
      \fi
      \ifx\BKM@color\@empty
        \let\BKM@RGBcolor\@empty
      \else
        \expandafter\BKM@toRGB\BKM@color\@nil
      \fi
      \special{%
        !outline \BKM@action;%
        p=\BKM@parent,%
        i=\number\BKM@id,%
        s=%
          \ifBKM@open
            \ifnum\BKM@level<\BKM@openlevel
              o%
            \else
              c%
            \fi
          \else
            c%
          \fi,%
        \ifx\BKM@RGBcolor\@empty
        \else
          c=\BKM@RGBcolor,%
        \fi
        \ifnum\BKM@FLAGS>\z@
          f=\BKM@FLAGS,%
        \fi
        t=\BKM@title
      }%
    \endgroup
  \fi
}
%    \end{macrocode}
%    \end{macro}
%    \begin{macro}{\BKM@getx}
%    \begin{macrocode}
\def\BKM@getx#1#2{%
  \def\BKM@x@parent{#1}%
  \def\BKM@x@level{#2}%
}
%    \end{macrocode}
%    \end{macro}
%    \begin{macro}{\BKM@toRGB}
%    \begin{macrocode}
\def\BKM@toRGB#1 #2 #3\@nil{%
  \let\BKM@RGBcolor\@empty
  \BKM@toRGBComponent{#1}%
  \BKM@toRGBComponent{#2}%
  \BKM@toRGBComponent{#3}%
}
%    \end{macrocode}
%    \end{macro}
%    \begin{macro}{\BKM@toRGBComponent}
%    \begin{macrocode}
\def\BKM@toRGBComponent#1{%
  \dimen@=#1pt\relax
  \ifdim\dimen@>\z@
    \ifdim\dimen@<\p@
      \dimen@=255\dimen@
      \advance\dimen@ by 32768sp\relax
      \divide\dimen@ by 65536\relax
      \dimen@ii=\dimen@
      \divide\dimen@ii by 16\relax
      \edef\BKM@RGBcolor{%
        \BKM@RGBcolor
        \BKM@toHexDigit\dimen@ii
      }%
      \dimen@ii=16\dimen@ii
      \advance\dimen@-\dimen@ii
      \edef\BKM@RGBcolor{%
        \BKM@RGBcolor
        \BKM@toHexDigit\dimen@
      }%
    \else
      \edef\BKM@RGBcolor{\BKM@RGBcolor FF}%
    \fi
  \else
    \edef\BKM@RGBcolor{\BKM@RGBcolor00}%
  \fi
}
%    \end{macrocode}
%    \end{macro}
%    \begin{macro}{\BKM@toHexDigit}
%    \begin{macrocode}
\def\BKM@toHexDigit#1{%
  \ifcase\expandafter\@firstofone\expandafter{\number#1} %
    0\or 1\or 2\or 3\or 4\or 5\or 6\or 7\or
    8\or 9\or A\or B\or C\or D\or E\or F%
  \fi
}
%    \end{macrocode}
%    \end{macro}
%    \begin{macrocode}
\begingroup
  \catcode`\|=0 %
  \catcode`\\=12 %
%    \end{macrocode}
%    \begin{macro}{\BKM@vtex@title}
%    \begin{macrocode}
  |gdef|BKM@vtex@title{%
    |@onelevel@sanitize|BKM@title
    |edef|BKM@title{|expandafter|BKM@vtex@leftparen|BKM@title\(|@nil}%
    |edef|BKM@title{|expandafter|BKM@vtex@rightparen|BKM@title\)|@nil}%
    |edef|BKM@title{|expandafter|BKM@vtex@zero|BKM@title\0|@nil}%
    |edef|BKM@title{|expandafter|BKM@vtex@one|BKM@title\1|@nil}%
    |edef|BKM@title{|expandafter|BKM@vtex@two|BKM@title\2|@nil}%
    |edef|BKM@title{|expandafter|BKM@vtex@three|BKM@title\3|@nil}%
  }%
%    \end{macrocode}
%    \end{macro}
%    \begin{macro}{\BKM@vtex@leftparen}
%    \begin{macrocode}
  |gdef|BKM@vtex@leftparen#1\(#2|@nil{%
    #1%
    |ifx||#2||%
    |else
      (%
      |ltx@ReturnAfterFi{%
        |BKM@vtex@leftparen#2|@nil
      }%
    |fi
  }%
%    \end{macrocode}
%    \end{macro}
%    \begin{macro}{\BKM@vtex@rightparen}
%    \begin{macrocode}
  |gdef|BKM@vtex@rightparen#1\)#2|@nil{%
    #1%
    |ifx||#2||%
    |else
      )%
      |ltx@ReturnAfterFi{%
        |BKM@vtex@rightparen#2|@nil
      }%
    |fi
  }%
%    \end{macrocode}
%    \end{macro}
%    \begin{macro}{\BKM@vtex@zero}
%    \begin{macrocode}
  |gdef|BKM@vtex@zero#1\0#2|@nil{%
    #1%
    |ifx||#2||%
    |else
      |noexpand|hv@pdf@char0%
      |ltx@ReturnAfterFi{%
        |BKM@vtex@zero#2|@nil
      }%
    |fi
  }%
%    \end{macrocode}
%    \end{macro}
%    \begin{macro}{\BKM@vtex@one}
%    \begin{macrocode}
  |gdef|BKM@vtex@one#1\1#2|@nil{%
    #1%
    |ifx||#2||%
    |else
      |noexpand|hv@pdf@char1%
      |ltx@ReturnAfterFi{%
        |BKM@vtex@one#2|@nil
      }%
    |fi
  }%
%    \end{macrocode}
%    \end{macro}
%    \begin{macro}{\BKM@vtex@two}
%    \begin{macrocode}
  |gdef|BKM@vtex@two#1\2#2|@nil{%
    #1%
    |ifx||#2||%
    |else
      |noexpand|hv@pdf@char2%
      |ltx@ReturnAfterFi{%
        |BKM@vtex@two#2|@nil
      }%
    |fi
  }%
%    \end{macrocode}
%    \end{macro}
%    \begin{macro}{\BKM@vtex@three}
%    \begin{macrocode}
  |gdef|BKM@vtex@three#1\3#2|@nil{%
    #1%
    |ifx||#2||%
    |else
      |noexpand|hv@pdf@char3%
      |ltx@ReturnAfterFi{%
        |BKM@vtex@three#2|@nil
      }%
    |fi
  }%
%    \end{macrocode}
%    \end{macro}
%    \begin{macrocode}
|endgroup
%    \end{macrocode}
%
%    \begin{macrocode}
%</vtex>
%    \end{macrocode}
%
% \subsection{\hologo{pdfTeX}\ 的驱动程序}
%
%    \begin{macrocode}
%<*pdftex>
\NeedsTeXFormat{LaTeX2e}
\ProvidesFile{bkm-pdftex.def}%
  [2020-11-06 v1.29 bookmark driver for pdfTeX (HO)]%
%    \end{macrocode}
%
%    \begin{macro}{\BKM@DO@entry}
%    \begin{macrocode}
\def\BKM@DO@entry#1#2{%
  \begingroup
    \kvsetkeys{BKM@DO}{#1}%
    \def\BKM@DO@title{#2}%
    \ifx\BKM@DO@srcfile\@empty
    \else
      \BKM@UnescapeHex\BKM@DO@srcfile
    \fi
    \BKM@UnescapeHex\BKM@DO@title
    \expandafter\expandafter\expandafter\BKM@getx
        \csname BKM@\BKM@DO@id\endcsname\@empty\@empty
    \let\BKM@attr\@empty
    \ifx\BKM@DO@flags\@empty
    \else
      \edef\BKM@attr{\BKM@attr/F \BKM@DO@flags}%
    \fi
    \ifx\BKM@DO@color\@empty
    \else
      \edef\BKM@attr{\BKM@attr/C[\BKM@DO@color]}%
    \fi
    \ifx\BKM@attr\@empty
    \else
      \edef\BKM@attr{attr{\BKM@attr}}%
    \fi
    \let\BKM@action\@empty
    \ifx\BKM@DO@gotor\@empty
      \ifx\BKM@DO@dest\@empty
        \ifx\BKM@DO@named\@empty
          \ifx\BKM@DO@rawaction\@empty
            \ifx\BKM@DO@uri\@empty
              \ifx\BKM@DO@page\@empty
                \PackageError{bookmark}{%
                  Missing action\BKM@SourceLocation
                }\@ehc
                \edef\BKM@action{goto page1{/Fit}}%
              \else
                \ifx\BKM@DO@view\@empty
                  \def\BKM@DO@view{Fit}%
                \fi
                \edef\BKM@action{goto page\BKM@DO@page{/\BKM@DO@view}}%
              \fi
            \else
              \BKM@UnescapeHex\BKM@DO@uri
              \BKM@EscapeString\BKM@DO@uri
              \edef\BKM@action{user{<</S/URI/URI(\BKM@DO@uri)>>}}%
            \fi
          \else
            \BKM@UnescapeHex\BKM@DO@rawaction
            \edef\BKM@action{%
              user{%
                <<%
                  \BKM@DO@rawaction
                >>%
              }%
            }%
          \fi
        \else
          \BKM@EscapeName\BKM@DO@named
          \edef\BKM@action{%
            user{<</S/Named/N/\BKM@DO@named>>}%
          }%
        \fi
      \else
        \BKM@UnescapeHex\BKM@DO@dest
        \BKM@DefGotoNameAction\BKM@action\BKM@DO@dest
      \fi
    \else
      \ifx\BKM@DO@dest\@empty
        \ifx\BKM@DO@page\@empty
          \def\BKM@DO@page{0}%
        \else
          \BKM@CalcExpr\BKM@DO@page\BKM@DO@page-1%
        \fi
        \ifx\BKM@DO@view\@empty
          \def\BKM@DO@view{Fit}%
        \fi
        \edef\BKM@action{/D[\BKM@DO@page/\BKM@DO@view]}%
      \else
        \BKM@UnescapeHex\BKM@DO@dest
        \BKM@EscapeString\BKM@DO@dest
        \edef\BKM@action{/D(\BKM@DO@dest)}%
      \fi
      \BKM@UnescapeHex\BKM@DO@gotor
      \BKM@EscapeString\BKM@DO@gotor
      \edef\BKM@action{%
        user{%
          <<%
            /S/GoToR%
            /F(\BKM@DO@gotor)%
            \BKM@action
          >>%
        }%
      }%
    \fi
    \pdfoutline\BKM@attr\BKM@action
                count\ifBKM@DO@open\else-\fi\BKM@x@childs
                {\BKM@DO@title}%
  \endgroup
}
%    \end{macrocode}
%    \end{macro}
%    \begin{macro}{\BKM@DefGotoNameAction}
%    \cs{BKM@DefGotoNameAction}\ 宏是一个用于 \xpackage{hypdestopt}\ 宏包的钩子(hook)。
%    \begin{macrocode}
\def\BKM@DefGotoNameAction#1#2{%
  \BKM@EscapeString\BKM@DO@dest
  \edef#1{goto name{#2}}%
}
%    \end{macrocode}
%    \end{macro}
%    \begin{macrocode}
%</pdftex>
%    \end{macrocode}
%
%    \begin{macrocode}
%<*pdftex|pdfmark>
%    \end{macrocode}
%    \begin{macro}{\BKM@SourceLocation}
%    \begin{macrocode}
\def\BKM@SourceLocation{%
  \ifx\BKM@DO@srcfile\@empty
    \ifx\BKM@DO@srcline\@empty
    \else
      .\MessageBreak
      Source: line \BKM@DO@srcline
    \fi
  \else
    \ifx\BKM@DO@srcline\@empty
      .\MessageBreak
      Source: file `\BKM@DO@srcfile'%
    \else
      .\MessageBreak
      Source: file `\BKM@DO@srcfile', line \BKM@DO@srcline
    \fi
  \fi
}
%    \end{macrocode}
%    \end{macro}
%    \begin{macrocode}
%</pdftex|pdfmark>
%    \end{macrocode}
%
% \subsection{具有 pdfmark 特色(specials)的驱动程序}
%
% \subsubsection{dvips 驱动程序}
%
%    \begin{macrocode}
%<*dvips>
\NeedsTeXFormat{LaTeX2e}
\ProvidesFile{bkm-dvips.def}%
  [2020-11-06 v1.29 bookmark driver for dvips (HO)]%
%    \end{macrocode}
%    \begin{macro}{\BKM@PSHeaderFile}
%    \begin{macrocode}
\def\BKM@PSHeaderFile#1{%
  \special{PSfile=#1}%
}
%    \end{macrocode}
%    \begin{macro}{\BKM@filename}
%    \begin{macrocode}
\def\BKM@filename{\jobname.out.ps}
%    \end{macrocode}
%    \end{macro}
%    \begin{macrocode}
\AddToHook{shipout/lastpage}{%
  \BKM@pdfmark@out
  \BKM@PSHeaderFile\BKM@filename
  }
%    \end{macrocode}
%    \end{macro}
%    \begin{macrocode}
%</dvips>
%    \end{macrocode}
%
% \subsubsection{公共部分(Common part)}
%
%    \begin{macrocode}
%<*pdfmark>
%    \end{macrocode}
%
%    \begin{macro}{\BKM@pdfmark@out}
%    不要在这里使用 \xpackage{rerunfilecheck}\ 宏包,因为在 \hologo{TeX}\ 运行期间不会
%    读取 \cs{BKM@filename}\ 文件。
%    \begin{macrocode}
\def\BKM@pdfmark@out{%
  \if@filesw
    \newwrite\BKM@file
    \immediate\openout\BKM@file=\BKM@filename\relax
    \BKM@write{\@percentchar!}%
    \BKM@write{/pdfmark where{pop}}%
    \BKM@write{%
      {%
        /globaldict where{pop globaldict}{userdict}ifelse%
        /pdfmark/cleartomark load put%
      }%
    }%
    \BKM@write{ifelse}%
  \else
    \let\BKM@write\@gobble
    \let\BKM@DO@entry\@gobbletwo
  \fi
}
%    \end{macrocode}
%    \end{macro}
%    \begin{macro}{\BKM@write}
%    \begin{macrocode}
\def\BKM@write#{%
  \immediate\write\BKM@file
}
%    \end{macrocode}
%    \end{macro}
%
%    \begin{macro}{\BKM@DO@entry}
%    Pdfmark 的规范(specification)说明 |/Color| 是颜色(color)的键名(key name),
%    但是 ghostscript 只将键(key)传递到 PDF 文件中,因此键名必须是 |/C|。
%    \begin{macrocode}
\def\BKM@DO@entry#1#2{%
  \begingroup
    \kvsetkeys{BKM@DO}{#1}%
    \ifx\BKM@DO@srcfile\@empty
    \else
      \BKM@UnescapeHex\BKM@DO@srcfile
    \fi
    \def\BKM@DO@title{#2}%
    \BKM@UnescapeHex\BKM@DO@title
    \expandafter\expandafter\expandafter\BKM@getx
        \csname BKM@\BKM@DO@id\endcsname\@empty\@empty
    \let\BKM@attr\@empty
    \ifx\BKM@DO@flags\@empty
    \else
      \edef\BKM@attr{\BKM@attr/F \BKM@DO@flags}%
    \fi
    \ifx\BKM@DO@color\@empty
    \else
      \edef\BKM@attr{\BKM@attr/C[\BKM@DO@color]}%
    \fi
    \let\BKM@action\@empty
    \ifx\BKM@DO@gotor\@empty
      \ifx\BKM@DO@dest\@empty
        \ifx\BKM@DO@named\@empty
          \ifx\BKM@DO@rawaction\@empty
            \ifx\BKM@DO@uri\@empty
              \ifx\BKM@DO@page\@empty
                \PackageError{bookmark}{%
                  Missing action\BKM@SourceLocation
                }\@ehc
                \edef\BKM@action{%
                  /Action/GoTo%
                  /Page 1%
                  /View[/Fit]%
                }%
              \else
                \ifx\BKM@DO@view\@empty
                  \def\BKM@DO@view{Fit}%
                \fi
                \edef\BKM@action{%
                  /Action/GoTo%
                  /Page \BKM@DO@page
                  /View[/\BKM@DO@view]%
                }%
              \fi
            \else
              \BKM@UnescapeHex\BKM@DO@uri
              \BKM@EscapeString\BKM@DO@uri
              \edef\BKM@action{%
                /Action<<%
                  /Subtype/URI%
                  /URI(\BKM@DO@uri)%
                >>%
              }%
            \fi
          \else
            \BKM@UnescapeHex\BKM@DO@rawaction
            \edef\BKM@action{%
              /Action<<%
                \BKM@DO@rawaction
              >>%
            }%
          \fi
        \else
          \BKM@EscapeName\BKM@DO@named
          \edef\BKM@action{%
            /Action<<%
              /Subtype/Named%
              /N/\BKM@DO@named
            >>%
          }%
        \fi
      \else
        \BKM@UnescapeHex\BKM@DO@dest
        \BKM@EscapeString\BKM@DO@dest
        \edef\BKM@action{%
          /Action/GoTo%
          /Dest(\BKM@DO@dest)cvn%
        }%
      \fi
    \else
      \ifx\BKM@DO@dest\@empty
        \ifx\BKM@DO@page\@empty
          \def\BKM@DO@page{1}%
        \fi
        \ifx\BKM@DO@view\@empty
          \def\BKM@DO@view{Fit}%
        \fi
        \edef\BKM@action{%
          /Page \BKM@DO@page
          /View[/\BKM@DO@view]%
        }%
      \else
        \BKM@UnescapeHex\BKM@DO@dest
        \BKM@EscapeString\BKM@DO@dest
        \edef\BKM@action{%
          /Dest(\BKM@DO@dest)cvn%
        }%
      \fi
      \BKM@UnescapeHex\BKM@DO@gotor
      \BKM@EscapeString\BKM@DO@gotor
      \edef\BKM@action{%
        /Action/GoToR%
        /File(\BKM@DO@gotor)%
        \BKM@action
      }%
    \fi
    \BKM@write{[}%
    \BKM@write{/Title(\BKM@DO@title)}%
    \ifnum\BKM@x@childs>\z@
      \BKM@write{/Count \ifBKM@DO@open\else-\fi\BKM@x@childs}%
    \fi
    \ifx\BKM@attr\@empty
    \else
      \BKM@write{\BKM@attr}%
    \fi
    \BKM@write{\BKM@action}%
    \BKM@write{/OUT pdfmark}%
  \endgroup
}
%    \end{macrocode}
%    \end{macro}
%    \begin{macrocode}
%</pdfmark>
%    \end{macrocode}
%
% \subsection{\xoption{pdftex}\ 和 \xoption{pdfmark}\ 的公共部分}
%
%    \begin{macrocode}
%<*pdftex|pdfmark>
%    \end{macrocode}
%
% \subsubsection{写入辅助文件(auxiliary file)}
%
%    \begin{macrocode}
\AddToHook{begindocument}{%
 \immediate\write\@mainaux{\string\providecommand\string\BKM@entry[2]{}}}
%    \end{macrocode}
%
%    \begin{macro}{\BKM@id}
%    \begin{macrocode}
\newcount\BKM@id
\BKM@id=\z@
%    \end{macrocode}
%    \end{macro}
%
%    \begin{macro}{\BKM@0}
%    \begin{macrocode}
\@namedef{BKM@0}{000}
%    \end{macrocode}
%    \end{macro}
%    \begin{macro}{\ifBKM@sw}
%    \begin{macrocode}
\newif\ifBKM@sw
%    \end{macrocode}
%    \end{macro}
%
%    \begin{macro}{\bookmark}
%    \begin{macrocode}
\newcommand*{\bookmark}[2][]{%
  \if@filesw
    \begingroup
      \BKM@InitSourceLocation
      \def\bookmark@text{#2}%
      \BKM@setup{#1}%
      \ifx\BKM@srcfile\@empty
      \else
        \BKM@EscapeHex\BKM@srcfile
      \fi
      \edef\BKM@prev{\the\BKM@id}%
      \global\advance\BKM@id\@ne
      \BKM@swtrue
      \@whilesw\ifBKM@sw\fi{%
        \ifnum\ifBKM@startatroot\z@\else\BKM@prev\fi=\z@
          \BKM@startatrootfalse
          \expandafter\xdef\csname BKM@\the\BKM@id\endcsname{%
            0{\BKM@level}0%
          }%
          \BKM@swfalse
        \else
          \expandafter\expandafter\expandafter\BKM@getx
              \csname BKM@\BKM@prev\endcsname
          \ifnum\BKM@level>\BKM@x@level\relax
            \expandafter\xdef\csname BKM@\the\BKM@id\endcsname{%
              {\BKM@prev}{\BKM@level}0%
            }%
            \ifnum\BKM@prev>\z@
              \BKM@CalcExpr\BKM@CalcResult\BKM@x@childs+1%
              \expandafter\xdef\csname BKM@\BKM@prev\endcsname{%
                {\BKM@x@parent}{\BKM@x@level}{\BKM@CalcResult}%
              }%
            \fi
            \BKM@swfalse
          \else
            \let\BKM@prev\BKM@x@parent
          \fi
        \fi
      }%
      \pdfstringdef\BKM@title{\bookmark@text}%
      \edef\BKM@FLAGS{\BKM@PrintStyle}%
      \csname BKM@HypDestOptHook\endcsname
      \BKM@EscapeHex\BKM@dest
      \BKM@EscapeHex\BKM@uri
      \BKM@EscapeHex\BKM@gotor
      \BKM@EscapeHex\BKM@rawaction
      \BKM@EscapeHex\BKM@title
      \immediate\write\@mainaux{%
        \string\BKM@entry{%
          id=\number\BKM@id
          \ifBKM@open
            \ifnum\BKM@level<\BKM@openlevel
              ,open%
            \fi
          \fi
          \BKM@auxentry{dest}%
          \BKM@auxentry{named}%
          \BKM@auxentry{uri}%
          \BKM@auxentry{gotor}%
          \BKM@auxentry{page}%
          \BKM@auxentry{view}%
          \BKM@auxentry{rawaction}%
          \BKM@auxentry{color}%
          \ifnum\BKM@FLAGS>\z@
            ,flags=\BKM@FLAGS
          \fi
          \BKM@auxentry{srcline}%
          \BKM@auxentry{srcfile}%
        }{\BKM@title}%
      }%
    \endgroup
  \fi
}
%    \end{macrocode}
%    \end{macro}
%    \begin{macro}{\BKM@getx}
%    \begin{macrocode}
\def\BKM@getx#1#2#3{%
  \def\BKM@x@parent{#1}%
  \def\BKM@x@level{#2}%
  \def\BKM@x@childs{#3}%
}
%    \end{macrocode}
%    \end{macro}
%    \begin{macro}{\BKM@auxentry}
%    \begin{macrocode}
\def\BKM@auxentry#1{%
  \expandafter\ifx\csname BKM@#1\endcsname\@empty
  \else
    ,#1={\csname BKM@#1\endcsname}%
  \fi
}
%    \end{macrocode}
%    \end{macro}
%
%    \begin{macro}{\BKM@InitSourceLocation}
%    \begin{macrocode}
\def\BKM@InitSourceLocation{%
  \edef\BKM@srcline{\the\inputlineno}%
  \BKM@LuaTeX@InitFile
  \ifx\BKM@srcfile\@empty
    \ltx@IfUndefined{currfilepath}{}{%
      \edef\BKM@srcfile{\currfilepath}%
    }%
  \fi
}
%    \end{macrocode}
%    \end{macro}
%    \begin{macro}{\BKM@LuaTeX@InitFile}
%    \begin{macrocode}
\ifluatex
  \ifnum\luatexversion>36 %
    \def\BKM@LuaTeX@InitFile{%
      \begingroup
        \ltx@LocToksA={}%
      \edef\x{\endgroup
        \def\noexpand\BKM@srcfile{%
          \the\expandafter\ltx@LocToksA
          \directlua{%
             if status and status.filename then %
               tex.settoks('ltx@LocToksA', status.filename)%
             end%
          }%
        }%
      }\x
    }%
  \else
    \let\BKM@LuaTeX@InitFile\relax
  \fi
\else
  \let\BKM@LuaTeX@InitFile\relax
\fi
%    \end{macrocode}
%    \end{macro}
%
% \subsubsection{读取辅助数据(auxiliary data)}
%
%    \begin{macrocode}
\SetupKeyvalOptions{family=BKM@DO,prefix=BKM@DO@}
\DeclareStringOption[0]{id}
\DeclareBoolOption{open}
\DeclareStringOption{flags}
\DeclareStringOption{color}
\DeclareStringOption{dest}
\DeclareStringOption{named}
\DeclareStringOption{uri}
\DeclareStringOption{gotor}
\DeclareStringOption{page}
\DeclareStringOption{view}
\DeclareStringOption{rawaction}
\DeclareStringOption{srcline}
\DeclareStringOption{srcfile}
%    \end{macrocode}
%
%    \begin{macrocode}
\AtBeginDocument{%
  \let\BKM@entry\BKM@DO@entry
}
%    \end{macrocode}
%
%    \begin{macrocode}
%</pdftex|pdfmark>
%    \end{macrocode}
%
% \subsection{\xoption{atend}\ 选项}
%
% \subsubsection{钩子(Hook)}
%
%    \begin{macrocode}
%<*package>
%    \end{macrocode}
%    \begin{macrocode}
\ifBKM@atend
\else
%    \end{macrocode}
%    \begin{macro}{\BookmarkAtEnd}
%    这是一个虚拟定义(dummy definition),如果没有给出 \xoption{atend}\ 选项,它将生成一个警告。
%    \begin{macrocode}
  \newcommand{\BookmarkAtEnd}[1]{%
    \PackageWarning{bookmark}{%
      Ignored, because option `atend' is missing%
    }%
  }%
%    \end{macrocode}
%    \end{macro}
%    \begin{macrocode}
  \expandafter\endinput
\fi
%    \end{macrocode}
%    \begin{macro}{\BookmarkAtEnd}
%    \begin{macrocode}
\newcommand*{\BookmarkAtEnd}{%
  \g@addto@macro\BKM@EndHook
}
%    \end{macrocode}
%    \end{macro}
%    \begin{macrocode}
\let\BKM@EndHook\@empty
%    \end{macrocode}
%    \begin{macrocode}
%</package>
%    \end{macrocode}
%
% \subsubsection{在文档末尾使用钩子的驱动程序}
%
%    驱动程序 \xoption{pdftex}\ 使用 LaTeX 钩子 \xoption{enddocument/afterlastpage}
%    (相当于以前使用的 \xpackage{atveryend}\ 的 \cs{AfterLastShipout}),因为它仍然需要 \xext{aux}\ 文件。
%    它使用 \cs{pdfoutline}\ 作为最后一页之后可以使用的书签(bookmakrs)。
%    \begin{itemize}
%    \item
%      驱动程序 \xoption{pdftex}\ 使用 \cs{pdfoutline}, \cs{pdfoutline}\ 可以在最后一页之后使用。
%    \end{itemize}
%    \begin{macrocode}
%<*pdftex>
\ifBKM@atend
  \AddToHook{enddocument/afterlastpage}{%
    \BKM@EndHook
  }%
\fi
%</pdftex>
%    \end{macrocode}
%
% \subsubsection{使用 \xoption{shipout/lastpage}\ 的驱动程序}
%
%    其他驱动程序使用 \cs{special}\ 命令实现 \cs{bookmark}。因此,最后的书签(last bookmarks)
%    必须放在最后一页(last page),而不是之后。不能使用 \cs{AtEndDocument},因为为时已晚,
%    最后一页已经输出了。因此,我们使用 LaTeX 钩子 \xoption{shipout/lastpage}。至少需要运行
%    两次 \hologo{LaTeX}。PostScript 驱动程序 \xoption{dvips}\ 使用外部 PostScript 文件作为书签。
%    为了避免与 pgf 发生冲突,文件写入(file writing)也被移到了最后一个输出页面(shipout page)。
%    \begin{macrocode}
%<*dvipdfm|vtex|pdfmark>
\ifBKM@atend
  \AddToHook{shipout/lastpage}{\BKM@EndHook}%
\fi
%</dvipdfm|vtex|pdfmark>
%    \end{macrocode}
%
% \section{安装(Installation)}
%
% \subsection{下载(Download)}
%
% \paragraph{宏包(Package)。} 在 CTAN\footnote{\CTANpkg{bookmark}}上提供此宏包:
% \begin{description}
% \item[\CTAN{macros/latex/contrib/bookmark/bookmark.dtx}] 源文件(source file)。
% \item[\CTAN{macros/latex/contrib/bookmark/bookmark.pdf}] 文档(documentation)。
% \end{description}
%
%
% \paragraph{捆绑包(Bundle)。} “bookmark”捆绑包(bundle)的所有宏包(packages)都可以在兼
% 容 TDS 的 ZIP 归档文件中找到。在那里,宏包已经被解包,文档文件(documentation files)已经生成。
% 文件(files)和目录(directories)遵循 TDS 标准。
% \begin{description}
% \item[\CTANinstall{install/macros/latex/contrib/bookmark.tds.zip}]
% \end{description}
% \emph{TDS}\ 是指标准的“用于 \TeX\ 文件的目录结构(Directory Structure)”(\CTANpkg{tds})。
% 名称中带有 \xfile{texmf}\ 的目录(directories)通常以这种方式组织。
%
% \subsection{捆绑包(Bundle)的安装}
%
% \paragraph{解压(Unpacking)。} 在您选择的 TDS 树(也称为 \xfile{texmf}\ 树)中解
% 压 \xfile{bookmark.tds.zip},例如(在 linux 中):
% \begin{quote}
%   |unzip bookmark.tds.zip -d ~/texmf|
% \end{quote}
%
% \subsection{宏包(Package)的安装}
%
% \paragraph{解压(Unpacking)。} \xfile{.dtx}\ 文件是一个自解压 \docstrip\ 归档文件(archive)。
% 这些文件是通过 \plainTeX\ 运行 \xfile{.dtx}\ 来提取的:
% \begin{quote}
%   \verb|tex bookmark.dtx|
% \end{quote}
%
% \paragraph{TDS.} 现在,不同的文件必须移动到安装 TDS 树(installation TDS tree)
% (也称为 \xfile{texmf}\ 树)中的不同目录中:
% \begin{quote}
% \def\t{^^A
% \begin{tabular}{@{}>{\ttfamily}l@{ $\rightarrow$ }>{\ttfamily}l@{}}
%   bookmark.sty & tex/latex/bookmark/bookmark.sty\\
%   bkm-dvipdfm.def & tex/latex/bookmark/bkm-dvipdfm.def\\
%   bkm-dvips.def & tex/latex/bookmark/bkm-dvips.def\\
%   bkm-pdftex.def & tex/latex/bookmark/bkm-pdftex.def\\
%   bkm-vtex.def & tex/latex/bookmark/bkm-vtex.def\\
%   bookmark.pdf & doc/latex/bookmark/bookmark.pdf\\
%   bookmark-example.tex & doc/latex/bookmark/bookmark-example.tex\\
%   bookmark.dtx & source/latex/bookmark/bookmark.dtx\\
% \end{tabular}^^A
% }^^A
% \sbox0{\t}^^A
% \ifdim\wd0>\linewidth
%   \begingroup
%     \advance\linewidth by\leftmargin
%     \advance\linewidth by\rightmargin
%   \edef\x{\endgroup
%     \def\noexpand\lw{\the\linewidth}^^A
%   }\x
%   \def\lwbox{^^A
%     \leavevmode
%     \hbox to \linewidth{^^A
%       \kern-\leftmargin\relax
%       \hss
%       \usebox0
%       \hss
%       \kern-\rightmargin\relax
%     }^^A
%   }^^A
%   \ifdim\wd0>\lw
%     \sbox0{\small\t}^^A
%     \ifdim\wd0>\linewidth
%       \ifdim\wd0>\lw
%         \sbox0{\footnotesize\t}^^A
%         \ifdim\wd0>\linewidth
%           \ifdim\wd0>\lw
%             \sbox0{\scriptsize\t}^^A
%             \ifdim\wd0>\linewidth
%               \ifdim\wd0>\lw
%                 \sbox0{\tiny\t}^^A
%                 \ifdim\wd0>\linewidth
%                   \lwbox
%                 \else
%                   \usebox0
%                 \fi
%               \else
%                 \lwbox
%               \fi
%             \else
%               \usebox0
%             \fi
%           \else
%             \lwbox
%           \fi
%         \else
%           \usebox0
%         \fi
%       \else
%         \lwbox
%       \fi
%     \else
%       \usebox0
%     \fi
%   \else
%     \lwbox
%   \fi
% \else
%   \usebox0
% \fi
% \end{quote}
% 如果你有一个 \xfile{docstrip.cfg}\ 文件,该文件能配置并启用 \docstrip\ 的 TDS 安装功能,
% 则一些文件可能已经在正确的位置了,请参阅 \docstrip\ 的文档(documentation)。
%
% \subsection{刷新文件名数据库}
%
% 如果您的 \TeX~发行版(\TeX\,Live、\mikTeX、\dots)依赖于文件名数据库(file name databases),
% 则必须刷新这些文件名数据库。例如,\TeX\,Live\ 用户运行 \verb|texhash| 或 \verb|mktexlsr|。
%
% \subsection{一些感兴趣的细节}
%
% \paragraph{用 \LaTeX\ 解压。}
% \xfile{.dtx}\ 根据格式(format)选择其操作(action):
% \begin{description}
% \item[\plainTeX:] 运行 \docstrip\ 并解压文件。
% \item[\LaTeX:] 生成文档。
% \end{description}
% 如果您坚持通过 \LaTeX\ 使用\docstrip (实际上 \docstrip\ 并不需要 \LaTeX),那么请您的意图告知自动检测程序:
% \begin{quote}
%   \verb|latex \let\install=y% \iffalse meta-comment
%
% File: bookmark.dtx
% Version: 2020-11-06 v1.29
% Info: PDF bookmarks
%
% Copyright (C)
%    2007-2011 Heiko Oberdiek
%    2016-2020 Oberdiek Package Support Group
%    https://github.com/ho-tex/bookmark/issues
%
% This work may be distributed and/or modified under the
% conditions of the LaTeX Project Public License, either
% version 1.3c of this license or (at your option) any later
% version. This version of this license is in
%    https://www.latex-project.org/lppl/lppl-1-3c.txt
% and the latest version of this license is in
%    https://www.latex-project.org/lppl.txt
% and version 1.3 or later is part of all distributions of
% LaTeX version 2005/12/01 or later.
%
% This work has the LPPL maintenance status "maintained".
%
% The Current Maintainers of this work are
% Heiko Oberdiek and the Oberdiek Package Support Group
% https://github.com/ho-tex/bookmark/issues
%
% This work consists of the main source file bookmark.dtx
% and the derived files
%    bookmark.sty, bookmark.pdf, bookmark.ins, bookmark.drv,
%    bkm-dvipdfm.def, bkm-dvips.def,
%    bkm-pdftex.def, bkm-vtex.def,
%    bkm-dvipdfm-2019-12-03.def, bkm-dvips-2019-12-03.def,
%    bkm-pdftex-2019-12-03.def, bkm-vtex-2019-12-03.def,
%    bookmark-example.tex.
%
% Distribution:
%    CTAN:macros/latex/contrib/bookmark/bookmark.dtx
%    CTAN:macros/latex/contrib/bookmark/bookmark-frozen.dtx
%    CTAN:macros/latex/contrib/bookmark/bookmark.pdf
%
% Unpacking:
%    (a) If bookmark.ins is present:
%           tex bookmark.ins
%    (b) Without bookmark.ins:
%           tex bookmark.dtx
%    (c) If you insist on using LaTeX
%           latex \let\install=y\input{bookmark.dtx}
%        (quote the arguments according to the demands of your shell)
%
% Documentation:
%    (a) If bookmark.drv is present:
%           latex bookmark.drv
%    (b) Without bookmark.drv:
%           latex bookmark.dtx; ...
%    The class ltxdoc loads the configuration file ltxdoc.cfg
%    if available. Here you can specify further options, e.g.
%    use A4 as paper format:
%       \PassOptionsToClass{a4paper}{article}
%
%    Programm calls to get the documentation (example):
%       pdflatex bookmark.dtx
%       makeindex -s gind.ist bookmark.idx
%       pdflatex bookmark.dtx
%       makeindex -s gind.ist bookmark.idx
%       pdflatex bookmark.dtx
%
% Installation:
%    TDS:tex/latex/bookmark/bookmark.sty
%    TDS:tex/latex/bookmark/bkm-dvipdfm.def
%    TDS:tex/latex/bookmark/bkm-dvips.def
%    TDS:tex/latex/bookmark/bkm-pdftex.def
%    TDS:tex/latex/bookmark/bkm-vtex.def
%    TDS:tex/latex/bookmark/bkm-dvipdfm-2019-12-03.def
%    TDS:tex/latex/bookmark/bkm-dvips-2019-12-03.def
%    TDS:tex/latex/bookmark/bkm-pdftex-2019-12-03.def
%    TDS:tex/latex/bookmark/bkm-vtex-2019-12-03.def%
%    TDS:doc/latex/bookmark/bookmark.pdf
%    TDS:doc/latex/bookmark/bookmark-example.tex
%    TDS:source/latex/bookmark/bookmark.dtx
%    TDS:source/latex/bookmark/bookmark-frozen.dtx
%
%<*ignore>
\begingroup
  \catcode123=1 %
  \catcode125=2 %
  \def\x{LaTeX2e}%
\expandafter\endgroup
\ifcase 0\ifx\install y1\fi\expandafter
         \ifx\csname processbatchFile\endcsname\relax\else1\fi
         \ifx\fmtname\x\else 1\fi\relax
\else\csname fi\endcsname
%</ignore>
%<*install>
\input docstrip.tex
\Msg{************************************************************************}
\Msg{* Installation}
\Msg{* Package: bookmark 2020-11-06 v1.29 PDF bookmarks (HO)}
\Msg{************************************************************************}

\keepsilent
\askforoverwritefalse

\let\MetaPrefix\relax
\preamble

This is a generated file.

Project: bookmark
Version: 2020-11-06 v1.29

Copyright (C)
   2007-2011 Heiko Oberdiek
   2016-2020 Oberdiek Package Support Group

This work may be distributed and/or modified under the
conditions of the LaTeX Project Public License, either
version 1.3c of this license or (at your option) any later
version. This version of this license is in
   https://www.latex-project.org/lppl/lppl-1-3c.txt
and the latest version of this license is in
   https://www.latex-project.org/lppl.txt
and version 1.3 or later is part of all distributions of
LaTeX version 2005/12/01 or later.

This work has the LPPL maintenance status "maintained".

The Current Maintainers of this work are
Heiko Oberdiek and the Oberdiek Package Support Group
https://github.com/ho-tex/bookmark/issues


This work consists of the main source file bookmark.dtx and bookmark-frozen.dtx
and the derived files
   bookmark.sty, bookmark.pdf, bookmark.ins, bookmark.drv,
   bkm-dvipdfm.def, bkm-dvips.def, bkm-pdftex.def, bkm-vtex.def,
   bkm-dvipdfm-2019-12-03.def, bkm-dvips-2019-12-03.def,
   bkm-pdftex-2019-12-03.def, bkm-vtex-2019-12-03.def,
   bookmark-example.tex.

\endpreamble
\let\MetaPrefix\DoubleperCent

\generate{%
  \file{bookmark.ins}{\from{bookmark.dtx}{install}}%
  \file{bookmark.drv}{\from{bookmark.dtx}{driver}}%
  \usedir{tex/latex/bookmark}%
  \file{bookmark.sty}{\from{bookmark.dtx}{package}}%
  \file{bkm-dvipdfm.def}{\from{bookmark.dtx}{dvipdfm}}%
  \file{bkm-dvips.def}{\from{bookmark.dtx}{dvips,pdfmark}}%
  \file{bkm-pdftex.def}{\from{bookmark.dtx}{pdftex}}%
  \file{bkm-vtex.def}{\from{bookmark.dtx}{vtex}}%
  \usedir{doc/latex/bookmark}%
  \file{bookmark-example.tex}{\from{bookmark.dtx}{example}}%
  \file{bkm-pdftex-2019-12-03.def}{\from{bookmark-frozen.dtx}{pdftexfrozen}}%
  \file{bkm-dvips-2019-12-03.def}{\from{bookmark-frozen.dtx}{dvipsfrozen}}%
  \file{bkm-vtex-2019-12-03.def}{\from{bookmark-frozen.dtx}{vtexfrozen}}%
  \file{bkm-dvipdfm-2019-12-03.def}{\from{bookmark-frozen.dtx}{dvipdfmfrozen}}%
}

\catcode32=13\relax% active space
\let =\space%
\Msg{************************************************************************}
\Msg{*}
\Msg{* To finish the installation you have to move the following}
\Msg{* files into a directory searched by TeX:}
\Msg{*}
\Msg{*     bookmark.sty, bkm-dvipdfm.def, bkm-dvips.def,}
\Msg{*     bkm-pdftex.def, bkm-vtex.def, bkm-dvipdfm-2019-12-03.def,}
\Msg{*     bkm-dvips-2019-12-03.def, bkm-pdftex-2019-12-03.def,}
\Msg{*     and bkm-vtex-2019-12-03.def}
\Msg{*}
\Msg{* To produce the documentation run the file `bookmark.drv'}
\Msg{* through LaTeX.}
\Msg{*}
\Msg{* Happy TeXing!}
\Msg{*}
\Msg{************************************************************************}

\endbatchfile
%</install>
%<*ignore>
\fi
%</ignore>
%<*driver>
\NeedsTeXFormat{LaTeX2e}
\ProvidesFile{bookmark.drv}%
  [2020-11-06 v1.29 PDF bookmarks (HO)]%
\documentclass{ltxdoc}
\usepackage{ctex}
\usepackage{indentfirst}
\setlength{\parindent}{2em}
\usepackage{holtxdoc}[2011/11/22]
\usepackage{xcolor}
\usepackage{hyperref}
\usepackage[open,openlevel=3,atend]{bookmark}[2020/11/06] %%%打开书签,显示的深度为3级,即显示part、section、subsection。
\bookmarksetup{color=red}
\begin{document}

  \renewcommand{\contentsname}{目\quad 录}
  \renewcommand{\abstractname}{摘\quad 要}
  \renewcommand{\historyname}{历史}
  \DocInput{bookmark.dtx}%
\end{document}
%</driver>
% \fi
%
%
%
% \GetFileInfo{bookmark.drv}
%
%% \title{\xpackage{bookmark} 宏包}
% \title{\heiti {\Huge \textbf{\xpackage{bookmark}\ 宏包}}}
% \date{2020-11-06\ \ \ v1.29}
% \author{Heiko Oberdiek \thanks
% {如有问题请点击:\url{https://github.com/ho-tex/bookmark/issues}}\\[5pt]赣医一附院神经科\ \ 黄旭华\ \ \ \ 译}
%
% \maketitle
%
% \begin{abstract}
% 这个宏包为 \xpackage{hyperref}\ 宏包实现了一个新的书签(bookmark)(大纲[outline])组织。现在
% 可以设置样式(style)和颜色(color)等书签属性(bookmark properties)。其他动作类型(action types)可用
% (URI、GoToR、Named)。书签是在第一次编译运行(compile run)中生成的。\xpackage{hyperref}\
% 宏包必需运行两次。
% \end{abstract}
%
% \tableofcontents
%
% \section{文档(Documentation)}
%
% \subsection{介绍}
%
% 这个 \xpackage{bookmark}\ 宏包试图为书签(bookmarks)提供一个更现代的管理:
% \begin{itemize}
% \item 书签已经在第一次 \hologo{TeX}\ 编译运行(compile run)中生成。
% \item 可以更改书签的字体样式(font style)和颜色(color)。
% \item 可以执行比简单的 GoTo 操作(actions)更多的操作。
% \end{itemize}
%
% 与 \xpackage{hyperref} \cite{hyperref} 一样,书签(bookmarks)也是按照书签生成宏
% (bookmark generating macros)(\cs{bookmark})的顺序生成的。级别号(level number)用于
% 定义书签的树结构(tree structure)。限制没有那么严格:
% \begin{itemize}
% \item 级别值(level values)可以跳变(jump)和省略(omit)。\cs{subsubsection}\ 可以跟在
%       \cs{chapter}\ 之后。这种情况如在 \xpackage{hyperref}\ 中则产生错误,它将显示一个警告(warning)
%       并尝试修复此错误。
% \item 多个书签可能指向同一目标(destination)。在 \xpackage{hyperref}\ 中,这会完全弄乱
%       书签树(bookmark tree),因为算法假设(algorithm assumes)目标名称(destination names)
%       是键(keys)(唯一的)。
% \end{itemize}
%
% 注意,这个宏包是作为书签管理(bookmark management)的实验平台(experimentation platform)。
% 欢迎反馈。此外,在未来的版本中,接口(interfaces)也可能发生变化。
%
% \subsection{选项(Options)}
%
% 可在以下四个地方放置选项(options):
% \begin{enumerate}
% \item \cs{usepackage}|[|\meta{options}|]{bookmark}|\\
%       这是放置驱动程序选项(driver options)和 \xoption{atend}\ 选项的唯一位置。
% \item \cs{bookmarksetup}|{|\meta{options}|}|\\
%       此命令仅用于设置选项(setting options)。
% \item \cs{bookmarksetupnext}|{|\meta{options}|}|\\
%       这些选项在下一个 \cs{bookmark}\ 命令的选项之后存储(stored)和调用(called)。
% \item \cs{bookmark}|[|\meta{options}|]{|\meta{title}|}|\\
%       此命令设置书签。选项设置(option settings)仅限于此书签。
% \end{enumerate}
% 异常(Exception):加载该宏包后,无法更改驱动程序选项(Driver options)、\xoption{atend}\ 选项
% 、\xoption{draft}\slash\xoption{final}选项。
%
% \subsubsection{\xoption{draft} 和 \xoption{final}\ 选项}
%
% 如果一个\LaTeX\ 文件要被编译了多次,那么可以使用 \xoption{draft}\ 选项来禁用该宏包的书签内
% 容(bookmark stuff),这样可以节省一点时间。默认 \xoption{final}\ 选项。两个选项都是
% 布尔选项(boolean options),如果没有值,则使用值 |true|。|draft=true| 与 |final=false| 相同。
%
% 除了驱动程序选项(driver options)之外,\xpackage{bookmark}\ 宏包选项都是局部选项(local options)。
% \xoption{draft}\ 选项和 \xoption{final}\ 选项均属于文档类选项(class option)(译者注:文档类选项为全局选项),
% 因此,在 \xpackage{bookmark}\ 宏包中未能看到这两个选项。如果您想使用全局的(global) \xoption{draft}选项
% 来优化第一次 \LaTeX\ 运行(runs),可以在导言(preamble)中引入 \xpackage{ifdraft}\ 宏包并设置 \LaTeX\ 的
% \cs{PassOptionsToPackage},例如:
%\begin{quote}
%\begin{verbatim}
%\documentclass[draft]{article}
%\usepackage{ifdraft}
%\ifdraft{%
%   \PassOptionsToPackage{draft}{bookmark}%
%}{}
%\end{verbatim}
%\end{quote}
%
% \subsubsection{驱动程序选项(Driver options)}
%
% 支持的驱动程序( drivers)包括 \xoption{pdftex}、\xoption{dvips}、\xoption{dvipdfm} (\xoption{xetex})、
% \xoption{vtex}。\hologo{TeX}\ 引擎 \hologo{pdfTeX}、\hologo{XeTeX}、\hologo{VTeX}\ 能被自动检测到。
% 默认的 DVI 驱动程序是 \xoption{dvips}。这可以通过 \cs{BookmarkDriverDefault}\ 在配置
% 文件 \xfile{bookmark.cfg}\ 中进行更改,例如:
% \begin{quote}
% |\def\BookmarkDriverDefault{dvipdfm}|
% \end{quote}
% 当前版本的(current versions)驱动程序使用新的 \LaTeX\ 钩子(\LaTeX-hooks)。如果检测到比
% 2020-10-01 更旧的格式,则将以前驱动程序的冻结版本(frozen versions)作为备份(fallback)。
%
% \paragraph{用 dvipdfmx 打开书签(bookmarks)。}旧版本的宏包有一个 \xoption{dvipdfmx-outline-open}\ 选项
% 可以激活代码,而该代码可以指定一个大纲条目(outline entry)是否打开。该宏包现在假设所有使用的 dvipdfmx 版本都是
% 最新版本,足以理解该代码,因此始终激活该代码。选项本身将被忽略。
%
%
% \subsubsection{布局选项(Layout options)}
%
% \paragraph{字体(Font)选项:}
%
% \begin{description}
% \item[\xoption{bold}:] 如果受 PDF 浏览器(PDF viewer)支持,书签将以粗体字体(bold font)显示(自 PDF 1.4起)。
% \item[\xoption{italic}:] 使用斜体字体(italic font)(自 PDF 1.4起)。
% \end{description}
% \xoption{bold}(粗体) 和 \xoption{italic}(斜体)可以同时使用。而 |false| 值(value)禁用字体选项。
%
% \paragraph{颜色(Color)选项:}
%
% 彩色书签(Colored bookmarks)是 PDF 1.4 的一个特性(feature),并非所有的 PDF 浏览器(PDF viewers)都支持彩色书签。
% \begin{description}
% \item[\xoption{color}:] 这里 color(颜色)可以作为 \xpackage{color}\ 宏包或 \xpackage{xcolor}\ 宏包的
% 颜色规范(color specification)给出。空值(empty value)表示未设置颜色属性。如果未加载 \xpackage{xcolor}\ 宏包,
% 能识别的值(recognized values)只有:
%   \begin{itemize}
%   \item 空值(empty value)表示未设置颜色属性,\\
%         例如:|color={}|
%   \item 颜色模型(color model) rgb 的显式颜色规范(explicit color specification),\\
%         例如,红色(red):|color=[rgb]{1,0,0}|
%   \item 颜色模型(color model)灰(gray)的显式颜色规范(explicit color specification),\\
%         例如,深灰色(dark gray):|color=[gray]{0.25}|
%   \end{itemize}
%   请注意,如果加载了 \xpackage{color}\ 宏包,此限制(restriction)也适用。然而,如果加载了 \xpackage{xcolor}\ 宏包,
%   则可以使用所有颜色规范(color specifications)。
% \end{description}
%
% \subsubsection{动作选项(Action options)}
%
% \begin{description}
% \item[\xoption{dest}:] 目的地名称(destination name)。
% \item[\xoption{page}:] 页码(page number),第一页(first page)为 1。
% \item[\xoption{view}:] 浏览规范(view specification),示例如下:\\
%   |view={FitB}|, |view={FitH 842}|, |view={XYZ 0 100 null}|\ \  一些浏览规范参数(view specification parameters)
%   将数字(numbers)视为具有单位 bp 的参数。它们可以作为普通数字(plain numbers)或在 \cs{calc}\ 内部以
%   长度表达式(length expressions)给出。如果加载了 \xpackage{calc}\ 宏包,则支持该宏包的表达式(expressions)。否则,
%   使用 \hologo{eTeX}\ 的 \cs{dimexpr}。例如:\\
%   |view={FitH \calc{\paperheight-\topmargin-1in}}|\\
%   |view={XYZ 0 \calc{\paperheight} null}|\\
%   注意 \cs{calc}\ 不能用于 |XYZ| 的第三个参数,因为该参数是缩放值(zoom value),而不是长度(length)。

% \item[\xoption{named}:] 已命名的动作(Named action)的名称:\\
%   |FirstPage|(第一页),|LastPage|(最后一页),|NextPage|(下一页),|PrevPage|(前一页)
% \item[\xoption{gotor}:] 外部(external) PDF 文件的名称。
% \item[\xoption{uri}:] URI 规范(URI specification)。
% \item[\xoption{rawaction}:] 原始动作规范(raw action specification)。由于这些规范取决于驱动程序(driver),因此不应使用此选项。
% \end{description}
% 通过分析指定的选项来选择书签的适当动作。动作由不同的选项集(sets of options)区分:
% \begin{quote}
 \begin{tabular}{|@{}r|l@{}|}
%   \hline
%   \ \textbf{动作(Action)}\  & \ \textbf{选项(Options)}\ \\ \hline
%   \ \textsf{GoTo}\  &\  \xoption{dest}\ \\ \hline
%   \ \textsf{GoTo}\  & \ \xoption{page} + \xoption{view}\ \\ \hline
%   \ \textsf{GoToR}\  & \ \xoption{gotor} + \xoption{dest}\ \\ \hline
%   \ \textsf{GoToR}\  & \ \xoption{gotor} + \xoption{page} + \xoption{view}\ \ \ \\ \hline
%   \ \textsf{Named}\  &\  \xoption{named}\ \\ \hline
%   \ \textsf{URI}\  & \ \xoption{uri}\ \\ \hline
% \end{tabular}
% \end{quote}
%
% \paragraph{缺少动作(Missing actions)。}
% 如果动作缺少 \xpackage{bookmark}\ 宏包,则抛出错误消息(error message)。根据驱动程序(driver)
% (\xoption{pdftex}、\xoption{dvips}\ 和好友[friends]),宏包在文档末尾很晚才检测到它。
% 自 2011/04/21 v1.21 版本以后,该宏包尝试打印 \cs{bookmark}\ 的相应出现的行号(line number)和文件名(file name)。
% 然而,\hologo{TeX}\ 确实提供了行号,但不幸的是,文件名是一个秘密(secret)。但该宏包有如下获取文件名的方法:
% \begin{itemize}
% \item 如果 \hologo{LuaTeX} (独立于 DVI 或 PDF 模式)正在运行,则自动使用其 |status.filename|。
% \item 宏包的 \cs{currfile} \cite{currfile}\ 重新定义了 \hologo{LaTeX}\ 的内部结构,以跟踪文件名(file name)。
% 如果加载了该宏包,那么它的 \cs{currfilepath}\ 将被检测到并由 \xpackage{bookmark}\ 自动使用。
% \item 可以通过 \cs{bookmarksetup}\ 或 \cs{bookmark}\ 中的 \xoption{scrfile}\ 选项手动设置(set manually)文件名。
% 但是要小心,手动设置会禁用以前的文件名检测方法。错误的(wrong)或丢失的(missed)文件名设置(file name setting)可能会在错误消息中
% 为您提供错误的源位置(source location)。
% \end{itemize}
%
% \subsubsection{级别选项(Level options)}
%
% 书签条目(bookmark entries)的顺序由 \cs{bookmark}\ 命令的的出现顺序(appearance order)定义。
% 树结构(tree structure)由书签节点(bookmark nodes)的属性 \xoption{level}(级别)构建。
% \xoption{level}\ 的值是整数(integers)。如果书签条目级别的值高于前一个节点,则该条目将成为
% 前一个节点的子(child)节点。差值的绝对值并不重要。
%
% \xpackage{bookmark}\ 宏包能记住全局属性(global property)“current level(当前级别)”中上
% 一个书签条目(previous bookmark entry)的级别。
%
% 级别系统的(level system)行为(behaviour)可以通过以下选项进行配置:
% \begin{description}
% \item[\xoption{level}:]
%    设置级别(level),请参阅上面的说明。如果给出的选项 \xoption{level}\ 没有值,那么将恢复默
%    认行为,即将“当前级别(current level)”用作级别值(level value)。自 2010/10/19 v1.16 版本以来,
%    如果宏 \cs{toclevel@part}、\cs{toclevel@section}\ 被定义过(通过 \xpackage{hyperref}\ 宏包完成,
%    请参阅它的 \xoption{bookmarkdepth}\ 选项),则 \xpackage{bookmark}\ 宏包还支持 |part|、|section| 等名称。
%
% \item[\xoption{rellevel}:]
%    设置相对于前一级别的(previous level)级别。正值表示书签条目成为前一个书签条目的子条目。
% \item[\xoption{keeplevel}:]
%    使用由\xoption{level}\ 或 \xoption{rellevel}\ 设置的级别,但不要更改全局属性“current level(当前级别)”。
%    可以通过设置为 |false| 来禁用该选项。
% \item[\xoption{startatroot}:]
%    此时,书签树(bookmark tree)再次从顶层(top level)开始。下一个书签条目不会作为上一个条目的子条目进行排序。
%    示例场景:文档使用 part。但是,最后几章(last chapters)不应放在最后一部分(last part)下面:
%    \begin{quote}
%\begin{verbatim}
%\documentclass{book}
%[...]
%\begin{document}
%  \part{第一部分}
%    \chapter{第一部分的第1章}
%    [...]
%  \part{第二部分(Second part)}
%    \chapter{第二部分的第1章}
%    [...]
%  \bookmarksetup{startatroot}
%  \chapter{Index}% 不属于第二部分
%\end{document}
%\end{verbatim}
%    \end{quote}
% \end{description}
%
% \subsubsection{样式定义(Style definitions)}
%
% 样式(style)是一组选项设置(option settings)。它可以由宏 \cs{bookmarkdefinestyle}\ 定义,
% 并由它的 \xoption{style}\ 选项使用。
% \begin{declcs}{bookmarkdefinestyle} \M{name} \M{key value list}
% \end{declcs}
% 选项设置(option settings)的 \meta{key value list}(键值列表)被指定为样式名(style \meta{name})。
%
% \begin{description}
% \item[\xoption{style}:]
%   \xoption{style}\ 选项的值是以前定义的样式的名称(name)。现在执行其选项设置(option settings)。
%   选项可以包括 \xoption{style}\ 选项。通过递归调用相同样式的无限递归(endless recursion)被阻止并抛出一个错误。
% \end{description}
%
% \subsubsection{钩子支持(Hook support)}
%
% 处理宏\cs{bookmark}\ 的可选选项(optional options)后,就会调用钩子(hook)。
% \begin{description}
% \item[\xoption{addtohook}:]
%   代码(code)作为该选项的值添加到钩子中。
% \end{description}
%
% \begin{declcs}{bookmarkget} \M{option}
% \end{declcs}
% \cs{bookmarkget}\ 宏提取 \meta{option}\ 选项的最新选项设置(latest option setting)的值。
% 对于布尔选项(boolean option),如果启用布尔选项,则返回 1,否则结果为零。结果数字(resulting numbers)
% 可以直接用于 \cs{ifnum}\ 或 \cs{ifcase}。如果您想要数字 \texttt{0}\ 和 \texttt{1},
% 请在 \cs{bookmarkget}\ 前面加上 \cs{number}\ 作为前缀。\cs{bookmarkget}\ 宏是可展开的(expandable)。
% 如果选项不受支持,则返回空字符串(empty string)。受支持的布尔选项有:
% \begin{quote}
%   \xoption{bold}、
%   \xoption{italic}、
%   \xoption{open}
% \end{quote}
% 其他受支持的选项有:
% \begin{quote}
%   \xoption{depth}、
%   \xoption{dest}、
%   \xoption{color}、
%   \xoption{gotor}、
%   \xoption{level}、
%   \xoption{named}、
%   \xoption{openlevel}、
%   \xoption{page}、
%   \xoption{rawaction}、
%   \xoption{uri}、
%   \xoption{view}、
% \end{quote}
% 另外,以下键(key)是可用的:
% \begin{quote}
%   \xoption{text}
% \end{quote}
% 它返回大纲条目(outline entry)的文本(text)。
%
% \paragraph{选项设置(Option setting)。}
% 在钩子(hook)内部可以使用 \cs{bookmarksetup}\ 设置选项。
%
% \subsection{与 \xpackage{hyperref}\ 的兼容性}
%
% \xpackage{bookmark}\ 宏包自动禁用 \xpackage{hyperref}\ 宏包的书签(bookmarks)。但是,
% \xpackage{bookmark}\ 宏包使用了 \xpackage{hyperref}\ 宏包的一些代码。例如,
% \xpackage{bookmark}\ 宏包重新定义了 \xpackage{hyperref}\ 宏包在 \cs{addcontentsline}\ 命令
% 和其他命令中插入的\cs{Hy@writebookmark}\ 钩子。因此,不应禁用 \xpackage{hyperref}\ 宏包的书签。
%
% \xpackage{bookmark}\ 宏包使用 \xpackage{hyperref}\ 宏包的 \cs{pdfstringdef},且不提供替换(replacement)。
%
% \xpackage{hyperref}\ 宏包的一些选项也能在 \xpackage{bookmark}\ 宏包中实现(implemented):
% \begin{quote}
% \begin{tabular}{|l@{}|l@{}|}
%   \hline
%   \xpackage{hyperref}\ 宏包的选项\  &\ \xpackage{bookmark}\ 宏包的选项\ \ \\ \hline
%   \xoption{bookmarksdepth} &\ \xoption{depth}\\ \hline
%   \xoption{bookmarksopen} & \ \xoption{open}\\ \hline
%   \xoption{bookmarksopenlevel}\ \ \  &\ \xoption{openlevel}\\ \hline
%   \xoption{bookmarksnumbered} \ \ \ &\ \xoption{numbered}\\ \hline
% \end{tabular}
% \end{quote}
%
% 还可以使用以下命令:
% \begin{quote}
%   \cs{pdfbookmark}\\
%   \cs{currentpdfbookmark}\\
%   \cs{subpdfbookmark}\\
%   \cs{belowpdfbookmark}
% \end{quote}
%
% \subsection{在末尾添加书签}
%
% 宏包选项 \xoption{atend}\ 启用以下宏(macro):
% \begin{declcs}{BookmarkAtEnd}
%   \M{stuff}
% \end{declcs}
% \cs{BookmarkAtEnd}\ 宏将 \meta{stuff}\ 放在文档末尾。\meta{stuff}\ 表示书签命令(bookmark commands)。举例:
% \begin{quote}
%\begin{verbatim}
%\usepackage[atend]{bookmark}
%\BookmarkAtEnd{%
%  \bookmarksetup{startatroot}%
%  \bookmark[named=LastPage, level=0]{Last page}%
%}
%\end{verbatim}
% \end{quote}
%
% 或者,可以在 \cs{bookmark}\ 中给出 \xoption{startatroot}\ 选项:
% \begin{quote}
%\begin{verbatim}
%\BookmarkAtEnd{%
%  \bookmark[
%    startatroot,
%    named=LastPage,
%    level=0,
%  ]{Last page}%
%}
%\end{verbatim}
% \end{quote}
%
% \paragraph{备注(Remarks):}
% \begin{itemize}
% \item
%   \cs{BookmarkAtEnd} 隐藏了这样一个事实,即在文档末尾添加书签的方法取决于驱动程序(driver)。
%
%   为此,驱动程序 \xoption{pdftex}\ 使用 \xpackage{atveryend}\ 宏包。如果 \cs{AtEndDocument}\ 太早,
%   最后一个页面(last page)可能不会被发送出去(shipped out)。由于需要 \xext{aux}\ 文件,此驱动程序使
%   用 \cs{AfterLastShipout}。
%
%   其他驱动程序(\xoption{dvipdfm}、\xoption{xetex}、\xoption{vtex})的实现(implementation)
%   取决于 \cs{special},\cs{special}\ 在最后一页之后没有效果。在这种情况下,\xpackage{atenddvi}\ 宏包的
%   \cs{AtEndDvi}\ 有帮助。它将其参数(argument)放在文档的最后一页(last page)。至少需要运行 \hologo{LaTeX}\ 两次,
%   因为最后一页是由引用(reference)检测到的。
%
%   \xoption{dvips}\ 现在使用新的 LaTeX 钩子 \texttt{shipout/lastpage}。
% \item
%   未指定 \cs{BookmarkAtEnd}\ 参数的扩展时间(time of expansion)。这可以立即发生,也可以在文档末尾发生。
% \end{itemize}
%
% \subsection{限制/行动计划}
%
% \begin{itemize}
% \item 支持缺失动作(missing actions)(启动,\dots)。
% \item 对 \xpackage{hyperref}\ 的 \xoption{bookmarkstype}\ 选项进行了更好的设计(design)。
% \end{itemize}
%
% \section{示例(Example)}
%
%    \begin{macrocode}
%<*example>
%    \end{macrocode}
%    \begin{macrocode}
\documentclass{article}
\usepackage{xcolor}[2007/01/21]
\usepackage{hyperref}
\usepackage[
  open,
  openlevel=2,
  atend
]{bookmark}[2019/12/03]

\bookmarksetup{color=blue}

\BookmarkAtEnd{%
  \bookmarksetup{startatroot}%
  \bookmark[named=LastPage, level=0]{End/Last page}%
  \bookmark[named=FirstPage, level=1]{First page}%
}

\begin{document}
\section{First section}
\subsection{Subsection A}
\begin{figure}
  \hypertarget{fig}{}%
  A figure.
\end{figure}
\bookmark[
  rellevel=1,
  keeplevel,
  dest=fig
]{A figure}
\subsection{Subsection B}
\subsubsection{Subsubsection C}
\subsection{Umlauts: \"A\"O\"U\"a\"o\"u\ss}
\newpage
\bookmarksetup{
  bold,
  color=[rgb]{1,0,0}
}
\section{Very important section}
\bookmarksetup{
  italic,
  bold=false,
  color=blue
}
\subsection{Italic section}
\bookmarksetup{
  italic=false
}
\part{Misc}
\section{Diverse}
\subsubsection{Subsubsection, omitting subsection}
\bookmarksetup{
  startatroot
}
\section{Last section outside part}
\subsection{Subsection}
\bookmarksetup{
  color={}
}
\begingroup
  \bookmarksetup{level=0, color=green!80!black}
  \bookmark[named=FirstPage]{First page}
  \bookmark[named=LastPage]{Last page}
  \bookmark[named=PrevPage]{Previous page}
  \bookmark[named=NextPage]{Next page}
\endgroup
\bookmark[
  page=2,
  view=FitH 800
]{Page 2, FitH 800}
\bookmark[
  page=2,
  view=FitBH \calc{\paperheight-\topmargin-1in-\headheight-\headsep}
]{Page 2, FitBH top of text body}
\bookmark[
  uri={http://www.dante.de/},
  color=magenta
]{Dante homepage}
\bookmark[
  gotor={t.pdf},
  page=1,
  view={XYZ 0 1000 null},
  color=cyan!75!black
]{File t.pdf}
\bookmark[named=FirstPage]{First page}
\bookmark[rellevel=1, named=LastPage]{Last page (rellevel=1)}
\bookmark[named=PrevPage]{Previous page}
\bookmark[level=0, named=FirstPage]{First page (level=0)}
\bookmark[
  rellevel=1,
  keeplevel,
  named=LastPage
]{Last page (rellevel=1, keeplevel)}
\bookmark[named=PrevPage]{Previous page}
\end{document}
%    \end{macrocode}
%    \begin{macrocode}
%</example>
%    \end{macrocode}
%
% \StopEventually{
% }
%
% \section{实现(Implementation)}
%
% \subsection{宏包(Package)}
%
%    \begin{macrocode}
%<*package>
\NeedsTeXFormat{LaTeX2e}
\ProvidesPackage{bookmark}%
  [2020-11-06 v1.29 PDF bookmarks (HO)]%
%    \end{macrocode}
%
% \subsubsection{要求(Requirements)}
%
% \paragraph{\hologo{eTeX}.}
%
%    \begin{macro}{\BKM@CalcExpr}
%    \begin{macrocode}
\begingroup\expandafter\expandafter\expandafter\endgroup
\expandafter\ifx\csname numexpr\endcsname\relax
  \def\BKM@CalcExpr#1#2#3#4{%
    \begingroup
      \count@=#2\relax
      \advance\count@ by#3#4\relax
      \edef\x{\endgroup
        \def\noexpand#1{\the\count@}%
      }%
    \x
  }%
\else
  \def\BKM@CalcExpr#1#2#3#4{%
    \edef#1{%
      \the\numexpr#2#3#4\relax
    }%
  }%
\fi
%    \end{macrocode}
%    \end{macro}
%
% \paragraph{\hologo{pdfTeX}\ 的转义功能(escape features)}
%
%    \begin{macro}{\BKM@EscapeName}
%    \begin{macrocode}
\def\BKM@EscapeName#1{%
  \ifx#1\@empty
  \else
    \EdefEscapeName#1#1%
  \fi
}%
%    \end{macrocode}
%    \end{macro}
%    \begin{macro}{\BKM@EscapeString}
%    \begin{macrocode}
\def\BKM@EscapeString#1{%
  \ifx#1\@empty
  \else
    \EdefEscapeString#1#1%
  \fi
}%
%    \end{macrocode}
%    \end{macro}
%    \begin{macro}{\BKM@EscapeHex}
%    \begin{macrocode}
\def\BKM@EscapeHex#1{%
  \ifx#1\@empty
  \else
    \EdefEscapeHex#1#1%
  \fi
}%
%    \end{macrocode}
%    \end{macro}
%    \begin{macro}{\BKM@UnescapeHex}
%    \begin{macrocode}
\def\BKM@UnescapeHex#1{%
  \EdefUnescapeHex#1#1%
}%
%    \end{macrocode}
%    \end{macro}
%
% \paragraph{宏包(Packages)。}
%
% 不要加载由 \xpackage{hyperref}\ 加载的宏包
%    \begin{macrocode}
\RequirePackage{hyperref}[2010/06/18]
%    \end{macrocode}
%
% \subsubsection{宏包选项(Package options)}
%
%    \begin{macrocode}
\SetupKeyvalOptions{family=BKM,prefix=BKM@}
\DeclareLocalOptions{%
  atend,%
  bold,%
  color,%
  depth,%
  dest,%
  draft,%
  final,%
  gotor,%
  italic,%
  keeplevel,%
  level,%
  named,%
  numbered,%
  open,%
  openlevel,%
  page,%
  rawaction,%
  rellevel,%
  srcfile,%
  srcline,%
  startatroot,%
  uri,%
  view,%
}
%    \end{macrocode}
%    \begin{macro}{\bookmarksetup}
%    \begin{macrocode}
\newcommand*{\bookmarksetup}{\kvsetkeys{BKM}}
%    \end{macrocode}
%    \end{macro}
%    \begin{macro}{\BKM@setup}
%    \begin{macrocode}
\def\BKM@setup#1{%
  \bookmarksetup{#1}%
  \ifx\BKM@HookNext\ltx@empty
  \else
    \expandafter\bookmarksetup\expandafter{\BKM@HookNext}%
    \BKM@HookNextClear
  \fi
  \BKM@hook
  \ifBKM@keeplevel
  \else
    \xdef\BKM@currentlevel{\BKM@level}%
  \fi
}
%    \end{macrocode}
%    \end{macro}
%
%    \begin{macro}{\bookmarksetupnext}
%    \begin{macrocode}
\newcommand*{\bookmarksetupnext}[1]{%
  \ltx@GlobalAppendToMacro\BKM@HookNext{,#1}%
}
%    \end{macrocode}
%    \end{macro}
%    \begin{macro}{\BKM@setupnext}
%    \begin{macrocode}
%    \end{macrocode}
%    \end{macro}
%    \begin{macro}{\BKM@HookNextClear}
%    \begin{macrocode}
\def\BKM@HookNextClear{%
  \global\let\BKM@HookNext\ltx@empty
}
%    \end{macrocode}
%    \end{macro}
%    \begin{macro}{\BKM@HookNext}
%    \begin{macrocode}
\BKM@HookNextClear
%    \end{macrocode}
%    \end{macro}
%
%    \begin{macrocode}
\DeclareBoolOption{draft}
\DeclareComplementaryOption{final}{draft}
%    \end{macrocode}
%    \begin{macro}{\BKM@DisableOptions}
%    \begin{macrocode}
\def\BKM@DisableOptions{%
  \DisableKeyvalOption[action=warning,package=bookmark]%
      {BKM}{draft}%
  \DisableKeyvalOption[action=warning,package=bookmark]%
      {BKM}{final}%
}
%    \end{macrocode}
%    \end{macro}
%    \begin{macrocode}
\DeclareBoolOption[\ifHy@bookmarksopen true\else false\fi]{open}
%    \end{macrocode}
%    \begin{macro}{\bookmark@open}
%    \begin{macrocode}
\def\bookmark@open{%
  \ifBKM@open\ltx@one\else\ltx@zero\fi
}
%    \end{macrocode}
%    \end{macro}
%    \begin{macrocode}
\DeclareStringOption[\maxdimen]{openlevel}
%    \end{macrocode}
%    \begin{macro}{\BKM@openlevel}
%    \begin{macrocode}
\edef\BKM@openlevel{\number\@bookmarksopenlevel}
%    \end{macrocode}
%    \end{macro}
%    \begin{macrocode}
%\DeclareStringOption[\c@tocdepth]{depth}
\ltx@IfUndefined{Hy@bookmarksdepth}{%
  \def\BKM@depth{\c@tocdepth}%
}{%
  \let\BKM@depth\Hy@bookmarksdepth
}
\define@key{BKM}{depth}[]{%
  \edef\BKM@param{#1}%
  \ifx\BKM@param\@empty
    \def\BKM@depth{\c@tocdepth}%
  \else
    \ltx@IfUndefined{toclevel@\BKM@param}{%
      \@onelevel@sanitize\BKM@param
      \edef\BKM@temp{\expandafter\@car\BKM@param\@nil}%
      \ifcase 0\expandafter\ifx\BKM@temp-1\fi
              \expandafter\ifnum\expandafter`\BKM@temp>47 %
                \expandafter\ifnum\expandafter`\BKM@temp<58 %
                  1%
                \fi
              \fi
              \relax
        \PackageWarning{bookmark}{%
          Unknown document division name (\BKM@param)\MessageBreak
          for option `depth'%
        }%
      \else
        \BKM@SetDepthOrLevel\BKM@depth\BKM@param
      \fi
    }{%
      \BKM@SetDepthOrLevel\BKM@depth{%
        \csname toclevel@\BKM@param\endcsname
      }%
    }%
  \fi
}
%    \end{macrocode}
%    \begin{macro}{\bookmark@depth}
%    \begin{macrocode}
\def\bookmark@depth{\BKM@depth}
%    \end{macrocode}
%    \end{macro}
%    \begin{macro}{\BKM@SetDepthOrLevel}
%    \begin{macrocode}
\def\BKM@SetDepthOrLevel#1#2{%
  \begingroup
    \setbox\z@=\hbox{%
      \count@=#2\relax
      \expandafter
    }%
  \expandafter\endgroup
  \expandafter\def\expandafter#1\expandafter{\the\count@}%
}
%    \end{macrocode}
%    \end{macro}
%    \begin{macrocode}
\DeclareStringOption[\BKM@currentlevel]{level}[\BKM@currentlevel]
\define@key{BKM}{level}{%
  \edef\BKM@param{#1}%
  \ifx\BKM@param\BKM@MacroCurrentLevel
    \let\BKM@level\BKM@param
  \else
    \ltx@IfUndefined{toclevel@\BKM@param}{%
      \@onelevel@sanitize\BKM@param
      \edef\BKM@temp{\expandafter\@car\BKM@param\@nil}%
      \ifcase 0\expandafter\ifx\BKM@temp-1\fi
              \expandafter\ifnum\expandafter`\BKM@temp>47 %
                \expandafter\ifnum\expandafter`\BKM@temp<58 %
                  1%
                \fi
              \fi
              \relax
        \PackageWarning{bookmark}{%
          Unknown document division name (\BKM@param)\MessageBreak
          for option `level'%
        }%
      \else
        \BKM@SetDepthOrLevel\BKM@level\BKM@param
      \fi
    }{%
      \BKM@SetDepthOrLevel\BKM@level{%
        \csname toclevel@\BKM@param\endcsname
      }%
    }%
  \fi
}
%    \end{macrocode}
%    \begin{macro}{\BKM@MacroCurrentLevel}
%    \begin{macrocode}
\def\BKM@MacroCurrentLevel{\BKM@currentlevel}
%    \end{macrocode}
%    \end{macro}
%    \begin{macrocode}
\DeclareBoolOption{keeplevel}
\DeclareBoolOption{startatroot}
%    \end{macrocode}
%    \begin{macro}{\BKM@startatrootfalse}
%    \begin{macrocode}
\def\BKM@startatrootfalse{%
  \global\let\ifBKM@startatroot\iffalse
}
%    \end{macrocode}
%    \end{macro}
%    \begin{macro}{\BKM@startatroottrue}
%    \begin{macrocode}
\def\BKM@startatroottrue{%
  \global\let\ifBKM@startatroot\iftrue
}
%    \end{macrocode}
%    \end{macro}
%    \begin{macrocode}
\define@key{BKM}{rellevel}{%
  \BKM@CalcExpr\BKM@level{#1}+\BKM@currentlevel
}
%    \end{macrocode}
%    \begin{macro}{\bookmark@level}
%    \begin{macrocode}
\def\bookmark@level{\BKM@level}
%    \end{macrocode}
%    \end{macro}
%    \begin{macro}{\BKM@currentlevel}
%    \begin{macrocode}
\def\BKM@currentlevel{0}
%    \end{macrocode}
%    \end{macro}
%    Make \xpackage{bookmark}'s option \xoption{numbered} an alias
%    for \xpackage{hyperref}'s \xoption{bookmarksnumbered}.
%    \begin{macrocode}
\DeclareBoolOption[%
  \ifHy@bookmarksnumbered true\else false\fi
]{numbered}
\g@addto@macro\BKM@numberedtrue{%
  \let\ifHy@bookmarksnumbered\iftrue
}
\g@addto@macro\BKM@numberedfalse{%
  \let\ifHy@bookmarksnumbered\iffalse
}
\g@addto@macro\Hy@bookmarksnumberedtrue{%
  \let\ifBKM@numbered\iftrue
}
\g@addto@macro\Hy@bookmarksnumberedfalse{%
  \let\ifBKM@numbered\iffalse
}
%    \end{macrocode}
%    \begin{macro}{\bookmark@numbered}
%    \begin{macrocode}
\def\bookmark@numbered{%
  \ifBKM@numbered\ltx@one\else\ltx@zero\fi
}
%    \end{macrocode}
%    \end{macro}
%
% \paragraph{重定义 \xpackage{hyperref}\ 宏包的选项}
%
%    \begin{macro}{\BKM@PatchHyperrefOption}
%    \begin{macrocode}
\def\BKM@PatchHyperrefOption#1{%
  \expandafter\BKM@@PatchHyperrefOption\csname KV@Hyp@#1\endcsname%
}
%    \end{macrocode}
%    \end{macro}
%    \begin{macro}{\BKM@@PatchHyperrefOption}
%    \begin{macrocode}
\def\BKM@@PatchHyperrefOption#1{%
  \expandafter\BKM@@@PatchHyperrefOption#1{##1}\BKM@nil#1%
}
%    \end{macrocode}
%    \end{macro}
%    \begin{macro}{\BKM@@@PatchHyperrefOption}
%    \begin{macrocode}
\def\BKM@@@PatchHyperrefOption#1\BKM@nil#2#3{%
  \def#2##1{%
    #1%
    \bookmarksetup{#3={##1}}%
  }%
}
%    \end{macrocode}
%    \end{macro}
%    \begin{macrocode}
\BKM@PatchHyperrefOption{bookmarksopen}{open}
\BKM@PatchHyperrefOption{bookmarksopenlevel}{openlevel}
\BKM@PatchHyperrefOption{bookmarksdepth}{depth}
%    \end{macrocode}
%
% \paragraph{字体样式(font style)选项。}
%
%    注意:\xpackage{bitset}\ 宏是基于零的,PDF 规范(PDF specifications)以1开头。
%    \begin{macrocode}
\bitsetReset{BKM@FontStyle}%
\define@key{BKM}{italic}[true]{%
  \expandafter\ifx\csname if#1\endcsname\iftrue
    \bitsetSet{BKM@FontStyle}{0}%
  \else
    \bitsetClear{BKM@FontStyle}{0}%
  \fi
}%
\define@key{BKM}{bold}[true]{%
  \expandafter\ifx\csname if#1\endcsname\iftrue
    \bitsetSet{BKM@FontStyle}{1}%
  \else
    \bitsetClear{BKM@FontStyle}{1}%
  \fi
}%
%    \end{macrocode}
%    \begin{macro}{\bookmark@italic}
%    \begin{macrocode}
\def\bookmark@italic{%
  \ifnum\bitsetGet{BKM@FontStyle}{0}=1 \ltx@one\else\ltx@zero\fi
}
%    \end{macrocode}
%    \end{macro}
%    \begin{macro}{\bookmark@bold}
%    \begin{macrocode}
\def\bookmark@bold{%
  \ifnum\bitsetGet{BKM@FontStyle}{1}=1 \ltx@one\else\ltx@zero\fi
}
%    \end{macrocode}
%    \end{macro}
%    \begin{macro}{\BKM@PrintStyle}
%    \begin{macrocode}
\def\BKM@PrintStyle{%
  \bitsetGetDec{BKM@FontStyle}%
}%
%    \end{macrocode}
%    \end{macro}
%
% \paragraph{颜色(color)选项。}
%
%    \begin{macrocode}
\define@key{BKM}{color}{%
  \HyColor@BookmarkColor{#1}\BKM@color{bookmark}{color}%
}
%    \end{macrocode}
%    \begin{macro}{\BKM@color}
%    \begin{macrocode}
\let\BKM@color\@empty
%    \end{macrocode}
%    \end{macro}
%    \begin{macro}{\bookmark@color}
%    \begin{macrocode}
\def\bookmark@color{\BKM@color}
%    \end{macrocode}
%    \end{macro}
%
% \subsubsection{动作(action)选项}
%
%    \begin{macrocode}
\def\BKM@temp#1{%
  \DeclareStringOption{#1}%
  \expandafter\edef\csname bookmark@#1\endcsname{%
    \expandafter\noexpand\csname BKM@#1\endcsname
  }%
}
%    \end{macrocode}
%    \begin{macro}{\bookmark@dest}
%    \begin{macrocode}
\BKM@temp{dest}
%    \end{macrocode}
%    \end{macro}
%    \begin{macro}{\bookmark@named}
%    \begin{macrocode}
\BKM@temp{named}
%    \end{macrocode}
%    \end{macro}
%    \begin{macro}{\bookmark@uri}
%    \begin{macrocode}
\BKM@temp{uri}
%    \end{macrocode}
%    \end{macro}
%    \begin{macro}{\bookmark@gotor}
%    \begin{macrocode}
\BKM@temp{gotor}
%    \end{macrocode}
%    \end{macro}
%    \begin{macro}{\bookmark@rawaction}
%    \begin{macrocode}
\BKM@temp{rawaction}
%    \end{macrocode}
%    \end{macro}
%
%    \begin{macrocode}
\define@key{BKM}{page}{%
  \def\BKM@page{#1}%
  \ifx\BKM@page\@empty
  \else
    \edef\BKM@page{\number\BKM@page}%
    \ifnum\BKM@page>\z@
    \else
      \PackageError{bookmark}{Page must be positive}\@ehc
      \def\BKM@page{1}%
    \fi
  \fi
}
%    \end{macrocode}
%    \begin{macro}{\BKM@page}
%    \begin{macrocode}
\let\BKM@page\@empty
%    \end{macrocode}
%    \end{macro}
%    \begin{macro}{\bookmark@page}
%    \begin{macrocode}
\def\bookmark@page{\BKM@@page}
%    \end{macrocode}
%    \end{macro}
%
%    \begin{macrocode}
\define@key{BKM}{view}{%
  \BKM@CheckView{#1}%
}
%    \end{macrocode}
%    \begin{macro}{\BKM@view}
%    \begin{macrocode}
\let\BKM@view\@empty
%    \end{macrocode}
%    \end{macro}
%    \begin{macro}{\bookmark@view}
%    \begin{macrocode}
\def\bookmark@view{\BKM@view}
%    \end{macrocode}
%    \end{macro}
%    \begin{macro}{BKM@CheckView}
%    \begin{macrocode}
\def\BKM@CheckView#1{%
  \BKM@CheckViewType#1 \@nil
}
%    \end{macrocode}
%    \end{macro}
%    \begin{macro}{\BKM@CheckViewType}
%    \begin{macrocode}
\def\BKM@CheckViewType#1 #2\@nil{%
  \def\BKM@type{#1}%
  \@onelevel@sanitize\BKM@type
  \BKM@TestViewType{Fit}{}%
  \BKM@TestViewType{FitB}{}%
  \BKM@TestViewType{FitH}{%
    \BKM@CheckParam#2 \@nil{top}%
  }%
  \BKM@TestViewType{FitBH}{%
    \BKM@CheckParam#2 \@nil{top}%
  }%
  \BKM@TestViewType{FitV}{%
    \BKM@CheckParam#2 \@nil{bottom}%
  }%
  \BKM@TestViewType{FitBV}{%
    \BKM@CheckParam#2 \@nil{bottom}%
  }%
  \BKM@TestViewType{FitR}{%
    \BKM@CheckRect{#2}{ }%
  }%
  \BKM@TestViewType{XYZ}{%
    \BKM@CheckXYZ{#2}{ }%
  }%
  \@car{%
    \PackageError{bookmark}{%
      Unknown view type `\BKM@type',\MessageBreak
      using `FitH' instead%
    }\@ehc
    \def\BKM@view{FitH}%
  }%
  \@nil
}
%    \end{macrocode}
%    \end{macro}
%    \begin{macro}{\BKM@TestViewType}
%    \begin{macrocode}
\def\BKM@TestViewType#1{%
  \def\BKM@temp{#1}%
  \@onelevel@sanitize\BKM@temp
  \ifx\BKM@type\BKM@temp
    \let\BKM@view\BKM@temp
    \expandafter\@car
  \else
    \expandafter\@gobble
  \fi
}
%    \end{macrocode}
%    \end{macro}
%    \begin{macro}{BKM@CheckParam}
%    \begin{macrocode}
\def\BKM@CheckParam#1 #2\@nil#3{%
  \def\BKM@param{#1}%
  \ifx\BKM@param\@empty
    \PackageWarning{bookmark}{%
      Missing parameter (#3) for `\BKM@type',\MessageBreak
      using 0%
    }%
    \def\BKM@param{0}%
  \else
    \BKM@CalcParam
  \fi
  \edef\BKM@view{\BKM@view\space\BKM@param}%
}
%    \end{macrocode}
%    \end{macro}
%    \begin{macro}{BKM@CheckRect}
%    \begin{macrocode}
\def\BKM@CheckRect#1#2{%
  \BKM@@CheckRect#1#2#2#2#2\@nil
}
%    \end{macrocode}
%    \end{macro}
%    \begin{macro}{\BKM@@CheckRect}
%    \begin{macrocode}
\def\BKM@@CheckRect#1 #2 #3 #4 #5\@nil{%
  \def\BKM@temp{0}%
  \def\BKM@param{#1}%
  \ifx\BKM@param\@empty
    \def\BKM@param{0}%
    \def\BKM@temp{1}%
  \else
    \BKM@CalcParam
  \fi
  \edef\BKM@view{\BKM@view\space\BKM@param}%
  \def\BKM@param{#2}%
  \ifx\BKM@param\@empty
    \def\BKM@param{0}%
    \def\BKM@temp{1}%
  \else
    \BKM@CalcParam
  \fi
  \edef\BKM@view{\BKM@view\space\BKM@param}%
  \def\BKM@param{#3}%
  \ifx\BKM@param\@empty
    \def\BKM@param{0}%
    \def\BKM@temp{1}%
  \else
    \BKM@CalcParam
  \fi
  \edef\BKM@view{\BKM@view\space\BKM@param}%
  \def\BKM@param{#4}%
  \ifx\BKM@param\@empty
    \def\BKM@param{0}%
    \def\BKM@temp{1}%
  \else
    \BKM@CalcParam
  \fi
  \edef\BKM@view{\BKM@view\space\BKM@param}%
  \ifnum\BKM@temp>\z@
    \PackageWarning{bookmark}{Missing parameters for `\BKM@type'}%
  \fi
}
%    \end{macrocode}
%    \end{macro}
%    \begin{macro}{\BKM@CheckXYZ}
%    \begin{macrocode}
\def\BKM@CheckXYZ#1#2{%
  \BKM@@CheckXYZ#1#2#2#2\@nil
}
%    \end{macrocode}
%    \end{macro}
%    \begin{macro}{\BKM@@CheckXYZ}
%    \begin{macrocode}
\def\BKM@@CheckXYZ#1 #2 #3 #4\@nil{%
  \def\BKM@param{#1}%
  \let\BKM@temp\BKM@param
  \@onelevel@sanitize\BKM@temp
  \ifx\BKM@param\@empty
    \let\BKM@param\BKM@null
  \else
    \ifx\BKM@temp\BKM@null
    \else
      \BKM@CalcParam
    \fi
  \fi
  \edef\BKM@view{\BKM@view\space\BKM@param}%
  \def\BKM@param{#2}%
  \let\BKM@temp\BKM@param
  \@onelevel@sanitize\BKM@temp
  \ifx\BKM@param\@empty
    \let\BKM@param\BKM@null
  \else
    \ifx\BKM@temp\BKM@null
    \else
      \BKM@CalcParam
    \fi
  \fi
  \edef\BKM@view{\BKM@view\space\BKM@param}%
  \def\BKM@param{#3}%
  \ifx\BKM@param\@empty
    \let\BKM@param\BKM@null
  \fi
  \edef\BKM@view{\BKM@view\space\BKM@param}%
}
%    \end{macrocode}
%    \end{macro}
%    \begin{macro}{\BKM@null}
%    \begin{macrocode}
\def\BKM@null{null}
\@onelevel@sanitize\BKM@null
%    \end{macrocode}
%    \end{macro}
%
%    \begin{macro}{\BKM@CalcParam}
%    \begin{macrocode}
\def\BKM@CalcParam{%
  \begingroup
  \let\calc\@firstofone
  \expandafter\BKM@@CalcParam\BKM@param\@empty\@empty\@nil
}
%    \end{macrocode}
%    \end{macro}
%    \begin{macro}{\BKM@@CalcParam}
%    \begin{macrocode}
\def\BKM@@CalcParam#1#2#3\@nil{%
  \ifx\calc#1%
    \@ifundefined{calc@assign@dimen}{%
      \@ifundefined{dimexpr}{%
        \setlength{\dimen@}{#2}%
      }{%
        \setlength{\dimen@}{\dimexpr#2\relax}%
      }%
    }{%
      \setlength{\dimen@}{#2}%
    }%
    \dimen@.99626\dimen@
    \edef\BKM@param{\strip@pt\dimen@}%
    \expandafter\endgroup
    \expandafter\def\expandafter\BKM@param\expandafter{\BKM@param}%
  \else
    \endgroup
  \fi
}
%    \end{macrocode}
%    \end{macro}
%
% \subsubsection{\xoption{atend}\ 选项}
%
%    \begin{macrocode}
\DeclareBoolOption{atend}
\g@addto@macro\BKM@DisableOptions{%
  \DisableKeyvalOption[action=warning,package=bookmark]%
      {BKM}{atend}%
}
%    \end{macrocode}
%
% \subsubsection{\xoption{style}\ 选项}
%
%    \begin{macro}{\bookmarkdefinestyle}
%    \begin{macrocode}
\newcommand*{\bookmarkdefinestyle}[2]{%
  \@ifundefined{BKM@style@#1}{%
  }{%
    \PackageInfo{bookmark}{Redefining style `#1'}%
  }%
  \@namedef{BKM@style@#1}{#2}%
}
%    \end{macrocode}
%    \end{macro}
%    \begin{macrocode}
\define@key{BKM}{style}{%
  \BKM@StyleCall{#1}%
}
\newif\ifBKM@ok
%    \end{macrocode}
%    \begin{macro}{\BKM@StyleCall}
%    \begin{macrocode}
\def\BKM@StyleCall#1{%
  \@ifundefined{BKM@style@#1}{%
    \PackageWarning{bookmark}{%
      Ignoring unknown style `#1'%
    }%
  }{%
%    \end{macrocode}
%    检查样式堆栈(style stack)。
%    \begin{macrocode}
    \BKM@oktrue
    \edef\BKM@StyleCurrent{#1}%
    \@onelevel@sanitize\BKM@StyleCurrent
    \let\BKM@StyleEntry\BKM@StyleEntryCheck
    \BKM@StyleStack
    \ifBKM@ok
      \expandafter\@firstofone
    \else
      \PackageError{bookmark}{%
        Ignoring recursive call of style `\BKM@StyleCurrent'%
      }\@ehc
      \expandafter\@gobble
    \fi
    {%
%    \end{macrocode}
%    在堆栈上推送当前样式(Push current style on stack)。
%    \begin{macrocode}
      \let\BKM@StyleEntry\relax
      \edef\BKM@StyleStack{%
        \BKM@StyleEntry{\BKM@StyleCurrent}%
        \BKM@StyleStack
      }%
%    \end{macrocode}
%   调用样式(Call style)。
%    \begin{macrocode}
      \expandafter\expandafter\expandafter\bookmarksetup
      \expandafter\expandafter\expandafter{%
        \csname BKM@style@\BKM@StyleCurrent\endcsname
      }%
%    \end{macrocode}
%    从堆栈中弹出当前样式(Pop current style from stack)。
%    \begin{macrocode}
      \BKM@StyleStackPop
    }%
  }%
}
%    \end{macrocode}
%    \end{macro}
%    \begin{macro}{\BKM@StyleStackPop}
%    \begin{macrocode}
\def\BKM@StyleStackPop{%
  \let\BKM@StyleEntry\relax
  \edef\BKM@StyleStack{%
    \expandafter\@gobbletwo\BKM@StyleStack
  }%
}
%    \end{macrocode}
%    \end{macro}
%    \begin{macro}{\BKM@StyleEntryCheck}
%    \begin{macrocode}
\def\BKM@StyleEntryCheck#1{%
  \def\BKM@temp{#1}%
  \ifx\BKM@temp\BKM@StyleCurrent
    \BKM@okfalse
  \fi
}
%    \end{macrocode}
%    \end{macro}
%    \begin{macro}{\BKM@StyleStack}
%    \begin{macrocode}
\def\BKM@StyleStack{}
%    \end{macrocode}
%    \end{macro}
%
% \subsubsection{源文件位置(source file location)选项}
%
%    \begin{macrocode}
\DeclareStringOption{srcline}
\DeclareStringOption{srcfile}
%    \end{macrocode}
%
% \subsubsection{钩子支持(Hook support)}
%
%    \begin{macro}{\BKM@hook}
%    \begin{macrocode}
\def\BKM@hook{}
%    \end{macrocode}
%    \end{macro}
%    \begin{macrocode}
\define@key{BKM}{addtohook}{%
  \ltx@LocalAppendToMacro\BKM@hook{#1}%
}
%    \end{macrocode}
%
%    \begin{macro}{bookmarkget}
%    \begin{macrocode}
\newcommand*{\bookmarkget}[1]{%
  \romannumeral0%
  \ltx@ifundefined{bookmark@#1}{%
    \ltx@space
  }{%
    \expandafter\expandafter\expandafter\ltx@space
    \csname bookmark@#1\endcsname
  }%
}
%    \end{macrocode}
%    \end{macro}
%
% \subsubsection{设置和加载驱动程序}
%
% \paragraph{检测驱动程序。}
%
%    \begin{macro}{\BKM@DefineDriverKey}
%    \begin{macrocode}
\def\BKM@DefineDriverKey#1{%
  \define@key{BKM}{#1}[]{%
    \def\BKM@driver{#1}%
  }%
  \g@addto@macro\BKM@DisableOptions{%
    \DisableKeyvalOption[action=warning,package=bookmark]%
        {BKM}{#1}%
  }%
}
%    \end{macrocode}
%    \end{macro}
%    \begin{macrocode}
\BKM@DefineDriverKey{pdftex}
\BKM@DefineDriverKey{dvips}
\BKM@DefineDriverKey{dvipdfm}
\BKM@DefineDriverKey{dvipdfmx}
\BKM@DefineDriverKey{xetex}
\BKM@DefineDriverKey{vtex}
\define@key{BKM}{dvipdfmx-outline-open}[true]{%
 \PackageWarning{bookmark}{Option 'dvipdfmx-outline-open' is obsolete
   and ignored}{}}
%    \end{macrocode}
%    \begin{macro}{\bookmark@driver}
%    \begin{macrocode}
\def\bookmark@driver{\BKM@driver}
%    \end{macrocode}
%    \end{macro}
%    \begin{macrocode}
\InputIfFileExists{bookmark.cfg}{}{}
%    \end{macrocode}
%    \begin{macro}{\BookmarkDriverDefault}
%    \begin{macrocode}
\providecommand*{\BookmarkDriverDefault}{dvips}
%    \end{macrocode}
%    \end{macro}
%    \begin{macro}{\BKM@driver}
% Lua\TeX\ 和 pdf\TeX\ 共享驱动程序。
%    \begin{macrocode}
\ifpdf
  \def\BKM@driver{pdftex}%
  \ifx\pdfoutline\@undefined
    \ifx\pdfextension\@undefined\else
      \protected\def\pdfoutline{\pdfextension outline }
    \fi
  \fi
\else
  \ifxetex
    \def\BKM@driver{dvipdfm}%
  \else
    \ifvtex
      \def\BKM@driver{vtex}%
    \else
      \edef\BKM@driver{\BookmarkDriverDefault}%
    \fi
  \fi
\fi
%    \end{macrocode}
%    \end{macro}
%
% \paragraph{过程选项(Process options)。}
%
%    \begin{macrocode}
\ProcessKeyvalOptions*
\BKM@DisableOptions
%    \end{macrocode}
%
% \paragraph{\xoption{draft}\ 选项}
%
%    \begin{macrocode}
\ifBKM@draft
  \PackageWarningNoLine{bookmark}{Draft mode on}%
  \let\bookmarksetup\ltx@gobble
  \let\BookmarkAtEnd\ltx@gobble
  \let\bookmarkdefinestyle\ltx@gobbletwo
  \let\bookmarkget\ltx@gobble
  \let\pdfbookmark\ltx@undefined
  \newcommand*{\pdfbookmark}[3][]{}%
  \let\currentpdfbookmark\ltx@gobbletwo
  \let\subpdfbookmark\ltx@gobbletwo
  \let\belowpdfbookmark\ltx@gobbletwo
  \newcommand*{\bookmark}[2][]{}%
  \renewcommand*{\Hy@writebookmark}[5]{}%
  \let\ReadBookmarks\relax
  \let\BKM@DefGotoNameAction\ltx@gobbletwo % package `hypdestopt'
  \expandafter\endinput
\fi
%    \end{macrocode}
%
% \paragraph{验证和加载驱动程序。}
%
%    \begin{macrocode}
\def\BKM@temp{dvipdfmx}%
\ifx\BKM@temp\BKM@driver
  \def\BKM@driver{dvipdfm}%
\fi
\def\BKM@temp{pdftex}%
\ifpdf
  \ifx\BKM@temp\BKM@driver
  \else
    \PackageWarningNoLine{bookmark}{%
      Wrong driver `\BKM@driver', using `pdftex' instead%
    }%
    \let\BKM@driver\BKM@temp
  \fi
\else
  \ifx\BKM@temp\BKM@driver
    \PackageError{bookmark}{%
      Wrong driver, pdfTeX is not running in PDF mode.\MessageBreak
      Package loading is aborted%
    }\@ehc
    \expandafter\expandafter\expandafter\endinput
  \fi
  \def\BKM@temp{dvipdfm}%
  \ifxetex
    \ifx\BKM@temp\BKM@driver
    \else
      \PackageWarningNoLine{bookmark}{%
        Wrong driver `\BKM@driver',\MessageBreak
        using `dvipdfm' for XeTeX instead%
      }%
      \let\BKM@driver\BKM@temp
    \fi
  \else
    \def\BKM@temp{vtex}%
    \ifvtex
      \ifx\BKM@temp\BKM@driver
      \else
        \PackageWarningNoLine{bookmark}{%
          Wrong driver `\BKM@driver',\MessageBreak
          using `vtex' for VTeX instead%
        }%
        \let\BKM@driver\BKM@temp
      \fi
    \else
      \ifx\BKM@temp\BKM@driver
        \PackageError{bookmark}{%
          Wrong driver, VTeX is not running in PDF mode.\MessageBreak
          Package loading is aborted%
        }\@ehc
        \expandafter\expandafter\expandafter\endinput
      \fi
    \fi
  \fi
\fi
\providecommand\IfFormatAtLeastTF{\@ifl@t@r\fmtversion}
\IfFormatAtLeastTF{2020/10/01}{}{\edef\BKM@driver{\BKM@driver-2019-12-03}}
\InputIfFileExists{bkm-\BKM@driver.def}{}{%
  \PackageError{bookmark}{%
    Unsupported driver `\BKM@driver'.\MessageBreak
    Package loading is aborted%
  }\@ehc
  \endinput
}
%    \end{macrocode}
%
% \subsubsection{与 \xpackage{hyperref}\ 的兼容性}
%
%    \begin{macro}{\pdfbookmark}
%    \begin{macrocode}
\let\pdfbookmark\ltx@undefined
\newcommand*{\pdfbookmark}[3][0]{%
  \bookmark[level=#1,dest={#3.#1}]{#2}%
  \hyper@anchorstart{#3.#1}\hyper@anchorend
}
%    \end{macrocode}
%    \end{macro}
%    \begin{macro}{\currentpdfbookmark}
%    \begin{macrocode}
\def\currentpdfbookmark{%
  \pdfbookmark[\BKM@currentlevel]%
}
%    \end{macrocode}
%    \end{macro}
%    \begin{macro}{\subpdfbookmark}
%    \begin{macrocode}
\def\subpdfbookmark{%
  \BKM@CalcExpr\BKM@CalcResult\BKM@currentlevel+1%
  \expandafter\pdfbookmark\expandafter[\BKM@CalcResult]%
}
%    \end{macrocode}
%    \end{macro}
%    \begin{macro}{\belowpdfbookmark}
%    \begin{macrocode}
\def\belowpdfbookmark#1#2{%
  \xdef\BKM@gtemp{\number\BKM@currentlevel}%
  \subpdfbookmark{#1}{#2}%
  \global\let\BKM@currentlevel\BKM@gtemp
}
%    \end{macrocode}
%    \end{macro}
%
%    节号(section number)、文本(text)、标签(label)、级别(level)、文件(file)
%    \begin{macro}{\Hy@writebookmark}
%    \begin{macrocode}
\def\Hy@writebookmark#1#2#3#4#5{%
  \ifnum#4>\BKM@depth\relax
  \else
    \def\BKM@type{#5}%
    \ifx\BKM@type\Hy@bookmarkstype
      \begingroup
        \ifBKM@numbered
          \let\numberline\Hy@numberline
          \let\booknumberline\Hy@numberline
          \let\partnumberline\Hy@numberline
          \let\chapternumberline\Hy@numberline
        \else
          \let\numberline\@gobble
          \let\booknumberline\@gobble
          \let\partnumberline\@gobble
          \let\chapternumberline\@gobble
        \fi
        \bookmark[level=#4,dest={\HyperDestNameFilter{#3}}]{#2}%
      \endgroup
    \fi
  \fi
}
%    \end{macrocode}
%    \end{macro}
%
%    \begin{macro}{\ReadBookmarks}
%    \begin{macrocode}
\let\ReadBookmarks\relax
%    \end{macrocode}
%    \end{macro}
%
%    \begin{macrocode}
%</package>
%    \end{macrocode}
%
% \subsection{dvipdfm 的驱动程序}
%
%    \begin{macrocode}
%<*dvipdfm>
\NeedsTeXFormat{LaTeX2e}
\ProvidesFile{bkm-dvipdfm.def}%
  [2020-11-06 v1.29 bookmark driver for dvipdfm (HO)]%
%    \end{macrocode}
%
%    \begin{macro}{\BKM@id}
%    \begin{macrocode}
\newcount\BKM@id
\BKM@id=\z@
%    \end{macrocode}
%    \end{macro}
%
%    \begin{macro}{\BKM@0}
%    \begin{macrocode}
\@namedef{BKM@0}{000}
%    \end{macrocode}
%    \end{macro}
%    \begin{macro}{\ifBKM@sw}
%    \begin{macrocode}
\newif\ifBKM@sw
%    \end{macrocode}
%    \end{macro}
%
%    \begin{macro}{\bookmark}
%    \begin{macrocode}
\newcommand*{\bookmark}[2][]{%
  \if@filesw
    \begingroup
      \def\bookmark@text{#2}%
      \BKM@setup{#1}%
      \edef\BKM@prev{\the\BKM@id}%
      \global\advance\BKM@id\@ne
      \BKM@swtrue
      \@whilesw\ifBKM@sw\fi{%
        \def\BKM@abslevel{1}%
        \ifnum\ifBKM@startatroot\z@\else\BKM@prev\fi=\z@
          \BKM@startatrootfalse
          \expandafter\xdef\csname BKM@\the\BKM@id\endcsname{%
            0{\BKM@level}\BKM@abslevel
          }%
          \BKM@swfalse
        \else
          \expandafter\expandafter\expandafter\BKM@getx
              \csname BKM@\BKM@prev\endcsname
          \ifnum\BKM@level>\BKM@x@level\relax
            \BKM@CalcExpr\BKM@abslevel\BKM@x@abslevel+1%
            \expandafter\xdef\csname BKM@\the\BKM@id\endcsname{%
              {\BKM@prev}{\BKM@level}\BKM@abslevel
            }%
            \BKM@swfalse
          \else
            \let\BKM@prev\BKM@x@parent
          \fi
        \fi
      }%
      \csname HyPsd@XeTeXBigCharstrue\endcsname
      \pdfstringdef\BKM@title{\bookmark@text}%
      \edef\BKM@FLAGS{\BKM@PrintStyle}%
      \let\BKM@action\@empty
      \ifx\BKM@gotor\@empty
        \ifx\BKM@dest\@empty
          \ifx\BKM@named\@empty
            \ifx\BKM@rawaction\@empty
              \ifx\BKM@uri\@empty
                \ifx\BKM@page\@empty
                  \PackageError{bookmark}{Missing action}\@ehc
                  \edef\BKM@action{/Dest[@page1/Fit]}%
                \else
                  \ifx\BKM@view\@empty
                    \def\BKM@view{Fit}%
                  \fi
                  \edef\BKM@action{/Dest[@page\BKM@page/\BKM@view]}%
                \fi
              \else
                \BKM@EscapeString\BKM@uri
                \edef\BKM@action{%
                  /A<<%
                    /S/URI%
                    /URI(\BKM@uri)%
                  >>%
                }%
              \fi
            \else
              \edef\BKM@action{/A<<\BKM@rawaction>>}%
            \fi
          \else
            \BKM@EscapeName\BKM@named
            \edef\BKM@action{%
              /A<</S/Named/N/\BKM@named>>%
            }%
          \fi
        \else
          \BKM@EscapeString\BKM@dest
          \edef\BKM@action{%
            /A<<%
              /S/GoTo%
              /D(\BKM@dest)%
            >>%
          }%
        \fi
      \else
        \ifx\BKM@dest\@empty
          \ifx\BKM@page\@empty
            \def\BKM@page{0}%
          \else
            \BKM@CalcExpr\BKM@page\BKM@page-1%
          \fi
          \ifx\BKM@view\@empty
            \def\BKM@view{Fit}%
          \fi
          \edef\BKM@action{/D[\BKM@page/\BKM@view]}%
        \else
          \BKM@EscapeString\BKM@dest
          \edef\BKM@action{/D(\BKM@dest)}%
        \fi
        \BKM@EscapeString\BKM@gotor
        \edef\BKM@action{%
          /A<<%
            /S/GoToR%
            /F(\BKM@gotor)%
            \BKM@action
          >>%
        }%
      \fi
      \special{pdf:%
        out
              [%
              \ifBKM@open
                \ifnum\BKM@level<%
                    \expandafter\ltx@firstofone\expandafter
                    {\number\BKM@openlevel} %
                \else
                  -%
                \fi
              \else
                -%
              \fi
              ] %
            \BKM@abslevel
        <<%
          /Title(\BKM@title)%
          \ifx\BKM@color\@empty
          \else
            /C[\BKM@color]%
          \fi
          \ifnum\BKM@FLAGS>\z@
            /F \BKM@FLAGS
          \fi
          \BKM@action
        >>%
      }%
    \endgroup
  \fi
}
%    \end{macrocode}
%    \end{macro}
%    \begin{macro}{\BKM@getx}
%    \begin{macrocode}
\def\BKM@getx#1#2#3{%
  \def\BKM@x@parent{#1}%
  \def\BKM@x@level{#2}%
  \def\BKM@x@abslevel{#3}%
}
%    \end{macrocode}
%    \end{macro}
%
%    \begin{macrocode}
%</dvipdfm>
%    \end{macrocode}
%
% \subsection{\hologo{VTeX}\ 的驱动程序}
%
%    \begin{macrocode}
%<*vtex>
\NeedsTeXFormat{LaTeX2e}
\ProvidesFile{bkm-vtex.def}%
  [2020-11-06 v1.29 bookmark driver for VTeX (HO)]%
%    \end{macrocode}
%
%    \begin{macrocode}
\ifvtexpdf
\else
  \PackageWarningNoLine{bookmark}{%
    The VTeX driver only supports PDF mode%
  }%
\fi
%    \end{macrocode}
%
%    \begin{macro}{\BKM@id}
%    \begin{macrocode}
\newcount\BKM@id
\BKM@id=\z@
%    \end{macrocode}
%    \end{macro}
%
%    \begin{macro}{\BKM@0}
%    \begin{macrocode}
\@namedef{BKM@0}{00}
%    \end{macrocode}
%    \end{macro}
%    \begin{macro}{\ifBKM@sw}
%    \begin{macrocode}
\newif\ifBKM@sw
%    \end{macrocode}
%    \end{macro}
%
%    \begin{macro}{\bookmark}
%    \begin{macrocode}
\newcommand*{\bookmark}[2][]{%
  \if@filesw
    \begingroup
      \def\bookmark@text{#2}%
      \BKM@setup{#1}%
      \edef\BKM@prev{\the\BKM@id}%
      \global\advance\BKM@id\@ne
      \BKM@swtrue
      \@whilesw\ifBKM@sw\fi{%
        \ifnum\ifBKM@startatroot\z@\else\BKM@prev\fi=\z@
          \BKM@startatrootfalse
          \def\BKM@parent{0}%
          \expandafter\xdef\csname BKM@\the\BKM@id\endcsname{%
            0{\BKM@level}%
          }%
          \BKM@swfalse
        \else
          \expandafter\expandafter\expandafter\BKM@getx
              \csname BKM@\BKM@prev\endcsname
          \ifnum\BKM@level>\BKM@x@level\relax
            \let\BKM@parent\BKM@prev
            \expandafter\xdef\csname BKM@\the\BKM@id\endcsname{%
              {\BKM@prev}{\BKM@level}%
            }%
            \BKM@swfalse
          \else
            \let\BKM@prev\BKM@x@parent
          \fi
        \fi
      }%
      \pdfstringdef\BKM@title{\bookmark@text}%
      \BKM@vtex@title
      \edef\BKM@FLAGS{\BKM@PrintStyle}%
      \let\BKM@action\@empty
      \ifx\BKM@gotor\@empty
        \ifx\BKM@dest\@empty
          \ifx\BKM@named\@empty
            \ifx\BKM@rawaction\@empty
              \ifx\BKM@uri\@empty
                \ifx\BKM@page\@empty
                  \PackageError{bookmark}{Missing action}\@ehc
                  \def\BKM@action{!1}%
                \else
                  \edef\BKM@action{!\BKM@page}%
                \fi
              \else
                \BKM@EscapeString\BKM@uri
                \edef\BKM@action{%
                  <u=%
                    /S/URI%
                    /URI(\BKM@uri)%
                  >%
                }%
              \fi
            \else
              \edef\BKM@action{<u=\BKM@rawaction>}%
            \fi
          \else
            \BKM@EscapeName\BKM@named
            \edef\BKM@action{%
              <u=%
                /S/Named%
                /N/\BKM@named
              >%
            }%
          \fi
        \else
          \BKM@EscapeString\BKM@dest
          \edef\BKM@action{\BKM@dest}%
        \fi
      \else
        \ifx\BKM@dest\@empty
          \ifx\BKM@page\@empty
            \def\BKM@page{1}%
          \fi
          \ifx\BKM@view\@empty
            \def\BKM@view{Fit}%
          \fi
          \edef\BKM@action{/D[\BKM@page/\BKM@view]}%
        \else
          \BKM@EscapeString\BKM@dest
          \edef\BKM@action{/D(\BKM@dest)}%
        \fi
        \BKM@EscapeString\BKM@gotor
        \edef\BKM@action{%
          <u=%
            /S/GoToR%
            /F(\BKM@gotor)%
            \BKM@action
          >>%
        }%
      \fi
      \ifx\BKM@color\@empty
        \let\BKM@RGBcolor\@empty
      \else
        \expandafter\BKM@toRGB\BKM@color\@nil
      \fi
      \special{%
        !outline \BKM@action;%
        p=\BKM@parent,%
        i=\number\BKM@id,%
        s=%
          \ifBKM@open
            \ifnum\BKM@level<\BKM@openlevel
              o%
            \else
              c%
            \fi
          \else
            c%
          \fi,%
        \ifx\BKM@RGBcolor\@empty
        \else
          c=\BKM@RGBcolor,%
        \fi
        \ifnum\BKM@FLAGS>\z@
          f=\BKM@FLAGS,%
        \fi
        t=\BKM@title
      }%
    \endgroup
  \fi
}
%    \end{macrocode}
%    \end{macro}
%    \begin{macro}{\BKM@getx}
%    \begin{macrocode}
\def\BKM@getx#1#2{%
  \def\BKM@x@parent{#1}%
  \def\BKM@x@level{#2}%
}
%    \end{macrocode}
%    \end{macro}
%    \begin{macro}{\BKM@toRGB}
%    \begin{macrocode}
\def\BKM@toRGB#1 #2 #3\@nil{%
  \let\BKM@RGBcolor\@empty
  \BKM@toRGBComponent{#1}%
  \BKM@toRGBComponent{#2}%
  \BKM@toRGBComponent{#3}%
}
%    \end{macrocode}
%    \end{macro}
%    \begin{macro}{\BKM@toRGBComponent}
%    \begin{macrocode}
\def\BKM@toRGBComponent#1{%
  \dimen@=#1pt\relax
  \ifdim\dimen@>\z@
    \ifdim\dimen@<\p@
      \dimen@=255\dimen@
      \advance\dimen@ by 32768sp\relax
      \divide\dimen@ by 65536\relax
      \dimen@ii=\dimen@
      \divide\dimen@ii by 16\relax
      \edef\BKM@RGBcolor{%
        \BKM@RGBcolor
        \BKM@toHexDigit\dimen@ii
      }%
      \dimen@ii=16\dimen@ii
      \advance\dimen@-\dimen@ii
      \edef\BKM@RGBcolor{%
        \BKM@RGBcolor
        \BKM@toHexDigit\dimen@
      }%
    \else
      \edef\BKM@RGBcolor{\BKM@RGBcolor FF}%
    \fi
  \else
    \edef\BKM@RGBcolor{\BKM@RGBcolor00}%
  \fi
}
%    \end{macrocode}
%    \end{macro}
%    \begin{macro}{\BKM@toHexDigit}
%    \begin{macrocode}
\def\BKM@toHexDigit#1{%
  \ifcase\expandafter\@firstofone\expandafter{\number#1} %
    0\or 1\or 2\or 3\or 4\or 5\or 6\or 7\or
    8\or 9\or A\or B\or C\or D\or E\or F%
  \fi
}
%    \end{macrocode}
%    \end{macro}
%    \begin{macrocode}
\begingroup
  \catcode`\|=0 %
  \catcode`\\=12 %
%    \end{macrocode}
%    \begin{macro}{\BKM@vtex@title}
%    \begin{macrocode}
  |gdef|BKM@vtex@title{%
    |@onelevel@sanitize|BKM@title
    |edef|BKM@title{|expandafter|BKM@vtex@leftparen|BKM@title\(|@nil}%
    |edef|BKM@title{|expandafter|BKM@vtex@rightparen|BKM@title\)|@nil}%
    |edef|BKM@title{|expandafter|BKM@vtex@zero|BKM@title\0|@nil}%
    |edef|BKM@title{|expandafter|BKM@vtex@one|BKM@title\1|@nil}%
    |edef|BKM@title{|expandafter|BKM@vtex@two|BKM@title\2|@nil}%
    |edef|BKM@title{|expandafter|BKM@vtex@three|BKM@title\3|@nil}%
  }%
%    \end{macrocode}
%    \end{macro}
%    \begin{macro}{\BKM@vtex@leftparen}
%    \begin{macrocode}
  |gdef|BKM@vtex@leftparen#1\(#2|@nil{%
    #1%
    |ifx||#2||%
    |else
      (%
      |ltx@ReturnAfterFi{%
        |BKM@vtex@leftparen#2|@nil
      }%
    |fi
  }%
%    \end{macrocode}
%    \end{macro}
%    \begin{macro}{\BKM@vtex@rightparen}
%    \begin{macrocode}
  |gdef|BKM@vtex@rightparen#1\)#2|@nil{%
    #1%
    |ifx||#2||%
    |else
      )%
      |ltx@ReturnAfterFi{%
        |BKM@vtex@rightparen#2|@nil
      }%
    |fi
  }%
%    \end{macrocode}
%    \end{macro}
%    \begin{macro}{\BKM@vtex@zero}
%    \begin{macrocode}
  |gdef|BKM@vtex@zero#1\0#2|@nil{%
    #1%
    |ifx||#2||%
    |else
      |noexpand|hv@pdf@char0%
      |ltx@ReturnAfterFi{%
        |BKM@vtex@zero#2|@nil
      }%
    |fi
  }%
%    \end{macrocode}
%    \end{macro}
%    \begin{macro}{\BKM@vtex@one}
%    \begin{macrocode}
  |gdef|BKM@vtex@one#1\1#2|@nil{%
    #1%
    |ifx||#2||%
    |else
      |noexpand|hv@pdf@char1%
      |ltx@ReturnAfterFi{%
        |BKM@vtex@one#2|@nil
      }%
    |fi
  }%
%    \end{macrocode}
%    \end{macro}
%    \begin{macro}{\BKM@vtex@two}
%    \begin{macrocode}
  |gdef|BKM@vtex@two#1\2#2|@nil{%
    #1%
    |ifx||#2||%
    |else
      |noexpand|hv@pdf@char2%
      |ltx@ReturnAfterFi{%
        |BKM@vtex@two#2|@nil
      }%
    |fi
  }%
%    \end{macrocode}
%    \end{macro}
%    \begin{macro}{\BKM@vtex@three}
%    \begin{macrocode}
  |gdef|BKM@vtex@three#1\3#2|@nil{%
    #1%
    |ifx||#2||%
    |else
      |noexpand|hv@pdf@char3%
      |ltx@ReturnAfterFi{%
        |BKM@vtex@three#2|@nil
      }%
    |fi
  }%
%    \end{macrocode}
%    \end{macro}
%    \begin{macrocode}
|endgroup
%    \end{macrocode}
%
%    \begin{macrocode}
%</vtex>
%    \end{macrocode}
%
% \subsection{\hologo{pdfTeX}\ 的驱动程序}
%
%    \begin{macrocode}
%<*pdftex>
\NeedsTeXFormat{LaTeX2e}
\ProvidesFile{bkm-pdftex.def}%
  [2020-11-06 v1.29 bookmark driver for pdfTeX (HO)]%
%    \end{macrocode}
%
%    \begin{macro}{\BKM@DO@entry}
%    \begin{macrocode}
\def\BKM@DO@entry#1#2{%
  \begingroup
    \kvsetkeys{BKM@DO}{#1}%
    \def\BKM@DO@title{#2}%
    \ifx\BKM@DO@srcfile\@empty
    \else
      \BKM@UnescapeHex\BKM@DO@srcfile
    \fi
    \BKM@UnescapeHex\BKM@DO@title
    \expandafter\expandafter\expandafter\BKM@getx
        \csname BKM@\BKM@DO@id\endcsname\@empty\@empty
    \let\BKM@attr\@empty
    \ifx\BKM@DO@flags\@empty
    \else
      \edef\BKM@attr{\BKM@attr/F \BKM@DO@flags}%
    \fi
    \ifx\BKM@DO@color\@empty
    \else
      \edef\BKM@attr{\BKM@attr/C[\BKM@DO@color]}%
    \fi
    \ifx\BKM@attr\@empty
    \else
      \edef\BKM@attr{attr{\BKM@attr}}%
    \fi
    \let\BKM@action\@empty
    \ifx\BKM@DO@gotor\@empty
      \ifx\BKM@DO@dest\@empty
        \ifx\BKM@DO@named\@empty
          \ifx\BKM@DO@rawaction\@empty
            \ifx\BKM@DO@uri\@empty
              \ifx\BKM@DO@page\@empty
                \PackageError{bookmark}{%
                  Missing action\BKM@SourceLocation
                }\@ehc
                \edef\BKM@action{goto page1{/Fit}}%
              \else
                \ifx\BKM@DO@view\@empty
                  \def\BKM@DO@view{Fit}%
                \fi
                \edef\BKM@action{goto page\BKM@DO@page{/\BKM@DO@view}}%
              \fi
            \else
              \BKM@UnescapeHex\BKM@DO@uri
              \BKM@EscapeString\BKM@DO@uri
              \edef\BKM@action{user{<</S/URI/URI(\BKM@DO@uri)>>}}%
            \fi
          \else
            \BKM@UnescapeHex\BKM@DO@rawaction
            \edef\BKM@action{%
              user{%
                <<%
                  \BKM@DO@rawaction
                >>%
              }%
            }%
          \fi
        \else
          \BKM@EscapeName\BKM@DO@named
          \edef\BKM@action{%
            user{<</S/Named/N/\BKM@DO@named>>}%
          }%
        \fi
      \else
        \BKM@UnescapeHex\BKM@DO@dest
        \BKM@DefGotoNameAction\BKM@action\BKM@DO@dest
      \fi
    \else
      \ifx\BKM@DO@dest\@empty
        \ifx\BKM@DO@page\@empty
          \def\BKM@DO@page{0}%
        \else
          \BKM@CalcExpr\BKM@DO@page\BKM@DO@page-1%
        \fi
        \ifx\BKM@DO@view\@empty
          \def\BKM@DO@view{Fit}%
        \fi
        \edef\BKM@action{/D[\BKM@DO@page/\BKM@DO@view]}%
      \else
        \BKM@UnescapeHex\BKM@DO@dest
        \BKM@EscapeString\BKM@DO@dest
        \edef\BKM@action{/D(\BKM@DO@dest)}%
      \fi
      \BKM@UnescapeHex\BKM@DO@gotor
      \BKM@EscapeString\BKM@DO@gotor
      \edef\BKM@action{%
        user{%
          <<%
            /S/GoToR%
            /F(\BKM@DO@gotor)%
            \BKM@action
          >>%
        }%
      }%
    \fi
    \pdfoutline\BKM@attr\BKM@action
                count\ifBKM@DO@open\else-\fi\BKM@x@childs
                {\BKM@DO@title}%
  \endgroup
}
%    \end{macrocode}
%    \end{macro}
%    \begin{macro}{\BKM@DefGotoNameAction}
%    \cs{BKM@DefGotoNameAction}\ 宏是一个用于 \xpackage{hypdestopt}\ 宏包的钩子(hook)。
%    \begin{macrocode}
\def\BKM@DefGotoNameAction#1#2{%
  \BKM@EscapeString\BKM@DO@dest
  \edef#1{goto name{#2}}%
}
%    \end{macrocode}
%    \end{macro}
%    \begin{macrocode}
%</pdftex>
%    \end{macrocode}
%
%    \begin{macrocode}
%<*pdftex|pdfmark>
%    \end{macrocode}
%    \begin{macro}{\BKM@SourceLocation}
%    \begin{macrocode}
\def\BKM@SourceLocation{%
  \ifx\BKM@DO@srcfile\@empty
    \ifx\BKM@DO@srcline\@empty
    \else
      .\MessageBreak
      Source: line \BKM@DO@srcline
    \fi
  \else
    \ifx\BKM@DO@srcline\@empty
      .\MessageBreak
      Source: file `\BKM@DO@srcfile'%
    \else
      .\MessageBreak
      Source: file `\BKM@DO@srcfile', line \BKM@DO@srcline
    \fi
  \fi
}
%    \end{macrocode}
%    \end{macro}
%    \begin{macrocode}
%</pdftex|pdfmark>
%    \end{macrocode}
%
% \subsection{具有 pdfmark 特色(specials)的驱动程序}
%
% \subsubsection{dvips 驱动程序}
%
%    \begin{macrocode}
%<*dvips>
\NeedsTeXFormat{LaTeX2e}
\ProvidesFile{bkm-dvips.def}%
  [2020-11-06 v1.29 bookmark driver for dvips (HO)]%
%    \end{macrocode}
%    \begin{macro}{\BKM@PSHeaderFile}
%    \begin{macrocode}
\def\BKM@PSHeaderFile#1{%
  \special{PSfile=#1}%
}
%    \end{macrocode}
%    \begin{macro}{\BKM@filename}
%    \begin{macrocode}
\def\BKM@filename{\jobname.out.ps}
%    \end{macrocode}
%    \end{macro}
%    \begin{macrocode}
\AddToHook{shipout/lastpage}{%
  \BKM@pdfmark@out
  \BKM@PSHeaderFile\BKM@filename
  }
%    \end{macrocode}
%    \end{macro}
%    \begin{macrocode}
%</dvips>
%    \end{macrocode}
%
% \subsubsection{公共部分(Common part)}
%
%    \begin{macrocode}
%<*pdfmark>
%    \end{macrocode}
%
%    \begin{macro}{\BKM@pdfmark@out}
%    不要在这里使用 \xpackage{rerunfilecheck}\ 宏包,因为在 \hologo{TeX}\ 运行期间不会
%    读取 \cs{BKM@filename}\ 文件。
%    \begin{macrocode}
\def\BKM@pdfmark@out{%
  \if@filesw
    \newwrite\BKM@file
    \immediate\openout\BKM@file=\BKM@filename\relax
    \BKM@write{\@percentchar!}%
    \BKM@write{/pdfmark where{pop}}%
    \BKM@write{%
      {%
        /globaldict where{pop globaldict}{userdict}ifelse%
        /pdfmark/cleartomark load put%
      }%
    }%
    \BKM@write{ifelse}%
  \else
    \let\BKM@write\@gobble
    \let\BKM@DO@entry\@gobbletwo
  \fi
}
%    \end{macrocode}
%    \end{macro}
%    \begin{macro}{\BKM@write}
%    \begin{macrocode}
\def\BKM@write#{%
  \immediate\write\BKM@file
}
%    \end{macrocode}
%    \end{macro}
%
%    \begin{macro}{\BKM@DO@entry}
%    Pdfmark 的规范(specification)说明 |/Color| 是颜色(color)的键名(key name),
%    但是 ghostscript 只将键(key)传递到 PDF 文件中,因此键名必须是 |/C|。
%    \begin{macrocode}
\def\BKM@DO@entry#1#2{%
  \begingroup
    \kvsetkeys{BKM@DO}{#1}%
    \ifx\BKM@DO@srcfile\@empty
    \else
      \BKM@UnescapeHex\BKM@DO@srcfile
    \fi
    \def\BKM@DO@title{#2}%
    \BKM@UnescapeHex\BKM@DO@title
    \expandafter\expandafter\expandafter\BKM@getx
        \csname BKM@\BKM@DO@id\endcsname\@empty\@empty
    \let\BKM@attr\@empty
    \ifx\BKM@DO@flags\@empty
    \else
      \edef\BKM@attr{\BKM@attr/F \BKM@DO@flags}%
    \fi
    \ifx\BKM@DO@color\@empty
    \else
      \edef\BKM@attr{\BKM@attr/C[\BKM@DO@color]}%
    \fi
    \let\BKM@action\@empty
    \ifx\BKM@DO@gotor\@empty
      \ifx\BKM@DO@dest\@empty
        \ifx\BKM@DO@named\@empty
          \ifx\BKM@DO@rawaction\@empty
            \ifx\BKM@DO@uri\@empty
              \ifx\BKM@DO@page\@empty
                \PackageError{bookmark}{%
                  Missing action\BKM@SourceLocation
                }\@ehc
                \edef\BKM@action{%
                  /Action/GoTo%
                  /Page 1%
                  /View[/Fit]%
                }%
              \else
                \ifx\BKM@DO@view\@empty
                  \def\BKM@DO@view{Fit}%
                \fi
                \edef\BKM@action{%
                  /Action/GoTo%
                  /Page \BKM@DO@page
                  /View[/\BKM@DO@view]%
                }%
              \fi
            \else
              \BKM@UnescapeHex\BKM@DO@uri
              \BKM@EscapeString\BKM@DO@uri
              \edef\BKM@action{%
                /Action<<%
                  /Subtype/URI%
                  /URI(\BKM@DO@uri)%
                >>%
              }%
            \fi
          \else
            \BKM@UnescapeHex\BKM@DO@rawaction
            \edef\BKM@action{%
              /Action<<%
                \BKM@DO@rawaction
              >>%
            }%
          \fi
        \else
          \BKM@EscapeName\BKM@DO@named
          \edef\BKM@action{%
            /Action<<%
              /Subtype/Named%
              /N/\BKM@DO@named
            >>%
          }%
        \fi
      \else
        \BKM@UnescapeHex\BKM@DO@dest
        \BKM@EscapeString\BKM@DO@dest
        \edef\BKM@action{%
          /Action/GoTo%
          /Dest(\BKM@DO@dest)cvn%
        }%
      \fi
    \else
      \ifx\BKM@DO@dest\@empty
        \ifx\BKM@DO@page\@empty
          \def\BKM@DO@page{1}%
        \fi
        \ifx\BKM@DO@view\@empty
          \def\BKM@DO@view{Fit}%
        \fi
        \edef\BKM@action{%
          /Page \BKM@DO@page
          /View[/\BKM@DO@view]%
        }%
      \else
        \BKM@UnescapeHex\BKM@DO@dest
        \BKM@EscapeString\BKM@DO@dest
        \edef\BKM@action{%
          /Dest(\BKM@DO@dest)cvn%
        }%
      \fi
      \BKM@UnescapeHex\BKM@DO@gotor
      \BKM@EscapeString\BKM@DO@gotor
      \edef\BKM@action{%
        /Action/GoToR%
        /File(\BKM@DO@gotor)%
        \BKM@action
      }%
    \fi
    \BKM@write{[}%
    \BKM@write{/Title(\BKM@DO@title)}%
    \ifnum\BKM@x@childs>\z@
      \BKM@write{/Count \ifBKM@DO@open\else-\fi\BKM@x@childs}%
    \fi
    \ifx\BKM@attr\@empty
    \else
      \BKM@write{\BKM@attr}%
    \fi
    \BKM@write{\BKM@action}%
    \BKM@write{/OUT pdfmark}%
  \endgroup
}
%    \end{macrocode}
%    \end{macro}
%    \begin{macrocode}
%</pdfmark>
%    \end{macrocode}
%
% \subsection{\xoption{pdftex}\ 和 \xoption{pdfmark}\ 的公共部分}
%
%    \begin{macrocode}
%<*pdftex|pdfmark>
%    \end{macrocode}
%
% \subsubsection{写入辅助文件(auxiliary file)}
%
%    \begin{macrocode}
\AddToHook{begindocument}{%
 \immediate\write\@mainaux{\string\providecommand\string\BKM@entry[2]{}}}
%    \end{macrocode}
%
%    \begin{macro}{\BKM@id}
%    \begin{macrocode}
\newcount\BKM@id
\BKM@id=\z@
%    \end{macrocode}
%    \end{macro}
%
%    \begin{macro}{\BKM@0}
%    \begin{macrocode}
\@namedef{BKM@0}{000}
%    \end{macrocode}
%    \end{macro}
%    \begin{macro}{\ifBKM@sw}
%    \begin{macrocode}
\newif\ifBKM@sw
%    \end{macrocode}
%    \end{macro}
%
%    \begin{macro}{\bookmark}
%    \begin{macrocode}
\newcommand*{\bookmark}[2][]{%
  \if@filesw
    \begingroup
      \BKM@InitSourceLocation
      \def\bookmark@text{#2}%
      \BKM@setup{#1}%
      \ifx\BKM@srcfile\@empty
      \else
        \BKM@EscapeHex\BKM@srcfile
      \fi
      \edef\BKM@prev{\the\BKM@id}%
      \global\advance\BKM@id\@ne
      \BKM@swtrue
      \@whilesw\ifBKM@sw\fi{%
        \ifnum\ifBKM@startatroot\z@\else\BKM@prev\fi=\z@
          \BKM@startatrootfalse
          \expandafter\xdef\csname BKM@\the\BKM@id\endcsname{%
            0{\BKM@level}0%
          }%
          \BKM@swfalse
        \else
          \expandafter\expandafter\expandafter\BKM@getx
              \csname BKM@\BKM@prev\endcsname
          \ifnum\BKM@level>\BKM@x@level\relax
            \expandafter\xdef\csname BKM@\the\BKM@id\endcsname{%
              {\BKM@prev}{\BKM@level}0%
            }%
            \ifnum\BKM@prev>\z@
              \BKM@CalcExpr\BKM@CalcResult\BKM@x@childs+1%
              \expandafter\xdef\csname BKM@\BKM@prev\endcsname{%
                {\BKM@x@parent}{\BKM@x@level}{\BKM@CalcResult}%
              }%
            \fi
            \BKM@swfalse
          \else
            \let\BKM@prev\BKM@x@parent
          \fi
        \fi
      }%
      \pdfstringdef\BKM@title{\bookmark@text}%
      \edef\BKM@FLAGS{\BKM@PrintStyle}%
      \csname BKM@HypDestOptHook\endcsname
      \BKM@EscapeHex\BKM@dest
      \BKM@EscapeHex\BKM@uri
      \BKM@EscapeHex\BKM@gotor
      \BKM@EscapeHex\BKM@rawaction
      \BKM@EscapeHex\BKM@title
      \immediate\write\@mainaux{%
        \string\BKM@entry{%
          id=\number\BKM@id
          \ifBKM@open
            \ifnum\BKM@level<\BKM@openlevel
              ,open%
            \fi
          \fi
          \BKM@auxentry{dest}%
          \BKM@auxentry{named}%
          \BKM@auxentry{uri}%
          \BKM@auxentry{gotor}%
          \BKM@auxentry{page}%
          \BKM@auxentry{view}%
          \BKM@auxentry{rawaction}%
          \BKM@auxentry{color}%
          \ifnum\BKM@FLAGS>\z@
            ,flags=\BKM@FLAGS
          \fi
          \BKM@auxentry{srcline}%
          \BKM@auxentry{srcfile}%
        }{\BKM@title}%
      }%
    \endgroup
  \fi
}
%    \end{macrocode}
%    \end{macro}
%    \begin{macro}{\BKM@getx}
%    \begin{macrocode}
\def\BKM@getx#1#2#3{%
  \def\BKM@x@parent{#1}%
  \def\BKM@x@level{#2}%
  \def\BKM@x@childs{#3}%
}
%    \end{macrocode}
%    \end{macro}
%    \begin{macro}{\BKM@auxentry}
%    \begin{macrocode}
\def\BKM@auxentry#1{%
  \expandafter\ifx\csname BKM@#1\endcsname\@empty
  \else
    ,#1={\csname BKM@#1\endcsname}%
  \fi
}
%    \end{macrocode}
%    \end{macro}
%
%    \begin{macro}{\BKM@InitSourceLocation}
%    \begin{macrocode}
\def\BKM@InitSourceLocation{%
  \edef\BKM@srcline{\the\inputlineno}%
  \BKM@LuaTeX@InitFile
  \ifx\BKM@srcfile\@empty
    \ltx@IfUndefined{currfilepath}{}{%
      \edef\BKM@srcfile{\currfilepath}%
    }%
  \fi
}
%    \end{macrocode}
%    \end{macro}
%    \begin{macro}{\BKM@LuaTeX@InitFile}
%    \begin{macrocode}
\ifluatex
  \ifnum\luatexversion>36 %
    \def\BKM@LuaTeX@InitFile{%
      \begingroup
        \ltx@LocToksA={}%
      \edef\x{\endgroup
        \def\noexpand\BKM@srcfile{%
          \the\expandafter\ltx@LocToksA
          \directlua{%
             if status and status.filename then %
               tex.settoks('ltx@LocToksA', status.filename)%
             end%
          }%
        }%
      }\x
    }%
  \else
    \let\BKM@LuaTeX@InitFile\relax
  \fi
\else
  \let\BKM@LuaTeX@InitFile\relax
\fi
%    \end{macrocode}
%    \end{macro}
%
% \subsubsection{读取辅助数据(auxiliary data)}
%
%    \begin{macrocode}
\SetupKeyvalOptions{family=BKM@DO,prefix=BKM@DO@}
\DeclareStringOption[0]{id}
\DeclareBoolOption{open}
\DeclareStringOption{flags}
\DeclareStringOption{color}
\DeclareStringOption{dest}
\DeclareStringOption{named}
\DeclareStringOption{uri}
\DeclareStringOption{gotor}
\DeclareStringOption{page}
\DeclareStringOption{view}
\DeclareStringOption{rawaction}
\DeclareStringOption{srcline}
\DeclareStringOption{srcfile}
%    \end{macrocode}
%
%    \begin{macrocode}
\AtBeginDocument{%
  \let\BKM@entry\BKM@DO@entry
}
%    \end{macrocode}
%
%    \begin{macrocode}
%</pdftex|pdfmark>
%    \end{macrocode}
%
% \subsection{\xoption{atend}\ 选项}
%
% \subsubsection{钩子(Hook)}
%
%    \begin{macrocode}
%<*package>
%    \end{macrocode}
%    \begin{macrocode}
\ifBKM@atend
\else
%    \end{macrocode}
%    \begin{macro}{\BookmarkAtEnd}
%    这是一个虚拟定义(dummy definition),如果没有给出 \xoption{atend}\ 选项,它将生成一个警告。
%    \begin{macrocode}
  \newcommand{\BookmarkAtEnd}[1]{%
    \PackageWarning{bookmark}{%
      Ignored, because option `atend' is missing%
    }%
  }%
%    \end{macrocode}
%    \end{macro}
%    \begin{macrocode}
  \expandafter\endinput
\fi
%    \end{macrocode}
%    \begin{macro}{\BookmarkAtEnd}
%    \begin{macrocode}
\newcommand*{\BookmarkAtEnd}{%
  \g@addto@macro\BKM@EndHook
}
%    \end{macrocode}
%    \end{macro}
%    \begin{macrocode}
\let\BKM@EndHook\@empty
%    \end{macrocode}
%    \begin{macrocode}
%</package>
%    \end{macrocode}
%
% \subsubsection{在文档末尾使用钩子的驱动程序}
%
%    驱动程序 \xoption{pdftex}\ 使用 LaTeX 钩子 \xoption{enddocument/afterlastpage}
%    (相当于以前使用的 \xpackage{atveryend}\ 的 \cs{AfterLastShipout}),因为它仍然需要 \xext{aux}\ 文件。
%    它使用 \cs{pdfoutline}\ 作为最后一页之后可以使用的书签(bookmakrs)。
%    \begin{itemize}
%    \item
%      驱动程序 \xoption{pdftex}\ 使用 \cs{pdfoutline}, \cs{pdfoutline}\ 可以在最后一页之后使用。
%    \end{itemize}
%    \begin{macrocode}
%<*pdftex>
\ifBKM@atend
  \AddToHook{enddocument/afterlastpage}{%
    \BKM@EndHook
  }%
\fi
%</pdftex>
%    \end{macrocode}
%
% \subsubsection{使用 \xoption{shipout/lastpage}\ 的驱动程序}
%
%    其他驱动程序使用 \cs{special}\ 命令实现 \cs{bookmark}。因此,最后的书签(last bookmarks)
%    必须放在最后一页(last page),而不是之后。不能使用 \cs{AtEndDocument},因为为时已晚,
%    最后一页已经输出了。因此,我们使用 LaTeX 钩子 \xoption{shipout/lastpage}。至少需要运行
%    两次 \hologo{LaTeX}。PostScript 驱动程序 \xoption{dvips}\ 使用外部 PostScript 文件作为书签。
%    为了避免与 pgf 发生冲突,文件写入(file writing)也被移到了最后一个输出页面(shipout page)。
%    \begin{macrocode}
%<*dvipdfm|vtex|pdfmark>
\ifBKM@atend
  \AddToHook{shipout/lastpage}{\BKM@EndHook}%
\fi
%</dvipdfm|vtex|pdfmark>
%    \end{macrocode}
%
% \section{安装(Installation)}
%
% \subsection{下载(Download)}
%
% \paragraph{宏包(Package)。} 在 CTAN\footnote{\CTANpkg{bookmark}}上提供此宏包:
% \begin{description}
% \item[\CTAN{macros/latex/contrib/bookmark/bookmark.dtx}] 源文件(source file)。
% \item[\CTAN{macros/latex/contrib/bookmark/bookmark.pdf}] 文档(documentation)。
% \end{description}
%
%
% \paragraph{捆绑包(Bundle)。} “bookmark”捆绑包(bundle)的所有宏包(packages)都可以在兼
% 容 TDS 的 ZIP 归档文件中找到。在那里,宏包已经被解包,文档文件(documentation files)已经生成。
% 文件(files)和目录(directories)遵循 TDS 标准。
% \begin{description}
% \item[\CTANinstall{install/macros/latex/contrib/bookmark.tds.zip}]
% \end{description}
% \emph{TDS}\ 是指标准的“用于 \TeX\ 文件的目录结构(Directory Structure)”(\CTANpkg{tds})。
% 名称中带有 \xfile{texmf}\ 的目录(directories)通常以这种方式组织。
%
% \subsection{捆绑包(Bundle)的安装}
%
% \paragraph{解压(Unpacking)。} 在您选择的 TDS 树(也称为 \xfile{texmf}\ 树)中解
% 压 \xfile{bookmark.tds.zip},例如(在 linux 中):
% \begin{quote}
%   |unzip bookmark.tds.zip -d ~/texmf|
% \end{quote}
%
% \subsection{宏包(Package)的安装}
%
% \paragraph{解压(Unpacking)。} \xfile{.dtx}\ 文件是一个自解压 \docstrip\ 归档文件(archive)。
% 这些文件是通过 \plainTeX\ 运行 \xfile{.dtx}\ 来提取的:
% \begin{quote}
%   \verb|tex bookmark.dtx|
% \end{quote}
%
% \paragraph{TDS.} 现在,不同的文件必须移动到安装 TDS 树(installation TDS tree)
% (也称为 \xfile{texmf}\ 树)中的不同目录中:
% \begin{quote}
% \def\t{^^A
% \begin{tabular}{@{}>{\ttfamily}l@{ $\rightarrow$ }>{\ttfamily}l@{}}
%   bookmark.sty & tex/latex/bookmark/bookmark.sty\\
%   bkm-dvipdfm.def & tex/latex/bookmark/bkm-dvipdfm.def\\
%   bkm-dvips.def & tex/latex/bookmark/bkm-dvips.def\\
%   bkm-pdftex.def & tex/latex/bookmark/bkm-pdftex.def\\
%   bkm-vtex.def & tex/latex/bookmark/bkm-vtex.def\\
%   bookmark.pdf & doc/latex/bookmark/bookmark.pdf\\
%   bookmark-example.tex & doc/latex/bookmark/bookmark-example.tex\\
%   bookmark.dtx & source/latex/bookmark/bookmark.dtx\\
% \end{tabular}^^A
% }^^A
% \sbox0{\t}^^A
% \ifdim\wd0>\linewidth
%   \begingroup
%     \advance\linewidth by\leftmargin
%     \advance\linewidth by\rightmargin
%   \edef\x{\endgroup
%     \def\noexpand\lw{\the\linewidth}^^A
%   }\x
%   \def\lwbox{^^A
%     \leavevmode
%     \hbox to \linewidth{^^A
%       \kern-\leftmargin\relax
%       \hss
%       \usebox0
%       \hss
%       \kern-\rightmargin\relax
%     }^^A
%   }^^A
%   \ifdim\wd0>\lw
%     \sbox0{\small\t}^^A
%     \ifdim\wd0>\linewidth
%       \ifdim\wd0>\lw
%         \sbox0{\footnotesize\t}^^A
%         \ifdim\wd0>\linewidth
%           \ifdim\wd0>\lw
%             \sbox0{\scriptsize\t}^^A
%             \ifdim\wd0>\linewidth
%               \ifdim\wd0>\lw
%                 \sbox0{\tiny\t}^^A
%                 \ifdim\wd0>\linewidth
%                   \lwbox
%                 \else
%                   \usebox0
%                 \fi
%               \else
%                 \lwbox
%               \fi
%             \else
%               \usebox0
%             \fi
%           \else
%             \lwbox
%           \fi
%         \else
%           \usebox0
%         \fi
%       \else
%         \lwbox
%       \fi
%     \else
%       \usebox0
%     \fi
%   \else
%     \lwbox
%   \fi
% \else
%   \usebox0
% \fi
% \end{quote}
% 如果你有一个 \xfile{docstrip.cfg}\ 文件,该文件能配置并启用 \docstrip\ 的 TDS 安装功能,
% 则一些文件可能已经在正确的位置了,请参阅 \docstrip\ 的文档(documentation)。
%
% \subsection{刷新文件名数据库}
%
% 如果您的 \TeX~发行版(\TeX\,Live、\mikTeX、\dots)依赖于文件名数据库(file name databases),
% 则必须刷新这些文件名数据库。例如,\TeX\,Live\ 用户运行 \verb|texhash| 或 \verb|mktexlsr|。
%
% \subsection{一些感兴趣的细节}
%
% \paragraph{用 \LaTeX\ 解压。}
% \xfile{.dtx}\ 根据格式(format)选择其操作(action):
% \begin{description}
% \item[\plainTeX:] 运行 \docstrip\ 并解压文件。
% \item[\LaTeX:] 生成文档。
% \end{description}
% 如果您坚持通过 \LaTeX\ 使用\docstrip (实际上 \docstrip\ 并不需要 \LaTeX),那么请您的意图告知自动检测程序:
% \begin{quote}
%   \verb|latex \let\install=y\input{bookmark.dtx}|
% \end{quote}
% 不要忘记根据 shell 的要求引用这个参数(argument)。
%
% \paragraph{知生成文档。}
% 您可以同时使用 \xfile{.dtx}\ 或 \xfile{.drv}\ 来生成文档。可以通过配置文件 \xfile{ltxdoc.cfg}\ 配置该进程。
% 例如,如果您希望 A4 作为纸张格式,请将下面这行写入此文件中:
% \begin{quote}
%   \verb|\PassOptionsToClass{a4paper}{article}|
% \end{quote}
% 下面是一个如何使用 pdf\LaTeX\ 生成文档的示例:
% \begin{quote}
%\begin{verbatim}
%pdflatex bookmark.dtx
%makeindex -s gind.ist bookmark.idx
%pdflatex bookmark.dtx
%makeindex -s gind.ist bookmark.idx
%pdflatex bookmark.dtx
%\end{verbatim}
% \end{quote}
%
% \begin{thebibliography}{9}
%
% \bibitem{hyperref}
%   Sebastian Rahtz, Heiko Oberdiek:
%   \textit{The \xpackage{hyperref} package};
%   2011/04/17 v6.82g;
%   \CTANpkg{hyperref}
%
% \bibitem{currfile}
%   Martin Scharrer:
%   \textit{The \xpackage{currfile} package};
%   2011/01/09 v0.4.
%   \CTANpkg{currfile}
%
% \end{thebibliography}
%
% \begin{History}
%   \begin{Version}{2007/02/19 v0.1}
%   \item
%     First experimental version.
%   \end{Version}
%   \begin{Version}{2007/02/20 v0.2}
%   \item
%     Option \xoption{startatroot} added.
%   \item
%     Dummies for \cs{pdf(un)escape...} commands added to get
%     the package basically work for non-\hologo{pdfTeX} users.
%   \end{Version}
%   \begin{Version}{2007/02/21 v0.3}
%   \item
%     Dependency from \hologo{pdfTeX} 1.30 removed by using package
%     \xpackage{pdfescape}.
%   \end{Version}
%   \begin{Version}{2007/02/22 v0.4}
%   \item
%     \xpackage{hyperref}'s \xoption{bookmarkstype} respected.
%   \end{Version}
%   \begin{Version}{2007/03/02 v0.5}
%   \item
%     Driver options \xoption{vtex} (PDF mode), \xoption{dvipsone},
%     and \xoption{textures} added.
%   \item
%     Implementation of option \xoption{depth} completed. Division names
%     are supported, see \xpackage{hyperref}'s
%     option \xoption{bookmarksdepth}.
%   \item
%     \xpackage{hyperref}'s options \xoption{bookmarksopen},
%     \xoption{bookmarksopenlevel}, and \xoption{bookmarksdepth} respected.
%   \end{Version}
%   \begin{Version}{2007/03/03 v0.6}
%   \item
%     Option \xoption{numbered} as alias for \xpackage{hyperref}'s
%     \xoption{bookmarksnumbered}.
%   \end{Version}
%   \begin{Version}{2007/03/07 v0.7}
%   \item
%     Dependency from \hologo{eTeX} removed.
%   \end{Version}
%   \begin{Version}{2007/04/09 v0.8}
%   \item
%     Option \xoption{atend} added.
%   \item
%     Option \xoption{rgbcolor} removed.
%     \verb|rgbcolor=<r> <g> <b>| can be replaced by
%     \verb|color=[rgb]{<r>,<g>,<b>}|.
%   \item
%     Support of recent cvs version (2007-03-29) of dvipdfmx
%     that extends the \cs{special} for bookmarks to specify
%     open outline entries. Option \xoption{dvipdfmx-outline-open}
%     or \cs{SpecialDvipdfmxOutlineOpen} notify the package.
%   \end{Version}
%   \begin{Version}{2007/04/25 v0.9}
%   \item
%     The syntax of \cs{special} of dvipdfmx, if feature
%     \xoption{dvipdfmx-outline-open} is enabled, has changed.
%     Now cvs version 2007-04-25 is needed.
%   \end{Version}
%   \begin{Version}{2007/05/29 v1.0}
%   \item
%     Bug fix in code for second parameter of XYZ.
%   \end{Version}
%   \begin{Version}{2007/07/13 v1.1}
%   \item
%     Fix for pdfmark with GoToR action.
%   \end{Version}
%   \begin{Version}{2007/09/25 v1.2}
%   \item
%     pdfmark driver respects \cs{nofiles}.
%   \end{Version}
%   \begin{Version}{2008/08/08 v1.3}
%   \item
%     Package \xpackage{flags} replaced by package \xpackage{bitset}.
%     Now flags are also supported without \hologo{eTeX}.
%   \item
%     Hook for package \xpackage{hypdestopt} added.
%   \end{Version}
%   \begin{Version}{2008/09/13 v1.4}
%   \item
%     Fix for bug introduced in v1.3, package \xpackage{flags} is one-based,
%     but package \xpackage{bitset} is zero-based. Thus options \xoption{bold}
%     and \xoption{italic} are wrong in v1.3. (Daniel M\"ullner)
%   \end{Version}
%   \begin{Version}{2009/08/13 v1.5}
%   \item
%     Except for driver options the other options are now local options.
%     This resolves a problem with KOMA-Script v3.00 and its option \xoption{open}.
%   \end{Version}
%   \begin{Version}{2009/12/06 v1.6}
%   \item
%     Use of package \xpackage{atveryend} for drivers \xoption{pdftex}
%     and \xoption{pdfmark}.
%   \end{Version}
%   \begin{Version}{2009/12/07 v1.7}
%   \item
%     Use of package \xpackage{atveryend} fixed.
%   \end{Version}
%   \begin{Version}{2009/12/17 v1.8}
%   \item
%     Support of \xpackage{hyperref} 2009/12/17 v6.79v for \hologo{XeTeX}.
%   \end{Version}
%   \begin{Version}{2010/03/30 v1.9}
%   \item
%     Package name in an error message fixed.
%   \end{Version}
%   \begin{Version}{2010/04/03 v1.10}
%   \item
%     Option \xoption{style} and macro \cs{bookmarkdefinestyle} added.
%   \item
%     Hook support with option \xoption{addtohook} added.
%   \item
%     \cs{bookmarkget} added.
%   \end{Version}
%   \begin{Version}{2010/04/04 v1.11}
%   \item
%     Bug fix (introduced in v1.10).
%   \end{Version}
%   \begin{Version}{2010/04/08 v1.12}
%   \item
%     Requires \xpackage{ltxcmds} 2010/04/08.
%   \end{Version}
%   \begin{Version}{2010/07/23 v1.13}
%   \item
%     Support for \xclass{memoir}'s \cs{booknumberline} added.
%   \end{Version}
%   \begin{Version}{2010/09/02 v1.14}
%   \item
%     (Local) options \xoption{draft} and \xoption{final} added.
%   \end{Version}
%   \begin{Version}{2010/09/25 v1.15}
%   \item
%     Fix for option \xoption{dvipdfmx-outline-open}.
%   \item
%     Option \xoption{dvipdfmx-outline-open} is set automatically,
%     if XeTeX $\geq$ 0.9995 is detected.
%   \end{Version}
%   \begin{Version}{2010/10/19 v1.16}
%   \item
%     Option `startatroot' now acts globally.
%   \item
%     Option `level' also accepts names the same way as option `depth'.
%   \end{Version}
%   \begin{Version}{2010/10/25 v1.17}
%   \item
%     \cs{bookmarksetupnext} added.
%   \item
%     Using \cs{kvsetkeys} of package \xpackage{kvsetkeys}, because
%     \cs{setkeys} of package \xpackage{keyval} is not reentrant.
%     This can cause problems (unknown keys) with older versions of
%     hyperref that also uses \cs{setkeys} (found by GL).
%   \end{Version}
%   \begin{Version}{2010/11/05 v1.18}
%   \item
%     Use of \cs{pdf@ifdraftmode} of package \xpackage{pdftexcmds} for
%     the default of option \xoption{draft}.
%   \end{Version}
%   \begin{Version}{2011/03/20 v1.19}
%   \item
%     Use of \cs{dimexpr} fixed, if \hologo{eTeX} is not used.
%     (Bug found by Martin M\"unch.)
%   \item
%     Fix in documentation. Also layout options work without \hologo{eTeX}.
%   \end{Version}
%   \begin{Version}{2011/04/13 v1.20}
%   \item
%     Bug fix: \cs{BKM@SetDepth} renamed to \cs{BKM@SetDepthOrLevel}.
%   \end{Version}
%   \begin{Version}{2011/04/21 v1.21}
%   \item
%     Some support for file name and line number in error messages
%     at end of document (pdfTeX and pdfmark based drivers).
%   \end{Version}
%   \begin{Version}{2011/05/13 v1.22}
%   \item
%     Change of version 2010/11/05 v1.18 reverted, because otherwise
%     draftmode disables some \xext{aux} file entries.
%   \end{Version}
%   \begin{Version}{2011/09/19 v1.23}
%   \item
%     Some \cs{renewcommand}s changed to \cs{def} to avoid trouble
%     if the commands are not defined, because hyperref stopped early.
%   \end{Version}
%   \begin{Version}{2011/12/02 v1.24}
%   \item
%     Small optimization in \cs{BKM@toHexDigit}.
%   \end{Version}
%   \begin{Version}{2016/05/16 v1.25}
%   \item
%     Documentation updates.
%   \end{Version}
%   \begin{Version}{2016/05/17 v1.26}
%   \item
%     define \cs{pdfoutline} to allow pdftex driver to be used with Lua\TeX.
%   \end{Version}
%   \begin{Version}{2019/06/04 v1.27}
%   \item
%     unknown style options are ignored (issue 67)
%   \end{Version}

%   \begin{Version}{2019/12/03 v1.28}
%   \item
%     Documentation updates.
%   \item adjust package loading (all required packages already loaded
%     by \xpackage{hyperref}).
%   \end{Version}
%   \begin{Version}{2020-11-06 v1.29}
%   \item Adapted the dvips to avoid a clash with pgf.
%         https://github.com/pgf-tikz/pgf/issues/944
%   \item All drivers now use the new LaTeX hooks
%         and so require a format 2020-10-01 or newer. The older
%         drivers are provided as frozen versions and are used if an older
%         format is detected.
%   \item Added support for destlabel option of hyperref, https://github.com/ho-tex/bookmark/issues/1
%   \item Removed the \xoption{dvipsone} and \xoption{textures} driver.
%   \item Removed the code for option \xoption{dvipdfmx-outline-open}
%     and \cs{SpecialDvipdfmxOutlineOpen}. All dvipdfmx version should now support
%     this out-of-the-box.
%   \end{Version}
% \end{History}
%
% \PrintIndex
%
% \Finale
\endinput
|
% \end{quote}
% 不要忘记根据 shell 的要求引用这个参数(argument)。
%
% \paragraph{知生成文档。}
% 您可以同时使用 \xfile{.dtx}\ 或 \xfile{.drv}\ 来生成文档。可以通过配置文件 \xfile{ltxdoc.cfg}\ 配置该进程。
% 例如,如果您希望 A4 作为纸张格式,请将下面这行写入此文件中:
% \begin{quote}
%   \verb|\PassOptionsToClass{a4paper}{article}|
% \end{quote}
% 下面是一个如何使用 pdf\LaTeX\ 生成文档的示例:
% \begin{quote}
%\begin{verbatim}
%pdflatex bookmark.dtx
%makeindex -s gind.ist bookmark.idx
%pdflatex bookmark.dtx
%makeindex -s gind.ist bookmark.idx
%pdflatex bookmark.dtx
%\end{verbatim}
% \end{quote}
%
% \begin{thebibliography}{9}
%
% \bibitem{hyperref}
%   Sebastian Rahtz, Heiko Oberdiek:
%   \textit{The \xpackage{hyperref} package};
%   2011/04/17 v6.82g;
%   \CTANpkg{hyperref}
%
% \bibitem{currfile}
%   Martin Scharrer:
%   \textit{The \xpackage{currfile} package};
%   2011/01/09 v0.4.
%   \CTANpkg{currfile}
%
% \end{thebibliography}
%
% \begin{History}
%   \begin{Version}{2007/02/19 v0.1}
%   \item
%     First experimental version.
%   \end{Version}
%   \begin{Version}{2007/02/20 v0.2}
%   \item
%     Option \xoption{startatroot} added.
%   \item
%     Dummies for \cs{pdf(un)escape...} commands added to get
%     the package basically work for non-\hologo{pdfTeX} users.
%   \end{Version}
%   \begin{Version}{2007/02/21 v0.3}
%   \item
%     Dependency from \hologo{pdfTeX} 1.30 removed by using package
%     \xpackage{pdfescape}.
%   \end{Version}
%   \begin{Version}{2007/02/22 v0.4}
%   \item
%     \xpackage{hyperref}'s \xoption{bookmarkstype} respected.
%   \end{Version}
%   \begin{Version}{2007/03/02 v0.5}
%   \item
%     Driver options \xoption{vtex} (PDF mode), \xoption{dvipsone},
%     and \xoption{textures} added.
%   \item
%     Implementation of option \xoption{depth} completed. Division names
%     are supported, see \xpackage{hyperref}'s
%     option \xoption{bookmarksdepth}.
%   \item
%     \xpackage{hyperref}'s options \xoption{bookmarksopen},
%     \xoption{bookmarksopenlevel}, and \xoption{bookmarksdepth} respected.
%   \end{Version}
%   \begin{Version}{2007/03/03 v0.6}
%   \item
%     Option \xoption{numbered} as alias for \xpackage{hyperref}'s
%     \xoption{bookmarksnumbered}.
%   \end{Version}
%   \begin{Version}{2007/03/07 v0.7}
%   \item
%     Dependency from \hologo{eTeX} removed.
%   \end{Version}
%   \begin{Version}{2007/04/09 v0.8}
%   \item
%     Option \xoption{atend} added.
%   \item
%     Option \xoption{rgbcolor} removed.
%     \verb|rgbcolor=<r> <g> <b>| can be replaced by
%     \verb|color=[rgb]{<r>,<g>,<b>}|.
%   \item
%     Support of recent cvs version (2007-03-29) of dvipdfmx
%     that extends the \cs{special} for bookmarks to specify
%     open outline entries. Option \xoption{dvipdfmx-outline-open}
%     or \cs{SpecialDvipdfmxOutlineOpen} notify the package.
%   \end{Version}
%   \begin{Version}{2007/04/25 v0.9}
%   \item
%     The syntax of \cs{special} of dvipdfmx, if feature
%     \xoption{dvipdfmx-outline-open} is enabled, has changed.
%     Now cvs version 2007-04-25 is needed.
%   \end{Version}
%   \begin{Version}{2007/05/29 v1.0}
%   \item
%     Bug fix in code for second parameter of XYZ.
%   \end{Version}
%   \begin{Version}{2007/07/13 v1.1}
%   \item
%     Fix for pdfmark with GoToR action.
%   \end{Version}
%   \begin{Version}{2007/09/25 v1.2}
%   \item
%     pdfmark driver respects \cs{nofiles}.
%   \end{Version}
%   \begin{Version}{2008/08/08 v1.3}
%   \item
%     Package \xpackage{flags} replaced by package \xpackage{bitset}.
%     Now flags are also supported without \hologo{eTeX}.
%   \item
%     Hook for package \xpackage{hypdestopt} added.
%   \end{Version}
%   \begin{Version}{2008/09/13 v1.4}
%   \item
%     Fix for bug introduced in v1.3, package \xpackage{flags} is one-based,
%     but package \xpackage{bitset} is zero-based. Thus options \xoption{bold}
%     and \xoption{italic} are wrong in v1.3. (Daniel M\"ullner)
%   \end{Version}
%   \begin{Version}{2009/08/13 v1.5}
%   \item
%     Except for driver options the other options are now local options.
%     This resolves a problem with KOMA-Script v3.00 and its option \xoption{open}.
%   \end{Version}
%   \begin{Version}{2009/12/06 v1.6}
%   \item
%     Use of package \xpackage{atveryend} for drivers \xoption{pdftex}
%     and \xoption{pdfmark}.
%   \end{Version}
%   \begin{Version}{2009/12/07 v1.7}
%   \item
%     Use of package \xpackage{atveryend} fixed.
%   \end{Version}
%   \begin{Version}{2009/12/17 v1.8}
%   \item
%     Support of \xpackage{hyperref} 2009/12/17 v6.79v for \hologo{XeTeX}.
%   \end{Version}
%   \begin{Version}{2010/03/30 v1.9}
%   \item
%     Package name in an error message fixed.
%   \end{Version}
%   \begin{Version}{2010/04/03 v1.10}
%   \item
%     Option \xoption{style} and macro \cs{bookmarkdefinestyle} added.
%   \item
%     Hook support with option \xoption{addtohook} added.
%   \item
%     \cs{bookmarkget} added.
%   \end{Version}
%   \begin{Version}{2010/04/04 v1.11}
%   \item
%     Bug fix (introduced in v1.10).
%   \end{Version}
%   \begin{Version}{2010/04/08 v1.12}
%   \item
%     Requires \xpackage{ltxcmds} 2010/04/08.
%   \end{Version}
%   \begin{Version}{2010/07/23 v1.13}
%   \item
%     Support for \xclass{memoir}'s \cs{booknumberline} added.
%   \end{Version}
%   \begin{Version}{2010/09/02 v1.14}
%   \item
%     (Local) options \xoption{draft} and \xoption{final} added.
%   \end{Version}
%   \begin{Version}{2010/09/25 v1.15}
%   \item
%     Fix for option \xoption{dvipdfmx-outline-open}.
%   \item
%     Option \xoption{dvipdfmx-outline-open} is set automatically,
%     if XeTeX $\geq$ 0.9995 is detected.
%   \end{Version}
%   \begin{Version}{2010/10/19 v1.16}
%   \item
%     Option `startatroot' now acts globally.
%   \item
%     Option `level' also accepts names the same way as option `depth'.
%   \end{Version}
%   \begin{Version}{2010/10/25 v1.17}
%   \item
%     \cs{bookmarksetupnext} added.
%   \item
%     Using \cs{kvsetkeys} of package \xpackage{kvsetkeys}, because
%     \cs{setkeys} of package \xpackage{keyval} is not reentrant.
%     This can cause problems (unknown keys) with older versions of
%     hyperref that also uses \cs{setkeys} (found by GL).
%   \end{Version}
%   \begin{Version}{2010/11/05 v1.18}
%   \item
%     Use of \cs{pdf@ifdraftmode} of package \xpackage{pdftexcmds} for
%     the default of option \xoption{draft}.
%   \end{Version}
%   \begin{Version}{2011/03/20 v1.19}
%   \item
%     Use of \cs{dimexpr} fixed, if \hologo{eTeX} is not used.
%     (Bug found by Martin M\"unch.)
%   \item
%     Fix in documentation. Also layout options work without \hologo{eTeX}.
%   \end{Version}
%   \begin{Version}{2011/04/13 v1.20}
%   \item
%     Bug fix: \cs{BKM@SetDepth} renamed to \cs{BKM@SetDepthOrLevel}.
%   \end{Version}
%   \begin{Version}{2011/04/21 v1.21}
%   \item
%     Some support for file name and line number in error messages
%     at end of document (pdfTeX and pdfmark based drivers).
%   \end{Version}
%   \begin{Version}{2011/05/13 v1.22}
%   \item
%     Change of version 2010/11/05 v1.18 reverted, because otherwise
%     draftmode disables some \xext{aux} file entries.
%   \end{Version}
%   \begin{Version}{2011/09/19 v1.23}
%   \item
%     Some \cs{renewcommand}s changed to \cs{def} to avoid trouble
%     if the commands are not defined, because hyperref stopped early.
%   \end{Version}
%   \begin{Version}{2011/12/02 v1.24}
%   \item
%     Small optimization in \cs{BKM@toHexDigit}.
%   \end{Version}
%   \begin{Version}{2016/05/16 v1.25}
%   \item
%     Documentation updates.
%   \end{Version}
%   \begin{Version}{2016/05/17 v1.26}
%   \item
%     define \cs{pdfoutline} to allow pdftex driver to be used with Lua\TeX.
%   \end{Version}
%   \begin{Version}{2019/06/04 v1.27}
%   \item
%     unknown style options are ignored (issue 67)
%   \end{Version}

%   \begin{Version}{2019/12/03 v1.28}
%   \item
%     Documentation updates.
%   \item adjust package loading (all required packages already loaded
%     by \xpackage{hyperref}).
%   \end{Version}
%   \begin{Version}{2020-11-06 v1.29}
%   \item Adapted the dvips to avoid a clash with pgf.
%         https://github.com/pgf-tikz/pgf/issues/944
%   \item All drivers now use the new LaTeX hooks
%         and so require a format 2020-10-01 or newer. The older
%         drivers are provided as frozen versions and are used if an older
%         format is detected.
%   \item Added support for destlabel option of hyperref, https://github.com/ho-tex/bookmark/issues/1
%   \item Removed the \xoption{dvipsone} and \xoption{textures} driver.
%   \item Removed the code for option \xoption{dvipdfmx-outline-open}
%     and \cs{SpecialDvipdfmxOutlineOpen}. All dvipdfmx version should now support
%     this out-of-the-box.
%   \end{Version}
% \end{History}
%
% \PrintIndex
%
% \Finale
\endinput

%        (quote the arguments according to the demands of your shell)
%
% Documentation:
%    (a) If bookmark.drv is present:
%           latex bookmark.drv
%    (b) Without bookmark.drv:
%           latex bookmark.dtx; ...
%    The class ltxdoc loads the configuration file ltxdoc.cfg
%    if available. Here you can specify further options, e.g.
%    use A4 as paper format:
%       \PassOptionsToClass{a4paper}{article}
%
%    Programm calls to get the documentation (example):
%       pdflatex bookmark.dtx
%       makeindex -s gind.ist bookmark.idx
%       pdflatex bookmark.dtx
%       makeindex -s gind.ist bookmark.idx
%       pdflatex bookmark.dtx
%
% Installation:
%    TDS:tex/latex/bookmark/bookmark.sty
%    TDS:tex/latex/bookmark/bkm-dvipdfm.def
%    TDS:tex/latex/bookmark/bkm-dvips.def
%    TDS:tex/latex/bookmark/bkm-pdftex.def
%    TDS:tex/latex/bookmark/bkm-vtex.def
%    TDS:tex/latex/bookmark/bkm-dvipdfm-2019-12-03.def
%    TDS:tex/latex/bookmark/bkm-dvips-2019-12-03.def
%    TDS:tex/latex/bookmark/bkm-pdftex-2019-12-03.def
%    TDS:tex/latex/bookmark/bkm-vtex-2019-12-03.def%
%    TDS:doc/latex/bookmark/bookmark.pdf
%    TDS:doc/latex/bookmark/bookmark-example.tex
%    TDS:source/latex/bookmark/bookmark.dtx
%    TDS:source/latex/bookmark/bookmark-frozen.dtx
%
%<*ignore>
\begingroup
  \catcode123=1 %
  \catcode125=2 %
  \def\x{LaTeX2e}%
\expandafter\endgroup
\ifcase 0\ifx\install y1\fi\expandafter
         \ifx\csname processbatchFile\endcsname\relax\else1\fi
         \ifx\fmtname\x\else 1\fi\relax
\else\csname fi\endcsname
%</ignore>
%<*install>
\input docstrip.tex
\Msg{************************************************************************}
\Msg{* Installation}
\Msg{* Package: bookmark 2020-11-06 v1.29 PDF bookmarks (HO)}
\Msg{************************************************************************}

\keepsilent
\askforoverwritefalse

\let\MetaPrefix\relax
\preamble

This is a generated file.

Project: bookmark
Version: 2020-11-06 v1.29

Copyright (C)
   2007-2011 Heiko Oberdiek
   2016-2020 Oberdiek Package Support Group

This work may be distributed and/or modified under the
conditions of the LaTeX Project Public License, either
version 1.3c of this license or (at your option) any later
version. This version of this license is in
   https://www.latex-project.org/lppl/lppl-1-3c.txt
and the latest version of this license is in
   https://www.latex-project.org/lppl.txt
and version 1.3 or later is part of all distributions of
LaTeX version 2005/12/01 or later.

This work has the LPPL maintenance status "maintained".

The Current Maintainers of this work are
Heiko Oberdiek and the Oberdiek Package Support Group
https://github.com/ho-tex/bookmark/issues


This work consists of the main source file bookmark.dtx and bookmark-frozen.dtx
and the derived files
   bookmark.sty, bookmark.pdf, bookmark.ins, bookmark.drv,
   bkm-dvipdfm.def, bkm-dvips.def, bkm-pdftex.def, bkm-vtex.def,
   bkm-dvipdfm-2019-12-03.def, bkm-dvips-2019-12-03.def,
   bkm-pdftex-2019-12-03.def, bkm-vtex-2019-12-03.def,
   bookmark-example.tex.

\endpreamble
\let\MetaPrefix\DoubleperCent

\generate{%
  \file{bookmark.ins}{\from{bookmark.dtx}{install}}%
  \file{bookmark.drv}{\from{bookmark.dtx}{driver}}%
  \usedir{tex/latex/bookmark}%
  \file{bookmark.sty}{\from{bookmark.dtx}{package}}%
  \file{bkm-dvipdfm.def}{\from{bookmark.dtx}{dvipdfm}}%
  \file{bkm-dvips.def}{\from{bookmark.dtx}{dvips,pdfmark}}%
  \file{bkm-pdftex.def}{\from{bookmark.dtx}{pdftex}}%
  \file{bkm-vtex.def}{\from{bookmark.dtx}{vtex}}%
  \usedir{doc/latex/bookmark}%
  \file{bookmark-example.tex}{\from{bookmark.dtx}{example}}%
  \file{bkm-pdftex-2019-12-03.def}{\from{bookmark-frozen.dtx}{pdftexfrozen}}%
  \file{bkm-dvips-2019-12-03.def}{\from{bookmark-frozen.dtx}{dvipsfrozen}}%
  \file{bkm-vtex-2019-12-03.def}{\from{bookmark-frozen.dtx}{vtexfrozen}}%
  \file{bkm-dvipdfm-2019-12-03.def}{\from{bookmark-frozen.dtx}{dvipdfmfrozen}}%
}

\catcode32=13\relax% active space
\let =\space%
\Msg{************************************************************************}
\Msg{*}
\Msg{* To finish the installation you have to move the following}
\Msg{* files into a directory searched by TeX:}
\Msg{*}
\Msg{*     bookmark.sty, bkm-dvipdfm.def, bkm-dvips.def,}
\Msg{*     bkm-pdftex.def, bkm-vtex.def, bkm-dvipdfm-2019-12-03.def,}
\Msg{*     bkm-dvips-2019-12-03.def, bkm-pdftex-2019-12-03.def,}
\Msg{*     and bkm-vtex-2019-12-03.def}
\Msg{*}
\Msg{* To produce the documentation run the file `bookmark.drv'}
\Msg{* through LaTeX.}
\Msg{*}
\Msg{* Happy TeXing!}
\Msg{*}
\Msg{************************************************************************}

\endbatchfile
%</install>
%<*ignore>
\fi
%</ignore>
%<*driver>
\NeedsTeXFormat{LaTeX2e}
\ProvidesFile{bookmark.drv}%
  [2020-11-06 v1.29 PDF bookmarks (HO)]%
\documentclass{ltxdoc}
\usepackage{ctex}
\usepackage{indentfirst}
\setlength{\parindent}{2em}
\usepackage{holtxdoc}[2011/11/22]
\usepackage{xcolor}
\usepackage{hyperref}
\usepackage[open,openlevel=3,atend]{bookmark}[2020/11/06] %%%打开书签,显示的深度为3级,即显示part、section、subsection。
\bookmarksetup{color=red}
\begin{document}

  \renewcommand{\contentsname}{目\quad 录}
  \renewcommand{\abstractname}{摘\quad 要}
  \renewcommand{\historyname}{历史}
  \DocInput{bookmark.dtx}%
\end{document}
%</driver>
% \fi
%
%
%
% \GetFileInfo{bookmark.drv}
%
%% \title{\xpackage{bookmark} 宏包}
% \title{\heiti {\Huge \textbf{\xpackage{bookmark}\ 宏包}}}
% \date{2020-11-06\ \ \ v1.29}
% \author{Heiko Oberdiek \thanks
% {如有问题请点击:\url{https://github.com/ho-tex/bookmark/issues}}\\[5pt]赣医一附院神经科\ \ 黄旭华\ \ \ \ 译}
%
% \maketitle
%
% \begin{abstract}
% 这个宏包为 \xpackage{hyperref}\ 宏包实现了一个新的书签(bookmark)(大纲[outline])组织。现在
% 可以设置样式(style)和颜色(color)等书签属性(bookmark properties)。其他动作类型(action types)可用
% (URI、GoToR、Named)。书签是在第一次编译运行(compile run)中生成的。\xpackage{hyperref}\
% 宏包必需运行两次。
% \end{abstract}
%
% \tableofcontents
%
% \section{文档(Documentation)}
%
% \subsection{介绍}
%
% 这个 \xpackage{bookmark}\ 宏包试图为书签(bookmarks)提供一个更现代的管理:
% \begin{itemize}
% \item 书签已经在第一次 \hologo{TeX}\ 编译运行(compile run)中生成。
% \item 可以更改书签的字体样式(font style)和颜色(color)。
% \item 可以执行比简单的 GoTo 操作(actions)更多的操作。
% \end{itemize}
%
% 与 \xpackage{hyperref} \cite{hyperref} 一样,书签(bookmarks)也是按照书签生成宏
% (bookmark generating macros)(\cs{bookmark})的顺序生成的。级别号(level number)用于
% 定义书签的树结构(tree structure)。限制没有那么严格:
% \begin{itemize}
% \item 级别值(level values)可以跳变(jump)和省略(omit)。\cs{subsubsection}\ 可以跟在
%       \cs{chapter}\ 之后。这种情况如在 \xpackage{hyperref}\ 中则产生错误,它将显示一个警告(warning)
%       并尝试修复此错误。
% \item 多个书签可能指向同一目标(destination)。在 \xpackage{hyperref}\ 中,这会完全弄乱
%       书签树(bookmark tree),因为算法假设(algorithm assumes)目标名称(destination names)
%       是键(keys)(唯一的)。
% \end{itemize}
%
% 注意,这个宏包是作为书签管理(bookmark management)的实验平台(experimentation platform)。
% 欢迎反馈。此外,在未来的版本中,接口(interfaces)也可能发生变化。
%
% \subsection{选项(Options)}
%
% 可在以下四个地方放置选项(options):
% \begin{enumerate}
% \item \cs{usepackage}|[|\meta{options}|]{bookmark}|\\
%       这是放置驱动程序选项(driver options)和 \xoption{atend}\ 选项的唯一位置。
% \item \cs{bookmarksetup}|{|\meta{options}|}|\\
%       此命令仅用于设置选项(setting options)。
% \item \cs{bookmarksetupnext}|{|\meta{options}|}|\\
%       这些选项在下一个 \cs{bookmark}\ 命令的选项之后存储(stored)和调用(called)。
% \item \cs{bookmark}|[|\meta{options}|]{|\meta{title}|}|\\
%       此命令设置书签。选项设置(option settings)仅限于此书签。
% \end{enumerate}
% 异常(Exception):加载该宏包后,无法更改驱动程序选项(Driver options)、\xoption{atend}\ 选项
% 、\xoption{draft}\slash\xoption{final}选项。
%
% \subsubsection{\xoption{draft} 和 \xoption{final}\ 选项}
%
% 如果一个\LaTeX\ 文件要被编译了多次,那么可以使用 \xoption{draft}\ 选项来禁用该宏包的书签内
% 容(bookmark stuff),这样可以节省一点时间。默认 \xoption{final}\ 选项。两个选项都是
% 布尔选项(boolean options),如果没有值,则使用值 |true|。|draft=true| 与 |final=false| 相同。
%
% 除了驱动程序选项(driver options)之外,\xpackage{bookmark}\ 宏包选项都是局部选项(local options)。
% \xoption{draft}\ 选项和 \xoption{final}\ 选项均属于文档类选项(class option)(译者注:文档类选项为全局选项),
% 因此,在 \xpackage{bookmark}\ 宏包中未能看到这两个选项。如果您想使用全局的(global) \xoption{draft}选项
% 来优化第一次 \LaTeX\ 运行(runs),可以在导言(preamble)中引入 \xpackage{ifdraft}\ 宏包并设置 \LaTeX\ 的
% \cs{PassOptionsToPackage},例如:
%\begin{quote}
%\begin{verbatim}
%\documentclass[draft]{article}
%\usepackage{ifdraft}
%\ifdraft{%
%   \PassOptionsToPackage{draft}{bookmark}%
%}{}
%\end{verbatim}
%\end{quote}
%
% \subsubsection{驱动程序选项(Driver options)}
%
% 支持的驱动程序( drivers)包括 \xoption{pdftex}、\xoption{dvips}、\xoption{dvipdfm} (\xoption{xetex})、
% \xoption{vtex}。\hologo{TeX}\ 引擎 \hologo{pdfTeX}、\hologo{XeTeX}、\hologo{VTeX}\ 能被自动检测到。
% 默认的 DVI 驱动程序是 \xoption{dvips}。这可以通过 \cs{BookmarkDriverDefault}\ 在配置
% 文件 \xfile{bookmark.cfg}\ 中进行更改,例如:
% \begin{quote}
% |\def\BookmarkDriverDefault{dvipdfm}|
% \end{quote}
% 当前版本的(current versions)驱动程序使用新的 \LaTeX\ 钩子(\LaTeX-hooks)。如果检测到比
% 2020-10-01 更旧的格式,则将以前驱动程序的冻结版本(frozen versions)作为备份(fallback)。
%
% \paragraph{用 dvipdfmx 打开书签(bookmarks)。}旧版本的宏包有一个 \xoption{dvipdfmx-outline-open}\ 选项
% 可以激活代码,而该代码可以指定一个大纲条目(outline entry)是否打开。该宏包现在假设所有使用的 dvipdfmx 版本都是
% 最新版本,足以理解该代码,因此始终激活该代码。选项本身将被忽略。
%
%
% \subsubsection{布局选项(Layout options)}
%
% \paragraph{字体(Font)选项:}
%
% \begin{description}
% \item[\xoption{bold}:] 如果受 PDF 浏览器(PDF viewer)支持,书签将以粗体字体(bold font)显示(自 PDF 1.4起)。
% \item[\xoption{italic}:] 使用斜体字体(italic font)(自 PDF 1.4起)。
% \end{description}
% \xoption{bold}(粗体) 和 \xoption{italic}(斜体)可以同时使用。而 |false| 值(value)禁用字体选项。
%
% \paragraph{颜色(Color)选项:}
%
% 彩色书签(Colored bookmarks)是 PDF 1.4 的一个特性(feature),并非所有的 PDF 浏览器(PDF viewers)都支持彩色书签。
% \begin{description}
% \item[\xoption{color}:] 这里 color(颜色)可以作为 \xpackage{color}\ 宏包或 \xpackage{xcolor}\ 宏包的
% 颜色规范(color specification)给出。空值(empty value)表示未设置颜色属性。如果未加载 \xpackage{xcolor}\ 宏包,
% 能识别的值(recognized values)只有:
%   \begin{itemize}
%   \item 空值(empty value)表示未设置颜色属性,\\
%         例如:|color={}|
%   \item 颜色模型(color model) rgb 的显式颜色规范(explicit color specification),\\
%         例如,红色(red):|color=[rgb]{1,0,0}|
%   \item 颜色模型(color model)灰(gray)的显式颜色规范(explicit color specification),\\
%         例如,深灰色(dark gray):|color=[gray]{0.25}|
%   \end{itemize}
%   请注意,如果加载了 \xpackage{color}\ 宏包,此限制(restriction)也适用。然而,如果加载了 \xpackage{xcolor}\ 宏包,
%   则可以使用所有颜色规范(color specifications)。
% \end{description}
%
% \subsubsection{动作选项(Action options)}
%
% \begin{description}
% \item[\xoption{dest}:] 目的地名称(destination name)。
% \item[\xoption{page}:] 页码(page number),第一页(first page)为 1。
% \item[\xoption{view}:] 浏览规范(view specification),示例如下:\\
%   |view={FitB}|, |view={FitH 842}|, |view={XYZ 0 100 null}|\ \  一些浏览规范参数(view specification parameters)
%   将数字(numbers)视为具有单位 bp 的参数。它们可以作为普通数字(plain numbers)或在 \cs{calc}\ 内部以
%   长度表达式(length expressions)给出。如果加载了 \xpackage{calc}\ 宏包,则支持该宏包的表达式(expressions)。否则,
%   使用 \hologo{eTeX}\ 的 \cs{dimexpr}。例如:\\
%   |view={FitH \calc{\paperheight-\topmargin-1in}}|\\
%   |view={XYZ 0 \calc{\paperheight} null}|\\
%   注意 \cs{calc}\ 不能用于 |XYZ| 的第三个参数,因为该参数是缩放值(zoom value),而不是长度(length)。

% \item[\xoption{named}:] 已命名的动作(Named action)的名称:\\
%   |FirstPage|(第一页),|LastPage|(最后一页),|NextPage|(下一页),|PrevPage|(前一页)
% \item[\xoption{gotor}:] 外部(external) PDF 文件的名称。
% \item[\xoption{uri}:] URI 规范(URI specification)。
% \item[\xoption{rawaction}:] 原始动作规范(raw action specification)。由于这些规范取决于驱动程序(driver),因此不应使用此选项。
% \end{description}
% 通过分析指定的选项来选择书签的适当动作。动作由不同的选项集(sets of options)区分:
% \begin{quote}
 \begin{tabular}{|@{}r|l@{}|}
%   \hline
%   \ \textbf{动作(Action)}\  & \ \textbf{选项(Options)}\ \\ \hline
%   \ \textsf{GoTo}\  &\  \xoption{dest}\ \\ \hline
%   \ \textsf{GoTo}\  & \ \xoption{page} + \xoption{view}\ \\ \hline
%   \ \textsf{GoToR}\  & \ \xoption{gotor} + \xoption{dest}\ \\ \hline
%   \ \textsf{GoToR}\  & \ \xoption{gotor} + \xoption{page} + \xoption{view}\ \ \ \\ \hline
%   \ \textsf{Named}\  &\  \xoption{named}\ \\ \hline
%   \ \textsf{URI}\  & \ \xoption{uri}\ \\ \hline
% \end{tabular}
% \end{quote}
%
% \paragraph{缺少动作(Missing actions)。}
% 如果动作缺少 \xpackage{bookmark}\ 宏包,则抛出错误消息(error message)。根据驱动程序(driver)
% (\xoption{pdftex}、\xoption{dvips}\ 和好友[friends]),宏包在文档末尾很晚才检测到它。
% 自 2011/04/21 v1.21 版本以后,该宏包尝试打印 \cs{bookmark}\ 的相应出现的行号(line number)和文件名(file name)。
% 然而,\hologo{TeX}\ 确实提供了行号,但不幸的是,文件名是一个秘密(secret)。但该宏包有如下获取文件名的方法:
% \begin{itemize}
% \item 如果 \hologo{LuaTeX} (独立于 DVI 或 PDF 模式)正在运行,则自动使用其 |status.filename|。
% \item 宏包的 \cs{currfile} \cite{currfile}\ 重新定义了 \hologo{LaTeX}\ 的内部结构,以跟踪文件名(file name)。
% 如果加载了该宏包,那么它的 \cs{currfilepath}\ 将被检测到并由 \xpackage{bookmark}\ 自动使用。
% \item 可以通过 \cs{bookmarksetup}\ 或 \cs{bookmark}\ 中的 \xoption{scrfile}\ 选项手动设置(set manually)文件名。
% 但是要小心,手动设置会禁用以前的文件名检测方法。错误的(wrong)或丢失的(missed)文件名设置(file name setting)可能会在错误消息中
% 为您提供错误的源位置(source location)。
% \end{itemize}
%
% \subsubsection{级别选项(Level options)}
%
% 书签条目(bookmark entries)的顺序由 \cs{bookmark}\ 命令的的出现顺序(appearance order)定义。
% 树结构(tree structure)由书签节点(bookmark nodes)的属性 \xoption{level}(级别)构建。
% \xoption{level}\ 的值是整数(integers)。如果书签条目级别的值高于前一个节点,则该条目将成为
% 前一个节点的子(child)节点。差值的绝对值并不重要。
%
% \xpackage{bookmark}\ 宏包能记住全局属性(global property)“current level(当前级别)”中上
% 一个书签条目(previous bookmark entry)的级别。
%
% 级别系统的(level system)行为(behaviour)可以通过以下选项进行配置:
% \begin{description}
% \item[\xoption{level}:]
%    设置级别(level),请参阅上面的说明。如果给出的选项 \xoption{level}\ 没有值,那么将恢复默
%    认行为,即将“当前级别(current level)”用作级别值(level value)。自 2010/10/19 v1.16 版本以来,
%    如果宏 \cs{toclevel@part}、\cs{toclevel@section}\ 被定义过(通过 \xpackage{hyperref}\ 宏包完成,
%    请参阅它的 \xoption{bookmarkdepth}\ 选项),则 \xpackage{bookmark}\ 宏包还支持 |part|、|section| 等名称。
%
% \item[\xoption{rellevel}:]
%    设置相对于前一级别的(previous level)级别。正值表示书签条目成为前一个书签条目的子条目。
% \item[\xoption{keeplevel}:]
%    使用由\xoption{level}\ 或 \xoption{rellevel}\ 设置的级别,但不要更改全局属性“current level(当前级别)”。
%    可以通过设置为 |false| 来禁用该选项。
% \item[\xoption{startatroot}:]
%    此时,书签树(bookmark tree)再次从顶层(top level)开始。下一个书签条目不会作为上一个条目的子条目进行排序。
%    示例场景:文档使用 part。但是,最后几章(last chapters)不应放在最后一部分(last part)下面:
%    \begin{quote}
%\begin{verbatim}
%\documentclass{book}
%[...]
%\begin{document}
%  \part{第一部分}
%    \chapter{第一部分的第1章}
%    [...]
%  \part{第二部分(Second part)}
%    \chapter{第二部分的第1章}
%    [...]
%  \bookmarksetup{startatroot}
%  \chapter{Index}% 不属于第二部分
%\end{document}
%\end{verbatim}
%    \end{quote}
% \end{description}
%
% \subsubsection{样式定义(Style definitions)}
%
% 样式(style)是一组选项设置(option settings)。它可以由宏 \cs{bookmarkdefinestyle}\ 定义,
% 并由它的 \xoption{style}\ 选项使用。
% \begin{declcs}{bookmarkdefinestyle} \M{name} \M{key value list}
% \end{declcs}
% 选项设置(option settings)的 \meta{key value list}(键值列表)被指定为样式名(style \meta{name})。
%
% \begin{description}
% \item[\xoption{style}:]
%   \xoption{style}\ 选项的值是以前定义的样式的名称(name)。现在执行其选项设置(option settings)。
%   选项可以包括 \xoption{style}\ 选项。通过递归调用相同样式的无限递归(endless recursion)被阻止并抛出一个错误。
% \end{description}
%
% \subsubsection{钩子支持(Hook support)}
%
% 处理宏\cs{bookmark}\ 的可选选项(optional options)后,就会调用钩子(hook)。
% \begin{description}
% \item[\xoption{addtohook}:]
%   代码(code)作为该选项的值添加到钩子中。
% \end{description}
%
% \begin{declcs}{bookmarkget} \M{option}
% \end{declcs}
% \cs{bookmarkget}\ 宏提取 \meta{option}\ 选项的最新选项设置(latest option setting)的值。
% 对于布尔选项(boolean option),如果启用布尔选项,则返回 1,否则结果为零。结果数字(resulting numbers)
% 可以直接用于 \cs{ifnum}\ 或 \cs{ifcase}。如果您想要数字 \texttt{0}\ 和 \texttt{1},
% 请在 \cs{bookmarkget}\ 前面加上 \cs{number}\ 作为前缀。\cs{bookmarkget}\ 宏是可展开的(expandable)。
% 如果选项不受支持,则返回空字符串(empty string)。受支持的布尔选项有:
% \begin{quote}
%   \xoption{bold}、
%   \xoption{italic}、
%   \xoption{open}
% \end{quote}
% 其他受支持的选项有:
% \begin{quote}
%   \xoption{depth}、
%   \xoption{dest}、
%   \xoption{color}、
%   \xoption{gotor}、
%   \xoption{level}、
%   \xoption{named}、
%   \xoption{openlevel}、
%   \xoption{page}、
%   \xoption{rawaction}、
%   \xoption{uri}、
%   \xoption{view}、
% \end{quote}
% 另外,以下键(key)是可用的:
% \begin{quote}
%   \xoption{text}
% \end{quote}
% 它返回大纲条目(outline entry)的文本(text)。
%
% \paragraph{选项设置(Option setting)。}
% 在钩子(hook)内部可以使用 \cs{bookmarksetup}\ 设置选项。
%
% \subsection{与 \xpackage{hyperref}\ 的兼容性}
%
% \xpackage{bookmark}\ 宏包自动禁用 \xpackage{hyperref}\ 宏包的书签(bookmarks)。但是,
% \xpackage{bookmark}\ 宏包使用了 \xpackage{hyperref}\ 宏包的一些代码。例如,
% \xpackage{bookmark}\ 宏包重新定义了 \xpackage{hyperref}\ 宏包在 \cs{addcontentsline}\ 命令
% 和其他命令中插入的\cs{Hy@writebookmark}\ 钩子。因此,不应禁用 \xpackage{hyperref}\ 宏包的书签。
%
% \xpackage{bookmark}\ 宏包使用 \xpackage{hyperref}\ 宏包的 \cs{pdfstringdef},且不提供替换(replacement)。
%
% \xpackage{hyperref}\ 宏包的一些选项也能在 \xpackage{bookmark}\ 宏包中实现(implemented):
% \begin{quote}
% \begin{tabular}{|l@{}|l@{}|}
%   \hline
%   \xpackage{hyperref}\ 宏包的选项\  &\ \xpackage{bookmark}\ 宏包的选项\ \ \\ \hline
%   \xoption{bookmarksdepth} &\ \xoption{depth}\\ \hline
%   \xoption{bookmarksopen} & \ \xoption{open}\\ \hline
%   \xoption{bookmarksopenlevel}\ \ \  &\ \xoption{openlevel}\\ \hline
%   \xoption{bookmarksnumbered} \ \ \ &\ \xoption{numbered}\\ \hline
% \end{tabular}
% \end{quote}
%
% 还可以使用以下命令:
% \begin{quote}
%   \cs{pdfbookmark}\\
%   \cs{currentpdfbookmark}\\
%   \cs{subpdfbookmark}\\
%   \cs{belowpdfbookmark}
% \end{quote}
%
% \subsection{在末尾添加书签}
%
% 宏包选项 \xoption{atend}\ 启用以下宏(macro):
% \begin{declcs}{BookmarkAtEnd}
%   \M{stuff}
% \end{declcs}
% \cs{BookmarkAtEnd}\ 宏将 \meta{stuff}\ 放在文档末尾。\meta{stuff}\ 表示书签命令(bookmark commands)。举例:
% \begin{quote}
%\begin{verbatim}
%\usepackage[atend]{bookmark}
%\BookmarkAtEnd{%
%  \bookmarksetup{startatroot}%
%  \bookmark[named=LastPage, level=0]{Last page}%
%}
%\end{verbatim}
% \end{quote}
%
% 或者,可以在 \cs{bookmark}\ 中给出 \xoption{startatroot}\ 选项:
% \begin{quote}
%\begin{verbatim}
%\BookmarkAtEnd{%
%  \bookmark[
%    startatroot,
%    named=LastPage,
%    level=0,
%  ]{Last page}%
%}
%\end{verbatim}
% \end{quote}
%
% \paragraph{备注(Remarks):}
% \begin{itemize}
% \item
%   \cs{BookmarkAtEnd} 隐藏了这样一个事实,即在文档末尾添加书签的方法取决于驱动程序(driver)。
%
%   为此,驱动程序 \xoption{pdftex}\ 使用 \xpackage{atveryend}\ 宏包。如果 \cs{AtEndDocument}\ 太早,
%   最后一个页面(last page)可能不会被发送出去(shipped out)。由于需要 \xext{aux}\ 文件,此驱动程序使
%   用 \cs{AfterLastShipout}。
%
%   其他驱动程序(\xoption{dvipdfm}、\xoption{xetex}、\xoption{vtex})的实现(implementation)
%   取决于 \cs{special},\cs{special}\ 在最后一页之后没有效果。在这种情况下,\xpackage{atenddvi}\ 宏包的
%   \cs{AtEndDvi}\ 有帮助。它将其参数(argument)放在文档的最后一页(last page)。至少需要运行 \hologo{LaTeX}\ 两次,
%   因为最后一页是由引用(reference)检测到的。
%
%   \xoption{dvips}\ 现在使用新的 LaTeX 钩子 \texttt{shipout/lastpage}。
% \item
%   未指定 \cs{BookmarkAtEnd}\ 参数的扩展时间(time of expansion)。这可以立即发生,也可以在文档末尾发生。
% \end{itemize}
%
% \subsection{限制/行动计划}
%
% \begin{itemize}
% \item 支持缺失动作(missing actions)(启动,\dots)。
% \item 对 \xpackage{hyperref}\ 的 \xoption{bookmarkstype}\ 选项进行了更好的设计(design)。
% \end{itemize}
%
% \section{示例(Example)}
%
%    \begin{macrocode}
%<*example>
%    \end{macrocode}
%    \begin{macrocode}
\documentclass{article}
\usepackage{xcolor}[2007/01/21]
\usepackage{hyperref}
\usepackage[
  open,
  openlevel=2,
  atend
]{bookmark}[2019/12/03]

\bookmarksetup{color=blue}

\BookmarkAtEnd{%
  \bookmarksetup{startatroot}%
  \bookmark[named=LastPage, level=0]{End/Last page}%
  \bookmark[named=FirstPage, level=1]{First page}%
}

\begin{document}
\section{First section}
\subsection{Subsection A}
\begin{figure}
  \hypertarget{fig}{}%
  A figure.
\end{figure}
\bookmark[
  rellevel=1,
  keeplevel,
  dest=fig
]{A figure}
\subsection{Subsection B}
\subsubsection{Subsubsection C}
\subsection{Umlauts: \"A\"O\"U\"a\"o\"u\ss}
\newpage
\bookmarksetup{
  bold,
  color=[rgb]{1,0,0}
}
\section{Very important section}
\bookmarksetup{
  italic,
  bold=false,
  color=blue
}
\subsection{Italic section}
\bookmarksetup{
  italic=false
}
\part{Misc}
\section{Diverse}
\subsubsection{Subsubsection, omitting subsection}
\bookmarksetup{
  startatroot
}
\section{Last section outside part}
\subsection{Subsection}
\bookmarksetup{
  color={}
}
\begingroup
  \bookmarksetup{level=0, color=green!80!black}
  \bookmark[named=FirstPage]{First page}
  \bookmark[named=LastPage]{Last page}
  \bookmark[named=PrevPage]{Previous page}
  \bookmark[named=NextPage]{Next page}
\endgroup
\bookmark[
  page=2,
  view=FitH 800
]{Page 2, FitH 800}
\bookmark[
  page=2,
  view=FitBH \calc{\paperheight-\topmargin-1in-\headheight-\headsep}
]{Page 2, FitBH top of text body}
\bookmark[
  uri={http://www.dante.de/},
  color=magenta
]{Dante homepage}
\bookmark[
  gotor={t.pdf},
  page=1,
  view={XYZ 0 1000 null},
  color=cyan!75!black
]{File t.pdf}
\bookmark[named=FirstPage]{First page}
\bookmark[rellevel=1, named=LastPage]{Last page (rellevel=1)}
\bookmark[named=PrevPage]{Previous page}
\bookmark[level=0, named=FirstPage]{First page (level=0)}
\bookmark[
  rellevel=1,
  keeplevel,
  named=LastPage
]{Last page (rellevel=1, keeplevel)}
\bookmark[named=PrevPage]{Previous page}
\end{document}
%    \end{macrocode}
%    \begin{macrocode}
%</example>
%    \end{macrocode}
%
% \StopEventually{
% }
%
% \section{实现(Implementation)}
%
% \subsection{宏包(Package)}
%
%    \begin{macrocode}
%<*package>
\NeedsTeXFormat{LaTeX2e}
\ProvidesPackage{bookmark}%
  [2020-11-06 v1.29 PDF bookmarks (HO)]%
%    \end{macrocode}
%
% \subsubsection{要求(Requirements)}
%
% \paragraph{\hologo{eTeX}.}
%
%    \begin{macro}{\BKM@CalcExpr}
%    \begin{macrocode}
\begingroup\expandafter\expandafter\expandafter\endgroup
\expandafter\ifx\csname numexpr\endcsname\relax
  \def\BKM@CalcExpr#1#2#3#4{%
    \begingroup
      \count@=#2\relax
      \advance\count@ by#3#4\relax
      \edef\x{\endgroup
        \def\noexpand#1{\the\count@}%
      }%
    \x
  }%
\else
  \def\BKM@CalcExpr#1#2#3#4{%
    \edef#1{%
      \the\numexpr#2#3#4\relax
    }%
  }%
\fi
%    \end{macrocode}
%    \end{macro}
%
% \paragraph{\hologo{pdfTeX}\ 的转义功能(escape features)}
%
%    \begin{macro}{\BKM@EscapeName}
%    \begin{macrocode}
\def\BKM@EscapeName#1{%
  \ifx#1\@empty
  \else
    \EdefEscapeName#1#1%
  \fi
}%
%    \end{macrocode}
%    \end{macro}
%    \begin{macro}{\BKM@EscapeString}
%    \begin{macrocode}
\def\BKM@EscapeString#1{%
  \ifx#1\@empty
  \else
    \EdefEscapeString#1#1%
  \fi
}%
%    \end{macrocode}
%    \end{macro}
%    \begin{macro}{\BKM@EscapeHex}
%    \begin{macrocode}
\def\BKM@EscapeHex#1{%
  \ifx#1\@empty
  \else
    \EdefEscapeHex#1#1%
  \fi
}%
%    \end{macrocode}
%    \end{macro}
%    \begin{macro}{\BKM@UnescapeHex}
%    \begin{macrocode}
\def\BKM@UnescapeHex#1{%
  \EdefUnescapeHex#1#1%
}%
%    \end{macrocode}
%    \end{macro}
%
% \paragraph{宏包(Packages)。}
%
% 不要加载由 \xpackage{hyperref}\ 加载的宏包
%    \begin{macrocode}
\RequirePackage{hyperref}[2010/06/18]
%    \end{macrocode}
%
% \subsubsection{宏包选项(Package options)}
%
%    \begin{macrocode}
\SetupKeyvalOptions{family=BKM,prefix=BKM@}
\DeclareLocalOptions{%
  atend,%
  bold,%
  color,%
  depth,%
  dest,%
  draft,%
  final,%
  gotor,%
  italic,%
  keeplevel,%
  level,%
  named,%
  numbered,%
  open,%
  openlevel,%
  page,%
  rawaction,%
  rellevel,%
  srcfile,%
  srcline,%
  startatroot,%
  uri,%
  view,%
}
%    \end{macrocode}
%    \begin{macro}{\bookmarksetup}
%    \begin{macrocode}
\newcommand*{\bookmarksetup}{\kvsetkeys{BKM}}
%    \end{macrocode}
%    \end{macro}
%    \begin{macro}{\BKM@setup}
%    \begin{macrocode}
\def\BKM@setup#1{%
  \bookmarksetup{#1}%
  \ifx\BKM@HookNext\ltx@empty
  \else
    \expandafter\bookmarksetup\expandafter{\BKM@HookNext}%
    \BKM@HookNextClear
  \fi
  \BKM@hook
  \ifBKM@keeplevel
  \else
    \xdef\BKM@currentlevel{\BKM@level}%
  \fi
}
%    \end{macrocode}
%    \end{macro}
%
%    \begin{macro}{\bookmarksetupnext}
%    \begin{macrocode}
\newcommand*{\bookmarksetupnext}[1]{%
  \ltx@GlobalAppendToMacro\BKM@HookNext{,#1}%
}
%    \end{macrocode}
%    \end{macro}
%    \begin{macro}{\BKM@setupnext}
%    \begin{macrocode}
%    \end{macrocode}
%    \end{macro}
%    \begin{macro}{\BKM@HookNextClear}
%    \begin{macrocode}
\def\BKM@HookNextClear{%
  \global\let\BKM@HookNext\ltx@empty
}
%    \end{macrocode}
%    \end{macro}
%    \begin{macro}{\BKM@HookNext}
%    \begin{macrocode}
\BKM@HookNextClear
%    \end{macrocode}
%    \end{macro}
%
%    \begin{macrocode}
\DeclareBoolOption{draft}
\DeclareComplementaryOption{final}{draft}
%    \end{macrocode}
%    \begin{macro}{\BKM@DisableOptions}
%    \begin{macrocode}
\def\BKM@DisableOptions{%
  \DisableKeyvalOption[action=warning,package=bookmark]%
      {BKM}{draft}%
  \DisableKeyvalOption[action=warning,package=bookmark]%
      {BKM}{final}%
}
%    \end{macrocode}
%    \end{macro}
%    \begin{macrocode}
\DeclareBoolOption[\ifHy@bookmarksopen true\else false\fi]{open}
%    \end{macrocode}
%    \begin{macro}{\bookmark@open}
%    \begin{macrocode}
\def\bookmark@open{%
  \ifBKM@open\ltx@one\else\ltx@zero\fi
}
%    \end{macrocode}
%    \end{macro}
%    \begin{macrocode}
\DeclareStringOption[\maxdimen]{openlevel}
%    \end{macrocode}
%    \begin{macro}{\BKM@openlevel}
%    \begin{macrocode}
\edef\BKM@openlevel{\number\@bookmarksopenlevel}
%    \end{macrocode}
%    \end{macro}
%    \begin{macrocode}
%\DeclareStringOption[\c@tocdepth]{depth}
\ltx@IfUndefined{Hy@bookmarksdepth}{%
  \def\BKM@depth{\c@tocdepth}%
}{%
  \let\BKM@depth\Hy@bookmarksdepth
}
\define@key{BKM}{depth}[]{%
  \edef\BKM@param{#1}%
  \ifx\BKM@param\@empty
    \def\BKM@depth{\c@tocdepth}%
  \else
    \ltx@IfUndefined{toclevel@\BKM@param}{%
      \@onelevel@sanitize\BKM@param
      \edef\BKM@temp{\expandafter\@car\BKM@param\@nil}%
      \ifcase 0\expandafter\ifx\BKM@temp-1\fi
              \expandafter\ifnum\expandafter`\BKM@temp>47 %
                \expandafter\ifnum\expandafter`\BKM@temp<58 %
                  1%
                \fi
              \fi
              \relax
        \PackageWarning{bookmark}{%
          Unknown document division name (\BKM@param)\MessageBreak
          for option `depth'%
        }%
      \else
        \BKM@SetDepthOrLevel\BKM@depth\BKM@param
      \fi
    }{%
      \BKM@SetDepthOrLevel\BKM@depth{%
        \csname toclevel@\BKM@param\endcsname
      }%
    }%
  \fi
}
%    \end{macrocode}
%    \begin{macro}{\bookmark@depth}
%    \begin{macrocode}
\def\bookmark@depth{\BKM@depth}
%    \end{macrocode}
%    \end{macro}
%    \begin{macro}{\BKM@SetDepthOrLevel}
%    \begin{macrocode}
\def\BKM@SetDepthOrLevel#1#2{%
  \begingroup
    \setbox\z@=\hbox{%
      \count@=#2\relax
      \expandafter
    }%
  \expandafter\endgroup
  \expandafter\def\expandafter#1\expandafter{\the\count@}%
}
%    \end{macrocode}
%    \end{macro}
%    \begin{macrocode}
\DeclareStringOption[\BKM@currentlevel]{level}[\BKM@currentlevel]
\define@key{BKM}{level}{%
  \edef\BKM@param{#1}%
  \ifx\BKM@param\BKM@MacroCurrentLevel
    \let\BKM@level\BKM@param
  \else
    \ltx@IfUndefined{toclevel@\BKM@param}{%
      \@onelevel@sanitize\BKM@param
      \edef\BKM@temp{\expandafter\@car\BKM@param\@nil}%
      \ifcase 0\expandafter\ifx\BKM@temp-1\fi
              \expandafter\ifnum\expandafter`\BKM@temp>47 %
                \expandafter\ifnum\expandafter`\BKM@temp<58 %
                  1%
                \fi
              \fi
              \relax
        \PackageWarning{bookmark}{%
          Unknown document division name (\BKM@param)\MessageBreak
          for option `level'%
        }%
      \else
        \BKM@SetDepthOrLevel\BKM@level\BKM@param
      \fi
    }{%
      \BKM@SetDepthOrLevel\BKM@level{%
        \csname toclevel@\BKM@param\endcsname
      }%
    }%
  \fi
}
%    \end{macrocode}
%    \begin{macro}{\BKM@MacroCurrentLevel}
%    \begin{macrocode}
\def\BKM@MacroCurrentLevel{\BKM@currentlevel}
%    \end{macrocode}
%    \end{macro}
%    \begin{macrocode}
\DeclareBoolOption{keeplevel}
\DeclareBoolOption{startatroot}
%    \end{macrocode}
%    \begin{macro}{\BKM@startatrootfalse}
%    \begin{macrocode}
\def\BKM@startatrootfalse{%
  \global\let\ifBKM@startatroot\iffalse
}
%    \end{macrocode}
%    \end{macro}
%    \begin{macro}{\BKM@startatroottrue}
%    \begin{macrocode}
\def\BKM@startatroottrue{%
  \global\let\ifBKM@startatroot\iftrue
}
%    \end{macrocode}
%    \end{macro}
%    \begin{macrocode}
\define@key{BKM}{rellevel}{%
  \BKM@CalcExpr\BKM@level{#1}+\BKM@currentlevel
}
%    \end{macrocode}
%    \begin{macro}{\bookmark@level}
%    \begin{macrocode}
\def\bookmark@level{\BKM@level}
%    \end{macrocode}
%    \end{macro}
%    \begin{macro}{\BKM@currentlevel}
%    \begin{macrocode}
\def\BKM@currentlevel{0}
%    \end{macrocode}
%    \end{macro}
%    Make \xpackage{bookmark}'s option \xoption{numbered} an alias
%    for \xpackage{hyperref}'s \xoption{bookmarksnumbered}.
%    \begin{macrocode}
\DeclareBoolOption[%
  \ifHy@bookmarksnumbered true\else false\fi
]{numbered}
\g@addto@macro\BKM@numberedtrue{%
  \let\ifHy@bookmarksnumbered\iftrue
}
\g@addto@macro\BKM@numberedfalse{%
  \let\ifHy@bookmarksnumbered\iffalse
}
\g@addto@macro\Hy@bookmarksnumberedtrue{%
  \let\ifBKM@numbered\iftrue
}
\g@addto@macro\Hy@bookmarksnumberedfalse{%
  \let\ifBKM@numbered\iffalse
}
%    \end{macrocode}
%    \begin{macro}{\bookmark@numbered}
%    \begin{macrocode}
\def\bookmark@numbered{%
  \ifBKM@numbered\ltx@one\else\ltx@zero\fi
}
%    \end{macrocode}
%    \end{macro}
%
% \paragraph{重定义 \xpackage{hyperref}\ 宏包的选项}
%
%    \begin{macro}{\BKM@PatchHyperrefOption}
%    \begin{macrocode}
\def\BKM@PatchHyperrefOption#1{%
  \expandafter\BKM@@PatchHyperrefOption\csname KV@Hyp@#1\endcsname%
}
%    \end{macrocode}
%    \end{macro}
%    \begin{macro}{\BKM@@PatchHyperrefOption}
%    \begin{macrocode}
\def\BKM@@PatchHyperrefOption#1{%
  \expandafter\BKM@@@PatchHyperrefOption#1{##1}\BKM@nil#1%
}
%    \end{macrocode}
%    \end{macro}
%    \begin{macro}{\BKM@@@PatchHyperrefOption}
%    \begin{macrocode}
\def\BKM@@@PatchHyperrefOption#1\BKM@nil#2#3{%
  \def#2##1{%
    #1%
    \bookmarksetup{#3={##1}}%
  }%
}
%    \end{macrocode}
%    \end{macro}
%    \begin{macrocode}
\BKM@PatchHyperrefOption{bookmarksopen}{open}
\BKM@PatchHyperrefOption{bookmarksopenlevel}{openlevel}
\BKM@PatchHyperrefOption{bookmarksdepth}{depth}
%    \end{macrocode}
%
% \paragraph{字体样式(font style)选项。}
%
%    注意:\xpackage{bitset}\ 宏是基于零的,PDF 规范(PDF specifications)以1开头。
%    \begin{macrocode}
\bitsetReset{BKM@FontStyle}%
\define@key{BKM}{italic}[true]{%
  \expandafter\ifx\csname if#1\endcsname\iftrue
    \bitsetSet{BKM@FontStyle}{0}%
  \else
    \bitsetClear{BKM@FontStyle}{0}%
  \fi
}%
\define@key{BKM}{bold}[true]{%
  \expandafter\ifx\csname if#1\endcsname\iftrue
    \bitsetSet{BKM@FontStyle}{1}%
  \else
    \bitsetClear{BKM@FontStyle}{1}%
  \fi
}%
%    \end{macrocode}
%    \begin{macro}{\bookmark@italic}
%    \begin{macrocode}
\def\bookmark@italic{%
  \ifnum\bitsetGet{BKM@FontStyle}{0}=1 \ltx@one\else\ltx@zero\fi
}
%    \end{macrocode}
%    \end{macro}
%    \begin{macro}{\bookmark@bold}
%    \begin{macrocode}
\def\bookmark@bold{%
  \ifnum\bitsetGet{BKM@FontStyle}{1}=1 \ltx@one\else\ltx@zero\fi
}
%    \end{macrocode}
%    \end{macro}
%    \begin{macro}{\BKM@PrintStyle}
%    \begin{macrocode}
\def\BKM@PrintStyle{%
  \bitsetGetDec{BKM@FontStyle}%
}%
%    \end{macrocode}
%    \end{macro}
%
% \paragraph{颜色(color)选项。}
%
%    \begin{macrocode}
\define@key{BKM}{color}{%
  \HyColor@BookmarkColor{#1}\BKM@color{bookmark}{color}%
}
%    \end{macrocode}
%    \begin{macro}{\BKM@color}
%    \begin{macrocode}
\let\BKM@color\@empty
%    \end{macrocode}
%    \end{macro}
%    \begin{macro}{\bookmark@color}
%    \begin{macrocode}
\def\bookmark@color{\BKM@color}
%    \end{macrocode}
%    \end{macro}
%
% \subsubsection{动作(action)选项}
%
%    \begin{macrocode}
\def\BKM@temp#1{%
  \DeclareStringOption{#1}%
  \expandafter\edef\csname bookmark@#1\endcsname{%
    \expandafter\noexpand\csname BKM@#1\endcsname
  }%
}
%    \end{macrocode}
%    \begin{macro}{\bookmark@dest}
%    \begin{macrocode}
\BKM@temp{dest}
%    \end{macrocode}
%    \end{macro}
%    \begin{macro}{\bookmark@named}
%    \begin{macrocode}
\BKM@temp{named}
%    \end{macrocode}
%    \end{macro}
%    \begin{macro}{\bookmark@uri}
%    \begin{macrocode}
\BKM@temp{uri}
%    \end{macrocode}
%    \end{macro}
%    \begin{macro}{\bookmark@gotor}
%    \begin{macrocode}
\BKM@temp{gotor}
%    \end{macrocode}
%    \end{macro}
%    \begin{macro}{\bookmark@rawaction}
%    \begin{macrocode}
\BKM@temp{rawaction}
%    \end{macrocode}
%    \end{macro}
%
%    \begin{macrocode}
\define@key{BKM}{page}{%
  \def\BKM@page{#1}%
  \ifx\BKM@page\@empty
  \else
    \edef\BKM@page{\number\BKM@page}%
    \ifnum\BKM@page>\z@
    \else
      \PackageError{bookmark}{Page must be positive}\@ehc
      \def\BKM@page{1}%
    \fi
  \fi
}
%    \end{macrocode}
%    \begin{macro}{\BKM@page}
%    \begin{macrocode}
\let\BKM@page\@empty
%    \end{macrocode}
%    \end{macro}
%    \begin{macro}{\bookmark@page}
%    \begin{macrocode}
\def\bookmark@page{\BKM@@page}
%    \end{macrocode}
%    \end{macro}
%
%    \begin{macrocode}
\define@key{BKM}{view}{%
  \BKM@CheckView{#1}%
}
%    \end{macrocode}
%    \begin{macro}{\BKM@view}
%    \begin{macrocode}
\let\BKM@view\@empty
%    \end{macrocode}
%    \end{macro}
%    \begin{macro}{\bookmark@view}
%    \begin{macrocode}
\def\bookmark@view{\BKM@view}
%    \end{macrocode}
%    \end{macro}
%    \begin{macro}{BKM@CheckView}
%    \begin{macrocode}
\def\BKM@CheckView#1{%
  \BKM@CheckViewType#1 \@nil
}
%    \end{macrocode}
%    \end{macro}
%    \begin{macro}{\BKM@CheckViewType}
%    \begin{macrocode}
\def\BKM@CheckViewType#1 #2\@nil{%
  \def\BKM@type{#1}%
  \@onelevel@sanitize\BKM@type
  \BKM@TestViewType{Fit}{}%
  \BKM@TestViewType{FitB}{}%
  \BKM@TestViewType{FitH}{%
    \BKM@CheckParam#2 \@nil{top}%
  }%
  \BKM@TestViewType{FitBH}{%
    \BKM@CheckParam#2 \@nil{top}%
  }%
  \BKM@TestViewType{FitV}{%
    \BKM@CheckParam#2 \@nil{bottom}%
  }%
  \BKM@TestViewType{FitBV}{%
    \BKM@CheckParam#2 \@nil{bottom}%
  }%
  \BKM@TestViewType{FitR}{%
    \BKM@CheckRect{#2}{ }%
  }%
  \BKM@TestViewType{XYZ}{%
    \BKM@CheckXYZ{#2}{ }%
  }%
  \@car{%
    \PackageError{bookmark}{%
      Unknown view type `\BKM@type',\MessageBreak
      using `FitH' instead%
    }\@ehc
    \def\BKM@view{FitH}%
  }%
  \@nil
}
%    \end{macrocode}
%    \end{macro}
%    \begin{macro}{\BKM@TestViewType}
%    \begin{macrocode}
\def\BKM@TestViewType#1{%
  \def\BKM@temp{#1}%
  \@onelevel@sanitize\BKM@temp
  \ifx\BKM@type\BKM@temp
    \let\BKM@view\BKM@temp
    \expandafter\@car
  \else
    \expandafter\@gobble
  \fi
}
%    \end{macrocode}
%    \end{macro}
%    \begin{macro}{BKM@CheckParam}
%    \begin{macrocode}
\def\BKM@CheckParam#1 #2\@nil#3{%
  \def\BKM@param{#1}%
  \ifx\BKM@param\@empty
    \PackageWarning{bookmark}{%
      Missing parameter (#3) for `\BKM@type',\MessageBreak
      using 0%
    }%
    \def\BKM@param{0}%
  \else
    \BKM@CalcParam
  \fi
  \edef\BKM@view{\BKM@view\space\BKM@param}%
}
%    \end{macrocode}
%    \end{macro}
%    \begin{macro}{BKM@CheckRect}
%    \begin{macrocode}
\def\BKM@CheckRect#1#2{%
  \BKM@@CheckRect#1#2#2#2#2\@nil
}
%    \end{macrocode}
%    \end{macro}
%    \begin{macro}{\BKM@@CheckRect}
%    \begin{macrocode}
\def\BKM@@CheckRect#1 #2 #3 #4 #5\@nil{%
  \def\BKM@temp{0}%
  \def\BKM@param{#1}%
  \ifx\BKM@param\@empty
    \def\BKM@param{0}%
    \def\BKM@temp{1}%
  \else
    \BKM@CalcParam
  \fi
  \edef\BKM@view{\BKM@view\space\BKM@param}%
  \def\BKM@param{#2}%
  \ifx\BKM@param\@empty
    \def\BKM@param{0}%
    \def\BKM@temp{1}%
  \else
    \BKM@CalcParam
  \fi
  \edef\BKM@view{\BKM@view\space\BKM@param}%
  \def\BKM@param{#3}%
  \ifx\BKM@param\@empty
    \def\BKM@param{0}%
    \def\BKM@temp{1}%
  \else
    \BKM@CalcParam
  \fi
  \edef\BKM@view{\BKM@view\space\BKM@param}%
  \def\BKM@param{#4}%
  \ifx\BKM@param\@empty
    \def\BKM@param{0}%
    \def\BKM@temp{1}%
  \else
    \BKM@CalcParam
  \fi
  \edef\BKM@view{\BKM@view\space\BKM@param}%
  \ifnum\BKM@temp>\z@
    \PackageWarning{bookmark}{Missing parameters for `\BKM@type'}%
  \fi
}
%    \end{macrocode}
%    \end{macro}
%    \begin{macro}{\BKM@CheckXYZ}
%    \begin{macrocode}
\def\BKM@CheckXYZ#1#2{%
  \BKM@@CheckXYZ#1#2#2#2\@nil
}
%    \end{macrocode}
%    \end{macro}
%    \begin{macro}{\BKM@@CheckXYZ}
%    \begin{macrocode}
\def\BKM@@CheckXYZ#1 #2 #3 #4\@nil{%
  \def\BKM@param{#1}%
  \let\BKM@temp\BKM@param
  \@onelevel@sanitize\BKM@temp
  \ifx\BKM@param\@empty
    \let\BKM@param\BKM@null
  \else
    \ifx\BKM@temp\BKM@null
    \else
      \BKM@CalcParam
    \fi
  \fi
  \edef\BKM@view{\BKM@view\space\BKM@param}%
  \def\BKM@param{#2}%
  \let\BKM@temp\BKM@param
  \@onelevel@sanitize\BKM@temp
  \ifx\BKM@param\@empty
    \let\BKM@param\BKM@null
  \else
    \ifx\BKM@temp\BKM@null
    \else
      \BKM@CalcParam
    \fi
  \fi
  \edef\BKM@view{\BKM@view\space\BKM@param}%
  \def\BKM@param{#3}%
  \ifx\BKM@param\@empty
    \let\BKM@param\BKM@null
  \fi
  \edef\BKM@view{\BKM@view\space\BKM@param}%
}
%    \end{macrocode}
%    \end{macro}
%    \begin{macro}{\BKM@null}
%    \begin{macrocode}
\def\BKM@null{null}
\@onelevel@sanitize\BKM@null
%    \end{macrocode}
%    \end{macro}
%
%    \begin{macro}{\BKM@CalcParam}
%    \begin{macrocode}
\def\BKM@CalcParam{%
  \begingroup
  \let\calc\@firstofone
  \expandafter\BKM@@CalcParam\BKM@param\@empty\@empty\@nil
}
%    \end{macrocode}
%    \end{macro}
%    \begin{macro}{\BKM@@CalcParam}
%    \begin{macrocode}
\def\BKM@@CalcParam#1#2#3\@nil{%
  \ifx\calc#1%
    \@ifundefined{calc@assign@dimen}{%
      \@ifundefined{dimexpr}{%
        \setlength{\dimen@}{#2}%
      }{%
        \setlength{\dimen@}{\dimexpr#2\relax}%
      }%
    }{%
      \setlength{\dimen@}{#2}%
    }%
    \dimen@.99626\dimen@
    \edef\BKM@param{\strip@pt\dimen@}%
    \expandafter\endgroup
    \expandafter\def\expandafter\BKM@param\expandafter{\BKM@param}%
  \else
    \endgroup
  \fi
}
%    \end{macrocode}
%    \end{macro}
%
% \subsubsection{\xoption{atend}\ 选项}
%
%    \begin{macrocode}
\DeclareBoolOption{atend}
\g@addto@macro\BKM@DisableOptions{%
  \DisableKeyvalOption[action=warning,package=bookmark]%
      {BKM}{atend}%
}
%    \end{macrocode}
%
% \subsubsection{\xoption{style}\ 选项}
%
%    \begin{macro}{\bookmarkdefinestyle}
%    \begin{macrocode}
\newcommand*{\bookmarkdefinestyle}[2]{%
  \@ifundefined{BKM@style@#1}{%
  }{%
    \PackageInfo{bookmark}{Redefining style `#1'}%
  }%
  \@namedef{BKM@style@#1}{#2}%
}
%    \end{macrocode}
%    \end{macro}
%    \begin{macrocode}
\define@key{BKM}{style}{%
  \BKM@StyleCall{#1}%
}
\newif\ifBKM@ok
%    \end{macrocode}
%    \begin{macro}{\BKM@StyleCall}
%    \begin{macrocode}
\def\BKM@StyleCall#1{%
  \@ifundefined{BKM@style@#1}{%
    \PackageWarning{bookmark}{%
      Ignoring unknown style `#1'%
    }%
  }{%
%    \end{macrocode}
%    检查样式堆栈(style stack)。
%    \begin{macrocode}
    \BKM@oktrue
    \edef\BKM@StyleCurrent{#1}%
    \@onelevel@sanitize\BKM@StyleCurrent
    \let\BKM@StyleEntry\BKM@StyleEntryCheck
    \BKM@StyleStack
    \ifBKM@ok
      \expandafter\@firstofone
    \else
      \PackageError{bookmark}{%
        Ignoring recursive call of style `\BKM@StyleCurrent'%
      }\@ehc
      \expandafter\@gobble
    \fi
    {%
%    \end{macrocode}
%    在堆栈上推送当前样式(Push current style on stack)。
%    \begin{macrocode}
      \let\BKM@StyleEntry\relax
      \edef\BKM@StyleStack{%
        \BKM@StyleEntry{\BKM@StyleCurrent}%
        \BKM@StyleStack
      }%
%    \end{macrocode}
%   调用样式(Call style)。
%    \begin{macrocode}
      \expandafter\expandafter\expandafter\bookmarksetup
      \expandafter\expandafter\expandafter{%
        \csname BKM@style@\BKM@StyleCurrent\endcsname
      }%
%    \end{macrocode}
%    从堆栈中弹出当前样式(Pop current style from stack)。
%    \begin{macrocode}
      \BKM@StyleStackPop
    }%
  }%
}
%    \end{macrocode}
%    \end{macro}
%    \begin{macro}{\BKM@StyleStackPop}
%    \begin{macrocode}
\def\BKM@StyleStackPop{%
  \let\BKM@StyleEntry\relax
  \edef\BKM@StyleStack{%
    \expandafter\@gobbletwo\BKM@StyleStack
  }%
}
%    \end{macrocode}
%    \end{macro}
%    \begin{macro}{\BKM@StyleEntryCheck}
%    \begin{macrocode}
\def\BKM@StyleEntryCheck#1{%
  \def\BKM@temp{#1}%
  \ifx\BKM@temp\BKM@StyleCurrent
    \BKM@okfalse
  \fi
}
%    \end{macrocode}
%    \end{macro}
%    \begin{macro}{\BKM@StyleStack}
%    \begin{macrocode}
\def\BKM@StyleStack{}
%    \end{macrocode}
%    \end{macro}
%
% \subsubsection{源文件位置(source file location)选项}
%
%    \begin{macrocode}
\DeclareStringOption{srcline}
\DeclareStringOption{srcfile}
%    \end{macrocode}
%
% \subsubsection{钩子支持(Hook support)}
%
%    \begin{macro}{\BKM@hook}
%    \begin{macrocode}
\def\BKM@hook{}
%    \end{macrocode}
%    \end{macro}
%    \begin{macrocode}
\define@key{BKM}{addtohook}{%
  \ltx@LocalAppendToMacro\BKM@hook{#1}%
}
%    \end{macrocode}
%
%    \begin{macro}{bookmarkget}
%    \begin{macrocode}
\newcommand*{\bookmarkget}[1]{%
  \romannumeral0%
  \ltx@ifundefined{bookmark@#1}{%
    \ltx@space
  }{%
    \expandafter\expandafter\expandafter\ltx@space
    \csname bookmark@#1\endcsname
  }%
}
%    \end{macrocode}
%    \end{macro}
%
% \subsubsection{设置和加载驱动程序}
%
% \paragraph{检测驱动程序。}
%
%    \begin{macro}{\BKM@DefineDriverKey}
%    \begin{macrocode}
\def\BKM@DefineDriverKey#1{%
  \define@key{BKM}{#1}[]{%
    \def\BKM@driver{#1}%
  }%
  \g@addto@macro\BKM@DisableOptions{%
    \DisableKeyvalOption[action=warning,package=bookmark]%
        {BKM}{#1}%
  }%
}
%    \end{macrocode}
%    \end{macro}
%    \begin{macrocode}
\BKM@DefineDriverKey{pdftex}
\BKM@DefineDriverKey{dvips}
\BKM@DefineDriverKey{dvipdfm}
\BKM@DefineDriverKey{dvipdfmx}
\BKM@DefineDriverKey{xetex}
\BKM@DefineDriverKey{vtex}
\define@key{BKM}{dvipdfmx-outline-open}[true]{%
 \PackageWarning{bookmark}{Option 'dvipdfmx-outline-open' is obsolete
   and ignored}{}}
%    \end{macrocode}
%    \begin{macro}{\bookmark@driver}
%    \begin{macrocode}
\def\bookmark@driver{\BKM@driver}
%    \end{macrocode}
%    \end{macro}
%    \begin{macrocode}
\InputIfFileExists{bookmark.cfg}{}{}
%    \end{macrocode}
%    \begin{macro}{\BookmarkDriverDefault}
%    \begin{macrocode}
\providecommand*{\BookmarkDriverDefault}{dvips}
%    \end{macrocode}
%    \end{macro}
%    \begin{macro}{\BKM@driver}
% Lua\TeX\ 和 pdf\TeX\ 共享驱动程序。
%    \begin{macrocode}
\ifpdf
  \def\BKM@driver{pdftex}%
  \ifx\pdfoutline\@undefined
    \ifx\pdfextension\@undefined\else
      \protected\def\pdfoutline{\pdfextension outline }
    \fi
  \fi
\else
  \ifxetex
    \def\BKM@driver{dvipdfm}%
  \else
    \ifvtex
      \def\BKM@driver{vtex}%
    \else
      \edef\BKM@driver{\BookmarkDriverDefault}%
    \fi
  \fi
\fi
%    \end{macrocode}
%    \end{macro}
%
% \paragraph{过程选项(Process options)。}
%
%    \begin{macrocode}
\ProcessKeyvalOptions*
\BKM@DisableOptions
%    \end{macrocode}
%
% \paragraph{\xoption{draft}\ 选项}
%
%    \begin{macrocode}
\ifBKM@draft
  \PackageWarningNoLine{bookmark}{Draft mode on}%
  \let\bookmarksetup\ltx@gobble
  \let\BookmarkAtEnd\ltx@gobble
  \let\bookmarkdefinestyle\ltx@gobbletwo
  \let\bookmarkget\ltx@gobble
  \let\pdfbookmark\ltx@undefined
  \newcommand*{\pdfbookmark}[3][]{}%
  \let\currentpdfbookmark\ltx@gobbletwo
  \let\subpdfbookmark\ltx@gobbletwo
  \let\belowpdfbookmark\ltx@gobbletwo
  \newcommand*{\bookmark}[2][]{}%
  \renewcommand*{\Hy@writebookmark}[5]{}%
  \let\ReadBookmarks\relax
  \let\BKM@DefGotoNameAction\ltx@gobbletwo % package `hypdestopt'
  \expandafter\endinput
\fi
%    \end{macrocode}
%
% \paragraph{验证和加载驱动程序。}
%
%    \begin{macrocode}
\def\BKM@temp{dvipdfmx}%
\ifx\BKM@temp\BKM@driver
  \def\BKM@driver{dvipdfm}%
\fi
\def\BKM@temp{pdftex}%
\ifpdf
  \ifx\BKM@temp\BKM@driver
  \else
    \PackageWarningNoLine{bookmark}{%
      Wrong driver `\BKM@driver', using `pdftex' instead%
    }%
    \let\BKM@driver\BKM@temp
  \fi
\else
  \ifx\BKM@temp\BKM@driver
    \PackageError{bookmark}{%
      Wrong driver, pdfTeX is not running in PDF mode.\MessageBreak
      Package loading is aborted%
    }\@ehc
    \expandafter\expandafter\expandafter\endinput
  \fi
  \def\BKM@temp{dvipdfm}%
  \ifxetex
    \ifx\BKM@temp\BKM@driver
    \else
      \PackageWarningNoLine{bookmark}{%
        Wrong driver `\BKM@driver',\MessageBreak
        using `dvipdfm' for XeTeX instead%
      }%
      \let\BKM@driver\BKM@temp
    \fi
  \else
    \def\BKM@temp{vtex}%
    \ifvtex
      \ifx\BKM@temp\BKM@driver
      \else
        \PackageWarningNoLine{bookmark}{%
          Wrong driver `\BKM@driver',\MessageBreak
          using `vtex' for VTeX instead%
        }%
        \let\BKM@driver\BKM@temp
      \fi
    \else
      \ifx\BKM@temp\BKM@driver
        \PackageError{bookmark}{%
          Wrong driver, VTeX is not running in PDF mode.\MessageBreak
          Package loading is aborted%
        }\@ehc
        \expandafter\expandafter\expandafter\endinput
      \fi
    \fi
  \fi
\fi
\providecommand\IfFormatAtLeastTF{\@ifl@t@r\fmtversion}
\IfFormatAtLeastTF{2020/10/01}{}{\edef\BKM@driver{\BKM@driver-2019-12-03}}
\InputIfFileExists{bkm-\BKM@driver.def}{}{%
  \PackageError{bookmark}{%
    Unsupported driver `\BKM@driver'.\MessageBreak
    Package loading is aborted%
  }\@ehc
  \endinput
}
%    \end{macrocode}
%
% \subsubsection{与 \xpackage{hyperref}\ 的兼容性}
%
%    \begin{macro}{\pdfbookmark}
%    \begin{macrocode}
\let\pdfbookmark\ltx@undefined
\newcommand*{\pdfbookmark}[3][0]{%
  \bookmark[level=#1,dest={#3.#1}]{#2}%
  \hyper@anchorstart{#3.#1}\hyper@anchorend
}
%    \end{macrocode}
%    \end{macro}
%    \begin{macro}{\currentpdfbookmark}
%    \begin{macrocode}
\def\currentpdfbookmark{%
  \pdfbookmark[\BKM@currentlevel]%
}
%    \end{macrocode}
%    \end{macro}
%    \begin{macro}{\subpdfbookmark}
%    \begin{macrocode}
\def\subpdfbookmark{%
  \BKM@CalcExpr\BKM@CalcResult\BKM@currentlevel+1%
  \expandafter\pdfbookmark\expandafter[\BKM@CalcResult]%
}
%    \end{macrocode}
%    \end{macro}
%    \begin{macro}{\belowpdfbookmark}
%    \begin{macrocode}
\def\belowpdfbookmark#1#2{%
  \xdef\BKM@gtemp{\number\BKM@currentlevel}%
  \subpdfbookmark{#1}{#2}%
  \global\let\BKM@currentlevel\BKM@gtemp
}
%    \end{macrocode}
%    \end{macro}
%
%    节号(section number)、文本(text)、标签(label)、级别(level)、文件(file)
%    \begin{macro}{\Hy@writebookmark}
%    \begin{macrocode}
\def\Hy@writebookmark#1#2#3#4#5{%
  \ifnum#4>\BKM@depth\relax
  \else
    \def\BKM@type{#5}%
    \ifx\BKM@type\Hy@bookmarkstype
      \begingroup
        \ifBKM@numbered
          \let\numberline\Hy@numberline
          \let\booknumberline\Hy@numberline
          \let\partnumberline\Hy@numberline
          \let\chapternumberline\Hy@numberline
        \else
          \let\numberline\@gobble
          \let\booknumberline\@gobble
          \let\partnumberline\@gobble
          \let\chapternumberline\@gobble
        \fi
        \bookmark[level=#4,dest={\HyperDestNameFilter{#3}}]{#2}%
      \endgroup
    \fi
  \fi
}
%    \end{macrocode}
%    \end{macro}
%
%    \begin{macro}{\ReadBookmarks}
%    \begin{macrocode}
\let\ReadBookmarks\relax
%    \end{macrocode}
%    \end{macro}
%
%    \begin{macrocode}
%</package>
%    \end{macrocode}
%
% \subsection{dvipdfm 的驱动程序}
%
%    \begin{macrocode}
%<*dvipdfm>
\NeedsTeXFormat{LaTeX2e}
\ProvidesFile{bkm-dvipdfm.def}%
  [2020-11-06 v1.29 bookmark driver for dvipdfm (HO)]%
%    \end{macrocode}
%
%    \begin{macro}{\BKM@id}
%    \begin{macrocode}
\newcount\BKM@id
\BKM@id=\z@
%    \end{macrocode}
%    \end{macro}
%
%    \begin{macro}{\BKM@0}
%    \begin{macrocode}
\@namedef{BKM@0}{000}
%    \end{macrocode}
%    \end{macro}
%    \begin{macro}{\ifBKM@sw}
%    \begin{macrocode}
\newif\ifBKM@sw
%    \end{macrocode}
%    \end{macro}
%
%    \begin{macro}{\bookmark}
%    \begin{macrocode}
\newcommand*{\bookmark}[2][]{%
  \if@filesw
    \begingroup
      \def\bookmark@text{#2}%
      \BKM@setup{#1}%
      \edef\BKM@prev{\the\BKM@id}%
      \global\advance\BKM@id\@ne
      \BKM@swtrue
      \@whilesw\ifBKM@sw\fi{%
        \def\BKM@abslevel{1}%
        \ifnum\ifBKM@startatroot\z@\else\BKM@prev\fi=\z@
          \BKM@startatrootfalse
          \expandafter\xdef\csname BKM@\the\BKM@id\endcsname{%
            0{\BKM@level}\BKM@abslevel
          }%
          \BKM@swfalse
        \else
          \expandafter\expandafter\expandafter\BKM@getx
              \csname BKM@\BKM@prev\endcsname
          \ifnum\BKM@level>\BKM@x@level\relax
            \BKM@CalcExpr\BKM@abslevel\BKM@x@abslevel+1%
            \expandafter\xdef\csname BKM@\the\BKM@id\endcsname{%
              {\BKM@prev}{\BKM@level}\BKM@abslevel
            }%
            \BKM@swfalse
          \else
            \let\BKM@prev\BKM@x@parent
          \fi
        \fi
      }%
      \csname HyPsd@XeTeXBigCharstrue\endcsname
      \pdfstringdef\BKM@title{\bookmark@text}%
      \edef\BKM@FLAGS{\BKM@PrintStyle}%
      \let\BKM@action\@empty
      \ifx\BKM@gotor\@empty
        \ifx\BKM@dest\@empty
          \ifx\BKM@named\@empty
            \ifx\BKM@rawaction\@empty
              \ifx\BKM@uri\@empty
                \ifx\BKM@page\@empty
                  \PackageError{bookmark}{Missing action}\@ehc
                  \edef\BKM@action{/Dest[@page1/Fit]}%
                \else
                  \ifx\BKM@view\@empty
                    \def\BKM@view{Fit}%
                  \fi
                  \edef\BKM@action{/Dest[@page\BKM@page/\BKM@view]}%
                \fi
              \else
                \BKM@EscapeString\BKM@uri
                \edef\BKM@action{%
                  /A<<%
                    /S/URI%
                    /URI(\BKM@uri)%
                  >>%
                }%
              \fi
            \else
              \edef\BKM@action{/A<<\BKM@rawaction>>}%
            \fi
          \else
            \BKM@EscapeName\BKM@named
            \edef\BKM@action{%
              /A<</S/Named/N/\BKM@named>>%
            }%
          \fi
        \else
          \BKM@EscapeString\BKM@dest
          \edef\BKM@action{%
            /A<<%
              /S/GoTo%
              /D(\BKM@dest)%
            >>%
          }%
        \fi
      \else
        \ifx\BKM@dest\@empty
          \ifx\BKM@page\@empty
            \def\BKM@page{0}%
          \else
            \BKM@CalcExpr\BKM@page\BKM@page-1%
          \fi
          \ifx\BKM@view\@empty
            \def\BKM@view{Fit}%
          \fi
          \edef\BKM@action{/D[\BKM@page/\BKM@view]}%
        \else
          \BKM@EscapeString\BKM@dest
          \edef\BKM@action{/D(\BKM@dest)}%
        \fi
        \BKM@EscapeString\BKM@gotor
        \edef\BKM@action{%
          /A<<%
            /S/GoToR%
            /F(\BKM@gotor)%
            \BKM@action
          >>%
        }%
      \fi
      \special{pdf:%
        out
              [%
              \ifBKM@open
                \ifnum\BKM@level<%
                    \expandafter\ltx@firstofone\expandafter
                    {\number\BKM@openlevel} %
                \else
                  -%
                \fi
              \else
                -%
              \fi
              ] %
            \BKM@abslevel
        <<%
          /Title(\BKM@title)%
          \ifx\BKM@color\@empty
          \else
            /C[\BKM@color]%
          \fi
          \ifnum\BKM@FLAGS>\z@
            /F \BKM@FLAGS
          \fi
          \BKM@action
        >>%
      }%
    \endgroup
  \fi
}
%    \end{macrocode}
%    \end{macro}
%    \begin{macro}{\BKM@getx}
%    \begin{macrocode}
\def\BKM@getx#1#2#3{%
  \def\BKM@x@parent{#1}%
  \def\BKM@x@level{#2}%
  \def\BKM@x@abslevel{#3}%
}
%    \end{macrocode}
%    \end{macro}
%
%    \begin{macrocode}
%</dvipdfm>
%    \end{macrocode}
%
% \subsection{\hologo{VTeX}\ 的驱动程序}
%
%    \begin{macrocode}
%<*vtex>
\NeedsTeXFormat{LaTeX2e}
\ProvidesFile{bkm-vtex.def}%
  [2020-11-06 v1.29 bookmark driver for VTeX (HO)]%
%    \end{macrocode}
%
%    \begin{macrocode}
\ifvtexpdf
\else
  \PackageWarningNoLine{bookmark}{%
    The VTeX driver only supports PDF mode%
  }%
\fi
%    \end{macrocode}
%
%    \begin{macro}{\BKM@id}
%    \begin{macrocode}
\newcount\BKM@id
\BKM@id=\z@
%    \end{macrocode}
%    \end{macro}
%
%    \begin{macro}{\BKM@0}
%    \begin{macrocode}
\@namedef{BKM@0}{00}
%    \end{macrocode}
%    \end{macro}
%    \begin{macro}{\ifBKM@sw}
%    \begin{macrocode}
\newif\ifBKM@sw
%    \end{macrocode}
%    \end{macro}
%
%    \begin{macro}{\bookmark}
%    \begin{macrocode}
\newcommand*{\bookmark}[2][]{%
  \if@filesw
    \begingroup
      \def\bookmark@text{#2}%
      \BKM@setup{#1}%
      \edef\BKM@prev{\the\BKM@id}%
      \global\advance\BKM@id\@ne
      \BKM@swtrue
      \@whilesw\ifBKM@sw\fi{%
        \ifnum\ifBKM@startatroot\z@\else\BKM@prev\fi=\z@
          \BKM@startatrootfalse
          \def\BKM@parent{0}%
          \expandafter\xdef\csname BKM@\the\BKM@id\endcsname{%
            0{\BKM@level}%
          }%
          \BKM@swfalse
        \else
          \expandafter\expandafter\expandafter\BKM@getx
              \csname BKM@\BKM@prev\endcsname
          \ifnum\BKM@level>\BKM@x@level\relax
            \let\BKM@parent\BKM@prev
            \expandafter\xdef\csname BKM@\the\BKM@id\endcsname{%
              {\BKM@prev}{\BKM@level}%
            }%
            \BKM@swfalse
          \else
            \let\BKM@prev\BKM@x@parent
          \fi
        \fi
      }%
      \pdfstringdef\BKM@title{\bookmark@text}%
      \BKM@vtex@title
      \edef\BKM@FLAGS{\BKM@PrintStyle}%
      \let\BKM@action\@empty
      \ifx\BKM@gotor\@empty
        \ifx\BKM@dest\@empty
          \ifx\BKM@named\@empty
            \ifx\BKM@rawaction\@empty
              \ifx\BKM@uri\@empty
                \ifx\BKM@page\@empty
                  \PackageError{bookmark}{Missing action}\@ehc
                  \def\BKM@action{!1}%
                \else
                  \edef\BKM@action{!\BKM@page}%
                \fi
              \else
                \BKM@EscapeString\BKM@uri
                \edef\BKM@action{%
                  <u=%
                    /S/URI%
                    /URI(\BKM@uri)%
                  >%
                }%
              \fi
            \else
              \edef\BKM@action{<u=\BKM@rawaction>}%
            \fi
          \else
            \BKM@EscapeName\BKM@named
            \edef\BKM@action{%
              <u=%
                /S/Named%
                /N/\BKM@named
              >%
            }%
          \fi
        \else
          \BKM@EscapeString\BKM@dest
          \edef\BKM@action{\BKM@dest}%
        \fi
      \else
        \ifx\BKM@dest\@empty
          \ifx\BKM@page\@empty
            \def\BKM@page{1}%
          \fi
          \ifx\BKM@view\@empty
            \def\BKM@view{Fit}%
          \fi
          \edef\BKM@action{/D[\BKM@page/\BKM@view]}%
        \else
          \BKM@EscapeString\BKM@dest
          \edef\BKM@action{/D(\BKM@dest)}%
        \fi
        \BKM@EscapeString\BKM@gotor
        \edef\BKM@action{%
          <u=%
            /S/GoToR%
            /F(\BKM@gotor)%
            \BKM@action
          >>%
        }%
      \fi
      \ifx\BKM@color\@empty
        \let\BKM@RGBcolor\@empty
      \else
        \expandafter\BKM@toRGB\BKM@color\@nil
      \fi
      \special{%
        !outline \BKM@action;%
        p=\BKM@parent,%
        i=\number\BKM@id,%
        s=%
          \ifBKM@open
            \ifnum\BKM@level<\BKM@openlevel
              o%
            \else
              c%
            \fi
          \else
            c%
          \fi,%
        \ifx\BKM@RGBcolor\@empty
        \else
          c=\BKM@RGBcolor,%
        \fi
        \ifnum\BKM@FLAGS>\z@
          f=\BKM@FLAGS,%
        \fi
        t=\BKM@title
      }%
    \endgroup
  \fi
}
%    \end{macrocode}
%    \end{macro}
%    \begin{macro}{\BKM@getx}
%    \begin{macrocode}
\def\BKM@getx#1#2{%
  \def\BKM@x@parent{#1}%
  \def\BKM@x@level{#2}%
}
%    \end{macrocode}
%    \end{macro}
%    \begin{macro}{\BKM@toRGB}
%    \begin{macrocode}
\def\BKM@toRGB#1 #2 #3\@nil{%
  \let\BKM@RGBcolor\@empty
  \BKM@toRGBComponent{#1}%
  \BKM@toRGBComponent{#2}%
  \BKM@toRGBComponent{#3}%
}
%    \end{macrocode}
%    \end{macro}
%    \begin{macro}{\BKM@toRGBComponent}
%    \begin{macrocode}
\def\BKM@toRGBComponent#1{%
  \dimen@=#1pt\relax
  \ifdim\dimen@>\z@
    \ifdim\dimen@<\p@
      \dimen@=255\dimen@
      \advance\dimen@ by 32768sp\relax
      \divide\dimen@ by 65536\relax
      \dimen@ii=\dimen@
      \divide\dimen@ii by 16\relax
      \edef\BKM@RGBcolor{%
        \BKM@RGBcolor
        \BKM@toHexDigit\dimen@ii
      }%
      \dimen@ii=16\dimen@ii
      \advance\dimen@-\dimen@ii
      \edef\BKM@RGBcolor{%
        \BKM@RGBcolor
        \BKM@toHexDigit\dimen@
      }%
    \else
      \edef\BKM@RGBcolor{\BKM@RGBcolor FF}%
    \fi
  \else
    \edef\BKM@RGBcolor{\BKM@RGBcolor00}%
  \fi
}
%    \end{macrocode}
%    \end{macro}
%    \begin{macro}{\BKM@toHexDigit}
%    \begin{macrocode}
\def\BKM@toHexDigit#1{%
  \ifcase\expandafter\@firstofone\expandafter{\number#1} %
    0\or 1\or 2\or 3\or 4\or 5\or 6\or 7\or
    8\or 9\or A\or B\or C\or D\or E\or F%
  \fi
}
%    \end{macrocode}
%    \end{macro}
%    \begin{macrocode}
\begingroup
  \catcode`\|=0 %
  \catcode`\\=12 %
%    \end{macrocode}
%    \begin{macro}{\BKM@vtex@title}
%    \begin{macrocode}
  |gdef|BKM@vtex@title{%
    |@onelevel@sanitize|BKM@title
    |edef|BKM@title{|expandafter|BKM@vtex@leftparen|BKM@title\(|@nil}%
    |edef|BKM@title{|expandafter|BKM@vtex@rightparen|BKM@title\)|@nil}%
    |edef|BKM@title{|expandafter|BKM@vtex@zero|BKM@title\0|@nil}%
    |edef|BKM@title{|expandafter|BKM@vtex@one|BKM@title\1|@nil}%
    |edef|BKM@title{|expandafter|BKM@vtex@two|BKM@title\2|@nil}%
    |edef|BKM@title{|expandafter|BKM@vtex@three|BKM@title\3|@nil}%
  }%
%    \end{macrocode}
%    \end{macro}
%    \begin{macro}{\BKM@vtex@leftparen}
%    \begin{macrocode}
  |gdef|BKM@vtex@leftparen#1\(#2|@nil{%
    #1%
    |ifx||#2||%
    |else
      (%
      |ltx@ReturnAfterFi{%
        |BKM@vtex@leftparen#2|@nil
      }%
    |fi
  }%
%    \end{macrocode}
%    \end{macro}
%    \begin{macro}{\BKM@vtex@rightparen}
%    \begin{macrocode}
  |gdef|BKM@vtex@rightparen#1\)#2|@nil{%
    #1%
    |ifx||#2||%
    |else
      )%
      |ltx@ReturnAfterFi{%
        |BKM@vtex@rightparen#2|@nil
      }%
    |fi
  }%
%    \end{macrocode}
%    \end{macro}
%    \begin{macro}{\BKM@vtex@zero}
%    \begin{macrocode}
  |gdef|BKM@vtex@zero#1\0#2|@nil{%
    #1%
    |ifx||#2||%
    |else
      |noexpand|hv@pdf@char0%
      |ltx@ReturnAfterFi{%
        |BKM@vtex@zero#2|@nil
      }%
    |fi
  }%
%    \end{macrocode}
%    \end{macro}
%    \begin{macro}{\BKM@vtex@one}
%    \begin{macrocode}
  |gdef|BKM@vtex@one#1\1#2|@nil{%
    #1%
    |ifx||#2||%
    |else
      |noexpand|hv@pdf@char1%
      |ltx@ReturnAfterFi{%
        |BKM@vtex@one#2|@nil
      }%
    |fi
  }%
%    \end{macrocode}
%    \end{macro}
%    \begin{macro}{\BKM@vtex@two}
%    \begin{macrocode}
  |gdef|BKM@vtex@two#1\2#2|@nil{%
    #1%
    |ifx||#2||%
    |else
      |noexpand|hv@pdf@char2%
      |ltx@ReturnAfterFi{%
        |BKM@vtex@two#2|@nil
      }%
    |fi
  }%
%    \end{macrocode}
%    \end{macro}
%    \begin{macro}{\BKM@vtex@three}
%    \begin{macrocode}
  |gdef|BKM@vtex@three#1\3#2|@nil{%
    #1%
    |ifx||#2||%
    |else
      |noexpand|hv@pdf@char3%
      |ltx@ReturnAfterFi{%
        |BKM@vtex@three#2|@nil
      }%
    |fi
  }%
%    \end{macrocode}
%    \end{macro}
%    \begin{macrocode}
|endgroup
%    \end{macrocode}
%
%    \begin{macrocode}
%</vtex>
%    \end{macrocode}
%
% \subsection{\hologo{pdfTeX}\ 的驱动程序}
%
%    \begin{macrocode}
%<*pdftex>
\NeedsTeXFormat{LaTeX2e}
\ProvidesFile{bkm-pdftex.def}%
  [2020-11-06 v1.29 bookmark driver for pdfTeX (HO)]%
%    \end{macrocode}
%
%    \begin{macro}{\BKM@DO@entry}
%    \begin{macrocode}
\def\BKM@DO@entry#1#2{%
  \begingroup
    \kvsetkeys{BKM@DO}{#1}%
    \def\BKM@DO@title{#2}%
    \ifx\BKM@DO@srcfile\@empty
    \else
      \BKM@UnescapeHex\BKM@DO@srcfile
    \fi
    \BKM@UnescapeHex\BKM@DO@title
    \expandafter\expandafter\expandafter\BKM@getx
        \csname BKM@\BKM@DO@id\endcsname\@empty\@empty
    \let\BKM@attr\@empty
    \ifx\BKM@DO@flags\@empty
    \else
      \edef\BKM@attr{\BKM@attr/F \BKM@DO@flags}%
    \fi
    \ifx\BKM@DO@color\@empty
    \else
      \edef\BKM@attr{\BKM@attr/C[\BKM@DO@color]}%
    \fi
    \ifx\BKM@attr\@empty
    \else
      \edef\BKM@attr{attr{\BKM@attr}}%
    \fi
    \let\BKM@action\@empty
    \ifx\BKM@DO@gotor\@empty
      \ifx\BKM@DO@dest\@empty
        \ifx\BKM@DO@named\@empty
          \ifx\BKM@DO@rawaction\@empty
            \ifx\BKM@DO@uri\@empty
              \ifx\BKM@DO@page\@empty
                \PackageError{bookmark}{%
                  Missing action\BKM@SourceLocation
                }\@ehc
                \edef\BKM@action{goto page1{/Fit}}%
              \else
                \ifx\BKM@DO@view\@empty
                  \def\BKM@DO@view{Fit}%
                \fi
                \edef\BKM@action{goto page\BKM@DO@page{/\BKM@DO@view}}%
              \fi
            \else
              \BKM@UnescapeHex\BKM@DO@uri
              \BKM@EscapeString\BKM@DO@uri
              \edef\BKM@action{user{<</S/URI/URI(\BKM@DO@uri)>>}}%
            \fi
          \else
            \BKM@UnescapeHex\BKM@DO@rawaction
            \edef\BKM@action{%
              user{%
                <<%
                  \BKM@DO@rawaction
                >>%
              }%
            }%
          \fi
        \else
          \BKM@EscapeName\BKM@DO@named
          \edef\BKM@action{%
            user{<</S/Named/N/\BKM@DO@named>>}%
          }%
        \fi
      \else
        \BKM@UnescapeHex\BKM@DO@dest
        \BKM@DefGotoNameAction\BKM@action\BKM@DO@dest
      \fi
    \else
      \ifx\BKM@DO@dest\@empty
        \ifx\BKM@DO@page\@empty
          \def\BKM@DO@page{0}%
        \else
          \BKM@CalcExpr\BKM@DO@page\BKM@DO@page-1%
        \fi
        \ifx\BKM@DO@view\@empty
          \def\BKM@DO@view{Fit}%
        \fi
        \edef\BKM@action{/D[\BKM@DO@page/\BKM@DO@view]}%
      \else
        \BKM@UnescapeHex\BKM@DO@dest
        \BKM@EscapeString\BKM@DO@dest
        \edef\BKM@action{/D(\BKM@DO@dest)}%
      \fi
      \BKM@UnescapeHex\BKM@DO@gotor
      \BKM@EscapeString\BKM@DO@gotor
      \edef\BKM@action{%
        user{%
          <<%
            /S/GoToR%
            /F(\BKM@DO@gotor)%
            \BKM@action
          >>%
        }%
      }%
    \fi
    \pdfoutline\BKM@attr\BKM@action
                count\ifBKM@DO@open\else-\fi\BKM@x@childs
                {\BKM@DO@title}%
  \endgroup
}
%    \end{macrocode}
%    \end{macro}
%    \begin{macro}{\BKM@DefGotoNameAction}
%    \cs{BKM@DefGotoNameAction}\ 宏是一个用于 \xpackage{hypdestopt}\ 宏包的钩子(hook)。
%    \begin{macrocode}
\def\BKM@DefGotoNameAction#1#2{%
  \BKM@EscapeString\BKM@DO@dest
  \edef#1{goto name{#2}}%
}
%    \end{macrocode}
%    \end{macro}
%    \begin{macrocode}
%</pdftex>
%    \end{macrocode}
%
%    \begin{macrocode}
%<*pdftex|pdfmark>
%    \end{macrocode}
%    \begin{macro}{\BKM@SourceLocation}
%    \begin{macrocode}
\def\BKM@SourceLocation{%
  \ifx\BKM@DO@srcfile\@empty
    \ifx\BKM@DO@srcline\@empty
    \else
      .\MessageBreak
      Source: line \BKM@DO@srcline
    \fi
  \else
    \ifx\BKM@DO@srcline\@empty
      .\MessageBreak
      Source: file `\BKM@DO@srcfile'%
    \else
      .\MessageBreak
      Source: file `\BKM@DO@srcfile', line \BKM@DO@srcline
    \fi
  \fi
}
%    \end{macrocode}
%    \end{macro}
%    \begin{macrocode}
%</pdftex|pdfmark>
%    \end{macrocode}
%
% \subsection{具有 pdfmark 特色(specials)的驱动程序}
%
% \subsubsection{dvips 驱动程序}
%
%    \begin{macrocode}
%<*dvips>
\NeedsTeXFormat{LaTeX2e}
\ProvidesFile{bkm-dvips.def}%
  [2020-11-06 v1.29 bookmark driver for dvips (HO)]%
%    \end{macrocode}
%    \begin{macro}{\BKM@PSHeaderFile}
%    \begin{macrocode}
\def\BKM@PSHeaderFile#1{%
  \special{PSfile=#1}%
}
%    \end{macrocode}
%    \begin{macro}{\BKM@filename}
%    \begin{macrocode}
\def\BKM@filename{\jobname.out.ps}
%    \end{macrocode}
%    \end{macro}
%    \begin{macrocode}
\AddToHook{shipout/lastpage}{%
  \BKM@pdfmark@out
  \BKM@PSHeaderFile\BKM@filename
  }
%    \end{macrocode}
%    \end{macro}
%    \begin{macrocode}
%</dvips>
%    \end{macrocode}
%
% \subsubsection{公共部分(Common part)}
%
%    \begin{macrocode}
%<*pdfmark>
%    \end{macrocode}
%
%    \begin{macro}{\BKM@pdfmark@out}
%    不要在这里使用 \xpackage{rerunfilecheck}\ 宏包,因为在 \hologo{TeX}\ 运行期间不会
%    读取 \cs{BKM@filename}\ 文件。
%    \begin{macrocode}
\def\BKM@pdfmark@out{%
  \if@filesw
    \newwrite\BKM@file
    \immediate\openout\BKM@file=\BKM@filename\relax
    \BKM@write{\@percentchar!}%
    \BKM@write{/pdfmark where{pop}}%
    \BKM@write{%
      {%
        /globaldict where{pop globaldict}{userdict}ifelse%
        /pdfmark/cleartomark load put%
      }%
    }%
    \BKM@write{ifelse}%
  \else
    \let\BKM@write\@gobble
    \let\BKM@DO@entry\@gobbletwo
  \fi
}
%    \end{macrocode}
%    \end{macro}
%    \begin{macro}{\BKM@write}
%    \begin{macrocode}
\def\BKM@write#{%
  \immediate\write\BKM@file
}
%    \end{macrocode}
%    \end{macro}
%
%    \begin{macro}{\BKM@DO@entry}
%    Pdfmark 的规范(specification)说明 |/Color| 是颜色(color)的键名(key name),
%    但是 ghostscript 只将键(key)传递到 PDF 文件中,因此键名必须是 |/C|。
%    \begin{macrocode}
\def\BKM@DO@entry#1#2{%
  \begingroup
    \kvsetkeys{BKM@DO}{#1}%
    \ifx\BKM@DO@srcfile\@empty
    \else
      \BKM@UnescapeHex\BKM@DO@srcfile
    \fi
    \def\BKM@DO@title{#2}%
    \BKM@UnescapeHex\BKM@DO@title
    \expandafter\expandafter\expandafter\BKM@getx
        \csname BKM@\BKM@DO@id\endcsname\@empty\@empty
    \let\BKM@attr\@empty
    \ifx\BKM@DO@flags\@empty
    \else
      \edef\BKM@attr{\BKM@attr/F \BKM@DO@flags}%
    \fi
    \ifx\BKM@DO@color\@empty
    \else
      \edef\BKM@attr{\BKM@attr/C[\BKM@DO@color]}%
    \fi
    \let\BKM@action\@empty
    \ifx\BKM@DO@gotor\@empty
      \ifx\BKM@DO@dest\@empty
        \ifx\BKM@DO@named\@empty
          \ifx\BKM@DO@rawaction\@empty
            \ifx\BKM@DO@uri\@empty
              \ifx\BKM@DO@page\@empty
                \PackageError{bookmark}{%
                  Missing action\BKM@SourceLocation
                }\@ehc
                \edef\BKM@action{%
                  /Action/GoTo%
                  /Page 1%
                  /View[/Fit]%
                }%
              \else
                \ifx\BKM@DO@view\@empty
                  \def\BKM@DO@view{Fit}%
                \fi
                \edef\BKM@action{%
                  /Action/GoTo%
                  /Page \BKM@DO@page
                  /View[/\BKM@DO@view]%
                }%
              \fi
            \else
              \BKM@UnescapeHex\BKM@DO@uri
              \BKM@EscapeString\BKM@DO@uri
              \edef\BKM@action{%
                /Action<<%
                  /Subtype/URI%
                  /URI(\BKM@DO@uri)%
                >>%
              }%
            \fi
          \else
            \BKM@UnescapeHex\BKM@DO@rawaction
            \edef\BKM@action{%
              /Action<<%
                \BKM@DO@rawaction
              >>%
            }%
          \fi
        \else
          \BKM@EscapeName\BKM@DO@named
          \edef\BKM@action{%
            /Action<<%
              /Subtype/Named%
              /N/\BKM@DO@named
            >>%
          }%
        \fi
      \else
        \BKM@UnescapeHex\BKM@DO@dest
        \BKM@EscapeString\BKM@DO@dest
        \edef\BKM@action{%
          /Action/GoTo%
          /Dest(\BKM@DO@dest)cvn%
        }%
      \fi
    \else
      \ifx\BKM@DO@dest\@empty
        \ifx\BKM@DO@page\@empty
          \def\BKM@DO@page{1}%
        \fi
        \ifx\BKM@DO@view\@empty
          \def\BKM@DO@view{Fit}%
        \fi
        \edef\BKM@action{%
          /Page \BKM@DO@page
          /View[/\BKM@DO@view]%
        }%
      \else
        \BKM@UnescapeHex\BKM@DO@dest
        \BKM@EscapeString\BKM@DO@dest
        \edef\BKM@action{%
          /Dest(\BKM@DO@dest)cvn%
        }%
      \fi
      \BKM@UnescapeHex\BKM@DO@gotor
      \BKM@EscapeString\BKM@DO@gotor
      \edef\BKM@action{%
        /Action/GoToR%
        /File(\BKM@DO@gotor)%
        \BKM@action
      }%
    \fi
    \BKM@write{[}%
    \BKM@write{/Title(\BKM@DO@title)}%
    \ifnum\BKM@x@childs>\z@
      \BKM@write{/Count \ifBKM@DO@open\else-\fi\BKM@x@childs}%
    \fi
    \ifx\BKM@attr\@empty
    \else
      \BKM@write{\BKM@attr}%
    \fi
    \BKM@write{\BKM@action}%
    \BKM@write{/OUT pdfmark}%
  \endgroup
}
%    \end{macrocode}
%    \end{macro}
%    \begin{macrocode}
%</pdfmark>
%    \end{macrocode}
%
% \subsection{\xoption{pdftex}\ 和 \xoption{pdfmark}\ 的公共部分}
%
%    \begin{macrocode}
%<*pdftex|pdfmark>
%    \end{macrocode}
%
% \subsubsection{写入辅助文件(auxiliary file)}
%
%    \begin{macrocode}
\AddToHook{begindocument}{%
 \immediate\write\@mainaux{\string\providecommand\string\BKM@entry[2]{}}}
%    \end{macrocode}
%
%    \begin{macro}{\BKM@id}
%    \begin{macrocode}
\newcount\BKM@id
\BKM@id=\z@
%    \end{macrocode}
%    \end{macro}
%
%    \begin{macro}{\BKM@0}
%    \begin{macrocode}
\@namedef{BKM@0}{000}
%    \end{macrocode}
%    \end{macro}
%    \begin{macro}{\ifBKM@sw}
%    \begin{macrocode}
\newif\ifBKM@sw
%    \end{macrocode}
%    \end{macro}
%
%    \begin{macro}{\bookmark}
%    \begin{macrocode}
\newcommand*{\bookmark}[2][]{%
  \if@filesw
    \begingroup
      \BKM@InitSourceLocation
      \def\bookmark@text{#2}%
      \BKM@setup{#1}%
      \ifx\BKM@srcfile\@empty
      \else
        \BKM@EscapeHex\BKM@srcfile
      \fi
      \edef\BKM@prev{\the\BKM@id}%
      \global\advance\BKM@id\@ne
      \BKM@swtrue
      \@whilesw\ifBKM@sw\fi{%
        \ifnum\ifBKM@startatroot\z@\else\BKM@prev\fi=\z@
          \BKM@startatrootfalse
          \expandafter\xdef\csname BKM@\the\BKM@id\endcsname{%
            0{\BKM@level}0%
          }%
          \BKM@swfalse
        \else
          \expandafter\expandafter\expandafter\BKM@getx
              \csname BKM@\BKM@prev\endcsname
          \ifnum\BKM@level>\BKM@x@level\relax
            \expandafter\xdef\csname BKM@\the\BKM@id\endcsname{%
              {\BKM@prev}{\BKM@level}0%
            }%
            \ifnum\BKM@prev>\z@
              \BKM@CalcExpr\BKM@CalcResult\BKM@x@childs+1%
              \expandafter\xdef\csname BKM@\BKM@prev\endcsname{%
                {\BKM@x@parent}{\BKM@x@level}{\BKM@CalcResult}%
              }%
            \fi
            \BKM@swfalse
          \else
            \let\BKM@prev\BKM@x@parent
          \fi
        \fi
      }%
      \pdfstringdef\BKM@title{\bookmark@text}%
      \edef\BKM@FLAGS{\BKM@PrintStyle}%
      \csname BKM@HypDestOptHook\endcsname
      \BKM@EscapeHex\BKM@dest
      \BKM@EscapeHex\BKM@uri
      \BKM@EscapeHex\BKM@gotor
      \BKM@EscapeHex\BKM@rawaction
      \BKM@EscapeHex\BKM@title
      \immediate\write\@mainaux{%
        \string\BKM@entry{%
          id=\number\BKM@id
          \ifBKM@open
            \ifnum\BKM@level<\BKM@openlevel
              ,open%
            \fi
          \fi
          \BKM@auxentry{dest}%
          \BKM@auxentry{named}%
          \BKM@auxentry{uri}%
          \BKM@auxentry{gotor}%
          \BKM@auxentry{page}%
          \BKM@auxentry{view}%
          \BKM@auxentry{rawaction}%
          \BKM@auxentry{color}%
          \ifnum\BKM@FLAGS>\z@
            ,flags=\BKM@FLAGS
          \fi
          \BKM@auxentry{srcline}%
          \BKM@auxentry{srcfile}%
        }{\BKM@title}%
      }%
    \endgroup
  \fi
}
%    \end{macrocode}
%    \end{macro}
%    \begin{macro}{\BKM@getx}
%    \begin{macrocode}
\def\BKM@getx#1#2#3{%
  \def\BKM@x@parent{#1}%
  \def\BKM@x@level{#2}%
  \def\BKM@x@childs{#3}%
}
%    \end{macrocode}
%    \end{macro}
%    \begin{macro}{\BKM@auxentry}
%    \begin{macrocode}
\def\BKM@auxentry#1{%
  \expandafter\ifx\csname BKM@#1\endcsname\@empty
  \else
    ,#1={\csname BKM@#1\endcsname}%
  \fi
}
%    \end{macrocode}
%    \end{macro}
%
%    \begin{macro}{\BKM@InitSourceLocation}
%    \begin{macrocode}
\def\BKM@InitSourceLocation{%
  \edef\BKM@srcline{\the\inputlineno}%
  \BKM@LuaTeX@InitFile
  \ifx\BKM@srcfile\@empty
    \ltx@IfUndefined{currfilepath}{}{%
      \edef\BKM@srcfile{\currfilepath}%
    }%
  \fi
}
%    \end{macrocode}
%    \end{macro}
%    \begin{macro}{\BKM@LuaTeX@InitFile}
%    \begin{macrocode}
\ifluatex
  \ifnum\luatexversion>36 %
    \def\BKM@LuaTeX@InitFile{%
      \begingroup
        \ltx@LocToksA={}%
      \edef\x{\endgroup
        \def\noexpand\BKM@srcfile{%
          \the\expandafter\ltx@LocToksA
          \directlua{%
             if status and status.filename then %
               tex.settoks('ltx@LocToksA', status.filename)%
             end%
          }%
        }%
      }\x
    }%
  \else
    \let\BKM@LuaTeX@InitFile\relax
  \fi
\else
  \let\BKM@LuaTeX@InitFile\relax
\fi
%    \end{macrocode}
%    \end{macro}
%
% \subsubsection{读取辅助数据(auxiliary data)}
%
%    \begin{macrocode}
\SetupKeyvalOptions{family=BKM@DO,prefix=BKM@DO@}
\DeclareStringOption[0]{id}
\DeclareBoolOption{open}
\DeclareStringOption{flags}
\DeclareStringOption{color}
\DeclareStringOption{dest}
\DeclareStringOption{named}
\DeclareStringOption{uri}
\DeclareStringOption{gotor}
\DeclareStringOption{page}
\DeclareStringOption{view}
\DeclareStringOption{rawaction}
\DeclareStringOption{srcline}
\DeclareStringOption{srcfile}
%    \end{macrocode}
%
%    \begin{macrocode}
\AtBeginDocument{%
  \let\BKM@entry\BKM@DO@entry
}
%    \end{macrocode}
%
%    \begin{macrocode}
%</pdftex|pdfmark>
%    \end{macrocode}
%
% \subsection{\xoption{atend}\ 选项}
%
% \subsubsection{钩子(Hook)}
%
%    \begin{macrocode}
%<*package>
%    \end{macrocode}
%    \begin{macrocode}
\ifBKM@atend
\else
%    \end{macrocode}
%    \begin{macro}{\BookmarkAtEnd}
%    这是一个虚拟定义(dummy definition),如果没有给出 \xoption{atend}\ 选项,它将生成一个警告。
%    \begin{macrocode}
  \newcommand{\BookmarkAtEnd}[1]{%
    \PackageWarning{bookmark}{%
      Ignored, because option `atend' is missing%
    }%
  }%
%    \end{macrocode}
%    \end{macro}
%    \begin{macrocode}
  \expandafter\endinput
\fi
%    \end{macrocode}
%    \begin{macro}{\BookmarkAtEnd}
%    \begin{macrocode}
\newcommand*{\BookmarkAtEnd}{%
  \g@addto@macro\BKM@EndHook
}
%    \end{macrocode}
%    \end{macro}
%    \begin{macrocode}
\let\BKM@EndHook\@empty
%    \end{macrocode}
%    \begin{macrocode}
%</package>
%    \end{macrocode}
%
% \subsubsection{在文档末尾使用钩子的驱动程序}
%
%    驱动程序 \xoption{pdftex}\ 使用 LaTeX 钩子 \xoption{enddocument/afterlastpage}
%    (相当于以前使用的 \xpackage{atveryend}\ 的 \cs{AfterLastShipout}),因为它仍然需要 \xext{aux}\ 文件。
%    它使用 \cs{pdfoutline}\ 作为最后一页之后可以使用的书签(bookmakrs)。
%    \begin{itemize}
%    \item
%      驱动程序 \xoption{pdftex}\ 使用 \cs{pdfoutline}, \cs{pdfoutline}\ 可以在最后一页之后使用。
%    \end{itemize}
%    \begin{macrocode}
%<*pdftex>
\ifBKM@atend
  \AddToHook{enddocument/afterlastpage}{%
    \BKM@EndHook
  }%
\fi
%</pdftex>
%    \end{macrocode}
%
% \subsubsection{使用 \xoption{shipout/lastpage}\ 的驱动程序}
%
%    其他驱动程序使用 \cs{special}\ 命令实现 \cs{bookmark}。因此,最后的书签(last bookmarks)
%    必须放在最后一页(last page),而不是之后。不能使用 \cs{AtEndDocument},因为为时已晚,
%    最后一页已经输出了。因此,我们使用 LaTeX 钩子 \xoption{shipout/lastpage}。至少需要运行
%    两次 \hologo{LaTeX}。PostScript 驱动程序 \xoption{dvips}\ 使用外部 PostScript 文件作为书签。
%    为了避免与 pgf 发生冲突,文件写入(file writing)也被移到了最后一个输出页面(shipout page)。
%    \begin{macrocode}
%<*dvipdfm|vtex|pdfmark>
\ifBKM@atend
  \AddToHook{shipout/lastpage}{\BKM@EndHook}%
\fi
%</dvipdfm|vtex|pdfmark>
%    \end{macrocode}
%
% \section{安装(Installation)}
%
% \subsection{下载(Download)}
%
% \paragraph{宏包(Package)。} 在 CTAN\footnote{\CTANpkg{bookmark}}上提供此宏包:
% \begin{description}
% \item[\CTAN{macros/latex/contrib/bookmark/bookmark.dtx}] 源文件(source file)。
% \item[\CTAN{macros/latex/contrib/bookmark/bookmark.pdf}] 文档(documentation)。
% \end{description}
%
%
% \paragraph{捆绑包(Bundle)。} “bookmark”捆绑包(bundle)的所有宏包(packages)都可以在兼
% 容 TDS 的 ZIP 归档文件中找到。在那里,宏包已经被解包,文档文件(documentation files)已经生成。
% 文件(files)和目录(directories)遵循 TDS 标准。
% \begin{description}
% \item[\CTANinstall{install/macros/latex/contrib/bookmark.tds.zip}]
% \end{description}
% \emph{TDS}\ 是指标准的“用于 \TeX\ 文件的目录结构(Directory Structure)”(\CTANpkg{tds})。
% 名称中带有 \xfile{texmf}\ 的目录(directories)通常以这种方式组织。
%
% \subsection{捆绑包(Bundle)的安装}
%
% \paragraph{解压(Unpacking)。} 在您选择的 TDS 树(也称为 \xfile{texmf}\ 树)中解
% 压 \xfile{bookmark.tds.zip},例如(在 linux 中):
% \begin{quote}
%   |unzip bookmark.tds.zip -d ~/texmf|
% \end{quote}
%
% \subsection{宏包(Package)的安装}
%
% \paragraph{解压(Unpacking)。} \xfile{.dtx}\ 文件是一个自解压 \docstrip\ 归档文件(archive)。
% 这些文件是通过 \plainTeX\ 运行 \xfile{.dtx}\ 来提取的:
% \begin{quote}
%   \verb|tex bookmark.dtx|
% \end{quote}
%
% \paragraph{TDS.} 现在,不同的文件必须移动到安装 TDS 树(installation TDS tree)
% (也称为 \xfile{texmf}\ 树)中的不同目录中:
% \begin{quote}
% \def\t{^^A
% \begin{tabular}{@{}>{\ttfamily}l@{ $\rightarrow$ }>{\ttfamily}l@{}}
%   bookmark.sty & tex/latex/bookmark/bookmark.sty\\
%   bkm-dvipdfm.def & tex/latex/bookmark/bkm-dvipdfm.def\\
%   bkm-dvips.def & tex/latex/bookmark/bkm-dvips.def\\
%   bkm-pdftex.def & tex/latex/bookmark/bkm-pdftex.def\\
%   bkm-vtex.def & tex/latex/bookmark/bkm-vtex.def\\
%   bookmark.pdf & doc/latex/bookmark/bookmark.pdf\\
%   bookmark-example.tex & doc/latex/bookmark/bookmark-example.tex\\
%   bookmark.dtx & source/latex/bookmark/bookmark.dtx\\
% \end{tabular}^^A
% }^^A
% \sbox0{\t}^^A
% \ifdim\wd0>\linewidth
%   \begingroup
%     \advance\linewidth by\leftmargin
%     \advance\linewidth by\rightmargin
%   \edef\x{\endgroup
%     \def\noexpand\lw{\the\linewidth}^^A
%   }\x
%   \def\lwbox{^^A
%     \leavevmode
%     \hbox to \linewidth{^^A
%       \kern-\leftmargin\relax
%       \hss
%       \usebox0
%       \hss
%       \kern-\rightmargin\relax
%     }^^A
%   }^^A
%   \ifdim\wd0>\lw
%     \sbox0{\small\t}^^A
%     \ifdim\wd0>\linewidth
%       \ifdim\wd0>\lw
%         \sbox0{\footnotesize\t}^^A
%         \ifdim\wd0>\linewidth
%           \ifdim\wd0>\lw
%             \sbox0{\scriptsize\t}^^A
%             \ifdim\wd0>\linewidth
%               \ifdim\wd0>\lw
%                 \sbox0{\tiny\t}^^A
%                 \ifdim\wd0>\linewidth
%                   \lwbox
%                 \else
%                   \usebox0
%                 \fi
%               \else
%                 \lwbox
%               \fi
%             \else
%               \usebox0
%             \fi
%           \else
%             \lwbox
%           \fi
%         \else
%           \usebox0
%         \fi
%       \else
%         \lwbox
%       \fi
%     \else
%       \usebox0
%     \fi
%   \else
%     \lwbox
%   \fi
% \else
%   \usebox0
% \fi
% \end{quote}
% 如果你有一个 \xfile{docstrip.cfg}\ 文件,该文件能配置并启用 \docstrip\ 的 TDS 安装功能,
% 则一些文件可能已经在正确的位置了,请参阅 \docstrip\ 的文档(documentation)。
%
% \subsection{刷新文件名数据库}
%
% 如果您的 \TeX~发行版(\TeX\,Live、\mikTeX、\dots)依赖于文件名数据库(file name databases),
% 则必须刷新这些文件名数据库。例如,\TeX\,Live\ 用户运行 \verb|texhash| 或 \verb|mktexlsr|。
%
% \subsection{一些感兴趣的细节}
%
% \paragraph{用 \LaTeX\ 解压。}
% \xfile{.dtx}\ 根据格式(format)选择其操作(action):
% \begin{description}
% \item[\plainTeX:] 运行 \docstrip\ 并解压文件。
% \item[\LaTeX:] 生成文档。
% \end{description}
% 如果您坚持通过 \LaTeX\ 使用\docstrip (实际上 \docstrip\ 并不需要 \LaTeX),那么请您的意图告知自动检测程序:
% \begin{quote}
%   \verb|latex \let\install=y% \iffalse meta-comment
%
% File: bookmark.dtx
% Version: 2020-11-06 v1.29
% Info: PDF bookmarks
%
% Copyright (C)
%    2007-2011 Heiko Oberdiek
%    2016-2020 Oberdiek Package Support Group
%    https://github.com/ho-tex/bookmark/issues
%
% This work may be distributed and/or modified under the
% conditions of the LaTeX Project Public License, either
% version 1.3c of this license or (at your option) any later
% version. This version of this license is in
%    https://www.latex-project.org/lppl/lppl-1-3c.txt
% and the latest version of this license is in
%    https://www.latex-project.org/lppl.txt
% and version 1.3 or later is part of all distributions of
% LaTeX version 2005/12/01 or later.
%
% This work has the LPPL maintenance status "maintained".
%
% The Current Maintainers of this work are
% Heiko Oberdiek and the Oberdiek Package Support Group
% https://github.com/ho-tex/bookmark/issues
%
% This work consists of the main source file bookmark.dtx
% and the derived files
%    bookmark.sty, bookmark.pdf, bookmark.ins, bookmark.drv,
%    bkm-dvipdfm.def, bkm-dvips.def,
%    bkm-pdftex.def, bkm-vtex.def,
%    bkm-dvipdfm-2019-12-03.def, bkm-dvips-2019-12-03.def,
%    bkm-pdftex-2019-12-03.def, bkm-vtex-2019-12-03.def,
%    bookmark-example.tex.
%
% Distribution:
%    CTAN:macros/latex/contrib/bookmark/bookmark.dtx
%    CTAN:macros/latex/contrib/bookmark/bookmark-frozen.dtx
%    CTAN:macros/latex/contrib/bookmark/bookmark.pdf
%
% Unpacking:
%    (a) If bookmark.ins is present:
%           tex bookmark.ins
%    (b) Without bookmark.ins:
%           tex bookmark.dtx
%    (c) If you insist on using LaTeX
%           latex \let\install=y% \iffalse meta-comment
%
% File: bookmark.dtx
% Version: 2020-11-06 v1.29
% Info: PDF bookmarks
%
% Copyright (C)
%    2007-2011 Heiko Oberdiek
%    2016-2020 Oberdiek Package Support Group
%    https://github.com/ho-tex/bookmark/issues
%
% This work may be distributed and/or modified under the
% conditions of the LaTeX Project Public License, either
% version 1.3c of this license or (at your option) any later
% version. This version of this license is in
%    https://www.latex-project.org/lppl/lppl-1-3c.txt
% and the latest version of this license is in
%    https://www.latex-project.org/lppl.txt
% and version 1.3 or later is part of all distributions of
% LaTeX version 2005/12/01 or later.
%
% This work has the LPPL maintenance status "maintained".
%
% The Current Maintainers of this work are
% Heiko Oberdiek and the Oberdiek Package Support Group
% https://github.com/ho-tex/bookmark/issues
%
% This work consists of the main source file bookmark.dtx
% and the derived files
%    bookmark.sty, bookmark.pdf, bookmark.ins, bookmark.drv,
%    bkm-dvipdfm.def, bkm-dvips.def,
%    bkm-pdftex.def, bkm-vtex.def,
%    bkm-dvipdfm-2019-12-03.def, bkm-dvips-2019-12-03.def,
%    bkm-pdftex-2019-12-03.def, bkm-vtex-2019-12-03.def,
%    bookmark-example.tex.
%
% Distribution:
%    CTAN:macros/latex/contrib/bookmark/bookmark.dtx
%    CTAN:macros/latex/contrib/bookmark/bookmark-frozen.dtx
%    CTAN:macros/latex/contrib/bookmark/bookmark.pdf
%
% Unpacking:
%    (a) If bookmark.ins is present:
%           tex bookmark.ins
%    (b) Without bookmark.ins:
%           tex bookmark.dtx
%    (c) If you insist on using LaTeX
%           latex \let\install=y\input{bookmark.dtx}
%        (quote the arguments according to the demands of your shell)
%
% Documentation:
%    (a) If bookmark.drv is present:
%           latex bookmark.drv
%    (b) Without bookmark.drv:
%           latex bookmark.dtx; ...
%    The class ltxdoc loads the configuration file ltxdoc.cfg
%    if available. Here you can specify further options, e.g.
%    use A4 as paper format:
%       \PassOptionsToClass{a4paper}{article}
%
%    Programm calls to get the documentation (example):
%       pdflatex bookmark.dtx
%       makeindex -s gind.ist bookmark.idx
%       pdflatex bookmark.dtx
%       makeindex -s gind.ist bookmark.idx
%       pdflatex bookmark.dtx
%
% Installation:
%    TDS:tex/latex/bookmark/bookmark.sty
%    TDS:tex/latex/bookmark/bkm-dvipdfm.def
%    TDS:tex/latex/bookmark/bkm-dvips.def
%    TDS:tex/latex/bookmark/bkm-pdftex.def
%    TDS:tex/latex/bookmark/bkm-vtex.def
%    TDS:tex/latex/bookmark/bkm-dvipdfm-2019-12-03.def
%    TDS:tex/latex/bookmark/bkm-dvips-2019-12-03.def
%    TDS:tex/latex/bookmark/bkm-pdftex-2019-12-03.def
%    TDS:tex/latex/bookmark/bkm-vtex-2019-12-03.def%
%    TDS:doc/latex/bookmark/bookmark.pdf
%    TDS:doc/latex/bookmark/bookmark-example.tex
%    TDS:source/latex/bookmark/bookmark.dtx
%    TDS:source/latex/bookmark/bookmark-frozen.dtx
%
%<*ignore>
\begingroup
  \catcode123=1 %
  \catcode125=2 %
  \def\x{LaTeX2e}%
\expandafter\endgroup
\ifcase 0\ifx\install y1\fi\expandafter
         \ifx\csname processbatchFile\endcsname\relax\else1\fi
         \ifx\fmtname\x\else 1\fi\relax
\else\csname fi\endcsname
%</ignore>
%<*install>
\input docstrip.tex
\Msg{************************************************************************}
\Msg{* Installation}
\Msg{* Package: bookmark 2020-11-06 v1.29 PDF bookmarks (HO)}
\Msg{************************************************************************}

\keepsilent
\askforoverwritefalse

\let\MetaPrefix\relax
\preamble

This is a generated file.

Project: bookmark
Version: 2020-11-06 v1.29

Copyright (C)
   2007-2011 Heiko Oberdiek
   2016-2020 Oberdiek Package Support Group

This work may be distributed and/or modified under the
conditions of the LaTeX Project Public License, either
version 1.3c of this license or (at your option) any later
version. This version of this license is in
   https://www.latex-project.org/lppl/lppl-1-3c.txt
and the latest version of this license is in
   https://www.latex-project.org/lppl.txt
and version 1.3 or later is part of all distributions of
LaTeX version 2005/12/01 or later.

This work has the LPPL maintenance status "maintained".

The Current Maintainers of this work are
Heiko Oberdiek and the Oberdiek Package Support Group
https://github.com/ho-tex/bookmark/issues


This work consists of the main source file bookmark.dtx and bookmark-frozen.dtx
and the derived files
   bookmark.sty, bookmark.pdf, bookmark.ins, bookmark.drv,
   bkm-dvipdfm.def, bkm-dvips.def, bkm-pdftex.def, bkm-vtex.def,
   bkm-dvipdfm-2019-12-03.def, bkm-dvips-2019-12-03.def,
   bkm-pdftex-2019-12-03.def, bkm-vtex-2019-12-03.def,
   bookmark-example.tex.

\endpreamble
\let\MetaPrefix\DoubleperCent

\generate{%
  \file{bookmark.ins}{\from{bookmark.dtx}{install}}%
  \file{bookmark.drv}{\from{bookmark.dtx}{driver}}%
  \usedir{tex/latex/bookmark}%
  \file{bookmark.sty}{\from{bookmark.dtx}{package}}%
  \file{bkm-dvipdfm.def}{\from{bookmark.dtx}{dvipdfm}}%
  \file{bkm-dvips.def}{\from{bookmark.dtx}{dvips,pdfmark}}%
  \file{bkm-pdftex.def}{\from{bookmark.dtx}{pdftex}}%
  \file{bkm-vtex.def}{\from{bookmark.dtx}{vtex}}%
  \usedir{doc/latex/bookmark}%
  \file{bookmark-example.tex}{\from{bookmark.dtx}{example}}%
  \file{bkm-pdftex-2019-12-03.def}{\from{bookmark-frozen.dtx}{pdftexfrozen}}%
  \file{bkm-dvips-2019-12-03.def}{\from{bookmark-frozen.dtx}{dvipsfrozen}}%
  \file{bkm-vtex-2019-12-03.def}{\from{bookmark-frozen.dtx}{vtexfrozen}}%
  \file{bkm-dvipdfm-2019-12-03.def}{\from{bookmark-frozen.dtx}{dvipdfmfrozen}}%
}

\catcode32=13\relax% active space
\let =\space%
\Msg{************************************************************************}
\Msg{*}
\Msg{* To finish the installation you have to move the following}
\Msg{* files into a directory searched by TeX:}
\Msg{*}
\Msg{*     bookmark.sty, bkm-dvipdfm.def, bkm-dvips.def,}
\Msg{*     bkm-pdftex.def, bkm-vtex.def, bkm-dvipdfm-2019-12-03.def,}
\Msg{*     bkm-dvips-2019-12-03.def, bkm-pdftex-2019-12-03.def,}
\Msg{*     and bkm-vtex-2019-12-03.def}
\Msg{*}
\Msg{* To produce the documentation run the file `bookmark.drv'}
\Msg{* through LaTeX.}
\Msg{*}
\Msg{* Happy TeXing!}
\Msg{*}
\Msg{************************************************************************}

\endbatchfile
%</install>
%<*ignore>
\fi
%</ignore>
%<*driver>
\NeedsTeXFormat{LaTeX2e}
\ProvidesFile{bookmark.drv}%
  [2020-11-06 v1.29 PDF bookmarks (HO)]%
\documentclass{ltxdoc}
\usepackage{ctex}
\usepackage{indentfirst}
\setlength{\parindent}{2em}
\usepackage{holtxdoc}[2011/11/22]
\usepackage{xcolor}
\usepackage{hyperref}
\usepackage[open,openlevel=3,atend]{bookmark}[2020/11/06] %%%打开书签,显示的深度为3级,即显示part、section、subsection。
\bookmarksetup{color=red}
\begin{document}

  \renewcommand{\contentsname}{目\quad 录}
  \renewcommand{\abstractname}{摘\quad 要}
  \renewcommand{\historyname}{历史}
  \DocInput{bookmark.dtx}%
\end{document}
%</driver>
% \fi
%
%
%
% \GetFileInfo{bookmark.drv}
%
%% \title{\xpackage{bookmark} 宏包}
% \title{\heiti {\Huge \textbf{\xpackage{bookmark}\ 宏包}}}
% \date{2020-11-06\ \ \ v1.29}
% \author{Heiko Oberdiek \thanks
% {如有问题请点击:\url{https://github.com/ho-tex/bookmark/issues}}\\[5pt]赣医一附院神经科\ \ 黄旭华\ \ \ \ 译}
%
% \maketitle
%
% \begin{abstract}
% 这个宏包为 \xpackage{hyperref}\ 宏包实现了一个新的书签(bookmark)(大纲[outline])组织。现在
% 可以设置样式(style)和颜色(color)等书签属性(bookmark properties)。其他动作类型(action types)可用
% (URI、GoToR、Named)。书签是在第一次编译运行(compile run)中生成的。\xpackage{hyperref}\
% 宏包必需运行两次。
% \end{abstract}
%
% \tableofcontents
%
% \section{文档(Documentation)}
%
% \subsection{介绍}
%
% 这个 \xpackage{bookmark}\ 宏包试图为书签(bookmarks)提供一个更现代的管理:
% \begin{itemize}
% \item 书签已经在第一次 \hologo{TeX}\ 编译运行(compile run)中生成。
% \item 可以更改书签的字体样式(font style)和颜色(color)。
% \item 可以执行比简单的 GoTo 操作(actions)更多的操作。
% \end{itemize}
%
% 与 \xpackage{hyperref} \cite{hyperref} 一样,书签(bookmarks)也是按照书签生成宏
% (bookmark generating macros)(\cs{bookmark})的顺序生成的。级别号(level number)用于
% 定义书签的树结构(tree structure)。限制没有那么严格:
% \begin{itemize}
% \item 级别值(level values)可以跳变(jump)和省略(omit)。\cs{subsubsection}\ 可以跟在
%       \cs{chapter}\ 之后。这种情况如在 \xpackage{hyperref}\ 中则产生错误,它将显示一个警告(warning)
%       并尝试修复此错误。
% \item 多个书签可能指向同一目标(destination)。在 \xpackage{hyperref}\ 中,这会完全弄乱
%       书签树(bookmark tree),因为算法假设(algorithm assumes)目标名称(destination names)
%       是键(keys)(唯一的)。
% \end{itemize}
%
% 注意,这个宏包是作为书签管理(bookmark management)的实验平台(experimentation platform)。
% 欢迎反馈。此外,在未来的版本中,接口(interfaces)也可能发生变化。
%
% \subsection{选项(Options)}
%
% 可在以下四个地方放置选项(options):
% \begin{enumerate}
% \item \cs{usepackage}|[|\meta{options}|]{bookmark}|\\
%       这是放置驱动程序选项(driver options)和 \xoption{atend}\ 选项的唯一位置。
% \item \cs{bookmarksetup}|{|\meta{options}|}|\\
%       此命令仅用于设置选项(setting options)。
% \item \cs{bookmarksetupnext}|{|\meta{options}|}|\\
%       这些选项在下一个 \cs{bookmark}\ 命令的选项之后存储(stored)和调用(called)。
% \item \cs{bookmark}|[|\meta{options}|]{|\meta{title}|}|\\
%       此命令设置书签。选项设置(option settings)仅限于此书签。
% \end{enumerate}
% 异常(Exception):加载该宏包后,无法更改驱动程序选项(Driver options)、\xoption{atend}\ 选项
% 、\xoption{draft}\slash\xoption{final}选项。
%
% \subsubsection{\xoption{draft} 和 \xoption{final}\ 选项}
%
% 如果一个\LaTeX\ 文件要被编译了多次,那么可以使用 \xoption{draft}\ 选项来禁用该宏包的书签内
% 容(bookmark stuff),这样可以节省一点时间。默认 \xoption{final}\ 选项。两个选项都是
% 布尔选项(boolean options),如果没有值,则使用值 |true|。|draft=true| 与 |final=false| 相同。
%
% 除了驱动程序选项(driver options)之外,\xpackage{bookmark}\ 宏包选项都是局部选项(local options)。
% \xoption{draft}\ 选项和 \xoption{final}\ 选项均属于文档类选项(class option)(译者注:文档类选项为全局选项),
% 因此,在 \xpackage{bookmark}\ 宏包中未能看到这两个选项。如果您想使用全局的(global) \xoption{draft}选项
% 来优化第一次 \LaTeX\ 运行(runs),可以在导言(preamble)中引入 \xpackage{ifdraft}\ 宏包并设置 \LaTeX\ 的
% \cs{PassOptionsToPackage},例如:
%\begin{quote}
%\begin{verbatim}
%\documentclass[draft]{article}
%\usepackage{ifdraft}
%\ifdraft{%
%   \PassOptionsToPackage{draft}{bookmark}%
%}{}
%\end{verbatim}
%\end{quote}
%
% \subsubsection{驱动程序选项(Driver options)}
%
% 支持的驱动程序( drivers)包括 \xoption{pdftex}、\xoption{dvips}、\xoption{dvipdfm} (\xoption{xetex})、
% \xoption{vtex}。\hologo{TeX}\ 引擎 \hologo{pdfTeX}、\hologo{XeTeX}、\hologo{VTeX}\ 能被自动检测到。
% 默认的 DVI 驱动程序是 \xoption{dvips}。这可以通过 \cs{BookmarkDriverDefault}\ 在配置
% 文件 \xfile{bookmark.cfg}\ 中进行更改,例如:
% \begin{quote}
% |\def\BookmarkDriverDefault{dvipdfm}|
% \end{quote}
% 当前版本的(current versions)驱动程序使用新的 \LaTeX\ 钩子(\LaTeX-hooks)。如果检测到比
% 2020-10-01 更旧的格式,则将以前驱动程序的冻结版本(frozen versions)作为备份(fallback)。
%
% \paragraph{用 dvipdfmx 打开书签(bookmarks)。}旧版本的宏包有一个 \xoption{dvipdfmx-outline-open}\ 选项
% 可以激活代码,而该代码可以指定一个大纲条目(outline entry)是否打开。该宏包现在假设所有使用的 dvipdfmx 版本都是
% 最新版本,足以理解该代码,因此始终激活该代码。选项本身将被忽略。
%
%
% \subsubsection{布局选项(Layout options)}
%
% \paragraph{字体(Font)选项:}
%
% \begin{description}
% \item[\xoption{bold}:] 如果受 PDF 浏览器(PDF viewer)支持,书签将以粗体字体(bold font)显示(自 PDF 1.4起)。
% \item[\xoption{italic}:] 使用斜体字体(italic font)(自 PDF 1.4起)。
% \end{description}
% \xoption{bold}(粗体) 和 \xoption{italic}(斜体)可以同时使用。而 |false| 值(value)禁用字体选项。
%
% \paragraph{颜色(Color)选项:}
%
% 彩色书签(Colored bookmarks)是 PDF 1.4 的一个特性(feature),并非所有的 PDF 浏览器(PDF viewers)都支持彩色书签。
% \begin{description}
% \item[\xoption{color}:] 这里 color(颜色)可以作为 \xpackage{color}\ 宏包或 \xpackage{xcolor}\ 宏包的
% 颜色规范(color specification)给出。空值(empty value)表示未设置颜色属性。如果未加载 \xpackage{xcolor}\ 宏包,
% 能识别的值(recognized values)只有:
%   \begin{itemize}
%   \item 空值(empty value)表示未设置颜色属性,\\
%         例如:|color={}|
%   \item 颜色模型(color model) rgb 的显式颜色规范(explicit color specification),\\
%         例如,红色(red):|color=[rgb]{1,0,0}|
%   \item 颜色模型(color model)灰(gray)的显式颜色规范(explicit color specification),\\
%         例如,深灰色(dark gray):|color=[gray]{0.25}|
%   \end{itemize}
%   请注意,如果加载了 \xpackage{color}\ 宏包,此限制(restriction)也适用。然而,如果加载了 \xpackage{xcolor}\ 宏包,
%   则可以使用所有颜色规范(color specifications)。
% \end{description}
%
% \subsubsection{动作选项(Action options)}
%
% \begin{description}
% \item[\xoption{dest}:] 目的地名称(destination name)。
% \item[\xoption{page}:] 页码(page number),第一页(first page)为 1。
% \item[\xoption{view}:] 浏览规范(view specification),示例如下:\\
%   |view={FitB}|, |view={FitH 842}|, |view={XYZ 0 100 null}|\ \  一些浏览规范参数(view specification parameters)
%   将数字(numbers)视为具有单位 bp 的参数。它们可以作为普通数字(plain numbers)或在 \cs{calc}\ 内部以
%   长度表达式(length expressions)给出。如果加载了 \xpackage{calc}\ 宏包,则支持该宏包的表达式(expressions)。否则,
%   使用 \hologo{eTeX}\ 的 \cs{dimexpr}。例如:\\
%   |view={FitH \calc{\paperheight-\topmargin-1in}}|\\
%   |view={XYZ 0 \calc{\paperheight} null}|\\
%   注意 \cs{calc}\ 不能用于 |XYZ| 的第三个参数,因为该参数是缩放值(zoom value),而不是长度(length)。

% \item[\xoption{named}:] 已命名的动作(Named action)的名称:\\
%   |FirstPage|(第一页),|LastPage|(最后一页),|NextPage|(下一页),|PrevPage|(前一页)
% \item[\xoption{gotor}:] 外部(external) PDF 文件的名称。
% \item[\xoption{uri}:] URI 规范(URI specification)。
% \item[\xoption{rawaction}:] 原始动作规范(raw action specification)。由于这些规范取决于驱动程序(driver),因此不应使用此选项。
% \end{description}
% 通过分析指定的选项来选择书签的适当动作。动作由不同的选项集(sets of options)区分:
% \begin{quote}
 \begin{tabular}{|@{}r|l@{}|}
%   \hline
%   \ \textbf{动作(Action)}\  & \ \textbf{选项(Options)}\ \\ \hline
%   \ \textsf{GoTo}\  &\  \xoption{dest}\ \\ \hline
%   \ \textsf{GoTo}\  & \ \xoption{page} + \xoption{view}\ \\ \hline
%   \ \textsf{GoToR}\  & \ \xoption{gotor} + \xoption{dest}\ \\ \hline
%   \ \textsf{GoToR}\  & \ \xoption{gotor} + \xoption{page} + \xoption{view}\ \ \ \\ \hline
%   \ \textsf{Named}\  &\  \xoption{named}\ \\ \hline
%   \ \textsf{URI}\  & \ \xoption{uri}\ \\ \hline
% \end{tabular}
% \end{quote}
%
% \paragraph{缺少动作(Missing actions)。}
% 如果动作缺少 \xpackage{bookmark}\ 宏包,则抛出错误消息(error message)。根据驱动程序(driver)
% (\xoption{pdftex}、\xoption{dvips}\ 和好友[friends]),宏包在文档末尾很晚才检测到它。
% 自 2011/04/21 v1.21 版本以后,该宏包尝试打印 \cs{bookmark}\ 的相应出现的行号(line number)和文件名(file name)。
% 然而,\hologo{TeX}\ 确实提供了行号,但不幸的是,文件名是一个秘密(secret)。但该宏包有如下获取文件名的方法:
% \begin{itemize}
% \item 如果 \hologo{LuaTeX} (独立于 DVI 或 PDF 模式)正在运行,则自动使用其 |status.filename|。
% \item 宏包的 \cs{currfile} \cite{currfile}\ 重新定义了 \hologo{LaTeX}\ 的内部结构,以跟踪文件名(file name)。
% 如果加载了该宏包,那么它的 \cs{currfilepath}\ 将被检测到并由 \xpackage{bookmark}\ 自动使用。
% \item 可以通过 \cs{bookmarksetup}\ 或 \cs{bookmark}\ 中的 \xoption{scrfile}\ 选项手动设置(set manually)文件名。
% 但是要小心,手动设置会禁用以前的文件名检测方法。错误的(wrong)或丢失的(missed)文件名设置(file name setting)可能会在错误消息中
% 为您提供错误的源位置(source location)。
% \end{itemize}
%
% \subsubsection{级别选项(Level options)}
%
% 书签条目(bookmark entries)的顺序由 \cs{bookmark}\ 命令的的出现顺序(appearance order)定义。
% 树结构(tree structure)由书签节点(bookmark nodes)的属性 \xoption{level}(级别)构建。
% \xoption{level}\ 的值是整数(integers)。如果书签条目级别的值高于前一个节点,则该条目将成为
% 前一个节点的子(child)节点。差值的绝对值并不重要。
%
% \xpackage{bookmark}\ 宏包能记住全局属性(global property)“current level(当前级别)”中上
% 一个书签条目(previous bookmark entry)的级别。
%
% 级别系统的(level system)行为(behaviour)可以通过以下选项进行配置:
% \begin{description}
% \item[\xoption{level}:]
%    设置级别(level),请参阅上面的说明。如果给出的选项 \xoption{level}\ 没有值,那么将恢复默
%    认行为,即将“当前级别(current level)”用作级别值(level value)。自 2010/10/19 v1.16 版本以来,
%    如果宏 \cs{toclevel@part}、\cs{toclevel@section}\ 被定义过(通过 \xpackage{hyperref}\ 宏包完成,
%    请参阅它的 \xoption{bookmarkdepth}\ 选项),则 \xpackage{bookmark}\ 宏包还支持 |part|、|section| 等名称。
%
% \item[\xoption{rellevel}:]
%    设置相对于前一级别的(previous level)级别。正值表示书签条目成为前一个书签条目的子条目。
% \item[\xoption{keeplevel}:]
%    使用由\xoption{level}\ 或 \xoption{rellevel}\ 设置的级别,但不要更改全局属性“current level(当前级别)”。
%    可以通过设置为 |false| 来禁用该选项。
% \item[\xoption{startatroot}:]
%    此时,书签树(bookmark tree)再次从顶层(top level)开始。下一个书签条目不会作为上一个条目的子条目进行排序。
%    示例场景:文档使用 part。但是,最后几章(last chapters)不应放在最后一部分(last part)下面:
%    \begin{quote}
%\begin{verbatim}
%\documentclass{book}
%[...]
%\begin{document}
%  \part{第一部分}
%    \chapter{第一部分的第1章}
%    [...]
%  \part{第二部分(Second part)}
%    \chapter{第二部分的第1章}
%    [...]
%  \bookmarksetup{startatroot}
%  \chapter{Index}% 不属于第二部分
%\end{document}
%\end{verbatim}
%    \end{quote}
% \end{description}
%
% \subsubsection{样式定义(Style definitions)}
%
% 样式(style)是一组选项设置(option settings)。它可以由宏 \cs{bookmarkdefinestyle}\ 定义,
% 并由它的 \xoption{style}\ 选项使用。
% \begin{declcs}{bookmarkdefinestyle} \M{name} \M{key value list}
% \end{declcs}
% 选项设置(option settings)的 \meta{key value list}(键值列表)被指定为样式名(style \meta{name})。
%
% \begin{description}
% \item[\xoption{style}:]
%   \xoption{style}\ 选项的值是以前定义的样式的名称(name)。现在执行其选项设置(option settings)。
%   选项可以包括 \xoption{style}\ 选项。通过递归调用相同样式的无限递归(endless recursion)被阻止并抛出一个错误。
% \end{description}
%
% \subsubsection{钩子支持(Hook support)}
%
% 处理宏\cs{bookmark}\ 的可选选项(optional options)后,就会调用钩子(hook)。
% \begin{description}
% \item[\xoption{addtohook}:]
%   代码(code)作为该选项的值添加到钩子中。
% \end{description}
%
% \begin{declcs}{bookmarkget} \M{option}
% \end{declcs}
% \cs{bookmarkget}\ 宏提取 \meta{option}\ 选项的最新选项设置(latest option setting)的值。
% 对于布尔选项(boolean option),如果启用布尔选项,则返回 1,否则结果为零。结果数字(resulting numbers)
% 可以直接用于 \cs{ifnum}\ 或 \cs{ifcase}。如果您想要数字 \texttt{0}\ 和 \texttt{1},
% 请在 \cs{bookmarkget}\ 前面加上 \cs{number}\ 作为前缀。\cs{bookmarkget}\ 宏是可展开的(expandable)。
% 如果选项不受支持,则返回空字符串(empty string)。受支持的布尔选项有:
% \begin{quote}
%   \xoption{bold}、
%   \xoption{italic}、
%   \xoption{open}
% \end{quote}
% 其他受支持的选项有:
% \begin{quote}
%   \xoption{depth}、
%   \xoption{dest}、
%   \xoption{color}、
%   \xoption{gotor}、
%   \xoption{level}、
%   \xoption{named}、
%   \xoption{openlevel}、
%   \xoption{page}、
%   \xoption{rawaction}、
%   \xoption{uri}、
%   \xoption{view}、
% \end{quote}
% 另外,以下键(key)是可用的:
% \begin{quote}
%   \xoption{text}
% \end{quote}
% 它返回大纲条目(outline entry)的文本(text)。
%
% \paragraph{选项设置(Option setting)。}
% 在钩子(hook)内部可以使用 \cs{bookmarksetup}\ 设置选项。
%
% \subsection{与 \xpackage{hyperref}\ 的兼容性}
%
% \xpackage{bookmark}\ 宏包自动禁用 \xpackage{hyperref}\ 宏包的书签(bookmarks)。但是,
% \xpackage{bookmark}\ 宏包使用了 \xpackage{hyperref}\ 宏包的一些代码。例如,
% \xpackage{bookmark}\ 宏包重新定义了 \xpackage{hyperref}\ 宏包在 \cs{addcontentsline}\ 命令
% 和其他命令中插入的\cs{Hy@writebookmark}\ 钩子。因此,不应禁用 \xpackage{hyperref}\ 宏包的书签。
%
% \xpackage{bookmark}\ 宏包使用 \xpackage{hyperref}\ 宏包的 \cs{pdfstringdef},且不提供替换(replacement)。
%
% \xpackage{hyperref}\ 宏包的一些选项也能在 \xpackage{bookmark}\ 宏包中实现(implemented):
% \begin{quote}
% \begin{tabular}{|l@{}|l@{}|}
%   \hline
%   \xpackage{hyperref}\ 宏包的选项\  &\ \xpackage{bookmark}\ 宏包的选项\ \ \\ \hline
%   \xoption{bookmarksdepth} &\ \xoption{depth}\\ \hline
%   \xoption{bookmarksopen} & \ \xoption{open}\\ \hline
%   \xoption{bookmarksopenlevel}\ \ \  &\ \xoption{openlevel}\\ \hline
%   \xoption{bookmarksnumbered} \ \ \ &\ \xoption{numbered}\\ \hline
% \end{tabular}
% \end{quote}
%
% 还可以使用以下命令:
% \begin{quote}
%   \cs{pdfbookmark}\\
%   \cs{currentpdfbookmark}\\
%   \cs{subpdfbookmark}\\
%   \cs{belowpdfbookmark}
% \end{quote}
%
% \subsection{在末尾添加书签}
%
% 宏包选项 \xoption{atend}\ 启用以下宏(macro):
% \begin{declcs}{BookmarkAtEnd}
%   \M{stuff}
% \end{declcs}
% \cs{BookmarkAtEnd}\ 宏将 \meta{stuff}\ 放在文档末尾。\meta{stuff}\ 表示书签命令(bookmark commands)。举例:
% \begin{quote}
%\begin{verbatim}
%\usepackage[atend]{bookmark}
%\BookmarkAtEnd{%
%  \bookmarksetup{startatroot}%
%  \bookmark[named=LastPage, level=0]{Last page}%
%}
%\end{verbatim}
% \end{quote}
%
% 或者,可以在 \cs{bookmark}\ 中给出 \xoption{startatroot}\ 选项:
% \begin{quote}
%\begin{verbatim}
%\BookmarkAtEnd{%
%  \bookmark[
%    startatroot,
%    named=LastPage,
%    level=0,
%  ]{Last page}%
%}
%\end{verbatim}
% \end{quote}
%
% \paragraph{备注(Remarks):}
% \begin{itemize}
% \item
%   \cs{BookmarkAtEnd} 隐藏了这样一个事实,即在文档末尾添加书签的方法取决于驱动程序(driver)。
%
%   为此,驱动程序 \xoption{pdftex}\ 使用 \xpackage{atveryend}\ 宏包。如果 \cs{AtEndDocument}\ 太早,
%   最后一个页面(last page)可能不会被发送出去(shipped out)。由于需要 \xext{aux}\ 文件,此驱动程序使
%   用 \cs{AfterLastShipout}。
%
%   其他驱动程序(\xoption{dvipdfm}、\xoption{xetex}、\xoption{vtex})的实现(implementation)
%   取决于 \cs{special},\cs{special}\ 在最后一页之后没有效果。在这种情况下,\xpackage{atenddvi}\ 宏包的
%   \cs{AtEndDvi}\ 有帮助。它将其参数(argument)放在文档的最后一页(last page)。至少需要运行 \hologo{LaTeX}\ 两次,
%   因为最后一页是由引用(reference)检测到的。
%
%   \xoption{dvips}\ 现在使用新的 LaTeX 钩子 \texttt{shipout/lastpage}。
% \item
%   未指定 \cs{BookmarkAtEnd}\ 参数的扩展时间(time of expansion)。这可以立即发生,也可以在文档末尾发生。
% \end{itemize}
%
% \subsection{限制/行动计划}
%
% \begin{itemize}
% \item 支持缺失动作(missing actions)(启动,\dots)。
% \item 对 \xpackage{hyperref}\ 的 \xoption{bookmarkstype}\ 选项进行了更好的设计(design)。
% \end{itemize}
%
% \section{示例(Example)}
%
%    \begin{macrocode}
%<*example>
%    \end{macrocode}
%    \begin{macrocode}
\documentclass{article}
\usepackage{xcolor}[2007/01/21]
\usepackage{hyperref}
\usepackage[
  open,
  openlevel=2,
  atend
]{bookmark}[2019/12/03]

\bookmarksetup{color=blue}

\BookmarkAtEnd{%
  \bookmarksetup{startatroot}%
  \bookmark[named=LastPage, level=0]{End/Last page}%
  \bookmark[named=FirstPage, level=1]{First page}%
}

\begin{document}
\section{First section}
\subsection{Subsection A}
\begin{figure}
  \hypertarget{fig}{}%
  A figure.
\end{figure}
\bookmark[
  rellevel=1,
  keeplevel,
  dest=fig
]{A figure}
\subsection{Subsection B}
\subsubsection{Subsubsection C}
\subsection{Umlauts: \"A\"O\"U\"a\"o\"u\ss}
\newpage
\bookmarksetup{
  bold,
  color=[rgb]{1,0,0}
}
\section{Very important section}
\bookmarksetup{
  italic,
  bold=false,
  color=blue
}
\subsection{Italic section}
\bookmarksetup{
  italic=false
}
\part{Misc}
\section{Diverse}
\subsubsection{Subsubsection, omitting subsection}
\bookmarksetup{
  startatroot
}
\section{Last section outside part}
\subsection{Subsection}
\bookmarksetup{
  color={}
}
\begingroup
  \bookmarksetup{level=0, color=green!80!black}
  \bookmark[named=FirstPage]{First page}
  \bookmark[named=LastPage]{Last page}
  \bookmark[named=PrevPage]{Previous page}
  \bookmark[named=NextPage]{Next page}
\endgroup
\bookmark[
  page=2,
  view=FitH 800
]{Page 2, FitH 800}
\bookmark[
  page=2,
  view=FitBH \calc{\paperheight-\topmargin-1in-\headheight-\headsep}
]{Page 2, FitBH top of text body}
\bookmark[
  uri={http://www.dante.de/},
  color=magenta
]{Dante homepage}
\bookmark[
  gotor={t.pdf},
  page=1,
  view={XYZ 0 1000 null},
  color=cyan!75!black
]{File t.pdf}
\bookmark[named=FirstPage]{First page}
\bookmark[rellevel=1, named=LastPage]{Last page (rellevel=1)}
\bookmark[named=PrevPage]{Previous page}
\bookmark[level=0, named=FirstPage]{First page (level=0)}
\bookmark[
  rellevel=1,
  keeplevel,
  named=LastPage
]{Last page (rellevel=1, keeplevel)}
\bookmark[named=PrevPage]{Previous page}
\end{document}
%    \end{macrocode}
%    \begin{macrocode}
%</example>
%    \end{macrocode}
%
% \StopEventually{
% }
%
% \section{实现(Implementation)}
%
% \subsection{宏包(Package)}
%
%    \begin{macrocode}
%<*package>
\NeedsTeXFormat{LaTeX2e}
\ProvidesPackage{bookmark}%
  [2020-11-06 v1.29 PDF bookmarks (HO)]%
%    \end{macrocode}
%
% \subsubsection{要求(Requirements)}
%
% \paragraph{\hologo{eTeX}.}
%
%    \begin{macro}{\BKM@CalcExpr}
%    \begin{macrocode}
\begingroup\expandafter\expandafter\expandafter\endgroup
\expandafter\ifx\csname numexpr\endcsname\relax
  \def\BKM@CalcExpr#1#2#3#4{%
    \begingroup
      \count@=#2\relax
      \advance\count@ by#3#4\relax
      \edef\x{\endgroup
        \def\noexpand#1{\the\count@}%
      }%
    \x
  }%
\else
  \def\BKM@CalcExpr#1#2#3#4{%
    \edef#1{%
      \the\numexpr#2#3#4\relax
    }%
  }%
\fi
%    \end{macrocode}
%    \end{macro}
%
% \paragraph{\hologo{pdfTeX}\ 的转义功能(escape features)}
%
%    \begin{macro}{\BKM@EscapeName}
%    \begin{macrocode}
\def\BKM@EscapeName#1{%
  \ifx#1\@empty
  \else
    \EdefEscapeName#1#1%
  \fi
}%
%    \end{macrocode}
%    \end{macro}
%    \begin{macro}{\BKM@EscapeString}
%    \begin{macrocode}
\def\BKM@EscapeString#1{%
  \ifx#1\@empty
  \else
    \EdefEscapeString#1#1%
  \fi
}%
%    \end{macrocode}
%    \end{macro}
%    \begin{macro}{\BKM@EscapeHex}
%    \begin{macrocode}
\def\BKM@EscapeHex#1{%
  \ifx#1\@empty
  \else
    \EdefEscapeHex#1#1%
  \fi
}%
%    \end{macrocode}
%    \end{macro}
%    \begin{macro}{\BKM@UnescapeHex}
%    \begin{macrocode}
\def\BKM@UnescapeHex#1{%
  \EdefUnescapeHex#1#1%
}%
%    \end{macrocode}
%    \end{macro}
%
% \paragraph{宏包(Packages)。}
%
% 不要加载由 \xpackage{hyperref}\ 加载的宏包
%    \begin{macrocode}
\RequirePackage{hyperref}[2010/06/18]
%    \end{macrocode}
%
% \subsubsection{宏包选项(Package options)}
%
%    \begin{macrocode}
\SetupKeyvalOptions{family=BKM,prefix=BKM@}
\DeclareLocalOptions{%
  atend,%
  bold,%
  color,%
  depth,%
  dest,%
  draft,%
  final,%
  gotor,%
  italic,%
  keeplevel,%
  level,%
  named,%
  numbered,%
  open,%
  openlevel,%
  page,%
  rawaction,%
  rellevel,%
  srcfile,%
  srcline,%
  startatroot,%
  uri,%
  view,%
}
%    \end{macrocode}
%    \begin{macro}{\bookmarksetup}
%    \begin{macrocode}
\newcommand*{\bookmarksetup}{\kvsetkeys{BKM}}
%    \end{macrocode}
%    \end{macro}
%    \begin{macro}{\BKM@setup}
%    \begin{macrocode}
\def\BKM@setup#1{%
  \bookmarksetup{#1}%
  \ifx\BKM@HookNext\ltx@empty
  \else
    \expandafter\bookmarksetup\expandafter{\BKM@HookNext}%
    \BKM@HookNextClear
  \fi
  \BKM@hook
  \ifBKM@keeplevel
  \else
    \xdef\BKM@currentlevel{\BKM@level}%
  \fi
}
%    \end{macrocode}
%    \end{macro}
%
%    \begin{macro}{\bookmarksetupnext}
%    \begin{macrocode}
\newcommand*{\bookmarksetupnext}[1]{%
  \ltx@GlobalAppendToMacro\BKM@HookNext{,#1}%
}
%    \end{macrocode}
%    \end{macro}
%    \begin{macro}{\BKM@setupnext}
%    \begin{macrocode}
%    \end{macrocode}
%    \end{macro}
%    \begin{macro}{\BKM@HookNextClear}
%    \begin{macrocode}
\def\BKM@HookNextClear{%
  \global\let\BKM@HookNext\ltx@empty
}
%    \end{macrocode}
%    \end{macro}
%    \begin{macro}{\BKM@HookNext}
%    \begin{macrocode}
\BKM@HookNextClear
%    \end{macrocode}
%    \end{macro}
%
%    \begin{macrocode}
\DeclareBoolOption{draft}
\DeclareComplementaryOption{final}{draft}
%    \end{macrocode}
%    \begin{macro}{\BKM@DisableOptions}
%    \begin{macrocode}
\def\BKM@DisableOptions{%
  \DisableKeyvalOption[action=warning,package=bookmark]%
      {BKM}{draft}%
  \DisableKeyvalOption[action=warning,package=bookmark]%
      {BKM}{final}%
}
%    \end{macrocode}
%    \end{macro}
%    \begin{macrocode}
\DeclareBoolOption[\ifHy@bookmarksopen true\else false\fi]{open}
%    \end{macrocode}
%    \begin{macro}{\bookmark@open}
%    \begin{macrocode}
\def\bookmark@open{%
  \ifBKM@open\ltx@one\else\ltx@zero\fi
}
%    \end{macrocode}
%    \end{macro}
%    \begin{macrocode}
\DeclareStringOption[\maxdimen]{openlevel}
%    \end{macrocode}
%    \begin{macro}{\BKM@openlevel}
%    \begin{macrocode}
\edef\BKM@openlevel{\number\@bookmarksopenlevel}
%    \end{macrocode}
%    \end{macro}
%    \begin{macrocode}
%\DeclareStringOption[\c@tocdepth]{depth}
\ltx@IfUndefined{Hy@bookmarksdepth}{%
  \def\BKM@depth{\c@tocdepth}%
}{%
  \let\BKM@depth\Hy@bookmarksdepth
}
\define@key{BKM}{depth}[]{%
  \edef\BKM@param{#1}%
  \ifx\BKM@param\@empty
    \def\BKM@depth{\c@tocdepth}%
  \else
    \ltx@IfUndefined{toclevel@\BKM@param}{%
      \@onelevel@sanitize\BKM@param
      \edef\BKM@temp{\expandafter\@car\BKM@param\@nil}%
      \ifcase 0\expandafter\ifx\BKM@temp-1\fi
              \expandafter\ifnum\expandafter`\BKM@temp>47 %
                \expandafter\ifnum\expandafter`\BKM@temp<58 %
                  1%
                \fi
              \fi
              \relax
        \PackageWarning{bookmark}{%
          Unknown document division name (\BKM@param)\MessageBreak
          for option `depth'%
        }%
      \else
        \BKM@SetDepthOrLevel\BKM@depth\BKM@param
      \fi
    }{%
      \BKM@SetDepthOrLevel\BKM@depth{%
        \csname toclevel@\BKM@param\endcsname
      }%
    }%
  \fi
}
%    \end{macrocode}
%    \begin{macro}{\bookmark@depth}
%    \begin{macrocode}
\def\bookmark@depth{\BKM@depth}
%    \end{macrocode}
%    \end{macro}
%    \begin{macro}{\BKM@SetDepthOrLevel}
%    \begin{macrocode}
\def\BKM@SetDepthOrLevel#1#2{%
  \begingroup
    \setbox\z@=\hbox{%
      \count@=#2\relax
      \expandafter
    }%
  \expandafter\endgroup
  \expandafter\def\expandafter#1\expandafter{\the\count@}%
}
%    \end{macrocode}
%    \end{macro}
%    \begin{macrocode}
\DeclareStringOption[\BKM@currentlevel]{level}[\BKM@currentlevel]
\define@key{BKM}{level}{%
  \edef\BKM@param{#1}%
  \ifx\BKM@param\BKM@MacroCurrentLevel
    \let\BKM@level\BKM@param
  \else
    \ltx@IfUndefined{toclevel@\BKM@param}{%
      \@onelevel@sanitize\BKM@param
      \edef\BKM@temp{\expandafter\@car\BKM@param\@nil}%
      \ifcase 0\expandafter\ifx\BKM@temp-1\fi
              \expandafter\ifnum\expandafter`\BKM@temp>47 %
                \expandafter\ifnum\expandafter`\BKM@temp<58 %
                  1%
                \fi
              \fi
              \relax
        \PackageWarning{bookmark}{%
          Unknown document division name (\BKM@param)\MessageBreak
          for option `level'%
        }%
      \else
        \BKM@SetDepthOrLevel\BKM@level\BKM@param
      \fi
    }{%
      \BKM@SetDepthOrLevel\BKM@level{%
        \csname toclevel@\BKM@param\endcsname
      }%
    }%
  \fi
}
%    \end{macrocode}
%    \begin{macro}{\BKM@MacroCurrentLevel}
%    \begin{macrocode}
\def\BKM@MacroCurrentLevel{\BKM@currentlevel}
%    \end{macrocode}
%    \end{macro}
%    \begin{macrocode}
\DeclareBoolOption{keeplevel}
\DeclareBoolOption{startatroot}
%    \end{macrocode}
%    \begin{macro}{\BKM@startatrootfalse}
%    \begin{macrocode}
\def\BKM@startatrootfalse{%
  \global\let\ifBKM@startatroot\iffalse
}
%    \end{macrocode}
%    \end{macro}
%    \begin{macro}{\BKM@startatroottrue}
%    \begin{macrocode}
\def\BKM@startatroottrue{%
  \global\let\ifBKM@startatroot\iftrue
}
%    \end{macrocode}
%    \end{macro}
%    \begin{macrocode}
\define@key{BKM}{rellevel}{%
  \BKM@CalcExpr\BKM@level{#1}+\BKM@currentlevel
}
%    \end{macrocode}
%    \begin{macro}{\bookmark@level}
%    \begin{macrocode}
\def\bookmark@level{\BKM@level}
%    \end{macrocode}
%    \end{macro}
%    \begin{macro}{\BKM@currentlevel}
%    \begin{macrocode}
\def\BKM@currentlevel{0}
%    \end{macrocode}
%    \end{macro}
%    Make \xpackage{bookmark}'s option \xoption{numbered} an alias
%    for \xpackage{hyperref}'s \xoption{bookmarksnumbered}.
%    \begin{macrocode}
\DeclareBoolOption[%
  \ifHy@bookmarksnumbered true\else false\fi
]{numbered}
\g@addto@macro\BKM@numberedtrue{%
  \let\ifHy@bookmarksnumbered\iftrue
}
\g@addto@macro\BKM@numberedfalse{%
  \let\ifHy@bookmarksnumbered\iffalse
}
\g@addto@macro\Hy@bookmarksnumberedtrue{%
  \let\ifBKM@numbered\iftrue
}
\g@addto@macro\Hy@bookmarksnumberedfalse{%
  \let\ifBKM@numbered\iffalse
}
%    \end{macrocode}
%    \begin{macro}{\bookmark@numbered}
%    \begin{macrocode}
\def\bookmark@numbered{%
  \ifBKM@numbered\ltx@one\else\ltx@zero\fi
}
%    \end{macrocode}
%    \end{macro}
%
% \paragraph{重定义 \xpackage{hyperref}\ 宏包的选项}
%
%    \begin{macro}{\BKM@PatchHyperrefOption}
%    \begin{macrocode}
\def\BKM@PatchHyperrefOption#1{%
  \expandafter\BKM@@PatchHyperrefOption\csname KV@Hyp@#1\endcsname%
}
%    \end{macrocode}
%    \end{macro}
%    \begin{macro}{\BKM@@PatchHyperrefOption}
%    \begin{macrocode}
\def\BKM@@PatchHyperrefOption#1{%
  \expandafter\BKM@@@PatchHyperrefOption#1{##1}\BKM@nil#1%
}
%    \end{macrocode}
%    \end{macro}
%    \begin{macro}{\BKM@@@PatchHyperrefOption}
%    \begin{macrocode}
\def\BKM@@@PatchHyperrefOption#1\BKM@nil#2#3{%
  \def#2##1{%
    #1%
    \bookmarksetup{#3={##1}}%
  }%
}
%    \end{macrocode}
%    \end{macro}
%    \begin{macrocode}
\BKM@PatchHyperrefOption{bookmarksopen}{open}
\BKM@PatchHyperrefOption{bookmarksopenlevel}{openlevel}
\BKM@PatchHyperrefOption{bookmarksdepth}{depth}
%    \end{macrocode}
%
% \paragraph{字体样式(font style)选项。}
%
%    注意:\xpackage{bitset}\ 宏是基于零的,PDF 规范(PDF specifications)以1开头。
%    \begin{macrocode}
\bitsetReset{BKM@FontStyle}%
\define@key{BKM}{italic}[true]{%
  \expandafter\ifx\csname if#1\endcsname\iftrue
    \bitsetSet{BKM@FontStyle}{0}%
  \else
    \bitsetClear{BKM@FontStyle}{0}%
  \fi
}%
\define@key{BKM}{bold}[true]{%
  \expandafter\ifx\csname if#1\endcsname\iftrue
    \bitsetSet{BKM@FontStyle}{1}%
  \else
    \bitsetClear{BKM@FontStyle}{1}%
  \fi
}%
%    \end{macrocode}
%    \begin{macro}{\bookmark@italic}
%    \begin{macrocode}
\def\bookmark@italic{%
  \ifnum\bitsetGet{BKM@FontStyle}{0}=1 \ltx@one\else\ltx@zero\fi
}
%    \end{macrocode}
%    \end{macro}
%    \begin{macro}{\bookmark@bold}
%    \begin{macrocode}
\def\bookmark@bold{%
  \ifnum\bitsetGet{BKM@FontStyle}{1}=1 \ltx@one\else\ltx@zero\fi
}
%    \end{macrocode}
%    \end{macro}
%    \begin{macro}{\BKM@PrintStyle}
%    \begin{macrocode}
\def\BKM@PrintStyle{%
  \bitsetGetDec{BKM@FontStyle}%
}%
%    \end{macrocode}
%    \end{macro}
%
% \paragraph{颜色(color)选项。}
%
%    \begin{macrocode}
\define@key{BKM}{color}{%
  \HyColor@BookmarkColor{#1}\BKM@color{bookmark}{color}%
}
%    \end{macrocode}
%    \begin{macro}{\BKM@color}
%    \begin{macrocode}
\let\BKM@color\@empty
%    \end{macrocode}
%    \end{macro}
%    \begin{macro}{\bookmark@color}
%    \begin{macrocode}
\def\bookmark@color{\BKM@color}
%    \end{macrocode}
%    \end{macro}
%
% \subsubsection{动作(action)选项}
%
%    \begin{macrocode}
\def\BKM@temp#1{%
  \DeclareStringOption{#1}%
  \expandafter\edef\csname bookmark@#1\endcsname{%
    \expandafter\noexpand\csname BKM@#1\endcsname
  }%
}
%    \end{macrocode}
%    \begin{macro}{\bookmark@dest}
%    \begin{macrocode}
\BKM@temp{dest}
%    \end{macrocode}
%    \end{macro}
%    \begin{macro}{\bookmark@named}
%    \begin{macrocode}
\BKM@temp{named}
%    \end{macrocode}
%    \end{macro}
%    \begin{macro}{\bookmark@uri}
%    \begin{macrocode}
\BKM@temp{uri}
%    \end{macrocode}
%    \end{macro}
%    \begin{macro}{\bookmark@gotor}
%    \begin{macrocode}
\BKM@temp{gotor}
%    \end{macrocode}
%    \end{macro}
%    \begin{macro}{\bookmark@rawaction}
%    \begin{macrocode}
\BKM@temp{rawaction}
%    \end{macrocode}
%    \end{macro}
%
%    \begin{macrocode}
\define@key{BKM}{page}{%
  \def\BKM@page{#1}%
  \ifx\BKM@page\@empty
  \else
    \edef\BKM@page{\number\BKM@page}%
    \ifnum\BKM@page>\z@
    \else
      \PackageError{bookmark}{Page must be positive}\@ehc
      \def\BKM@page{1}%
    \fi
  \fi
}
%    \end{macrocode}
%    \begin{macro}{\BKM@page}
%    \begin{macrocode}
\let\BKM@page\@empty
%    \end{macrocode}
%    \end{macro}
%    \begin{macro}{\bookmark@page}
%    \begin{macrocode}
\def\bookmark@page{\BKM@@page}
%    \end{macrocode}
%    \end{macro}
%
%    \begin{macrocode}
\define@key{BKM}{view}{%
  \BKM@CheckView{#1}%
}
%    \end{macrocode}
%    \begin{macro}{\BKM@view}
%    \begin{macrocode}
\let\BKM@view\@empty
%    \end{macrocode}
%    \end{macro}
%    \begin{macro}{\bookmark@view}
%    \begin{macrocode}
\def\bookmark@view{\BKM@view}
%    \end{macrocode}
%    \end{macro}
%    \begin{macro}{BKM@CheckView}
%    \begin{macrocode}
\def\BKM@CheckView#1{%
  \BKM@CheckViewType#1 \@nil
}
%    \end{macrocode}
%    \end{macro}
%    \begin{macro}{\BKM@CheckViewType}
%    \begin{macrocode}
\def\BKM@CheckViewType#1 #2\@nil{%
  \def\BKM@type{#1}%
  \@onelevel@sanitize\BKM@type
  \BKM@TestViewType{Fit}{}%
  \BKM@TestViewType{FitB}{}%
  \BKM@TestViewType{FitH}{%
    \BKM@CheckParam#2 \@nil{top}%
  }%
  \BKM@TestViewType{FitBH}{%
    \BKM@CheckParam#2 \@nil{top}%
  }%
  \BKM@TestViewType{FitV}{%
    \BKM@CheckParam#2 \@nil{bottom}%
  }%
  \BKM@TestViewType{FitBV}{%
    \BKM@CheckParam#2 \@nil{bottom}%
  }%
  \BKM@TestViewType{FitR}{%
    \BKM@CheckRect{#2}{ }%
  }%
  \BKM@TestViewType{XYZ}{%
    \BKM@CheckXYZ{#2}{ }%
  }%
  \@car{%
    \PackageError{bookmark}{%
      Unknown view type `\BKM@type',\MessageBreak
      using `FitH' instead%
    }\@ehc
    \def\BKM@view{FitH}%
  }%
  \@nil
}
%    \end{macrocode}
%    \end{macro}
%    \begin{macro}{\BKM@TestViewType}
%    \begin{macrocode}
\def\BKM@TestViewType#1{%
  \def\BKM@temp{#1}%
  \@onelevel@sanitize\BKM@temp
  \ifx\BKM@type\BKM@temp
    \let\BKM@view\BKM@temp
    \expandafter\@car
  \else
    \expandafter\@gobble
  \fi
}
%    \end{macrocode}
%    \end{macro}
%    \begin{macro}{BKM@CheckParam}
%    \begin{macrocode}
\def\BKM@CheckParam#1 #2\@nil#3{%
  \def\BKM@param{#1}%
  \ifx\BKM@param\@empty
    \PackageWarning{bookmark}{%
      Missing parameter (#3) for `\BKM@type',\MessageBreak
      using 0%
    }%
    \def\BKM@param{0}%
  \else
    \BKM@CalcParam
  \fi
  \edef\BKM@view{\BKM@view\space\BKM@param}%
}
%    \end{macrocode}
%    \end{macro}
%    \begin{macro}{BKM@CheckRect}
%    \begin{macrocode}
\def\BKM@CheckRect#1#2{%
  \BKM@@CheckRect#1#2#2#2#2\@nil
}
%    \end{macrocode}
%    \end{macro}
%    \begin{macro}{\BKM@@CheckRect}
%    \begin{macrocode}
\def\BKM@@CheckRect#1 #2 #3 #4 #5\@nil{%
  \def\BKM@temp{0}%
  \def\BKM@param{#1}%
  \ifx\BKM@param\@empty
    \def\BKM@param{0}%
    \def\BKM@temp{1}%
  \else
    \BKM@CalcParam
  \fi
  \edef\BKM@view{\BKM@view\space\BKM@param}%
  \def\BKM@param{#2}%
  \ifx\BKM@param\@empty
    \def\BKM@param{0}%
    \def\BKM@temp{1}%
  \else
    \BKM@CalcParam
  \fi
  \edef\BKM@view{\BKM@view\space\BKM@param}%
  \def\BKM@param{#3}%
  \ifx\BKM@param\@empty
    \def\BKM@param{0}%
    \def\BKM@temp{1}%
  \else
    \BKM@CalcParam
  \fi
  \edef\BKM@view{\BKM@view\space\BKM@param}%
  \def\BKM@param{#4}%
  \ifx\BKM@param\@empty
    \def\BKM@param{0}%
    \def\BKM@temp{1}%
  \else
    \BKM@CalcParam
  \fi
  \edef\BKM@view{\BKM@view\space\BKM@param}%
  \ifnum\BKM@temp>\z@
    \PackageWarning{bookmark}{Missing parameters for `\BKM@type'}%
  \fi
}
%    \end{macrocode}
%    \end{macro}
%    \begin{macro}{\BKM@CheckXYZ}
%    \begin{macrocode}
\def\BKM@CheckXYZ#1#2{%
  \BKM@@CheckXYZ#1#2#2#2\@nil
}
%    \end{macrocode}
%    \end{macro}
%    \begin{macro}{\BKM@@CheckXYZ}
%    \begin{macrocode}
\def\BKM@@CheckXYZ#1 #2 #3 #4\@nil{%
  \def\BKM@param{#1}%
  \let\BKM@temp\BKM@param
  \@onelevel@sanitize\BKM@temp
  \ifx\BKM@param\@empty
    \let\BKM@param\BKM@null
  \else
    \ifx\BKM@temp\BKM@null
    \else
      \BKM@CalcParam
    \fi
  \fi
  \edef\BKM@view{\BKM@view\space\BKM@param}%
  \def\BKM@param{#2}%
  \let\BKM@temp\BKM@param
  \@onelevel@sanitize\BKM@temp
  \ifx\BKM@param\@empty
    \let\BKM@param\BKM@null
  \else
    \ifx\BKM@temp\BKM@null
    \else
      \BKM@CalcParam
    \fi
  \fi
  \edef\BKM@view{\BKM@view\space\BKM@param}%
  \def\BKM@param{#3}%
  \ifx\BKM@param\@empty
    \let\BKM@param\BKM@null
  \fi
  \edef\BKM@view{\BKM@view\space\BKM@param}%
}
%    \end{macrocode}
%    \end{macro}
%    \begin{macro}{\BKM@null}
%    \begin{macrocode}
\def\BKM@null{null}
\@onelevel@sanitize\BKM@null
%    \end{macrocode}
%    \end{macro}
%
%    \begin{macro}{\BKM@CalcParam}
%    \begin{macrocode}
\def\BKM@CalcParam{%
  \begingroup
  \let\calc\@firstofone
  \expandafter\BKM@@CalcParam\BKM@param\@empty\@empty\@nil
}
%    \end{macrocode}
%    \end{macro}
%    \begin{macro}{\BKM@@CalcParam}
%    \begin{macrocode}
\def\BKM@@CalcParam#1#2#3\@nil{%
  \ifx\calc#1%
    \@ifundefined{calc@assign@dimen}{%
      \@ifundefined{dimexpr}{%
        \setlength{\dimen@}{#2}%
      }{%
        \setlength{\dimen@}{\dimexpr#2\relax}%
      }%
    }{%
      \setlength{\dimen@}{#2}%
    }%
    \dimen@.99626\dimen@
    \edef\BKM@param{\strip@pt\dimen@}%
    \expandafter\endgroup
    \expandafter\def\expandafter\BKM@param\expandafter{\BKM@param}%
  \else
    \endgroup
  \fi
}
%    \end{macrocode}
%    \end{macro}
%
% \subsubsection{\xoption{atend}\ 选项}
%
%    \begin{macrocode}
\DeclareBoolOption{atend}
\g@addto@macro\BKM@DisableOptions{%
  \DisableKeyvalOption[action=warning,package=bookmark]%
      {BKM}{atend}%
}
%    \end{macrocode}
%
% \subsubsection{\xoption{style}\ 选项}
%
%    \begin{macro}{\bookmarkdefinestyle}
%    \begin{macrocode}
\newcommand*{\bookmarkdefinestyle}[2]{%
  \@ifundefined{BKM@style@#1}{%
  }{%
    \PackageInfo{bookmark}{Redefining style `#1'}%
  }%
  \@namedef{BKM@style@#1}{#2}%
}
%    \end{macrocode}
%    \end{macro}
%    \begin{macrocode}
\define@key{BKM}{style}{%
  \BKM@StyleCall{#1}%
}
\newif\ifBKM@ok
%    \end{macrocode}
%    \begin{macro}{\BKM@StyleCall}
%    \begin{macrocode}
\def\BKM@StyleCall#1{%
  \@ifundefined{BKM@style@#1}{%
    \PackageWarning{bookmark}{%
      Ignoring unknown style `#1'%
    }%
  }{%
%    \end{macrocode}
%    检查样式堆栈(style stack)。
%    \begin{macrocode}
    \BKM@oktrue
    \edef\BKM@StyleCurrent{#1}%
    \@onelevel@sanitize\BKM@StyleCurrent
    \let\BKM@StyleEntry\BKM@StyleEntryCheck
    \BKM@StyleStack
    \ifBKM@ok
      \expandafter\@firstofone
    \else
      \PackageError{bookmark}{%
        Ignoring recursive call of style `\BKM@StyleCurrent'%
      }\@ehc
      \expandafter\@gobble
    \fi
    {%
%    \end{macrocode}
%    在堆栈上推送当前样式(Push current style on stack)。
%    \begin{macrocode}
      \let\BKM@StyleEntry\relax
      \edef\BKM@StyleStack{%
        \BKM@StyleEntry{\BKM@StyleCurrent}%
        \BKM@StyleStack
      }%
%    \end{macrocode}
%   调用样式(Call style)。
%    \begin{macrocode}
      \expandafter\expandafter\expandafter\bookmarksetup
      \expandafter\expandafter\expandafter{%
        \csname BKM@style@\BKM@StyleCurrent\endcsname
      }%
%    \end{macrocode}
%    从堆栈中弹出当前样式(Pop current style from stack)。
%    \begin{macrocode}
      \BKM@StyleStackPop
    }%
  }%
}
%    \end{macrocode}
%    \end{macro}
%    \begin{macro}{\BKM@StyleStackPop}
%    \begin{macrocode}
\def\BKM@StyleStackPop{%
  \let\BKM@StyleEntry\relax
  \edef\BKM@StyleStack{%
    \expandafter\@gobbletwo\BKM@StyleStack
  }%
}
%    \end{macrocode}
%    \end{macro}
%    \begin{macro}{\BKM@StyleEntryCheck}
%    \begin{macrocode}
\def\BKM@StyleEntryCheck#1{%
  \def\BKM@temp{#1}%
  \ifx\BKM@temp\BKM@StyleCurrent
    \BKM@okfalse
  \fi
}
%    \end{macrocode}
%    \end{macro}
%    \begin{macro}{\BKM@StyleStack}
%    \begin{macrocode}
\def\BKM@StyleStack{}
%    \end{macrocode}
%    \end{macro}
%
% \subsubsection{源文件位置(source file location)选项}
%
%    \begin{macrocode}
\DeclareStringOption{srcline}
\DeclareStringOption{srcfile}
%    \end{macrocode}
%
% \subsubsection{钩子支持(Hook support)}
%
%    \begin{macro}{\BKM@hook}
%    \begin{macrocode}
\def\BKM@hook{}
%    \end{macrocode}
%    \end{macro}
%    \begin{macrocode}
\define@key{BKM}{addtohook}{%
  \ltx@LocalAppendToMacro\BKM@hook{#1}%
}
%    \end{macrocode}
%
%    \begin{macro}{bookmarkget}
%    \begin{macrocode}
\newcommand*{\bookmarkget}[1]{%
  \romannumeral0%
  \ltx@ifundefined{bookmark@#1}{%
    \ltx@space
  }{%
    \expandafter\expandafter\expandafter\ltx@space
    \csname bookmark@#1\endcsname
  }%
}
%    \end{macrocode}
%    \end{macro}
%
% \subsubsection{设置和加载驱动程序}
%
% \paragraph{检测驱动程序。}
%
%    \begin{macro}{\BKM@DefineDriverKey}
%    \begin{macrocode}
\def\BKM@DefineDriverKey#1{%
  \define@key{BKM}{#1}[]{%
    \def\BKM@driver{#1}%
  }%
  \g@addto@macro\BKM@DisableOptions{%
    \DisableKeyvalOption[action=warning,package=bookmark]%
        {BKM}{#1}%
  }%
}
%    \end{macrocode}
%    \end{macro}
%    \begin{macrocode}
\BKM@DefineDriverKey{pdftex}
\BKM@DefineDriverKey{dvips}
\BKM@DefineDriverKey{dvipdfm}
\BKM@DefineDriverKey{dvipdfmx}
\BKM@DefineDriverKey{xetex}
\BKM@DefineDriverKey{vtex}
\define@key{BKM}{dvipdfmx-outline-open}[true]{%
 \PackageWarning{bookmark}{Option 'dvipdfmx-outline-open' is obsolete
   and ignored}{}}
%    \end{macrocode}
%    \begin{macro}{\bookmark@driver}
%    \begin{macrocode}
\def\bookmark@driver{\BKM@driver}
%    \end{macrocode}
%    \end{macro}
%    \begin{macrocode}
\InputIfFileExists{bookmark.cfg}{}{}
%    \end{macrocode}
%    \begin{macro}{\BookmarkDriverDefault}
%    \begin{macrocode}
\providecommand*{\BookmarkDriverDefault}{dvips}
%    \end{macrocode}
%    \end{macro}
%    \begin{macro}{\BKM@driver}
% Lua\TeX\ 和 pdf\TeX\ 共享驱动程序。
%    \begin{macrocode}
\ifpdf
  \def\BKM@driver{pdftex}%
  \ifx\pdfoutline\@undefined
    \ifx\pdfextension\@undefined\else
      \protected\def\pdfoutline{\pdfextension outline }
    \fi
  \fi
\else
  \ifxetex
    \def\BKM@driver{dvipdfm}%
  \else
    \ifvtex
      \def\BKM@driver{vtex}%
    \else
      \edef\BKM@driver{\BookmarkDriverDefault}%
    \fi
  \fi
\fi
%    \end{macrocode}
%    \end{macro}
%
% \paragraph{过程选项(Process options)。}
%
%    \begin{macrocode}
\ProcessKeyvalOptions*
\BKM@DisableOptions
%    \end{macrocode}
%
% \paragraph{\xoption{draft}\ 选项}
%
%    \begin{macrocode}
\ifBKM@draft
  \PackageWarningNoLine{bookmark}{Draft mode on}%
  \let\bookmarksetup\ltx@gobble
  \let\BookmarkAtEnd\ltx@gobble
  \let\bookmarkdefinestyle\ltx@gobbletwo
  \let\bookmarkget\ltx@gobble
  \let\pdfbookmark\ltx@undefined
  \newcommand*{\pdfbookmark}[3][]{}%
  \let\currentpdfbookmark\ltx@gobbletwo
  \let\subpdfbookmark\ltx@gobbletwo
  \let\belowpdfbookmark\ltx@gobbletwo
  \newcommand*{\bookmark}[2][]{}%
  \renewcommand*{\Hy@writebookmark}[5]{}%
  \let\ReadBookmarks\relax
  \let\BKM@DefGotoNameAction\ltx@gobbletwo % package `hypdestopt'
  \expandafter\endinput
\fi
%    \end{macrocode}
%
% \paragraph{验证和加载驱动程序。}
%
%    \begin{macrocode}
\def\BKM@temp{dvipdfmx}%
\ifx\BKM@temp\BKM@driver
  \def\BKM@driver{dvipdfm}%
\fi
\def\BKM@temp{pdftex}%
\ifpdf
  \ifx\BKM@temp\BKM@driver
  \else
    \PackageWarningNoLine{bookmark}{%
      Wrong driver `\BKM@driver', using `pdftex' instead%
    }%
    \let\BKM@driver\BKM@temp
  \fi
\else
  \ifx\BKM@temp\BKM@driver
    \PackageError{bookmark}{%
      Wrong driver, pdfTeX is not running in PDF mode.\MessageBreak
      Package loading is aborted%
    }\@ehc
    \expandafter\expandafter\expandafter\endinput
  \fi
  \def\BKM@temp{dvipdfm}%
  \ifxetex
    \ifx\BKM@temp\BKM@driver
    \else
      \PackageWarningNoLine{bookmark}{%
        Wrong driver `\BKM@driver',\MessageBreak
        using `dvipdfm' for XeTeX instead%
      }%
      \let\BKM@driver\BKM@temp
    \fi
  \else
    \def\BKM@temp{vtex}%
    \ifvtex
      \ifx\BKM@temp\BKM@driver
      \else
        \PackageWarningNoLine{bookmark}{%
          Wrong driver `\BKM@driver',\MessageBreak
          using `vtex' for VTeX instead%
        }%
        \let\BKM@driver\BKM@temp
      \fi
    \else
      \ifx\BKM@temp\BKM@driver
        \PackageError{bookmark}{%
          Wrong driver, VTeX is not running in PDF mode.\MessageBreak
          Package loading is aborted%
        }\@ehc
        \expandafter\expandafter\expandafter\endinput
      \fi
    \fi
  \fi
\fi
\providecommand\IfFormatAtLeastTF{\@ifl@t@r\fmtversion}
\IfFormatAtLeastTF{2020/10/01}{}{\edef\BKM@driver{\BKM@driver-2019-12-03}}
\InputIfFileExists{bkm-\BKM@driver.def}{}{%
  \PackageError{bookmark}{%
    Unsupported driver `\BKM@driver'.\MessageBreak
    Package loading is aborted%
  }\@ehc
  \endinput
}
%    \end{macrocode}
%
% \subsubsection{与 \xpackage{hyperref}\ 的兼容性}
%
%    \begin{macro}{\pdfbookmark}
%    \begin{macrocode}
\let\pdfbookmark\ltx@undefined
\newcommand*{\pdfbookmark}[3][0]{%
  \bookmark[level=#1,dest={#3.#1}]{#2}%
  \hyper@anchorstart{#3.#1}\hyper@anchorend
}
%    \end{macrocode}
%    \end{macro}
%    \begin{macro}{\currentpdfbookmark}
%    \begin{macrocode}
\def\currentpdfbookmark{%
  \pdfbookmark[\BKM@currentlevel]%
}
%    \end{macrocode}
%    \end{macro}
%    \begin{macro}{\subpdfbookmark}
%    \begin{macrocode}
\def\subpdfbookmark{%
  \BKM@CalcExpr\BKM@CalcResult\BKM@currentlevel+1%
  \expandafter\pdfbookmark\expandafter[\BKM@CalcResult]%
}
%    \end{macrocode}
%    \end{macro}
%    \begin{macro}{\belowpdfbookmark}
%    \begin{macrocode}
\def\belowpdfbookmark#1#2{%
  \xdef\BKM@gtemp{\number\BKM@currentlevel}%
  \subpdfbookmark{#1}{#2}%
  \global\let\BKM@currentlevel\BKM@gtemp
}
%    \end{macrocode}
%    \end{macro}
%
%    节号(section number)、文本(text)、标签(label)、级别(level)、文件(file)
%    \begin{macro}{\Hy@writebookmark}
%    \begin{macrocode}
\def\Hy@writebookmark#1#2#3#4#5{%
  \ifnum#4>\BKM@depth\relax
  \else
    \def\BKM@type{#5}%
    \ifx\BKM@type\Hy@bookmarkstype
      \begingroup
        \ifBKM@numbered
          \let\numberline\Hy@numberline
          \let\booknumberline\Hy@numberline
          \let\partnumberline\Hy@numberline
          \let\chapternumberline\Hy@numberline
        \else
          \let\numberline\@gobble
          \let\booknumberline\@gobble
          \let\partnumberline\@gobble
          \let\chapternumberline\@gobble
        \fi
        \bookmark[level=#4,dest={\HyperDestNameFilter{#3}}]{#2}%
      \endgroup
    \fi
  \fi
}
%    \end{macrocode}
%    \end{macro}
%
%    \begin{macro}{\ReadBookmarks}
%    \begin{macrocode}
\let\ReadBookmarks\relax
%    \end{macrocode}
%    \end{macro}
%
%    \begin{macrocode}
%</package>
%    \end{macrocode}
%
% \subsection{dvipdfm 的驱动程序}
%
%    \begin{macrocode}
%<*dvipdfm>
\NeedsTeXFormat{LaTeX2e}
\ProvidesFile{bkm-dvipdfm.def}%
  [2020-11-06 v1.29 bookmark driver for dvipdfm (HO)]%
%    \end{macrocode}
%
%    \begin{macro}{\BKM@id}
%    \begin{macrocode}
\newcount\BKM@id
\BKM@id=\z@
%    \end{macrocode}
%    \end{macro}
%
%    \begin{macro}{\BKM@0}
%    \begin{macrocode}
\@namedef{BKM@0}{000}
%    \end{macrocode}
%    \end{macro}
%    \begin{macro}{\ifBKM@sw}
%    \begin{macrocode}
\newif\ifBKM@sw
%    \end{macrocode}
%    \end{macro}
%
%    \begin{macro}{\bookmark}
%    \begin{macrocode}
\newcommand*{\bookmark}[2][]{%
  \if@filesw
    \begingroup
      \def\bookmark@text{#2}%
      \BKM@setup{#1}%
      \edef\BKM@prev{\the\BKM@id}%
      \global\advance\BKM@id\@ne
      \BKM@swtrue
      \@whilesw\ifBKM@sw\fi{%
        \def\BKM@abslevel{1}%
        \ifnum\ifBKM@startatroot\z@\else\BKM@prev\fi=\z@
          \BKM@startatrootfalse
          \expandafter\xdef\csname BKM@\the\BKM@id\endcsname{%
            0{\BKM@level}\BKM@abslevel
          }%
          \BKM@swfalse
        \else
          \expandafter\expandafter\expandafter\BKM@getx
              \csname BKM@\BKM@prev\endcsname
          \ifnum\BKM@level>\BKM@x@level\relax
            \BKM@CalcExpr\BKM@abslevel\BKM@x@abslevel+1%
            \expandafter\xdef\csname BKM@\the\BKM@id\endcsname{%
              {\BKM@prev}{\BKM@level}\BKM@abslevel
            }%
            \BKM@swfalse
          \else
            \let\BKM@prev\BKM@x@parent
          \fi
        \fi
      }%
      \csname HyPsd@XeTeXBigCharstrue\endcsname
      \pdfstringdef\BKM@title{\bookmark@text}%
      \edef\BKM@FLAGS{\BKM@PrintStyle}%
      \let\BKM@action\@empty
      \ifx\BKM@gotor\@empty
        \ifx\BKM@dest\@empty
          \ifx\BKM@named\@empty
            \ifx\BKM@rawaction\@empty
              \ifx\BKM@uri\@empty
                \ifx\BKM@page\@empty
                  \PackageError{bookmark}{Missing action}\@ehc
                  \edef\BKM@action{/Dest[@page1/Fit]}%
                \else
                  \ifx\BKM@view\@empty
                    \def\BKM@view{Fit}%
                  \fi
                  \edef\BKM@action{/Dest[@page\BKM@page/\BKM@view]}%
                \fi
              \else
                \BKM@EscapeString\BKM@uri
                \edef\BKM@action{%
                  /A<<%
                    /S/URI%
                    /URI(\BKM@uri)%
                  >>%
                }%
              \fi
            \else
              \edef\BKM@action{/A<<\BKM@rawaction>>}%
            \fi
          \else
            \BKM@EscapeName\BKM@named
            \edef\BKM@action{%
              /A<</S/Named/N/\BKM@named>>%
            }%
          \fi
        \else
          \BKM@EscapeString\BKM@dest
          \edef\BKM@action{%
            /A<<%
              /S/GoTo%
              /D(\BKM@dest)%
            >>%
          }%
        \fi
      \else
        \ifx\BKM@dest\@empty
          \ifx\BKM@page\@empty
            \def\BKM@page{0}%
          \else
            \BKM@CalcExpr\BKM@page\BKM@page-1%
          \fi
          \ifx\BKM@view\@empty
            \def\BKM@view{Fit}%
          \fi
          \edef\BKM@action{/D[\BKM@page/\BKM@view]}%
        \else
          \BKM@EscapeString\BKM@dest
          \edef\BKM@action{/D(\BKM@dest)}%
        \fi
        \BKM@EscapeString\BKM@gotor
        \edef\BKM@action{%
          /A<<%
            /S/GoToR%
            /F(\BKM@gotor)%
            \BKM@action
          >>%
        }%
      \fi
      \special{pdf:%
        out
              [%
              \ifBKM@open
                \ifnum\BKM@level<%
                    \expandafter\ltx@firstofone\expandafter
                    {\number\BKM@openlevel} %
                \else
                  -%
                \fi
              \else
                -%
              \fi
              ] %
            \BKM@abslevel
        <<%
          /Title(\BKM@title)%
          \ifx\BKM@color\@empty
          \else
            /C[\BKM@color]%
          \fi
          \ifnum\BKM@FLAGS>\z@
            /F \BKM@FLAGS
          \fi
          \BKM@action
        >>%
      }%
    \endgroup
  \fi
}
%    \end{macrocode}
%    \end{macro}
%    \begin{macro}{\BKM@getx}
%    \begin{macrocode}
\def\BKM@getx#1#2#3{%
  \def\BKM@x@parent{#1}%
  \def\BKM@x@level{#2}%
  \def\BKM@x@abslevel{#3}%
}
%    \end{macrocode}
%    \end{macro}
%
%    \begin{macrocode}
%</dvipdfm>
%    \end{macrocode}
%
% \subsection{\hologo{VTeX}\ 的驱动程序}
%
%    \begin{macrocode}
%<*vtex>
\NeedsTeXFormat{LaTeX2e}
\ProvidesFile{bkm-vtex.def}%
  [2020-11-06 v1.29 bookmark driver for VTeX (HO)]%
%    \end{macrocode}
%
%    \begin{macrocode}
\ifvtexpdf
\else
  \PackageWarningNoLine{bookmark}{%
    The VTeX driver only supports PDF mode%
  }%
\fi
%    \end{macrocode}
%
%    \begin{macro}{\BKM@id}
%    \begin{macrocode}
\newcount\BKM@id
\BKM@id=\z@
%    \end{macrocode}
%    \end{macro}
%
%    \begin{macro}{\BKM@0}
%    \begin{macrocode}
\@namedef{BKM@0}{00}
%    \end{macrocode}
%    \end{macro}
%    \begin{macro}{\ifBKM@sw}
%    \begin{macrocode}
\newif\ifBKM@sw
%    \end{macrocode}
%    \end{macro}
%
%    \begin{macro}{\bookmark}
%    \begin{macrocode}
\newcommand*{\bookmark}[2][]{%
  \if@filesw
    \begingroup
      \def\bookmark@text{#2}%
      \BKM@setup{#1}%
      \edef\BKM@prev{\the\BKM@id}%
      \global\advance\BKM@id\@ne
      \BKM@swtrue
      \@whilesw\ifBKM@sw\fi{%
        \ifnum\ifBKM@startatroot\z@\else\BKM@prev\fi=\z@
          \BKM@startatrootfalse
          \def\BKM@parent{0}%
          \expandafter\xdef\csname BKM@\the\BKM@id\endcsname{%
            0{\BKM@level}%
          }%
          \BKM@swfalse
        \else
          \expandafter\expandafter\expandafter\BKM@getx
              \csname BKM@\BKM@prev\endcsname
          \ifnum\BKM@level>\BKM@x@level\relax
            \let\BKM@parent\BKM@prev
            \expandafter\xdef\csname BKM@\the\BKM@id\endcsname{%
              {\BKM@prev}{\BKM@level}%
            }%
            \BKM@swfalse
          \else
            \let\BKM@prev\BKM@x@parent
          \fi
        \fi
      }%
      \pdfstringdef\BKM@title{\bookmark@text}%
      \BKM@vtex@title
      \edef\BKM@FLAGS{\BKM@PrintStyle}%
      \let\BKM@action\@empty
      \ifx\BKM@gotor\@empty
        \ifx\BKM@dest\@empty
          \ifx\BKM@named\@empty
            \ifx\BKM@rawaction\@empty
              \ifx\BKM@uri\@empty
                \ifx\BKM@page\@empty
                  \PackageError{bookmark}{Missing action}\@ehc
                  \def\BKM@action{!1}%
                \else
                  \edef\BKM@action{!\BKM@page}%
                \fi
              \else
                \BKM@EscapeString\BKM@uri
                \edef\BKM@action{%
                  <u=%
                    /S/URI%
                    /URI(\BKM@uri)%
                  >%
                }%
              \fi
            \else
              \edef\BKM@action{<u=\BKM@rawaction>}%
            \fi
          \else
            \BKM@EscapeName\BKM@named
            \edef\BKM@action{%
              <u=%
                /S/Named%
                /N/\BKM@named
              >%
            }%
          \fi
        \else
          \BKM@EscapeString\BKM@dest
          \edef\BKM@action{\BKM@dest}%
        \fi
      \else
        \ifx\BKM@dest\@empty
          \ifx\BKM@page\@empty
            \def\BKM@page{1}%
          \fi
          \ifx\BKM@view\@empty
            \def\BKM@view{Fit}%
          \fi
          \edef\BKM@action{/D[\BKM@page/\BKM@view]}%
        \else
          \BKM@EscapeString\BKM@dest
          \edef\BKM@action{/D(\BKM@dest)}%
        \fi
        \BKM@EscapeString\BKM@gotor
        \edef\BKM@action{%
          <u=%
            /S/GoToR%
            /F(\BKM@gotor)%
            \BKM@action
          >>%
        }%
      \fi
      \ifx\BKM@color\@empty
        \let\BKM@RGBcolor\@empty
      \else
        \expandafter\BKM@toRGB\BKM@color\@nil
      \fi
      \special{%
        !outline \BKM@action;%
        p=\BKM@parent,%
        i=\number\BKM@id,%
        s=%
          \ifBKM@open
            \ifnum\BKM@level<\BKM@openlevel
              o%
            \else
              c%
            \fi
          \else
            c%
          \fi,%
        \ifx\BKM@RGBcolor\@empty
        \else
          c=\BKM@RGBcolor,%
        \fi
        \ifnum\BKM@FLAGS>\z@
          f=\BKM@FLAGS,%
        \fi
        t=\BKM@title
      }%
    \endgroup
  \fi
}
%    \end{macrocode}
%    \end{macro}
%    \begin{macro}{\BKM@getx}
%    \begin{macrocode}
\def\BKM@getx#1#2{%
  \def\BKM@x@parent{#1}%
  \def\BKM@x@level{#2}%
}
%    \end{macrocode}
%    \end{macro}
%    \begin{macro}{\BKM@toRGB}
%    \begin{macrocode}
\def\BKM@toRGB#1 #2 #3\@nil{%
  \let\BKM@RGBcolor\@empty
  \BKM@toRGBComponent{#1}%
  \BKM@toRGBComponent{#2}%
  \BKM@toRGBComponent{#3}%
}
%    \end{macrocode}
%    \end{macro}
%    \begin{macro}{\BKM@toRGBComponent}
%    \begin{macrocode}
\def\BKM@toRGBComponent#1{%
  \dimen@=#1pt\relax
  \ifdim\dimen@>\z@
    \ifdim\dimen@<\p@
      \dimen@=255\dimen@
      \advance\dimen@ by 32768sp\relax
      \divide\dimen@ by 65536\relax
      \dimen@ii=\dimen@
      \divide\dimen@ii by 16\relax
      \edef\BKM@RGBcolor{%
        \BKM@RGBcolor
        \BKM@toHexDigit\dimen@ii
      }%
      \dimen@ii=16\dimen@ii
      \advance\dimen@-\dimen@ii
      \edef\BKM@RGBcolor{%
        \BKM@RGBcolor
        \BKM@toHexDigit\dimen@
      }%
    \else
      \edef\BKM@RGBcolor{\BKM@RGBcolor FF}%
    \fi
  \else
    \edef\BKM@RGBcolor{\BKM@RGBcolor00}%
  \fi
}
%    \end{macrocode}
%    \end{macro}
%    \begin{macro}{\BKM@toHexDigit}
%    \begin{macrocode}
\def\BKM@toHexDigit#1{%
  \ifcase\expandafter\@firstofone\expandafter{\number#1} %
    0\or 1\or 2\or 3\or 4\or 5\or 6\or 7\or
    8\or 9\or A\or B\or C\or D\or E\or F%
  \fi
}
%    \end{macrocode}
%    \end{macro}
%    \begin{macrocode}
\begingroup
  \catcode`\|=0 %
  \catcode`\\=12 %
%    \end{macrocode}
%    \begin{macro}{\BKM@vtex@title}
%    \begin{macrocode}
  |gdef|BKM@vtex@title{%
    |@onelevel@sanitize|BKM@title
    |edef|BKM@title{|expandafter|BKM@vtex@leftparen|BKM@title\(|@nil}%
    |edef|BKM@title{|expandafter|BKM@vtex@rightparen|BKM@title\)|@nil}%
    |edef|BKM@title{|expandafter|BKM@vtex@zero|BKM@title\0|@nil}%
    |edef|BKM@title{|expandafter|BKM@vtex@one|BKM@title\1|@nil}%
    |edef|BKM@title{|expandafter|BKM@vtex@two|BKM@title\2|@nil}%
    |edef|BKM@title{|expandafter|BKM@vtex@three|BKM@title\3|@nil}%
  }%
%    \end{macrocode}
%    \end{macro}
%    \begin{macro}{\BKM@vtex@leftparen}
%    \begin{macrocode}
  |gdef|BKM@vtex@leftparen#1\(#2|@nil{%
    #1%
    |ifx||#2||%
    |else
      (%
      |ltx@ReturnAfterFi{%
        |BKM@vtex@leftparen#2|@nil
      }%
    |fi
  }%
%    \end{macrocode}
%    \end{macro}
%    \begin{macro}{\BKM@vtex@rightparen}
%    \begin{macrocode}
  |gdef|BKM@vtex@rightparen#1\)#2|@nil{%
    #1%
    |ifx||#2||%
    |else
      )%
      |ltx@ReturnAfterFi{%
        |BKM@vtex@rightparen#2|@nil
      }%
    |fi
  }%
%    \end{macrocode}
%    \end{macro}
%    \begin{macro}{\BKM@vtex@zero}
%    \begin{macrocode}
  |gdef|BKM@vtex@zero#1\0#2|@nil{%
    #1%
    |ifx||#2||%
    |else
      |noexpand|hv@pdf@char0%
      |ltx@ReturnAfterFi{%
        |BKM@vtex@zero#2|@nil
      }%
    |fi
  }%
%    \end{macrocode}
%    \end{macro}
%    \begin{macro}{\BKM@vtex@one}
%    \begin{macrocode}
  |gdef|BKM@vtex@one#1\1#2|@nil{%
    #1%
    |ifx||#2||%
    |else
      |noexpand|hv@pdf@char1%
      |ltx@ReturnAfterFi{%
        |BKM@vtex@one#2|@nil
      }%
    |fi
  }%
%    \end{macrocode}
%    \end{macro}
%    \begin{macro}{\BKM@vtex@two}
%    \begin{macrocode}
  |gdef|BKM@vtex@two#1\2#2|@nil{%
    #1%
    |ifx||#2||%
    |else
      |noexpand|hv@pdf@char2%
      |ltx@ReturnAfterFi{%
        |BKM@vtex@two#2|@nil
      }%
    |fi
  }%
%    \end{macrocode}
%    \end{macro}
%    \begin{macro}{\BKM@vtex@three}
%    \begin{macrocode}
  |gdef|BKM@vtex@three#1\3#2|@nil{%
    #1%
    |ifx||#2||%
    |else
      |noexpand|hv@pdf@char3%
      |ltx@ReturnAfterFi{%
        |BKM@vtex@three#2|@nil
      }%
    |fi
  }%
%    \end{macrocode}
%    \end{macro}
%    \begin{macrocode}
|endgroup
%    \end{macrocode}
%
%    \begin{macrocode}
%</vtex>
%    \end{macrocode}
%
% \subsection{\hologo{pdfTeX}\ 的驱动程序}
%
%    \begin{macrocode}
%<*pdftex>
\NeedsTeXFormat{LaTeX2e}
\ProvidesFile{bkm-pdftex.def}%
  [2020-11-06 v1.29 bookmark driver for pdfTeX (HO)]%
%    \end{macrocode}
%
%    \begin{macro}{\BKM@DO@entry}
%    \begin{macrocode}
\def\BKM@DO@entry#1#2{%
  \begingroup
    \kvsetkeys{BKM@DO}{#1}%
    \def\BKM@DO@title{#2}%
    \ifx\BKM@DO@srcfile\@empty
    \else
      \BKM@UnescapeHex\BKM@DO@srcfile
    \fi
    \BKM@UnescapeHex\BKM@DO@title
    \expandafter\expandafter\expandafter\BKM@getx
        \csname BKM@\BKM@DO@id\endcsname\@empty\@empty
    \let\BKM@attr\@empty
    \ifx\BKM@DO@flags\@empty
    \else
      \edef\BKM@attr{\BKM@attr/F \BKM@DO@flags}%
    \fi
    \ifx\BKM@DO@color\@empty
    \else
      \edef\BKM@attr{\BKM@attr/C[\BKM@DO@color]}%
    \fi
    \ifx\BKM@attr\@empty
    \else
      \edef\BKM@attr{attr{\BKM@attr}}%
    \fi
    \let\BKM@action\@empty
    \ifx\BKM@DO@gotor\@empty
      \ifx\BKM@DO@dest\@empty
        \ifx\BKM@DO@named\@empty
          \ifx\BKM@DO@rawaction\@empty
            \ifx\BKM@DO@uri\@empty
              \ifx\BKM@DO@page\@empty
                \PackageError{bookmark}{%
                  Missing action\BKM@SourceLocation
                }\@ehc
                \edef\BKM@action{goto page1{/Fit}}%
              \else
                \ifx\BKM@DO@view\@empty
                  \def\BKM@DO@view{Fit}%
                \fi
                \edef\BKM@action{goto page\BKM@DO@page{/\BKM@DO@view}}%
              \fi
            \else
              \BKM@UnescapeHex\BKM@DO@uri
              \BKM@EscapeString\BKM@DO@uri
              \edef\BKM@action{user{<</S/URI/URI(\BKM@DO@uri)>>}}%
            \fi
          \else
            \BKM@UnescapeHex\BKM@DO@rawaction
            \edef\BKM@action{%
              user{%
                <<%
                  \BKM@DO@rawaction
                >>%
              }%
            }%
          \fi
        \else
          \BKM@EscapeName\BKM@DO@named
          \edef\BKM@action{%
            user{<</S/Named/N/\BKM@DO@named>>}%
          }%
        \fi
      \else
        \BKM@UnescapeHex\BKM@DO@dest
        \BKM@DefGotoNameAction\BKM@action\BKM@DO@dest
      \fi
    \else
      \ifx\BKM@DO@dest\@empty
        \ifx\BKM@DO@page\@empty
          \def\BKM@DO@page{0}%
        \else
          \BKM@CalcExpr\BKM@DO@page\BKM@DO@page-1%
        \fi
        \ifx\BKM@DO@view\@empty
          \def\BKM@DO@view{Fit}%
        \fi
        \edef\BKM@action{/D[\BKM@DO@page/\BKM@DO@view]}%
      \else
        \BKM@UnescapeHex\BKM@DO@dest
        \BKM@EscapeString\BKM@DO@dest
        \edef\BKM@action{/D(\BKM@DO@dest)}%
      \fi
      \BKM@UnescapeHex\BKM@DO@gotor
      \BKM@EscapeString\BKM@DO@gotor
      \edef\BKM@action{%
        user{%
          <<%
            /S/GoToR%
            /F(\BKM@DO@gotor)%
            \BKM@action
          >>%
        }%
      }%
    \fi
    \pdfoutline\BKM@attr\BKM@action
                count\ifBKM@DO@open\else-\fi\BKM@x@childs
                {\BKM@DO@title}%
  \endgroup
}
%    \end{macrocode}
%    \end{macro}
%    \begin{macro}{\BKM@DefGotoNameAction}
%    \cs{BKM@DefGotoNameAction}\ 宏是一个用于 \xpackage{hypdestopt}\ 宏包的钩子(hook)。
%    \begin{macrocode}
\def\BKM@DefGotoNameAction#1#2{%
  \BKM@EscapeString\BKM@DO@dest
  \edef#1{goto name{#2}}%
}
%    \end{macrocode}
%    \end{macro}
%    \begin{macrocode}
%</pdftex>
%    \end{macrocode}
%
%    \begin{macrocode}
%<*pdftex|pdfmark>
%    \end{macrocode}
%    \begin{macro}{\BKM@SourceLocation}
%    \begin{macrocode}
\def\BKM@SourceLocation{%
  \ifx\BKM@DO@srcfile\@empty
    \ifx\BKM@DO@srcline\@empty
    \else
      .\MessageBreak
      Source: line \BKM@DO@srcline
    \fi
  \else
    \ifx\BKM@DO@srcline\@empty
      .\MessageBreak
      Source: file `\BKM@DO@srcfile'%
    \else
      .\MessageBreak
      Source: file `\BKM@DO@srcfile', line \BKM@DO@srcline
    \fi
  \fi
}
%    \end{macrocode}
%    \end{macro}
%    \begin{macrocode}
%</pdftex|pdfmark>
%    \end{macrocode}
%
% \subsection{具有 pdfmark 特色(specials)的驱动程序}
%
% \subsubsection{dvips 驱动程序}
%
%    \begin{macrocode}
%<*dvips>
\NeedsTeXFormat{LaTeX2e}
\ProvidesFile{bkm-dvips.def}%
  [2020-11-06 v1.29 bookmark driver for dvips (HO)]%
%    \end{macrocode}
%    \begin{macro}{\BKM@PSHeaderFile}
%    \begin{macrocode}
\def\BKM@PSHeaderFile#1{%
  \special{PSfile=#1}%
}
%    \end{macrocode}
%    \begin{macro}{\BKM@filename}
%    \begin{macrocode}
\def\BKM@filename{\jobname.out.ps}
%    \end{macrocode}
%    \end{macro}
%    \begin{macrocode}
\AddToHook{shipout/lastpage}{%
  \BKM@pdfmark@out
  \BKM@PSHeaderFile\BKM@filename
  }
%    \end{macrocode}
%    \end{macro}
%    \begin{macrocode}
%</dvips>
%    \end{macrocode}
%
% \subsubsection{公共部分(Common part)}
%
%    \begin{macrocode}
%<*pdfmark>
%    \end{macrocode}
%
%    \begin{macro}{\BKM@pdfmark@out}
%    不要在这里使用 \xpackage{rerunfilecheck}\ 宏包,因为在 \hologo{TeX}\ 运行期间不会
%    读取 \cs{BKM@filename}\ 文件。
%    \begin{macrocode}
\def\BKM@pdfmark@out{%
  \if@filesw
    \newwrite\BKM@file
    \immediate\openout\BKM@file=\BKM@filename\relax
    \BKM@write{\@percentchar!}%
    \BKM@write{/pdfmark where{pop}}%
    \BKM@write{%
      {%
        /globaldict where{pop globaldict}{userdict}ifelse%
        /pdfmark/cleartomark load put%
      }%
    }%
    \BKM@write{ifelse}%
  \else
    \let\BKM@write\@gobble
    \let\BKM@DO@entry\@gobbletwo
  \fi
}
%    \end{macrocode}
%    \end{macro}
%    \begin{macro}{\BKM@write}
%    \begin{macrocode}
\def\BKM@write#{%
  \immediate\write\BKM@file
}
%    \end{macrocode}
%    \end{macro}
%
%    \begin{macro}{\BKM@DO@entry}
%    Pdfmark 的规范(specification)说明 |/Color| 是颜色(color)的键名(key name),
%    但是 ghostscript 只将键(key)传递到 PDF 文件中,因此键名必须是 |/C|。
%    \begin{macrocode}
\def\BKM@DO@entry#1#2{%
  \begingroup
    \kvsetkeys{BKM@DO}{#1}%
    \ifx\BKM@DO@srcfile\@empty
    \else
      \BKM@UnescapeHex\BKM@DO@srcfile
    \fi
    \def\BKM@DO@title{#2}%
    \BKM@UnescapeHex\BKM@DO@title
    \expandafter\expandafter\expandafter\BKM@getx
        \csname BKM@\BKM@DO@id\endcsname\@empty\@empty
    \let\BKM@attr\@empty
    \ifx\BKM@DO@flags\@empty
    \else
      \edef\BKM@attr{\BKM@attr/F \BKM@DO@flags}%
    \fi
    \ifx\BKM@DO@color\@empty
    \else
      \edef\BKM@attr{\BKM@attr/C[\BKM@DO@color]}%
    \fi
    \let\BKM@action\@empty
    \ifx\BKM@DO@gotor\@empty
      \ifx\BKM@DO@dest\@empty
        \ifx\BKM@DO@named\@empty
          \ifx\BKM@DO@rawaction\@empty
            \ifx\BKM@DO@uri\@empty
              \ifx\BKM@DO@page\@empty
                \PackageError{bookmark}{%
                  Missing action\BKM@SourceLocation
                }\@ehc
                \edef\BKM@action{%
                  /Action/GoTo%
                  /Page 1%
                  /View[/Fit]%
                }%
              \else
                \ifx\BKM@DO@view\@empty
                  \def\BKM@DO@view{Fit}%
                \fi
                \edef\BKM@action{%
                  /Action/GoTo%
                  /Page \BKM@DO@page
                  /View[/\BKM@DO@view]%
                }%
              \fi
            \else
              \BKM@UnescapeHex\BKM@DO@uri
              \BKM@EscapeString\BKM@DO@uri
              \edef\BKM@action{%
                /Action<<%
                  /Subtype/URI%
                  /URI(\BKM@DO@uri)%
                >>%
              }%
            \fi
          \else
            \BKM@UnescapeHex\BKM@DO@rawaction
            \edef\BKM@action{%
              /Action<<%
                \BKM@DO@rawaction
              >>%
            }%
          \fi
        \else
          \BKM@EscapeName\BKM@DO@named
          \edef\BKM@action{%
            /Action<<%
              /Subtype/Named%
              /N/\BKM@DO@named
            >>%
          }%
        \fi
      \else
        \BKM@UnescapeHex\BKM@DO@dest
        \BKM@EscapeString\BKM@DO@dest
        \edef\BKM@action{%
          /Action/GoTo%
          /Dest(\BKM@DO@dest)cvn%
        }%
      \fi
    \else
      \ifx\BKM@DO@dest\@empty
        \ifx\BKM@DO@page\@empty
          \def\BKM@DO@page{1}%
        \fi
        \ifx\BKM@DO@view\@empty
          \def\BKM@DO@view{Fit}%
        \fi
        \edef\BKM@action{%
          /Page \BKM@DO@page
          /View[/\BKM@DO@view]%
        }%
      \else
        \BKM@UnescapeHex\BKM@DO@dest
        \BKM@EscapeString\BKM@DO@dest
        \edef\BKM@action{%
          /Dest(\BKM@DO@dest)cvn%
        }%
      \fi
      \BKM@UnescapeHex\BKM@DO@gotor
      \BKM@EscapeString\BKM@DO@gotor
      \edef\BKM@action{%
        /Action/GoToR%
        /File(\BKM@DO@gotor)%
        \BKM@action
      }%
    \fi
    \BKM@write{[}%
    \BKM@write{/Title(\BKM@DO@title)}%
    \ifnum\BKM@x@childs>\z@
      \BKM@write{/Count \ifBKM@DO@open\else-\fi\BKM@x@childs}%
    \fi
    \ifx\BKM@attr\@empty
    \else
      \BKM@write{\BKM@attr}%
    \fi
    \BKM@write{\BKM@action}%
    \BKM@write{/OUT pdfmark}%
  \endgroup
}
%    \end{macrocode}
%    \end{macro}
%    \begin{macrocode}
%</pdfmark>
%    \end{macrocode}
%
% \subsection{\xoption{pdftex}\ 和 \xoption{pdfmark}\ 的公共部分}
%
%    \begin{macrocode}
%<*pdftex|pdfmark>
%    \end{macrocode}
%
% \subsubsection{写入辅助文件(auxiliary file)}
%
%    \begin{macrocode}
\AddToHook{begindocument}{%
 \immediate\write\@mainaux{\string\providecommand\string\BKM@entry[2]{}}}
%    \end{macrocode}
%
%    \begin{macro}{\BKM@id}
%    \begin{macrocode}
\newcount\BKM@id
\BKM@id=\z@
%    \end{macrocode}
%    \end{macro}
%
%    \begin{macro}{\BKM@0}
%    \begin{macrocode}
\@namedef{BKM@0}{000}
%    \end{macrocode}
%    \end{macro}
%    \begin{macro}{\ifBKM@sw}
%    \begin{macrocode}
\newif\ifBKM@sw
%    \end{macrocode}
%    \end{macro}
%
%    \begin{macro}{\bookmark}
%    \begin{macrocode}
\newcommand*{\bookmark}[2][]{%
  \if@filesw
    \begingroup
      \BKM@InitSourceLocation
      \def\bookmark@text{#2}%
      \BKM@setup{#1}%
      \ifx\BKM@srcfile\@empty
      \else
        \BKM@EscapeHex\BKM@srcfile
      \fi
      \edef\BKM@prev{\the\BKM@id}%
      \global\advance\BKM@id\@ne
      \BKM@swtrue
      \@whilesw\ifBKM@sw\fi{%
        \ifnum\ifBKM@startatroot\z@\else\BKM@prev\fi=\z@
          \BKM@startatrootfalse
          \expandafter\xdef\csname BKM@\the\BKM@id\endcsname{%
            0{\BKM@level}0%
          }%
          \BKM@swfalse
        \else
          \expandafter\expandafter\expandafter\BKM@getx
              \csname BKM@\BKM@prev\endcsname
          \ifnum\BKM@level>\BKM@x@level\relax
            \expandafter\xdef\csname BKM@\the\BKM@id\endcsname{%
              {\BKM@prev}{\BKM@level}0%
            }%
            \ifnum\BKM@prev>\z@
              \BKM@CalcExpr\BKM@CalcResult\BKM@x@childs+1%
              \expandafter\xdef\csname BKM@\BKM@prev\endcsname{%
                {\BKM@x@parent}{\BKM@x@level}{\BKM@CalcResult}%
              }%
            \fi
            \BKM@swfalse
          \else
            \let\BKM@prev\BKM@x@parent
          \fi
        \fi
      }%
      \pdfstringdef\BKM@title{\bookmark@text}%
      \edef\BKM@FLAGS{\BKM@PrintStyle}%
      \csname BKM@HypDestOptHook\endcsname
      \BKM@EscapeHex\BKM@dest
      \BKM@EscapeHex\BKM@uri
      \BKM@EscapeHex\BKM@gotor
      \BKM@EscapeHex\BKM@rawaction
      \BKM@EscapeHex\BKM@title
      \immediate\write\@mainaux{%
        \string\BKM@entry{%
          id=\number\BKM@id
          \ifBKM@open
            \ifnum\BKM@level<\BKM@openlevel
              ,open%
            \fi
          \fi
          \BKM@auxentry{dest}%
          \BKM@auxentry{named}%
          \BKM@auxentry{uri}%
          \BKM@auxentry{gotor}%
          \BKM@auxentry{page}%
          \BKM@auxentry{view}%
          \BKM@auxentry{rawaction}%
          \BKM@auxentry{color}%
          \ifnum\BKM@FLAGS>\z@
            ,flags=\BKM@FLAGS
          \fi
          \BKM@auxentry{srcline}%
          \BKM@auxentry{srcfile}%
        }{\BKM@title}%
      }%
    \endgroup
  \fi
}
%    \end{macrocode}
%    \end{macro}
%    \begin{macro}{\BKM@getx}
%    \begin{macrocode}
\def\BKM@getx#1#2#3{%
  \def\BKM@x@parent{#1}%
  \def\BKM@x@level{#2}%
  \def\BKM@x@childs{#3}%
}
%    \end{macrocode}
%    \end{macro}
%    \begin{macro}{\BKM@auxentry}
%    \begin{macrocode}
\def\BKM@auxentry#1{%
  \expandafter\ifx\csname BKM@#1\endcsname\@empty
  \else
    ,#1={\csname BKM@#1\endcsname}%
  \fi
}
%    \end{macrocode}
%    \end{macro}
%
%    \begin{macro}{\BKM@InitSourceLocation}
%    \begin{macrocode}
\def\BKM@InitSourceLocation{%
  \edef\BKM@srcline{\the\inputlineno}%
  \BKM@LuaTeX@InitFile
  \ifx\BKM@srcfile\@empty
    \ltx@IfUndefined{currfilepath}{}{%
      \edef\BKM@srcfile{\currfilepath}%
    }%
  \fi
}
%    \end{macrocode}
%    \end{macro}
%    \begin{macro}{\BKM@LuaTeX@InitFile}
%    \begin{macrocode}
\ifluatex
  \ifnum\luatexversion>36 %
    \def\BKM@LuaTeX@InitFile{%
      \begingroup
        \ltx@LocToksA={}%
      \edef\x{\endgroup
        \def\noexpand\BKM@srcfile{%
          \the\expandafter\ltx@LocToksA
          \directlua{%
             if status and status.filename then %
               tex.settoks('ltx@LocToksA', status.filename)%
             end%
          }%
        }%
      }\x
    }%
  \else
    \let\BKM@LuaTeX@InitFile\relax
  \fi
\else
  \let\BKM@LuaTeX@InitFile\relax
\fi
%    \end{macrocode}
%    \end{macro}
%
% \subsubsection{读取辅助数据(auxiliary data)}
%
%    \begin{macrocode}
\SetupKeyvalOptions{family=BKM@DO,prefix=BKM@DO@}
\DeclareStringOption[0]{id}
\DeclareBoolOption{open}
\DeclareStringOption{flags}
\DeclareStringOption{color}
\DeclareStringOption{dest}
\DeclareStringOption{named}
\DeclareStringOption{uri}
\DeclareStringOption{gotor}
\DeclareStringOption{page}
\DeclareStringOption{view}
\DeclareStringOption{rawaction}
\DeclareStringOption{srcline}
\DeclareStringOption{srcfile}
%    \end{macrocode}
%
%    \begin{macrocode}
\AtBeginDocument{%
  \let\BKM@entry\BKM@DO@entry
}
%    \end{macrocode}
%
%    \begin{macrocode}
%</pdftex|pdfmark>
%    \end{macrocode}
%
% \subsection{\xoption{atend}\ 选项}
%
% \subsubsection{钩子(Hook)}
%
%    \begin{macrocode}
%<*package>
%    \end{macrocode}
%    \begin{macrocode}
\ifBKM@atend
\else
%    \end{macrocode}
%    \begin{macro}{\BookmarkAtEnd}
%    这是一个虚拟定义(dummy definition),如果没有给出 \xoption{atend}\ 选项,它将生成一个警告。
%    \begin{macrocode}
  \newcommand{\BookmarkAtEnd}[1]{%
    \PackageWarning{bookmark}{%
      Ignored, because option `atend' is missing%
    }%
  }%
%    \end{macrocode}
%    \end{macro}
%    \begin{macrocode}
  \expandafter\endinput
\fi
%    \end{macrocode}
%    \begin{macro}{\BookmarkAtEnd}
%    \begin{macrocode}
\newcommand*{\BookmarkAtEnd}{%
  \g@addto@macro\BKM@EndHook
}
%    \end{macrocode}
%    \end{macro}
%    \begin{macrocode}
\let\BKM@EndHook\@empty
%    \end{macrocode}
%    \begin{macrocode}
%</package>
%    \end{macrocode}
%
% \subsubsection{在文档末尾使用钩子的驱动程序}
%
%    驱动程序 \xoption{pdftex}\ 使用 LaTeX 钩子 \xoption{enddocument/afterlastpage}
%    (相当于以前使用的 \xpackage{atveryend}\ 的 \cs{AfterLastShipout}),因为它仍然需要 \xext{aux}\ 文件。
%    它使用 \cs{pdfoutline}\ 作为最后一页之后可以使用的书签(bookmakrs)。
%    \begin{itemize}
%    \item
%      驱动程序 \xoption{pdftex}\ 使用 \cs{pdfoutline}, \cs{pdfoutline}\ 可以在最后一页之后使用。
%    \end{itemize}
%    \begin{macrocode}
%<*pdftex>
\ifBKM@atend
  \AddToHook{enddocument/afterlastpage}{%
    \BKM@EndHook
  }%
\fi
%</pdftex>
%    \end{macrocode}
%
% \subsubsection{使用 \xoption{shipout/lastpage}\ 的驱动程序}
%
%    其他驱动程序使用 \cs{special}\ 命令实现 \cs{bookmark}。因此,最后的书签(last bookmarks)
%    必须放在最后一页(last page),而不是之后。不能使用 \cs{AtEndDocument},因为为时已晚,
%    最后一页已经输出了。因此,我们使用 LaTeX 钩子 \xoption{shipout/lastpage}。至少需要运行
%    两次 \hologo{LaTeX}。PostScript 驱动程序 \xoption{dvips}\ 使用外部 PostScript 文件作为书签。
%    为了避免与 pgf 发生冲突,文件写入(file writing)也被移到了最后一个输出页面(shipout page)。
%    \begin{macrocode}
%<*dvipdfm|vtex|pdfmark>
\ifBKM@atend
  \AddToHook{shipout/lastpage}{\BKM@EndHook}%
\fi
%</dvipdfm|vtex|pdfmark>
%    \end{macrocode}
%
% \section{安装(Installation)}
%
% \subsection{下载(Download)}
%
% \paragraph{宏包(Package)。} 在 CTAN\footnote{\CTANpkg{bookmark}}上提供此宏包:
% \begin{description}
% \item[\CTAN{macros/latex/contrib/bookmark/bookmark.dtx}] 源文件(source file)。
% \item[\CTAN{macros/latex/contrib/bookmark/bookmark.pdf}] 文档(documentation)。
% \end{description}
%
%
% \paragraph{捆绑包(Bundle)。} “bookmark”捆绑包(bundle)的所有宏包(packages)都可以在兼
% 容 TDS 的 ZIP 归档文件中找到。在那里,宏包已经被解包,文档文件(documentation files)已经生成。
% 文件(files)和目录(directories)遵循 TDS 标准。
% \begin{description}
% \item[\CTANinstall{install/macros/latex/contrib/bookmark.tds.zip}]
% \end{description}
% \emph{TDS}\ 是指标准的“用于 \TeX\ 文件的目录结构(Directory Structure)”(\CTANpkg{tds})。
% 名称中带有 \xfile{texmf}\ 的目录(directories)通常以这种方式组织。
%
% \subsection{捆绑包(Bundle)的安装}
%
% \paragraph{解压(Unpacking)。} 在您选择的 TDS 树(也称为 \xfile{texmf}\ 树)中解
% 压 \xfile{bookmark.tds.zip},例如(在 linux 中):
% \begin{quote}
%   |unzip bookmark.tds.zip -d ~/texmf|
% \end{quote}
%
% \subsection{宏包(Package)的安装}
%
% \paragraph{解压(Unpacking)。} \xfile{.dtx}\ 文件是一个自解压 \docstrip\ 归档文件(archive)。
% 这些文件是通过 \plainTeX\ 运行 \xfile{.dtx}\ 来提取的:
% \begin{quote}
%   \verb|tex bookmark.dtx|
% \end{quote}
%
% \paragraph{TDS.} 现在,不同的文件必须移动到安装 TDS 树(installation TDS tree)
% (也称为 \xfile{texmf}\ 树)中的不同目录中:
% \begin{quote}
% \def\t{^^A
% \begin{tabular}{@{}>{\ttfamily}l@{ $\rightarrow$ }>{\ttfamily}l@{}}
%   bookmark.sty & tex/latex/bookmark/bookmark.sty\\
%   bkm-dvipdfm.def & tex/latex/bookmark/bkm-dvipdfm.def\\
%   bkm-dvips.def & tex/latex/bookmark/bkm-dvips.def\\
%   bkm-pdftex.def & tex/latex/bookmark/bkm-pdftex.def\\
%   bkm-vtex.def & tex/latex/bookmark/bkm-vtex.def\\
%   bookmark.pdf & doc/latex/bookmark/bookmark.pdf\\
%   bookmark-example.tex & doc/latex/bookmark/bookmark-example.tex\\
%   bookmark.dtx & source/latex/bookmark/bookmark.dtx\\
% \end{tabular}^^A
% }^^A
% \sbox0{\t}^^A
% \ifdim\wd0>\linewidth
%   \begingroup
%     \advance\linewidth by\leftmargin
%     \advance\linewidth by\rightmargin
%   \edef\x{\endgroup
%     \def\noexpand\lw{\the\linewidth}^^A
%   }\x
%   \def\lwbox{^^A
%     \leavevmode
%     \hbox to \linewidth{^^A
%       \kern-\leftmargin\relax
%       \hss
%       \usebox0
%       \hss
%       \kern-\rightmargin\relax
%     }^^A
%   }^^A
%   \ifdim\wd0>\lw
%     \sbox0{\small\t}^^A
%     \ifdim\wd0>\linewidth
%       \ifdim\wd0>\lw
%         \sbox0{\footnotesize\t}^^A
%         \ifdim\wd0>\linewidth
%           \ifdim\wd0>\lw
%             \sbox0{\scriptsize\t}^^A
%             \ifdim\wd0>\linewidth
%               \ifdim\wd0>\lw
%                 \sbox0{\tiny\t}^^A
%                 \ifdim\wd0>\linewidth
%                   \lwbox
%                 \else
%                   \usebox0
%                 \fi
%               \else
%                 \lwbox
%               \fi
%             \else
%               \usebox0
%             \fi
%           \else
%             \lwbox
%           \fi
%         \else
%           \usebox0
%         \fi
%       \else
%         \lwbox
%       \fi
%     \else
%       \usebox0
%     \fi
%   \else
%     \lwbox
%   \fi
% \else
%   \usebox0
% \fi
% \end{quote}
% 如果你有一个 \xfile{docstrip.cfg}\ 文件,该文件能配置并启用 \docstrip\ 的 TDS 安装功能,
% 则一些文件可能已经在正确的位置了,请参阅 \docstrip\ 的文档(documentation)。
%
% \subsection{刷新文件名数据库}
%
% 如果您的 \TeX~发行版(\TeX\,Live、\mikTeX、\dots)依赖于文件名数据库(file name databases),
% 则必须刷新这些文件名数据库。例如,\TeX\,Live\ 用户运行 \verb|texhash| 或 \verb|mktexlsr|。
%
% \subsection{一些感兴趣的细节}
%
% \paragraph{用 \LaTeX\ 解压。}
% \xfile{.dtx}\ 根据格式(format)选择其操作(action):
% \begin{description}
% \item[\plainTeX:] 运行 \docstrip\ 并解压文件。
% \item[\LaTeX:] 生成文档。
% \end{description}
% 如果您坚持通过 \LaTeX\ 使用\docstrip (实际上 \docstrip\ 并不需要 \LaTeX),那么请您的意图告知自动检测程序:
% \begin{quote}
%   \verb|latex \let\install=y\input{bookmark.dtx}|
% \end{quote}
% 不要忘记根据 shell 的要求引用这个参数(argument)。
%
% \paragraph{知生成文档。}
% 您可以同时使用 \xfile{.dtx}\ 或 \xfile{.drv}\ 来生成文档。可以通过配置文件 \xfile{ltxdoc.cfg}\ 配置该进程。
% 例如,如果您希望 A4 作为纸张格式,请将下面这行写入此文件中:
% \begin{quote}
%   \verb|\PassOptionsToClass{a4paper}{article}|
% \end{quote}
% 下面是一个如何使用 pdf\LaTeX\ 生成文档的示例:
% \begin{quote}
%\begin{verbatim}
%pdflatex bookmark.dtx
%makeindex -s gind.ist bookmark.idx
%pdflatex bookmark.dtx
%makeindex -s gind.ist bookmark.idx
%pdflatex bookmark.dtx
%\end{verbatim}
% \end{quote}
%
% \begin{thebibliography}{9}
%
% \bibitem{hyperref}
%   Sebastian Rahtz, Heiko Oberdiek:
%   \textit{The \xpackage{hyperref} package};
%   2011/04/17 v6.82g;
%   \CTANpkg{hyperref}
%
% \bibitem{currfile}
%   Martin Scharrer:
%   \textit{The \xpackage{currfile} package};
%   2011/01/09 v0.4.
%   \CTANpkg{currfile}
%
% \end{thebibliography}
%
% \begin{History}
%   \begin{Version}{2007/02/19 v0.1}
%   \item
%     First experimental version.
%   \end{Version}
%   \begin{Version}{2007/02/20 v0.2}
%   \item
%     Option \xoption{startatroot} added.
%   \item
%     Dummies for \cs{pdf(un)escape...} commands added to get
%     the package basically work for non-\hologo{pdfTeX} users.
%   \end{Version}
%   \begin{Version}{2007/02/21 v0.3}
%   \item
%     Dependency from \hologo{pdfTeX} 1.30 removed by using package
%     \xpackage{pdfescape}.
%   \end{Version}
%   \begin{Version}{2007/02/22 v0.4}
%   \item
%     \xpackage{hyperref}'s \xoption{bookmarkstype} respected.
%   \end{Version}
%   \begin{Version}{2007/03/02 v0.5}
%   \item
%     Driver options \xoption{vtex} (PDF mode), \xoption{dvipsone},
%     and \xoption{textures} added.
%   \item
%     Implementation of option \xoption{depth} completed. Division names
%     are supported, see \xpackage{hyperref}'s
%     option \xoption{bookmarksdepth}.
%   \item
%     \xpackage{hyperref}'s options \xoption{bookmarksopen},
%     \xoption{bookmarksopenlevel}, and \xoption{bookmarksdepth} respected.
%   \end{Version}
%   \begin{Version}{2007/03/03 v0.6}
%   \item
%     Option \xoption{numbered} as alias for \xpackage{hyperref}'s
%     \xoption{bookmarksnumbered}.
%   \end{Version}
%   \begin{Version}{2007/03/07 v0.7}
%   \item
%     Dependency from \hologo{eTeX} removed.
%   \end{Version}
%   \begin{Version}{2007/04/09 v0.8}
%   \item
%     Option \xoption{atend} added.
%   \item
%     Option \xoption{rgbcolor} removed.
%     \verb|rgbcolor=<r> <g> <b>| can be replaced by
%     \verb|color=[rgb]{<r>,<g>,<b>}|.
%   \item
%     Support of recent cvs version (2007-03-29) of dvipdfmx
%     that extends the \cs{special} for bookmarks to specify
%     open outline entries. Option \xoption{dvipdfmx-outline-open}
%     or \cs{SpecialDvipdfmxOutlineOpen} notify the package.
%   \end{Version}
%   \begin{Version}{2007/04/25 v0.9}
%   \item
%     The syntax of \cs{special} of dvipdfmx, if feature
%     \xoption{dvipdfmx-outline-open} is enabled, has changed.
%     Now cvs version 2007-04-25 is needed.
%   \end{Version}
%   \begin{Version}{2007/05/29 v1.0}
%   \item
%     Bug fix in code for second parameter of XYZ.
%   \end{Version}
%   \begin{Version}{2007/07/13 v1.1}
%   \item
%     Fix for pdfmark with GoToR action.
%   \end{Version}
%   \begin{Version}{2007/09/25 v1.2}
%   \item
%     pdfmark driver respects \cs{nofiles}.
%   \end{Version}
%   \begin{Version}{2008/08/08 v1.3}
%   \item
%     Package \xpackage{flags} replaced by package \xpackage{bitset}.
%     Now flags are also supported without \hologo{eTeX}.
%   \item
%     Hook for package \xpackage{hypdestopt} added.
%   \end{Version}
%   \begin{Version}{2008/09/13 v1.4}
%   \item
%     Fix for bug introduced in v1.3, package \xpackage{flags} is one-based,
%     but package \xpackage{bitset} is zero-based. Thus options \xoption{bold}
%     and \xoption{italic} are wrong in v1.3. (Daniel M\"ullner)
%   \end{Version}
%   \begin{Version}{2009/08/13 v1.5}
%   \item
%     Except for driver options the other options are now local options.
%     This resolves a problem with KOMA-Script v3.00 and its option \xoption{open}.
%   \end{Version}
%   \begin{Version}{2009/12/06 v1.6}
%   \item
%     Use of package \xpackage{atveryend} for drivers \xoption{pdftex}
%     and \xoption{pdfmark}.
%   \end{Version}
%   \begin{Version}{2009/12/07 v1.7}
%   \item
%     Use of package \xpackage{atveryend} fixed.
%   \end{Version}
%   \begin{Version}{2009/12/17 v1.8}
%   \item
%     Support of \xpackage{hyperref} 2009/12/17 v6.79v for \hologo{XeTeX}.
%   \end{Version}
%   \begin{Version}{2010/03/30 v1.9}
%   \item
%     Package name in an error message fixed.
%   \end{Version}
%   \begin{Version}{2010/04/03 v1.10}
%   \item
%     Option \xoption{style} and macro \cs{bookmarkdefinestyle} added.
%   \item
%     Hook support with option \xoption{addtohook} added.
%   \item
%     \cs{bookmarkget} added.
%   \end{Version}
%   \begin{Version}{2010/04/04 v1.11}
%   \item
%     Bug fix (introduced in v1.10).
%   \end{Version}
%   \begin{Version}{2010/04/08 v1.12}
%   \item
%     Requires \xpackage{ltxcmds} 2010/04/08.
%   \end{Version}
%   \begin{Version}{2010/07/23 v1.13}
%   \item
%     Support for \xclass{memoir}'s \cs{booknumberline} added.
%   \end{Version}
%   \begin{Version}{2010/09/02 v1.14}
%   \item
%     (Local) options \xoption{draft} and \xoption{final} added.
%   \end{Version}
%   \begin{Version}{2010/09/25 v1.15}
%   \item
%     Fix for option \xoption{dvipdfmx-outline-open}.
%   \item
%     Option \xoption{dvipdfmx-outline-open} is set automatically,
%     if XeTeX $\geq$ 0.9995 is detected.
%   \end{Version}
%   \begin{Version}{2010/10/19 v1.16}
%   \item
%     Option `startatroot' now acts globally.
%   \item
%     Option `level' also accepts names the same way as option `depth'.
%   \end{Version}
%   \begin{Version}{2010/10/25 v1.17}
%   \item
%     \cs{bookmarksetupnext} added.
%   \item
%     Using \cs{kvsetkeys} of package \xpackage{kvsetkeys}, because
%     \cs{setkeys} of package \xpackage{keyval} is not reentrant.
%     This can cause problems (unknown keys) with older versions of
%     hyperref that also uses \cs{setkeys} (found by GL).
%   \end{Version}
%   \begin{Version}{2010/11/05 v1.18}
%   \item
%     Use of \cs{pdf@ifdraftmode} of package \xpackage{pdftexcmds} for
%     the default of option \xoption{draft}.
%   \end{Version}
%   \begin{Version}{2011/03/20 v1.19}
%   \item
%     Use of \cs{dimexpr} fixed, if \hologo{eTeX} is not used.
%     (Bug found by Martin M\"unch.)
%   \item
%     Fix in documentation. Also layout options work without \hologo{eTeX}.
%   \end{Version}
%   \begin{Version}{2011/04/13 v1.20}
%   \item
%     Bug fix: \cs{BKM@SetDepth} renamed to \cs{BKM@SetDepthOrLevel}.
%   \end{Version}
%   \begin{Version}{2011/04/21 v1.21}
%   \item
%     Some support for file name and line number in error messages
%     at end of document (pdfTeX and pdfmark based drivers).
%   \end{Version}
%   \begin{Version}{2011/05/13 v1.22}
%   \item
%     Change of version 2010/11/05 v1.18 reverted, because otherwise
%     draftmode disables some \xext{aux} file entries.
%   \end{Version}
%   \begin{Version}{2011/09/19 v1.23}
%   \item
%     Some \cs{renewcommand}s changed to \cs{def} to avoid trouble
%     if the commands are not defined, because hyperref stopped early.
%   \end{Version}
%   \begin{Version}{2011/12/02 v1.24}
%   \item
%     Small optimization in \cs{BKM@toHexDigit}.
%   \end{Version}
%   \begin{Version}{2016/05/16 v1.25}
%   \item
%     Documentation updates.
%   \end{Version}
%   \begin{Version}{2016/05/17 v1.26}
%   \item
%     define \cs{pdfoutline} to allow pdftex driver to be used with Lua\TeX.
%   \end{Version}
%   \begin{Version}{2019/06/04 v1.27}
%   \item
%     unknown style options are ignored (issue 67)
%   \end{Version}

%   \begin{Version}{2019/12/03 v1.28}
%   \item
%     Documentation updates.
%   \item adjust package loading (all required packages already loaded
%     by \xpackage{hyperref}).
%   \end{Version}
%   \begin{Version}{2020-11-06 v1.29}
%   \item Adapted the dvips to avoid a clash with pgf.
%         https://github.com/pgf-tikz/pgf/issues/944
%   \item All drivers now use the new LaTeX hooks
%         and so require a format 2020-10-01 or newer. The older
%         drivers are provided as frozen versions and are used if an older
%         format is detected.
%   \item Added support for destlabel option of hyperref, https://github.com/ho-tex/bookmark/issues/1
%   \item Removed the \xoption{dvipsone} and \xoption{textures} driver.
%   \item Removed the code for option \xoption{dvipdfmx-outline-open}
%     and \cs{SpecialDvipdfmxOutlineOpen}. All dvipdfmx version should now support
%     this out-of-the-box.
%   \end{Version}
% \end{History}
%
% \PrintIndex
%
% \Finale
\endinput

%        (quote the arguments according to the demands of your shell)
%
% Documentation:
%    (a) If bookmark.drv is present:
%           latex bookmark.drv
%    (b) Without bookmark.drv:
%           latex bookmark.dtx; ...
%    The class ltxdoc loads the configuration file ltxdoc.cfg
%    if available. Here you can specify further options, e.g.
%    use A4 as paper format:
%       \PassOptionsToClass{a4paper}{article}
%
%    Programm calls to get the documentation (example):
%       pdflatex bookmark.dtx
%       makeindex -s gind.ist bookmark.idx
%       pdflatex bookmark.dtx
%       makeindex -s gind.ist bookmark.idx
%       pdflatex bookmark.dtx
%
% Installation:
%    TDS:tex/latex/bookmark/bookmark.sty
%    TDS:tex/latex/bookmark/bkm-dvipdfm.def
%    TDS:tex/latex/bookmark/bkm-dvips.def
%    TDS:tex/latex/bookmark/bkm-pdftex.def
%    TDS:tex/latex/bookmark/bkm-vtex.def
%    TDS:tex/latex/bookmark/bkm-dvipdfm-2019-12-03.def
%    TDS:tex/latex/bookmark/bkm-dvips-2019-12-03.def
%    TDS:tex/latex/bookmark/bkm-pdftex-2019-12-03.def
%    TDS:tex/latex/bookmark/bkm-vtex-2019-12-03.def%
%    TDS:doc/latex/bookmark/bookmark.pdf
%    TDS:doc/latex/bookmark/bookmark-example.tex
%    TDS:source/latex/bookmark/bookmark.dtx
%    TDS:source/latex/bookmark/bookmark-frozen.dtx
%
%<*ignore>
\begingroup
  \catcode123=1 %
  \catcode125=2 %
  \def\x{LaTeX2e}%
\expandafter\endgroup
\ifcase 0\ifx\install y1\fi\expandafter
         \ifx\csname processbatchFile\endcsname\relax\else1\fi
         \ifx\fmtname\x\else 1\fi\relax
\else\csname fi\endcsname
%</ignore>
%<*install>
\input docstrip.tex
\Msg{************************************************************************}
\Msg{* Installation}
\Msg{* Package: bookmark 2020-11-06 v1.29 PDF bookmarks (HO)}
\Msg{************************************************************************}

\keepsilent
\askforoverwritefalse

\let\MetaPrefix\relax
\preamble

This is a generated file.

Project: bookmark
Version: 2020-11-06 v1.29

Copyright (C)
   2007-2011 Heiko Oberdiek
   2016-2020 Oberdiek Package Support Group

This work may be distributed and/or modified under the
conditions of the LaTeX Project Public License, either
version 1.3c of this license or (at your option) any later
version. This version of this license is in
   https://www.latex-project.org/lppl/lppl-1-3c.txt
and the latest version of this license is in
   https://www.latex-project.org/lppl.txt
and version 1.3 or later is part of all distributions of
LaTeX version 2005/12/01 or later.

This work has the LPPL maintenance status "maintained".

The Current Maintainers of this work are
Heiko Oberdiek and the Oberdiek Package Support Group
https://github.com/ho-tex/bookmark/issues


This work consists of the main source file bookmark.dtx and bookmark-frozen.dtx
and the derived files
   bookmark.sty, bookmark.pdf, bookmark.ins, bookmark.drv,
   bkm-dvipdfm.def, bkm-dvips.def, bkm-pdftex.def, bkm-vtex.def,
   bkm-dvipdfm-2019-12-03.def, bkm-dvips-2019-12-03.def,
   bkm-pdftex-2019-12-03.def, bkm-vtex-2019-12-03.def,
   bookmark-example.tex.

\endpreamble
\let\MetaPrefix\DoubleperCent

\generate{%
  \file{bookmark.ins}{\from{bookmark.dtx}{install}}%
  \file{bookmark.drv}{\from{bookmark.dtx}{driver}}%
  \usedir{tex/latex/bookmark}%
  \file{bookmark.sty}{\from{bookmark.dtx}{package}}%
  \file{bkm-dvipdfm.def}{\from{bookmark.dtx}{dvipdfm}}%
  \file{bkm-dvips.def}{\from{bookmark.dtx}{dvips,pdfmark}}%
  \file{bkm-pdftex.def}{\from{bookmark.dtx}{pdftex}}%
  \file{bkm-vtex.def}{\from{bookmark.dtx}{vtex}}%
  \usedir{doc/latex/bookmark}%
  \file{bookmark-example.tex}{\from{bookmark.dtx}{example}}%
  \file{bkm-pdftex-2019-12-03.def}{\from{bookmark-frozen.dtx}{pdftexfrozen}}%
  \file{bkm-dvips-2019-12-03.def}{\from{bookmark-frozen.dtx}{dvipsfrozen}}%
  \file{bkm-vtex-2019-12-03.def}{\from{bookmark-frozen.dtx}{vtexfrozen}}%
  \file{bkm-dvipdfm-2019-12-03.def}{\from{bookmark-frozen.dtx}{dvipdfmfrozen}}%
}

\catcode32=13\relax% active space
\let =\space%
\Msg{************************************************************************}
\Msg{*}
\Msg{* To finish the installation you have to move the following}
\Msg{* files into a directory searched by TeX:}
\Msg{*}
\Msg{*     bookmark.sty, bkm-dvipdfm.def, bkm-dvips.def,}
\Msg{*     bkm-pdftex.def, bkm-vtex.def, bkm-dvipdfm-2019-12-03.def,}
\Msg{*     bkm-dvips-2019-12-03.def, bkm-pdftex-2019-12-03.def,}
\Msg{*     and bkm-vtex-2019-12-03.def}
\Msg{*}
\Msg{* To produce the documentation run the file `bookmark.drv'}
\Msg{* through LaTeX.}
\Msg{*}
\Msg{* Happy TeXing!}
\Msg{*}
\Msg{************************************************************************}

\endbatchfile
%</install>
%<*ignore>
\fi
%</ignore>
%<*driver>
\NeedsTeXFormat{LaTeX2e}
\ProvidesFile{bookmark.drv}%
  [2020-11-06 v1.29 PDF bookmarks (HO)]%
\documentclass{ltxdoc}
\usepackage{ctex}
\usepackage{indentfirst}
\setlength{\parindent}{2em}
\usepackage{holtxdoc}[2011/11/22]
\usepackage{xcolor}
\usepackage{hyperref}
\usepackage[open,openlevel=3,atend]{bookmark}[2020/11/06] %%%打开书签,显示的深度为3级,即显示part、section、subsection。
\bookmarksetup{color=red}
\begin{document}

  \renewcommand{\contentsname}{目\quad 录}
  \renewcommand{\abstractname}{摘\quad 要}
  \renewcommand{\historyname}{历史}
  \DocInput{bookmark.dtx}%
\end{document}
%</driver>
% \fi
%
%
%
% \GetFileInfo{bookmark.drv}
%
%% \title{\xpackage{bookmark} 宏包}
% \title{\heiti {\Huge \textbf{\xpackage{bookmark}\ 宏包}}}
% \date{2020-11-06\ \ \ v1.29}
% \author{Heiko Oberdiek \thanks
% {如有问题请点击:\url{https://github.com/ho-tex/bookmark/issues}}\\[5pt]赣医一附院神经科\ \ 黄旭华\ \ \ \ 译}
%
% \maketitle
%
% \begin{abstract}
% 这个宏包为 \xpackage{hyperref}\ 宏包实现了一个新的书签(bookmark)(大纲[outline])组织。现在
% 可以设置样式(style)和颜色(color)等书签属性(bookmark properties)。其他动作类型(action types)可用
% (URI、GoToR、Named)。书签是在第一次编译运行(compile run)中生成的。\xpackage{hyperref}\
% 宏包必需运行两次。
% \end{abstract}
%
% \tableofcontents
%
% \section{文档(Documentation)}
%
% \subsection{介绍}
%
% 这个 \xpackage{bookmark}\ 宏包试图为书签(bookmarks)提供一个更现代的管理:
% \begin{itemize}
% \item 书签已经在第一次 \hologo{TeX}\ 编译运行(compile run)中生成。
% \item 可以更改书签的字体样式(font style)和颜色(color)。
% \item 可以执行比简单的 GoTo 操作(actions)更多的操作。
% \end{itemize}
%
% 与 \xpackage{hyperref} \cite{hyperref} 一样,书签(bookmarks)也是按照书签生成宏
% (bookmark generating macros)(\cs{bookmark})的顺序生成的。级别号(level number)用于
% 定义书签的树结构(tree structure)。限制没有那么严格:
% \begin{itemize}
% \item 级别值(level values)可以跳变(jump)和省略(omit)。\cs{subsubsection}\ 可以跟在
%       \cs{chapter}\ 之后。这种情况如在 \xpackage{hyperref}\ 中则产生错误,它将显示一个警告(warning)
%       并尝试修复此错误。
% \item 多个书签可能指向同一目标(destination)。在 \xpackage{hyperref}\ 中,这会完全弄乱
%       书签树(bookmark tree),因为算法假设(algorithm assumes)目标名称(destination names)
%       是键(keys)(唯一的)。
% \end{itemize}
%
% 注意,这个宏包是作为书签管理(bookmark management)的实验平台(experimentation platform)。
% 欢迎反馈。此外,在未来的版本中,接口(interfaces)也可能发生变化。
%
% \subsection{选项(Options)}
%
% 可在以下四个地方放置选项(options):
% \begin{enumerate}
% \item \cs{usepackage}|[|\meta{options}|]{bookmark}|\\
%       这是放置驱动程序选项(driver options)和 \xoption{atend}\ 选项的唯一位置。
% \item \cs{bookmarksetup}|{|\meta{options}|}|\\
%       此命令仅用于设置选项(setting options)。
% \item \cs{bookmarksetupnext}|{|\meta{options}|}|\\
%       这些选项在下一个 \cs{bookmark}\ 命令的选项之后存储(stored)和调用(called)。
% \item \cs{bookmark}|[|\meta{options}|]{|\meta{title}|}|\\
%       此命令设置书签。选项设置(option settings)仅限于此书签。
% \end{enumerate}
% 异常(Exception):加载该宏包后,无法更改驱动程序选项(Driver options)、\xoption{atend}\ 选项
% 、\xoption{draft}\slash\xoption{final}选项。
%
% \subsubsection{\xoption{draft} 和 \xoption{final}\ 选项}
%
% 如果一个\LaTeX\ 文件要被编译了多次,那么可以使用 \xoption{draft}\ 选项来禁用该宏包的书签内
% 容(bookmark stuff),这样可以节省一点时间。默认 \xoption{final}\ 选项。两个选项都是
% 布尔选项(boolean options),如果没有值,则使用值 |true|。|draft=true| 与 |final=false| 相同。
%
% 除了驱动程序选项(driver options)之外,\xpackage{bookmark}\ 宏包选项都是局部选项(local options)。
% \xoption{draft}\ 选项和 \xoption{final}\ 选项均属于文档类选项(class option)(译者注:文档类选项为全局选项),
% 因此,在 \xpackage{bookmark}\ 宏包中未能看到这两个选项。如果您想使用全局的(global) \xoption{draft}选项
% 来优化第一次 \LaTeX\ 运行(runs),可以在导言(preamble)中引入 \xpackage{ifdraft}\ 宏包并设置 \LaTeX\ 的
% \cs{PassOptionsToPackage},例如:
%\begin{quote}
%\begin{verbatim}
%\documentclass[draft]{article}
%\usepackage{ifdraft}
%\ifdraft{%
%   \PassOptionsToPackage{draft}{bookmark}%
%}{}
%\end{verbatim}
%\end{quote}
%
% \subsubsection{驱动程序选项(Driver options)}
%
% 支持的驱动程序( drivers)包括 \xoption{pdftex}、\xoption{dvips}、\xoption{dvipdfm} (\xoption{xetex})、
% \xoption{vtex}。\hologo{TeX}\ 引擎 \hologo{pdfTeX}、\hologo{XeTeX}、\hologo{VTeX}\ 能被自动检测到。
% 默认的 DVI 驱动程序是 \xoption{dvips}。这可以通过 \cs{BookmarkDriverDefault}\ 在配置
% 文件 \xfile{bookmark.cfg}\ 中进行更改,例如:
% \begin{quote}
% |\def\BookmarkDriverDefault{dvipdfm}|
% \end{quote}
% 当前版本的(current versions)驱动程序使用新的 \LaTeX\ 钩子(\LaTeX-hooks)。如果检测到比
% 2020-10-01 更旧的格式,则将以前驱动程序的冻结版本(frozen versions)作为备份(fallback)。
%
% \paragraph{用 dvipdfmx 打开书签(bookmarks)。}旧版本的宏包有一个 \xoption{dvipdfmx-outline-open}\ 选项
% 可以激活代码,而该代码可以指定一个大纲条目(outline entry)是否打开。该宏包现在假设所有使用的 dvipdfmx 版本都是
% 最新版本,足以理解该代码,因此始终激活该代码。选项本身将被忽略。
%
%
% \subsubsection{布局选项(Layout options)}
%
% \paragraph{字体(Font)选项:}
%
% \begin{description}
% \item[\xoption{bold}:] 如果受 PDF 浏览器(PDF viewer)支持,书签将以粗体字体(bold font)显示(自 PDF 1.4起)。
% \item[\xoption{italic}:] 使用斜体字体(italic font)(自 PDF 1.4起)。
% \end{description}
% \xoption{bold}(粗体) 和 \xoption{italic}(斜体)可以同时使用。而 |false| 值(value)禁用字体选项。
%
% \paragraph{颜色(Color)选项:}
%
% 彩色书签(Colored bookmarks)是 PDF 1.4 的一个特性(feature),并非所有的 PDF 浏览器(PDF viewers)都支持彩色书签。
% \begin{description}
% \item[\xoption{color}:] 这里 color(颜色)可以作为 \xpackage{color}\ 宏包或 \xpackage{xcolor}\ 宏包的
% 颜色规范(color specification)给出。空值(empty value)表示未设置颜色属性。如果未加载 \xpackage{xcolor}\ 宏包,
% 能识别的值(recognized values)只有:
%   \begin{itemize}
%   \item 空值(empty value)表示未设置颜色属性,\\
%         例如:|color={}|
%   \item 颜色模型(color model) rgb 的显式颜色规范(explicit color specification),\\
%         例如,红色(red):|color=[rgb]{1,0,0}|
%   \item 颜色模型(color model)灰(gray)的显式颜色规范(explicit color specification),\\
%         例如,深灰色(dark gray):|color=[gray]{0.25}|
%   \end{itemize}
%   请注意,如果加载了 \xpackage{color}\ 宏包,此限制(restriction)也适用。然而,如果加载了 \xpackage{xcolor}\ 宏包,
%   则可以使用所有颜色规范(color specifications)。
% \end{description}
%
% \subsubsection{动作选项(Action options)}
%
% \begin{description}
% \item[\xoption{dest}:] 目的地名称(destination name)。
% \item[\xoption{page}:] 页码(page number),第一页(first page)为 1。
% \item[\xoption{view}:] 浏览规范(view specification),示例如下:\\
%   |view={FitB}|, |view={FitH 842}|, |view={XYZ 0 100 null}|\ \  一些浏览规范参数(view specification parameters)
%   将数字(numbers)视为具有单位 bp 的参数。它们可以作为普通数字(plain numbers)或在 \cs{calc}\ 内部以
%   长度表达式(length expressions)给出。如果加载了 \xpackage{calc}\ 宏包,则支持该宏包的表达式(expressions)。否则,
%   使用 \hologo{eTeX}\ 的 \cs{dimexpr}。例如:\\
%   |view={FitH \calc{\paperheight-\topmargin-1in}}|\\
%   |view={XYZ 0 \calc{\paperheight} null}|\\
%   注意 \cs{calc}\ 不能用于 |XYZ| 的第三个参数,因为该参数是缩放值(zoom value),而不是长度(length)。

% \item[\xoption{named}:] 已命名的动作(Named action)的名称:\\
%   |FirstPage|(第一页),|LastPage|(最后一页),|NextPage|(下一页),|PrevPage|(前一页)
% \item[\xoption{gotor}:] 外部(external) PDF 文件的名称。
% \item[\xoption{uri}:] URI 规范(URI specification)。
% \item[\xoption{rawaction}:] 原始动作规范(raw action specification)。由于这些规范取决于驱动程序(driver),因此不应使用此选项。
% \end{description}
% 通过分析指定的选项来选择书签的适当动作。动作由不同的选项集(sets of options)区分:
% \begin{quote}
 \begin{tabular}{|@{}r|l@{}|}
%   \hline
%   \ \textbf{动作(Action)}\  & \ \textbf{选项(Options)}\ \\ \hline
%   \ \textsf{GoTo}\  &\  \xoption{dest}\ \\ \hline
%   \ \textsf{GoTo}\  & \ \xoption{page} + \xoption{view}\ \\ \hline
%   \ \textsf{GoToR}\  & \ \xoption{gotor} + \xoption{dest}\ \\ \hline
%   \ \textsf{GoToR}\  & \ \xoption{gotor} + \xoption{page} + \xoption{view}\ \ \ \\ \hline
%   \ \textsf{Named}\  &\  \xoption{named}\ \\ \hline
%   \ \textsf{URI}\  & \ \xoption{uri}\ \\ \hline
% \end{tabular}
% \end{quote}
%
% \paragraph{缺少动作(Missing actions)。}
% 如果动作缺少 \xpackage{bookmark}\ 宏包,则抛出错误消息(error message)。根据驱动程序(driver)
% (\xoption{pdftex}、\xoption{dvips}\ 和好友[friends]),宏包在文档末尾很晚才检测到它。
% 自 2011/04/21 v1.21 版本以后,该宏包尝试打印 \cs{bookmark}\ 的相应出现的行号(line number)和文件名(file name)。
% 然而,\hologo{TeX}\ 确实提供了行号,但不幸的是,文件名是一个秘密(secret)。但该宏包有如下获取文件名的方法:
% \begin{itemize}
% \item 如果 \hologo{LuaTeX} (独立于 DVI 或 PDF 模式)正在运行,则自动使用其 |status.filename|。
% \item 宏包的 \cs{currfile} \cite{currfile}\ 重新定义了 \hologo{LaTeX}\ 的内部结构,以跟踪文件名(file name)。
% 如果加载了该宏包,那么它的 \cs{currfilepath}\ 将被检测到并由 \xpackage{bookmark}\ 自动使用。
% \item 可以通过 \cs{bookmarksetup}\ 或 \cs{bookmark}\ 中的 \xoption{scrfile}\ 选项手动设置(set manually)文件名。
% 但是要小心,手动设置会禁用以前的文件名检测方法。错误的(wrong)或丢失的(missed)文件名设置(file name setting)可能会在错误消息中
% 为您提供错误的源位置(source location)。
% \end{itemize}
%
% \subsubsection{级别选项(Level options)}
%
% 书签条目(bookmark entries)的顺序由 \cs{bookmark}\ 命令的的出现顺序(appearance order)定义。
% 树结构(tree structure)由书签节点(bookmark nodes)的属性 \xoption{level}(级别)构建。
% \xoption{level}\ 的值是整数(integers)。如果书签条目级别的值高于前一个节点,则该条目将成为
% 前一个节点的子(child)节点。差值的绝对值并不重要。
%
% \xpackage{bookmark}\ 宏包能记住全局属性(global property)“current level(当前级别)”中上
% 一个书签条目(previous bookmark entry)的级别。
%
% 级别系统的(level system)行为(behaviour)可以通过以下选项进行配置:
% \begin{description}
% \item[\xoption{level}:]
%    设置级别(level),请参阅上面的说明。如果给出的选项 \xoption{level}\ 没有值,那么将恢复默
%    认行为,即将“当前级别(current level)”用作级别值(level value)。自 2010/10/19 v1.16 版本以来,
%    如果宏 \cs{toclevel@part}、\cs{toclevel@section}\ 被定义过(通过 \xpackage{hyperref}\ 宏包完成,
%    请参阅它的 \xoption{bookmarkdepth}\ 选项),则 \xpackage{bookmark}\ 宏包还支持 |part|、|section| 等名称。
%
% \item[\xoption{rellevel}:]
%    设置相对于前一级别的(previous level)级别。正值表示书签条目成为前一个书签条目的子条目。
% \item[\xoption{keeplevel}:]
%    使用由\xoption{level}\ 或 \xoption{rellevel}\ 设置的级别,但不要更改全局属性“current level(当前级别)”。
%    可以通过设置为 |false| 来禁用该选项。
% \item[\xoption{startatroot}:]
%    此时,书签树(bookmark tree)再次从顶层(top level)开始。下一个书签条目不会作为上一个条目的子条目进行排序。
%    示例场景:文档使用 part。但是,最后几章(last chapters)不应放在最后一部分(last part)下面:
%    \begin{quote}
%\begin{verbatim}
%\documentclass{book}
%[...]
%\begin{document}
%  \part{第一部分}
%    \chapter{第一部分的第1章}
%    [...]
%  \part{第二部分(Second part)}
%    \chapter{第二部分的第1章}
%    [...]
%  \bookmarksetup{startatroot}
%  \chapter{Index}% 不属于第二部分
%\end{document}
%\end{verbatim}
%    \end{quote}
% \end{description}
%
% \subsubsection{样式定义(Style definitions)}
%
% 样式(style)是一组选项设置(option settings)。它可以由宏 \cs{bookmarkdefinestyle}\ 定义,
% 并由它的 \xoption{style}\ 选项使用。
% \begin{declcs}{bookmarkdefinestyle} \M{name} \M{key value list}
% \end{declcs}
% 选项设置(option settings)的 \meta{key value list}(键值列表)被指定为样式名(style \meta{name})。
%
% \begin{description}
% \item[\xoption{style}:]
%   \xoption{style}\ 选项的值是以前定义的样式的名称(name)。现在执行其选项设置(option settings)。
%   选项可以包括 \xoption{style}\ 选项。通过递归调用相同样式的无限递归(endless recursion)被阻止并抛出一个错误。
% \end{description}
%
% \subsubsection{钩子支持(Hook support)}
%
% 处理宏\cs{bookmark}\ 的可选选项(optional options)后,就会调用钩子(hook)。
% \begin{description}
% \item[\xoption{addtohook}:]
%   代码(code)作为该选项的值添加到钩子中。
% \end{description}
%
% \begin{declcs}{bookmarkget} \M{option}
% \end{declcs}
% \cs{bookmarkget}\ 宏提取 \meta{option}\ 选项的最新选项设置(latest option setting)的值。
% 对于布尔选项(boolean option),如果启用布尔选项,则返回 1,否则结果为零。结果数字(resulting numbers)
% 可以直接用于 \cs{ifnum}\ 或 \cs{ifcase}。如果您想要数字 \texttt{0}\ 和 \texttt{1},
% 请在 \cs{bookmarkget}\ 前面加上 \cs{number}\ 作为前缀。\cs{bookmarkget}\ 宏是可展开的(expandable)。
% 如果选项不受支持,则返回空字符串(empty string)。受支持的布尔选项有:
% \begin{quote}
%   \xoption{bold}、
%   \xoption{italic}、
%   \xoption{open}
% \end{quote}
% 其他受支持的选项有:
% \begin{quote}
%   \xoption{depth}、
%   \xoption{dest}、
%   \xoption{color}、
%   \xoption{gotor}、
%   \xoption{level}、
%   \xoption{named}、
%   \xoption{openlevel}、
%   \xoption{page}、
%   \xoption{rawaction}、
%   \xoption{uri}、
%   \xoption{view}、
% \end{quote}
% 另外,以下键(key)是可用的:
% \begin{quote}
%   \xoption{text}
% \end{quote}
% 它返回大纲条目(outline entry)的文本(text)。
%
% \paragraph{选项设置(Option setting)。}
% 在钩子(hook)内部可以使用 \cs{bookmarksetup}\ 设置选项。
%
% \subsection{与 \xpackage{hyperref}\ 的兼容性}
%
% \xpackage{bookmark}\ 宏包自动禁用 \xpackage{hyperref}\ 宏包的书签(bookmarks)。但是,
% \xpackage{bookmark}\ 宏包使用了 \xpackage{hyperref}\ 宏包的一些代码。例如,
% \xpackage{bookmark}\ 宏包重新定义了 \xpackage{hyperref}\ 宏包在 \cs{addcontentsline}\ 命令
% 和其他命令中插入的\cs{Hy@writebookmark}\ 钩子。因此,不应禁用 \xpackage{hyperref}\ 宏包的书签。
%
% \xpackage{bookmark}\ 宏包使用 \xpackage{hyperref}\ 宏包的 \cs{pdfstringdef},且不提供替换(replacement)。
%
% \xpackage{hyperref}\ 宏包的一些选项也能在 \xpackage{bookmark}\ 宏包中实现(implemented):
% \begin{quote}
% \begin{tabular}{|l@{}|l@{}|}
%   \hline
%   \xpackage{hyperref}\ 宏包的选项\  &\ \xpackage{bookmark}\ 宏包的选项\ \ \\ \hline
%   \xoption{bookmarksdepth} &\ \xoption{depth}\\ \hline
%   \xoption{bookmarksopen} & \ \xoption{open}\\ \hline
%   \xoption{bookmarksopenlevel}\ \ \  &\ \xoption{openlevel}\\ \hline
%   \xoption{bookmarksnumbered} \ \ \ &\ \xoption{numbered}\\ \hline
% \end{tabular}
% \end{quote}
%
% 还可以使用以下命令:
% \begin{quote}
%   \cs{pdfbookmark}\\
%   \cs{currentpdfbookmark}\\
%   \cs{subpdfbookmark}\\
%   \cs{belowpdfbookmark}
% \end{quote}
%
% \subsection{在末尾添加书签}
%
% 宏包选项 \xoption{atend}\ 启用以下宏(macro):
% \begin{declcs}{BookmarkAtEnd}
%   \M{stuff}
% \end{declcs}
% \cs{BookmarkAtEnd}\ 宏将 \meta{stuff}\ 放在文档末尾。\meta{stuff}\ 表示书签命令(bookmark commands)。举例:
% \begin{quote}
%\begin{verbatim}
%\usepackage[atend]{bookmark}
%\BookmarkAtEnd{%
%  \bookmarksetup{startatroot}%
%  \bookmark[named=LastPage, level=0]{Last page}%
%}
%\end{verbatim}
% \end{quote}
%
% 或者,可以在 \cs{bookmark}\ 中给出 \xoption{startatroot}\ 选项:
% \begin{quote}
%\begin{verbatim}
%\BookmarkAtEnd{%
%  \bookmark[
%    startatroot,
%    named=LastPage,
%    level=0,
%  ]{Last page}%
%}
%\end{verbatim}
% \end{quote}
%
% \paragraph{备注(Remarks):}
% \begin{itemize}
% \item
%   \cs{BookmarkAtEnd} 隐藏了这样一个事实,即在文档末尾添加书签的方法取决于驱动程序(driver)。
%
%   为此,驱动程序 \xoption{pdftex}\ 使用 \xpackage{atveryend}\ 宏包。如果 \cs{AtEndDocument}\ 太早,
%   最后一个页面(last page)可能不会被发送出去(shipped out)。由于需要 \xext{aux}\ 文件,此驱动程序使
%   用 \cs{AfterLastShipout}。
%
%   其他驱动程序(\xoption{dvipdfm}、\xoption{xetex}、\xoption{vtex})的实现(implementation)
%   取决于 \cs{special},\cs{special}\ 在最后一页之后没有效果。在这种情况下,\xpackage{atenddvi}\ 宏包的
%   \cs{AtEndDvi}\ 有帮助。它将其参数(argument)放在文档的最后一页(last page)。至少需要运行 \hologo{LaTeX}\ 两次,
%   因为最后一页是由引用(reference)检测到的。
%
%   \xoption{dvips}\ 现在使用新的 LaTeX 钩子 \texttt{shipout/lastpage}。
% \item
%   未指定 \cs{BookmarkAtEnd}\ 参数的扩展时间(time of expansion)。这可以立即发生,也可以在文档末尾发生。
% \end{itemize}
%
% \subsection{限制/行动计划}
%
% \begin{itemize}
% \item 支持缺失动作(missing actions)(启动,\dots)。
% \item 对 \xpackage{hyperref}\ 的 \xoption{bookmarkstype}\ 选项进行了更好的设计(design)。
% \end{itemize}
%
% \section{示例(Example)}
%
%    \begin{macrocode}
%<*example>
%    \end{macrocode}
%    \begin{macrocode}
\documentclass{article}
\usepackage{xcolor}[2007/01/21]
\usepackage{hyperref}
\usepackage[
  open,
  openlevel=2,
  atend
]{bookmark}[2019/12/03]

\bookmarksetup{color=blue}

\BookmarkAtEnd{%
  \bookmarksetup{startatroot}%
  \bookmark[named=LastPage, level=0]{End/Last page}%
  \bookmark[named=FirstPage, level=1]{First page}%
}

\begin{document}
\section{First section}
\subsection{Subsection A}
\begin{figure}
  \hypertarget{fig}{}%
  A figure.
\end{figure}
\bookmark[
  rellevel=1,
  keeplevel,
  dest=fig
]{A figure}
\subsection{Subsection B}
\subsubsection{Subsubsection C}
\subsection{Umlauts: \"A\"O\"U\"a\"o\"u\ss}
\newpage
\bookmarksetup{
  bold,
  color=[rgb]{1,0,0}
}
\section{Very important section}
\bookmarksetup{
  italic,
  bold=false,
  color=blue
}
\subsection{Italic section}
\bookmarksetup{
  italic=false
}
\part{Misc}
\section{Diverse}
\subsubsection{Subsubsection, omitting subsection}
\bookmarksetup{
  startatroot
}
\section{Last section outside part}
\subsection{Subsection}
\bookmarksetup{
  color={}
}
\begingroup
  \bookmarksetup{level=0, color=green!80!black}
  \bookmark[named=FirstPage]{First page}
  \bookmark[named=LastPage]{Last page}
  \bookmark[named=PrevPage]{Previous page}
  \bookmark[named=NextPage]{Next page}
\endgroup
\bookmark[
  page=2,
  view=FitH 800
]{Page 2, FitH 800}
\bookmark[
  page=2,
  view=FitBH \calc{\paperheight-\topmargin-1in-\headheight-\headsep}
]{Page 2, FitBH top of text body}
\bookmark[
  uri={http://www.dante.de/},
  color=magenta
]{Dante homepage}
\bookmark[
  gotor={t.pdf},
  page=1,
  view={XYZ 0 1000 null},
  color=cyan!75!black
]{File t.pdf}
\bookmark[named=FirstPage]{First page}
\bookmark[rellevel=1, named=LastPage]{Last page (rellevel=1)}
\bookmark[named=PrevPage]{Previous page}
\bookmark[level=0, named=FirstPage]{First page (level=0)}
\bookmark[
  rellevel=1,
  keeplevel,
  named=LastPage
]{Last page (rellevel=1, keeplevel)}
\bookmark[named=PrevPage]{Previous page}
\end{document}
%    \end{macrocode}
%    \begin{macrocode}
%</example>
%    \end{macrocode}
%
% \StopEventually{
% }
%
% \section{实现(Implementation)}
%
% \subsection{宏包(Package)}
%
%    \begin{macrocode}
%<*package>
\NeedsTeXFormat{LaTeX2e}
\ProvidesPackage{bookmark}%
  [2020-11-06 v1.29 PDF bookmarks (HO)]%
%    \end{macrocode}
%
% \subsubsection{要求(Requirements)}
%
% \paragraph{\hologo{eTeX}.}
%
%    \begin{macro}{\BKM@CalcExpr}
%    \begin{macrocode}
\begingroup\expandafter\expandafter\expandafter\endgroup
\expandafter\ifx\csname numexpr\endcsname\relax
  \def\BKM@CalcExpr#1#2#3#4{%
    \begingroup
      \count@=#2\relax
      \advance\count@ by#3#4\relax
      \edef\x{\endgroup
        \def\noexpand#1{\the\count@}%
      }%
    \x
  }%
\else
  \def\BKM@CalcExpr#1#2#3#4{%
    \edef#1{%
      \the\numexpr#2#3#4\relax
    }%
  }%
\fi
%    \end{macrocode}
%    \end{macro}
%
% \paragraph{\hologo{pdfTeX}\ 的转义功能(escape features)}
%
%    \begin{macro}{\BKM@EscapeName}
%    \begin{macrocode}
\def\BKM@EscapeName#1{%
  \ifx#1\@empty
  \else
    \EdefEscapeName#1#1%
  \fi
}%
%    \end{macrocode}
%    \end{macro}
%    \begin{macro}{\BKM@EscapeString}
%    \begin{macrocode}
\def\BKM@EscapeString#1{%
  \ifx#1\@empty
  \else
    \EdefEscapeString#1#1%
  \fi
}%
%    \end{macrocode}
%    \end{macro}
%    \begin{macro}{\BKM@EscapeHex}
%    \begin{macrocode}
\def\BKM@EscapeHex#1{%
  \ifx#1\@empty
  \else
    \EdefEscapeHex#1#1%
  \fi
}%
%    \end{macrocode}
%    \end{macro}
%    \begin{macro}{\BKM@UnescapeHex}
%    \begin{macrocode}
\def\BKM@UnescapeHex#1{%
  \EdefUnescapeHex#1#1%
}%
%    \end{macrocode}
%    \end{macro}
%
% \paragraph{宏包(Packages)。}
%
% 不要加载由 \xpackage{hyperref}\ 加载的宏包
%    \begin{macrocode}
\RequirePackage{hyperref}[2010/06/18]
%    \end{macrocode}
%
% \subsubsection{宏包选项(Package options)}
%
%    \begin{macrocode}
\SetupKeyvalOptions{family=BKM,prefix=BKM@}
\DeclareLocalOptions{%
  atend,%
  bold,%
  color,%
  depth,%
  dest,%
  draft,%
  final,%
  gotor,%
  italic,%
  keeplevel,%
  level,%
  named,%
  numbered,%
  open,%
  openlevel,%
  page,%
  rawaction,%
  rellevel,%
  srcfile,%
  srcline,%
  startatroot,%
  uri,%
  view,%
}
%    \end{macrocode}
%    \begin{macro}{\bookmarksetup}
%    \begin{macrocode}
\newcommand*{\bookmarksetup}{\kvsetkeys{BKM}}
%    \end{macrocode}
%    \end{macro}
%    \begin{macro}{\BKM@setup}
%    \begin{macrocode}
\def\BKM@setup#1{%
  \bookmarksetup{#1}%
  \ifx\BKM@HookNext\ltx@empty
  \else
    \expandafter\bookmarksetup\expandafter{\BKM@HookNext}%
    \BKM@HookNextClear
  \fi
  \BKM@hook
  \ifBKM@keeplevel
  \else
    \xdef\BKM@currentlevel{\BKM@level}%
  \fi
}
%    \end{macrocode}
%    \end{macro}
%
%    \begin{macro}{\bookmarksetupnext}
%    \begin{macrocode}
\newcommand*{\bookmarksetupnext}[1]{%
  \ltx@GlobalAppendToMacro\BKM@HookNext{,#1}%
}
%    \end{macrocode}
%    \end{macro}
%    \begin{macro}{\BKM@setupnext}
%    \begin{macrocode}
%    \end{macrocode}
%    \end{macro}
%    \begin{macro}{\BKM@HookNextClear}
%    \begin{macrocode}
\def\BKM@HookNextClear{%
  \global\let\BKM@HookNext\ltx@empty
}
%    \end{macrocode}
%    \end{macro}
%    \begin{macro}{\BKM@HookNext}
%    \begin{macrocode}
\BKM@HookNextClear
%    \end{macrocode}
%    \end{macro}
%
%    \begin{macrocode}
\DeclareBoolOption{draft}
\DeclareComplementaryOption{final}{draft}
%    \end{macrocode}
%    \begin{macro}{\BKM@DisableOptions}
%    \begin{macrocode}
\def\BKM@DisableOptions{%
  \DisableKeyvalOption[action=warning,package=bookmark]%
      {BKM}{draft}%
  \DisableKeyvalOption[action=warning,package=bookmark]%
      {BKM}{final}%
}
%    \end{macrocode}
%    \end{macro}
%    \begin{macrocode}
\DeclareBoolOption[\ifHy@bookmarksopen true\else false\fi]{open}
%    \end{macrocode}
%    \begin{macro}{\bookmark@open}
%    \begin{macrocode}
\def\bookmark@open{%
  \ifBKM@open\ltx@one\else\ltx@zero\fi
}
%    \end{macrocode}
%    \end{macro}
%    \begin{macrocode}
\DeclareStringOption[\maxdimen]{openlevel}
%    \end{macrocode}
%    \begin{macro}{\BKM@openlevel}
%    \begin{macrocode}
\edef\BKM@openlevel{\number\@bookmarksopenlevel}
%    \end{macrocode}
%    \end{macro}
%    \begin{macrocode}
%\DeclareStringOption[\c@tocdepth]{depth}
\ltx@IfUndefined{Hy@bookmarksdepth}{%
  \def\BKM@depth{\c@tocdepth}%
}{%
  \let\BKM@depth\Hy@bookmarksdepth
}
\define@key{BKM}{depth}[]{%
  \edef\BKM@param{#1}%
  \ifx\BKM@param\@empty
    \def\BKM@depth{\c@tocdepth}%
  \else
    \ltx@IfUndefined{toclevel@\BKM@param}{%
      \@onelevel@sanitize\BKM@param
      \edef\BKM@temp{\expandafter\@car\BKM@param\@nil}%
      \ifcase 0\expandafter\ifx\BKM@temp-1\fi
              \expandafter\ifnum\expandafter`\BKM@temp>47 %
                \expandafter\ifnum\expandafter`\BKM@temp<58 %
                  1%
                \fi
              \fi
              \relax
        \PackageWarning{bookmark}{%
          Unknown document division name (\BKM@param)\MessageBreak
          for option `depth'%
        }%
      \else
        \BKM@SetDepthOrLevel\BKM@depth\BKM@param
      \fi
    }{%
      \BKM@SetDepthOrLevel\BKM@depth{%
        \csname toclevel@\BKM@param\endcsname
      }%
    }%
  \fi
}
%    \end{macrocode}
%    \begin{macro}{\bookmark@depth}
%    \begin{macrocode}
\def\bookmark@depth{\BKM@depth}
%    \end{macrocode}
%    \end{macro}
%    \begin{macro}{\BKM@SetDepthOrLevel}
%    \begin{macrocode}
\def\BKM@SetDepthOrLevel#1#2{%
  \begingroup
    \setbox\z@=\hbox{%
      \count@=#2\relax
      \expandafter
    }%
  \expandafter\endgroup
  \expandafter\def\expandafter#1\expandafter{\the\count@}%
}
%    \end{macrocode}
%    \end{macro}
%    \begin{macrocode}
\DeclareStringOption[\BKM@currentlevel]{level}[\BKM@currentlevel]
\define@key{BKM}{level}{%
  \edef\BKM@param{#1}%
  \ifx\BKM@param\BKM@MacroCurrentLevel
    \let\BKM@level\BKM@param
  \else
    \ltx@IfUndefined{toclevel@\BKM@param}{%
      \@onelevel@sanitize\BKM@param
      \edef\BKM@temp{\expandafter\@car\BKM@param\@nil}%
      \ifcase 0\expandafter\ifx\BKM@temp-1\fi
              \expandafter\ifnum\expandafter`\BKM@temp>47 %
                \expandafter\ifnum\expandafter`\BKM@temp<58 %
                  1%
                \fi
              \fi
              \relax
        \PackageWarning{bookmark}{%
          Unknown document division name (\BKM@param)\MessageBreak
          for option `level'%
        }%
      \else
        \BKM@SetDepthOrLevel\BKM@level\BKM@param
      \fi
    }{%
      \BKM@SetDepthOrLevel\BKM@level{%
        \csname toclevel@\BKM@param\endcsname
      }%
    }%
  \fi
}
%    \end{macrocode}
%    \begin{macro}{\BKM@MacroCurrentLevel}
%    \begin{macrocode}
\def\BKM@MacroCurrentLevel{\BKM@currentlevel}
%    \end{macrocode}
%    \end{macro}
%    \begin{macrocode}
\DeclareBoolOption{keeplevel}
\DeclareBoolOption{startatroot}
%    \end{macrocode}
%    \begin{macro}{\BKM@startatrootfalse}
%    \begin{macrocode}
\def\BKM@startatrootfalse{%
  \global\let\ifBKM@startatroot\iffalse
}
%    \end{macrocode}
%    \end{macro}
%    \begin{macro}{\BKM@startatroottrue}
%    \begin{macrocode}
\def\BKM@startatroottrue{%
  \global\let\ifBKM@startatroot\iftrue
}
%    \end{macrocode}
%    \end{macro}
%    \begin{macrocode}
\define@key{BKM}{rellevel}{%
  \BKM@CalcExpr\BKM@level{#1}+\BKM@currentlevel
}
%    \end{macrocode}
%    \begin{macro}{\bookmark@level}
%    \begin{macrocode}
\def\bookmark@level{\BKM@level}
%    \end{macrocode}
%    \end{macro}
%    \begin{macro}{\BKM@currentlevel}
%    \begin{macrocode}
\def\BKM@currentlevel{0}
%    \end{macrocode}
%    \end{macro}
%    Make \xpackage{bookmark}'s option \xoption{numbered} an alias
%    for \xpackage{hyperref}'s \xoption{bookmarksnumbered}.
%    \begin{macrocode}
\DeclareBoolOption[%
  \ifHy@bookmarksnumbered true\else false\fi
]{numbered}
\g@addto@macro\BKM@numberedtrue{%
  \let\ifHy@bookmarksnumbered\iftrue
}
\g@addto@macro\BKM@numberedfalse{%
  \let\ifHy@bookmarksnumbered\iffalse
}
\g@addto@macro\Hy@bookmarksnumberedtrue{%
  \let\ifBKM@numbered\iftrue
}
\g@addto@macro\Hy@bookmarksnumberedfalse{%
  \let\ifBKM@numbered\iffalse
}
%    \end{macrocode}
%    \begin{macro}{\bookmark@numbered}
%    \begin{macrocode}
\def\bookmark@numbered{%
  \ifBKM@numbered\ltx@one\else\ltx@zero\fi
}
%    \end{macrocode}
%    \end{macro}
%
% \paragraph{重定义 \xpackage{hyperref}\ 宏包的选项}
%
%    \begin{macro}{\BKM@PatchHyperrefOption}
%    \begin{macrocode}
\def\BKM@PatchHyperrefOption#1{%
  \expandafter\BKM@@PatchHyperrefOption\csname KV@Hyp@#1\endcsname%
}
%    \end{macrocode}
%    \end{macro}
%    \begin{macro}{\BKM@@PatchHyperrefOption}
%    \begin{macrocode}
\def\BKM@@PatchHyperrefOption#1{%
  \expandafter\BKM@@@PatchHyperrefOption#1{##1}\BKM@nil#1%
}
%    \end{macrocode}
%    \end{macro}
%    \begin{macro}{\BKM@@@PatchHyperrefOption}
%    \begin{macrocode}
\def\BKM@@@PatchHyperrefOption#1\BKM@nil#2#3{%
  \def#2##1{%
    #1%
    \bookmarksetup{#3={##1}}%
  }%
}
%    \end{macrocode}
%    \end{macro}
%    \begin{macrocode}
\BKM@PatchHyperrefOption{bookmarksopen}{open}
\BKM@PatchHyperrefOption{bookmarksopenlevel}{openlevel}
\BKM@PatchHyperrefOption{bookmarksdepth}{depth}
%    \end{macrocode}
%
% \paragraph{字体样式(font style)选项。}
%
%    注意:\xpackage{bitset}\ 宏是基于零的,PDF 规范(PDF specifications)以1开头。
%    \begin{macrocode}
\bitsetReset{BKM@FontStyle}%
\define@key{BKM}{italic}[true]{%
  \expandafter\ifx\csname if#1\endcsname\iftrue
    \bitsetSet{BKM@FontStyle}{0}%
  \else
    \bitsetClear{BKM@FontStyle}{0}%
  \fi
}%
\define@key{BKM}{bold}[true]{%
  \expandafter\ifx\csname if#1\endcsname\iftrue
    \bitsetSet{BKM@FontStyle}{1}%
  \else
    \bitsetClear{BKM@FontStyle}{1}%
  \fi
}%
%    \end{macrocode}
%    \begin{macro}{\bookmark@italic}
%    \begin{macrocode}
\def\bookmark@italic{%
  \ifnum\bitsetGet{BKM@FontStyle}{0}=1 \ltx@one\else\ltx@zero\fi
}
%    \end{macrocode}
%    \end{macro}
%    \begin{macro}{\bookmark@bold}
%    \begin{macrocode}
\def\bookmark@bold{%
  \ifnum\bitsetGet{BKM@FontStyle}{1}=1 \ltx@one\else\ltx@zero\fi
}
%    \end{macrocode}
%    \end{macro}
%    \begin{macro}{\BKM@PrintStyle}
%    \begin{macrocode}
\def\BKM@PrintStyle{%
  \bitsetGetDec{BKM@FontStyle}%
}%
%    \end{macrocode}
%    \end{macro}
%
% \paragraph{颜色(color)选项。}
%
%    \begin{macrocode}
\define@key{BKM}{color}{%
  \HyColor@BookmarkColor{#1}\BKM@color{bookmark}{color}%
}
%    \end{macrocode}
%    \begin{macro}{\BKM@color}
%    \begin{macrocode}
\let\BKM@color\@empty
%    \end{macrocode}
%    \end{macro}
%    \begin{macro}{\bookmark@color}
%    \begin{macrocode}
\def\bookmark@color{\BKM@color}
%    \end{macrocode}
%    \end{macro}
%
% \subsubsection{动作(action)选项}
%
%    \begin{macrocode}
\def\BKM@temp#1{%
  \DeclareStringOption{#1}%
  \expandafter\edef\csname bookmark@#1\endcsname{%
    \expandafter\noexpand\csname BKM@#1\endcsname
  }%
}
%    \end{macrocode}
%    \begin{macro}{\bookmark@dest}
%    \begin{macrocode}
\BKM@temp{dest}
%    \end{macrocode}
%    \end{macro}
%    \begin{macro}{\bookmark@named}
%    \begin{macrocode}
\BKM@temp{named}
%    \end{macrocode}
%    \end{macro}
%    \begin{macro}{\bookmark@uri}
%    \begin{macrocode}
\BKM@temp{uri}
%    \end{macrocode}
%    \end{macro}
%    \begin{macro}{\bookmark@gotor}
%    \begin{macrocode}
\BKM@temp{gotor}
%    \end{macrocode}
%    \end{macro}
%    \begin{macro}{\bookmark@rawaction}
%    \begin{macrocode}
\BKM@temp{rawaction}
%    \end{macrocode}
%    \end{macro}
%
%    \begin{macrocode}
\define@key{BKM}{page}{%
  \def\BKM@page{#1}%
  \ifx\BKM@page\@empty
  \else
    \edef\BKM@page{\number\BKM@page}%
    \ifnum\BKM@page>\z@
    \else
      \PackageError{bookmark}{Page must be positive}\@ehc
      \def\BKM@page{1}%
    \fi
  \fi
}
%    \end{macrocode}
%    \begin{macro}{\BKM@page}
%    \begin{macrocode}
\let\BKM@page\@empty
%    \end{macrocode}
%    \end{macro}
%    \begin{macro}{\bookmark@page}
%    \begin{macrocode}
\def\bookmark@page{\BKM@@page}
%    \end{macrocode}
%    \end{macro}
%
%    \begin{macrocode}
\define@key{BKM}{view}{%
  \BKM@CheckView{#1}%
}
%    \end{macrocode}
%    \begin{macro}{\BKM@view}
%    \begin{macrocode}
\let\BKM@view\@empty
%    \end{macrocode}
%    \end{macro}
%    \begin{macro}{\bookmark@view}
%    \begin{macrocode}
\def\bookmark@view{\BKM@view}
%    \end{macrocode}
%    \end{macro}
%    \begin{macro}{BKM@CheckView}
%    \begin{macrocode}
\def\BKM@CheckView#1{%
  \BKM@CheckViewType#1 \@nil
}
%    \end{macrocode}
%    \end{macro}
%    \begin{macro}{\BKM@CheckViewType}
%    \begin{macrocode}
\def\BKM@CheckViewType#1 #2\@nil{%
  \def\BKM@type{#1}%
  \@onelevel@sanitize\BKM@type
  \BKM@TestViewType{Fit}{}%
  \BKM@TestViewType{FitB}{}%
  \BKM@TestViewType{FitH}{%
    \BKM@CheckParam#2 \@nil{top}%
  }%
  \BKM@TestViewType{FitBH}{%
    \BKM@CheckParam#2 \@nil{top}%
  }%
  \BKM@TestViewType{FitV}{%
    \BKM@CheckParam#2 \@nil{bottom}%
  }%
  \BKM@TestViewType{FitBV}{%
    \BKM@CheckParam#2 \@nil{bottom}%
  }%
  \BKM@TestViewType{FitR}{%
    \BKM@CheckRect{#2}{ }%
  }%
  \BKM@TestViewType{XYZ}{%
    \BKM@CheckXYZ{#2}{ }%
  }%
  \@car{%
    \PackageError{bookmark}{%
      Unknown view type `\BKM@type',\MessageBreak
      using `FitH' instead%
    }\@ehc
    \def\BKM@view{FitH}%
  }%
  \@nil
}
%    \end{macrocode}
%    \end{macro}
%    \begin{macro}{\BKM@TestViewType}
%    \begin{macrocode}
\def\BKM@TestViewType#1{%
  \def\BKM@temp{#1}%
  \@onelevel@sanitize\BKM@temp
  \ifx\BKM@type\BKM@temp
    \let\BKM@view\BKM@temp
    \expandafter\@car
  \else
    \expandafter\@gobble
  \fi
}
%    \end{macrocode}
%    \end{macro}
%    \begin{macro}{BKM@CheckParam}
%    \begin{macrocode}
\def\BKM@CheckParam#1 #2\@nil#3{%
  \def\BKM@param{#1}%
  \ifx\BKM@param\@empty
    \PackageWarning{bookmark}{%
      Missing parameter (#3) for `\BKM@type',\MessageBreak
      using 0%
    }%
    \def\BKM@param{0}%
  \else
    \BKM@CalcParam
  \fi
  \edef\BKM@view{\BKM@view\space\BKM@param}%
}
%    \end{macrocode}
%    \end{macro}
%    \begin{macro}{BKM@CheckRect}
%    \begin{macrocode}
\def\BKM@CheckRect#1#2{%
  \BKM@@CheckRect#1#2#2#2#2\@nil
}
%    \end{macrocode}
%    \end{macro}
%    \begin{macro}{\BKM@@CheckRect}
%    \begin{macrocode}
\def\BKM@@CheckRect#1 #2 #3 #4 #5\@nil{%
  \def\BKM@temp{0}%
  \def\BKM@param{#1}%
  \ifx\BKM@param\@empty
    \def\BKM@param{0}%
    \def\BKM@temp{1}%
  \else
    \BKM@CalcParam
  \fi
  \edef\BKM@view{\BKM@view\space\BKM@param}%
  \def\BKM@param{#2}%
  \ifx\BKM@param\@empty
    \def\BKM@param{0}%
    \def\BKM@temp{1}%
  \else
    \BKM@CalcParam
  \fi
  \edef\BKM@view{\BKM@view\space\BKM@param}%
  \def\BKM@param{#3}%
  \ifx\BKM@param\@empty
    \def\BKM@param{0}%
    \def\BKM@temp{1}%
  \else
    \BKM@CalcParam
  \fi
  \edef\BKM@view{\BKM@view\space\BKM@param}%
  \def\BKM@param{#4}%
  \ifx\BKM@param\@empty
    \def\BKM@param{0}%
    \def\BKM@temp{1}%
  \else
    \BKM@CalcParam
  \fi
  \edef\BKM@view{\BKM@view\space\BKM@param}%
  \ifnum\BKM@temp>\z@
    \PackageWarning{bookmark}{Missing parameters for `\BKM@type'}%
  \fi
}
%    \end{macrocode}
%    \end{macro}
%    \begin{macro}{\BKM@CheckXYZ}
%    \begin{macrocode}
\def\BKM@CheckXYZ#1#2{%
  \BKM@@CheckXYZ#1#2#2#2\@nil
}
%    \end{macrocode}
%    \end{macro}
%    \begin{macro}{\BKM@@CheckXYZ}
%    \begin{macrocode}
\def\BKM@@CheckXYZ#1 #2 #3 #4\@nil{%
  \def\BKM@param{#1}%
  \let\BKM@temp\BKM@param
  \@onelevel@sanitize\BKM@temp
  \ifx\BKM@param\@empty
    \let\BKM@param\BKM@null
  \else
    \ifx\BKM@temp\BKM@null
    \else
      \BKM@CalcParam
    \fi
  \fi
  \edef\BKM@view{\BKM@view\space\BKM@param}%
  \def\BKM@param{#2}%
  \let\BKM@temp\BKM@param
  \@onelevel@sanitize\BKM@temp
  \ifx\BKM@param\@empty
    \let\BKM@param\BKM@null
  \else
    \ifx\BKM@temp\BKM@null
    \else
      \BKM@CalcParam
    \fi
  \fi
  \edef\BKM@view{\BKM@view\space\BKM@param}%
  \def\BKM@param{#3}%
  \ifx\BKM@param\@empty
    \let\BKM@param\BKM@null
  \fi
  \edef\BKM@view{\BKM@view\space\BKM@param}%
}
%    \end{macrocode}
%    \end{macro}
%    \begin{macro}{\BKM@null}
%    \begin{macrocode}
\def\BKM@null{null}
\@onelevel@sanitize\BKM@null
%    \end{macrocode}
%    \end{macro}
%
%    \begin{macro}{\BKM@CalcParam}
%    \begin{macrocode}
\def\BKM@CalcParam{%
  \begingroup
  \let\calc\@firstofone
  \expandafter\BKM@@CalcParam\BKM@param\@empty\@empty\@nil
}
%    \end{macrocode}
%    \end{macro}
%    \begin{macro}{\BKM@@CalcParam}
%    \begin{macrocode}
\def\BKM@@CalcParam#1#2#3\@nil{%
  \ifx\calc#1%
    \@ifundefined{calc@assign@dimen}{%
      \@ifundefined{dimexpr}{%
        \setlength{\dimen@}{#2}%
      }{%
        \setlength{\dimen@}{\dimexpr#2\relax}%
      }%
    }{%
      \setlength{\dimen@}{#2}%
    }%
    \dimen@.99626\dimen@
    \edef\BKM@param{\strip@pt\dimen@}%
    \expandafter\endgroup
    \expandafter\def\expandafter\BKM@param\expandafter{\BKM@param}%
  \else
    \endgroup
  \fi
}
%    \end{macrocode}
%    \end{macro}
%
% \subsubsection{\xoption{atend}\ 选项}
%
%    \begin{macrocode}
\DeclareBoolOption{atend}
\g@addto@macro\BKM@DisableOptions{%
  \DisableKeyvalOption[action=warning,package=bookmark]%
      {BKM}{atend}%
}
%    \end{macrocode}
%
% \subsubsection{\xoption{style}\ 选项}
%
%    \begin{macro}{\bookmarkdefinestyle}
%    \begin{macrocode}
\newcommand*{\bookmarkdefinestyle}[2]{%
  \@ifundefined{BKM@style@#1}{%
  }{%
    \PackageInfo{bookmark}{Redefining style `#1'}%
  }%
  \@namedef{BKM@style@#1}{#2}%
}
%    \end{macrocode}
%    \end{macro}
%    \begin{macrocode}
\define@key{BKM}{style}{%
  \BKM@StyleCall{#1}%
}
\newif\ifBKM@ok
%    \end{macrocode}
%    \begin{macro}{\BKM@StyleCall}
%    \begin{macrocode}
\def\BKM@StyleCall#1{%
  \@ifundefined{BKM@style@#1}{%
    \PackageWarning{bookmark}{%
      Ignoring unknown style `#1'%
    }%
  }{%
%    \end{macrocode}
%    检查样式堆栈(style stack)。
%    \begin{macrocode}
    \BKM@oktrue
    \edef\BKM@StyleCurrent{#1}%
    \@onelevel@sanitize\BKM@StyleCurrent
    \let\BKM@StyleEntry\BKM@StyleEntryCheck
    \BKM@StyleStack
    \ifBKM@ok
      \expandafter\@firstofone
    \else
      \PackageError{bookmark}{%
        Ignoring recursive call of style `\BKM@StyleCurrent'%
      }\@ehc
      \expandafter\@gobble
    \fi
    {%
%    \end{macrocode}
%    在堆栈上推送当前样式(Push current style on stack)。
%    \begin{macrocode}
      \let\BKM@StyleEntry\relax
      \edef\BKM@StyleStack{%
        \BKM@StyleEntry{\BKM@StyleCurrent}%
        \BKM@StyleStack
      }%
%    \end{macrocode}
%   调用样式(Call style)。
%    \begin{macrocode}
      \expandafter\expandafter\expandafter\bookmarksetup
      \expandafter\expandafter\expandafter{%
        \csname BKM@style@\BKM@StyleCurrent\endcsname
      }%
%    \end{macrocode}
%    从堆栈中弹出当前样式(Pop current style from stack)。
%    \begin{macrocode}
      \BKM@StyleStackPop
    }%
  }%
}
%    \end{macrocode}
%    \end{macro}
%    \begin{macro}{\BKM@StyleStackPop}
%    \begin{macrocode}
\def\BKM@StyleStackPop{%
  \let\BKM@StyleEntry\relax
  \edef\BKM@StyleStack{%
    \expandafter\@gobbletwo\BKM@StyleStack
  }%
}
%    \end{macrocode}
%    \end{macro}
%    \begin{macro}{\BKM@StyleEntryCheck}
%    \begin{macrocode}
\def\BKM@StyleEntryCheck#1{%
  \def\BKM@temp{#1}%
  \ifx\BKM@temp\BKM@StyleCurrent
    \BKM@okfalse
  \fi
}
%    \end{macrocode}
%    \end{macro}
%    \begin{macro}{\BKM@StyleStack}
%    \begin{macrocode}
\def\BKM@StyleStack{}
%    \end{macrocode}
%    \end{macro}
%
% \subsubsection{源文件位置(source file location)选项}
%
%    \begin{macrocode}
\DeclareStringOption{srcline}
\DeclareStringOption{srcfile}
%    \end{macrocode}
%
% \subsubsection{钩子支持(Hook support)}
%
%    \begin{macro}{\BKM@hook}
%    \begin{macrocode}
\def\BKM@hook{}
%    \end{macrocode}
%    \end{macro}
%    \begin{macrocode}
\define@key{BKM}{addtohook}{%
  \ltx@LocalAppendToMacro\BKM@hook{#1}%
}
%    \end{macrocode}
%
%    \begin{macro}{bookmarkget}
%    \begin{macrocode}
\newcommand*{\bookmarkget}[1]{%
  \romannumeral0%
  \ltx@ifundefined{bookmark@#1}{%
    \ltx@space
  }{%
    \expandafter\expandafter\expandafter\ltx@space
    \csname bookmark@#1\endcsname
  }%
}
%    \end{macrocode}
%    \end{macro}
%
% \subsubsection{设置和加载驱动程序}
%
% \paragraph{检测驱动程序。}
%
%    \begin{macro}{\BKM@DefineDriverKey}
%    \begin{macrocode}
\def\BKM@DefineDriverKey#1{%
  \define@key{BKM}{#1}[]{%
    \def\BKM@driver{#1}%
  }%
  \g@addto@macro\BKM@DisableOptions{%
    \DisableKeyvalOption[action=warning,package=bookmark]%
        {BKM}{#1}%
  }%
}
%    \end{macrocode}
%    \end{macro}
%    \begin{macrocode}
\BKM@DefineDriverKey{pdftex}
\BKM@DefineDriverKey{dvips}
\BKM@DefineDriverKey{dvipdfm}
\BKM@DefineDriverKey{dvipdfmx}
\BKM@DefineDriverKey{xetex}
\BKM@DefineDriverKey{vtex}
\define@key{BKM}{dvipdfmx-outline-open}[true]{%
 \PackageWarning{bookmark}{Option 'dvipdfmx-outline-open' is obsolete
   and ignored}{}}
%    \end{macrocode}
%    \begin{macro}{\bookmark@driver}
%    \begin{macrocode}
\def\bookmark@driver{\BKM@driver}
%    \end{macrocode}
%    \end{macro}
%    \begin{macrocode}
\InputIfFileExists{bookmark.cfg}{}{}
%    \end{macrocode}
%    \begin{macro}{\BookmarkDriverDefault}
%    \begin{macrocode}
\providecommand*{\BookmarkDriverDefault}{dvips}
%    \end{macrocode}
%    \end{macro}
%    \begin{macro}{\BKM@driver}
% Lua\TeX\ 和 pdf\TeX\ 共享驱动程序。
%    \begin{macrocode}
\ifpdf
  \def\BKM@driver{pdftex}%
  \ifx\pdfoutline\@undefined
    \ifx\pdfextension\@undefined\else
      \protected\def\pdfoutline{\pdfextension outline }
    \fi
  \fi
\else
  \ifxetex
    \def\BKM@driver{dvipdfm}%
  \else
    \ifvtex
      \def\BKM@driver{vtex}%
    \else
      \edef\BKM@driver{\BookmarkDriverDefault}%
    \fi
  \fi
\fi
%    \end{macrocode}
%    \end{macro}
%
% \paragraph{过程选项(Process options)。}
%
%    \begin{macrocode}
\ProcessKeyvalOptions*
\BKM@DisableOptions
%    \end{macrocode}
%
% \paragraph{\xoption{draft}\ 选项}
%
%    \begin{macrocode}
\ifBKM@draft
  \PackageWarningNoLine{bookmark}{Draft mode on}%
  \let\bookmarksetup\ltx@gobble
  \let\BookmarkAtEnd\ltx@gobble
  \let\bookmarkdefinestyle\ltx@gobbletwo
  \let\bookmarkget\ltx@gobble
  \let\pdfbookmark\ltx@undefined
  \newcommand*{\pdfbookmark}[3][]{}%
  \let\currentpdfbookmark\ltx@gobbletwo
  \let\subpdfbookmark\ltx@gobbletwo
  \let\belowpdfbookmark\ltx@gobbletwo
  \newcommand*{\bookmark}[2][]{}%
  \renewcommand*{\Hy@writebookmark}[5]{}%
  \let\ReadBookmarks\relax
  \let\BKM@DefGotoNameAction\ltx@gobbletwo % package `hypdestopt'
  \expandafter\endinput
\fi
%    \end{macrocode}
%
% \paragraph{验证和加载驱动程序。}
%
%    \begin{macrocode}
\def\BKM@temp{dvipdfmx}%
\ifx\BKM@temp\BKM@driver
  \def\BKM@driver{dvipdfm}%
\fi
\def\BKM@temp{pdftex}%
\ifpdf
  \ifx\BKM@temp\BKM@driver
  \else
    \PackageWarningNoLine{bookmark}{%
      Wrong driver `\BKM@driver', using `pdftex' instead%
    }%
    \let\BKM@driver\BKM@temp
  \fi
\else
  \ifx\BKM@temp\BKM@driver
    \PackageError{bookmark}{%
      Wrong driver, pdfTeX is not running in PDF mode.\MessageBreak
      Package loading is aborted%
    }\@ehc
    \expandafter\expandafter\expandafter\endinput
  \fi
  \def\BKM@temp{dvipdfm}%
  \ifxetex
    \ifx\BKM@temp\BKM@driver
    \else
      \PackageWarningNoLine{bookmark}{%
        Wrong driver `\BKM@driver',\MessageBreak
        using `dvipdfm' for XeTeX instead%
      }%
      \let\BKM@driver\BKM@temp
    \fi
  \else
    \def\BKM@temp{vtex}%
    \ifvtex
      \ifx\BKM@temp\BKM@driver
      \else
        \PackageWarningNoLine{bookmark}{%
          Wrong driver `\BKM@driver',\MessageBreak
          using `vtex' for VTeX instead%
        }%
        \let\BKM@driver\BKM@temp
      \fi
    \else
      \ifx\BKM@temp\BKM@driver
        \PackageError{bookmark}{%
          Wrong driver, VTeX is not running in PDF mode.\MessageBreak
          Package loading is aborted%
        }\@ehc
        \expandafter\expandafter\expandafter\endinput
      \fi
    \fi
  \fi
\fi
\providecommand\IfFormatAtLeastTF{\@ifl@t@r\fmtversion}
\IfFormatAtLeastTF{2020/10/01}{}{\edef\BKM@driver{\BKM@driver-2019-12-03}}
\InputIfFileExists{bkm-\BKM@driver.def}{}{%
  \PackageError{bookmark}{%
    Unsupported driver `\BKM@driver'.\MessageBreak
    Package loading is aborted%
  }\@ehc
  \endinput
}
%    \end{macrocode}
%
% \subsubsection{与 \xpackage{hyperref}\ 的兼容性}
%
%    \begin{macro}{\pdfbookmark}
%    \begin{macrocode}
\let\pdfbookmark\ltx@undefined
\newcommand*{\pdfbookmark}[3][0]{%
  \bookmark[level=#1,dest={#3.#1}]{#2}%
  \hyper@anchorstart{#3.#1}\hyper@anchorend
}
%    \end{macrocode}
%    \end{macro}
%    \begin{macro}{\currentpdfbookmark}
%    \begin{macrocode}
\def\currentpdfbookmark{%
  \pdfbookmark[\BKM@currentlevel]%
}
%    \end{macrocode}
%    \end{macro}
%    \begin{macro}{\subpdfbookmark}
%    \begin{macrocode}
\def\subpdfbookmark{%
  \BKM@CalcExpr\BKM@CalcResult\BKM@currentlevel+1%
  \expandafter\pdfbookmark\expandafter[\BKM@CalcResult]%
}
%    \end{macrocode}
%    \end{macro}
%    \begin{macro}{\belowpdfbookmark}
%    \begin{macrocode}
\def\belowpdfbookmark#1#2{%
  \xdef\BKM@gtemp{\number\BKM@currentlevel}%
  \subpdfbookmark{#1}{#2}%
  \global\let\BKM@currentlevel\BKM@gtemp
}
%    \end{macrocode}
%    \end{macro}
%
%    节号(section number)、文本(text)、标签(label)、级别(level)、文件(file)
%    \begin{macro}{\Hy@writebookmark}
%    \begin{macrocode}
\def\Hy@writebookmark#1#2#3#4#5{%
  \ifnum#4>\BKM@depth\relax
  \else
    \def\BKM@type{#5}%
    \ifx\BKM@type\Hy@bookmarkstype
      \begingroup
        \ifBKM@numbered
          \let\numberline\Hy@numberline
          \let\booknumberline\Hy@numberline
          \let\partnumberline\Hy@numberline
          \let\chapternumberline\Hy@numberline
        \else
          \let\numberline\@gobble
          \let\booknumberline\@gobble
          \let\partnumberline\@gobble
          \let\chapternumberline\@gobble
        \fi
        \bookmark[level=#4,dest={\HyperDestNameFilter{#3}}]{#2}%
      \endgroup
    \fi
  \fi
}
%    \end{macrocode}
%    \end{macro}
%
%    \begin{macro}{\ReadBookmarks}
%    \begin{macrocode}
\let\ReadBookmarks\relax
%    \end{macrocode}
%    \end{macro}
%
%    \begin{macrocode}
%</package>
%    \end{macrocode}
%
% \subsection{dvipdfm 的驱动程序}
%
%    \begin{macrocode}
%<*dvipdfm>
\NeedsTeXFormat{LaTeX2e}
\ProvidesFile{bkm-dvipdfm.def}%
  [2020-11-06 v1.29 bookmark driver for dvipdfm (HO)]%
%    \end{macrocode}
%
%    \begin{macro}{\BKM@id}
%    \begin{macrocode}
\newcount\BKM@id
\BKM@id=\z@
%    \end{macrocode}
%    \end{macro}
%
%    \begin{macro}{\BKM@0}
%    \begin{macrocode}
\@namedef{BKM@0}{000}
%    \end{macrocode}
%    \end{macro}
%    \begin{macro}{\ifBKM@sw}
%    \begin{macrocode}
\newif\ifBKM@sw
%    \end{macrocode}
%    \end{macro}
%
%    \begin{macro}{\bookmark}
%    \begin{macrocode}
\newcommand*{\bookmark}[2][]{%
  \if@filesw
    \begingroup
      \def\bookmark@text{#2}%
      \BKM@setup{#1}%
      \edef\BKM@prev{\the\BKM@id}%
      \global\advance\BKM@id\@ne
      \BKM@swtrue
      \@whilesw\ifBKM@sw\fi{%
        \def\BKM@abslevel{1}%
        \ifnum\ifBKM@startatroot\z@\else\BKM@prev\fi=\z@
          \BKM@startatrootfalse
          \expandafter\xdef\csname BKM@\the\BKM@id\endcsname{%
            0{\BKM@level}\BKM@abslevel
          }%
          \BKM@swfalse
        \else
          \expandafter\expandafter\expandafter\BKM@getx
              \csname BKM@\BKM@prev\endcsname
          \ifnum\BKM@level>\BKM@x@level\relax
            \BKM@CalcExpr\BKM@abslevel\BKM@x@abslevel+1%
            \expandafter\xdef\csname BKM@\the\BKM@id\endcsname{%
              {\BKM@prev}{\BKM@level}\BKM@abslevel
            }%
            \BKM@swfalse
          \else
            \let\BKM@prev\BKM@x@parent
          \fi
        \fi
      }%
      \csname HyPsd@XeTeXBigCharstrue\endcsname
      \pdfstringdef\BKM@title{\bookmark@text}%
      \edef\BKM@FLAGS{\BKM@PrintStyle}%
      \let\BKM@action\@empty
      \ifx\BKM@gotor\@empty
        \ifx\BKM@dest\@empty
          \ifx\BKM@named\@empty
            \ifx\BKM@rawaction\@empty
              \ifx\BKM@uri\@empty
                \ifx\BKM@page\@empty
                  \PackageError{bookmark}{Missing action}\@ehc
                  \edef\BKM@action{/Dest[@page1/Fit]}%
                \else
                  \ifx\BKM@view\@empty
                    \def\BKM@view{Fit}%
                  \fi
                  \edef\BKM@action{/Dest[@page\BKM@page/\BKM@view]}%
                \fi
              \else
                \BKM@EscapeString\BKM@uri
                \edef\BKM@action{%
                  /A<<%
                    /S/URI%
                    /URI(\BKM@uri)%
                  >>%
                }%
              \fi
            \else
              \edef\BKM@action{/A<<\BKM@rawaction>>}%
            \fi
          \else
            \BKM@EscapeName\BKM@named
            \edef\BKM@action{%
              /A<</S/Named/N/\BKM@named>>%
            }%
          \fi
        \else
          \BKM@EscapeString\BKM@dest
          \edef\BKM@action{%
            /A<<%
              /S/GoTo%
              /D(\BKM@dest)%
            >>%
          }%
        \fi
      \else
        \ifx\BKM@dest\@empty
          \ifx\BKM@page\@empty
            \def\BKM@page{0}%
          \else
            \BKM@CalcExpr\BKM@page\BKM@page-1%
          \fi
          \ifx\BKM@view\@empty
            \def\BKM@view{Fit}%
          \fi
          \edef\BKM@action{/D[\BKM@page/\BKM@view]}%
        \else
          \BKM@EscapeString\BKM@dest
          \edef\BKM@action{/D(\BKM@dest)}%
        \fi
        \BKM@EscapeString\BKM@gotor
        \edef\BKM@action{%
          /A<<%
            /S/GoToR%
            /F(\BKM@gotor)%
            \BKM@action
          >>%
        }%
      \fi
      \special{pdf:%
        out
              [%
              \ifBKM@open
                \ifnum\BKM@level<%
                    \expandafter\ltx@firstofone\expandafter
                    {\number\BKM@openlevel} %
                \else
                  -%
                \fi
              \else
                -%
              \fi
              ] %
            \BKM@abslevel
        <<%
          /Title(\BKM@title)%
          \ifx\BKM@color\@empty
          \else
            /C[\BKM@color]%
          \fi
          \ifnum\BKM@FLAGS>\z@
            /F \BKM@FLAGS
          \fi
          \BKM@action
        >>%
      }%
    \endgroup
  \fi
}
%    \end{macrocode}
%    \end{macro}
%    \begin{macro}{\BKM@getx}
%    \begin{macrocode}
\def\BKM@getx#1#2#3{%
  \def\BKM@x@parent{#1}%
  \def\BKM@x@level{#2}%
  \def\BKM@x@abslevel{#3}%
}
%    \end{macrocode}
%    \end{macro}
%
%    \begin{macrocode}
%</dvipdfm>
%    \end{macrocode}
%
% \subsection{\hologo{VTeX}\ 的驱动程序}
%
%    \begin{macrocode}
%<*vtex>
\NeedsTeXFormat{LaTeX2e}
\ProvidesFile{bkm-vtex.def}%
  [2020-11-06 v1.29 bookmark driver for VTeX (HO)]%
%    \end{macrocode}
%
%    \begin{macrocode}
\ifvtexpdf
\else
  \PackageWarningNoLine{bookmark}{%
    The VTeX driver only supports PDF mode%
  }%
\fi
%    \end{macrocode}
%
%    \begin{macro}{\BKM@id}
%    \begin{macrocode}
\newcount\BKM@id
\BKM@id=\z@
%    \end{macrocode}
%    \end{macro}
%
%    \begin{macro}{\BKM@0}
%    \begin{macrocode}
\@namedef{BKM@0}{00}
%    \end{macrocode}
%    \end{macro}
%    \begin{macro}{\ifBKM@sw}
%    \begin{macrocode}
\newif\ifBKM@sw
%    \end{macrocode}
%    \end{macro}
%
%    \begin{macro}{\bookmark}
%    \begin{macrocode}
\newcommand*{\bookmark}[2][]{%
  \if@filesw
    \begingroup
      \def\bookmark@text{#2}%
      \BKM@setup{#1}%
      \edef\BKM@prev{\the\BKM@id}%
      \global\advance\BKM@id\@ne
      \BKM@swtrue
      \@whilesw\ifBKM@sw\fi{%
        \ifnum\ifBKM@startatroot\z@\else\BKM@prev\fi=\z@
          \BKM@startatrootfalse
          \def\BKM@parent{0}%
          \expandafter\xdef\csname BKM@\the\BKM@id\endcsname{%
            0{\BKM@level}%
          }%
          \BKM@swfalse
        \else
          \expandafter\expandafter\expandafter\BKM@getx
              \csname BKM@\BKM@prev\endcsname
          \ifnum\BKM@level>\BKM@x@level\relax
            \let\BKM@parent\BKM@prev
            \expandafter\xdef\csname BKM@\the\BKM@id\endcsname{%
              {\BKM@prev}{\BKM@level}%
            }%
            \BKM@swfalse
          \else
            \let\BKM@prev\BKM@x@parent
          \fi
        \fi
      }%
      \pdfstringdef\BKM@title{\bookmark@text}%
      \BKM@vtex@title
      \edef\BKM@FLAGS{\BKM@PrintStyle}%
      \let\BKM@action\@empty
      \ifx\BKM@gotor\@empty
        \ifx\BKM@dest\@empty
          \ifx\BKM@named\@empty
            \ifx\BKM@rawaction\@empty
              \ifx\BKM@uri\@empty
                \ifx\BKM@page\@empty
                  \PackageError{bookmark}{Missing action}\@ehc
                  \def\BKM@action{!1}%
                \else
                  \edef\BKM@action{!\BKM@page}%
                \fi
              \else
                \BKM@EscapeString\BKM@uri
                \edef\BKM@action{%
                  <u=%
                    /S/URI%
                    /URI(\BKM@uri)%
                  >%
                }%
              \fi
            \else
              \edef\BKM@action{<u=\BKM@rawaction>}%
            \fi
          \else
            \BKM@EscapeName\BKM@named
            \edef\BKM@action{%
              <u=%
                /S/Named%
                /N/\BKM@named
              >%
            }%
          \fi
        \else
          \BKM@EscapeString\BKM@dest
          \edef\BKM@action{\BKM@dest}%
        \fi
      \else
        \ifx\BKM@dest\@empty
          \ifx\BKM@page\@empty
            \def\BKM@page{1}%
          \fi
          \ifx\BKM@view\@empty
            \def\BKM@view{Fit}%
          \fi
          \edef\BKM@action{/D[\BKM@page/\BKM@view]}%
        \else
          \BKM@EscapeString\BKM@dest
          \edef\BKM@action{/D(\BKM@dest)}%
        \fi
        \BKM@EscapeString\BKM@gotor
        \edef\BKM@action{%
          <u=%
            /S/GoToR%
            /F(\BKM@gotor)%
            \BKM@action
          >>%
        }%
      \fi
      \ifx\BKM@color\@empty
        \let\BKM@RGBcolor\@empty
      \else
        \expandafter\BKM@toRGB\BKM@color\@nil
      \fi
      \special{%
        !outline \BKM@action;%
        p=\BKM@parent,%
        i=\number\BKM@id,%
        s=%
          \ifBKM@open
            \ifnum\BKM@level<\BKM@openlevel
              o%
            \else
              c%
            \fi
          \else
            c%
          \fi,%
        \ifx\BKM@RGBcolor\@empty
        \else
          c=\BKM@RGBcolor,%
        \fi
        \ifnum\BKM@FLAGS>\z@
          f=\BKM@FLAGS,%
        \fi
        t=\BKM@title
      }%
    \endgroup
  \fi
}
%    \end{macrocode}
%    \end{macro}
%    \begin{macro}{\BKM@getx}
%    \begin{macrocode}
\def\BKM@getx#1#2{%
  \def\BKM@x@parent{#1}%
  \def\BKM@x@level{#2}%
}
%    \end{macrocode}
%    \end{macro}
%    \begin{macro}{\BKM@toRGB}
%    \begin{macrocode}
\def\BKM@toRGB#1 #2 #3\@nil{%
  \let\BKM@RGBcolor\@empty
  \BKM@toRGBComponent{#1}%
  \BKM@toRGBComponent{#2}%
  \BKM@toRGBComponent{#3}%
}
%    \end{macrocode}
%    \end{macro}
%    \begin{macro}{\BKM@toRGBComponent}
%    \begin{macrocode}
\def\BKM@toRGBComponent#1{%
  \dimen@=#1pt\relax
  \ifdim\dimen@>\z@
    \ifdim\dimen@<\p@
      \dimen@=255\dimen@
      \advance\dimen@ by 32768sp\relax
      \divide\dimen@ by 65536\relax
      \dimen@ii=\dimen@
      \divide\dimen@ii by 16\relax
      \edef\BKM@RGBcolor{%
        \BKM@RGBcolor
        \BKM@toHexDigit\dimen@ii
      }%
      \dimen@ii=16\dimen@ii
      \advance\dimen@-\dimen@ii
      \edef\BKM@RGBcolor{%
        \BKM@RGBcolor
        \BKM@toHexDigit\dimen@
      }%
    \else
      \edef\BKM@RGBcolor{\BKM@RGBcolor FF}%
    \fi
  \else
    \edef\BKM@RGBcolor{\BKM@RGBcolor00}%
  \fi
}
%    \end{macrocode}
%    \end{macro}
%    \begin{macro}{\BKM@toHexDigit}
%    \begin{macrocode}
\def\BKM@toHexDigit#1{%
  \ifcase\expandafter\@firstofone\expandafter{\number#1} %
    0\or 1\or 2\or 3\or 4\or 5\or 6\or 7\or
    8\or 9\or A\or B\or C\or D\or E\or F%
  \fi
}
%    \end{macrocode}
%    \end{macro}
%    \begin{macrocode}
\begingroup
  \catcode`\|=0 %
  \catcode`\\=12 %
%    \end{macrocode}
%    \begin{macro}{\BKM@vtex@title}
%    \begin{macrocode}
  |gdef|BKM@vtex@title{%
    |@onelevel@sanitize|BKM@title
    |edef|BKM@title{|expandafter|BKM@vtex@leftparen|BKM@title\(|@nil}%
    |edef|BKM@title{|expandafter|BKM@vtex@rightparen|BKM@title\)|@nil}%
    |edef|BKM@title{|expandafter|BKM@vtex@zero|BKM@title\0|@nil}%
    |edef|BKM@title{|expandafter|BKM@vtex@one|BKM@title\1|@nil}%
    |edef|BKM@title{|expandafter|BKM@vtex@two|BKM@title\2|@nil}%
    |edef|BKM@title{|expandafter|BKM@vtex@three|BKM@title\3|@nil}%
  }%
%    \end{macrocode}
%    \end{macro}
%    \begin{macro}{\BKM@vtex@leftparen}
%    \begin{macrocode}
  |gdef|BKM@vtex@leftparen#1\(#2|@nil{%
    #1%
    |ifx||#2||%
    |else
      (%
      |ltx@ReturnAfterFi{%
        |BKM@vtex@leftparen#2|@nil
      }%
    |fi
  }%
%    \end{macrocode}
%    \end{macro}
%    \begin{macro}{\BKM@vtex@rightparen}
%    \begin{macrocode}
  |gdef|BKM@vtex@rightparen#1\)#2|@nil{%
    #1%
    |ifx||#2||%
    |else
      )%
      |ltx@ReturnAfterFi{%
        |BKM@vtex@rightparen#2|@nil
      }%
    |fi
  }%
%    \end{macrocode}
%    \end{macro}
%    \begin{macro}{\BKM@vtex@zero}
%    \begin{macrocode}
  |gdef|BKM@vtex@zero#1\0#2|@nil{%
    #1%
    |ifx||#2||%
    |else
      |noexpand|hv@pdf@char0%
      |ltx@ReturnAfterFi{%
        |BKM@vtex@zero#2|@nil
      }%
    |fi
  }%
%    \end{macrocode}
%    \end{macro}
%    \begin{macro}{\BKM@vtex@one}
%    \begin{macrocode}
  |gdef|BKM@vtex@one#1\1#2|@nil{%
    #1%
    |ifx||#2||%
    |else
      |noexpand|hv@pdf@char1%
      |ltx@ReturnAfterFi{%
        |BKM@vtex@one#2|@nil
      }%
    |fi
  }%
%    \end{macrocode}
%    \end{macro}
%    \begin{macro}{\BKM@vtex@two}
%    \begin{macrocode}
  |gdef|BKM@vtex@two#1\2#2|@nil{%
    #1%
    |ifx||#2||%
    |else
      |noexpand|hv@pdf@char2%
      |ltx@ReturnAfterFi{%
        |BKM@vtex@two#2|@nil
      }%
    |fi
  }%
%    \end{macrocode}
%    \end{macro}
%    \begin{macro}{\BKM@vtex@three}
%    \begin{macrocode}
  |gdef|BKM@vtex@three#1\3#2|@nil{%
    #1%
    |ifx||#2||%
    |else
      |noexpand|hv@pdf@char3%
      |ltx@ReturnAfterFi{%
        |BKM@vtex@three#2|@nil
      }%
    |fi
  }%
%    \end{macrocode}
%    \end{macro}
%    \begin{macrocode}
|endgroup
%    \end{macrocode}
%
%    \begin{macrocode}
%</vtex>
%    \end{macrocode}
%
% \subsection{\hologo{pdfTeX}\ 的驱动程序}
%
%    \begin{macrocode}
%<*pdftex>
\NeedsTeXFormat{LaTeX2e}
\ProvidesFile{bkm-pdftex.def}%
  [2020-11-06 v1.29 bookmark driver for pdfTeX (HO)]%
%    \end{macrocode}
%
%    \begin{macro}{\BKM@DO@entry}
%    \begin{macrocode}
\def\BKM@DO@entry#1#2{%
  \begingroup
    \kvsetkeys{BKM@DO}{#1}%
    \def\BKM@DO@title{#2}%
    \ifx\BKM@DO@srcfile\@empty
    \else
      \BKM@UnescapeHex\BKM@DO@srcfile
    \fi
    \BKM@UnescapeHex\BKM@DO@title
    \expandafter\expandafter\expandafter\BKM@getx
        \csname BKM@\BKM@DO@id\endcsname\@empty\@empty
    \let\BKM@attr\@empty
    \ifx\BKM@DO@flags\@empty
    \else
      \edef\BKM@attr{\BKM@attr/F \BKM@DO@flags}%
    \fi
    \ifx\BKM@DO@color\@empty
    \else
      \edef\BKM@attr{\BKM@attr/C[\BKM@DO@color]}%
    \fi
    \ifx\BKM@attr\@empty
    \else
      \edef\BKM@attr{attr{\BKM@attr}}%
    \fi
    \let\BKM@action\@empty
    \ifx\BKM@DO@gotor\@empty
      \ifx\BKM@DO@dest\@empty
        \ifx\BKM@DO@named\@empty
          \ifx\BKM@DO@rawaction\@empty
            \ifx\BKM@DO@uri\@empty
              \ifx\BKM@DO@page\@empty
                \PackageError{bookmark}{%
                  Missing action\BKM@SourceLocation
                }\@ehc
                \edef\BKM@action{goto page1{/Fit}}%
              \else
                \ifx\BKM@DO@view\@empty
                  \def\BKM@DO@view{Fit}%
                \fi
                \edef\BKM@action{goto page\BKM@DO@page{/\BKM@DO@view}}%
              \fi
            \else
              \BKM@UnescapeHex\BKM@DO@uri
              \BKM@EscapeString\BKM@DO@uri
              \edef\BKM@action{user{<</S/URI/URI(\BKM@DO@uri)>>}}%
            \fi
          \else
            \BKM@UnescapeHex\BKM@DO@rawaction
            \edef\BKM@action{%
              user{%
                <<%
                  \BKM@DO@rawaction
                >>%
              }%
            }%
          \fi
        \else
          \BKM@EscapeName\BKM@DO@named
          \edef\BKM@action{%
            user{<</S/Named/N/\BKM@DO@named>>}%
          }%
        \fi
      \else
        \BKM@UnescapeHex\BKM@DO@dest
        \BKM@DefGotoNameAction\BKM@action\BKM@DO@dest
      \fi
    \else
      \ifx\BKM@DO@dest\@empty
        \ifx\BKM@DO@page\@empty
          \def\BKM@DO@page{0}%
        \else
          \BKM@CalcExpr\BKM@DO@page\BKM@DO@page-1%
        \fi
        \ifx\BKM@DO@view\@empty
          \def\BKM@DO@view{Fit}%
        \fi
        \edef\BKM@action{/D[\BKM@DO@page/\BKM@DO@view]}%
      \else
        \BKM@UnescapeHex\BKM@DO@dest
        \BKM@EscapeString\BKM@DO@dest
        \edef\BKM@action{/D(\BKM@DO@dest)}%
      \fi
      \BKM@UnescapeHex\BKM@DO@gotor
      \BKM@EscapeString\BKM@DO@gotor
      \edef\BKM@action{%
        user{%
          <<%
            /S/GoToR%
            /F(\BKM@DO@gotor)%
            \BKM@action
          >>%
        }%
      }%
    \fi
    \pdfoutline\BKM@attr\BKM@action
                count\ifBKM@DO@open\else-\fi\BKM@x@childs
                {\BKM@DO@title}%
  \endgroup
}
%    \end{macrocode}
%    \end{macro}
%    \begin{macro}{\BKM@DefGotoNameAction}
%    \cs{BKM@DefGotoNameAction}\ 宏是一个用于 \xpackage{hypdestopt}\ 宏包的钩子(hook)。
%    \begin{macrocode}
\def\BKM@DefGotoNameAction#1#2{%
  \BKM@EscapeString\BKM@DO@dest
  \edef#1{goto name{#2}}%
}
%    \end{macrocode}
%    \end{macro}
%    \begin{macrocode}
%</pdftex>
%    \end{macrocode}
%
%    \begin{macrocode}
%<*pdftex|pdfmark>
%    \end{macrocode}
%    \begin{macro}{\BKM@SourceLocation}
%    \begin{macrocode}
\def\BKM@SourceLocation{%
  \ifx\BKM@DO@srcfile\@empty
    \ifx\BKM@DO@srcline\@empty
    \else
      .\MessageBreak
      Source: line \BKM@DO@srcline
    \fi
  \else
    \ifx\BKM@DO@srcline\@empty
      .\MessageBreak
      Source: file `\BKM@DO@srcfile'%
    \else
      .\MessageBreak
      Source: file `\BKM@DO@srcfile', line \BKM@DO@srcline
    \fi
  \fi
}
%    \end{macrocode}
%    \end{macro}
%    \begin{macrocode}
%</pdftex|pdfmark>
%    \end{macrocode}
%
% \subsection{具有 pdfmark 特色(specials)的驱动程序}
%
% \subsubsection{dvips 驱动程序}
%
%    \begin{macrocode}
%<*dvips>
\NeedsTeXFormat{LaTeX2e}
\ProvidesFile{bkm-dvips.def}%
  [2020-11-06 v1.29 bookmark driver for dvips (HO)]%
%    \end{macrocode}
%    \begin{macro}{\BKM@PSHeaderFile}
%    \begin{macrocode}
\def\BKM@PSHeaderFile#1{%
  \special{PSfile=#1}%
}
%    \end{macrocode}
%    \begin{macro}{\BKM@filename}
%    \begin{macrocode}
\def\BKM@filename{\jobname.out.ps}
%    \end{macrocode}
%    \end{macro}
%    \begin{macrocode}
\AddToHook{shipout/lastpage}{%
  \BKM@pdfmark@out
  \BKM@PSHeaderFile\BKM@filename
  }
%    \end{macrocode}
%    \end{macro}
%    \begin{macrocode}
%</dvips>
%    \end{macrocode}
%
% \subsubsection{公共部分(Common part)}
%
%    \begin{macrocode}
%<*pdfmark>
%    \end{macrocode}
%
%    \begin{macro}{\BKM@pdfmark@out}
%    不要在这里使用 \xpackage{rerunfilecheck}\ 宏包,因为在 \hologo{TeX}\ 运行期间不会
%    读取 \cs{BKM@filename}\ 文件。
%    \begin{macrocode}
\def\BKM@pdfmark@out{%
  \if@filesw
    \newwrite\BKM@file
    \immediate\openout\BKM@file=\BKM@filename\relax
    \BKM@write{\@percentchar!}%
    \BKM@write{/pdfmark where{pop}}%
    \BKM@write{%
      {%
        /globaldict where{pop globaldict}{userdict}ifelse%
        /pdfmark/cleartomark load put%
      }%
    }%
    \BKM@write{ifelse}%
  \else
    \let\BKM@write\@gobble
    \let\BKM@DO@entry\@gobbletwo
  \fi
}
%    \end{macrocode}
%    \end{macro}
%    \begin{macro}{\BKM@write}
%    \begin{macrocode}
\def\BKM@write#{%
  \immediate\write\BKM@file
}
%    \end{macrocode}
%    \end{macro}
%
%    \begin{macro}{\BKM@DO@entry}
%    Pdfmark 的规范(specification)说明 |/Color| 是颜色(color)的键名(key name),
%    但是 ghostscript 只将键(key)传递到 PDF 文件中,因此键名必须是 |/C|。
%    \begin{macrocode}
\def\BKM@DO@entry#1#2{%
  \begingroup
    \kvsetkeys{BKM@DO}{#1}%
    \ifx\BKM@DO@srcfile\@empty
    \else
      \BKM@UnescapeHex\BKM@DO@srcfile
    \fi
    \def\BKM@DO@title{#2}%
    \BKM@UnescapeHex\BKM@DO@title
    \expandafter\expandafter\expandafter\BKM@getx
        \csname BKM@\BKM@DO@id\endcsname\@empty\@empty
    \let\BKM@attr\@empty
    \ifx\BKM@DO@flags\@empty
    \else
      \edef\BKM@attr{\BKM@attr/F \BKM@DO@flags}%
    \fi
    \ifx\BKM@DO@color\@empty
    \else
      \edef\BKM@attr{\BKM@attr/C[\BKM@DO@color]}%
    \fi
    \let\BKM@action\@empty
    \ifx\BKM@DO@gotor\@empty
      \ifx\BKM@DO@dest\@empty
        \ifx\BKM@DO@named\@empty
          \ifx\BKM@DO@rawaction\@empty
            \ifx\BKM@DO@uri\@empty
              \ifx\BKM@DO@page\@empty
                \PackageError{bookmark}{%
                  Missing action\BKM@SourceLocation
                }\@ehc
                \edef\BKM@action{%
                  /Action/GoTo%
                  /Page 1%
                  /View[/Fit]%
                }%
              \else
                \ifx\BKM@DO@view\@empty
                  \def\BKM@DO@view{Fit}%
                \fi
                \edef\BKM@action{%
                  /Action/GoTo%
                  /Page \BKM@DO@page
                  /View[/\BKM@DO@view]%
                }%
              \fi
            \else
              \BKM@UnescapeHex\BKM@DO@uri
              \BKM@EscapeString\BKM@DO@uri
              \edef\BKM@action{%
                /Action<<%
                  /Subtype/URI%
                  /URI(\BKM@DO@uri)%
                >>%
              }%
            \fi
          \else
            \BKM@UnescapeHex\BKM@DO@rawaction
            \edef\BKM@action{%
              /Action<<%
                \BKM@DO@rawaction
              >>%
            }%
          \fi
        \else
          \BKM@EscapeName\BKM@DO@named
          \edef\BKM@action{%
            /Action<<%
              /Subtype/Named%
              /N/\BKM@DO@named
            >>%
          }%
        \fi
      \else
        \BKM@UnescapeHex\BKM@DO@dest
        \BKM@EscapeString\BKM@DO@dest
        \edef\BKM@action{%
          /Action/GoTo%
          /Dest(\BKM@DO@dest)cvn%
        }%
      \fi
    \else
      \ifx\BKM@DO@dest\@empty
        \ifx\BKM@DO@page\@empty
          \def\BKM@DO@page{1}%
        \fi
        \ifx\BKM@DO@view\@empty
          \def\BKM@DO@view{Fit}%
        \fi
        \edef\BKM@action{%
          /Page \BKM@DO@page
          /View[/\BKM@DO@view]%
        }%
      \else
        \BKM@UnescapeHex\BKM@DO@dest
        \BKM@EscapeString\BKM@DO@dest
        \edef\BKM@action{%
          /Dest(\BKM@DO@dest)cvn%
        }%
      \fi
      \BKM@UnescapeHex\BKM@DO@gotor
      \BKM@EscapeString\BKM@DO@gotor
      \edef\BKM@action{%
        /Action/GoToR%
        /File(\BKM@DO@gotor)%
        \BKM@action
      }%
    \fi
    \BKM@write{[}%
    \BKM@write{/Title(\BKM@DO@title)}%
    \ifnum\BKM@x@childs>\z@
      \BKM@write{/Count \ifBKM@DO@open\else-\fi\BKM@x@childs}%
    \fi
    \ifx\BKM@attr\@empty
    \else
      \BKM@write{\BKM@attr}%
    \fi
    \BKM@write{\BKM@action}%
    \BKM@write{/OUT pdfmark}%
  \endgroup
}
%    \end{macrocode}
%    \end{macro}
%    \begin{macrocode}
%</pdfmark>
%    \end{macrocode}
%
% \subsection{\xoption{pdftex}\ 和 \xoption{pdfmark}\ 的公共部分}
%
%    \begin{macrocode}
%<*pdftex|pdfmark>
%    \end{macrocode}
%
% \subsubsection{写入辅助文件(auxiliary file)}
%
%    \begin{macrocode}
\AddToHook{begindocument}{%
 \immediate\write\@mainaux{\string\providecommand\string\BKM@entry[2]{}}}
%    \end{macrocode}
%
%    \begin{macro}{\BKM@id}
%    \begin{macrocode}
\newcount\BKM@id
\BKM@id=\z@
%    \end{macrocode}
%    \end{macro}
%
%    \begin{macro}{\BKM@0}
%    \begin{macrocode}
\@namedef{BKM@0}{000}
%    \end{macrocode}
%    \end{macro}
%    \begin{macro}{\ifBKM@sw}
%    \begin{macrocode}
\newif\ifBKM@sw
%    \end{macrocode}
%    \end{macro}
%
%    \begin{macro}{\bookmark}
%    \begin{macrocode}
\newcommand*{\bookmark}[2][]{%
  \if@filesw
    \begingroup
      \BKM@InitSourceLocation
      \def\bookmark@text{#2}%
      \BKM@setup{#1}%
      \ifx\BKM@srcfile\@empty
      \else
        \BKM@EscapeHex\BKM@srcfile
      \fi
      \edef\BKM@prev{\the\BKM@id}%
      \global\advance\BKM@id\@ne
      \BKM@swtrue
      \@whilesw\ifBKM@sw\fi{%
        \ifnum\ifBKM@startatroot\z@\else\BKM@prev\fi=\z@
          \BKM@startatrootfalse
          \expandafter\xdef\csname BKM@\the\BKM@id\endcsname{%
            0{\BKM@level}0%
          }%
          \BKM@swfalse
        \else
          \expandafter\expandafter\expandafter\BKM@getx
              \csname BKM@\BKM@prev\endcsname
          \ifnum\BKM@level>\BKM@x@level\relax
            \expandafter\xdef\csname BKM@\the\BKM@id\endcsname{%
              {\BKM@prev}{\BKM@level}0%
            }%
            \ifnum\BKM@prev>\z@
              \BKM@CalcExpr\BKM@CalcResult\BKM@x@childs+1%
              \expandafter\xdef\csname BKM@\BKM@prev\endcsname{%
                {\BKM@x@parent}{\BKM@x@level}{\BKM@CalcResult}%
              }%
            \fi
            \BKM@swfalse
          \else
            \let\BKM@prev\BKM@x@parent
          \fi
        \fi
      }%
      \pdfstringdef\BKM@title{\bookmark@text}%
      \edef\BKM@FLAGS{\BKM@PrintStyle}%
      \csname BKM@HypDestOptHook\endcsname
      \BKM@EscapeHex\BKM@dest
      \BKM@EscapeHex\BKM@uri
      \BKM@EscapeHex\BKM@gotor
      \BKM@EscapeHex\BKM@rawaction
      \BKM@EscapeHex\BKM@title
      \immediate\write\@mainaux{%
        \string\BKM@entry{%
          id=\number\BKM@id
          \ifBKM@open
            \ifnum\BKM@level<\BKM@openlevel
              ,open%
            \fi
          \fi
          \BKM@auxentry{dest}%
          \BKM@auxentry{named}%
          \BKM@auxentry{uri}%
          \BKM@auxentry{gotor}%
          \BKM@auxentry{page}%
          \BKM@auxentry{view}%
          \BKM@auxentry{rawaction}%
          \BKM@auxentry{color}%
          \ifnum\BKM@FLAGS>\z@
            ,flags=\BKM@FLAGS
          \fi
          \BKM@auxentry{srcline}%
          \BKM@auxentry{srcfile}%
        }{\BKM@title}%
      }%
    \endgroup
  \fi
}
%    \end{macrocode}
%    \end{macro}
%    \begin{macro}{\BKM@getx}
%    \begin{macrocode}
\def\BKM@getx#1#2#3{%
  \def\BKM@x@parent{#1}%
  \def\BKM@x@level{#2}%
  \def\BKM@x@childs{#3}%
}
%    \end{macrocode}
%    \end{macro}
%    \begin{macro}{\BKM@auxentry}
%    \begin{macrocode}
\def\BKM@auxentry#1{%
  \expandafter\ifx\csname BKM@#1\endcsname\@empty
  \else
    ,#1={\csname BKM@#1\endcsname}%
  \fi
}
%    \end{macrocode}
%    \end{macro}
%
%    \begin{macro}{\BKM@InitSourceLocation}
%    \begin{macrocode}
\def\BKM@InitSourceLocation{%
  \edef\BKM@srcline{\the\inputlineno}%
  \BKM@LuaTeX@InitFile
  \ifx\BKM@srcfile\@empty
    \ltx@IfUndefined{currfilepath}{}{%
      \edef\BKM@srcfile{\currfilepath}%
    }%
  \fi
}
%    \end{macrocode}
%    \end{macro}
%    \begin{macro}{\BKM@LuaTeX@InitFile}
%    \begin{macrocode}
\ifluatex
  \ifnum\luatexversion>36 %
    \def\BKM@LuaTeX@InitFile{%
      \begingroup
        \ltx@LocToksA={}%
      \edef\x{\endgroup
        \def\noexpand\BKM@srcfile{%
          \the\expandafter\ltx@LocToksA
          \directlua{%
             if status and status.filename then %
               tex.settoks('ltx@LocToksA', status.filename)%
             end%
          }%
        }%
      }\x
    }%
  \else
    \let\BKM@LuaTeX@InitFile\relax
  \fi
\else
  \let\BKM@LuaTeX@InitFile\relax
\fi
%    \end{macrocode}
%    \end{macro}
%
% \subsubsection{读取辅助数据(auxiliary data)}
%
%    \begin{macrocode}
\SetupKeyvalOptions{family=BKM@DO,prefix=BKM@DO@}
\DeclareStringOption[0]{id}
\DeclareBoolOption{open}
\DeclareStringOption{flags}
\DeclareStringOption{color}
\DeclareStringOption{dest}
\DeclareStringOption{named}
\DeclareStringOption{uri}
\DeclareStringOption{gotor}
\DeclareStringOption{page}
\DeclareStringOption{view}
\DeclareStringOption{rawaction}
\DeclareStringOption{srcline}
\DeclareStringOption{srcfile}
%    \end{macrocode}
%
%    \begin{macrocode}
\AtBeginDocument{%
  \let\BKM@entry\BKM@DO@entry
}
%    \end{macrocode}
%
%    \begin{macrocode}
%</pdftex|pdfmark>
%    \end{macrocode}
%
% \subsection{\xoption{atend}\ 选项}
%
% \subsubsection{钩子(Hook)}
%
%    \begin{macrocode}
%<*package>
%    \end{macrocode}
%    \begin{macrocode}
\ifBKM@atend
\else
%    \end{macrocode}
%    \begin{macro}{\BookmarkAtEnd}
%    这是一个虚拟定义(dummy definition),如果没有给出 \xoption{atend}\ 选项,它将生成一个警告。
%    \begin{macrocode}
  \newcommand{\BookmarkAtEnd}[1]{%
    \PackageWarning{bookmark}{%
      Ignored, because option `atend' is missing%
    }%
  }%
%    \end{macrocode}
%    \end{macro}
%    \begin{macrocode}
  \expandafter\endinput
\fi
%    \end{macrocode}
%    \begin{macro}{\BookmarkAtEnd}
%    \begin{macrocode}
\newcommand*{\BookmarkAtEnd}{%
  \g@addto@macro\BKM@EndHook
}
%    \end{macrocode}
%    \end{macro}
%    \begin{macrocode}
\let\BKM@EndHook\@empty
%    \end{macrocode}
%    \begin{macrocode}
%</package>
%    \end{macrocode}
%
% \subsubsection{在文档末尾使用钩子的驱动程序}
%
%    驱动程序 \xoption{pdftex}\ 使用 LaTeX 钩子 \xoption{enddocument/afterlastpage}
%    (相当于以前使用的 \xpackage{atveryend}\ 的 \cs{AfterLastShipout}),因为它仍然需要 \xext{aux}\ 文件。
%    它使用 \cs{pdfoutline}\ 作为最后一页之后可以使用的书签(bookmakrs)。
%    \begin{itemize}
%    \item
%      驱动程序 \xoption{pdftex}\ 使用 \cs{pdfoutline}, \cs{pdfoutline}\ 可以在最后一页之后使用。
%    \end{itemize}
%    \begin{macrocode}
%<*pdftex>
\ifBKM@atend
  \AddToHook{enddocument/afterlastpage}{%
    \BKM@EndHook
  }%
\fi
%</pdftex>
%    \end{macrocode}
%
% \subsubsection{使用 \xoption{shipout/lastpage}\ 的驱动程序}
%
%    其他驱动程序使用 \cs{special}\ 命令实现 \cs{bookmark}。因此,最后的书签(last bookmarks)
%    必须放在最后一页(last page),而不是之后。不能使用 \cs{AtEndDocument},因为为时已晚,
%    最后一页已经输出了。因此,我们使用 LaTeX 钩子 \xoption{shipout/lastpage}。至少需要运行
%    两次 \hologo{LaTeX}。PostScript 驱动程序 \xoption{dvips}\ 使用外部 PostScript 文件作为书签。
%    为了避免与 pgf 发生冲突,文件写入(file writing)也被移到了最后一个输出页面(shipout page)。
%    \begin{macrocode}
%<*dvipdfm|vtex|pdfmark>
\ifBKM@atend
  \AddToHook{shipout/lastpage}{\BKM@EndHook}%
\fi
%</dvipdfm|vtex|pdfmark>
%    \end{macrocode}
%
% \section{安装(Installation)}
%
% \subsection{下载(Download)}
%
% \paragraph{宏包(Package)。} 在 CTAN\footnote{\CTANpkg{bookmark}}上提供此宏包:
% \begin{description}
% \item[\CTAN{macros/latex/contrib/bookmark/bookmark.dtx}] 源文件(source file)。
% \item[\CTAN{macros/latex/contrib/bookmark/bookmark.pdf}] 文档(documentation)。
% \end{description}
%
%
% \paragraph{捆绑包(Bundle)。} “bookmark”捆绑包(bundle)的所有宏包(packages)都可以在兼
% 容 TDS 的 ZIP 归档文件中找到。在那里,宏包已经被解包,文档文件(documentation files)已经生成。
% 文件(files)和目录(directories)遵循 TDS 标准。
% \begin{description}
% \item[\CTANinstall{install/macros/latex/contrib/bookmark.tds.zip}]
% \end{description}
% \emph{TDS}\ 是指标准的“用于 \TeX\ 文件的目录结构(Directory Structure)”(\CTANpkg{tds})。
% 名称中带有 \xfile{texmf}\ 的目录(directories)通常以这种方式组织。
%
% \subsection{捆绑包(Bundle)的安装}
%
% \paragraph{解压(Unpacking)。} 在您选择的 TDS 树(也称为 \xfile{texmf}\ 树)中解
% 压 \xfile{bookmark.tds.zip},例如(在 linux 中):
% \begin{quote}
%   |unzip bookmark.tds.zip -d ~/texmf|
% \end{quote}
%
% \subsection{宏包(Package)的安装}
%
% \paragraph{解压(Unpacking)。} \xfile{.dtx}\ 文件是一个自解压 \docstrip\ 归档文件(archive)。
% 这些文件是通过 \plainTeX\ 运行 \xfile{.dtx}\ 来提取的:
% \begin{quote}
%   \verb|tex bookmark.dtx|
% \end{quote}
%
% \paragraph{TDS.} 现在,不同的文件必须移动到安装 TDS 树(installation TDS tree)
% (也称为 \xfile{texmf}\ 树)中的不同目录中:
% \begin{quote}
% \def\t{^^A
% \begin{tabular}{@{}>{\ttfamily}l@{ $\rightarrow$ }>{\ttfamily}l@{}}
%   bookmark.sty & tex/latex/bookmark/bookmark.sty\\
%   bkm-dvipdfm.def & tex/latex/bookmark/bkm-dvipdfm.def\\
%   bkm-dvips.def & tex/latex/bookmark/bkm-dvips.def\\
%   bkm-pdftex.def & tex/latex/bookmark/bkm-pdftex.def\\
%   bkm-vtex.def & tex/latex/bookmark/bkm-vtex.def\\
%   bookmark.pdf & doc/latex/bookmark/bookmark.pdf\\
%   bookmark-example.tex & doc/latex/bookmark/bookmark-example.tex\\
%   bookmark.dtx & source/latex/bookmark/bookmark.dtx\\
% \end{tabular}^^A
% }^^A
% \sbox0{\t}^^A
% \ifdim\wd0>\linewidth
%   \begingroup
%     \advance\linewidth by\leftmargin
%     \advance\linewidth by\rightmargin
%   \edef\x{\endgroup
%     \def\noexpand\lw{\the\linewidth}^^A
%   }\x
%   \def\lwbox{^^A
%     \leavevmode
%     \hbox to \linewidth{^^A
%       \kern-\leftmargin\relax
%       \hss
%       \usebox0
%       \hss
%       \kern-\rightmargin\relax
%     }^^A
%   }^^A
%   \ifdim\wd0>\lw
%     \sbox0{\small\t}^^A
%     \ifdim\wd0>\linewidth
%       \ifdim\wd0>\lw
%         \sbox0{\footnotesize\t}^^A
%         \ifdim\wd0>\linewidth
%           \ifdim\wd0>\lw
%             \sbox0{\scriptsize\t}^^A
%             \ifdim\wd0>\linewidth
%               \ifdim\wd0>\lw
%                 \sbox0{\tiny\t}^^A
%                 \ifdim\wd0>\linewidth
%                   \lwbox
%                 \else
%                   \usebox0
%                 \fi
%               \else
%                 \lwbox
%               \fi
%             \else
%               \usebox0
%             \fi
%           \else
%             \lwbox
%           \fi
%         \else
%           \usebox0
%         \fi
%       \else
%         \lwbox
%       \fi
%     \else
%       \usebox0
%     \fi
%   \else
%     \lwbox
%   \fi
% \else
%   \usebox0
% \fi
% \end{quote}
% 如果你有一个 \xfile{docstrip.cfg}\ 文件,该文件能配置并启用 \docstrip\ 的 TDS 安装功能,
% 则一些文件可能已经在正确的位置了,请参阅 \docstrip\ 的文档(documentation)。
%
% \subsection{刷新文件名数据库}
%
% 如果您的 \TeX~发行版(\TeX\,Live、\mikTeX、\dots)依赖于文件名数据库(file name databases),
% 则必须刷新这些文件名数据库。例如,\TeX\,Live\ 用户运行 \verb|texhash| 或 \verb|mktexlsr|。
%
% \subsection{一些感兴趣的细节}
%
% \paragraph{用 \LaTeX\ 解压。}
% \xfile{.dtx}\ 根据格式(format)选择其操作(action):
% \begin{description}
% \item[\plainTeX:] 运行 \docstrip\ 并解压文件。
% \item[\LaTeX:] 生成文档。
% \end{description}
% 如果您坚持通过 \LaTeX\ 使用\docstrip (实际上 \docstrip\ 并不需要 \LaTeX),那么请您的意图告知自动检测程序:
% \begin{quote}
%   \verb|latex \let\install=y% \iffalse meta-comment
%
% File: bookmark.dtx
% Version: 2020-11-06 v1.29
% Info: PDF bookmarks
%
% Copyright (C)
%    2007-2011 Heiko Oberdiek
%    2016-2020 Oberdiek Package Support Group
%    https://github.com/ho-tex/bookmark/issues
%
% This work may be distributed and/or modified under the
% conditions of the LaTeX Project Public License, either
% version 1.3c of this license or (at your option) any later
% version. This version of this license is in
%    https://www.latex-project.org/lppl/lppl-1-3c.txt
% and the latest version of this license is in
%    https://www.latex-project.org/lppl.txt
% and version 1.3 or later is part of all distributions of
% LaTeX version 2005/12/01 or later.
%
% This work has the LPPL maintenance status "maintained".
%
% The Current Maintainers of this work are
% Heiko Oberdiek and the Oberdiek Package Support Group
% https://github.com/ho-tex/bookmark/issues
%
% This work consists of the main source file bookmark.dtx
% and the derived files
%    bookmark.sty, bookmark.pdf, bookmark.ins, bookmark.drv,
%    bkm-dvipdfm.def, bkm-dvips.def,
%    bkm-pdftex.def, bkm-vtex.def,
%    bkm-dvipdfm-2019-12-03.def, bkm-dvips-2019-12-03.def,
%    bkm-pdftex-2019-12-03.def, bkm-vtex-2019-12-03.def,
%    bookmark-example.tex.
%
% Distribution:
%    CTAN:macros/latex/contrib/bookmark/bookmark.dtx
%    CTAN:macros/latex/contrib/bookmark/bookmark-frozen.dtx
%    CTAN:macros/latex/contrib/bookmark/bookmark.pdf
%
% Unpacking:
%    (a) If bookmark.ins is present:
%           tex bookmark.ins
%    (b) Without bookmark.ins:
%           tex bookmark.dtx
%    (c) If you insist on using LaTeX
%           latex \let\install=y\input{bookmark.dtx}
%        (quote the arguments according to the demands of your shell)
%
% Documentation:
%    (a) If bookmark.drv is present:
%           latex bookmark.drv
%    (b) Without bookmark.drv:
%           latex bookmark.dtx; ...
%    The class ltxdoc loads the configuration file ltxdoc.cfg
%    if available. Here you can specify further options, e.g.
%    use A4 as paper format:
%       \PassOptionsToClass{a4paper}{article}
%
%    Programm calls to get the documentation (example):
%       pdflatex bookmark.dtx
%       makeindex -s gind.ist bookmark.idx
%       pdflatex bookmark.dtx
%       makeindex -s gind.ist bookmark.idx
%       pdflatex bookmark.dtx
%
% Installation:
%    TDS:tex/latex/bookmark/bookmark.sty
%    TDS:tex/latex/bookmark/bkm-dvipdfm.def
%    TDS:tex/latex/bookmark/bkm-dvips.def
%    TDS:tex/latex/bookmark/bkm-pdftex.def
%    TDS:tex/latex/bookmark/bkm-vtex.def
%    TDS:tex/latex/bookmark/bkm-dvipdfm-2019-12-03.def
%    TDS:tex/latex/bookmark/bkm-dvips-2019-12-03.def
%    TDS:tex/latex/bookmark/bkm-pdftex-2019-12-03.def
%    TDS:tex/latex/bookmark/bkm-vtex-2019-12-03.def%
%    TDS:doc/latex/bookmark/bookmark.pdf
%    TDS:doc/latex/bookmark/bookmark-example.tex
%    TDS:source/latex/bookmark/bookmark.dtx
%    TDS:source/latex/bookmark/bookmark-frozen.dtx
%
%<*ignore>
\begingroup
  \catcode123=1 %
  \catcode125=2 %
  \def\x{LaTeX2e}%
\expandafter\endgroup
\ifcase 0\ifx\install y1\fi\expandafter
         \ifx\csname processbatchFile\endcsname\relax\else1\fi
         \ifx\fmtname\x\else 1\fi\relax
\else\csname fi\endcsname
%</ignore>
%<*install>
\input docstrip.tex
\Msg{************************************************************************}
\Msg{* Installation}
\Msg{* Package: bookmark 2020-11-06 v1.29 PDF bookmarks (HO)}
\Msg{************************************************************************}

\keepsilent
\askforoverwritefalse

\let\MetaPrefix\relax
\preamble

This is a generated file.

Project: bookmark
Version: 2020-11-06 v1.29

Copyright (C)
   2007-2011 Heiko Oberdiek
   2016-2020 Oberdiek Package Support Group

This work may be distributed and/or modified under the
conditions of the LaTeX Project Public License, either
version 1.3c of this license or (at your option) any later
version. This version of this license is in
   https://www.latex-project.org/lppl/lppl-1-3c.txt
and the latest version of this license is in
   https://www.latex-project.org/lppl.txt
and version 1.3 or later is part of all distributions of
LaTeX version 2005/12/01 or later.

This work has the LPPL maintenance status "maintained".

The Current Maintainers of this work are
Heiko Oberdiek and the Oberdiek Package Support Group
https://github.com/ho-tex/bookmark/issues


This work consists of the main source file bookmark.dtx and bookmark-frozen.dtx
and the derived files
   bookmark.sty, bookmark.pdf, bookmark.ins, bookmark.drv,
   bkm-dvipdfm.def, bkm-dvips.def, bkm-pdftex.def, bkm-vtex.def,
   bkm-dvipdfm-2019-12-03.def, bkm-dvips-2019-12-03.def,
   bkm-pdftex-2019-12-03.def, bkm-vtex-2019-12-03.def,
   bookmark-example.tex.

\endpreamble
\let\MetaPrefix\DoubleperCent

\generate{%
  \file{bookmark.ins}{\from{bookmark.dtx}{install}}%
  \file{bookmark.drv}{\from{bookmark.dtx}{driver}}%
  \usedir{tex/latex/bookmark}%
  \file{bookmark.sty}{\from{bookmark.dtx}{package}}%
  \file{bkm-dvipdfm.def}{\from{bookmark.dtx}{dvipdfm}}%
  \file{bkm-dvips.def}{\from{bookmark.dtx}{dvips,pdfmark}}%
  \file{bkm-pdftex.def}{\from{bookmark.dtx}{pdftex}}%
  \file{bkm-vtex.def}{\from{bookmark.dtx}{vtex}}%
  \usedir{doc/latex/bookmark}%
  \file{bookmark-example.tex}{\from{bookmark.dtx}{example}}%
  \file{bkm-pdftex-2019-12-03.def}{\from{bookmark-frozen.dtx}{pdftexfrozen}}%
  \file{bkm-dvips-2019-12-03.def}{\from{bookmark-frozen.dtx}{dvipsfrozen}}%
  \file{bkm-vtex-2019-12-03.def}{\from{bookmark-frozen.dtx}{vtexfrozen}}%
  \file{bkm-dvipdfm-2019-12-03.def}{\from{bookmark-frozen.dtx}{dvipdfmfrozen}}%
}

\catcode32=13\relax% active space
\let =\space%
\Msg{************************************************************************}
\Msg{*}
\Msg{* To finish the installation you have to move the following}
\Msg{* files into a directory searched by TeX:}
\Msg{*}
\Msg{*     bookmark.sty, bkm-dvipdfm.def, bkm-dvips.def,}
\Msg{*     bkm-pdftex.def, bkm-vtex.def, bkm-dvipdfm-2019-12-03.def,}
\Msg{*     bkm-dvips-2019-12-03.def, bkm-pdftex-2019-12-03.def,}
\Msg{*     and bkm-vtex-2019-12-03.def}
\Msg{*}
\Msg{* To produce the documentation run the file `bookmark.drv'}
\Msg{* through LaTeX.}
\Msg{*}
\Msg{* Happy TeXing!}
\Msg{*}
\Msg{************************************************************************}

\endbatchfile
%</install>
%<*ignore>
\fi
%</ignore>
%<*driver>
\NeedsTeXFormat{LaTeX2e}
\ProvidesFile{bookmark.drv}%
  [2020-11-06 v1.29 PDF bookmarks (HO)]%
\documentclass{ltxdoc}
\usepackage{ctex}
\usepackage{indentfirst}
\setlength{\parindent}{2em}
\usepackage{holtxdoc}[2011/11/22]
\usepackage{xcolor}
\usepackage{hyperref}
\usepackage[open,openlevel=3,atend]{bookmark}[2020/11/06] %%%打开书签,显示的深度为3级,即显示part、section、subsection。
\bookmarksetup{color=red}
\begin{document}

  \renewcommand{\contentsname}{目\quad 录}
  \renewcommand{\abstractname}{摘\quad 要}
  \renewcommand{\historyname}{历史}
  \DocInput{bookmark.dtx}%
\end{document}
%</driver>
% \fi
%
%
%
% \GetFileInfo{bookmark.drv}
%
%% \title{\xpackage{bookmark} 宏包}
% \title{\heiti {\Huge \textbf{\xpackage{bookmark}\ 宏包}}}
% \date{2020-11-06\ \ \ v1.29}
% \author{Heiko Oberdiek \thanks
% {如有问题请点击:\url{https://github.com/ho-tex/bookmark/issues}}\\[5pt]赣医一附院神经科\ \ 黄旭华\ \ \ \ 译}
%
% \maketitle
%
% \begin{abstract}
% 这个宏包为 \xpackage{hyperref}\ 宏包实现了一个新的书签(bookmark)(大纲[outline])组织。现在
% 可以设置样式(style)和颜色(color)等书签属性(bookmark properties)。其他动作类型(action types)可用
% (URI、GoToR、Named)。书签是在第一次编译运行(compile run)中生成的。\xpackage{hyperref}\
% 宏包必需运行两次。
% \end{abstract}
%
% \tableofcontents
%
% \section{文档(Documentation)}
%
% \subsection{介绍}
%
% 这个 \xpackage{bookmark}\ 宏包试图为书签(bookmarks)提供一个更现代的管理:
% \begin{itemize}
% \item 书签已经在第一次 \hologo{TeX}\ 编译运行(compile run)中生成。
% \item 可以更改书签的字体样式(font style)和颜色(color)。
% \item 可以执行比简单的 GoTo 操作(actions)更多的操作。
% \end{itemize}
%
% 与 \xpackage{hyperref} \cite{hyperref} 一样,书签(bookmarks)也是按照书签生成宏
% (bookmark generating macros)(\cs{bookmark})的顺序生成的。级别号(level number)用于
% 定义书签的树结构(tree structure)。限制没有那么严格:
% \begin{itemize}
% \item 级别值(level values)可以跳变(jump)和省略(omit)。\cs{subsubsection}\ 可以跟在
%       \cs{chapter}\ 之后。这种情况如在 \xpackage{hyperref}\ 中则产生错误,它将显示一个警告(warning)
%       并尝试修复此错误。
% \item 多个书签可能指向同一目标(destination)。在 \xpackage{hyperref}\ 中,这会完全弄乱
%       书签树(bookmark tree),因为算法假设(algorithm assumes)目标名称(destination names)
%       是键(keys)(唯一的)。
% \end{itemize}
%
% 注意,这个宏包是作为书签管理(bookmark management)的实验平台(experimentation platform)。
% 欢迎反馈。此外,在未来的版本中,接口(interfaces)也可能发生变化。
%
% \subsection{选项(Options)}
%
% 可在以下四个地方放置选项(options):
% \begin{enumerate}
% \item \cs{usepackage}|[|\meta{options}|]{bookmark}|\\
%       这是放置驱动程序选项(driver options)和 \xoption{atend}\ 选项的唯一位置。
% \item \cs{bookmarksetup}|{|\meta{options}|}|\\
%       此命令仅用于设置选项(setting options)。
% \item \cs{bookmarksetupnext}|{|\meta{options}|}|\\
%       这些选项在下一个 \cs{bookmark}\ 命令的选项之后存储(stored)和调用(called)。
% \item \cs{bookmark}|[|\meta{options}|]{|\meta{title}|}|\\
%       此命令设置书签。选项设置(option settings)仅限于此书签。
% \end{enumerate}
% 异常(Exception):加载该宏包后,无法更改驱动程序选项(Driver options)、\xoption{atend}\ 选项
% 、\xoption{draft}\slash\xoption{final}选项。
%
% \subsubsection{\xoption{draft} 和 \xoption{final}\ 选项}
%
% 如果一个\LaTeX\ 文件要被编译了多次,那么可以使用 \xoption{draft}\ 选项来禁用该宏包的书签内
% 容(bookmark stuff),这样可以节省一点时间。默认 \xoption{final}\ 选项。两个选项都是
% 布尔选项(boolean options),如果没有值,则使用值 |true|。|draft=true| 与 |final=false| 相同。
%
% 除了驱动程序选项(driver options)之外,\xpackage{bookmark}\ 宏包选项都是局部选项(local options)。
% \xoption{draft}\ 选项和 \xoption{final}\ 选项均属于文档类选项(class option)(译者注:文档类选项为全局选项),
% 因此,在 \xpackage{bookmark}\ 宏包中未能看到这两个选项。如果您想使用全局的(global) \xoption{draft}选项
% 来优化第一次 \LaTeX\ 运行(runs),可以在导言(preamble)中引入 \xpackage{ifdraft}\ 宏包并设置 \LaTeX\ 的
% \cs{PassOptionsToPackage},例如:
%\begin{quote}
%\begin{verbatim}
%\documentclass[draft]{article}
%\usepackage{ifdraft}
%\ifdraft{%
%   \PassOptionsToPackage{draft}{bookmark}%
%}{}
%\end{verbatim}
%\end{quote}
%
% \subsubsection{驱动程序选项(Driver options)}
%
% 支持的驱动程序( drivers)包括 \xoption{pdftex}、\xoption{dvips}、\xoption{dvipdfm} (\xoption{xetex})、
% \xoption{vtex}。\hologo{TeX}\ 引擎 \hologo{pdfTeX}、\hologo{XeTeX}、\hologo{VTeX}\ 能被自动检测到。
% 默认的 DVI 驱动程序是 \xoption{dvips}。这可以通过 \cs{BookmarkDriverDefault}\ 在配置
% 文件 \xfile{bookmark.cfg}\ 中进行更改,例如:
% \begin{quote}
% |\def\BookmarkDriverDefault{dvipdfm}|
% \end{quote}
% 当前版本的(current versions)驱动程序使用新的 \LaTeX\ 钩子(\LaTeX-hooks)。如果检测到比
% 2020-10-01 更旧的格式,则将以前驱动程序的冻结版本(frozen versions)作为备份(fallback)。
%
% \paragraph{用 dvipdfmx 打开书签(bookmarks)。}旧版本的宏包有一个 \xoption{dvipdfmx-outline-open}\ 选项
% 可以激活代码,而该代码可以指定一个大纲条目(outline entry)是否打开。该宏包现在假设所有使用的 dvipdfmx 版本都是
% 最新版本,足以理解该代码,因此始终激活该代码。选项本身将被忽略。
%
%
% \subsubsection{布局选项(Layout options)}
%
% \paragraph{字体(Font)选项:}
%
% \begin{description}
% \item[\xoption{bold}:] 如果受 PDF 浏览器(PDF viewer)支持,书签将以粗体字体(bold font)显示(自 PDF 1.4起)。
% \item[\xoption{italic}:] 使用斜体字体(italic font)(自 PDF 1.4起)。
% \end{description}
% \xoption{bold}(粗体) 和 \xoption{italic}(斜体)可以同时使用。而 |false| 值(value)禁用字体选项。
%
% \paragraph{颜色(Color)选项:}
%
% 彩色书签(Colored bookmarks)是 PDF 1.4 的一个特性(feature),并非所有的 PDF 浏览器(PDF viewers)都支持彩色书签。
% \begin{description}
% \item[\xoption{color}:] 这里 color(颜色)可以作为 \xpackage{color}\ 宏包或 \xpackage{xcolor}\ 宏包的
% 颜色规范(color specification)给出。空值(empty value)表示未设置颜色属性。如果未加载 \xpackage{xcolor}\ 宏包,
% 能识别的值(recognized values)只有:
%   \begin{itemize}
%   \item 空值(empty value)表示未设置颜色属性,\\
%         例如:|color={}|
%   \item 颜色模型(color model) rgb 的显式颜色规范(explicit color specification),\\
%         例如,红色(red):|color=[rgb]{1,0,0}|
%   \item 颜色模型(color model)灰(gray)的显式颜色规范(explicit color specification),\\
%         例如,深灰色(dark gray):|color=[gray]{0.25}|
%   \end{itemize}
%   请注意,如果加载了 \xpackage{color}\ 宏包,此限制(restriction)也适用。然而,如果加载了 \xpackage{xcolor}\ 宏包,
%   则可以使用所有颜色规范(color specifications)。
% \end{description}
%
% \subsubsection{动作选项(Action options)}
%
% \begin{description}
% \item[\xoption{dest}:] 目的地名称(destination name)。
% \item[\xoption{page}:] 页码(page number),第一页(first page)为 1。
% \item[\xoption{view}:] 浏览规范(view specification),示例如下:\\
%   |view={FitB}|, |view={FitH 842}|, |view={XYZ 0 100 null}|\ \  一些浏览规范参数(view specification parameters)
%   将数字(numbers)视为具有单位 bp 的参数。它们可以作为普通数字(plain numbers)或在 \cs{calc}\ 内部以
%   长度表达式(length expressions)给出。如果加载了 \xpackage{calc}\ 宏包,则支持该宏包的表达式(expressions)。否则,
%   使用 \hologo{eTeX}\ 的 \cs{dimexpr}。例如:\\
%   |view={FitH \calc{\paperheight-\topmargin-1in}}|\\
%   |view={XYZ 0 \calc{\paperheight} null}|\\
%   注意 \cs{calc}\ 不能用于 |XYZ| 的第三个参数,因为该参数是缩放值(zoom value),而不是长度(length)。

% \item[\xoption{named}:] 已命名的动作(Named action)的名称:\\
%   |FirstPage|(第一页),|LastPage|(最后一页),|NextPage|(下一页),|PrevPage|(前一页)
% \item[\xoption{gotor}:] 外部(external) PDF 文件的名称。
% \item[\xoption{uri}:] URI 规范(URI specification)。
% \item[\xoption{rawaction}:] 原始动作规范(raw action specification)。由于这些规范取决于驱动程序(driver),因此不应使用此选项。
% \end{description}
% 通过分析指定的选项来选择书签的适当动作。动作由不同的选项集(sets of options)区分:
% \begin{quote}
 \begin{tabular}{|@{}r|l@{}|}
%   \hline
%   \ \textbf{动作(Action)}\  & \ \textbf{选项(Options)}\ \\ \hline
%   \ \textsf{GoTo}\  &\  \xoption{dest}\ \\ \hline
%   \ \textsf{GoTo}\  & \ \xoption{page} + \xoption{view}\ \\ \hline
%   \ \textsf{GoToR}\  & \ \xoption{gotor} + \xoption{dest}\ \\ \hline
%   \ \textsf{GoToR}\  & \ \xoption{gotor} + \xoption{page} + \xoption{view}\ \ \ \\ \hline
%   \ \textsf{Named}\  &\  \xoption{named}\ \\ \hline
%   \ \textsf{URI}\  & \ \xoption{uri}\ \\ \hline
% \end{tabular}
% \end{quote}
%
% \paragraph{缺少动作(Missing actions)。}
% 如果动作缺少 \xpackage{bookmark}\ 宏包,则抛出错误消息(error message)。根据驱动程序(driver)
% (\xoption{pdftex}、\xoption{dvips}\ 和好友[friends]),宏包在文档末尾很晚才检测到它。
% 自 2011/04/21 v1.21 版本以后,该宏包尝试打印 \cs{bookmark}\ 的相应出现的行号(line number)和文件名(file name)。
% 然而,\hologo{TeX}\ 确实提供了行号,但不幸的是,文件名是一个秘密(secret)。但该宏包有如下获取文件名的方法:
% \begin{itemize}
% \item 如果 \hologo{LuaTeX} (独立于 DVI 或 PDF 模式)正在运行,则自动使用其 |status.filename|。
% \item 宏包的 \cs{currfile} \cite{currfile}\ 重新定义了 \hologo{LaTeX}\ 的内部结构,以跟踪文件名(file name)。
% 如果加载了该宏包,那么它的 \cs{currfilepath}\ 将被检测到并由 \xpackage{bookmark}\ 自动使用。
% \item 可以通过 \cs{bookmarksetup}\ 或 \cs{bookmark}\ 中的 \xoption{scrfile}\ 选项手动设置(set manually)文件名。
% 但是要小心,手动设置会禁用以前的文件名检测方法。错误的(wrong)或丢失的(missed)文件名设置(file name setting)可能会在错误消息中
% 为您提供错误的源位置(source location)。
% \end{itemize}
%
% \subsubsection{级别选项(Level options)}
%
% 书签条目(bookmark entries)的顺序由 \cs{bookmark}\ 命令的的出现顺序(appearance order)定义。
% 树结构(tree structure)由书签节点(bookmark nodes)的属性 \xoption{level}(级别)构建。
% \xoption{level}\ 的值是整数(integers)。如果书签条目级别的值高于前一个节点,则该条目将成为
% 前一个节点的子(child)节点。差值的绝对值并不重要。
%
% \xpackage{bookmark}\ 宏包能记住全局属性(global property)“current level(当前级别)”中上
% 一个书签条目(previous bookmark entry)的级别。
%
% 级别系统的(level system)行为(behaviour)可以通过以下选项进行配置:
% \begin{description}
% \item[\xoption{level}:]
%    设置级别(level),请参阅上面的说明。如果给出的选项 \xoption{level}\ 没有值,那么将恢复默
%    认行为,即将“当前级别(current level)”用作级别值(level value)。自 2010/10/19 v1.16 版本以来,
%    如果宏 \cs{toclevel@part}、\cs{toclevel@section}\ 被定义过(通过 \xpackage{hyperref}\ 宏包完成,
%    请参阅它的 \xoption{bookmarkdepth}\ 选项),则 \xpackage{bookmark}\ 宏包还支持 |part|、|section| 等名称。
%
% \item[\xoption{rellevel}:]
%    设置相对于前一级别的(previous level)级别。正值表示书签条目成为前一个书签条目的子条目。
% \item[\xoption{keeplevel}:]
%    使用由\xoption{level}\ 或 \xoption{rellevel}\ 设置的级别,但不要更改全局属性“current level(当前级别)”。
%    可以通过设置为 |false| 来禁用该选项。
% \item[\xoption{startatroot}:]
%    此时,书签树(bookmark tree)再次从顶层(top level)开始。下一个书签条目不会作为上一个条目的子条目进行排序。
%    示例场景:文档使用 part。但是,最后几章(last chapters)不应放在最后一部分(last part)下面:
%    \begin{quote}
%\begin{verbatim}
%\documentclass{book}
%[...]
%\begin{document}
%  \part{第一部分}
%    \chapter{第一部分的第1章}
%    [...]
%  \part{第二部分(Second part)}
%    \chapter{第二部分的第1章}
%    [...]
%  \bookmarksetup{startatroot}
%  \chapter{Index}% 不属于第二部分
%\end{document}
%\end{verbatim}
%    \end{quote}
% \end{description}
%
% \subsubsection{样式定义(Style definitions)}
%
% 样式(style)是一组选项设置(option settings)。它可以由宏 \cs{bookmarkdefinestyle}\ 定义,
% 并由它的 \xoption{style}\ 选项使用。
% \begin{declcs}{bookmarkdefinestyle} \M{name} \M{key value list}
% \end{declcs}
% 选项设置(option settings)的 \meta{key value list}(键值列表)被指定为样式名(style \meta{name})。
%
% \begin{description}
% \item[\xoption{style}:]
%   \xoption{style}\ 选项的值是以前定义的样式的名称(name)。现在执行其选项设置(option settings)。
%   选项可以包括 \xoption{style}\ 选项。通过递归调用相同样式的无限递归(endless recursion)被阻止并抛出一个错误。
% \end{description}
%
% \subsubsection{钩子支持(Hook support)}
%
% 处理宏\cs{bookmark}\ 的可选选项(optional options)后,就会调用钩子(hook)。
% \begin{description}
% \item[\xoption{addtohook}:]
%   代码(code)作为该选项的值添加到钩子中。
% \end{description}
%
% \begin{declcs}{bookmarkget} \M{option}
% \end{declcs}
% \cs{bookmarkget}\ 宏提取 \meta{option}\ 选项的最新选项设置(latest option setting)的值。
% 对于布尔选项(boolean option),如果启用布尔选项,则返回 1,否则结果为零。结果数字(resulting numbers)
% 可以直接用于 \cs{ifnum}\ 或 \cs{ifcase}。如果您想要数字 \texttt{0}\ 和 \texttt{1},
% 请在 \cs{bookmarkget}\ 前面加上 \cs{number}\ 作为前缀。\cs{bookmarkget}\ 宏是可展开的(expandable)。
% 如果选项不受支持,则返回空字符串(empty string)。受支持的布尔选项有:
% \begin{quote}
%   \xoption{bold}、
%   \xoption{italic}、
%   \xoption{open}
% \end{quote}
% 其他受支持的选项有:
% \begin{quote}
%   \xoption{depth}、
%   \xoption{dest}、
%   \xoption{color}、
%   \xoption{gotor}、
%   \xoption{level}、
%   \xoption{named}、
%   \xoption{openlevel}、
%   \xoption{page}、
%   \xoption{rawaction}、
%   \xoption{uri}、
%   \xoption{view}、
% \end{quote}
% 另外,以下键(key)是可用的:
% \begin{quote}
%   \xoption{text}
% \end{quote}
% 它返回大纲条目(outline entry)的文本(text)。
%
% \paragraph{选项设置(Option setting)。}
% 在钩子(hook)内部可以使用 \cs{bookmarksetup}\ 设置选项。
%
% \subsection{与 \xpackage{hyperref}\ 的兼容性}
%
% \xpackage{bookmark}\ 宏包自动禁用 \xpackage{hyperref}\ 宏包的书签(bookmarks)。但是,
% \xpackage{bookmark}\ 宏包使用了 \xpackage{hyperref}\ 宏包的一些代码。例如,
% \xpackage{bookmark}\ 宏包重新定义了 \xpackage{hyperref}\ 宏包在 \cs{addcontentsline}\ 命令
% 和其他命令中插入的\cs{Hy@writebookmark}\ 钩子。因此,不应禁用 \xpackage{hyperref}\ 宏包的书签。
%
% \xpackage{bookmark}\ 宏包使用 \xpackage{hyperref}\ 宏包的 \cs{pdfstringdef},且不提供替换(replacement)。
%
% \xpackage{hyperref}\ 宏包的一些选项也能在 \xpackage{bookmark}\ 宏包中实现(implemented):
% \begin{quote}
% \begin{tabular}{|l@{}|l@{}|}
%   \hline
%   \xpackage{hyperref}\ 宏包的选项\  &\ \xpackage{bookmark}\ 宏包的选项\ \ \\ \hline
%   \xoption{bookmarksdepth} &\ \xoption{depth}\\ \hline
%   \xoption{bookmarksopen} & \ \xoption{open}\\ \hline
%   \xoption{bookmarksopenlevel}\ \ \  &\ \xoption{openlevel}\\ \hline
%   \xoption{bookmarksnumbered} \ \ \ &\ \xoption{numbered}\\ \hline
% \end{tabular}
% \end{quote}
%
% 还可以使用以下命令:
% \begin{quote}
%   \cs{pdfbookmark}\\
%   \cs{currentpdfbookmark}\\
%   \cs{subpdfbookmark}\\
%   \cs{belowpdfbookmark}
% \end{quote}
%
% \subsection{在末尾添加书签}
%
% 宏包选项 \xoption{atend}\ 启用以下宏(macro):
% \begin{declcs}{BookmarkAtEnd}
%   \M{stuff}
% \end{declcs}
% \cs{BookmarkAtEnd}\ 宏将 \meta{stuff}\ 放在文档末尾。\meta{stuff}\ 表示书签命令(bookmark commands)。举例:
% \begin{quote}
%\begin{verbatim}
%\usepackage[atend]{bookmark}
%\BookmarkAtEnd{%
%  \bookmarksetup{startatroot}%
%  \bookmark[named=LastPage, level=0]{Last page}%
%}
%\end{verbatim}
% \end{quote}
%
% 或者,可以在 \cs{bookmark}\ 中给出 \xoption{startatroot}\ 选项:
% \begin{quote}
%\begin{verbatim}
%\BookmarkAtEnd{%
%  \bookmark[
%    startatroot,
%    named=LastPage,
%    level=0,
%  ]{Last page}%
%}
%\end{verbatim}
% \end{quote}
%
% \paragraph{备注(Remarks):}
% \begin{itemize}
% \item
%   \cs{BookmarkAtEnd} 隐藏了这样一个事实,即在文档末尾添加书签的方法取决于驱动程序(driver)。
%
%   为此,驱动程序 \xoption{pdftex}\ 使用 \xpackage{atveryend}\ 宏包。如果 \cs{AtEndDocument}\ 太早,
%   最后一个页面(last page)可能不会被发送出去(shipped out)。由于需要 \xext{aux}\ 文件,此驱动程序使
%   用 \cs{AfterLastShipout}。
%
%   其他驱动程序(\xoption{dvipdfm}、\xoption{xetex}、\xoption{vtex})的实现(implementation)
%   取决于 \cs{special},\cs{special}\ 在最后一页之后没有效果。在这种情况下,\xpackage{atenddvi}\ 宏包的
%   \cs{AtEndDvi}\ 有帮助。它将其参数(argument)放在文档的最后一页(last page)。至少需要运行 \hologo{LaTeX}\ 两次,
%   因为最后一页是由引用(reference)检测到的。
%
%   \xoption{dvips}\ 现在使用新的 LaTeX 钩子 \texttt{shipout/lastpage}。
% \item
%   未指定 \cs{BookmarkAtEnd}\ 参数的扩展时间(time of expansion)。这可以立即发生,也可以在文档末尾发生。
% \end{itemize}
%
% \subsection{限制/行动计划}
%
% \begin{itemize}
% \item 支持缺失动作(missing actions)(启动,\dots)。
% \item 对 \xpackage{hyperref}\ 的 \xoption{bookmarkstype}\ 选项进行了更好的设计(design)。
% \end{itemize}
%
% \section{示例(Example)}
%
%    \begin{macrocode}
%<*example>
%    \end{macrocode}
%    \begin{macrocode}
\documentclass{article}
\usepackage{xcolor}[2007/01/21]
\usepackage{hyperref}
\usepackage[
  open,
  openlevel=2,
  atend
]{bookmark}[2019/12/03]

\bookmarksetup{color=blue}

\BookmarkAtEnd{%
  \bookmarksetup{startatroot}%
  \bookmark[named=LastPage, level=0]{End/Last page}%
  \bookmark[named=FirstPage, level=1]{First page}%
}

\begin{document}
\section{First section}
\subsection{Subsection A}
\begin{figure}
  \hypertarget{fig}{}%
  A figure.
\end{figure}
\bookmark[
  rellevel=1,
  keeplevel,
  dest=fig
]{A figure}
\subsection{Subsection B}
\subsubsection{Subsubsection C}
\subsection{Umlauts: \"A\"O\"U\"a\"o\"u\ss}
\newpage
\bookmarksetup{
  bold,
  color=[rgb]{1,0,0}
}
\section{Very important section}
\bookmarksetup{
  italic,
  bold=false,
  color=blue
}
\subsection{Italic section}
\bookmarksetup{
  italic=false
}
\part{Misc}
\section{Diverse}
\subsubsection{Subsubsection, omitting subsection}
\bookmarksetup{
  startatroot
}
\section{Last section outside part}
\subsection{Subsection}
\bookmarksetup{
  color={}
}
\begingroup
  \bookmarksetup{level=0, color=green!80!black}
  \bookmark[named=FirstPage]{First page}
  \bookmark[named=LastPage]{Last page}
  \bookmark[named=PrevPage]{Previous page}
  \bookmark[named=NextPage]{Next page}
\endgroup
\bookmark[
  page=2,
  view=FitH 800
]{Page 2, FitH 800}
\bookmark[
  page=2,
  view=FitBH \calc{\paperheight-\topmargin-1in-\headheight-\headsep}
]{Page 2, FitBH top of text body}
\bookmark[
  uri={http://www.dante.de/},
  color=magenta
]{Dante homepage}
\bookmark[
  gotor={t.pdf},
  page=1,
  view={XYZ 0 1000 null},
  color=cyan!75!black
]{File t.pdf}
\bookmark[named=FirstPage]{First page}
\bookmark[rellevel=1, named=LastPage]{Last page (rellevel=1)}
\bookmark[named=PrevPage]{Previous page}
\bookmark[level=0, named=FirstPage]{First page (level=0)}
\bookmark[
  rellevel=1,
  keeplevel,
  named=LastPage
]{Last page (rellevel=1, keeplevel)}
\bookmark[named=PrevPage]{Previous page}
\end{document}
%    \end{macrocode}
%    \begin{macrocode}
%</example>
%    \end{macrocode}
%
% \StopEventually{
% }
%
% \section{实现(Implementation)}
%
% \subsection{宏包(Package)}
%
%    \begin{macrocode}
%<*package>
\NeedsTeXFormat{LaTeX2e}
\ProvidesPackage{bookmark}%
  [2020-11-06 v1.29 PDF bookmarks (HO)]%
%    \end{macrocode}
%
% \subsubsection{要求(Requirements)}
%
% \paragraph{\hologo{eTeX}.}
%
%    \begin{macro}{\BKM@CalcExpr}
%    \begin{macrocode}
\begingroup\expandafter\expandafter\expandafter\endgroup
\expandafter\ifx\csname numexpr\endcsname\relax
  \def\BKM@CalcExpr#1#2#3#4{%
    \begingroup
      \count@=#2\relax
      \advance\count@ by#3#4\relax
      \edef\x{\endgroup
        \def\noexpand#1{\the\count@}%
      }%
    \x
  }%
\else
  \def\BKM@CalcExpr#1#2#3#4{%
    \edef#1{%
      \the\numexpr#2#3#4\relax
    }%
  }%
\fi
%    \end{macrocode}
%    \end{macro}
%
% \paragraph{\hologo{pdfTeX}\ 的转义功能(escape features)}
%
%    \begin{macro}{\BKM@EscapeName}
%    \begin{macrocode}
\def\BKM@EscapeName#1{%
  \ifx#1\@empty
  \else
    \EdefEscapeName#1#1%
  \fi
}%
%    \end{macrocode}
%    \end{macro}
%    \begin{macro}{\BKM@EscapeString}
%    \begin{macrocode}
\def\BKM@EscapeString#1{%
  \ifx#1\@empty
  \else
    \EdefEscapeString#1#1%
  \fi
}%
%    \end{macrocode}
%    \end{macro}
%    \begin{macro}{\BKM@EscapeHex}
%    \begin{macrocode}
\def\BKM@EscapeHex#1{%
  \ifx#1\@empty
  \else
    \EdefEscapeHex#1#1%
  \fi
}%
%    \end{macrocode}
%    \end{macro}
%    \begin{macro}{\BKM@UnescapeHex}
%    \begin{macrocode}
\def\BKM@UnescapeHex#1{%
  \EdefUnescapeHex#1#1%
}%
%    \end{macrocode}
%    \end{macro}
%
% \paragraph{宏包(Packages)。}
%
% 不要加载由 \xpackage{hyperref}\ 加载的宏包
%    \begin{macrocode}
\RequirePackage{hyperref}[2010/06/18]
%    \end{macrocode}
%
% \subsubsection{宏包选项(Package options)}
%
%    \begin{macrocode}
\SetupKeyvalOptions{family=BKM,prefix=BKM@}
\DeclareLocalOptions{%
  atend,%
  bold,%
  color,%
  depth,%
  dest,%
  draft,%
  final,%
  gotor,%
  italic,%
  keeplevel,%
  level,%
  named,%
  numbered,%
  open,%
  openlevel,%
  page,%
  rawaction,%
  rellevel,%
  srcfile,%
  srcline,%
  startatroot,%
  uri,%
  view,%
}
%    \end{macrocode}
%    \begin{macro}{\bookmarksetup}
%    \begin{macrocode}
\newcommand*{\bookmarksetup}{\kvsetkeys{BKM}}
%    \end{macrocode}
%    \end{macro}
%    \begin{macro}{\BKM@setup}
%    \begin{macrocode}
\def\BKM@setup#1{%
  \bookmarksetup{#1}%
  \ifx\BKM@HookNext\ltx@empty
  \else
    \expandafter\bookmarksetup\expandafter{\BKM@HookNext}%
    \BKM@HookNextClear
  \fi
  \BKM@hook
  \ifBKM@keeplevel
  \else
    \xdef\BKM@currentlevel{\BKM@level}%
  \fi
}
%    \end{macrocode}
%    \end{macro}
%
%    \begin{macro}{\bookmarksetupnext}
%    \begin{macrocode}
\newcommand*{\bookmarksetupnext}[1]{%
  \ltx@GlobalAppendToMacro\BKM@HookNext{,#1}%
}
%    \end{macrocode}
%    \end{macro}
%    \begin{macro}{\BKM@setupnext}
%    \begin{macrocode}
%    \end{macrocode}
%    \end{macro}
%    \begin{macro}{\BKM@HookNextClear}
%    \begin{macrocode}
\def\BKM@HookNextClear{%
  \global\let\BKM@HookNext\ltx@empty
}
%    \end{macrocode}
%    \end{macro}
%    \begin{macro}{\BKM@HookNext}
%    \begin{macrocode}
\BKM@HookNextClear
%    \end{macrocode}
%    \end{macro}
%
%    \begin{macrocode}
\DeclareBoolOption{draft}
\DeclareComplementaryOption{final}{draft}
%    \end{macrocode}
%    \begin{macro}{\BKM@DisableOptions}
%    \begin{macrocode}
\def\BKM@DisableOptions{%
  \DisableKeyvalOption[action=warning,package=bookmark]%
      {BKM}{draft}%
  \DisableKeyvalOption[action=warning,package=bookmark]%
      {BKM}{final}%
}
%    \end{macrocode}
%    \end{macro}
%    \begin{macrocode}
\DeclareBoolOption[\ifHy@bookmarksopen true\else false\fi]{open}
%    \end{macrocode}
%    \begin{macro}{\bookmark@open}
%    \begin{macrocode}
\def\bookmark@open{%
  \ifBKM@open\ltx@one\else\ltx@zero\fi
}
%    \end{macrocode}
%    \end{macro}
%    \begin{macrocode}
\DeclareStringOption[\maxdimen]{openlevel}
%    \end{macrocode}
%    \begin{macro}{\BKM@openlevel}
%    \begin{macrocode}
\edef\BKM@openlevel{\number\@bookmarksopenlevel}
%    \end{macrocode}
%    \end{macro}
%    \begin{macrocode}
%\DeclareStringOption[\c@tocdepth]{depth}
\ltx@IfUndefined{Hy@bookmarksdepth}{%
  \def\BKM@depth{\c@tocdepth}%
}{%
  \let\BKM@depth\Hy@bookmarksdepth
}
\define@key{BKM}{depth}[]{%
  \edef\BKM@param{#1}%
  \ifx\BKM@param\@empty
    \def\BKM@depth{\c@tocdepth}%
  \else
    \ltx@IfUndefined{toclevel@\BKM@param}{%
      \@onelevel@sanitize\BKM@param
      \edef\BKM@temp{\expandafter\@car\BKM@param\@nil}%
      \ifcase 0\expandafter\ifx\BKM@temp-1\fi
              \expandafter\ifnum\expandafter`\BKM@temp>47 %
                \expandafter\ifnum\expandafter`\BKM@temp<58 %
                  1%
                \fi
              \fi
              \relax
        \PackageWarning{bookmark}{%
          Unknown document division name (\BKM@param)\MessageBreak
          for option `depth'%
        }%
      \else
        \BKM@SetDepthOrLevel\BKM@depth\BKM@param
      \fi
    }{%
      \BKM@SetDepthOrLevel\BKM@depth{%
        \csname toclevel@\BKM@param\endcsname
      }%
    }%
  \fi
}
%    \end{macrocode}
%    \begin{macro}{\bookmark@depth}
%    \begin{macrocode}
\def\bookmark@depth{\BKM@depth}
%    \end{macrocode}
%    \end{macro}
%    \begin{macro}{\BKM@SetDepthOrLevel}
%    \begin{macrocode}
\def\BKM@SetDepthOrLevel#1#2{%
  \begingroup
    \setbox\z@=\hbox{%
      \count@=#2\relax
      \expandafter
    }%
  \expandafter\endgroup
  \expandafter\def\expandafter#1\expandafter{\the\count@}%
}
%    \end{macrocode}
%    \end{macro}
%    \begin{macrocode}
\DeclareStringOption[\BKM@currentlevel]{level}[\BKM@currentlevel]
\define@key{BKM}{level}{%
  \edef\BKM@param{#1}%
  \ifx\BKM@param\BKM@MacroCurrentLevel
    \let\BKM@level\BKM@param
  \else
    \ltx@IfUndefined{toclevel@\BKM@param}{%
      \@onelevel@sanitize\BKM@param
      \edef\BKM@temp{\expandafter\@car\BKM@param\@nil}%
      \ifcase 0\expandafter\ifx\BKM@temp-1\fi
              \expandafter\ifnum\expandafter`\BKM@temp>47 %
                \expandafter\ifnum\expandafter`\BKM@temp<58 %
                  1%
                \fi
              \fi
              \relax
        \PackageWarning{bookmark}{%
          Unknown document division name (\BKM@param)\MessageBreak
          for option `level'%
        }%
      \else
        \BKM@SetDepthOrLevel\BKM@level\BKM@param
      \fi
    }{%
      \BKM@SetDepthOrLevel\BKM@level{%
        \csname toclevel@\BKM@param\endcsname
      }%
    }%
  \fi
}
%    \end{macrocode}
%    \begin{macro}{\BKM@MacroCurrentLevel}
%    \begin{macrocode}
\def\BKM@MacroCurrentLevel{\BKM@currentlevel}
%    \end{macrocode}
%    \end{macro}
%    \begin{macrocode}
\DeclareBoolOption{keeplevel}
\DeclareBoolOption{startatroot}
%    \end{macrocode}
%    \begin{macro}{\BKM@startatrootfalse}
%    \begin{macrocode}
\def\BKM@startatrootfalse{%
  \global\let\ifBKM@startatroot\iffalse
}
%    \end{macrocode}
%    \end{macro}
%    \begin{macro}{\BKM@startatroottrue}
%    \begin{macrocode}
\def\BKM@startatroottrue{%
  \global\let\ifBKM@startatroot\iftrue
}
%    \end{macrocode}
%    \end{macro}
%    \begin{macrocode}
\define@key{BKM}{rellevel}{%
  \BKM@CalcExpr\BKM@level{#1}+\BKM@currentlevel
}
%    \end{macrocode}
%    \begin{macro}{\bookmark@level}
%    \begin{macrocode}
\def\bookmark@level{\BKM@level}
%    \end{macrocode}
%    \end{macro}
%    \begin{macro}{\BKM@currentlevel}
%    \begin{macrocode}
\def\BKM@currentlevel{0}
%    \end{macrocode}
%    \end{macro}
%    Make \xpackage{bookmark}'s option \xoption{numbered} an alias
%    for \xpackage{hyperref}'s \xoption{bookmarksnumbered}.
%    \begin{macrocode}
\DeclareBoolOption[%
  \ifHy@bookmarksnumbered true\else false\fi
]{numbered}
\g@addto@macro\BKM@numberedtrue{%
  \let\ifHy@bookmarksnumbered\iftrue
}
\g@addto@macro\BKM@numberedfalse{%
  \let\ifHy@bookmarksnumbered\iffalse
}
\g@addto@macro\Hy@bookmarksnumberedtrue{%
  \let\ifBKM@numbered\iftrue
}
\g@addto@macro\Hy@bookmarksnumberedfalse{%
  \let\ifBKM@numbered\iffalse
}
%    \end{macrocode}
%    \begin{macro}{\bookmark@numbered}
%    \begin{macrocode}
\def\bookmark@numbered{%
  \ifBKM@numbered\ltx@one\else\ltx@zero\fi
}
%    \end{macrocode}
%    \end{macro}
%
% \paragraph{重定义 \xpackage{hyperref}\ 宏包的选项}
%
%    \begin{macro}{\BKM@PatchHyperrefOption}
%    \begin{macrocode}
\def\BKM@PatchHyperrefOption#1{%
  \expandafter\BKM@@PatchHyperrefOption\csname KV@Hyp@#1\endcsname%
}
%    \end{macrocode}
%    \end{macro}
%    \begin{macro}{\BKM@@PatchHyperrefOption}
%    \begin{macrocode}
\def\BKM@@PatchHyperrefOption#1{%
  \expandafter\BKM@@@PatchHyperrefOption#1{##1}\BKM@nil#1%
}
%    \end{macrocode}
%    \end{macro}
%    \begin{macro}{\BKM@@@PatchHyperrefOption}
%    \begin{macrocode}
\def\BKM@@@PatchHyperrefOption#1\BKM@nil#2#3{%
  \def#2##1{%
    #1%
    \bookmarksetup{#3={##1}}%
  }%
}
%    \end{macrocode}
%    \end{macro}
%    \begin{macrocode}
\BKM@PatchHyperrefOption{bookmarksopen}{open}
\BKM@PatchHyperrefOption{bookmarksopenlevel}{openlevel}
\BKM@PatchHyperrefOption{bookmarksdepth}{depth}
%    \end{macrocode}
%
% \paragraph{字体样式(font style)选项。}
%
%    注意:\xpackage{bitset}\ 宏是基于零的,PDF 规范(PDF specifications)以1开头。
%    \begin{macrocode}
\bitsetReset{BKM@FontStyle}%
\define@key{BKM}{italic}[true]{%
  \expandafter\ifx\csname if#1\endcsname\iftrue
    \bitsetSet{BKM@FontStyle}{0}%
  \else
    \bitsetClear{BKM@FontStyle}{0}%
  \fi
}%
\define@key{BKM}{bold}[true]{%
  \expandafter\ifx\csname if#1\endcsname\iftrue
    \bitsetSet{BKM@FontStyle}{1}%
  \else
    \bitsetClear{BKM@FontStyle}{1}%
  \fi
}%
%    \end{macrocode}
%    \begin{macro}{\bookmark@italic}
%    \begin{macrocode}
\def\bookmark@italic{%
  \ifnum\bitsetGet{BKM@FontStyle}{0}=1 \ltx@one\else\ltx@zero\fi
}
%    \end{macrocode}
%    \end{macro}
%    \begin{macro}{\bookmark@bold}
%    \begin{macrocode}
\def\bookmark@bold{%
  \ifnum\bitsetGet{BKM@FontStyle}{1}=1 \ltx@one\else\ltx@zero\fi
}
%    \end{macrocode}
%    \end{macro}
%    \begin{macro}{\BKM@PrintStyle}
%    \begin{macrocode}
\def\BKM@PrintStyle{%
  \bitsetGetDec{BKM@FontStyle}%
}%
%    \end{macrocode}
%    \end{macro}
%
% \paragraph{颜色(color)选项。}
%
%    \begin{macrocode}
\define@key{BKM}{color}{%
  \HyColor@BookmarkColor{#1}\BKM@color{bookmark}{color}%
}
%    \end{macrocode}
%    \begin{macro}{\BKM@color}
%    \begin{macrocode}
\let\BKM@color\@empty
%    \end{macrocode}
%    \end{macro}
%    \begin{macro}{\bookmark@color}
%    \begin{macrocode}
\def\bookmark@color{\BKM@color}
%    \end{macrocode}
%    \end{macro}
%
% \subsubsection{动作(action)选项}
%
%    \begin{macrocode}
\def\BKM@temp#1{%
  \DeclareStringOption{#1}%
  \expandafter\edef\csname bookmark@#1\endcsname{%
    \expandafter\noexpand\csname BKM@#1\endcsname
  }%
}
%    \end{macrocode}
%    \begin{macro}{\bookmark@dest}
%    \begin{macrocode}
\BKM@temp{dest}
%    \end{macrocode}
%    \end{macro}
%    \begin{macro}{\bookmark@named}
%    \begin{macrocode}
\BKM@temp{named}
%    \end{macrocode}
%    \end{macro}
%    \begin{macro}{\bookmark@uri}
%    \begin{macrocode}
\BKM@temp{uri}
%    \end{macrocode}
%    \end{macro}
%    \begin{macro}{\bookmark@gotor}
%    \begin{macrocode}
\BKM@temp{gotor}
%    \end{macrocode}
%    \end{macro}
%    \begin{macro}{\bookmark@rawaction}
%    \begin{macrocode}
\BKM@temp{rawaction}
%    \end{macrocode}
%    \end{macro}
%
%    \begin{macrocode}
\define@key{BKM}{page}{%
  \def\BKM@page{#1}%
  \ifx\BKM@page\@empty
  \else
    \edef\BKM@page{\number\BKM@page}%
    \ifnum\BKM@page>\z@
    \else
      \PackageError{bookmark}{Page must be positive}\@ehc
      \def\BKM@page{1}%
    \fi
  \fi
}
%    \end{macrocode}
%    \begin{macro}{\BKM@page}
%    \begin{macrocode}
\let\BKM@page\@empty
%    \end{macrocode}
%    \end{macro}
%    \begin{macro}{\bookmark@page}
%    \begin{macrocode}
\def\bookmark@page{\BKM@@page}
%    \end{macrocode}
%    \end{macro}
%
%    \begin{macrocode}
\define@key{BKM}{view}{%
  \BKM@CheckView{#1}%
}
%    \end{macrocode}
%    \begin{macro}{\BKM@view}
%    \begin{macrocode}
\let\BKM@view\@empty
%    \end{macrocode}
%    \end{macro}
%    \begin{macro}{\bookmark@view}
%    \begin{macrocode}
\def\bookmark@view{\BKM@view}
%    \end{macrocode}
%    \end{macro}
%    \begin{macro}{BKM@CheckView}
%    \begin{macrocode}
\def\BKM@CheckView#1{%
  \BKM@CheckViewType#1 \@nil
}
%    \end{macrocode}
%    \end{macro}
%    \begin{macro}{\BKM@CheckViewType}
%    \begin{macrocode}
\def\BKM@CheckViewType#1 #2\@nil{%
  \def\BKM@type{#1}%
  \@onelevel@sanitize\BKM@type
  \BKM@TestViewType{Fit}{}%
  \BKM@TestViewType{FitB}{}%
  \BKM@TestViewType{FitH}{%
    \BKM@CheckParam#2 \@nil{top}%
  }%
  \BKM@TestViewType{FitBH}{%
    \BKM@CheckParam#2 \@nil{top}%
  }%
  \BKM@TestViewType{FitV}{%
    \BKM@CheckParam#2 \@nil{bottom}%
  }%
  \BKM@TestViewType{FitBV}{%
    \BKM@CheckParam#2 \@nil{bottom}%
  }%
  \BKM@TestViewType{FitR}{%
    \BKM@CheckRect{#2}{ }%
  }%
  \BKM@TestViewType{XYZ}{%
    \BKM@CheckXYZ{#2}{ }%
  }%
  \@car{%
    \PackageError{bookmark}{%
      Unknown view type `\BKM@type',\MessageBreak
      using `FitH' instead%
    }\@ehc
    \def\BKM@view{FitH}%
  }%
  \@nil
}
%    \end{macrocode}
%    \end{macro}
%    \begin{macro}{\BKM@TestViewType}
%    \begin{macrocode}
\def\BKM@TestViewType#1{%
  \def\BKM@temp{#1}%
  \@onelevel@sanitize\BKM@temp
  \ifx\BKM@type\BKM@temp
    \let\BKM@view\BKM@temp
    \expandafter\@car
  \else
    \expandafter\@gobble
  \fi
}
%    \end{macrocode}
%    \end{macro}
%    \begin{macro}{BKM@CheckParam}
%    \begin{macrocode}
\def\BKM@CheckParam#1 #2\@nil#3{%
  \def\BKM@param{#1}%
  \ifx\BKM@param\@empty
    \PackageWarning{bookmark}{%
      Missing parameter (#3) for `\BKM@type',\MessageBreak
      using 0%
    }%
    \def\BKM@param{0}%
  \else
    \BKM@CalcParam
  \fi
  \edef\BKM@view{\BKM@view\space\BKM@param}%
}
%    \end{macrocode}
%    \end{macro}
%    \begin{macro}{BKM@CheckRect}
%    \begin{macrocode}
\def\BKM@CheckRect#1#2{%
  \BKM@@CheckRect#1#2#2#2#2\@nil
}
%    \end{macrocode}
%    \end{macro}
%    \begin{macro}{\BKM@@CheckRect}
%    \begin{macrocode}
\def\BKM@@CheckRect#1 #2 #3 #4 #5\@nil{%
  \def\BKM@temp{0}%
  \def\BKM@param{#1}%
  \ifx\BKM@param\@empty
    \def\BKM@param{0}%
    \def\BKM@temp{1}%
  \else
    \BKM@CalcParam
  \fi
  \edef\BKM@view{\BKM@view\space\BKM@param}%
  \def\BKM@param{#2}%
  \ifx\BKM@param\@empty
    \def\BKM@param{0}%
    \def\BKM@temp{1}%
  \else
    \BKM@CalcParam
  \fi
  \edef\BKM@view{\BKM@view\space\BKM@param}%
  \def\BKM@param{#3}%
  \ifx\BKM@param\@empty
    \def\BKM@param{0}%
    \def\BKM@temp{1}%
  \else
    \BKM@CalcParam
  \fi
  \edef\BKM@view{\BKM@view\space\BKM@param}%
  \def\BKM@param{#4}%
  \ifx\BKM@param\@empty
    \def\BKM@param{0}%
    \def\BKM@temp{1}%
  \else
    \BKM@CalcParam
  \fi
  \edef\BKM@view{\BKM@view\space\BKM@param}%
  \ifnum\BKM@temp>\z@
    \PackageWarning{bookmark}{Missing parameters for `\BKM@type'}%
  \fi
}
%    \end{macrocode}
%    \end{macro}
%    \begin{macro}{\BKM@CheckXYZ}
%    \begin{macrocode}
\def\BKM@CheckXYZ#1#2{%
  \BKM@@CheckXYZ#1#2#2#2\@nil
}
%    \end{macrocode}
%    \end{macro}
%    \begin{macro}{\BKM@@CheckXYZ}
%    \begin{macrocode}
\def\BKM@@CheckXYZ#1 #2 #3 #4\@nil{%
  \def\BKM@param{#1}%
  \let\BKM@temp\BKM@param
  \@onelevel@sanitize\BKM@temp
  \ifx\BKM@param\@empty
    \let\BKM@param\BKM@null
  \else
    \ifx\BKM@temp\BKM@null
    \else
      \BKM@CalcParam
    \fi
  \fi
  \edef\BKM@view{\BKM@view\space\BKM@param}%
  \def\BKM@param{#2}%
  \let\BKM@temp\BKM@param
  \@onelevel@sanitize\BKM@temp
  \ifx\BKM@param\@empty
    \let\BKM@param\BKM@null
  \else
    \ifx\BKM@temp\BKM@null
    \else
      \BKM@CalcParam
    \fi
  \fi
  \edef\BKM@view{\BKM@view\space\BKM@param}%
  \def\BKM@param{#3}%
  \ifx\BKM@param\@empty
    \let\BKM@param\BKM@null
  \fi
  \edef\BKM@view{\BKM@view\space\BKM@param}%
}
%    \end{macrocode}
%    \end{macro}
%    \begin{macro}{\BKM@null}
%    \begin{macrocode}
\def\BKM@null{null}
\@onelevel@sanitize\BKM@null
%    \end{macrocode}
%    \end{macro}
%
%    \begin{macro}{\BKM@CalcParam}
%    \begin{macrocode}
\def\BKM@CalcParam{%
  \begingroup
  \let\calc\@firstofone
  \expandafter\BKM@@CalcParam\BKM@param\@empty\@empty\@nil
}
%    \end{macrocode}
%    \end{macro}
%    \begin{macro}{\BKM@@CalcParam}
%    \begin{macrocode}
\def\BKM@@CalcParam#1#2#3\@nil{%
  \ifx\calc#1%
    \@ifundefined{calc@assign@dimen}{%
      \@ifundefined{dimexpr}{%
        \setlength{\dimen@}{#2}%
      }{%
        \setlength{\dimen@}{\dimexpr#2\relax}%
      }%
    }{%
      \setlength{\dimen@}{#2}%
    }%
    \dimen@.99626\dimen@
    \edef\BKM@param{\strip@pt\dimen@}%
    \expandafter\endgroup
    \expandafter\def\expandafter\BKM@param\expandafter{\BKM@param}%
  \else
    \endgroup
  \fi
}
%    \end{macrocode}
%    \end{macro}
%
% \subsubsection{\xoption{atend}\ 选项}
%
%    \begin{macrocode}
\DeclareBoolOption{atend}
\g@addto@macro\BKM@DisableOptions{%
  \DisableKeyvalOption[action=warning,package=bookmark]%
      {BKM}{atend}%
}
%    \end{macrocode}
%
% \subsubsection{\xoption{style}\ 选项}
%
%    \begin{macro}{\bookmarkdefinestyle}
%    \begin{macrocode}
\newcommand*{\bookmarkdefinestyle}[2]{%
  \@ifundefined{BKM@style@#1}{%
  }{%
    \PackageInfo{bookmark}{Redefining style `#1'}%
  }%
  \@namedef{BKM@style@#1}{#2}%
}
%    \end{macrocode}
%    \end{macro}
%    \begin{macrocode}
\define@key{BKM}{style}{%
  \BKM@StyleCall{#1}%
}
\newif\ifBKM@ok
%    \end{macrocode}
%    \begin{macro}{\BKM@StyleCall}
%    \begin{macrocode}
\def\BKM@StyleCall#1{%
  \@ifundefined{BKM@style@#1}{%
    \PackageWarning{bookmark}{%
      Ignoring unknown style `#1'%
    }%
  }{%
%    \end{macrocode}
%    检查样式堆栈(style stack)。
%    \begin{macrocode}
    \BKM@oktrue
    \edef\BKM@StyleCurrent{#1}%
    \@onelevel@sanitize\BKM@StyleCurrent
    \let\BKM@StyleEntry\BKM@StyleEntryCheck
    \BKM@StyleStack
    \ifBKM@ok
      \expandafter\@firstofone
    \else
      \PackageError{bookmark}{%
        Ignoring recursive call of style `\BKM@StyleCurrent'%
      }\@ehc
      \expandafter\@gobble
    \fi
    {%
%    \end{macrocode}
%    在堆栈上推送当前样式(Push current style on stack)。
%    \begin{macrocode}
      \let\BKM@StyleEntry\relax
      \edef\BKM@StyleStack{%
        \BKM@StyleEntry{\BKM@StyleCurrent}%
        \BKM@StyleStack
      }%
%    \end{macrocode}
%   调用样式(Call style)。
%    \begin{macrocode}
      \expandafter\expandafter\expandafter\bookmarksetup
      \expandafter\expandafter\expandafter{%
        \csname BKM@style@\BKM@StyleCurrent\endcsname
      }%
%    \end{macrocode}
%    从堆栈中弹出当前样式(Pop current style from stack)。
%    \begin{macrocode}
      \BKM@StyleStackPop
    }%
  }%
}
%    \end{macrocode}
%    \end{macro}
%    \begin{macro}{\BKM@StyleStackPop}
%    \begin{macrocode}
\def\BKM@StyleStackPop{%
  \let\BKM@StyleEntry\relax
  \edef\BKM@StyleStack{%
    \expandafter\@gobbletwo\BKM@StyleStack
  }%
}
%    \end{macrocode}
%    \end{macro}
%    \begin{macro}{\BKM@StyleEntryCheck}
%    \begin{macrocode}
\def\BKM@StyleEntryCheck#1{%
  \def\BKM@temp{#1}%
  \ifx\BKM@temp\BKM@StyleCurrent
    \BKM@okfalse
  \fi
}
%    \end{macrocode}
%    \end{macro}
%    \begin{macro}{\BKM@StyleStack}
%    \begin{macrocode}
\def\BKM@StyleStack{}
%    \end{macrocode}
%    \end{macro}
%
% \subsubsection{源文件位置(source file location)选项}
%
%    \begin{macrocode}
\DeclareStringOption{srcline}
\DeclareStringOption{srcfile}
%    \end{macrocode}
%
% \subsubsection{钩子支持(Hook support)}
%
%    \begin{macro}{\BKM@hook}
%    \begin{macrocode}
\def\BKM@hook{}
%    \end{macrocode}
%    \end{macro}
%    \begin{macrocode}
\define@key{BKM}{addtohook}{%
  \ltx@LocalAppendToMacro\BKM@hook{#1}%
}
%    \end{macrocode}
%
%    \begin{macro}{bookmarkget}
%    \begin{macrocode}
\newcommand*{\bookmarkget}[1]{%
  \romannumeral0%
  \ltx@ifundefined{bookmark@#1}{%
    \ltx@space
  }{%
    \expandafter\expandafter\expandafter\ltx@space
    \csname bookmark@#1\endcsname
  }%
}
%    \end{macrocode}
%    \end{macro}
%
% \subsubsection{设置和加载驱动程序}
%
% \paragraph{检测驱动程序。}
%
%    \begin{macro}{\BKM@DefineDriverKey}
%    \begin{macrocode}
\def\BKM@DefineDriverKey#1{%
  \define@key{BKM}{#1}[]{%
    \def\BKM@driver{#1}%
  }%
  \g@addto@macro\BKM@DisableOptions{%
    \DisableKeyvalOption[action=warning,package=bookmark]%
        {BKM}{#1}%
  }%
}
%    \end{macrocode}
%    \end{macro}
%    \begin{macrocode}
\BKM@DefineDriverKey{pdftex}
\BKM@DefineDriverKey{dvips}
\BKM@DefineDriverKey{dvipdfm}
\BKM@DefineDriverKey{dvipdfmx}
\BKM@DefineDriverKey{xetex}
\BKM@DefineDriverKey{vtex}
\define@key{BKM}{dvipdfmx-outline-open}[true]{%
 \PackageWarning{bookmark}{Option 'dvipdfmx-outline-open' is obsolete
   and ignored}{}}
%    \end{macrocode}
%    \begin{macro}{\bookmark@driver}
%    \begin{macrocode}
\def\bookmark@driver{\BKM@driver}
%    \end{macrocode}
%    \end{macro}
%    \begin{macrocode}
\InputIfFileExists{bookmark.cfg}{}{}
%    \end{macrocode}
%    \begin{macro}{\BookmarkDriverDefault}
%    \begin{macrocode}
\providecommand*{\BookmarkDriverDefault}{dvips}
%    \end{macrocode}
%    \end{macro}
%    \begin{macro}{\BKM@driver}
% Lua\TeX\ 和 pdf\TeX\ 共享驱动程序。
%    \begin{macrocode}
\ifpdf
  \def\BKM@driver{pdftex}%
  \ifx\pdfoutline\@undefined
    \ifx\pdfextension\@undefined\else
      \protected\def\pdfoutline{\pdfextension outline }
    \fi
  \fi
\else
  \ifxetex
    \def\BKM@driver{dvipdfm}%
  \else
    \ifvtex
      \def\BKM@driver{vtex}%
    \else
      \edef\BKM@driver{\BookmarkDriverDefault}%
    \fi
  \fi
\fi
%    \end{macrocode}
%    \end{macro}
%
% \paragraph{过程选项(Process options)。}
%
%    \begin{macrocode}
\ProcessKeyvalOptions*
\BKM@DisableOptions
%    \end{macrocode}
%
% \paragraph{\xoption{draft}\ 选项}
%
%    \begin{macrocode}
\ifBKM@draft
  \PackageWarningNoLine{bookmark}{Draft mode on}%
  \let\bookmarksetup\ltx@gobble
  \let\BookmarkAtEnd\ltx@gobble
  \let\bookmarkdefinestyle\ltx@gobbletwo
  \let\bookmarkget\ltx@gobble
  \let\pdfbookmark\ltx@undefined
  \newcommand*{\pdfbookmark}[3][]{}%
  \let\currentpdfbookmark\ltx@gobbletwo
  \let\subpdfbookmark\ltx@gobbletwo
  \let\belowpdfbookmark\ltx@gobbletwo
  \newcommand*{\bookmark}[2][]{}%
  \renewcommand*{\Hy@writebookmark}[5]{}%
  \let\ReadBookmarks\relax
  \let\BKM@DefGotoNameAction\ltx@gobbletwo % package `hypdestopt'
  \expandafter\endinput
\fi
%    \end{macrocode}
%
% \paragraph{验证和加载驱动程序。}
%
%    \begin{macrocode}
\def\BKM@temp{dvipdfmx}%
\ifx\BKM@temp\BKM@driver
  \def\BKM@driver{dvipdfm}%
\fi
\def\BKM@temp{pdftex}%
\ifpdf
  \ifx\BKM@temp\BKM@driver
  \else
    \PackageWarningNoLine{bookmark}{%
      Wrong driver `\BKM@driver', using `pdftex' instead%
    }%
    \let\BKM@driver\BKM@temp
  \fi
\else
  \ifx\BKM@temp\BKM@driver
    \PackageError{bookmark}{%
      Wrong driver, pdfTeX is not running in PDF mode.\MessageBreak
      Package loading is aborted%
    }\@ehc
    \expandafter\expandafter\expandafter\endinput
  \fi
  \def\BKM@temp{dvipdfm}%
  \ifxetex
    \ifx\BKM@temp\BKM@driver
    \else
      \PackageWarningNoLine{bookmark}{%
        Wrong driver `\BKM@driver',\MessageBreak
        using `dvipdfm' for XeTeX instead%
      }%
      \let\BKM@driver\BKM@temp
    \fi
  \else
    \def\BKM@temp{vtex}%
    \ifvtex
      \ifx\BKM@temp\BKM@driver
      \else
        \PackageWarningNoLine{bookmark}{%
          Wrong driver `\BKM@driver',\MessageBreak
          using `vtex' for VTeX instead%
        }%
        \let\BKM@driver\BKM@temp
      \fi
    \else
      \ifx\BKM@temp\BKM@driver
        \PackageError{bookmark}{%
          Wrong driver, VTeX is not running in PDF mode.\MessageBreak
          Package loading is aborted%
        }\@ehc
        \expandafter\expandafter\expandafter\endinput
      \fi
    \fi
  \fi
\fi
\providecommand\IfFormatAtLeastTF{\@ifl@t@r\fmtversion}
\IfFormatAtLeastTF{2020/10/01}{}{\edef\BKM@driver{\BKM@driver-2019-12-03}}
\InputIfFileExists{bkm-\BKM@driver.def}{}{%
  \PackageError{bookmark}{%
    Unsupported driver `\BKM@driver'.\MessageBreak
    Package loading is aborted%
  }\@ehc
  \endinput
}
%    \end{macrocode}
%
% \subsubsection{与 \xpackage{hyperref}\ 的兼容性}
%
%    \begin{macro}{\pdfbookmark}
%    \begin{macrocode}
\let\pdfbookmark\ltx@undefined
\newcommand*{\pdfbookmark}[3][0]{%
  \bookmark[level=#1,dest={#3.#1}]{#2}%
  \hyper@anchorstart{#3.#1}\hyper@anchorend
}
%    \end{macrocode}
%    \end{macro}
%    \begin{macro}{\currentpdfbookmark}
%    \begin{macrocode}
\def\currentpdfbookmark{%
  \pdfbookmark[\BKM@currentlevel]%
}
%    \end{macrocode}
%    \end{macro}
%    \begin{macro}{\subpdfbookmark}
%    \begin{macrocode}
\def\subpdfbookmark{%
  \BKM@CalcExpr\BKM@CalcResult\BKM@currentlevel+1%
  \expandafter\pdfbookmark\expandafter[\BKM@CalcResult]%
}
%    \end{macrocode}
%    \end{macro}
%    \begin{macro}{\belowpdfbookmark}
%    \begin{macrocode}
\def\belowpdfbookmark#1#2{%
  \xdef\BKM@gtemp{\number\BKM@currentlevel}%
  \subpdfbookmark{#1}{#2}%
  \global\let\BKM@currentlevel\BKM@gtemp
}
%    \end{macrocode}
%    \end{macro}
%
%    节号(section number)、文本(text)、标签(label)、级别(level)、文件(file)
%    \begin{macro}{\Hy@writebookmark}
%    \begin{macrocode}
\def\Hy@writebookmark#1#2#3#4#5{%
  \ifnum#4>\BKM@depth\relax
  \else
    \def\BKM@type{#5}%
    \ifx\BKM@type\Hy@bookmarkstype
      \begingroup
        \ifBKM@numbered
          \let\numberline\Hy@numberline
          \let\booknumberline\Hy@numberline
          \let\partnumberline\Hy@numberline
          \let\chapternumberline\Hy@numberline
        \else
          \let\numberline\@gobble
          \let\booknumberline\@gobble
          \let\partnumberline\@gobble
          \let\chapternumberline\@gobble
        \fi
        \bookmark[level=#4,dest={\HyperDestNameFilter{#3}}]{#2}%
      \endgroup
    \fi
  \fi
}
%    \end{macrocode}
%    \end{macro}
%
%    \begin{macro}{\ReadBookmarks}
%    \begin{macrocode}
\let\ReadBookmarks\relax
%    \end{macrocode}
%    \end{macro}
%
%    \begin{macrocode}
%</package>
%    \end{macrocode}
%
% \subsection{dvipdfm 的驱动程序}
%
%    \begin{macrocode}
%<*dvipdfm>
\NeedsTeXFormat{LaTeX2e}
\ProvidesFile{bkm-dvipdfm.def}%
  [2020-11-06 v1.29 bookmark driver for dvipdfm (HO)]%
%    \end{macrocode}
%
%    \begin{macro}{\BKM@id}
%    \begin{macrocode}
\newcount\BKM@id
\BKM@id=\z@
%    \end{macrocode}
%    \end{macro}
%
%    \begin{macro}{\BKM@0}
%    \begin{macrocode}
\@namedef{BKM@0}{000}
%    \end{macrocode}
%    \end{macro}
%    \begin{macro}{\ifBKM@sw}
%    \begin{macrocode}
\newif\ifBKM@sw
%    \end{macrocode}
%    \end{macro}
%
%    \begin{macro}{\bookmark}
%    \begin{macrocode}
\newcommand*{\bookmark}[2][]{%
  \if@filesw
    \begingroup
      \def\bookmark@text{#2}%
      \BKM@setup{#1}%
      \edef\BKM@prev{\the\BKM@id}%
      \global\advance\BKM@id\@ne
      \BKM@swtrue
      \@whilesw\ifBKM@sw\fi{%
        \def\BKM@abslevel{1}%
        \ifnum\ifBKM@startatroot\z@\else\BKM@prev\fi=\z@
          \BKM@startatrootfalse
          \expandafter\xdef\csname BKM@\the\BKM@id\endcsname{%
            0{\BKM@level}\BKM@abslevel
          }%
          \BKM@swfalse
        \else
          \expandafter\expandafter\expandafter\BKM@getx
              \csname BKM@\BKM@prev\endcsname
          \ifnum\BKM@level>\BKM@x@level\relax
            \BKM@CalcExpr\BKM@abslevel\BKM@x@abslevel+1%
            \expandafter\xdef\csname BKM@\the\BKM@id\endcsname{%
              {\BKM@prev}{\BKM@level}\BKM@abslevel
            }%
            \BKM@swfalse
          \else
            \let\BKM@prev\BKM@x@parent
          \fi
        \fi
      }%
      \csname HyPsd@XeTeXBigCharstrue\endcsname
      \pdfstringdef\BKM@title{\bookmark@text}%
      \edef\BKM@FLAGS{\BKM@PrintStyle}%
      \let\BKM@action\@empty
      \ifx\BKM@gotor\@empty
        \ifx\BKM@dest\@empty
          \ifx\BKM@named\@empty
            \ifx\BKM@rawaction\@empty
              \ifx\BKM@uri\@empty
                \ifx\BKM@page\@empty
                  \PackageError{bookmark}{Missing action}\@ehc
                  \edef\BKM@action{/Dest[@page1/Fit]}%
                \else
                  \ifx\BKM@view\@empty
                    \def\BKM@view{Fit}%
                  \fi
                  \edef\BKM@action{/Dest[@page\BKM@page/\BKM@view]}%
                \fi
              \else
                \BKM@EscapeString\BKM@uri
                \edef\BKM@action{%
                  /A<<%
                    /S/URI%
                    /URI(\BKM@uri)%
                  >>%
                }%
              \fi
            \else
              \edef\BKM@action{/A<<\BKM@rawaction>>}%
            \fi
          \else
            \BKM@EscapeName\BKM@named
            \edef\BKM@action{%
              /A<</S/Named/N/\BKM@named>>%
            }%
          \fi
        \else
          \BKM@EscapeString\BKM@dest
          \edef\BKM@action{%
            /A<<%
              /S/GoTo%
              /D(\BKM@dest)%
            >>%
          }%
        \fi
      \else
        \ifx\BKM@dest\@empty
          \ifx\BKM@page\@empty
            \def\BKM@page{0}%
          \else
            \BKM@CalcExpr\BKM@page\BKM@page-1%
          \fi
          \ifx\BKM@view\@empty
            \def\BKM@view{Fit}%
          \fi
          \edef\BKM@action{/D[\BKM@page/\BKM@view]}%
        \else
          \BKM@EscapeString\BKM@dest
          \edef\BKM@action{/D(\BKM@dest)}%
        \fi
        \BKM@EscapeString\BKM@gotor
        \edef\BKM@action{%
          /A<<%
            /S/GoToR%
            /F(\BKM@gotor)%
            \BKM@action
          >>%
        }%
      \fi
      \special{pdf:%
        out
              [%
              \ifBKM@open
                \ifnum\BKM@level<%
                    \expandafter\ltx@firstofone\expandafter
                    {\number\BKM@openlevel} %
                \else
                  -%
                \fi
              \else
                -%
              \fi
              ] %
            \BKM@abslevel
        <<%
          /Title(\BKM@title)%
          \ifx\BKM@color\@empty
          \else
            /C[\BKM@color]%
          \fi
          \ifnum\BKM@FLAGS>\z@
            /F \BKM@FLAGS
          \fi
          \BKM@action
        >>%
      }%
    \endgroup
  \fi
}
%    \end{macrocode}
%    \end{macro}
%    \begin{macro}{\BKM@getx}
%    \begin{macrocode}
\def\BKM@getx#1#2#3{%
  \def\BKM@x@parent{#1}%
  \def\BKM@x@level{#2}%
  \def\BKM@x@abslevel{#3}%
}
%    \end{macrocode}
%    \end{macro}
%
%    \begin{macrocode}
%</dvipdfm>
%    \end{macrocode}
%
% \subsection{\hologo{VTeX}\ 的驱动程序}
%
%    \begin{macrocode}
%<*vtex>
\NeedsTeXFormat{LaTeX2e}
\ProvidesFile{bkm-vtex.def}%
  [2020-11-06 v1.29 bookmark driver for VTeX (HO)]%
%    \end{macrocode}
%
%    \begin{macrocode}
\ifvtexpdf
\else
  \PackageWarningNoLine{bookmark}{%
    The VTeX driver only supports PDF mode%
  }%
\fi
%    \end{macrocode}
%
%    \begin{macro}{\BKM@id}
%    \begin{macrocode}
\newcount\BKM@id
\BKM@id=\z@
%    \end{macrocode}
%    \end{macro}
%
%    \begin{macro}{\BKM@0}
%    \begin{macrocode}
\@namedef{BKM@0}{00}
%    \end{macrocode}
%    \end{macro}
%    \begin{macro}{\ifBKM@sw}
%    \begin{macrocode}
\newif\ifBKM@sw
%    \end{macrocode}
%    \end{macro}
%
%    \begin{macro}{\bookmark}
%    \begin{macrocode}
\newcommand*{\bookmark}[2][]{%
  \if@filesw
    \begingroup
      \def\bookmark@text{#2}%
      \BKM@setup{#1}%
      \edef\BKM@prev{\the\BKM@id}%
      \global\advance\BKM@id\@ne
      \BKM@swtrue
      \@whilesw\ifBKM@sw\fi{%
        \ifnum\ifBKM@startatroot\z@\else\BKM@prev\fi=\z@
          \BKM@startatrootfalse
          \def\BKM@parent{0}%
          \expandafter\xdef\csname BKM@\the\BKM@id\endcsname{%
            0{\BKM@level}%
          }%
          \BKM@swfalse
        \else
          \expandafter\expandafter\expandafter\BKM@getx
              \csname BKM@\BKM@prev\endcsname
          \ifnum\BKM@level>\BKM@x@level\relax
            \let\BKM@parent\BKM@prev
            \expandafter\xdef\csname BKM@\the\BKM@id\endcsname{%
              {\BKM@prev}{\BKM@level}%
            }%
            \BKM@swfalse
          \else
            \let\BKM@prev\BKM@x@parent
          \fi
        \fi
      }%
      \pdfstringdef\BKM@title{\bookmark@text}%
      \BKM@vtex@title
      \edef\BKM@FLAGS{\BKM@PrintStyle}%
      \let\BKM@action\@empty
      \ifx\BKM@gotor\@empty
        \ifx\BKM@dest\@empty
          \ifx\BKM@named\@empty
            \ifx\BKM@rawaction\@empty
              \ifx\BKM@uri\@empty
                \ifx\BKM@page\@empty
                  \PackageError{bookmark}{Missing action}\@ehc
                  \def\BKM@action{!1}%
                \else
                  \edef\BKM@action{!\BKM@page}%
                \fi
              \else
                \BKM@EscapeString\BKM@uri
                \edef\BKM@action{%
                  <u=%
                    /S/URI%
                    /URI(\BKM@uri)%
                  >%
                }%
              \fi
            \else
              \edef\BKM@action{<u=\BKM@rawaction>}%
            \fi
          \else
            \BKM@EscapeName\BKM@named
            \edef\BKM@action{%
              <u=%
                /S/Named%
                /N/\BKM@named
              >%
            }%
          \fi
        \else
          \BKM@EscapeString\BKM@dest
          \edef\BKM@action{\BKM@dest}%
        \fi
      \else
        \ifx\BKM@dest\@empty
          \ifx\BKM@page\@empty
            \def\BKM@page{1}%
          \fi
          \ifx\BKM@view\@empty
            \def\BKM@view{Fit}%
          \fi
          \edef\BKM@action{/D[\BKM@page/\BKM@view]}%
        \else
          \BKM@EscapeString\BKM@dest
          \edef\BKM@action{/D(\BKM@dest)}%
        \fi
        \BKM@EscapeString\BKM@gotor
        \edef\BKM@action{%
          <u=%
            /S/GoToR%
            /F(\BKM@gotor)%
            \BKM@action
          >>%
        }%
      \fi
      \ifx\BKM@color\@empty
        \let\BKM@RGBcolor\@empty
      \else
        \expandafter\BKM@toRGB\BKM@color\@nil
      \fi
      \special{%
        !outline \BKM@action;%
        p=\BKM@parent,%
        i=\number\BKM@id,%
        s=%
          \ifBKM@open
            \ifnum\BKM@level<\BKM@openlevel
              o%
            \else
              c%
            \fi
          \else
            c%
          \fi,%
        \ifx\BKM@RGBcolor\@empty
        \else
          c=\BKM@RGBcolor,%
        \fi
        \ifnum\BKM@FLAGS>\z@
          f=\BKM@FLAGS,%
        \fi
        t=\BKM@title
      }%
    \endgroup
  \fi
}
%    \end{macrocode}
%    \end{macro}
%    \begin{macro}{\BKM@getx}
%    \begin{macrocode}
\def\BKM@getx#1#2{%
  \def\BKM@x@parent{#1}%
  \def\BKM@x@level{#2}%
}
%    \end{macrocode}
%    \end{macro}
%    \begin{macro}{\BKM@toRGB}
%    \begin{macrocode}
\def\BKM@toRGB#1 #2 #3\@nil{%
  \let\BKM@RGBcolor\@empty
  \BKM@toRGBComponent{#1}%
  \BKM@toRGBComponent{#2}%
  \BKM@toRGBComponent{#3}%
}
%    \end{macrocode}
%    \end{macro}
%    \begin{macro}{\BKM@toRGBComponent}
%    \begin{macrocode}
\def\BKM@toRGBComponent#1{%
  \dimen@=#1pt\relax
  \ifdim\dimen@>\z@
    \ifdim\dimen@<\p@
      \dimen@=255\dimen@
      \advance\dimen@ by 32768sp\relax
      \divide\dimen@ by 65536\relax
      \dimen@ii=\dimen@
      \divide\dimen@ii by 16\relax
      \edef\BKM@RGBcolor{%
        \BKM@RGBcolor
        \BKM@toHexDigit\dimen@ii
      }%
      \dimen@ii=16\dimen@ii
      \advance\dimen@-\dimen@ii
      \edef\BKM@RGBcolor{%
        \BKM@RGBcolor
        \BKM@toHexDigit\dimen@
      }%
    \else
      \edef\BKM@RGBcolor{\BKM@RGBcolor FF}%
    \fi
  \else
    \edef\BKM@RGBcolor{\BKM@RGBcolor00}%
  \fi
}
%    \end{macrocode}
%    \end{macro}
%    \begin{macro}{\BKM@toHexDigit}
%    \begin{macrocode}
\def\BKM@toHexDigit#1{%
  \ifcase\expandafter\@firstofone\expandafter{\number#1} %
    0\or 1\or 2\or 3\or 4\or 5\or 6\or 7\or
    8\or 9\or A\or B\or C\or D\or E\or F%
  \fi
}
%    \end{macrocode}
%    \end{macro}
%    \begin{macrocode}
\begingroup
  \catcode`\|=0 %
  \catcode`\\=12 %
%    \end{macrocode}
%    \begin{macro}{\BKM@vtex@title}
%    \begin{macrocode}
  |gdef|BKM@vtex@title{%
    |@onelevel@sanitize|BKM@title
    |edef|BKM@title{|expandafter|BKM@vtex@leftparen|BKM@title\(|@nil}%
    |edef|BKM@title{|expandafter|BKM@vtex@rightparen|BKM@title\)|@nil}%
    |edef|BKM@title{|expandafter|BKM@vtex@zero|BKM@title\0|@nil}%
    |edef|BKM@title{|expandafter|BKM@vtex@one|BKM@title\1|@nil}%
    |edef|BKM@title{|expandafter|BKM@vtex@two|BKM@title\2|@nil}%
    |edef|BKM@title{|expandafter|BKM@vtex@three|BKM@title\3|@nil}%
  }%
%    \end{macrocode}
%    \end{macro}
%    \begin{macro}{\BKM@vtex@leftparen}
%    \begin{macrocode}
  |gdef|BKM@vtex@leftparen#1\(#2|@nil{%
    #1%
    |ifx||#2||%
    |else
      (%
      |ltx@ReturnAfterFi{%
        |BKM@vtex@leftparen#2|@nil
      }%
    |fi
  }%
%    \end{macrocode}
%    \end{macro}
%    \begin{macro}{\BKM@vtex@rightparen}
%    \begin{macrocode}
  |gdef|BKM@vtex@rightparen#1\)#2|@nil{%
    #1%
    |ifx||#2||%
    |else
      )%
      |ltx@ReturnAfterFi{%
        |BKM@vtex@rightparen#2|@nil
      }%
    |fi
  }%
%    \end{macrocode}
%    \end{macro}
%    \begin{macro}{\BKM@vtex@zero}
%    \begin{macrocode}
  |gdef|BKM@vtex@zero#1\0#2|@nil{%
    #1%
    |ifx||#2||%
    |else
      |noexpand|hv@pdf@char0%
      |ltx@ReturnAfterFi{%
        |BKM@vtex@zero#2|@nil
      }%
    |fi
  }%
%    \end{macrocode}
%    \end{macro}
%    \begin{macro}{\BKM@vtex@one}
%    \begin{macrocode}
  |gdef|BKM@vtex@one#1\1#2|@nil{%
    #1%
    |ifx||#2||%
    |else
      |noexpand|hv@pdf@char1%
      |ltx@ReturnAfterFi{%
        |BKM@vtex@one#2|@nil
      }%
    |fi
  }%
%    \end{macrocode}
%    \end{macro}
%    \begin{macro}{\BKM@vtex@two}
%    \begin{macrocode}
  |gdef|BKM@vtex@two#1\2#2|@nil{%
    #1%
    |ifx||#2||%
    |else
      |noexpand|hv@pdf@char2%
      |ltx@ReturnAfterFi{%
        |BKM@vtex@two#2|@nil
      }%
    |fi
  }%
%    \end{macrocode}
%    \end{macro}
%    \begin{macro}{\BKM@vtex@three}
%    \begin{macrocode}
  |gdef|BKM@vtex@three#1\3#2|@nil{%
    #1%
    |ifx||#2||%
    |else
      |noexpand|hv@pdf@char3%
      |ltx@ReturnAfterFi{%
        |BKM@vtex@three#2|@nil
      }%
    |fi
  }%
%    \end{macrocode}
%    \end{macro}
%    \begin{macrocode}
|endgroup
%    \end{macrocode}
%
%    \begin{macrocode}
%</vtex>
%    \end{macrocode}
%
% \subsection{\hologo{pdfTeX}\ 的驱动程序}
%
%    \begin{macrocode}
%<*pdftex>
\NeedsTeXFormat{LaTeX2e}
\ProvidesFile{bkm-pdftex.def}%
  [2020-11-06 v1.29 bookmark driver for pdfTeX (HO)]%
%    \end{macrocode}
%
%    \begin{macro}{\BKM@DO@entry}
%    \begin{macrocode}
\def\BKM@DO@entry#1#2{%
  \begingroup
    \kvsetkeys{BKM@DO}{#1}%
    \def\BKM@DO@title{#2}%
    \ifx\BKM@DO@srcfile\@empty
    \else
      \BKM@UnescapeHex\BKM@DO@srcfile
    \fi
    \BKM@UnescapeHex\BKM@DO@title
    \expandafter\expandafter\expandafter\BKM@getx
        \csname BKM@\BKM@DO@id\endcsname\@empty\@empty
    \let\BKM@attr\@empty
    \ifx\BKM@DO@flags\@empty
    \else
      \edef\BKM@attr{\BKM@attr/F \BKM@DO@flags}%
    \fi
    \ifx\BKM@DO@color\@empty
    \else
      \edef\BKM@attr{\BKM@attr/C[\BKM@DO@color]}%
    \fi
    \ifx\BKM@attr\@empty
    \else
      \edef\BKM@attr{attr{\BKM@attr}}%
    \fi
    \let\BKM@action\@empty
    \ifx\BKM@DO@gotor\@empty
      \ifx\BKM@DO@dest\@empty
        \ifx\BKM@DO@named\@empty
          \ifx\BKM@DO@rawaction\@empty
            \ifx\BKM@DO@uri\@empty
              \ifx\BKM@DO@page\@empty
                \PackageError{bookmark}{%
                  Missing action\BKM@SourceLocation
                }\@ehc
                \edef\BKM@action{goto page1{/Fit}}%
              \else
                \ifx\BKM@DO@view\@empty
                  \def\BKM@DO@view{Fit}%
                \fi
                \edef\BKM@action{goto page\BKM@DO@page{/\BKM@DO@view}}%
              \fi
            \else
              \BKM@UnescapeHex\BKM@DO@uri
              \BKM@EscapeString\BKM@DO@uri
              \edef\BKM@action{user{<</S/URI/URI(\BKM@DO@uri)>>}}%
            \fi
          \else
            \BKM@UnescapeHex\BKM@DO@rawaction
            \edef\BKM@action{%
              user{%
                <<%
                  \BKM@DO@rawaction
                >>%
              }%
            }%
          \fi
        \else
          \BKM@EscapeName\BKM@DO@named
          \edef\BKM@action{%
            user{<</S/Named/N/\BKM@DO@named>>}%
          }%
        \fi
      \else
        \BKM@UnescapeHex\BKM@DO@dest
        \BKM@DefGotoNameAction\BKM@action\BKM@DO@dest
      \fi
    \else
      \ifx\BKM@DO@dest\@empty
        \ifx\BKM@DO@page\@empty
          \def\BKM@DO@page{0}%
        \else
          \BKM@CalcExpr\BKM@DO@page\BKM@DO@page-1%
        \fi
        \ifx\BKM@DO@view\@empty
          \def\BKM@DO@view{Fit}%
        \fi
        \edef\BKM@action{/D[\BKM@DO@page/\BKM@DO@view]}%
      \else
        \BKM@UnescapeHex\BKM@DO@dest
        \BKM@EscapeString\BKM@DO@dest
        \edef\BKM@action{/D(\BKM@DO@dest)}%
      \fi
      \BKM@UnescapeHex\BKM@DO@gotor
      \BKM@EscapeString\BKM@DO@gotor
      \edef\BKM@action{%
        user{%
          <<%
            /S/GoToR%
            /F(\BKM@DO@gotor)%
            \BKM@action
          >>%
        }%
      }%
    \fi
    \pdfoutline\BKM@attr\BKM@action
                count\ifBKM@DO@open\else-\fi\BKM@x@childs
                {\BKM@DO@title}%
  \endgroup
}
%    \end{macrocode}
%    \end{macro}
%    \begin{macro}{\BKM@DefGotoNameAction}
%    \cs{BKM@DefGotoNameAction}\ 宏是一个用于 \xpackage{hypdestopt}\ 宏包的钩子(hook)。
%    \begin{macrocode}
\def\BKM@DefGotoNameAction#1#2{%
  \BKM@EscapeString\BKM@DO@dest
  \edef#1{goto name{#2}}%
}
%    \end{macrocode}
%    \end{macro}
%    \begin{macrocode}
%</pdftex>
%    \end{macrocode}
%
%    \begin{macrocode}
%<*pdftex|pdfmark>
%    \end{macrocode}
%    \begin{macro}{\BKM@SourceLocation}
%    \begin{macrocode}
\def\BKM@SourceLocation{%
  \ifx\BKM@DO@srcfile\@empty
    \ifx\BKM@DO@srcline\@empty
    \else
      .\MessageBreak
      Source: line \BKM@DO@srcline
    \fi
  \else
    \ifx\BKM@DO@srcline\@empty
      .\MessageBreak
      Source: file `\BKM@DO@srcfile'%
    \else
      .\MessageBreak
      Source: file `\BKM@DO@srcfile', line \BKM@DO@srcline
    \fi
  \fi
}
%    \end{macrocode}
%    \end{macro}
%    \begin{macrocode}
%</pdftex|pdfmark>
%    \end{macrocode}
%
% \subsection{具有 pdfmark 特色(specials)的驱动程序}
%
% \subsubsection{dvips 驱动程序}
%
%    \begin{macrocode}
%<*dvips>
\NeedsTeXFormat{LaTeX2e}
\ProvidesFile{bkm-dvips.def}%
  [2020-11-06 v1.29 bookmark driver for dvips (HO)]%
%    \end{macrocode}
%    \begin{macro}{\BKM@PSHeaderFile}
%    \begin{macrocode}
\def\BKM@PSHeaderFile#1{%
  \special{PSfile=#1}%
}
%    \end{macrocode}
%    \begin{macro}{\BKM@filename}
%    \begin{macrocode}
\def\BKM@filename{\jobname.out.ps}
%    \end{macrocode}
%    \end{macro}
%    \begin{macrocode}
\AddToHook{shipout/lastpage}{%
  \BKM@pdfmark@out
  \BKM@PSHeaderFile\BKM@filename
  }
%    \end{macrocode}
%    \end{macro}
%    \begin{macrocode}
%</dvips>
%    \end{macrocode}
%
% \subsubsection{公共部分(Common part)}
%
%    \begin{macrocode}
%<*pdfmark>
%    \end{macrocode}
%
%    \begin{macro}{\BKM@pdfmark@out}
%    不要在这里使用 \xpackage{rerunfilecheck}\ 宏包,因为在 \hologo{TeX}\ 运行期间不会
%    读取 \cs{BKM@filename}\ 文件。
%    \begin{macrocode}
\def\BKM@pdfmark@out{%
  \if@filesw
    \newwrite\BKM@file
    \immediate\openout\BKM@file=\BKM@filename\relax
    \BKM@write{\@percentchar!}%
    \BKM@write{/pdfmark where{pop}}%
    \BKM@write{%
      {%
        /globaldict where{pop globaldict}{userdict}ifelse%
        /pdfmark/cleartomark load put%
      }%
    }%
    \BKM@write{ifelse}%
  \else
    \let\BKM@write\@gobble
    \let\BKM@DO@entry\@gobbletwo
  \fi
}
%    \end{macrocode}
%    \end{macro}
%    \begin{macro}{\BKM@write}
%    \begin{macrocode}
\def\BKM@write#{%
  \immediate\write\BKM@file
}
%    \end{macrocode}
%    \end{macro}
%
%    \begin{macro}{\BKM@DO@entry}
%    Pdfmark 的规范(specification)说明 |/Color| 是颜色(color)的键名(key name),
%    但是 ghostscript 只将键(key)传递到 PDF 文件中,因此键名必须是 |/C|。
%    \begin{macrocode}
\def\BKM@DO@entry#1#2{%
  \begingroup
    \kvsetkeys{BKM@DO}{#1}%
    \ifx\BKM@DO@srcfile\@empty
    \else
      \BKM@UnescapeHex\BKM@DO@srcfile
    \fi
    \def\BKM@DO@title{#2}%
    \BKM@UnescapeHex\BKM@DO@title
    \expandafter\expandafter\expandafter\BKM@getx
        \csname BKM@\BKM@DO@id\endcsname\@empty\@empty
    \let\BKM@attr\@empty
    \ifx\BKM@DO@flags\@empty
    \else
      \edef\BKM@attr{\BKM@attr/F \BKM@DO@flags}%
    \fi
    \ifx\BKM@DO@color\@empty
    \else
      \edef\BKM@attr{\BKM@attr/C[\BKM@DO@color]}%
    \fi
    \let\BKM@action\@empty
    \ifx\BKM@DO@gotor\@empty
      \ifx\BKM@DO@dest\@empty
        \ifx\BKM@DO@named\@empty
          \ifx\BKM@DO@rawaction\@empty
            \ifx\BKM@DO@uri\@empty
              \ifx\BKM@DO@page\@empty
                \PackageError{bookmark}{%
                  Missing action\BKM@SourceLocation
                }\@ehc
                \edef\BKM@action{%
                  /Action/GoTo%
                  /Page 1%
                  /View[/Fit]%
                }%
              \else
                \ifx\BKM@DO@view\@empty
                  \def\BKM@DO@view{Fit}%
                \fi
                \edef\BKM@action{%
                  /Action/GoTo%
                  /Page \BKM@DO@page
                  /View[/\BKM@DO@view]%
                }%
              \fi
            \else
              \BKM@UnescapeHex\BKM@DO@uri
              \BKM@EscapeString\BKM@DO@uri
              \edef\BKM@action{%
                /Action<<%
                  /Subtype/URI%
                  /URI(\BKM@DO@uri)%
                >>%
              }%
            \fi
          \else
            \BKM@UnescapeHex\BKM@DO@rawaction
            \edef\BKM@action{%
              /Action<<%
                \BKM@DO@rawaction
              >>%
            }%
          \fi
        \else
          \BKM@EscapeName\BKM@DO@named
          \edef\BKM@action{%
            /Action<<%
              /Subtype/Named%
              /N/\BKM@DO@named
            >>%
          }%
        \fi
      \else
        \BKM@UnescapeHex\BKM@DO@dest
        \BKM@EscapeString\BKM@DO@dest
        \edef\BKM@action{%
          /Action/GoTo%
          /Dest(\BKM@DO@dest)cvn%
        }%
      \fi
    \else
      \ifx\BKM@DO@dest\@empty
        \ifx\BKM@DO@page\@empty
          \def\BKM@DO@page{1}%
        \fi
        \ifx\BKM@DO@view\@empty
          \def\BKM@DO@view{Fit}%
        \fi
        \edef\BKM@action{%
          /Page \BKM@DO@page
          /View[/\BKM@DO@view]%
        }%
      \else
        \BKM@UnescapeHex\BKM@DO@dest
        \BKM@EscapeString\BKM@DO@dest
        \edef\BKM@action{%
          /Dest(\BKM@DO@dest)cvn%
        }%
      \fi
      \BKM@UnescapeHex\BKM@DO@gotor
      \BKM@EscapeString\BKM@DO@gotor
      \edef\BKM@action{%
        /Action/GoToR%
        /File(\BKM@DO@gotor)%
        \BKM@action
      }%
    \fi
    \BKM@write{[}%
    \BKM@write{/Title(\BKM@DO@title)}%
    \ifnum\BKM@x@childs>\z@
      \BKM@write{/Count \ifBKM@DO@open\else-\fi\BKM@x@childs}%
    \fi
    \ifx\BKM@attr\@empty
    \else
      \BKM@write{\BKM@attr}%
    \fi
    \BKM@write{\BKM@action}%
    \BKM@write{/OUT pdfmark}%
  \endgroup
}
%    \end{macrocode}
%    \end{macro}
%    \begin{macrocode}
%</pdfmark>
%    \end{macrocode}
%
% \subsection{\xoption{pdftex}\ 和 \xoption{pdfmark}\ 的公共部分}
%
%    \begin{macrocode}
%<*pdftex|pdfmark>
%    \end{macrocode}
%
% \subsubsection{写入辅助文件(auxiliary file)}
%
%    \begin{macrocode}
\AddToHook{begindocument}{%
 \immediate\write\@mainaux{\string\providecommand\string\BKM@entry[2]{}}}
%    \end{macrocode}
%
%    \begin{macro}{\BKM@id}
%    \begin{macrocode}
\newcount\BKM@id
\BKM@id=\z@
%    \end{macrocode}
%    \end{macro}
%
%    \begin{macro}{\BKM@0}
%    \begin{macrocode}
\@namedef{BKM@0}{000}
%    \end{macrocode}
%    \end{macro}
%    \begin{macro}{\ifBKM@sw}
%    \begin{macrocode}
\newif\ifBKM@sw
%    \end{macrocode}
%    \end{macro}
%
%    \begin{macro}{\bookmark}
%    \begin{macrocode}
\newcommand*{\bookmark}[2][]{%
  \if@filesw
    \begingroup
      \BKM@InitSourceLocation
      \def\bookmark@text{#2}%
      \BKM@setup{#1}%
      \ifx\BKM@srcfile\@empty
      \else
        \BKM@EscapeHex\BKM@srcfile
      \fi
      \edef\BKM@prev{\the\BKM@id}%
      \global\advance\BKM@id\@ne
      \BKM@swtrue
      \@whilesw\ifBKM@sw\fi{%
        \ifnum\ifBKM@startatroot\z@\else\BKM@prev\fi=\z@
          \BKM@startatrootfalse
          \expandafter\xdef\csname BKM@\the\BKM@id\endcsname{%
            0{\BKM@level}0%
          }%
          \BKM@swfalse
        \else
          \expandafter\expandafter\expandafter\BKM@getx
              \csname BKM@\BKM@prev\endcsname
          \ifnum\BKM@level>\BKM@x@level\relax
            \expandafter\xdef\csname BKM@\the\BKM@id\endcsname{%
              {\BKM@prev}{\BKM@level}0%
            }%
            \ifnum\BKM@prev>\z@
              \BKM@CalcExpr\BKM@CalcResult\BKM@x@childs+1%
              \expandafter\xdef\csname BKM@\BKM@prev\endcsname{%
                {\BKM@x@parent}{\BKM@x@level}{\BKM@CalcResult}%
              }%
            \fi
            \BKM@swfalse
          \else
            \let\BKM@prev\BKM@x@parent
          \fi
        \fi
      }%
      \pdfstringdef\BKM@title{\bookmark@text}%
      \edef\BKM@FLAGS{\BKM@PrintStyle}%
      \csname BKM@HypDestOptHook\endcsname
      \BKM@EscapeHex\BKM@dest
      \BKM@EscapeHex\BKM@uri
      \BKM@EscapeHex\BKM@gotor
      \BKM@EscapeHex\BKM@rawaction
      \BKM@EscapeHex\BKM@title
      \immediate\write\@mainaux{%
        \string\BKM@entry{%
          id=\number\BKM@id
          \ifBKM@open
            \ifnum\BKM@level<\BKM@openlevel
              ,open%
            \fi
          \fi
          \BKM@auxentry{dest}%
          \BKM@auxentry{named}%
          \BKM@auxentry{uri}%
          \BKM@auxentry{gotor}%
          \BKM@auxentry{page}%
          \BKM@auxentry{view}%
          \BKM@auxentry{rawaction}%
          \BKM@auxentry{color}%
          \ifnum\BKM@FLAGS>\z@
            ,flags=\BKM@FLAGS
          \fi
          \BKM@auxentry{srcline}%
          \BKM@auxentry{srcfile}%
        }{\BKM@title}%
      }%
    \endgroup
  \fi
}
%    \end{macrocode}
%    \end{macro}
%    \begin{macro}{\BKM@getx}
%    \begin{macrocode}
\def\BKM@getx#1#2#3{%
  \def\BKM@x@parent{#1}%
  \def\BKM@x@level{#2}%
  \def\BKM@x@childs{#3}%
}
%    \end{macrocode}
%    \end{macro}
%    \begin{macro}{\BKM@auxentry}
%    \begin{macrocode}
\def\BKM@auxentry#1{%
  \expandafter\ifx\csname BKM@#1\endcsname\@empty
  \else
    ,#1={\csname BKM@#1\endcsname}%
  \fi
}
%    \end{macrocode}
%    \end{macro}
%
%    \begin{macro}{\BKM@InitSourceLocation}
%    \begin{macrocode}
\def\BKM@InitSourceLocation{%
  \edef\BKM@srcline{\the\inputlineno}%
  \BKM@LuaTeX@InitFile
  \ifx\BKM@srcfile\@empty
    \ltx@IfUndefined{currfilepath}{}{%
      \edef\BKM@srcfile{\currfilepath}%
    }%
  \fi
}
%    \end{macrocode}
%    \end{macro}
%    \begin{macro}{\BKM@LuaTeX@InitFile}
%    \begin{macrocode}
\ifluatex
  \ifnum\luatexversion>36 %
    \def\BKM@LuaTeX@InitFile{%
      \begingroup
        \ltx@LocToksA={}%
      \edef\x{\endgroup
        \def\noexpand\BKM@srcfile{%
          \the\expandafter\ltx@LocToksA
          \directlua{%
             if status and status.filename then %
               tex.settoks('ltx@LocToksA', status.filename)%
             end%
          }%
        }%
      }\x
    }%
  \else
    \let\BKM@LuaTeX@InitFile\relax
  \fi
\else
  \let\BKM@LuaTeX@InitFile\relax
\fi
%    \end{macrocode}
%    \end{macro}
%
% \subsubsection{读取辅助数据(auxiliary data)}
%
%    \begin{macrocode}
\SetupKeyvalOptions{family=BKM@DO,prefix=BKM@DO@}
\DeclareStringOption[0]{id}
\DeclareBoolOption{open}
\DeclareStringOption{flags}
\DeclareStringOption{color}
\DeclareStringOption{dest}
\DeclareStringOption{named}
\DeclareStringOption{uri}
\DeclareStringOption{gotor}
\DeclareStringOption{page}
\DeclareStringOption{view}
\DeclareStringOption{rawaction}
\DeclareStringOption{srcline}
\DeclareStringOption{srcfile}
%    \end{macrocode}
%
%    \begin{macrocode}
\AtBeginDocument{%
  \let\BKM@entry\BKM@DO@entry
}
%    \end{macrocode}
%
%    \begin{macrocode}
%</pdftex|pdfmark>
%    \end{macrocode}
%
% \subsection{\xoption{atend}\ 选项}
%
% \subsubsection{钩子(Hook)}
%
%    \begin{macrocode}
%<*package>
%    \end{macrocode}
%    \begin{macrocode}
\ifBKM@atend
\else
%    \end{macrocode}
%    \begin{macro}{\BookmarkAtEnd}
%    这是一个虚拟定义(dummy definition),如果没有给出 \xoption{atend}\ 选项,它将生成一个警告。
%    \begin{macrocode}
  \newcommand{\BookmarkAtEnd}[1]{%
    \PackageWarning{bookmark}{%
      Ignored, because option `atend' is missing%
    }%
  }%
%    \end{macrocode}
%    \end{macro}
%    \begin{macrocode}
  \expandafter\endinput
\fi
%    \end{macrocode}
%    \begin{macro}{\BookmarkAtEnd}
%    \begin{macrocode}
\newcommand*{\BookmarkAtEnd}{%
  \g@addto@macro\BKM@EndHook
}
%    \end{macrocode}
%    \end{macro}
%    \begin{macrocode}
\let\BKM@EndHook\@empty
%    \end{macrocode}
%    \begin{macrocode}
%</package>
%    \end{macrocode}
%
% \subsubsection{在文档末尾使用钩子的驱动程序}
%
%    驱动程序 \xoption{pdftex}\ 使用 LaTeX 钩子 \xoption{enddocument/afterlastpage}
%    (相当于以前使用的 \xpackage{atveryend}\ 的 \cs{AfterLastShipout}),因为它仍然需要 \xext{aux}\ 文件。
%    它使用 \cs{pdfoutline}\ 作为最后一页之后可以使用的书签(bookmakrs)。
%    \begin{itemize}
%    \item
%      驱动程序 \xoption{pdftex}\ 使用 \cs{pdfoutline}, \cs{pdfoutline}\ 可以在最后一页之后使用。
%    \end{itemize}
%    \begin{macrocode}
%<*pdftex>
\ifBKM@atend
  \AddToHook{enddocument/afterlastpage}{%
    \BKM@EndHook
  }%
\fi
%</pdftex>
%    \end{macrocode}
%
% \subsubsection{使用 \xoption{shipout/lastpage}\ 的驱动程序}
%
%    其他驱动程序使用 \cs{special}\ 命令实现 \cs{bookmark}。因此,最后的书签(last bookmarks)
%    必须放在最后一页(last page),而不是之后。不能使用 \cs{AtEndDocument},因为为时已晚,
%    最后一页已经输出了。因此,我们使用 LaTeX 钩子 \xoption{shipout/lastpage}。至少需要运行
%    两次 \hologo{LaTeX}。PostScript 驱动程序 \xoption{dvips}\ 使用外部 PostScript 文件作为书签。
%    为了避免与 pgf 发生冲突,文件写入(file writing)也被移到了最后一个输出页面(shipout page)。
%    \begin{macrocode}
%<*dvipdfm|vtex|pdfmark>
\ifBKM@atend
  \AddToHook{shipout/lastpage}{\BKM@EndHook}%
\fi
%</dvipdfm|vtex|pdfmark>
%    \end{macrocode}
%
% \section{安装(Installation)}
%
% \subsection{下载(Download)}
%
% \paragraph{宏包(Package)。} 在 CTAN\footnote{\CTANpkg{bookmark}}上提供此宏包:
% \begin{description}
% \item[\CTAN{macros/latex/contrib/bookmark/bookmark.dtx}] 源文件(source file)。
% \item[\CTAN{macros/latex/contrib/bookmark/bookmark.pdf}] 文档(documentation)。
% \end{description}
%
%
% \paragraph{捆绑包(Bundle)。} “bookmark”捆绑包(bundle)的所有宏包(packages)都可以在兼
% 容 TDS 的 ZIP 归档文件中找到。在那里,宏包已经被解包,文档文件(documentation files)已经生成。
% 文件(files)和目录(directories)遵循 TDS 标准。
% \begin{description}
% \item[\CTANinstall{install/macros/latex/contrib/bookmark.tds.zip}]
% \end{description}
% \emph{TDS}\ 是指标准的“用于 \TeX\ 文件的目录结构(Directory Structure)”(\CTANpkg{tds})。
% 名称中带有 \xfile{texmf}\ 的目录(directories)通常以这种方式组织。
%
% \subsection{捆绑包(Bundle)的安装}
%
% \paragraph{解压(Unpacking)。} 在您选择的 TDS 树(也称为 \xfile{texmf}\ 树)中解
% 压 \xfile{bookmark.tds.zip},例如(在 linux 中):
% \begin{quote}
%   |unzip bookmark.tds.zip -d ~/texmf|
% \end{quote}
%
% \subsection{宏包(Package)的安装}
%
% \paragraph{解压(Unpacking)。} \xfile{.dtx}\ 文件是一个自解压 \docstrip\ 归档文件(archive)。
% 这些文件是通过 \plainTeX\ 运行 \xfile{.dtx}\ 来提取的:
% \begin{quote}
%   \verb|tex bookmark.dtx|
% \end{quote}
%
% \paragraph{TDS.} 现在,不同的文件必须移动到安装 TDS 树(installation TDS tree)
% (也称为 \xfile{texmf}\ 树)中的不同目录中:
% \begin{quote}
% \def\t{^^A
% \begin{tabular}{@{}>{\ttfamily}l@{ $\rightarrow$ }>{\ttfamily}l@{}}
%   bookmark.sty & tex/latex/bookmark/bookmark.sty\\
%   bkm-dvipdfm.def & tex/latex/bookmark/bkm-dvipdfm.def\\
%   bkm-dvips.def & tex/latex/bookmark/bkm-dvips.def\\
%   bkm-pdftex.def & tex/latex/bookmark/bkm-pdftex.def\\
%   bkm-vtex.def & tex/latex/bookmark/bkm-vtex.def\\
%   bookmark.pdf & doc/latex/bookmark/bookmark.pdf\\
%   bookmark-example.tex & doc/latex/bookmark/bookmark-example.tex\\
%   bookmark.dtx & source/latex/bookmark/bookmark.dtx\\
% \end{tabular}^^A
% }^^A
% \sbox0{\t}^^A
% \ifdim\wd0>\linewidth
%   \begingroup
%     \advance\linewidth by\leftmargin
%     \advance\linewidth by\rightmargin
%   \edef\x{\endgroup
%     \def\noexpand\lw{\the\linewidth}^^A
%   }\x
%   \def\lwbox{^^A
%     \leavevmode
%     \hbox to \linewidth{^^A
%       \kern-\leftmargin\relax
%       \hss
%       \usebox0
%       \hss
%       \kern-\rightmargin\relax
%     }^^A
%   }^^A
%   \ifdim\wd0>\lw
%     \sbox0{\small\t}^^A
%     \ifdim\wd0>\linewidth
%       \ifdim\wd0>\lw
%         \sbox0{\footnotesize\t}^^A
%         \ifdim\wd0>\linewidth
%           \ifdim\wd0>\lw
%             \sbox0{\scriptsize\t}^^A
%             \ifdim\wd0>\linewidth
%               \ifdim\wd0>\lw
%                 \sbox0{\tiny\t}^^A
%                 \ifdim\wd0>\linewidth
%                   \lwbox
%                 \else
%                   \usebox0
%                 \fi
%               \else
%                 \lwbox
%               \fi
%             \else
%               \usebox0
%             \fi
%           \else
%             \lwbox
%           \fi
%         \else
%           \usebox0
%         \fi
%       \else
%         \lwbox
%       \fi
%     \else
%       \usebox0
%     \fi
%   \else
%     \lwbox
%   \fi
% \else
%   \usebox0
% \fi
% \end{quote}
% 如果你有一个 \xfile{docstrip.cfg}\ 文件,该文件能配置并启用 \docstrip\ 的 TDS 安装功能,
% 则一些文件可能已经在正确的位置了,请参阅 \docstrip\ 的文档(documentation)。
%
% \subsection{刷新文件名数据库}
%
% 如果您的 \TeX~发行版(\TeX\,Live、\mikTeX、\dots)依赖于文件名数据库(file name databases),
% 则必须刷新这些文件名数据库。例如,\TeX\,Live\ 用户运行 \verb|texhash| 或 \verb|mktexlsr|。
%
% \subsection{一些感兴趣的细节}
%
% \paragraph{用 \LaTeX\ 解压。}
% \xfile{.dtx}\ 根据格式(format)选择其操作(action):
% \begin{description}
% \item[\plainTeX:] 运行 \docstrip\ 并解压文件。
% \item[\LaTeX:] 生成文档。
% \end{description}
% 如果您坚持通过 \LaTeX\ 使用\docstrip (实际上 \docstrip\ 并不需要 \LaTeX),那么请您的意图告知自动检测程序:
% \begin{quote}
%   \verb|latex \let\install=y\input{bookmark.dtx}|
% \end{quote}
% 不要忘记根据 shell 的要求引用这个参数(argument)。
%
% \paragraph{知生成文档。}
% 您可以同时使用 \xfile{.dtx}\ 或 \xfile{.drv}\ 来生成文档。可以通过配置文件 \xfile{ltxdoc.cfg}\ 配置该进程。
% 例如,如果您希望 A4 作为纸张格式,请将下面这行写入此文件中:
% \begin{quote}
%   \verb|\PassOptionsToClass{a4paper}{article}|
% \end{quote}
% 下面是一个如何使用 pdf\LaTeX\ 生成文档的示例:
% \begin{quote}
%\begin{verbatim}
%pdflatex bookmark.dtx
%makeindex -s gind.ist bookmark.idx
%pdflatex bookmark.dtx
%makeindex -s gind.ist bookmark.idx
%pdflatex bookmark.dtx
%\end{verbatim}
% \end{quote}
%
% \begin{thebibliography}{9}
%
% \bibitem{hyperref}
%   Sebastian Rahtz, Heiko Oberdiek:
%   \textit{The \xpackage{hyperref} package};
%   2011/04/17 v6.82g;
%   \CTANpkg{hyperref}
%
% \bibitem{currfile}
%   Martin Scharrer:
%   \textit{The \xpackage{currfile} package};
%   2011/01/09 v0.4.
%   \CTANpkg{currfile}
%
% \end{thebibliography}
%
% \begin{History}
%   \begin{Version}{2007/02/19 v0.1}
%   \item
%     First experimental version.
%   \end{Version}
%   \begin{Version}{2007/02/20 v0.2}
%   \item
%     Option \xoption{startatroot} added.
%   \item
%     Dummies for \cs{pdf(un)escape...} commands added to get
%     the package basically work for non-\hologo{pdfTeX} users.
%   \end{Version}
%   \begin{Version}{2007/02/21 v0.3}
%   \item
%     Dependency from \hologo{pdfTeX} 1.30 removed by using package
%     \xpackage{pdfescape}.
%   \end{Version}
%   \begin{Version}{2007/02/22 v0.4}
%   \item
%     \xpackage{hyperref}'s \xoption{bookmarkstype} respected.
%   \end{Version}
%   \begin{Version}{2007/03/02 v0.5}
%   \item
%     Driver options \xoption{vtex} (PDF mode), \xoption{dvipsone},
%     and \xoption{textures} added.
%   \item
%     Implementation of option \xoption{depth} completed. Division names
%     are supported, see \xpackage{hyperref}'s
%     option \xoption{bookmarksdepth}.
%   \item
%     \xpackage{hyperref}'s options \xoption{bookmarksopen},
%     \xoption{bookmarksopenlevel}, and \xoption{bookmarksdepth} respected.
%   \end{Version}
%   \begin{Version}{2007/03/03 v0.6}
%   \item
%     Option \xoption{numbered} as alias for \xpackage{hyperref}'s
%     \xoption{bookmarksnumbered}.
%   \end{Version}
%   \begin{Version}{2007/03/07 v0.7}
%   \item
%     Dependency from \hologo{eTeX} removed.
%   \end{Version}
%   \begin{Version}{2007/04/09 v0.8}
%   \item
%     Option \xoption{atend} added.
%   \item
%     Option \xoption{rgbcolor} removed.
%     \verb|rgbcolor=<r> <g> <b>| can be replaced by
%     \verb|color=[rgb]{<r>,<g>,<b>}|.
%   \item
%     Support of recent cvs version (2007-03-29) of dvipdfmx
%     that extends the \cs{special} for bookmarks to specify
%     open outline entries. Option \xoption{dvipdfmx-outline-open}
%     or \cs{SpecialDvipdfmxOutlineOpen} notify the package.
%   \end{Version}
%   \begin{Version}{2007/04/25 v0.9}
%   \item
%     The syntax of \cs{special} of dvipdfmx, if feature
%     \xoption{dvipdfmx-outline-open} is enabled, has changed.
%     Now cvs version 2007-04-25 is needed.
%   \end{Version}
%   \begin{Version}{2007/05/29 v1.0}
%   \item
%     Bug fix in code for second parameter of XYZ.
%   \end{Version}
%   \begin{Version}{2007/07/13 v1.1}
%   \item
%     Fix for pdfmark with GoToR action.
%   \end{Version}
%   \begin{Version}{2007/09/25 v1.2}
%   \item
%     pdfmark driver respects \cs{nofiles}.
%   \end{Version}
%   \begin{Version}{2008/08/08 v1.3}
%   \item
%     Package \xpackage{flags} replaced by package \xpackage{bitset}.
%     Now flags are also supported without \hologo{eTeX}.
%   \item
%     Hook for package \xpackage{hypdestopt} added.
%   \end{Version}
%   \begin{Version}{2008/09/13 v1.4}
%   \item
%     Fix for bug introduced in v1.3, package \xpackage{flags} is one-based,
%     but package \xpackage{bitset} is zero-based. Thus options \xoption{bold}
%     and \xoption{italic} are wrong in v1.3. (Daniel M\"ullner)
%   \end{Version}
%   \begin{Version}{2009/08/13 v1.5}
%   \item
%     Except for driver options the other options are now local options.
%     This resolves a problem with KOMA-Script v3.00 and its option \xoption{open}.
%   \end{Version}
%   \begin{Version}{2009/12/06 v1.6}
%   \item
%     Use of package \xpackage{atveryend} for drivers \xoption{pdftex}
%     and \xoption{pdfmark}.
%   \end{Version}
%   \begin{Version}{2009/12/07 v1.7}
%   \item
%     Use of package \xpackage{atveryend} fixed.
%   \end{Version}
%   \begin{Version}{2009/12/17 v1.8}
%   \item
%     Support of \xpackage{hyperref} 2009/12/17 v6.79v for \hologo{XeTeX}.
%   \end{Version}
%   \begin{Version}{2010/03/30 v1.9}
%   \item
%     Package name in an error message fixed.
%   \end{Version}
%   \begin{Version}{2010/04/03 v1.10}
%   \item
%     Option \xoption{style} and macro \cs{bookmarkdefinestyle} added.
%   \item
%     Hook support with option \xoption{addtohook} added.
%   \item
%     \cs{bookmarkget} added.
%   \end{Version}
%   \begin{Version}{2010/04/04 v1.11}
%   \item
%     Bug fix (introduced in v1.10).
%   \end{Version}
%   \begin{Version}{2010/04/08 v1.12}
%   \item
%     Requires \xpackage{ltxcmds} 2010/04/08.
%   \end{Version}
%   \begin{Version}{2010/07/23 v1.13}
%   \item
%     Support for \xclass{memoir}'s \cs{booknumberline} added.
%   \end{Version}
%   \begin{Version}{2010/09/02 v1.14}
%   \item
%     (Local) options \xoption{draft} and \xoption{final} added.
%   \end{Version}
%   \begin{Version}{2010/09/25 v1.15}
%   \item
%     Fix for option \xoption{dvipdfmx-outline-open}.
%   \item
%     Option \xoption{dvipdfmx-outline-open} is set automatically,
%     if XeTeX $\geq$ 0.9995 is detected.
%   \end{Version}
%   \begin{Version}{2010/10/19 v1.16}
%   \item
%     Option `startatroot' now acts globally.
%   \item
%     Option `level' also accepts names the same way as option `depth'.
%   \end{Version}
%   \begin{Version}{2010/10/25 v1.17}
%   \item
%     \cs{bookmarksetupnext} added.
%   \item
%     Using \cs{kvsetkeys} of package \xpackage{kvsetkeys}, because
%     \cs{setkeys} of package \xpackage{keyval} is not reentrant.
%     This can cause problems (unknown keys) with older versions of
%     hyperref that also uses \cs{setkeys} (found by GL).
%   \end{Version}
%   \begin{Version}{2010/11/05 v1.18}
%   \item
%     Use of \cs{pdf@ifdraftmode} of package \xpackage{pdftexcmds} for
%     the default of option \xoption{draft}.
%   \end{Version}
%   \begin{Version}{2011/03/20 v1.19}
%   \item
%     Use of \cs{dimexpr} fixed, if \hologo{eTeX} is not used.
%     (Bug found by Martin M\"unch.)
%   \item
%     Fix in documentation. Also layout options work without \hologo{eTeX}.
%   \end{Version}
%   \begin{Version}{2011/04/13 v1.20}
%   \item
%     Bug fix: \cs{BKM@SetDepth} renamed to \cs{BKM@SetDepthOrLevel}.
%   \end{Version}
%   \begin{Version}{2011/04/21 v1.21}
%   \item
%     Some support for file name and line number in error messages
%     at end of document (pdfTeX and pdfmark based drivers).
%   \end{Version}
%   \begin{Version}{2011/05/13 v1.22}
%   \item
%     Change of version 2010/11/05 v1.18 reverted, because otherwise
%     draftmode disables some \xext{aux} file entries.
%   \end{Version}
%   \begin{Version}{2011/09/19 v1.23}
%   \item
%     Some \cs{renewcommand}s changed to \cs{def} to avoid trouble
%     if the commands are not defined, because hyperref stopped early.
%   \end{Version}
%   \begin{Version}{2011/12/02 v1.24}
%   \item
%     Small optimization in \cs{BKM@toHexDigit}.
%   \end{Version}
%   \begin{Version}{2016/05/16 v1.25}
%   \item
%     Documentation updates.
%   \end{Version}
%   \begin{Version}{2016/05/17 v1.26}
%   \item
%     define \cs{pdfoutline} to allow pdftex driver to be used with Lua\TeX.
%   \end{Version}
%   \begin{Version}{2019/06/04 v1.27}
%   \item
%     unknown style options are ignored (issue 67)
%   \end{Version}

%   \begin{Version}{2019/12/03 v1.28}
%   \item
%     Documentation updates.
%   \item adjust package loading (all required packages already loaded
%     by \xpackage{hyperref}).
%   \end{Version}
%   \begin{Version}{2020-11-06 v1.29}
%   \item Adapted the dvips to avoid a clash with pgf.
%         https://github.com/pgf-tikz/pgf/issues/944
%   \item All drivers now use the new LaTeX hooks
%         and so require a format 2020-10-01 or newer. The older
%         drivers are provided as frozen versions and are used if an older
%         format is detected.
%   \item Added support for destlabel option of hyperref, https://github.com/ho-tex/bookmark/issues/1
%   \item Removed the \xoption{dvipsone} and \xoption{textures} driver.
%   \item Removed the code for option \xoption{dvipdfmx-outline-open}
%     and \cs{SpecialDvipdfmxOutlineOpen}. All dvipdfmx version should now support
%     this out-of-the-box.
%   \end{Version}
% \end{History}
%
% \PrintIndex
%
% \Finale
\endinput
|
% \end{quote}
% 不要忘记根据 shell 的要求引用这个参数(argument)。
%
% \paragraph{知生成文档。}
% 您可以同时使用 \xfile{.dtx}\ 或 \xfile{.drv}\ 来生成文档。可以通过配置文件 \xfile{ltxdoc.cfg}\ 配置该进程。
% 例如,如果您希望 A4 作为纸张格式,请将下面这行写入此文件中:
% \begin{quote}
%   \verb|\PassOptionsToClass{a4paper}{article}|
% \end{quote}
% 下面是一个如何使用 pdf\LaTeX\ 生成文档的示例:
% \begin{quote}
%\begin{verbatim}
%pdflatex bookmark.dtx
%makeindex -s gind.ist bookmark.idx
%pdflatex bookmark.dtx
%makeindex -s gind.ist bookmark.idx
%pdflatex bookmark.dtx
%\end{verbatim}
% \end{quote}
%
% \begin{thebibliography}{9}
%
% \bibitem{hyperref}
%   Sebastian Rahtz, Heiko Oberdiek:
%   \textit{The \xpackage{hyperref} package};
%   2011/04/17 v6.82g;
%   \CTANpkg{hyperref}
%
% \bibitem{currfile}
%   Martin Scharrer:
%   \textit{The \xpackage{currfile} package};
%   2011/01/09 v0.4.
%   \CTANpkg{currfile}
%
% \end{thebibliography}
%
% \begin{History}
%   \begin{Version}{2007/02/19 v0.1}
%   \item
%     First experimental version.
%   \end{Version}
%   \begin{Version}{2007/02/20 v0.2}
%   \item
%     Option \xoption{startatroot} added.
%   \item
%     Dummies for \cs{pdf(un)escape...} commands added to get
%     the package basically work for non-\hologo{pdfTeX} users.
%   \end{Version}
%   \begin{Version}{2007/02/21 v0.3}
%   \item
%     Dependency from \hologo{pdfTeX} 1.30 removed by using package
%     \xpackage{pdfescape}.
%   \end{Version}
%   \begin{Version}{2007/02/22 v0.4}
%   \item
%     \xpackage{hyperref}'s \xoption{bookmarkstype} respected.
%   \end{Version}
%   \begin{Version}{2007/03/02 v0.5}
%   \item
%     Driver options \xoption{vtex} (PDF mode), \xoption{dvipsone},
%     and \xoption{textures} added.
%   \item
%     Implementation of option \xoption{depth} completed. Division names
%     are supported, see \xpackage{hyperref}'s
%     option \xoption{bookmarksdepth}.
%   \item
%     \xpackage{hyperref}'s options \xoption{bookmarksopen},
%     \xoption{bookmarksopenlevel}, and \xoption{bookmarksdepth} respected.
%   \end{Version}
%   \begin{Version}{2007/03/03 v0.6}
%   \item
%     Option \xoption{numbered} as alias for \xpackage{hyperref}'s
%     \xoption{bookmarksnumbered}.
%   \end{Version}
%   \begin{Version}{2007/03/07 v0.7}
%   \item
%     Dependency from \hologo{eTeX} removed.
%   \end{Version}
%   \begin{Version}{2007/04/09 v0.8}
%   \item
%     Option \xoption{atend} added.
%   \item
%     Option \xoption{rgbcolor} removed.
%     \verb|rgbcolor=<r> <g> <b>| can be replaced by
%     \verb|color=[rgb]{<r>,<g>,<b>}|.
%   \item
%     Support of recent cvs version (2007-03-29) of dvipdfmx
%     that extends the \cs{special} for bookmarks to specify
%     open outline entries. Option \xoption{dvipdfmx-outline-open}
%     or \cs{SpecialDvipdfmxOutlineOpen} notify the package.
%   \end{Version}
%   \begin{Version}{2007/04/25 v0.9}
%   \item
%     The syntax of \cs{special} of dvipdfmx, if feature
%     \xoption{dvipdfmx-outline-open} is enabled, has changed.
%     Now cvs version 2007-04-25 is needed.
%   \end{Version}
%   \begin{Version}{2007/05/29 v1.0}
%   \item
%     Bug fix in code for second parameter of XYZ.
%   \end{Version}
%   \begin{Version}{2007/07/13 v1.1}
%   \item
%     Fix for pdfmark with GoToR action.
%   \end{Version}
%   \begin{Version}{2007/09/25 v1.2}
%   \item
%     pdfmark driver respects \cs{nofiles}.
%   \end{Version}
%   \begin{Version}{2008/08/08 v1.3}
%   \item
%     Package \xpackage{flags} replaced by package \xpackage{bitset}.
%     Now flags are also supported without \hologo{eTeX}.
%   \item
%     Hook for package \xpackage{hypdestopt} added.
%   \end{Version}
%   \begin{Version}{2008/09/13 v1.4}
%   \item
%     Fix for bug introduced in v1.3, package \xpackage{flags} is one-based,
%     but package \xpackage{bitset} is zero-based. Thus options \xoption{bold}
%     and \xoption{italic} are wrong in v1.3. (Daniel M\"ullner)
%   \end{Version}
%   \begin{Version}{2009/08/13 v1.5}
%   \item
%     Except for driver options the other options are now local options.
%     This resolves a problem with KOMA-Script v3.00 and its option \xoption{open}.
%   \end{Version}
%   \begin{Version}{2009/12/06 v1.6}
%   \item
%     Use of package \xpackage{atveryend} for drivers \xoption{pdftex}
%     and \xoption{pdfmark}.
%   \end{Version}
%   \begin{Version}{2009/12/07 v1.7}
%   \item
%     Use of package \xpackage{atveryend} fixed.
%   \end{Version}
%   \begin{Version}{2009/12/17 v1.8}
%   \item
%     Support of \xpackage{hyperref} 2009/12/17 v6.79v for \hologo{XeTeX}.
%   \end{Version}
%   \begin{Version}{2010/03/30 v1.9}
%   \item
%     Package name in an error message fixed.
%   \end{Version}
%   \begin{Version}{2010/04/03 v1.10}
%   \item
%     Option \xoption{style} and macro \cs{bookmarkdefinestyle} added.
%   \item
%     Hook support with option \xoption{addtohook} added.
%   \item
%     \cs{bookmarkget} added.
%   \end{Version}
%   \begin{Version}{2010/04/04 v1.11}
%   \item
%     Bug fix (introduced in v1.10).
%   \end{Version}
%   \begin{Version}{2010/04/08 v1.12}
%   \item
%     Requires \xpackage{ltxcmds} 2010/04/08.
%   \end{Version}
%   \begin{Version}{2010/07/23 v1.13}
%   \item
%     Support for \xclass{memoir}'s \cs{booknumberline} added.
%   \end{Version}
%   \begin{Version}{2010/09/02 v1.14}
%   \item
%     (Local) options \xoption{draft} and \xoption{final} added.
%   \end{Version}
%   \begin{Version}{2010/09/25 v1.15}
%   \item
%     Fix for option \xoption{dvipdfmx-outline-open}.
%   \item
%     Option \xoption{dvipdfmx-outline-open} is set automatically,
%     if XeTeX $\geq$ 0.9995 is detected.
%   \end{Version}
%   \begin{Version}{2010/10/19 v1.16}
%   \item
%     Option `startatroot' now acts globally.
%   \item
%     Option `level' also accepts names the same way as option `depth'.
%   \end{Version}
%   \begin{Version}{2010/10/25 v1.17}
%   \item
%     \cs{bookmarksetupnext} added.
%   \item
%     Using \cs{kvsetkeys} of package \xpackage{kvsetkeys}, because
%     \cs{setkeys} of package \xpackage{keyval} is not reentrant.
%     This can cause problems (unknown keys) with older versions of
%     hyperref that also uses \cs{setkeys} (found by GL).
%   \end{Version}
%   \begin{Version}{2010/11/05 v1.18}
%   \item
%     Use of \cs{pdf@ifdraftmode} of package \xpackage{pdftexcmds} for
%     the default of option \xoption{draft}.
%   \end{Version}
%   \begin{Version}{2011/03/20 v1.19}
%   \item
%     Use of \cs{dimexpr} fixed, if \hologo{eTeX} is not used.
%     (Bug found by Martin M\"unch.)
%   \item
%     Fix in documentation. Also layout options work without \hologo{eTeX}.
%   \end{Version}
%   \begin{Version}{2011/04/13 v1.20}
%   \item
%     Bug fix: \cs{BKM@SetDepth} renamed to \cs{BKM@SetDepthOrLevel}.
%   \end{Version}
%   \begin{Version}{2011/04/21 v1.21}
%   \item
%     Some support for file name and line number in error messages
%     at end of document (pdfTeX and pdfmark based drivers).
%   \end{Version}
%   \begin{Version}{2011/05/13 v1.22}
%   \item
%     Change of version 2010/11/05 v1.18 reverted, because otherwise
%     draftmode disables some \xext{aux} file entries.
%   \end{Version}
%   \begin{Version}{2011/09/19 v1.23}
%   \item
%     Some \cs{renewcommand}s changed to \cs{def} to avoid trouble
%     if the commands are not defined, because hyperref stopped early.
%   \end{Version}
%   \begin{Version}{2011/12/02 v1.24}
%   \item
%     Small optimization in \cs{BKM@toHexDigit}.
%   \end{Version}
%   \begin{Version}{2016/05/16 v1.25}
%   \item
%     Documentation updates.
%   \end{Version}
%   \begin{Version}{2016/05/17 v1.26}
%   \item
%     define \cs{pdfoutline} to allow pdftex driver to be used with Lua\TeX.
%   \end{Version}
%   \begin{Version}{2019/06/04 v1.27}
%   \item
%     unknown style options are ignored (issue 67)
%   \end{Version}

%   \begin{Version}{2019/12/03 v1.28}
%   \item
%     Documentation updates.
%   \item adjust package loading (all required packages already loaded
%     by \xpackage{hyperref}).
%   \end{Version}
%   \begin{Version}{2020-11-06 v1.29}
%   \item Adapted the dvips to avoid a clash with pgf.
%         https://github.com/pgf-tikz/pgf/issues/944
%   \item All drivers now use the new LaTeX hooks
%         and so require a format 2020-10-01 or newer. The older
%         drivers are provided as frozen versions and are used if an older
%         format is detected.
%   \item Added support for destlabel option of hyperref, https://github.com/ho-tex/bookmark/issues/1
%   \item Removed the \xoption{dvipsone} and \xoption{textures} driver.
%   \item Removed the code for option \xoption{dvipdfmx-outline-open}
%     and \cs{SpecialDvipdfmxOutlineOpen}. All dvipdfmx version should now support
%     this out-of-the-box.
%   \end{Version}
% \end{History}
%
% \PrintIndex
%
% \Finale
\endinput
|
% \end{quote}
% 不要忘记根据 shell 的要求引用这个参数(argument)。
%
% \paragraph{知生成文档。}
% 您可以同时使用 \xfile{.dtx}\ 或 \xfile{.drv}\ 来生成文档。可以通过配置文件 \xfile{ltxdoc.cfg}\ 配置该进程。
% 例如,如果您希望 A4 作为纸张格式,请将下面这行写入此文件中:
% \begin{quote}
%   \verb|\PassOptionsToClass{a4paper}{article}|
% \end{quote}
% 下面是一个如何使用 pdf\LaTeX\ 生成文档的示例:
% \begin{quote}
%\begin{verbatim}
%pdflatex bookmark.dtx
%makeindex -s gind.ist bookmark.idx
%pdflatex bookmark.dtx
%makeindex -s gind.ist bookmark.idx
%pdflatex bookmark.dtx
%\end{verbatim}
% \end{quote}
%
% \begin{thebibliography}{9}
%
% \bibitem{hyperref}
%   Sebastian Rahtz, Heiko Oberdiek:
%   \textit{The \xpackage{hyperref} package};
%   2011/04/17 v6.82g;
%   \CTANpkg{hyperref}
%
% \bibitem{currfile}
%   Martin Scharrer:
%   \textit{The \xpackage{currfile} package};
%   2011/01/09 v0.4.
%   \CTANpkg{currfile}
%
% \end{thebibliography}
%
% \begin{History}
%   \begin{Version}{2007/02/19 v0.1}
%   \item
%     First experimental version.
%   \end{Version}
%   \begin{Version}{2007/02/20 v0.2}
%   \item
%     Option \xoption{startatroot} added.
%   \item
%     Dummies for \cs{pdf(un)escape...} commands added to get
%     the package basically work for non-\hologo{pdfTeX} users.
%   \end{Version}
%   \begin{Version}{2007/02/21 v0.3}
%   \item
%     Dependency from \hologo{pdfTeX} 1.30 removed by using package
%     \xpackage{pdfescape}.
%   \end{Version}
%   \begin{Version}{2007/02/22 v0.4}
%   \item
%     \xpackage{hyperref}'s \xoption{bookmarkstype} respected.
%   \end{Version}
%   \begin{Version}{2007/03/02 v0.5}
%   \item
%     Driver options \xoption{vtex} (PDF mode), \xoption{dvipsone},
%     and \xoption{textures} added.
%   \item
%     Implementation of option \xoption{depth} completed. Division names
%     are supported, see \xpackage{hyperref}'s
%     option \xoption{bookmarksdepth}.
%   \item
%     \xpackage{hyperref}'s options \xoption{bookmarksopen},
%     \xoption{bookmarksopenlevel}, and \xoption{bookmarksdepth} respected.
%   \end{Version}
%   \begin{Version}{2007/03/03 v0.6}
%   \item
%     Option \xoption{numbered} as alias for \xpackage{hyperref}'s
%     \xoption{bookmarksnumbered}.
%   \end{Version}
%   \begin{Version}{2007/03/07 v0.7}
%   \item
%     Dependency from \hologo{eTeX} removed.
%   \end{Version}
%   \begin{Version}{2007/04/09 v0.8}
%   \item
%     Option \xoption{atend} added.
%   \item
%     Option \xoption{rgbcolor} removed.
%     \verb|rgbcolor=<r> <g> <b>| can be replaced by
%     \verb|color=[rgb]{<r>,<g>,<b>}|.
%   \item
%     Support of recent cvs version (2007-03-29) of dvipdfmx
%     that extends the \cs{special} for bookmarks to specify
%     open outline entries. Option \xoption{dvipdfmx-outline-open}
%     or \cs{SpecialDvipdfmxOutlineOpen} notify the package.
%   \end{Version}
%   \begin{Version}{2007/04/25 v0.9}
%   \item
%     The syntax of \cs{special} of dvipdfmx, if feature
%     \xoption{dvipdfmx-outline-open} is enabled, has changed.
%     Now cvs version 2007-04-25 is needed.
%   \end{Version}
%   \begin{Version}{2007/05/29 v1.0}
%   \item
%     Bug fix in code for second parameter of XYZ.
%   \end{Version}
%   \begin{Version}{2007/07/13 v1.1}
%   \item
%     Fix for pdfmark with GoToR action.
%   \end{Version}
%   \begin{Version}{2007/09/25 v1.2}
%   \item
%     pdfmark driver respects \cs{nofiles}.
%   \end{Version}
%   \begin{Version}{2008/08/08 v1.3}
%   \item
%     Package \xpackage{flags} replaced by package \xpackage{bitset}.
%     Now flags are also supported without \hologo{eTeX}.
%   \item
%     Hook for package \xpackage{hypdestopt} added.
%   \end{Version}
%   \begin{Version}{2008/09/13 v1.4}
%   \item
%     Fix for bug introduced in v1.3, package \xpackage{flags} is one-based,
%     but package \xpackage{bitset} is zero-based. Thus options \xoption{bold}
%     and \xoption{italic} are wrong in v1.3. (Daniel M\"ullner)
%   \end{Version}
%   \begin{Version}{2009/08/13 v1.5}
%   \item
%     Except for driver options the other options are now local options.
%     This resolves a problem with KOMA-Script v3.00 and its option \xoption{open}.
%   \end{Version}
%   \begin{Version}{2009/12/06 v1.6}
%   \item
%     Use of package \xpackage{atveryend} for drivers \xoption{pdftex}
%     and \xoption{pdfmark}.
%   \end{Version}
%   \begin{Version}{2009/12/07 v1.7}
%   \item
%     Use of package \xpackage{atveryend} fixed.
%   \end{Version}
%   \begin{Version}{2009/12/17 v1.8}
%   \item
%     Support of \xpackage{hyperref} 2009/12/17 v6.79v for \hologo{XeTeX}.
%   \end{Version}
%   \begin{Version}{2010/03/30 v1.9}
%   \item
%     Package name in an error message fixed.
%   \end{Version}
%   \begin{Version}{2010/04/03 v1.10}
%   \item
%     Option \xoption{style} and macro \cs{bookmarkdefinestyle} added.
%   \item
%     Hook support with option \xoption{addtohook} added.
%   \item
%     \cs{bookmarkget} added.
%   \end{Version}
%   \begin{Version}{2010/04/04 v1.11}
%   \item
%     Bug fix (introduced in v1.10).
%   \end{Version}
%   \begin{Version}{2010/04/08 v1.12}
%   \item
%     Requires \xpackage{ltxcmds} 2010/04/08.
%   \end{Version}
%   \begin{Version}{2010/07/23 v1.13}
%   \item
%     Support for \xclass{memoir}'s \cs{booknumberline} added.
%   \end{Version}
%   \begin{Version}{2010/09/02 v1.14}
%   \item
%     (Local) options \xoption{draft} and \xoption{final} added.
%   \end{Version}
%   \begin{Version}{2010/09/25 v1.15}
%   \item
%     Fix for option \xoption{dvipdfmx-outline-open}.
%   \item
%     Option \xoption{dvipdfmx-outline-open} is set automatically,
%     if XeTeX $\geq$ 0.9995 is detected.
%   \end{Version}
%   \begin{Version}{2010/10/19 v1.16}
%   \item
%     Option `startatroot' now acts globally.
%   \item
%     Option `level' also accepts names the same way as option `depth'.
%   \end{Version}
%   \begin{Version}{2010/10/25 v1.17}
%   \item
%     \cs{bookmarksetupnext} added.
%   \item
%     Using \cs{kvsetkeys} of package \xpackage{kvsetkeys}, because
%     \cs{setkeys} of package \xpackage{keyval} is not reentrant.
%     This can cause problems (unknown keys) with older versions of
%     hyperref that also uses \cs{setkeys} (found by GL).
%   \end{Version}
%   \begin{Version}{2010/11/05 v1.18}
%   \item
%     Use of \cs{pdf@ifdraftmode} of package \xpackage{pdftexcmds} for
%     the default of option \xoption{draft}.
%   \end{Version}
%   \begin{Version}{2011/03/20 v1.19}
%   \item
%     Use of \cs{dimexpr} fixed, if \hologo{eTeX} is not used.
%     (Bug found by Martin M\"unch.)
%   \item
%     Fix in documentation. Also layout options work without \hologo{eTeX}.
%   \end{Version}
%   \begin{Version}{2011/04/13 v1.20}
%   \item
%     Bug fix: \cs{BKM@SetDepth} renamed to \cs{BKM@SetDepthOrLevel}.
%   \end{Version}
%   \begin{Version}{2011/04/21 v1.21}
%   \item
%     Some support for file name and line number in error messages
%     at end of document (pdfTeX and pdfmark based drivers).
%   \end{Version}
%   \begin{Version}{2011/05/13 v1.22}
%   \item
%     Change of version 2010/11/05 v1.18 reverted, because otherwise
%     draftmode disables some \xext{aux} file entries.
%   \end{Version}
%   \begin{Version}{2011/09/19 v1.23}
%   \item
%     Some \cs{renewcommand}s changed to \cs{def} to avoid trouble
%     if the commands are not defined, because hyperref stopped early.
%   \end{Version}
%   \begin{Version}{2011/12/02 v1.24}
%   \item
%     Small optimization in \cs{BKM@toHexDigit}.
%   \end{Version}
%   \begin{Version}{2016/05/16 v1.25}
%   \item
%     Documentation updates.
%   \end{Version}
%   \begin{Version}{2016/05/17 v1.26}
%   \item
%     define \cs{pdfoutline} to allow pdftex driver to be used with Lua\TeX.
%   \end{Version}
%   \begin{Version}{2019/06/04 v1.27}
%   \item
%     unknown style options are ignored (issue 67)
%   \end{Version}

%   \begin{Version}{2019/12/03 v1.28}
%   \item
%     Documentation updates.
%   \item adjust package loading (all required packages already loaded
%     by \xpackage{hyperref}).
%   \end{Version}
%   \begin{Version}{2020-11-06 v1.29}
%   \item Adapted the dvips to avoid a clash with pgf.
%         https://github.com/pgf-tikz/pgf/issues/944
%   \item All drivers now use the new LaTeX hooks
%         and so require a format 2020-10-01 or newer. The older
%         drivers are provided as frozen versions and are used if an older
%         format is detected.
%   \item Added support for destlabel option of hyperref, https://github.com/ho-tex/bookmark/issues/1
%   \item Removed the \xoption{dvipsone} and \xoption{textures} driver.
%   \item Removed the code for option \xoption{dvipdfmx-outline-open}
%     and \cs{SpecialDvipdfmxOutlineOpen}. All dvipdfmx version should now support
%     this out-of-the-box.
%   \end{Version}
% \end{History}
%
% \PrintIndex
%
% \Finale
\endinput

%        (quote the arguments according to the demands of your shell)
%
% Documentation:
%    (a) If bookmark.drv is present:
%           latex bookmark.drv
%    (b) Without bookmark.drv:
%           latex bookmark.dtx; ...
%    The class ltxdoc loads the configuration file ltxdoc.cfg
%    if available. Here you can specify further options, e.g.
%    use A4 as paper format:
%       \PassOptionsToClass{a4paper}{article}
%
%    Programm calls to get the documentation (example):
%       pdflatex bookmark.dtx
%       makeindex -s gind.ist bookmark.idx
%       pdflatex bookmark.dtx
%       makeindex -s gind.ist bookmark.idx
%       pdflatex bookmark.dtx
%
% Installation:
%    TDS:tex/latex/bookmark/bookmark.sty
%    TDS:tex/latex/bookmark/bkm-dvipdfm.def
%    TDS:tex/latex/bookmark/bkm-dvips.def
%    TDS:tex/latex/bookmark/bkm-pdftex.def
%    TDS:tex/latex/bookmark/bkm-vtex.def
%    TDS:tex/latex/bookmark/bkm-dvipdfm-2019-12-03.def
%    TDS:tex/latex/bookmark/bkm-dvips-2019-12-03.def
%    TDS:tex/latex/bookmark/bkm-pdftex-2019-12-03.def
%    TDS:tex/latex/bookmark/bkm-vtex-2019-12-03.def%
%    TDS:doc/latex/bookmark/bookmark.pdf
%    TDS:doc/latex/bookmark/bookmark-example.tex
%    TDS:source/latex/bookmark/bookmark.dtx
%    TDS:source/latex/bookmark/bookmark-frozen.dtx
%
%<*ignore>
\begingroup
  \catcode123=1 %
  \catcode125=2 %
  \def\x{LaTeX2e}%
\expandafter\endgroup
\ifcase 0\ifx\install y1\fi\expandafter
         \ifx\csname processbatchFile\endcsname\relax\else1\fi
         \ifx\fmtname\x\else 1\fi\relax
\else\csname fi\endcsname
%</ignore>
%<*install>
\input docstrip.tex
\Msg{************************************************************************}
\Msg{* Installation}
\Msg{* Package: bookmark 2020-11-06 v1.29 PDF bookmarks (HO)}
\Msg{************************************************************************}

\keepsilent
\askforoverwritefalse

\let\MetaPrefix\relax
\preamble

This is a generated file.

Project: bookmark
Version: 2020-11-06 v1.29

Copyright (C)
   2007-2011 Heiko Oberdiek
   2016-2020 Oberdiek Package Support Group

This work may be distributed and/or modified under the
conditions of the LaTeX Project Public License, either
version 1.3c of this license or (at your option) any later
version. This version of this license is in
   https://www.latex-project.org/lppl/lppl-1-3c.txt
and the latest version of this license is in
   https://www.latex-project.org/lppl.txt
and version 1.3 or later is part of all distributions of
LaTeX version 2005/12/01 or later.

This work has the LPPL maintenance status "maintained".

The Current Maintainers of this work are
Heiko Oberdiek and the Oberdiek Package Support Group
https://github.com/ho-tex/bookmark/issues


This work consists of the main source file bookmark.dtx and bookmark-frozen.dtx
and the derived files
   bookmark.sty, bookmark.pdf, bookmark.ins, bookmark.drv,
   bkm-dvipdfm.def, bkm-dvips.def, bkm-pdftex.def, bkm-vtex.def,
   bkm-dvipdfm-2019-12-03.def, bkm-dvips-2019-12-03.def,
   bkm-pdftex-2019-12-03.def, bkm-vtex-2019-12-03.def,
   bookmark-example.tex.

\endpreamble
\let\MetaPrefix\DoubleperCent

\generate{%
  \file{bookmark.ins}{\from{bookmark.dtx}{install}}%
  \file{bookmark.drv}{\from{bookmark.dtx}{driver}}%
  \usedir{tex/latex/bookmark}%
  \file{bookmark.sty}{\from{bookmark.dtx}{package}}%
  \file{bkm-dvipdfm.def}{\from{bookmark.dtx}{dvipdfm}}%
  \file{bkm-dvips.def}{\from{bookmark.dtx}{dvips,pdfmark}}%
  \file{bkm-pdftex.def}{\from{bookmark.dtx}{pdftex}}%
  \file{bkm-vtex.def}{\from{bookmark.dtx}{vtex}}%
  \usedir{doc/latex/bookmark}%
  \file{bookmark-example.tex}{\from{bookmark.dtx}{example}}%
  \file{bkm-pdftex-2019-12-03.def}{\from{bookmark-frozen.dtx}{pdftexfrozen}}%
  \file{bkm-dvips-2019-12-03.def}{\from{bookmark-frozen.dtx}{dvipsfrozen}}%
  \file{bkm-vtex-2019-12-03.def}{\from{bookmark-frozen.dtx}{vtexfrozen}}%
  \file{bkm-dvipdfm-2019-12-03.def}{\from{bookmark-frozen.dtx}{dvipdfmfrozen}}%
}

\catcode32=13\relax% active space
\let =\space%
\Msg{************************************************************************}
\Msg{*}
\Msg{* To finish the installation you have to move the following}
\Msg{* files into a directory searched by TeX:}
\Msg{*}
\Msg{*     bookmark.sty, bkm-dvipdfm.def, bkm-dvips.def,}
\Msg{*     bkm-pdftex.def, bkm-vtex.def, bkm-dvipdfm-2019-12-03.def,}
\Msg{*     bkm-dvips-2019-12-03.def, bkm-pdftex-2019-12-03.def,}
\Msg{*     and bkm-vtex-2019-12-03.def}
\Msg{*}
\Msg{* To produce the documentation run the file `bookmark.drv'}
\Msg{* through LaTeX.}
\Msg{*}
\Msg{* Happy TeXing!}
\Msg{*}
\Msg{************************************************************************}

\endbatchfile
%</install>
%<*ignore>
\fi
%</ignore>
%<*driver>
\NeedsTeXFormat{LaTeX2e}
\ProvidesFile{bookmark.drv}%
  [2020-11-06 v1.29 PDF bookmarks (HO)]%
\documentclass{ltxdoc}
\usepackage{ctex}
\usepackage{indentfirst}
\setlength{\parindent}{2em}
\usepackage{holtxdoc}[2011/11/22]
\usepackage{xcolor}
\usepackage{hyperref}
\usepackage[open,openlevel=3,atend]{bookmark}[2020/11/06] %%%打开书签,显示的深度为3级,即显示part、section、subsection。
\bookmarksetup{color=red}
\begin{document}

  \renewcommand{\contentsname}{目\quad 录}
  \renewcommand{\abstractname}{摘\quad 要}
  \renewcommand{\historyname}{历史}
  \DocInput{bookmark.dtx}%
\end{document}
%</driver>
% \fi
%
%
%
% \GetFileInfo{bookmark.drv}
%
%% \title{\xpackage{bookmark} 宏包}
% \title{\heiti {\Huge \textbf{\xpackage{bookmark}\ 宏包}}}
% \date{2020-11-06\ \ \ v1.29}
% \author{Heiko Oberdiek \thanks
% {如有问题请点击:\url{https://github.com/ho-tex/bookmark/issues}}\\[5pt]赣医一附院神经科\ \ 黄旭华\ \ \ \ 译}
%
% \maketitle
%
% \begin{abstract}
% 这个宏包为 \xpackage{hyperref}\ 宏包实现了一个新的书签(bookmark)(大纲[outline])组织。现在
% 可以设置样式(style)和颜色(color)等书签属性(bookmark properties)。其他动作类型(action types)可用
% (URI、GoToR、Named)。书签是在第一次编译运行(compile run)中生成的。\xpackage{hyperref}\
% 宏包必需运行两次。
% \end{abstract}
%
% \tableofcontents
%
% \section{文档(Documentation)}
%
% \subsection{介绍}
%
% 这个 \xpackage{bookmark}\ 宏包试图为书签(bookmarks)提供一个更现代的管理:
% \begin{itemize}
% \item 书签已经在第一次 \hologo{TeX}\ 编译运行(compile run)中生成。
% \item 可以更改书签的字体样式(font style)和颜色(color)。
% \item 可以执行比简单的 GoTo 操作(actions)更多的操作。
% \end{itemize}
%
% 与 \xpackage{hyperref} \cite{hyperref} 一样,书签(bookmarks)也是按照书签生成宏
% (bookmark generating macros)(\cs{bookmark})的顺序生成的。级别号(level number)用于
% 定义书签的树结构(tree structure)。限制没有那么严格:
% \begin{itemize}
% \item 级别值(level values)可以跳变(jump)和省略(omit)。\cs{subsubsection}\ 可以跟在
%       \cs{chapter}\ 之后。这种情况如在 \xpackage{hyperref}\ 中则产生错误,它将显示一个警告(warning)
%       并尝试修复此错误。
% \item 多个书签可能指向同一目标(destination)。在 \xpackage{hyperref}\ 中,这会完全弄乱
%       书签树(bookmark tree),因为算法假设(algorithm assumes)目标名称(destination names)
%       是键(keys)(唯一的)。
% \end{itemize}
%
% 注意,这个宏包是作为书签管理(bookmark management)的实验平台(experimentation platform)。
% 欢迎反馈。此外,在未来的版本中,接口(interfaces)也可能发生变化。
%
% \subsection{选项(Options)}
%
% 可在以下四个地方放置选项(options):
% \begin{enumerate}
% \item \cs{usepackage}|[|\meta{options}|]{bookmark}|\\
%       这是放置驱动程序选项(driver options)和 \xoption{atend}\ 选项的唯一位置。
% \item \cs{bookmarksetup}|{|\meta{options}|}|\\
%       此命令仅用于设置选项(setting options)。
% \item \cs{bookmarksetupnext}|{|\meta{options}|}|\\
%       这些选项在下一个 \cs{bookmark}\ 命令的选项之后存储(stored)和调用(called)。
% \item \cs{bookmark}|[|\meta{options}|]{|\meta{title}|}|\\
%       此命令设置书签。选项设置(option settings)仅限于此书签。
% \end{enumerate}
% 异常(Exception):加载该宏包后,无法更改驱动程序选项(Driver options)、\xoption{atend}\ 选项
% 、\xoption{draft}\slash\xoption{final}选项。
%
% \subsubsection{\xoption{draft} 和 \xoption{final}\ 选项}
%
% 如果一个\LaTeX\ 文件要被编译了多次,那么可以使用 \xoption{draft}\ 选项来禁用该宏包的书签内
% 容(bookmark stuff),这样可以节省一点时间。默认 \xoption{final}\ 选项。两个选项都是
% 布尔选项(boolean options),如果没有值,则使用值 |true|。|draft=true| 与 |final=false| 相同。
%
% 除了驱动程序选项(driver options)之外,\xpackage{bookmark}\ 宏包选项都是局部选项(local options)。
% \xoption{draft}\ 选项和 \xoption{final}\ 选项均属于文档类选项(class option)(译者注:文档类选项为全局选项),
% 因此,在 \xpackage{bookmark}\ 宏包中未能看到这两个选项。如果您想使用全局的(global) \xoption{draft}选项
% 来优化第一次 \LaTeX\ 运行(runs),可以在导言(preamble)中引入 \xpackage{ifdraft}\ 宏包并设置 \LaTeX\ 的
% \cs{PassOptionsToPackage},例如:
%\begin{quote}
%\begin{verbatim}
%\documentclass[draft]{article}
%\usepackage{ifdraft}
%\ifdraft{%
%   \PassOptionsToPackage{draft}{bookmark}%
%}{}
%\end{verbatim}
%\end{quote}
%
% \subsubsection{驱动程序选项(Driver options)}
%
% 支持的驱动程序( drivers)包括 \xoption{pdftex}、\xoption{dvips}、\xoption{dvipdfm} (\xoption{xetex})、
% \xoption{vtex}。\hologo{TeX}\ 引擎 \hologo{pdfTeX}、\hologo{XeTeX}、\hologo{VTeX}\ 能被自动检测到。
% 默认的 DVI 驱动程序是 \xoption{dvips}。这可以通过 \cs{BookmarkDriverDefault}\ 在配置
% 文件 \xfile{bookmark.cfg}\ 中进行更改,例如:
% \begin{quote}
% |\def\BookmarkDriverDefault{dvipdfm}|
% \end{quote}
% 当前版本的(current versions)驱动程序使用新的 \LaTeX\ 钩子(\LaTeX-hooks)。如果检测到比
% 2020-10-01 更旧的格式,则将以前驱动程序的冻结版本(frozen versions)作为备份(fallback)。
%
% \paragraph{用 dvipdfmx 打开书签(bookmarks)。}旧版本的宏包有一个 \xoption{dvipdfmx-outline-open}\ 选项
% 可以激活代码,而该代码可以指定一个大纲条目(outline entry)是否打开。该宏包现在假设所有使用的 dvipdfmx 版本都是
% 最新版本,足以理解该代码,因此始终激活该代码。选项本身将被忽略。
%
%
% \subsubsection{布局选项(Layout options)}
%
% \paragraph{字体(Font)选项:}
%
% \begin{description}
% \item[\xoption{bold}:] 如果受 PDF 浏览器(PDF viewer)支持,书签将以粗体字体(bold font)显示(自 PDF 1.4起)。
% \item[\xoption{italic}:] 使用斜体字体(italic font)(自 PDF 1.4起)。
% \end{description}
% \xoption{bold}(粗体) 和 \xoption{italic}(斜体)可以同时使用。而 |false| 值(value)禁用字体选项。
%
% \paragraph{颜色(Color)选项:}
%
% 彩色书签(Colored bookmarks)是 PDF 1.4 的一个特性(feature),并非所有的 PDF 浏览器(PDF viewers)都支持彩色书签。
% \begin{description}
% \item[\xoption{color}:] 这里 color(颜色)可以作为 \xpackage{color}\ 宏包或 \xpackage{xcolor}\ 宏包的
% 颜色规范(color specification)给出。空值(empty value)表示未设置颜色属性。如果未加载 \xpackage{xcolor}\ 宏包,
% 能识别的值(recognized values)只有:
%   \begin{itemize}
%   \item 空值(empty value)表示未设置颜色属性,\\
%         例如:|color={}|
%   \item 颜色模型(color model) rgb 的显式颜色规范(explicit color specification),\\
%         例如,红色(red):|color=[rgb]{1,0,0}|
%   \item 颜色模型(color model)灰(gray)的显式颜色规范(explicit color specification),\\
%         例如,深灰色(dark gray):|color=[gray]{0.25}|
%   \end{itemize}
%   请注意,如果加载了 \xpackage{color}\ 宏包,此限制(restriction)也适用。然而,如果加载了 \xpackage{xcolor}\ 宏包,
%   则可以使用所有颜色规范(color specifications)。
% \end{description}
%
% \subsubsection{动作选项(Action options)}
%
% \begin{description}
% \item[\xoption{dest}:] 目的地名称(destination name)。
% \item[\xoption{page}:] 页码(page number),第一页(first page)为 1。
% \item[\xoption{view}:] 浏览规范(view specification),示例如下:\\
%   |view={FitB}|, |view={FitH 842}|, |view={XYZ 0 100 null}|\ \  一些浏览规范参数(view specification parameters)
%   将数字(numbers)视为具有单位 bp 的参数。它们可以作为普通数字(plain numbers)或在 \cs{calc}\ 内部以
%   长度表达式(length expressions)给出。如果加载了 \xpackage{calc}\ 宏包,则支持该宏包的表达式(expressions)。否则,
%   使用 \hologo{eTeX}\ 的 \cs{dimexpr}。例如:\\
%   |view={FitH \calc{\paperheight-\topmargin-1in}}|\\
%   |view={XYZ 0 \calc{\paperheight} null}|\\
%   注意 \cs{calc}\ 不能用于 |XYZ| 的第三个参数,因为该参数是缩放值(zoom value),而不是长度(length)。

% \item[\xoption{named}:] 已命名的动作(Named action)的名称:\\
%   |FirstPage|(第一页),|LastPage|(最后一页),|NextPage|(下一页),|PrevPage|(前一页)
% \item[\xoption{gotor}:] 外部(external) PDF 文件的名称。
% \item[\xoption{uri}:] URI 规范(URI specification)。
% \item[\xoption{rawaction}:] 原始动作规范(raw action specification)。由于这些规范取决于驱动程序(driver),因此不应使用此选项。
% \end{description}
% 通过分析指定的选项来选择书签的适当动作。动作由不同的选项集(sets of options)区分:
% \begin{quote}
 \begin{tabular}{|@{}r|l@{}|}
%   \hline
%   \ \textbf{动作(Action)}\  & \ \textbf{选项(Options)}\ \\ \hline
%   \ \textsf{GoTo}\  &\  \xoption{dest}\ \\ \hline
%   \ \textsf{GoTo}\  & \ \xoption{page} + \xoption{view}\ \\ \hline
%   \ \textsf{GoToR}\  & \ \xoption{gotor} + \xoption{dest}\ \\ \hline
%   \ \textsf{GoToR}\  & \ \xoption{gotor} + \xoption{page} + \xoption{view}\ \ \ \\ \hline
%   \ \textsf{Named}\  &\  \xoption{named}\ \\ \hline
%   \ \textsf{URI}\  & \ \xoption{uri}\ \\ \hline
% \end{tabular}
% \end{quote}
%
% \paragraph{缺少动作(Missing actions)。}
% 如果动作缺少 \xpackage{bookmark}\ 宏包,则抛出错误消息(error message)。根据驱动程序(driver)
% (\xoption{pdftex}、\xoption{dvips}\ 和好友[friends]),宏包在文档末尾很晚才检测到它。
% 自 2011/04/21 v1.21 版本以后,该宏包尝试打印 \cs{bookmark}\ 的相应出现的行号(line number)和文件名(file name)。
% 然而,\hologo{TeX}\ 确实提供了行号,但不幸的是,文件名是一个秘密(secret)。但该宏包有如下获取文件名的方法:
% \begin{itemize}
% \item 如果 \hologo{LuaTeX} (独立于 DVI 或 PDF 模式)正在运行,则自动使用其 |status.filename|。
% \item 宏包的 \cs{currfile} \cite{currfile}\ 重新定义了 \hologo{LaTeX}\ 的内部结构,以跟踪文件名(file name)。
% 如果加载了该宏包,那么它的 \cs{currfilepath}\ 将被检测到并由 \xpackage{bookmark}\ 自动使用。
% \item 可以通过 \cs{bookmarksetup}\ 或 \cs{bookmark}\ 中的 \xoption{scrfile}\ 选项手动设置(set manually)文件名。
% 但是要小心,手动设置会禁用以前的文件名检测方法。错误的(wrong)或丢失的(missed)文件名设置(file name setting)可能会在错误消息中
% 为您提供错误的源位置(source location)。
% \end{itemize}
%
% \subsubsection{级别选项(Level options)}
%
% 书签条目(bookmark entries)的顺序由 \cs{bookmark}\ 命令的的出现顺序(appearance order)定义。
% 树结构(tree structure)由书签节点(bookmark nodes)的属性 \xoption{level}(级别)构建。
% \xoption{level}\ 的值是整数(integers)。如果书签条目级别的值高于前一个节点,则该条目将成为
% 前一个节点的子(child)节点。差值的绝对值并不重要。
%
% \xpackage{bookmark}\ 宏包能记住全局属性(global property)“current level(当前级别)”中上
% 一个书签条目(previous bookmark entry)的级别。
%
% 级别系统的(level system)行为(behaviour)可以通过以下选项进行配置:
% \begin{description}
% \item[\xoption{level}:]
%    设置级别(level),请参阅上面的说明。如果给出的选项 \xoption{level}\ 没有值,那么将恢复默
%    认行为,即将“当前级别(current level)”用作级别值(level value)。自 2010/10/19 v1.16 版本以来,
%    如果宏 \cs{toclevel@part}、\cs{toclevel@section}\ 被定义过(通过 \xpackage{hyperref}\ 宏包完成,
%    请参阅它的 \xoption{bookmarkdepth}\ 选项),则 \xpackage{bookmark}\ 宏包还支持 |part|、|section| 等名称。
%
% \item[\xoption{rellevel}:]
%    设置相对于前一级别的(previous level)级别。正值表示书签条目成为前一个书签条目的子条目。
% \item[\xoption{keeplevel}:]
%    使用由\xoption{level}\ 或 \xoption{rellevel}\ 设置的级别,但不要更改全局属性“current level(当前级别)”。
%    可以通过设置为 |false| 来禁用该选项。
% \item[\xoption{startatroot}:]
%    此时,书签树(bookmark tree)再次从顶层(top level)开始。下一个书签条目不会作为上一个条目的子条目进行排序。
%    示例场景:文档使用 part。但是,最后几章(last chapters)不应放在最后一部分(last part)下面:
%    \begin{quote}
%\begin{verbatim}
%\documentclass{book}
%[...]
%\begin{document}
%  \part{第一部分}
%    \chapter{第一部分的第1章}
%    [...]
%  \part{第二部分(Second part)}
%    \chapter{第二部分的第1章}
%    [...]
%  \bookmarksetup{startatroot}
%  \chapter{Index}% 不属于第二部分
%\end{document}
%\end{verbatim}
%    \end{quote}
% \end{description}
%
% \subsubsection{样式定义(Style definitions)}
%
% 样式(style)是一组选项设置(option settings)。它可以由宏 \cs{bookmarkdefinestyle}\ 定义,
% 并由它的 \xoption{style}\ 选项使用。
% \begin{declcs}{bookmarkdefinestyle} \M{name} \M{key value list}
% \end{declcs}
% 选项设置(option settings)的 \meta{key value list}(键值列表)被指定为样式名(style \meta{name})。
%
% \begin{description}
% \item[\xoption{style}:]
%   \xoption{style}\ 选项的值是以前定义的样式的名称(name)。现在执行其选项设置(option settings)。
%   选项可以包括 \xoption{style}\ 选项。通过递归调用相同样式的无限递归(endless recursion)被阻止并抛出一个错误。
% \end{description}
%
% \subsubsection{钩子支持(Hook support)}
%
% 处理宏\cs{bookmark}\ 的可选选项(optional options)后,就会调用钩子(hook)。
% \begin{description}
% \item[\xoption{addtohook}:]
%   代码(code)作为该选项的值添加到钩子中。
% \end{description}
%
% \begin{declcs}{bookmarkget} \M{option}
% \end{declcs}
% \cs{bookmarkget}\ 宏提取 \meta{option}\ 选项的最新选项设置(latest option setting)的值。
% 对于布尔选项(boolean option),如果启用布尔选项,则返回 1,否则结果为零。结果数字(resulting numbers)
% 可以直接用于 \cs{ifnum}\ 或 \cs{ifcase}。如果您想要数字 \texttt{0}\ 和 \texttt{1},
% 请在 \cs{bookmarkget}\ 前面加上 \cs{number}\ 作为前缀。\cs{bookmarkget}\ 宏是可展开的(expandable)。
% 如果选项不受支持,则返回空字符串(empty string)。受支持的布尔选项有:
% \begin{quote}
%   \xoption{bold}、
%   \xoption{italic}、
%   \xoption{open}
% \end{quote}
% 其他受支持的选项有:
% \begin{quote}
%   \xoption{depth}、
%   \xoption{dest}、
%   \xoption{color}、
%   \xoption{gotor}、
%   \xoption{level}、
%   \xoption{named}、
%   \xoption{openlevel}、
%   \xoption{page}、
%   \xoption{rawaction}、
%   \xoption{uri}、
%   \xoption{view}、
% \end{quote}
% 另外,以下键(key)是可用的:
% \begin{quote}
%   \xoption{text}
% \end{quote}
% 它返回大纲条目(outline entry)的文本(text)。
%
% \paragraph{选项设置(Option setting)。}
% 在钩子(hook)内部可以使用 \cs{bookmarksetup}\ 设置选项。
%
% \subsection{与 \xpackage{hyperref}\ 的兼容性}
%
% \xpackage{bookmark}\ 宏包自动禁用 \xpackage{hyperref}\ 宏包的书签(bookmarks)。但是,
% \xpackage{bookmark}\ 宏包使用了 \xpackage{hyperref}\ 宏包的一些代码。例如,
% \xpackage{bookmark}\ 宏包重新定义了 \xpackage{hyperref}\ 宏包在 \cs{addcontentsline}\ 命令
% 和其他命令中插入的\cs{Hy@writebookmark}\ 钩子。因此,不应禁用 \xpackage{hyperref}\ 宏包的书签。
%
% \xpackage{bookmark}\ 宏包使用 \xpackage{hyperref}\ 宏包的 \cs{pdfstringdef},且不提供替换(replacement)。
%
% \xpackage{hyperref}\ 宏包的一些选项也能在 \xpackage{bookmark}\ 宏包中实现(implemented):
% \begin{quote}
% \begin{tabular}{|l@{}|l@{}|}
%   \hline
%   \xpackage{hyperref}\ 宏包的选项\  &\ \xpackage{bookmark}\ 宏包的选项\ \ \\ \hline
%   \xoption{bookmarksdepth} &\ \xoption{depth}\\ \hline
%   \xoption{bookmarksopen} & \ \xoption{open}\\ \hline
%   \xoption{bookmarksopenlevel}\ \ \  &\ \xoption{openlevel}\\ \hline
%   \xoption{bookmarksnumbered} \ \ \ &\ \xoption{numbered}\\ \hline
% \end{tabular}
% \end{quote}
%
% 还可以使用以下命令:
% \begin{quote}
%   \cs{pdfbookmark}\\
%   \cs{currentpdfbookmark}\\
%   \cs{subpdfbookmark}\\
%   \cs{belowpdfbookmark}
% \end{quote}
%
% \subsection{在末尾添加书签}
%
% 宏包选项 \xoption{atend}\ 启用以下宏(macro):
% \begin{declcs}{BookmarkAtEnd}
%   \M{stuff}
% \end{declcs}
% \cs{BookmarkAtEnd}\ 宏将 \meta{stuff}\ 放在文档末尾。\meta{stuff}\ 表示书签命令(bookmark commands)。举例:
% \begin{quote}
%\begin{verbatim}
%\usepackage[atend]{bookmark}
%\BookmarkAtEnd{%
%  \bookmarksetup{startatroot}%
%  \bookmark[named=LastPage, level=0]{Last page}%
%}
%\end{verbatim}
% \end{quote}
%
% 或者,可以在 \cs{bookmark}\ 中给出 \xoption{startatroot}\ 选项:
% \begin{quote}
%\begin{verbatim}
%\BookmarkAtEnd{%
%  \bookmark[
%    startatroot,
%    named=LastPage,
%    level=0,
%  ]{Last page}%
%}
%\end{verbatim}
% \end{quote}
%
% \paragraph{备注(Remarks):}
% \begin{itemize}
% \item
%   \cs{BookmarkAtEnd} 隐藏了这样一个事实,即在文档末尾添加书签的方法取决于驱动程序(driver)。
%
%   为此,驱动程序 \xoption{pdftex}\ 使用 \xpackage{atveryend}\ 宏包。如果 \cs{AtEndDocument}\ 太早,
%   最后一个页面(last page)可能不会被发送出去(shipped out)。由于需要 \xext{aux}\ 文件,此驱动程序使
%   用 \cs{AfterLastShipout}。
%
%   其他驱动程序(\xoption{dvipdfm}、\xoption{xetex}、\xoption{vtex})的实现(implementation)
%   取决于 \cs{special},\cs{special}\ 在最后一页之后没有效果。在这种情况下,\xpackage{atenddvi}\ 宏包的
%   \cs{AtEndDvi}\ 有帮助。它将其参数(argument)放在文档的最后一页(last page)。至少需要运行 \hologo{LaTeX}\ 两次,
%   因为最后一页是由引用(reference)检测到的。
%
%   \xoption{dvips}\ 现在使用新的 LaTeX 钩子 \texttt{shipout/lastpage}。
% \item
%   未指定 \cs{BookmarkAtEnd}\ 参数的扩展时间(time of expansion)。这可以立即发生,也可以在文档末尾发生。
% \end{itemize}
%
% \subsection{限制/行动计划}
%
% \begin{itemize}
% \item 支持缺失动作(missing actions)(启动,\dots)。
% \item 对 \xpackage{hyperref}\ 的 \xoption{bookmarkstype}\ 选项进行了更好的设计(design)。
% \end{itemize}
%
% \section{示例(Example)}
%
%    \begin{macrocode}
%<*example>
%    \end{macrocode}
%    \begin{macrocode}
\documentclass{article}
\usepackage{xcolor}[2007/01/21]
\usepackage{hyperref}
\usepackage[
  open,
  openlevel=2,
  atend
]{bookmark}[2019/12/03]

\bookmarksetup{color=blue}

\BookmarkAtEnd{%
  \bookmarksetup{startatroot}%
  \bookmark[named=LastPage, level=0]{End/Last page}%
  \bookmark[named=FirstPage, level=1]{First page}%
}

\begin{document}
\section{First section}
\subsection{Subsection A}
\begin{figure}
  \hypertarget{fig}{}%
  A figure.
\end{figure}
\bookmark[
  rellevel=1,
  keeplevel,
  dest=fig
]{A figure}
\subsection{Subsection B}
\subsubsection{Subsubsection C}
\subsection{Umlauts: \"A\"O\"U\"a\"o\"u\ss}
\newpage
\bookmarksetup{
  bold,
  color=[rgb]{1,0,0}
}
\section{Very important section}
\bookmarksetup{
  italic,
  bold=false,
  color=blue
}
\subsection{Italic section}
\bookmarksetup{
  italic=false
}
\part{Misc}
\section{Diverse}
\subsubsection{Subsubsection, omitting subsection}
\bookmarksetup{
  startatroot
}
\section{Last section outside part}
\subsection{Subsection}
\bookmarksetup{
  color={}
}
\begingroup
  \bookmarksetup{level=0, color=green!80!black}
  \bookmark[named=FirstPage]{First page}
  \bookmark[named=LastPage]{Last page}
  \bookmark[named=PrevPage]{Previous page}
  \bookmark[named=NextPage]{Next page}
\endgroup
\bookmark[
  page=2,
  view=FitH 800
]{Page 2, FitH 800}
\bookmark[
  page=2,
  view=FitBH \calc{\paperheight-\topmargin-1in-\headheight-\headsep}
]{Page 2, FitBH top of text body}
\bookmark[
  uri={http://www.dante.de/},
  color=magenta
]{Dante homepage}
\bookmark[
  gotor={t.pdf},
  page=1,
  view={XYZ 0 1000 null},
  color=cyan!75!black
]{File t.pdf}
\bookmark[named=FirstPage]{First page}
\bookmark[rellevel=1, named=LastPage]{Last page (rellevel=1)}
\bookmark[named=PrevPage]{Previous page}
\bookmark[level=0, named=FirstPage]{First page (level=0)}
\bookmark[
  rellevel=1,
  keeplevel,
  named=LastPage
]{Last page (rellevel=1, keeplevel)}
\bookmark[named=PrevPage]{Previous page}
\end{document}
%    \end{macrocode}
%    \begin{macrocode}
%</example>
%    \end{macrocode}
%
% \StopEventually{
% }
%
% \section{实现(Implementation)}
%
% \subsection{宏包(Package)}
%
%    \begin{macrocode}
%<*package>
\NeedsTeXFormat{LaTeX2e}
\ProvidesPackage{bookmark}%
  [2020-11-06 v1.29 PDF bookmarks (HO)]%
%    \end{macrocode}
%
% \subsubsection{要求(Requirements)}
%
% \paragraph{\hologo{eTeX}.}
%
%    \begin{macro}{\BKM@CalcExpr}
%    \begin{macrocode}
\begingroup\expandafter\expandafter\expandafter\endgroup
\expandafter\ifx\csname numexpr\endcsname\relax
  \def\BKM@CalcExpr#1#2#3#4{%
    \begingroup
      \count@=#2\relax
      \advance\count@ by#3#4\relax
      \edef\x{\endgroup
        \def\noexpand#1{\the\count@}%
      }%
    \x
  }%
\else
  \def\BKM@CalcExpr#1#2#3#4{%
    \edef#1{%
      \the\numexpr#2#3#4\relax
    }%
  }%
\fi
%    \end{macrocode}
%    \end{macro}
%
% \paragraph{\hologo{pdfTeX}\ 的转义功能(escape features)}
%
%    \begin{macro}{\BKM@EscapeName}
%    \begin{macrocode}
\def\BKM@EscapeName#1{%
  \ifx#1\@empty
  \else
    \EdefEscapeName#1#1%
  \fi
}%
%    \end{macrocode}
%    \end{macro}
%    \begin{macro}{\BKM@EscapeString}
%    \begin{macrocode}
\def\BKM@EscapeString#1{%
  \ifx#1\@empty
  \else
    \EdefEscapeString#1#1%
  \fi
}%
%    \end{macrocode}
%    \end{macro}
%    \begin{macro}{\BKM@EscapeHex}
%    \begin{macrocode}
\def\BKM@EscapeHex#1{%
  \ifx#1\@empty
  \else
    \EdefEscapeHex#1#1%
  \fi
}%
%    \end{macrocode}
%    \end{macro}
%    \begin{macro}{\BKM@UnescapeHex}
%    \begin{macrocode}
\def\BKM@UnescapeHex#1{%
  \EdefUnescapeHex#1#1%
}%
%    \end{macrocode}
%    \end{macro}
%
% \paragraph{宏包(Packages)。}
%
% 不要加载由 \xpackage{hyperref}\ 加载的宏包
%    \begin{macrocode}
\RequirePackage{hyperref}[2010/06/18]
%    \end{macrocode}
%
% \subsubsection{宏包选项(Package options)}
%
%    \begin{macrocode}
\SetupKeyvalOptions{family=BKM,prefix=BKM@}
\DeclareLocalOptions{%
  atend,%
  bold,%
  color,%
  depth,%
  dest,%
  draft,%
  final,%
  gotor,%
  italic,%
  keeplevel,%
  level,%
  named,%
  numbered,%
  open,%
  openlevel,%
  page,%
  rawaction,%
  rellevel,%
  srcfile,%
  srcline,%
  startatroot,%
  uri,%
  view,%
}
%    \end{macrocode}
%    \begin{macro}{\bookmarksetup}
%    \begin{macrocode}
\newcommand*{\bookmarksetup}{\kvsetkeys{BKM}}
%    \end{macrocode}
%    \end{macro}
%    \begin{macro}{\BKM@setup}
%    \begin{macrocode}
\def\BKM@setup#1{%
  \bookmarksetup{#1}%
  \ifx\BKM@HookNext\ltx@empty
  \else
    \expandafter\bookmarksetup\expandafter{\BKM@HookNext}%
    \BKM@HookNextClear
  \fi
  \BKM@hook
  \ifBKM@keeplevel
  \else
    \xdef\BKM@currentlevel{\BKM@level}%
  \fi
}
%    \end{macrocode}
%    \end{macro}
%
%    \begin{macro}{\bookmarksetupnext}
%    \begin{macrocode}
\newcommand*{\bookmarksetupnext}[1]{%
  \ltx@GlobalAppendToMacro\BKM@HookNext{,#1}%
}
%    \end{macrocode}
%    \end{macro}
%    \begin{macro}{\BKM@setupnext}
%    \begin{macrocode}
%    \end{macrocode}
%    \end{macro}
%    \begin{macro}{\BKM@HookNextClear}
%    \begin{macrocode}
\def\BKM@HookNextClear{%
  \global\let\BKM@HookNext\ltx@empty
}
%    \end{macrocode}
%    \end{macro}
%    \begin{macro}{\BKM@HookNext}
%    \begin{macrocode}
\BKM@HookNextClear
%    \end{macrocode}
%    \end{macro}
%
%    \begin{macrocode}
\DeclareBoolOption{draft}
\DeclareComplementaryOption{final}{draft}
%    \end{macrocode}
%    \begin{macro}{\BKM@DisableOptions}
%    \begin{macrocode}
\def\BKM@DisableOptions{%
  \DisableKeyvalOption[action=warning,package=bookmark]%
      {BKM}{draft}%
  \DisableKeyvalOption[action=warning,package=bookmark]%
      {BKM}{final}%
}
%    \end{macrocode}
%    \end{macro}
%    \begin{macrocode}
\DeclareBoolOption[\ifHy@bookmarksopen true\else false\fi]{open}
%    \end{macrocode}
%    \begin{macro}{\bookmark@open}
%    \begin{macrocode}
\def\bookmark@open{%
  \ifBKM@open\ltx@one\else\ltx@zero\fi
}
%    \end{macrocode}
%    \end{macro}
%    \begin{macrocode}
\DeclareStringOption[\maxdimen]{openlevel}
%    \end{macrocode}
%    \begin{macro}{\BKM@openlevel}
%    \begin{macrocode}
\edef\BKM@openlevel{\number\@bookmarksopenlevel}
%    \end{macrocode}
%    \end{macro}
%    \begin{macrocode}
%\DeclareStringOption[\c@tocdepth]{depth}
\ltx@IfUndefined{Hy@bookmarksdepth}{%
  \def\BKM@depth{\c@tocdepth}%
}{%
  \let\BKM@depth\Hy@bookmarksdepth
}
\define@key{BKM}{depth}[]{%
  \edef\BKM@param{#1}%
  \ifx\BKM@param\@empty
    \def\BKM@depth{\c@tocdepth}%
  \else
    \ltx@IfUndefined{toclevel@\BKM@param}{%
      \@onelevel@sanitize\BKM@param
      \edef\BKM@temp{\expandafter\@car\BKM@param\@nil}%
      \ifcase 0\expandafter\ifx\BKM@temp-1\fi
              \expandafter\ifnum\expandafter`\BKM@temp>47 %
                \expandafter\ifnum\expandafter`\BKM@temp<58 %
                  1%
                \fi
              \fi
              \relax
        \PackageWarning{bookmark}{%
          Unknown document division name (\BKM@param)\MessageBreak
          for option `depth'%
        }%
      \else
        \BKM@SetDepthOrLevel\BKM@depth\BKM@param
      \fi
    }{%
      \BKM@SetDepthOrLevel\BKM@depth{%
        \csname toclevel@\BKM@param\endcsname
      }%
    }%
  \fi
}
%    \end{macrocode}
%    \begin{macro}{\bookmark@depth}
%    \begin{macrocode}
\def\bookmark@depth{\BKM@depth}
%    \end{macrocode}
%    \end{macro}
%    \begin{macro}{\BKM@SetDepthOrLevel}
%    \begin{macrocode}
\def\BKM@SetDepthOrLevel#1#2{%
  \begingroup
    \setbox\z@=\hbox{%
      \count@=#2\relax
      \expandafter
    }%
  \expandafter\endgroup
  \expandafter\def\expandafter#1\expandafter{\the\count@}%
}
%    \end{macrocode}
%    \end{macro}
%    \begin{macrocode}
\DeclareStringOption[\BKM@currentlevel]{level}[\BKM@currentlevel]
\define@key{BKM}{level}{%
  \edef\BKM@param{#1}%
  \ifx\BKM@param\BKM@MacroCurrentLevel
    \let\BKM@level\BKM@param
  \else
    \ltx@IfUndefined{toclevel@\BKM@param}{%
      \@onelevel@sanitize\BKM@param
      \edef\BKM@temp{\expandafter\@car\BKM@param\@nil}%
      \ifcase 0\expandafter\ifx\BKM@temp-1\fi
              \expandafter\ifnum\expandafter`\BKM@temp>47 %
                \expandafter\ifnum\expandafter`\BKM@temp<58 %
                  1%
                \fi
              \fi
              \relax
        \PackageWarning{bookmark}{%
          Unknown document division name (\BKM@param)\MessageBreak
          for option `level'%
        }%
      \else
        \BKM@SetDepthOrLevel\BKM@level\BKM@param
      \fi
    }{%
      \BKM@SetDepthOrLevel\BKM@level{%
        \csname toclevel@\BKM@param\endcsname
      }%
    }%
  \fi
}
%    \end{macrocode}
%    \begin{macro}{\BKM@MacroCurrentLevel}
%    \begin{macrocode}
\def\BKM@MacroCurrentLevel{\BKM@currentlevel}
%    \end{macrocode}
%    \end{macro}
%    \begin{macrocode}
\DeclareBoolOption{keeplevel}
\DeclareBoolOption{startatroot}
%    \end{macrocode}
%    \begin{macro}{\BKM@startatrootfalse}
%    \begin{macrocode}
\def\BKM@startatrootfalse{%
  \global\let\ifBKM@startatroot\iffalse
}
%    \end{macrocode}
%    \end{macro}
%    \begin{macro}{\BKM@startatroottrue}
%    \begin{macrocode}
\def\BKM@startatroottrue{%
  \global\let\ifBKM@startatroot\iftrue
}
%    \end{macrocode}
%    \end{macro}
%    \begin{macrocode}
\define@key{BKM}{rellevel}{%
  \BKM@CalcExpr\BKM@level{#1}+\BKM@currentlevel
}
%    \end{macrocode}
%    \begin{macro}{\bookmark@level}
%    \begin{macrocode}
\def\bookmark@level{\BKM@level}
%    \end{macrocode}
%    \end{macro}
%    \begin{macro}{\BKM@currentlevel}
%    \begin{macrocode}
\def\BKM@currentlevel{0}
%    \end{macrocode}
%    \end{macro}
%    Make \xpackage{bookmark}'s option \xoption{numbered} an alias
%    for \xpackage{hyperref}'s \xoption{bookmarksnumbered}.
%    \begin{macrocode}
\DeclareBoolOption[%
  \ifHy@bookmarksnumbered true\else false\fi
]{numbered}
\g@addto@macro\BKM@numberedtrue{%
  \let\ifHy@bookmarksnumbered\iftrue
}
\g@addto@macro\BKM@numberedfalse{%
  \let\ifHy@bookmarksnumbered\iffalse
}
\g@addto@macro\Hy@bookmarksnumberedtrue{%
  \let\ifBKM@numbered\iftrue
}
\g@addto@macro\Hy@bookmarksnumberedfalse{%
  \let\ifBKM@numbered\iffalse
}
%    \end{macrocode}
%    \begin{macro}{\bookmark@numbered}
%    \begin{macrocode}
\def\bookmark@numbered{%
  \ifBKM@numbered\ltx@one\else\ltx@zero\fi
}
%    \end{macrocode}
%    \end{macro}
%
% \paragraph{重定义 \xpackage{hyperref}\ 宏包的选项}
%
%    \begin{macro}{\BKM@PatchHyperrefOption}
%    \begin{macrocode}
\def\BKM@PatchHyperrefOption#1{%
  \expandafter\BKM@@PatchHyperrefOption\csname KV@Hyp@#1\endcsname%
}
%    \end{macrocode}
%    \end{macro}
%    \begin{macro}{\BKM@@PatchHyperrefOption}
%    \begin{macrocode}
\def\BKM@@PatchHyperrefOption#1{%
  \expandafter\BKM@@@PatchHyperrefOption#1{##1}\BKM@nil#1%
}
%    \end{macrocode}
%    \end{macro}
%    \begin{macro}{\BKM@@@PatchHyperrefOption}
%    \begin{macrocode}
\def\BKM@@@PatchHyperrefOption#1\BKM@nil#2#3{%
  \def#2##1{%
    #1%
    \bookmarksetup{#3={##1}}%
  }%
}
%    \end{macrocode}
%    \end{macro}
%    \begin{macrocode}
\BKM@PatchHyperrefOption{bookmarksopen}{open}
\BKM@PatchHyperrefOption{bookmarksopenlevel}{openlevel}
\BKM@PatchHyperrefOption{bookmarksdepth}{depth}
%    \end{macrocode}
%
% \paragraph{字体样式(font style)选项。}
%
%    注意:\xpackage{bitset}\ 宏是基于零的,PDF 规范(PDF specifications)以1开头。
%    \begin{macrocode}
\bitsetReset{BKM@FontStyle}%
\define@key{BKM}{italic}[true]{%
  \expandafter\ifx\csname if#1\endcsname\iftrue
    \bitsetSet{BKM@FontStyle}{0}%
  \else
    \bitsetClear{BKM@FontStyle}{0}%
  \fi
}%
\define@key{BKM}{bold}[true]{%
  \expandafter\ifx\csname if#1\endcsname\iftrue
    \bitsetSet{BKM@FontStyle}{1}%
  \else
    \bitsetClear{BKM@FontStyle}{1}%
  \fi
}%
%    \end{macrocode}
%    \begin{macro}{\bookmark@italic}
%    \begin{macrocode}
\def\bookmark@italic{%
  \ifnum\bitsetGet{BKM@FontStyle}{0}=1 \ltx@one\else\ltx@zero\fi
}
%    \end{macrocode}
%    \end{macro}
%    \begin{macro}{\bookmark@bold}
%    \begin{macrocode}
\def\bookmark@bold{%
  \ifnum\bitsetGet{BKM@FontStyle}{1}=1 \ltx@one\else\ltx@zero\fi
}
%    \end{macrocode}
%    \end{macro}
%    \begin{macro}{\BKM@PrintStyle}
%    \begin{macrocode}
\def\BKM@PrintStyle{%
  \bitsetGetDec{BKM@FontStyle}%
}%
%    \end{macrocode}
%    \end{macro}
%
% \paragraph{颜色(color)选项。}
%
%    \begin{macrocode}
\define@key{BKM}{color}{%
  \HyColor@BookmarkColor{#1}\BKM@color{bookmark}{color}%
}
%    \end{macrocode}
%    \begin{macro}{\BKM@color}
%    \begin{macrocode}
\let\BKM@color\@empty
%    \end{macrocode}
%    \end{macro}
%    \begin{macro}{\bookmark@color}
%    \begin{macrocode}
\def\bookmark@color{\BKM@color}
%    \end{macrocode}
%    \end{macro}
%
% \subsubsection{动作(action)选项}
%
%    \begin{macrocode}
\def\BKM@temp#1{%
  \DeclareStringOption{#1}%
  \expandafter\edef\csname bookmark@#1\endcsname{%
    \expandafter\noexpand\csname BKM@#1\endcsname
  }%
}
%    \end{macrocode}
%    \begin{macro}{\bookmark@dest}
%    \begin{macrocode}
\BKM@temp{dest}
%    \end{macrocode}
%    \end{macro}
%    \begin{macro}{\bookmark@named}
%    \begin{macrocode}
\BKM@temp{named}
%    \end{macrocode}
%    \end{macro}
%    \begin{macro}{\bookmark@uri}
%    \begin{macrocode}
\BKM@temp{uri}
%    \end{macrocode}
%    \end{macro}
%    \begin{macro}{\bookmark@gotor}
%    \begin{macrocode}
\BKM@temp{gotor}
%    \end{macrocode}
%    \end{macro}
%    \begin{macro}{\bookmark@rawaction}
%    \begin{macrocode}
\BKM@temp{rawaction}
%    \end{macrocode}
%    \end{macro}
%
%    \begin{macrocode}
\define@key{BKM}{page}{%
  \def\BKM@page{#1}%
  \ifx\BKM@page\@empty
  \else
    \edef\BKM@page{\number\BKM@page}%
    \ifnum\BKM@page>\z@
    \else
      \PackageError{bookmark}{Page must be positive}\@ehc
      \def\BKM@page{1}%
    \fi
  \fi
}
%    \end{macrocode}
%    \begin{macro}{\BKM@page}
%    \begin{macrocode}
\let\BKM@page\@empty
%    \end{macrocode}
%    \end{macro}
%    \begin{macro}{\bookmark@page}
%    \begin{macrocode}
\def\bookmark@page{\BKM@@page}
%    \end{macrocode}
%    \end{macro}
%
%    \begin{macrocode}
\define@key{BKM}{view}{%
  \BKM@CheckView{#1}%
}
%    \end{macrocode}
%    \begin{macro}{\BKM@view}
%    \begin{macrocode}
\let\BKM@view\@empty
%    \end{macrocode}
%    \end{macro}
%    \begin{macro}{\bookmark@view}
%    \begin{macrocode}
\def\bookmark@view{\BKM@view}
%    \end{macrocode}
%    \end{macro}
%    \begin{macro}{BKM@CheckView}
%    \begin{macrocode}
\def\BKM@CheckView#1{%
  \BKM@CheckViewType#1 \@nil
}
%    \end{macrocode}
%    \end{macro}
%    \begin{macro}{\BKM@CheckViewType}
%    \begin{macrocode}
\def\BKM@CheckViewType#1 #2\@nil{%
  \def\BKM@type{#1}%
  \@onelevel@sanitize\BKM@type
  \BKM@TestViewType{Fit}{}%
  \BKM@TestViewType{FitB}{}%
  \BKM@TestViewType{FitH}{%
    \BKM@CheckParam#2 \@nil{top}%
  }%
  \BKM@TestViewType{FitBH}{%
    \BKM@CheckParam#2 \@nil{top}%
  }%
  \BKM@TestViewType{FitV}{%
    \BKM@CheckParam#2 \@nil{bottom}%
  }%
  \BKM@TestViewType{FitBV}{%
    \BKM@CheckParam#2 \@nil{bottom}%
  }%
  \BKM@TestViewType{FitR}{%
    \BKM@CheckRect{#2}{ }%
  }%
  \BKM@TestViewType{XYZ}{%
    \BKM@CheckXYZ{#2}{ }%
  }%
  \@car{%
    \PackageError{bookmark}{%
      Unknown view type `\BKM@type',\MessageBreak
      using `FitH' instead%
    }\@ehc
    \def\BKM@view{FitH}%
  }%
  \@nil
}
%    \end{macrocode}
%    \end{macro}
%    \begin{macro}{\BKM@TestViewType}
%    \begin{macrocode}
\def\BKM@TestViewType#1{%
  \def\BKM@temp{#1}%
  \@onelevel@sanitize\BKM@temp
  \ifx\BKM@type\BKM@temp
    \let\BKM@view\BKM@temp
    \expandafter\@car
  \else
    \expandafter\@gobble
  \fi
}
%    \end{macrocode}
%    \end{macro}
%    \begin{macro}{BKM@CheckParam}
%    \begin{macrocode}
\def\BKM@CheckParam#1 #2\@nil#3{%
  \def\BKM@param{#1}%
  \ifx\BKM@param\@empty
    \PackageWarning{bookmark}{%
      Missing parameter (#3) for `\BKM@type',\MessageBreak
      using 0%
    }%
    \def\BKM@param{0}%
  \else
    \BKM@CalcParam
  \fi
  \edef\BKM@view{\BKM@view\space\BKM@param}%
}
%    \end{macrocode}
%    \end{macro}
%    \begin{macro}{BKM@CheckRect}
%    \begin{macrocode}
\def\BKM@CheckRect#1#2{%
  \BKM@@CheckRect#1#2#2#2#2\@nil
}
%    \end{macrocode}
%    \end{macro}
%    \begin{macro}{\BKM@@CheckRect}
%    \begin{macrocode}
\def\BKM@@CheckRect#1 #2 #3 #4 #5\@nil{%
  \def\BKM@temp{0}%
  \def\BKM@param{#1}%
  \ifx\BKM@param\@empty
    \def\BKM@param{0}%
    \def\BKM@temp{1}%
  \else
    \BKM@CalcParam
  \fi
  \edef\BKM@view{\BKM@view\space\BKM@param}%
  \def\BKM@param{#2}%
  \ifx\BKM@param\@empty
    \def\BKM@param{0}%
    \def\BKM@temp{1}%
  \else
    \BKM@CalcParam
  \fi
  \edef\BKM@view{\BKM@view\space\BKM@param}%
  \def\BKM@param{#3}%
  \ifx\BKM@param\@empty
    \def\BKM@param{0}%
    \def\BKM@temp{1}%
  \else
    \BKM@CalcParam
  \fi
  \edef\BKM@view{\BKM@view\space\BKM@param}%
  \def\BKM@param{#4}%
  \ifx\BKM@param\@empty
    \def\BKM@param{0}%
    \def\BKM@temp{1}%
  \else
    \BKM@CalcParam
  \fi
  \edef\BKM@view{\BKM@view\space\BKM@param}%
  \ifnum\BKM@temp>\z@
    \PackageWarning{bookmark}{Missing parameters for `\BKM@type'}%
  \fi
}
%    \end{macrocode}
%    \end{macro}
%    \begin{macro}{\BKM@CheckXYZ}
%    \begin{macrocode}
\def\BKM@CheckXYZ#1#2{%
  \BKM@@CheckXYZ#1#2#2#2\@nil
}
%    \end{macrocode}
%    \end{macro}
%    \begin{macro}{\BKM@@CheckXYZ}
%    \begin{macrocode}
\def\BKM@@CheckXYZ#1 #2 #3 #4\@nil{%
  \def\BKM@param{#1}%
  \let\BKM@temp\BKM@param
  \@onelevel@sanitize\BKM@temp
  \ifx\BKM@param\@empty
    \let\BKM@param\BKM@null
  \else
    \ifx\BKM@temp\BKM@null
    \else
      \BKM@CalcParam
    \fi
  \fi
  \edef\BKM@view{\BKM@view\space\BKM@param}%
  \def\BKM@param{#2}%
  \let\BKM@temp\BKM@param
  \@onelevel@sanitize\BKM@temp
  \ifx\BKM@param\@empty
    \let\BKM@param\BKM@null
  \else
    \ifx\BKM@temp\BKM@null
    \else
      \BKM@CalcParam
    \fi
  \fi
  \edef\BKM@view{\BKM@view\space\BKM@param}%
  \def\BKM@param{#3}%
  \ifx\BKM@param\@empty
    \let\BKM@param\BKM@null
  \fi
  \edef\BKM@view{\BKM@view\space\BKM@param}%
}
%    \end{macrocode}
%    \end{macro}
%    \begin{macro}{\BKM@null}
%    \begin{macrocode}
\def\BKM@null{null}
\@onelevel@sanitize\BKM@null
%    \end{macrocode}
%    \end{macro}
%
%    \begin{macro}{\BKM@CalcParam}
%    \begin{macrocode}
\def\BKM@CalcParam{%
  \begingroup
  \let\calc\@firstofone
  \expandafter\BKM@@CalcParam\BKM@param\@empty\@empty\@nil
}
%    \end{macrocode}
%    \end{macro}
%    \begin{macro}{\BKM@@CalcParam}
%    \begin{macrocode}
\def\BKM@@CalcParam#1#2#3\@nil{%
  \ifx\calc#1%
    \@ifundefined{calc@assign@dimen}{%
      \@ifundefined{dimexpr}{%
        \setlength{\dimen@}{#2}%
      }{%
        \setlength{\dimen@}{\dimexpr#2\relax}%
      }%
    }{%
      \setlength{\dimen@}{#2}%
    }%
    \dimen@.99626\dimen@
    \edef\BKM@param{\strip@pt\dimen@}%
    \expandafter\endgroup
    \expandafter\def\expandafter\BKM@param\expandafter{\BKM@param}%
  \else
    \endgroup
  \fi
}
%    \end{macrocode}
%    \end{macro}
%
% \subsubsection{\xoption{atend}\ 选项}
%
%    \begin{macrocode}
\DeclareBoolOption{atend}
\g@addto@macro\BKM@DisableOptions{%
  \DisableKeyvalOption[action=warning,package=bookmark]%
      {BKM}{atend}%
}
%    \end{macrocode}
%
% \subsubsection{\xoption{style}\ 选项}
%
%    \begin{macro}{\bookmarkdefinestyle}
%    \begin{macrocode}
\newcommand*{\bookmarkdefinestyle}[2]{%
  \@ifundefined{BKM@style@#1}{%
  }{%
    \PackageInfo{bookmark}{Redefining style `#1'}%
  }%
  \@namedef{BKM@style@#1}{#2}%
}
%    \end{macrocode}
%    \end{macro}
%    \begin{macrocode}
\define@key{BKM}{style}{%
  \BKM@StyleCall{#1}%
}
\newif\ifBKM@ok
%    \end{macrocode}
%    \begin{macro}{\BKM@StyleCall}
%    \begin{macrocode}
\def\BKM@StyleCall#1{%
  \@ifundefined{BKM@style@#1}{%
    \PackageWarning{bookmark}{%
      Ignoring unknown style `#1'%
    }%
  }{%
%    \end{macrocode}
%    检查样式堆栈(style stack)。
%    \begin{macrocode}
    \BKM@oktrue
    \edef\BKM@StyleCurrent{#1}%
    \@onelevel@sanitize\BKM@StyleCurrent
    \let\BKM@StyleEntry\BKM@StyleEntryCheck
    \BKM@StyleStack
    \ifBKM@ok
      \expandafter\@firstofone
    \else
      \PackageError{bookmark}{%
        Ignoring recursive call of style `\BKM@StyleCurrent'%
      }\@ehc
      \expandafter\@gobble
    \fi
    {%
%    \end{macrocode}
%    在堆栈上推送当前样式(Push current style on stack)。
%    \begin{macrocode}
      \let\BKM@StyleEntry\relax
      \edef\BKM@StyleStack{%
        \BKM@StyleEntry{\BKM@StyleCurrent}%
        \BKM@StyleStack
      }%
%    \end{macrocode}
%   调用样式(Call style)。
%    \begin{macrocode}
      \expandafter\expandafter\expandafter\bookmarksetup
      \expandafter\expandafter\expandafter{%
        \csname BKM@style@\BKM@StyleCurrent\endcsname
      }%
%    \end{macrocode}
%    从堆栈中弹出当前样式(Pop current style from stack)。
%    \begin{macrocode}
      \BKM@StyleStackPop
    }%
  }%
}
%    \end{macrocode}
%    \end{macro}
%    \begin{macro}{\BKM@StyleStackPop}
%    \begin{macrocode}
\def\BKM@StyleStackPop{%
  \let\BKM@StyleEntry\relax
  \edef\BKM@StyleStack{%
    \expandafter\@gobbletwo\BKM@StyleStack
  }%
}
%    \end{macrocode}
%    \end{macro}
%    \begin{macro}{\BKM@StyleEntryCheck}
%    \begin{macrocode}
\def\BKM@StyleEntryCheck#1{%
  \def\BKM@temp{#1}%
  \ifx\BKM@temp\BKM@StyleCurrent
    \BKM@okfalse
  \fi
}
%    \end{macrocode}
%    \end{macro}
%    \begin{macro}{\BKM@StyleStack}
%    \begin{macrocode}
\def\BKM@StyleStack{}
%    \end{macrocode}
%    \end{macro}
%
% \subsubsection{源文件位置(source file location)选项}
%
%    \begin{macrocode}
\DeclareStringOption{srcline}
\DeclareStringOption{srcfile}
%    \end{macrocode}
%
% \subsubsection{钩子支持(Hook support)}
%
%    \begin{macro}{\BKM@hook}
%    \begin{macrocode}
\def\BKM@hook{}
%    \end{macrocode}
%    \end{macro}
%    \begin{macrocode}
\define@key{BKM}{addtohook}{%
  \ltx@LocalAppendToMacro\BKM@hook{#1}%
}
%    \end{macrocode}
%
%    \begin{macro}{bookmarkget}
%    \begin{macrocode}
\newcommand*{\bookmarkget}[1]{%
  \romannumeral0%
  \ltx@ifundefined{bookmark@#1}{%
    \ltx@space
  }{%
    \expandafter\expandafter\expandafter\ltx@space
    \csname bookmark@#1\endcsname
  }%
}
%    \end{macrocode}
%    \end{macro}
%
% \subsubsection{设置和加载驱动程序}
%
% \paragraph{检测驱动程序。}
%
%    \begin{macro}{\BKM@DefineDriverKey}
%    \begin{macrocode}
\def\BKM@DefineDriverKey#1{%
  \define@key{BKM}{#1}[]{%
    \def\BKM@driver{#1}%
  }%
  \g@addto@macro\BKM@DisableOptions{%
    \DisableKeyvalOption[action=warning,package=bookmark]%
        {BKM}{#1}%
  }%
}
%    \end{macrocode}
%    \end{macro}
%    \begin{macrocode}
\BKM@DefineDriverKey{pdftex}
\BKM@DefineDriverKey{dvips}
\BKM@DefineDriverKey{dvipdfm}
\BKM@DefineDriverKey{dvipdfmx}
\BKM@DefineDriverKey{xetex}
\BKM@DefineDriverKey{vtex}
\define@key{BKM}{dvipdfmx-outline-open}[true]{%
 \PackageWarning{bookmark}{Option 'dvipdfmx-outline-open' is obsolete
   and ignored}{}}
%    \end{macrocode}
%    \begin{macro}{\bookmark@driver}
%    \begin{macrocode}
\def\bookmark@driver{\BKM@driver}
%    \end{macrocode}
%    \end{macro}
%    \begin{macrocode}
\InputIfFileExists{bookmark.cfg}{}{}
%    \end{macrocode}
%    \begin{macro}{\BookmarkDriverDefault}
%    \begin{macrocode}
\providecommand*{\BookmarkDriverDefault}{dvips}
%    \end{macrocode}
%    \end{macro}
%    \begin{macro}{\BKM@driver}
% Lua\TeX\ 和 pdf\TeX\ 共享驱动程序。
%    \begin{macrocode}
\ifpdf
  \def\BKM@driver{pdftex}%
  \ifx\pdfoutline\@undefined
    \ifx\pdfextension\@undefined\else
      \protected\def\pdfoutline{\pdfextension outline }
    \fi
  \fi
\else
  \ifxetex
    \def\BKM@driver{dvipdfm}%
  \else
    \ifvtex
      \def\BKM@driver{vtex}%
    \else
      \edef\BKM@driver{\BookmarkDriverDefault}%
    \fi
  \fi
\fi
%    \end{macrocode}
%    \end{macro}
%
% \paragraph{过程选项(Process options)。}
%
%    \begin{macrocode}
\ProcessKeyvalOptions*
\BKM@DisableOptions
%    \end{macrocode}
%
% \paragraph{\xoption{draft}\ 选项}
%
%    \begin{macrocode}
\ifBKM@draft
  \PackageWarningNoLine{bookmark}{Draft mode on}%
  \let\bookmarksetup\ltx@gobble
  \let\BookmarkAtEnd\ltx@gobble
  \let\bookmarkdefinestyle\ltx@gobbletwo
  \let\bookmarkget\ltx@gobble
  \let\pdfbookmark\ltx@undefined
  \newcommand*{\pdfbookmark}[3][]{}%
  \let\currentpdfbookmark\ltx@gobbletwo
  \let\subpdfbookmark\ltx@gobbletwo
  \let\belowpdfbookmark\ltx@gobbletwo
  \newcommand*{\bookmark}[2][]{}%
  \renewcommand*{\Hy@writebookmark}[5]{}%
  \let\ReadBookmarks\relax
  \let\BKM@DefGotoNameAction\ltx@gobbletwo % package `hypdestopt'
  \expandafter\endinput
\fi
%    \end{macrocode}
%
% \paragraph{验证和加载驱动程序。}
%
%    \begin{macrocode}
\def\BKM@temp{dvipdfmx}%
\ifx\BKM@temp\BKM@driver
  \def\BKM@driver{dvipdfm}%
\fi
\def\BKM@temp{pdftex}%
\ifpdf
  \ifx\BKM@temp\BKM@driver
  \else
    \PackageWarningNoLine{bookmark}{%
      Wrong driver `\BKM@driver', using `pdftex' instead%
    }%
    \let\BKM@driver\BKM@temp
  \fi
\else
  \ifx\BKM@temp\BKM@driver
    \PackageError{bookmark}{%
      Wrong driver, pdfTeX is not running in PDF mode.\MessageBreak
      Package loading is aborted%
    }\@ehc
    \expandafter\expandafter\expandafter\endinput
  \fi
  \def\BKM@temp{dvipdfm}%
  \ifxetex
    \ifx\BKM@temp\BKM@driver
    \else
      \PackageWarningNoLine{bookmark}{%
        Wrong driver `\BKM@driver',\MessageBreak
        using `dvipdfm' for XeTeX instead%
      }%
      \let\BKM@driver\BKM@temp
    \fi
  \else
    \def\BKM@temp{vtex}%
    \ifvtex
      \ifx\BKM@temp\BKM@driver
      \else
        \PackageWarningNoLine{bookmark}{%
          Wrong driver `\BKM@driver',\MessageBreak
          using `vtex' for VTeX instead%
        }%
        \let\BKM@driver\BKM@temp
      \fi
    \else
      \ifx\BKM@temp\BKM@driver
        \PackageError{bookmark}{%
          Wrong driver, VTeX is not running in PDF mode.\MessageBreak
          Package loading is aborted%
        }\@ehc
        \expandafter\expandafter\expandafter\endinput
      \fi
    \fi
  \fi
\fi
\providecommand\IfFormatAtLeastTF{\@ifl@t@r\fmtversion}
\IfFormatAtLeastTF{2020/10/01}{}{\edef\BKM@driver{\BKM@driver-2019-12-03}}
\InputIfFileExists{bkm-\BKM@driver.def}{}{%
  \PackageError{bookmark}{%
    Unsupported driver `\BKM@driver'.\MessageBreak
    Package loading is aborted%
  }\@ehc
  \endinput
}
%    \end{macrocode}
%
% \subsubsection{与 \xpackage{hyperref}\ 的兼容性}
%
%    \begin{macro}{\pdfbookmark}
%    \begin{macrocode}
\let\pdfbookmark\ltx@undefined
\newcommand*{\pdfbookmark}[3][0]{%
  \bookmark[level=#1,dest={#3.#1}]{#2}%
  \hyper@anchorstart{#3.#1}\hyper@anchorend
}
%    \end{macrocode}
%    \end{macro}
%    \begin{macro}{\currentpdfbookmark}
%    \begin{macrocode}
\def\currentpdfbookmark{%
  \pdfbookmark[\BKM@currentlevel]%
}
%    \end{macrocode}
%    \end{macro}
%    \begin{macro}{\subpdfbookmark}
%    \begin{macrocode}
\def\subpdfbookmark{%
  \BKM@CalcExpr\BKM@CalcResult\BKM@currentlevel+1%
  \expandafter\pdfbookmark\expandafter[\BKM@CalcResult]%
}
%    \end{macrocode}
%    \end{macro}
%    \begin{macro}{\belowpdfbookmark}
%    \begin{macrocode}
\def\belowpdfbookmark#1#2{%
  \xdef\BKM@gtemp{\number\BKM@currentlevel}%
  \subpdfbookmark{#1}{#2}%
  \global\let\BKM@currentlevel\BKM@gtemp
}
%    \end{macrocode}
%    \end{macro}
%
%    节号(section number)、文本(text)、标签(label)、级别(level)、文件(file)
%    \begin{macro}{\Hy@writebookmark}
%    \begin{macrocode}
\def\Hy@writebookmark#1#2#3#4#5{%
  \ifnum#4>\BKM@depth\relax
  \else
    \def\BKM@type{#5}%
    \ifx\BKM@type\Hy@bookmarkstype
      \begingroup
        \ifBKM@numbered
          \let\numberline\Hy@numberline
          \let\booknumberline\Hy@numberline
          \let\partnumberline\Hy@numberline
          \let\chapternumberline\Hy@numberline
        \else
          \let\numberline\@gobble
          \let\booknumberline\@gobble
          \let\partnumberline\@gobble
          \let\chapternumberline\@gobble
        \fi
        \bookmark[level=#4,dest={\HyperDestNameFilter{#3}}]{#2}%
      \endgroup
    \fi
  \fi
}
%    \end{macrocode}
%    \end{macro}
%
%    \begin{macro}{\ReadBookmarks}
%    \begin{macrocode}
\let\ReadBookmarks\relax
%    \end{macrocode}
%    \end{macro}
%
%    \begin{macrocode}
%</package>
%    \end{macrocode}
%
% \subsection{dvipdfm 的驱动程序}
%
%    \begin{macrocode}
%<*dvipdfm>
\NeedsTeXFormat{LaTeX2e}
\ProvidesFile{bkm-dvipdfm.def}%
  [2020-11-06 v1.29 bookmark driver for dvipdfm (HO)]%
%    \end{macrocode}
%
%    \begin{macro}{\BKM@id}
%    \begin{macrocode}
\newcount\BKM@id
\BKM@id=\z@
%    \end{macrocode}
%    \end{macro}
%
%    \begin{macro}{\BKM@0}
%    \begin{macrocode}
\@namedef{BKM@0}{000}
%    \end{macrocode}
%    \end{macro}
%    \begin{macro}{\ifBKM@sw}
%    \begin{macrocode}
\newif\ifBKM@sw
%    \end{macrocode}
%    \end{macro}
%
%    \begin{macro}{\bookmark}
%    \begin{macrocode}
\newcommand*{\bookmark}[2][]{%
  \if@filesw
    \begingroup
      \def\bookmark@text{#2}%
      \BKM@setup{#1}%
      \edef\BKM@prev{\the\BKM@id}%
      \global\advance\BKM@id\@ne
      \BKM@swtrue
      \@whilesw\ifBKM@sw\fi{%
        \def\BKM@abslevel{1}%
        \ifnum\ifBKM@startatroot\z@\else\BKM@prev\fi=\z@
          \BKM@startatrootfalse
          \expandafter\xdef\csname BKM@\the\BKM@id\endcsname{%
            0{\BKM@level}\BKM@abslevel
          }%
          \BKM@swfalse
        \else
          \expandafter\expandafter\expandafter\BKM@getx
              \csname BKM@\BKM@prev\endcsname
          \ifnum\BKM@level>\BKM@x@level\relax
            \BKM@CalcExpr\BKM@abslevel\BKM@x@abslevel+1%
            \expandafter\xdef\csname BKM@\the\BKM@id\endcsname{%
              {\BKM@prev}{\BKM@level}\BKM@abslevel
            }%
            \BKM@swfalse
          \else
            \let\BKM@prev\BKM@x@parent
          \fi
        \fi
      }%
      \csname HyPsd@XeTeXBigCharstrue\endcsname
      \pdfstringdef\BKM@title{\bookmark@text}%
      \edef\BKM@FLAGS{\BKM@PrintStyle}%
      \let\BKM@action\@empty
      \ifx\BKM@gotor\@empty
        \ifx\BKM@dest\@empty
          \ifx\BKM@named\@empty
            \ifx\BKM@rawaction\@empty
              \ifx\BKM@uri\@empty
                \ifx\BKM@page\@empty
                  \PackageError{bookmark}{Missing action}\@ehc
                  \edef\BKM@action{/Dest[@page1/Fit]}%
                \else
                  \ifx\BKM@view\@empty
                    \def\BKM@view{Fit}%
                  \fi
                  \edef\BKM@action{/Dest[@page\BKM@page/\BKM@view]}%
                \fi
              \else
                \BKM@EscapeString\BKM@uri
                \edef\BKM@action{%
                  /A<<%
                    /S/URI%
                    /URI(\BKM@uri)%
                  >>%
                }%
              \fi
            \else
              \edef\BKM@action{/A<<\BKM@rawaction>>}%
            \fi
          \else
            \BKM@EscapeName\BKM@named
            \edef\BKM@action{%
              /A<</S/Named/N/\BKM@named>>%
            }%
          \fi
        \else
          \BKM@EscapeString\BKM@dest
          \edef\BKM@action{%
            /A<<%
              /S/GoTo%
              /D(\BKM@dest)%
            >>%
          }%
        \fi
      \else
        \ifx\BKM@dest\@empty
          \ifx\BKM@page\@empty
            \def\BKM@page{0}%
          \else
            \BKM@CalcExpr\BKM@page\BKM@page-1%
          \fi
          \ifx\BKM@view\@empty
            \def\BKM@view{Fit}%
          \fi
          \edef\BKM@action{/D[\BKM@page/\BKM@view]}%
        \else
          \BKM@EscapeString\BKM@dest
          \edef\BKM@action{/D(\BKM@dest)}%
        \fi
        \BKM@EscapeString\BKM@gotor
        \edef\BKM@action{%
          /A<<%
            /S/GoToR%
            /F(\BKM@gotor)%
            \BKM@action
          >>%
        }%
      \fi
      \special{pdf:%
        out
              [%
              \ifBKM@open
                \ifnum\BKM@level<%
                    \expandafter\ltx@firstofone\expandafter
                    {\number\BKM@openlevel} %
                \else
                  -%
                \fi
              \else
                -%
              \fi
              ] %
            \BKM@abslevel
        <<%
          /Title(\BKM@title)%
          \ifx\BKM@color\@empty
          \else
            /C[\BKM@color]%
          \fi
          \ifnum\BKM@FLAGS>\z@
            /F \BKM@FLAGS
          \fi
          \BKM@action
        >>%
      }%
    \endgroup
  \fi
}
%    \end{macrocode}
%    \end{macro}
%    \begin{macro}{\BKM@getx}
%    \begin{macrocode}
\def\BKM@getx#1#2#3{%
  \def\BKM@x@parent{#1}%
  \def\BKM@x@level{#2}%
  \def\BKM@x@abslevel{#3}%
}
%    \end{macrocode}
%    \end{macro}
%
%    \begin{macrocode}
%</dvipdfm>
%    \end{macrocode}
%
% \subsection{\hologo{VTeX}\ 的驱动程序}
%
%    \begin{macrocode}
%<*vtex>
\NeedsTeXFormat{LaTeX2e}
\ProvidesFile{bkm-vtex.def}%
  [2020-11-06 v1.29 bookmark driver for VTeX (HO)]%
%    \end{macrocode}
%
%    \begin{macrocode}
\ifvtexpdf
\else
  \PackageWarningNoLine{bookmark}{%
    The VTeX driver only supports PDF mode%
  }%
\fi
%    \end{macrocode}
%
%    \begin{macro}{\BKM@id}
%    \begin{macrocode}
\newcount\BKM@id
\BKM@id=\z@
%    \end{macrocode}
%    \end{macro}
%
%    \begin{macro}{\BKM@0}
%    \begin{macrocode}
\@namedef{BKM@0}{00}
%    \end{macrocode}
%    \end{macro}
%    \begin{macro}{\ifBKM@sw}
%    \begin{macrocode}
\newif\ifBKM@sw
%    \end{macrocode}
%    \end{macro}
%
%    \begin{macro}{\bookmark}
%    \begin{macrocode}
\newcommand*{\bookmark}[2][]{%
  \if@filesw
    \begingroup
      \def\bookmark@text{#2}%
      \BKM@setup{#1}%
      \edef\BKM@prev{\the\BKM@id}%
      \global\advance\BKM@id\@ne
      \BKM@swtrue
      \@whilesw\ifBKM@sw\fi{%
        \ifnum\ifBKM@startatroot\z@\else\BKM@prev\fi=\z@
          \BKM@startatrootfalse
          \def\BKM@parent{0}%
          \expandafter\xdef\csname BKM@\the\BKM@id\endcsname{%
            0{\BKM@level}%
          }%
          \BKM@swfalse
        \else
          \expandafter\expandafter\expandafter\BKM@getx
              \csname BKM@\BKM@prev\endcsname
          \ifnum\BKM@level>\BKM@x@level\relax
            \let\BKM@parent\BKM@prev
            \expandafter\xdef\csname BKM@\the\BKM@id\endcsname{%
              {\BKM@prev}{\BKM@level}%
            }%
            \BKM@swfalse
          \else
            \let\BKM@prev\BKM@x@parent
          \fi
        \fi
      }%
      \pdfstringdef\BKM@title{\bookmark@text}%
      \BKM@vtex@title
      \edef\BKM@FLAGS{\BKM@PrintStyle}%
      \let\BKM@action\@empty
      \ifx\BKM@gotor\@empty
        \ifx\BKM@dest\@empty
          \ifx\BKM@named\@empty
            \ifx\BKM@rawaction\@empty
              \ifx\BKM@uri\@empty
                \ifx\BKM@page\@empty
                  \PackageError{bookmark}{Missing action}\@ehc
                  \def\BKM@action{!1}%
                \else
                  \edef\BKM@action{!\BKM@page}%
                \fi
              \else
                \BKM@EscapeString\BKM@uri
                \edef\BKM@action{%
                  <u=%
                    /S/URI%
                    /URI(\BKM@uri)%
                  >%
                }%
              \fi
            \else
              \edef\BKM@action{<u=\BKM@rawaction>}%
            \fi
          \else
            \BKM@EscapeName\BKM@named
            \edef\BKM@action{%
              <u=%
                /S/Named%
                /N/\BKM@named
              >%
            }%
          \fi
        \else
          \BKM@EscapeString\BKM@dest
          \edef\BKM@action{\BKM@dest}%
        \fi
      \else
        \ifx\BKM@dest\@empty
          \ifx\BKM@page\@empty
            \def\BKM@page{1}%
          \fi
          \ifx\BKM@view\@empty
            \def\BKM@view{Fit}%
          \fi
          \edef\BKM@action{/D[\BKM@page/\BKM@view]}%
        \else
          \BKM@EscapeString\BKM@dest
          \edef\BKM@action{/D(\BKM@dest)}%
        \fi
        \BKM@EscapeString\BKM@gotor
        \edef\BKM@action{%
          <u=%
            /S/GoToR%
            /F(\BKM@gotor)%
            \BKM@action
          >>%
        }%
      \fi
      \ifx\BKM@color\@empty
        \let\BKM@RGBcolor\@empty
      \else
        \expandafter\BKM@toRGB\BKM@color\@nil
      \fi
      \special{%
        !outline \BKM@action;%
        p=\BKM@parent,%
        i=\number\BKM@id,%
        s=%
          \ifBKM@open
            \ifnum\BKM@level<\BKM@openlevel
              o%
            \else
              c%
            \fi
          \else
            c%
          \fi,%
        \ifx\BKM@RGBcolor\@empty
        \else
          c=\BKM@RGBcolor,%
        \fi
        \ifnum\BKM@FLAGS>\z@
          f=\BKM@FLAGS,%
        \fi
        t=\BKM@title
      }%
    \endgroup
  \fi
}
%    \end{macrocode}
%    \end{macro}
%    \begin{macro}{\BKM@getx}
%    \begin{macrocode}
\def\BKM@getx#1#2{%
  \def\BKM@x@parent{#1}%
  \def\BKM@x@level{#2}%
}
%    \end{macrocode}
%    \end{macro}
%    \begin{macro}{\BKM@toRGB}
%    \begin{macrocode}
\def\BKM@toRGB#1 #2 #3\@nil{%
  \let\BKM@RGBcolor\@empty
  \BKM@toRGBComponent{#1}%
  \BKM@toRGBComponent{#2}%
  \BKM@toRGBComponent{#3}%
}
%    \end{macrocode}
%    \end{macro}
%    \begin{macro}{\BKM@toRGBComponent}
%    \begin{macrocode}
\def\BKM@toRGBComponent#1{%
  \dimen@=#1pt\relax
  \ifdim\dimen@>\z@
    \ifdim\dimen@<\p@
      \dimen@=255\dimen@
      \advance\dimen@ by 32768sp\relax
      \divide\dimen@ by 65536\relax
      \dimen@ii=\dimen@
      \divide\dimen@ii by 16\relax
      \edef\BKM@RGBcolor{%
        \BKM@RGBcolor
        \BKM@toHexDigit\dimen@ii
      }%
      \dimen@ii=16\dimen@ii
      \advance\dimen@-\dimen@ii
      \edef\BKM@RGBcolor{%
        \BKM@RGBcolor
        \BKM@toHexDigit\dimen@
      }%
    \else
      \edef\BKM@RGBcolor{\BKM@RGBcolor FF}%
    \fi
  \else
    \edef\BKM@RGBcolor{\BKM@RGBcolor00}%
  \fi
}
%    \end{macrocode}
%    \end{macro}
%    \begin{macro}{\BKM@toHexDigit}
%    \begin{macrocode}
\def\BKM@toHexDigit#1{%
  \ifcase\expandafter\@firstofone\expandafter{\number#1} %
    0\or 1\or 2\or 3\or 4\or 5\or 6\or 7\or
    8\or 9\or A\or B\or C\or D\or E\or F%
  \fi
}
%    \end{macrocode}
%    \end{macro}
%    \begin{macrocode}
\begingroup
  \catcode`\|=0 %
  \catcode`\\=12 %
%    \end{macrocode}
%    \begin{macro}{\BKM@vtex@title}
%    \begin{macrocode}
  |gdef|BKM@vtex@title{%
    |@onelevel@sanitize|BKM@title
    |edef|BKM@title{|expandafter|BKM@vtex@leftparen|BKM@title\(|@nil}%
    |edef|BKM@title{|expandafter|BKM@vtex@rightparen|BKM@title\)|@nil}%
    |edef|BKM@title{|expandafter|BKM@vtex@zero|BKM@title\0|@nil}%
    |edef|BKM@title{|expandafter|BKM@vtex@one|BKM@title\1|@nil}%
    |edef|BKM@title{|expandafter|BKM@vtex@two|BKM@title\2|@nil}%
    |edef|BKM@title{|expandafter|BKM@vtex@three|BKM@title\3|@nil}%
  }%
%    \end{macrocode}
%    \end{macro}
%    \begin{macro}{\BKM@vtex@leftparen}
%    \begin{macrocode}
  |gdef|BKM@vtex@leftparen#1\(#2|@nil{%
    #1%
    |ifx||#2||%
    |else
      (%
      |ltx@ReturnAfterFi{%
        |BKM@vtex@leftparen#2|@nil
      }%
    |fi
  }%
%    \end{macrocode}
%    \end{macro}
%    \begin{macro}{\BKM@vtex@rightparen}
%    \begin{macrocode}
  |gdef|BKM@vtex@rightparen#1\)#2|@nil{%
    #1%
    |ifx||#2||%
    |else
      )%
      |ltx@ReturnAfterFi{%
        |BKM@vtex@rightparen#2|@nil
      }%
    |fi
  }%
%    \end{macrocode}
%    \end{macro}
%    \begin{macro}{\BKM@vtex@zero}
%    \begin{macrocode}
  |gdef|BKM@vtex@zero#1\0#2|@nil{%
    #1%
    |ifx||#2||%
    |else
      |noexpand|hv@pdf@char0%
      |ltx@ReturnAfterFi{%
        |BKM@vtex@zero#2|@nil
      }%
    |fi
  }%
%    \end{macrocode}
%    \end{macro}
%    \begin{macro}{\BKM@vtex@one}
%    \begin{macrocode}
  |gdef|BKM@vtex@one#1\1#2|@nil{%
    #1%
    |ifx||#2||%
    |else
      |noexpand|hv@pdf@char1%
      |ltx@ReturnAfterFi{%
        |BKM@vtex@one#2|@nil
      }%
    |fi
  }%
%    \end{macrocode}
%    \end{macro}
%    \begin{macro}{\BKM@vtex@two}
%    \begin{macrocode}
  |gdef|BKM@vtex@two#1\2#2|@nil{%
    #1%
    |ifx||#2||%
    |else
      |noexpand|hv@pdf@char2%
      |ltx@ReturnAfterFi{%
        |BKM@vtex@two#2|@nil
      }%
    |fi
  }%
%    \end{macrocode}
%    \end{macro}
%    \begin{macro}{\BKM@vtex@three}
%    \begin{macrocode}
  |gdef|BKM@vtex@three#1\3#2|@nil{%
    #1%
    |ifx||#2||%
    |else
      |noexpand|hv@pdf@char3%
      |ltx@ReturnAfterFi{%
        |BKM@vtex@three#2|@nil
      }%
    |fi
  }%
%    \end{macrocode}
%    \end{macro}
%    \begin{macrocode}
|endgroup
%    \end{macrocode}
%
%    \begin{macrocode}
%</vtex>
%    \end{macrocode}
%
% \subsection{\hologo{pdfTeX}\ 的驱动程序}
%
%    \begin{macrocode}
%<*pdftex>
\NeedsTeXFormat{LaTeX2e}
\ProvidesFile{bkm-pdftex.def}%
  [2020-11-06 v1.29 bookmark driver for pdfTeX (HO)]%
%    \end{macrocode}
%
%    \begin{macro}{\BKM@DO@entry}
%    \begin{macrocode}
\def\BKM@DO@entry#1#2{%
  \begingroup
    \kvsetkeys{BKM@DO}{#1}%
    \def\BKM@DO@title{#2}%
    \ifx\BKM@DO@srcfile\@empty
    \else
      \BKM@UnescapeHex\BKM@DO@srcfile
    \fi
    \BKM@UnescapeHex\BKM@DO@title
    \expandafter\expandafter\expandafter\BKM@getx
        \csname BKM@\BKM@DO@id\endcsname\@empty\@empty
    \let\BKM@attr\@empty
    \ifx\BKM@DO@flags\@empty
    \else
      \edef\BKM@attr{\BKM@attr/F \BKM@DO@flags}%
    \fi
    \ifx\BKM@DO@color\@empty
    \else
      \edef\BKM@attr{\BKM@attr/C[\BKM@DO@color]}%
    \fi
    \ifx\BKM@attr\@empty
    \else
      \edef\BKM@attr{attr{\BKM@attr}}%
    \fi
    \let\BKM@action\@empty
    \ifx\BKM@DO@gotor\@empty
      \ifx\BKM@DO@dest\@empty
        \ifx\BKM@DO@named\@empty
          \ifx\BKM@DO@rawaction\@empty
            \ifx\BKM@DO@uri\@empty
              \ifx\BKM@DO@page\@empty
                \PackageError{bookmark}{%
                  Missing action\BKM@SourceLocation
                }\@ehc
                \edef\BKM@action{goto page1{/Fit}}%
              \else
                \ifx\BKM@DO@view\@empty
                  \def\BKM@DO@view{Fit}%
                \fi
                \edef\BKM@action{goto page\BKM@DO@page{/\BKM@DO@view}}%
              \fi
            \else
              \BKM@UnescapeHex\BKM@DO@uri
              \BKM@EscapeString\BKM@DO@uri
              \edef\BKM@action{user{<</S/URI/URI(\BKM@DO@uri)>>}}%
            \fi
          \else
            \BKM@UnescapeHex\BKM@DO@rawaction
            \edef\BKM@action{%
              user{%
                <<%
                  \BKM@DO@rawaction
                >>%
              }%
            }%
          \fi
        \else
          \BKM@EscapeName\BKM@DO@named
          \edef\BKM@action{%
            user{<</S/Named/N/\BKM@DO@named>>}%
          }%
        \fi
      \else
        \BKM@UnescapeHex\BKM@DO@dest
        \BKM@DefGotoNameAction\BKM@action\BKM@DO@dest
      \fi
    \else
      \ifx\BKM@DO@dest\@empty
        \ifx\BKM@DO@page\@empty
          \def\BKM@DO@page{0}%
        \else
          \BKM@CalcExpr\BKM@DO@page\BKM@DO@page-1%
        \fi
        \ifx\BKM@DO@view\@empty
          \def\BKM@DO@view{Fit}%
        \fi
        \edef\BKM@action{/D[\BKM@DO@page/\BKM@DO@view]}%
      \else
        \BKM@UnescapeHex\BKM@DO@dest
        \BKM@EscapeString\BKM@DO@dest
        \edef\BKM@action{/D(\BKM@DO@dest)}%
      \fi
      \BKM@UnescapeHex\BKM@DO@gotor
      \BKM@EscapeString\BKM@DO@gotor
      \edef\BKM@action{%
        user{%
          <<%
            /S/GoToR%
            /F(\BKM@DO@gotor)%
            \BKM@action
          >>%
        }%
      }%
    \fi
    \pdfoutline\BKM@attr\BKM@action
                count\ifBKM@DO@open\else-\fi\BKM@x@childs
                {\BKM@DO@title}%
  \endgroup
}
%    \end{macrocode}
%    \end{macro}
%    \begin{macro}{\BKM@DefGotoNameAction}
%    \cs{BKM@DefGotoNameAction}\ 宏是一个用于 \xpackage{hypdestopt}\ 宏包的钩子(hook)。
%    \begin{macrocode}
\def\BKM@DefGotoNameAction#1#2{%
  \BKM@EscapeString\BKM@DO@dest
  \edef#1{goto name{#2}}%
}
%    \end{macrocode}
%    \end{macro}
%    \begin{macrocode}
%</pdftex>
%    \end{macrocode}
%
%    \begin{macrocode}
%<*pdftex|pdfmark>
%    \end{macrocode}
%    \begin{macro}{\BKM@SourceLocation}
%    \begin{macrocode}
\def\BKM@SourceLocation{%
  \ifx\BKM@DO@srcfile\@empty
    \ifx\BKM@DO@srcline\@empty
    \else
      .\MessageBreak
      Source: line \BKM@DO@srcline
    \fi
  \else
    \ifx\BKM@DO@srcline\@empty
      .\MessageBreak
      Source: file `\BKM@DO@srcfile'%
    \else
      .\MessageBreak
      Source: file `\BKM@DO@srcfile', line \BKM@DO@srcline
    \fi
  \fi
}
%    \end{macrocode}
%    \end{macro}
%    \begin{macrocode}
%</pdftex|pdfmark>
%    \end{macrocode}
%
% \subsection{具有 pdfmark 特色(specials)的驱动程序}
%
% \subsubsection{dvips 驱动程序}
%
%    \begin{macrocode}
%<*dvips>
\NeedsTeXFormat{LaTeX2e}
\ProvidesFile{bkm-dvips.def}%
  [2020-11-06 v1.29 bookmark driver for dvips (HO)]%
%    \end{macrocode}
%    \begin{macro}{\BKM@PSHeaderFile}
%    \begin{macrocode}
\def\BKM@PSHeaderFile#1{%
  \special{PSfile=#1}%
}
%    \end{macrocode}
%    \begin{macro}{\BKM@filename}
%    \begin{macrocode}
\def\BKM@filename{\jobname.out.ps}
%    \end{macrocode}
%    \end{macro}
%    \begin{macrocode}
\AddToHook{shipout/lastpage}{%
  \BKM@pdfmark@out
  \BKM@PSHeaderFile\BKM@filename
  }
%    \end{macrocode}
%    \end{macro}
%    \begin{macrocode}
%</dvips>
%    \end{macrocode}
%
% \subsubsection{公共部分(Common part)}
%
%    \begin{macrocode}
%<*pdfmark>
%    \end{macrocode}
%
%    \begin{macro}{\BKM@pdfmark@out}
%    不要在这里使用 \xpackage{rerunfilecheck}\ 宏包,因为在 \hologo{TeX}\ 运行期间不会
%    读取 \cs{BKM@filename}\ 文件。
%    \begin{macrocode}
\def\BKM@pdfmark@out{%
  \if@filesw
    \newwrite\BKM@file
    \immediate\openout\BKM@file=\BKM@filename\relax
    \BKM@write{\@percentchar!}%
    \BKM@write{/pdfmark where{pop}}%
    \BKM@write{%
      {%
        /globaldict where{pop globaldict}{userdict}ifelse%
        /pdfmark/cleartomark load put%
      }%
    }%
    \BKM@write{ifelse}%
  \else
    \let\BKM@write\@gobble
    \let\BKM@DO@entry\@gobbletwo
  \fi
}
%    \end{macrocode}
%    \end{macro}
%    \begin{macro}{\BKM@write}
%    \begin{macrocode}
\def\BKM@write#{%
  \immediate\write\BKM@file
}
%    \end{macrocode}
%    \end{macro}
%
%    \begin{macro}{\BKM@DO@entry}
%    Pdfmark 的规范(specification)说明 |/Color| 是颜色(color)的键名(key name),
%    但是 ghostscript 只将键(key)传递到 PDF 文件中,因此键名必须是 |/C|。
%    \begin{macrocode}
\def\BKM@DO@entry#1#2{%
  \begingroup
    \kvsetkeys{BKM@DO}{#1}%
    \ifx\BKM@DO@srcfile\@empty
    \else
      \BKM@UnescapeHex\BKM@DO@srcfile
    \fi
    \def\BKM@DO@title{#2}%
    \BKM@UnescapeHex\BKM@DO@title
    \expandafter\expandafter\expandafter\BKM@getx
        \csname BKM@\BKM@DO@id\endcsname\@empty\@empty
    \let\BKM@attr\@empty
    \ifx\BKM@DO@flags\@empty
    \else
      \edef\BKM@attr{\BKM@attr/F \BKM@DO@flags}%
    \fi
    \ifx\BKM@DO@color\@empty
    \else
      \edef\BKM@attr{\BKM@attr/C[\BKM@DO@color]}%
    \fi
    \let\BKM@action\@empty
    \ifx\BKM@DO@gotor\@empty
      \ifx\BKM@DO@dest\@empty
        \ifx\BKM@DO@named\@empty
          \ifx\BKM@DO@rawaction\@empty
            \ifx\BKM@DO@uri\@empty
              \ifx\BKM@DO@page\@empty
                \PackageError{bookmark}{%
                  Missing action\BKM@SourceLocation
                }\@ehc
                \edef\BKM@action{%
                  /Action/GoTo%
                  /Page 1%
                  /View[/Fit]%
                }%
              \else
                \ifx\BKM@DO@view\@empty
                  \def\BKM@DO@view{Fit}%
                \fi
                \edef\BKM@action{%
                  /Action/GoTo%
                  /Page \BKM@DO@page
                  /View[/\BKM@DO@view]%
                }%
              \fi
            \else
              \BKM@UnescapeHex\BKM@DO@uri
              \BKM@EscapeString\BKM@DO@uri
              \edef\BKM@action{%
                /Action<<%
                  /Subtype/URI%
                  /URI(\BKM@DO@uri)%
                >>%
              }%
            \fi
          \else
            \BKM@UnescapeHex\BKM@DO@rawaction
            \edef\BKM@action{%
              /Action<<%
                \BKM@DO@rawaction
              >>%
            }%
          \fi
        \else
          \BKM@EscapeName\BKM@DO@named
          \edef\BKM@action{%
            /Action<<%
              /Subtype/Named%
              /N/\BKM@DO@named
            >>%
          }%
        \fi
      \else
        \BKM@UnescapeHex\BKM@DO@dest
        \BKM@EscapeString\BKM@DO@dest
        \edef\BKM@action{%
          /Action/GoTo%
          /Dest(\BKM@DO@dest)cvn%
        }%
      \fi
    \else
      \ifx\BKM@DO@dest\@empty
        \ifx\BKM@DO@page\@empty
          \def\BKM@DO@page{1}%
        \fi
        \ifx\BKM@DO@view\@empty
          \def\BKM@DO@view{Fit}%
        \fi
        \edef\BKM@action{%
          /Page \BKM@DO@page
          /View[/\BKM@DO@view]%
        }%
      \else
        \BKM@UnescapeHex\BKM@DO@dest
        \BKM@EscapeString\BKM@DO@dest
        \edef\BKM@action{%
          /Dest(\BKM@DO@dest)cvn%
        }%
      \fi
      \BKM@UnescapeHex\BKM@DO@gotor
      \BKM@EscapeString\BKM@DO@gotor
      \edef\BKM@action{%
        /Action/GoToR%
        /File(\BKM@DO@gotor)%
        \BKM@action
      }%
    \fi
    \BKM@write{[}%
    \BKM@write{/Title(\BKM@DO@title)}%
    \ifnum\BKM@x@childs>\z@
      \BKM@write{/Count \ifBKM@DO@open\else-\fi\BKM@x@childs}%
    \fi
    \ifx\BKM@attr\@empty
    \else
      \BKM@write{\BKM@attr}%
    \fi
    \BKM@write{\BKM@action}%
    \BKM@write{/OUT pdfmark}%
  \endgroup
}
%    \end{macrocode}
%    \end{macro}
%    \begin{macrocode}
%</pdfmark>
%    \end{macrocode}
%
% \subsection{\xoption{pdftex}\ 和 \xoption{pdfmark}\ 的公共部分}
%
%    \begin{macrocode}
%<*pdftex|pdfmark>
%    \end{macrocode}
%
% \subsubsection{写入辅助文件(auxiliary file)}
%
%    \begin{macrocode}
\AddToHook{begindocument}{%
 \immediate\write\@mainaux{\string\providecommand\string\BKM@entry[2]{}}}
%    \end{macrocode}
%
%    \begin{macro}{\BKM@id}
%    \begin{macrocode}
\newcount\BKM@id
\BKM@id=\z@
%    \end{macrocode}
%    \end{macro}
%
%    \begin{macro}{\BKM@0}
%    \begin{macrocode}
\@namedef{BKM@0}{000}
%    \end{macrocode}
%    \end{macro}
%    \begin{macro}{\ifBKM@sw}
%    \begin{macrocode}
\newif\ifBKM@sw
%    \end{macrocode}
%    \end{macro}
%
%    \begin{macro}{\bookmark}
%    \begin{macrocode}
\newcommand*{\bookmark}[2][]{%
  \if@filesw
    \begingroup
      \BKM@InitSourceLocation
      \def\bookmark@text{#2}%
      \BKM@setup{#1}%
      \ifx\BKM@srcfile\@empty
      \else
        \BKM@EscapeHex\BKM@srcfile
      \fi
      \edef\BKM@prev{\the\BKM@id}%
      \global\advance\BKM@id\@ne
      \BKM@swtrue
      \@whilesw\ifBKM@sw\fi{%
        \ifnum\ifBKM@startatroot\z@\else\BKM@prev\fi=\z@
          \BKM@startatrootfalse
          \expandafter\xdef\csname BKM@\the\BKM@id\endcsname{%
            0{\BKM@level}0%
          }%
          \BKM@swfalse
        \else
          \expandafter\expandafter\expandafter\BKM@getx
              \csname BKM@\BKM@prev\endcsname
          \ifnum\BKM@level>\BKM@x@level\relax
            \expandafter\xdef\csname BKM@\the\BKM@id\endcsname{%
              {\BKM@prev}{\BKM@level}0%
            }%
            \ifnum\BKM@prev>\z@
              \BKM@CalcExpr\BKM@CalcResult\BKM@x@childs+1%
              \expandafter\xdef\csname BKM@\BKM@prev\endcsname{%
                {\BKM@x@parent}{\BKM@x@level}{\BKM@CalcResult}%
              }%
            \fi
            \BKM@swfalse
          \else
            \let\BKM@prev\BKM@x@parent
          \fi
        \fi
      }%
      \pdfstringdef\BKM@title{\bookmark@text}%
      \edef\BKM@FLAGS{\BKM@PrintStyle}%
      \csname BKM@HypDestOptHook\endcsname
      \BKM@EscapeHex\BKM@dest
      \BKM@EscapeHex\BKM@uri
      \BKM@EscapeHex\BKM@gotor
      \BKM@EscapeHex\BKM@rawaction
      \BKM@EscapeHex\BKM@title
      \immediate\write\@mainaux{%
        \string\BKM@entry{%
          id=\number\BKM@id
          \ifBKM@open
            \ifnum\BKM@level<\BKM@openlevel
              ,open%
            \fi
          \fi
          \BKM@auxentry{dest}%
          \BKM@auxentry{named}%
          \BKM@auxentry{uri}%
          \BKM@auxentry{gotor}%
          \BKM@auxentry{page}%
          \BKM@auxentry{view}%
          \BKM@auxentry{rawaction}%
          \BKM@auxentry{color}%
          \ifnum\BKM@FLAGS>\z@
            ,flags=\BKM@FLAGS
          \fi
          \BKM@auxentry{srcline}%
          \BKM@auxentry{srcfile}%
        }{\BKM@title}%
      }%
    \endgroup
  \fi
}
%    \end{macrocode}
%    \end{macro}
%    \begin{macro}{\BKM@getx}
%    \begin{macrocode}
\def\BKM@getx#1#2#3{%
  \def\BKM@x@parent{#1}%
  \def\BKM@x@level{#2}%
  \def\BKM@x@childs{#3}%
}
%    \end{macrocode}
%    \end{macro}
%    \begin{macro}{\BKM@auxentry}
%    \begin{macrocode}
\def\BKM@auxentry#1{%
  \expandafter\ifx\csname BKM@#1\endcsname\@empty
  \else
    ,#1={\csname BKM@#1\endcsname}%
  \fi
}
%    \end{macrocode}
%    \end{macro}
%
%    \begin{macro}{\BKM@InitSourceLocation}
%    \begin{macrocode}
\def\BKM@InitSourceLocation{%
  \edef\BKM@srcline{\the\inputlineno}%
  \BKM@LuaTeX@InitFile
  \ifx\BKM@srcfile\@empty
    \ltx@IfUndefined{currfilepath}{}{%
      \edef\BKM@srcfile{\currfilepath}%
    }%
  \fi
}
%    \end{macrocode}
%    \end{macro}
%    \begin{macro}{\BKM@LuaTeX@InitFile}
%    \begin{macrocode}
\ifluatex
  \ifnum\luatexversion>36 %
    \def\BKM@LuaTeX@InitFile{%
      \begingroup
        \ltx@LocToksA={}%
      \edef\x{\endgroup
        \def\noexpand\BKM@srcfile{%
          \the\expandafter\ltx@LocToksA
          \directlua{%
             if status and status.filename then %
               tex.settoks('ltx@LocToksA', status.filename)%
             end%
          }%
        }%
      }\x
    }%
  \else
    \let\BKM@LuaTeX@InitFile\relax
  \fi
\else
  \let\BKM@LuaTeX@InitFile\relax
\fi
%    \end{macrocode}
%    \end{macro}
%
% \subsubsection{读取辅助数据(auxiliary data)}
%
%    \begin{macrocode}
\SetupKeyvalOptions{family=BKM@DO,prefix=BKM@DO@}
\DeclareStringOption[0]{id}
\DeclareBoolOption{open}
\DeclareStringOption{flags}
\DeclareStringOption{color}
\DeclareStringOption{dest}
\DeclareStringOption{named}
\DeclareStringOption{uri}
\DeclareStringOption{gotor}
\DeclareStringOption{page}
\DeclareStringOption{view}
\DeclareStringOption{rawaction}
\DeclareStringOption{srcline}
\DeclareStringOption{srcfile}
%    \end{macrocode}
%
%    \begin{macrocode}
\AtBeginDocument{%
  \let\BKM@entry\BKM@DO@entry
}
%    \end{macrocode}
%
%    \begin{macrocode}
%</pdftex|pdfmark>
%    \end{macrocode}
%
% \subsection{\xoption{atend}\ 选项}
%
% \subsubsection{钩子(Hook)}
%
%    \begin{macrocode}
%<*package>
%    \end{macrocode}
%    \begin{macrocode}
\ifBKM@atend
\else
%    \end{macrocode}
%    \begin{macro}{\BookmarkAtEnd}
%    这是一个虚拟定义(dummy definition),如果没有给出 \xoption{atend}\ 选项,它将生成一个警告。
%    \begin{macrocode}
  \newcommand{\BookmarkAtEnd}[1]{%
    \PackageWarning{bookmark}{%
      Ignored, because option `atend' is missing%
    }%
  }%
%    \end{macrocode}
%    \end{macro}
%    \begin{macrocode}
  \expandafter\endinput
\fi
%    \end{macrocode}
%    \begin{macro}{\BookmarkAtEnd}
%    \begin{macrocode}
\newcommand*{\BookmarkAtEnd}{%
  \g@addto@macro\BKM@EndHook
}
%    \end{macrocode}
%    \end{macro}
%    \begin{macrocode}
\let\BKM@EndHook\@empty
%    \end{macrocode}
%    \begin{macrocode}
%</package>
%    \end{macrocode}
%
% \subsubsection{在文档末尾使用钩子的驱动程序}
%
%    驱动程序 \xoption{pdftex}\ 使用 LaTeX 钩子 \xoption{enddocument/afterlastpage}
%    (相当于以前使用的 \xpackage{atveryend}\ 的 \cs{AfterLastShipout}),因为它仍然需要 \xext{aux}\ 文件。
%    它使用 \cs{pdfoutline}\ 作为最后一页之后可以使用的书签(bookmakrs)。
%    \begin{itemize}
%    \item
%      驱动程序 \xoption{pdftex}\ 使用 \cs{pdfoutline}, \cs{pdfoutline}\ 可以在最后一页之后使用。
%    \end{itemize}
%    \begin{macrocode}
%<*pdftex>
\ifBKM@atend
  \AddToHook{enddocument/afterlastpage}{%
    \BKM@EndHook
  }%
\fi
%</pdftex>
%    \end{macrocode}
%
% \subsubsection{使用 \xoption{shipout/lastpage}\ 的驱动程序}
%
%    其他驱动程序使用 \cs{special}\ 命令实现 \cs{bookmark}。因此,最后的书签(last bookmarks)
%    必须放在最后一页(last page),而不是之后。不能使用 \cs{AtEndDocument},因为为时已晚,
%    最后一页已经输出了。因此,我们使用 LaTeX 钩子 \xoption{shipout/lastpage}。至少需要运行
%    两次 \hologo{LaTeX}。PostScript 驱动程序 \xoption{dvips}\ 使用外部 PostScript 文件作为书签。
%    为了避免与 pgf 发生冲突,文件写入(file writing)也被移到了最后一个输出页面(shipout page)。
%    \begin{macrocode}
%<*dvipdfm|vtex|pdfmark>
\ifBKM@atend
  \AddToHook{shipout/lastpage}{\BKM@EndHook}%
\fi
%</dvipdfm|vtex|pdfmark>
%    \end{macrocode}
%
% \section{安装(Installation)}
%
% \subsection{下载(Download)}
%
% \paragraph{宏包(Package)。} 在 CTAN\footnote{\CTANpkg{bookmark}}上提供此宏包:
% \begin{description}
% \item[\CTAN{macros/latex/contrib/bookmark/bookmark.dtx}] 源文件(source file)。
% \item[\CTAN{macros/latex/contrib/bookmark/bookmark.pdf}] 文档(documentation)。
% \end{description}
%
%
% \paragraph{捆绑包(Bundle)。} “bookmark”捆绑包(bundle)的所有宏包(packages)都可以在兼
% 容 TDS 的 ZIP 归档文件中找到。在那里,宏包已经被解包,文档文件(documentation files)已经生成。
% 文件(files)和目录(directories)遵循 TDS 标准。
% \begin{description}
% \item[\CTANinstall{install/macros/latex/contrib/bookmark.tds.zip}]
% \end{description}
% \emph{TDS}\ 是指标准的“用于 \TeX\ 文件的目录结构(Directory Structure)”(\CTANpkg{tds})。
% 名称中带有 \xfile{texmf}\ 的目录(directories)通常以这种方式组织。
%
% \subsection{捆绑包(Bundle)的安装}
%
% \paragraph{解压(Unpacking)。} 在您选择的 TDS 树(也称为 \xfile{texmf}\ 树)中解
% 压 \xfile{bookmark.tds.zip},例如(在 linux 中):
% \begin{quote}
%   |unzip bookmark.tds.zip -d ~/texmf|
% \end{quote}
%
% \subsection{宏包(Package)的安装}
%
% \paragraph{解压(Unpacking)。} \xfile{.dtx}\ 文件是一个自解压 \docstrip\ 归档文件(archive)。
% 这些文件是通过 \plainTeX\ 运行 \xfile{.dtx}\ 来提取的:
% \begin{quote}
%   \verb|tex bookmark.dtx|
% \end{quote}
%
% \paragraph{TDS.} 现在,不同的文件必须移动到安装 TDS 树(installation TDS tree)
% (也称为 \xfile{texmf}\ 树)中的不同目录中:
% \begin{quote}
% \def\t{^^A
% \begin{tabular}{@{}>{\ttfamily}l@{ $\rightarrow$ }>{\ttfamily}l@{}}
%   bookmark.sty & tex/latex/bookmark/bookmark.sty\\
%   bkm-dvipdfm.def & tex/latex/bookmark/bkm-dvipdfm.def\\
%   bkm-dvips.def & tex/latex/bookmark/bkm-dvips.def\\
%   bkm-pdftex.def & tex/latex/bookmark/bkm-pdftex.def\\
%   bkm-vtex.def & tex/latex/bookmark/bkm-vtex.def\\
%   bookmark.pdf & doc/latex/bookmark/bookmark.pdf\\
%   bookmark-example.tex & doc/latex/bookmark/bookmark-example.tex\\
%   bookmark.dtx & source/latex/bookmark/bookmark.dtx\\
% \end{tabular}^^A
% }^^A
% \sbox0{\t}^^A
% \ifdim\wd0>\linewidth
%   \begingroup
%     \advance\linewidth by\leftmargin
%     \advance\linewidth by\rightmargin
%   \edef\x{\endgroup
%     \def\noexpand\lw{\the\linewidth}^^A
%   }\x
%   \def\lwbox{^^A
%     \leavevmode
%     \hbox to \linewidth{^^A
%       \kern-\leftmargin\relax
%       \hss
%       \usebox0
%       \hss
%       \kern-\rightmargin\relax
%     }^^A
%   }^^A
%   \ifdim\wd0>\lw
%     \sbox0{\small\t}^^A
%     \ifdim\wd0>\linewidth
%       \ifdim\wd0>\lw
%         \sbox0{\footnotesize\t}^^A
%         \ifdim\wd0>\linewidth
%           \ifdim\wd0>\lw
%             \sbox0{\scriptsize\t}^^A
%             \ifdim\wd0>\linewidth
%               \ifdim\wd0>\lw
%                 \sbox0{\tiny\t}^^A
%                 \ifdim\wd0>\linewidth
%                   \lwbox
%                 \else
%                   \usebox0
%                 \fi
%               \else
%                 \lwbox
%               \fi
%             \else
%               \usebox0
%             \fi
%           \else
%             \lwbox
%           \fi
%         \else
%           \usebox0
%         \fi
%       \else
%         \lwbox
%       \fi
%     \else
%       \usebox0
%     \fi
%   \else
%     \lwbox
%   \fi
% \else
%   \usebox0
% \fi
% \end{quote}
% 如果你有一个 \xfile{docstrip.cfg}\ 文件,该文件能配置并启用 \docstrip\ 的 TDS 安装功能,
% 则一些文件可能已经在正确的位置了,请参阅 \docstrip\ 的文档(documentation)。
%
% \subsection{刷新文件名数据库}
%
% 如果您的 \TeX~发行版(\TeX\,Live、\mikTeX、\dots)依赖于文件名数据库(file name databases),
% 则必须刷新这些文件名数据库。例如,\TeX\,Live\ 用户运行 \verb|texhash| 或 \verb|mktexlsr|。
%
% \subsection{一些感兴趣的细节}
%
% \paragraph{用 \LaTeX\ 解压。}
% \xfile{.dtx}\ 根据格式(format)选择其操作(action):
% \begin{description}
% \item[\plainTeX:] 运行 \docstrip\ 并解压文件。
% \item[\LaTeX:] 生成文档。
% \end{description}
% 如果您坚持通过 \LaTeX\ 使用\docstrip (实际上 \docstrip\ 并不需要 \LaTeX),那么请您的意图告知自动检测程序:
% \begin{quote}
%   \verb|latex \let\install=y% \iffalse meta-comment
%
% File: bookmark.dtx
% Version: 2020-11-06 v1.29
% Info: PDF bookmarks
%
% Copyright (C)
%    2007-2011 Heiko Oberdiek
%    2016-2020 Oberdiek Package Support Group
%    https://github.com/ho-tex/bookmark/issues
%
% This work may be distributed and/or modified under the
% conditions of the LaTeX Project Public License, either
% version 1.3c of this license or (at your option) any later
% version. This version of this license is in
%    https://www.latex-project.org/lppl/lppl-1-3c.txt
% and the latest version of this license is in
%    https://www.latex-project.org/lppl.txt
% and version 1.3 or later is part of all distributions of
% LaTeX version 2005/12/01 or later.
%
% This work has the LPPL maintenance status "maintained".
%
% The Current Maintainers of this work are
% Heiko Oberdiek and the Oberdiek Package Support Group
% https://github.com/ho-tex/bookmark/issues
%
% This work consists of the main source file bookmark.dtx
% and the derived files
%    bookmark.sty, bookmark.pdf, bookmark.ins, bookmark.drv,
%    bkm-dvipdfm.def, bkm-dvips.def,
%    bkm-pdftex.def, bkm-vtex.def,
%    bkm-dvipdfm-2019-12-03.def, bkm-dvips-2019-12-03.def,
%    bkm-pdftex-2019-12-03.def, bkm-vtex-2019-12-03.def,
%    bookmark-example.tex.
%
% Distribution:
%    CTAN:macros/latex/contrib/bookmark/bookmark.dtx
%    CTAN:macros/latex/contrib/bookmark/bookmark-frozen.dtx
%    CTAN:macros/latex/contrib/bookmark/bookmark.pdf
%
% Unpacking:
%    (a) If bookmark.ins is present:
%           tex bookmark.ins
%    (b) Without bookmark.ins:
%           tex bookmark.dtx
%    (c) If you insist on using LaTeX
%           latex \let\install=y% \iffalse meta-comment
%
% File: bookmark.dtx
% Version: 2020-11-06 v1.29
% Info: PDF bookmarks
%
% Copyright (C)
%    2007-2011 Heiko Oberdiek
%    2016-2020 Oberdiek Package Support Group
%    https://github.com/ho-tex/bookmark/issues
%
% This work may be distributed and/or modified under the
% conditions of the LaTeX Project Public License, either
% version 1.3c of this license or (at your option) any later
% version. This version of this license is in
%    https://www.latex-project.org/lppl/lppl-1-3c.txt
% and the latest version of this license is in
%    https://www.latex-project.org/lppl.txt
% and version 1.3 or later is part of all distributions of
% LaTeX version 2005/12/01 or later.
%
% This work has the LPPL maintenance status "maintained".
%
% The Current Maintainers of this work are
% Heiko Oberdiek and the Oberdiek Package Support Group
% https://github.com/ho-tex/bookmark/issues
%
% This work consists of the main source file bookmark.dtx
% and the derived files
%    bookmark.sty, bookmark.pdf, bookmark.ins, bookmark.drv,
%    bkm-dvipdfm.def, bkm-dvips.def,
%    bkm-pdftex.def, bkm-vtex.def,
%    bkm-dvipdfm-2019-12-03.def, bkm-dvips-2019-12-03.def,
%    bkm-pdftex-2019-12-03.def, bkm-vtex-2019-12-03.def,
%    bookmark-example.tex.
%
% Distribution:
%    CTAN:macros/latex/contrib/bookmark/bookmark.dtx
%    CTAN:macros/latex/contrib/bookmark/bookmark-frozen.dtx
%    CTAN:macros/latex/contrib/bookmark/bookmark.pdf
%
% Unpacking:
%    (a) If bookmark.ins is present:
%           tex bookmark.ins
%    (b) Without bookmark.ins:
%           tex bookmark.dtx
%    (c) If you insist on using LaTeX
%           latex \let\install=y% \iffalse meta-comment
%
% File: bookmark.dtx
% Version: 2020-11-06 v1.29
% Info: PDF bookmarks
%
% Copyright (C)
%    2007-2011 Heiko Oberdiek
%    2016-2020 Oberdiek Package Support Group
%    https://github.com/ho-tex/bookmark/issues
%
% This work may be distributed and/or modified under the
% conditions of the LaTeX Project Public License, either
% version 1.3c of this license or (at your option) any later
% version. This version of this license is in
%    https://www.latex-project.org/lppl/lppl-1-3c.txt
% and the latest version of this license is in
%    https://www.latex-project.org/lppl.txt
% and version 1.3 or later is part of all distributions of
% LaTeX version 2005/12/01 or later.
%
% This work has the LPPL maintenance status "maintained".
%
% The Current Maintainers of this work are
% Heiko Oberdiek and the Oberdiek Package Support Group
% https://github.com/ho-tex/bookmark/issues
%
% This work consists of the main source file bookmark.dtx
% and the derived files
%    bookmark.sty, bookmark.pdf, bookmark.ins, bookmark.drv,
%    bkm-dvipdfm.def, bkm-dvips.def,
%    bkm-pdftex.def, bkm-vtex.def,
%    bkm-dvipdfm-2019-12-03.def, bkm-dvips-2019-12-03.def,
%    bkm-pdftex-2019-12-03.def, bkm-vtex-2019-12-03.def,
%    bookmark-example.tex.
%
% Distribution:
%    CTAN:macros/latex/contrib/bookmark/bookmark.dtx
%    CTAN:macros/latex/contrib/bookmark/bookmark-frozen.dtx
%    CTAN:macros/latex/contrib/bookmark/bookmark.pdf
%
% Unpacking:
%    (a) If bookmark.ins is present:
%           tex bookmark.ins
%    (b) Without bookmark.ins:
%           tex bookmark.dtx
%    (c) If you insist on using LaTeX
%           latex \let\install=y\input{bookmark.dtx}
%        (quote the arguments according to the demands of your shell)
%
% Documentation:
%    (a) If bookmark.drv is present:
%           latex bookmark.drv
%    (b) Without bookmark.drv:
%           latex bookmark.dtx; ...
%    The class ltxdoc loads the configuration file ltxdoc.cfg
%    if available. Here you can specify further options, e.g.
%    use A4 as paper format:
%       \PassOptionsToClass{a4paper}{article}
%
%    Programm calls to get the documentation (example):
%       pdflatex bookmark.dtx
%       makeindex -s gind.ist bookmark.idx
%       pdflatex bookmark.dtx
%       makeindex -s gind.ist bookmark.idx
%       pdflatex bookmark.dtx
%
% Installation:
%    TDS:tex/latex/bookmark/bookmark.sty
%    TDS:tex/latex/bookmark/bkm-dvipdfm.def
%    TDS:tex/latex/bookmark/bkm-dvips.def
%    TDS:tex/latex/bookmark/bkm-pdftex.def
%    TDS:tex/latex/bookmark/bkm-vtex.def
%    TDS:tex/latex/bookmark/bkm-dvipdfm-2019-12-03.def
%    TDS:tex/latex/bookmark/bkm-dvips-2019-12-03.def
%    TDS:tex/latex/bookmark/bkm-pdftex-2019-12-03.def
%    TDS:tex/latex/bookmark/bkm-vtex-2019-12-03.def%
%    TDS:doc/latex/bookmark/bookmark.pdf
%    TDS:doc/latex/bookmark/bookmark-example.tex
%    TDS:source/latex/bookmark/bookmark.dtx
%    TDS:source/latex/bookmark/bookmark-frozen.dtx
%
%<*ignore>
\begingroup
  \catcode123=1 %
  \catcode125=2 %
  \def\x{LaTeX2e}%
\expandafter\endgroup
\ifcase 0\ifx\install y1\fi\expandafter
         \ifx\csname processbatchFile\endcsname\relax\else1\fi
         \ifx\fmtname\x\else 1\fi\relax
\else\csname fi\endcsname
%</ignore>
%<*install>
\input docstrip.tex
\Msg{************************************************************************}
\Msg{* Installation}
\Msg{* Package: bookmark 2020-11-06 v1.29 PDF bookmarks (HO)}
\Msg{************************************************************************}

\keepsilent
\askforoverwritefalse

\let\MetaPrefix\relax
\preamble

This is a generated file.

Project: bookmark
Version: 2020-11-06 v1.29

Copyright (C)
   2007-2011 Heiko Oberdiek
   2016-2020 Oberdiek Package Support Group

This work may be distributed and/or modified under the
conditions of the LaTeX Project Public License, either
version 1.3c of this license or (at your option) any later
version. This version of this license is in
   https://www.latex-project.org/lppl/lppl-1-3c.txt
and the latest version of this license is in
   https://www.latex-project.org/lppl.txt
and version 1.3 or later is part of all distributions of
LaTeX version 2005/12/01 or later.

This work has the LPPL maintenance status "maintained".

The Current Maintainers of this work are
Heiko Oberdiek and the Oberdiek Package Support Group
https://github.com/ho-tex/bookmark/issues


This work consists of the main source file bookmark.dtx and bookmark-frozen.dtx
and the derived files
   bookmark.sty, bookmark.pdf, bookmark.ins, bookmark.drv,
   bkm-dvipdfm.def, bkm-dvips.def, bkm-pdftex.def, bkm-vtex.def,
   bkm-dvipdfm-2019-12-03.def, bkm-dvips-2019-12-03.def,
   bkm-pdftex-2019-12-03.def, bkm-vtex-2019-12-03.def,
   bookmark-example.tex.

\endpreamble
\let\MetaPrefix\DoubleperCent

\generate{%
  \file{bookmark.ins}{\from{bookmark.dtx}{install}}%
  \file{bookmark.drv}{\from{bookmark.dtx}{driver}}%
  \usedir{tex/latex/bookmark}%
  \file{bookmark.sty}{\from{bookmark.dtx}{package}}%
  \file{bkm-dvipdfm.def}{\from{bookmark.dtx}{dvipdfm}}%
  \file{bkm-dvips.def}{\from{bookmark.dtx}{dvips,pdfmark}}%
  \file{bkm-pdftex.def}{\from{bookmark.dtx}{pdftex}}%
  \file{bkm-vtex.def}{\from{bookmark.dtx}{vtex}}%
  \usedir{doc/latex/bookmark}%
  \file{bookmark-example.tex}{\from{bookmark.dtx}{example}}%
  \file{bkm-pdftex-2019-12-03.def}{\from{bookmark-frozen.dtx}{pdftexfrozen}}%
  \file{bkm-dvips-2019-12-03.def}{\from{bookmark-frozen.dtx}{dvipsfrozen}}%
  \file{bkm-vtex-2019-12-03.def}{\from{bookmark-frozen.dtx}{vtexfrozen}}%
  \file{bkm-dvipdfm-2019-12-03.def}{\from{bookmark-frozen.dtx}{dvipdfmfrozen}}%
}

\catcode32=13\relax% active space
\let =\space%
\Msg{************************************************************************}
\Msg{*}
\Msg{* To finish the installation you have to move the following}
\Msg{* files into a directory searched by TeX:}
\Msg{*}
\Msg{*     bookmark.sty, bkm-dvipdfm.def, bkm-dvips.def,}
\Msg{*     bkm-pdftex.def, bkm-vtex.def, bkm-dvipdfm-2019-12-03.def,}
\Msg{*     bkm-dvips-2019-12-03.def, bkm-pdftex-2019-12-03.def,}
\Msg{*     and bkm-vtex-2019-12-03.def}
\Msg{*}
\Msg{* To produce the documentation run the file `bookmark.drv'}
\Msg{* through LaTeX.}
\Msg{*}
\Msg{* Happy TeXing!}
\Msg{*}
\Msg{************************************************************************}

\endbatchfile
%</install>
%<*ignore>
\fi
%</ignore>
%<*driver>
\NeedsTeXFormat{LaTeX2e}
\ProvidesFile{bookmark.drv}%
  [2020-11-06 v1.29 PDF bookmarks (HO)]%
\documentclass{ltxdoc}
\usepackage{ctex}
\usepackage{indentfirst}
\setlength{\parindent}{2em}
\usepackage{holtxdoc}[2011/11/22]
\usepackage{xcolor}
\usepackage{hyperref}
\usepackage[open,openlevel=3,atend]{bookmark}[2020/11/06] %%%打开书签,显示的深度为3级,即显示part、section、subsection。
\bookmarksetup{color=red}
\begin{document}

  \renewcommand{\contentsname}{目\quad 录}
  \renewcommand{\abstractname}{摘\quad 要}
  \renewcommand{\historyname}{历史}
  \DocInput{bookmark.dtx}%
\end{document}
%</driver>
% \fi
%
%
%
% \GetFileInfo{bookmark.drv}
%
%% \title{\xpackage{bookmark} 宏包}
% \title{\heiti {\Huge \textbf{\xpackage{bookmark}\ 宏包}}}
% \date{2020-11-06\ \ \ v1.29}
% \author{Heiko Oberdiek \thanks
% {如有问题请点击:\url{https://github.com/ho-tex/bookmark/issues}}\\[5pt]赣医一附院神经科\ \ 黄旭华\ \ \ \ 译}
%
% \maketitle
%
% \begin{abstract}
% 这个宏包为 \xpackage{hyperref}\ 宏包实现了一个新的书签(bookmark)(大纲[outline])组织。现在
% 可以设置样式(style)和颜色(color)等书签属性(bookmark properties)。其他动作类型(action types)可用
% (URI、GoToR、Named)。书签是在第一次编译运行(compile run)中生成的。\xpackage{hyperref}\
% 宏包必需运行两次。
% \end{abstract}
%
% \tableofcontents
%
% \section{文档(Documentation)}
%
% \subsection{介绍}
%
% 这个 \xpackage{bookmark}\ 宏包试图为书签(bookmarks)提供一个更现代的管理:
% \begin{itemize}
% \item 书签已经在第一次 \hologo{TeX}\ 编译运行(compile run)中生成。
% \item 可以更改书签的字体样式(font style)和颜色(color)。
% \item 可以执行比简单的 GoTo 操作(actions)更多的操作。
% \end{itemize}
%
% 与 \xpackage{hyperref} \cite{hyperref} 一样,书签(bookmarks)也是按照书签生成宏
% (bookmark generating macros)(\cs{bookmark})的顺序生成的。级别号(level number)用于
% 定义书签的树结构(tree structure)。限制没有那么严格:
% \begin{itemize}
% \item 级别值(level values)可以跳变(jump)和省略(omit)。\cs{subsubsection}\ 可以跟在
%       \cs{chapter}\ 之后。这种情况如在 \xpackage{hyperref}\ 中则产生错误,它将显示一个警告(warning)
%       并尝试修复此错误。
% \item 多个书签可能指向同一目标(destination)。在 \xpackage{hyperref}\ 中,这会完全弄乱
%       书签树(bookmark tree),因为算法假设(algorithm assumes)目标名称(destination names)
%       是键(keys)(唯一的)。
% \end{itemize}
%
% 注意,这个宏包是作为书签管理(bookmark management)的实验平台(experimentation platform)。
% 欢迎反馈。此外,在未来的版本中,接口(interfaces)也可能发生变化。
%
% \subsection{选项(Options)}
%
% 可在以下四个地方放置选项(options):
% \begin{enumerate}
% \item \cs{usepackage}|[|\meta{options}|]{bookmark}|\\
%       这是放置驱动程序选项(driver options)和 \xoption{atend}\ 选项的唯一位置。
% \item \cs{bookmarksetup}|{|\meta{options}|}|\\
%       此命令仅用于设置选项(setting options)。
% \item \cs{bookmarksetupnext}|{|\meta{options}|}|\\
%       这些选项在下一个 \cs{bookmark}\ 命令的选项之后存储(stored)和调用(called)。
% \item \cs{bookmark}|[|\meta{options}|]{|\meta{title}|}|\\
%       此命令设置书签。选项设置(option settings)仅限于此书签。
% \end{enumerate}
% 异常(Exception):加载该宏包后,无法更改驱动程序选项(Driver options)、\xoption{atend}\ 选项
% 、\xoption{draft}\slash\xoption{final}选项。
%
% \subsubsection{\xoption{draft} 和 \xoption{final}\ 选项}
%
% 如果一个\LaTeX\ 文件要被编译了多次,那么可以使用 \xoption{draft}\ 选项来禁用该宏包的书签内
% 容(bookmark stuff),这样可以节省一点时间。默认 \xoption{final}\ 选项。两个选项都是
% 布尔选项(boolean options),如果没有值,则使用值 |true|。|draft=true| 与 |final=false| 相同。
%
% 除了驱动程序选项(driver options)之外,\xpackage{bookmark}\ 宏包选项都是局部选项(local options)。
% \xoption{draft}\ 选项和 \xoption{final}\ 选项均属于文档类选项(class option)(译者注:文档类选项为全局选项),
% 因此,在 \xpackage{bookmark}\ 宏包中未能看到这两个选项。如果您想使用全局的(global) \xoption{draft}选项
% 来优化第一次 \LaTeX\ 运行(runs),可以在导言(preamble)中引入 \xpackage{ifdraft}\ 宏包并设置 \LaTeX\ 的
% \cs{PassOptionsToPackage},例如:
%\begin{quote}
%\begin{verbatim}
%\documentclass[draft]{article}
%\usepackage{ifdraft}
%\ifdraft{%
%   \PassOptionsToPackage{draft}{bookmark}%
%}{}
%\end{verbatim}
%\end{quote}
%
% \subsubsection{驱动程序选项(Driver options)}
%
% 支持的驱动程序( drivers)包括 \xoption{pdftex}、\xoption{dvips}、\xoption{dvipdfm} (\xoption{xetex})、
% \xoption{vtex}。\hologo{TeX}\ 引擎 \hologo{pdfTeX}、\hologo{XeTeX}、\hologo{VTeX}\ 能被自动检测到。
% 默认的 DVI 驱动程序是 \xoption{dvips}。这可以通过 \cs{BookmarkDriverDefault}\ 在配置
% 文件 \xfile{bookmark.cfg}\ 中进行更改,例如:
% \begin{quote}
% |\def\BookmarkDriverDefault{dvipdfm}|
% \end{quote}
% 当前版本的(current versions)驱动程序使用新的 \LaTeX\ 钩子(\LaTeX-hooks)。如果检测到比
% 2020-10-01 更旧的格式,则将以前驱动程序的冻结版本(frozen versions)作为备份(fallback)。
%
% \paragraph{用 dvipdfmx 打开书签(bookmarks)。}旧版本的宏包有一个 \xoption{dvipdfmx-outline-open}\ 选项
% 可以激活代码,而该代码可以指定一个大纲条目(outline entry)是否打开。该宏包现在假设所有使用的 dvipdfmx 版本都是
% 最新版本,足以理解该代码,因此始终激活该代码。选项本身将被忽略。
%
%
% \subsubsection{布局选项(Layout options)}
%
% \paragraph{字体(Font)选项:}
%
% \begin{description}
% \item[\xoption{bold}:] 如果受 PDF 浏览器(PDF viewer)支持,书签将以粗体字体(bold font)显示(自 PDF 1.4起)。
% \item[\xoption{italic}:] 使用斜体字体(italic font)(自 PDF 1.4起)。
% \end{description}
% \xoption{bold}(粗体) 和 \xoption{italic}(斜体)可以同时使用。而 |false| 值(value)禁用字体选项。
%
% \paragraph{颜色(Color)选项:}
%
% 彩色书签(Colored bookmarks)是 PDF 1.4 的一个特性(feature),并非所有的 PDF 浏览器(PDF viewers)都支持彩色书签。
% \begin{description}
% \item[\xoption{color}:] 这里 color(颜色)可以作为 \xpackage{color}\ 宏包或 \xpackage{xcolor}\ 宏包的
% 颜色规范(color specification)给出。空值(empty value)表示未设置颜色属性。如果未加载 \xpackage{xcolor}\ 宏包,
% 能识别的值(recognized values)只有:
%   \begin{itemize}
%   \item 空值(empty value)表示未设置颜色属性,\\
%         例如:|color={}|
%   \item 颜色模型(color model) rgb 的显式颜色规范(explicit color specification),\\
%         例如,红色(red):|color=[rgb]{1,0,0}|
%   \item 颜色模型(color model)灰(gray)的显式颜色规范(explicit color specification),\\
%         例如,深灰色(dark gray):|color=[gray]{0.25}|
%   \end{itemize}
%   请注意,如果加载了 \xpackage{color}\ 宏包,此限制(restriction)也适用。然而,如果加载了 \xpackage{xcolor}\ 宏包,
%   则可以使用所有颜色规范(color specifications)。
% \end{description}
%
% \subsubsection{动作选项(Action options)}
%
% \begin{description}
% \item[\xoption{dest}:] 目的地名称(destination name)。
% \item[\xoption{page}:] 页码(page number),第一页(first page)为 1。
% \item[\xoption{view}:] 浏览规范(view specification),示例如下:\\
%   |view={FitB}|, |view={FitH 842}|, |view={XYZ 0 100 null}|\ \  一些浏览规范参数(view specification parameters)
%   将数字(numbers)视为具有单位 bp 的参数。它们可以作为普通数字(plain numbers)或在 \cs{calc}\ 内部以
%   长度表达式(length expressions)给出。如果加载了 \xpackage{calc}\ 宏包,则支持该宏包的表达式(expressions)。否则,
%   使用 \hologo{eTeX}\ 的 \cs{dimexpr}。例如:\\
%   |view={FitH \calc{\paperheight-\topmargin-1in}}|\\
%   |view={XYZ 0 \calc{\paperheight} null}|\\
%   注意 \cs{calc}\ 不能用于 |XYZ| 的第三个参数,因为该参数是缩放值(zoom value),而不是长度(length)。

% \item[\xoption{named}:] 已命名的动作(Named action)的名称:\\
%   |FirstPage|(第一页),|LastPage|(最后一页),|NextPage|(下一页),|PrevPage|(前一页)
% \item[\xoption{gotor}:] 外部(external) PDF 文件的名称。
% \item[\xoption{uri}:] URI 规范(URI specification)。
% \item[\xoption{rawaction}:] 原始动作规范(raw action specification)。由于这些规范取决于驱动程序(driver),因此不应使用此选项。
% \end{description}
% 通过分析指定的选项来选择书签的适当动作。动作由不同的选项集(sets of options)区分:
% \begin{quote}
 \begin{tabular}{|@{}r|l@{}|}
%   \hline
%   \ \textbf{动作(Action)}\  & \ \textbf{选项(Options)}\ \\ \hline
%   \ \textsf{GoTo}\  &\  \xoption{dest}\ \\ \hline
%   \ \textsf{GoTo}\  & \ \xoption{page} + \xoption{view}\ \\ \hline
%   \ \textsf{GoToR}\  & \ \xoption{gotor} + \xoption{dest}\ \\ \hline
%   \ \textsf{GoToR}\  & \ \xoption{gotor} + \xoption{page} + \xoption{view}\ \ \ \\ \hline
%   \ \textsf{Named}\  &\  \xoption{named}\ \\ \hline
%   \ \textsf{URI}\  & \ \xoption{uri}\ \\ \hline
% \end{tabular}
% \end{quote}
%
% \paragraph{缺少动作(Missing actions)。}
% 如果动作缺少 \xpackage{bookmark}\ 宏包,则抛出错误消息(error message)。根据驱动程序(driver)
% (\xoption{pdftex}、\xoption{dvips}\ 和好友[friends]),宏包在文档末尾很晚才检测到它。
% 自 2011/04/21 v1.21 版本以后,该宏包尝试打印 \cs{bookmark}\ 的相应出现的行号(line number)和文件名(file name)。
% 然而,\hologo{TeX}\ 确实提供了行号,但不幸的是,文件名是一个秘密(secret)。但该宏包有如下获取文件名的方法:
% \begin{itemize}
% \item 如果 \hologo{LuaTeX} (独立于 DVI 或 PDF 模式)正在运行,则自动使用其 |status.filename|。
% \item 宏包的 \cs{currfile} \cite{currfile}\ 重新定义了 \hologo{LaTeX}\ 的内部结构,以跟踪文件名(file name)。
% 如果加载了该宏包,那么它的 \cs{currfilepath}\ 将被检测到并由 \xpackage{bookmark}\ 自动使用。
% \item 可以通过 \cs{bookmarksetup}\ 或 \cs{bookmark}\ 中的 \xoption{scrfile}\ 选项手动设置(set manually)文件名。
% 但是要小心,手动设置会禁用以前的文件名检测方法。错误的(wrong)或丢失的(missed)文件名设置(file name setting)可能会在错误消息中
% 为您提供错误的源位置(source location)。
% \end{itemize}
%
% \subsubsection{级别选项(Level options)}
%
% 书签条目(bookmark entries)的顺序由 \cs{bookmark}\ 命令的的出现顺序(appearance order)定义。
% 树结构(tree structure)由书签节点(bookmark nodes)的属性 \xoption{level}(级别)构建。
% \xoption{level}\ 的值是整数(integers)。如果书签条目级别的值高于前一个节点,则该条目将成为
% 前一个节点的子(child)节点。差值的绝对值并不重要。
%
% \xpackage{bookmark}\ 宏包能记住全局属性(global property)“current level(当前级别)”中上
% 一个书签条目(previous bookmark entry)的级别。
%
% 级别系统的(level system)行为(behaviour)可以通过以下选项进行配置:
% \begin{description}
% \item[\xoption{level}:]
%    设置级别(level),请参阅上面的说明。如果给出的选项 \xoption{level}\ 没有值,那么将恢复默
%    认行为,即将“当前级别(current level)”用作级别值(level value)。自 2010/10/19 v1.16 版本以来,
%    如果宏 \cs{toclevel@part}、\cs{toclevel@section}\ 被定义过(通过 \xpackage{hyperref}\ 宏包完成,
%    请参阅它的 \xoption{bookmarkdepth}\ 选项),则 \xpackage{bookmark}\ 宏包还支持 |part|、|section| 等名称。
%
% \item[\xoption{rellevel}:]
%    设置相对于前一级别的(previous level)级别。正值表示书签条目成为前一个书签条目的子条目。
% \item[\xoption{keeplevel}:]
%    使用由\xoption{level}\ 或 \xoption{rellevel}\ 设置的级别,但不要更改全局属性“current level(当前级别)”。
%    可以通过设置为 |false| 来禁用该选项。
% \item[\xoption{startatroot}:]
%    此时,书签树(bookmark tree)再次从顶层(top level)开始。下一个书签条目不会作为上一个条目的子条目进行排序。
%    示例场景:文档使用 part。但是,最后几章(last chapters)不应放在最后一部分(last part)下面:
%    \begin{quote}
%\begin{verbatim}
%\documentclass{book}
%[...]
%\begin{document}
%  \part{第一部分}
%    \chapter{第一部分的第1章}
%    [...]
%  \part{第二部分(Second part)}
%    \chapter{第二部分的第1章}
%    [...]
%  \bookmarksetup{startatroot}
%  \chapter{Index}% 不属于第二部分
%\end{document}
%\end{verbatim}
%    \end{quote}
% \end{description}
%
% \subsubsection{样式定义(Style definitions)}
%
% 样式(style)是一组选项设置(option settings)。它可以由宏 \cs{bookmarkdefinestyle}\ 定义,
% 并由它的 \xoption{style}\ 选项使用。
% \begin{declcs}{bookmarkdefinestyle} \M{name} \M{key value list}
% \end{declcs}
% 选项设置(option settings)的 \meta{key value list}(键值列表)被指定为样式名(style \meta{name})。
%
% \begin{description}
% \item[\xoption{style}:]
%   \xoption{style}\ 选项的值是以前定义的样式的名称(name)。现在执行其选项设置(option settings)。
%   选项可以包括 \xoption{style}\ 选项。通过递归调用相同样式的无限递归(endless recursion)被阻止并抛出一个错误。
% \end{description}
%
% \subsubsection{钩子支持(Hook support)}
%
% 处理宏\cs{bookmark}\ 的可选选项(optional options)后,就会调用钩子(hook)。
% \begin{description}
% \item[\xoption{addtohook}:]
%   代码(code)作为该选项的值添加到钩子中。
% \end{description}
%
% \begin{declcs}{bookmarkget} \M{option}
% \end{declcs}
% \cs{bookmarkget}\ 宏提取 \meta{option}\ 选项的最新选项设置(latest option setting)的值。
% 对于布尔选项(boolean option),如果启用布尔选项,则返回 1,否则结果为零。结果数字(resulting numbers)
% 可以直接用于 \cs{ifnum}\ 或 \cs{ifcase}。如果您想要数字 \texttt{0}\ 和 \texttt{1},
% 请在 \cs{bookmarkget}\ 前面加上 \cs{number}\ 作为前缀。\cs{bookmarkget}\ 宏是可展开的(expandable)。
% 如果选项不受支持,则返回空字符串(empty string)。受支持的布尔选项有:
% \begin{quote}
%   \xoption{bold}、
%   \xoption{italic}、
%   \xoption{open}
% \end{quote}
% 其他受支持的选项有:
% \begin{quote}
%   \xoption{depth}、
%   \xoption{dest}、
%   \xoption{color}、
%   \xoption{gotor}、
%   \xoption{level}、
%   \xoption{named}、
%   \xoption{openlevel}、
%   \xoption{page}、
%   \xoption{rawaction}、
%   \xoption{uri}、
%   \xoption{view}、
% \end{quote}
% 另外,以下键(key)是可用的:
% \begin{quote}
%   \xoption{text}
% \end{quote}
% 它返回大纲条目(outline entry)的文本(text)。
%
% \paragraph{选项设置(Option setting)。}
% 在钩子(hook)内部可以使用 \cs{bookmarksetup}\ 设置选项。
%
% \subsection{与 \xpackage{hyperref}\ 的兼容性}
%
% \xpackage{bookmark}\ 宏包自动禁用 \xpackage{hyperref}\ 宏包的书签(bookmarks)。但是,
% \xpackage{bookmark}\ 宏包使用了 \xpackage{hyperref}\ 宏包的一些代码。例如,
% \xpackage{bookmark}\ 宏包重新定义了 \xpackage{hyperref}\ 宏包在 \cs{addcontentsline}\ 命令
% 和其他命令中插入的\cs{Hy@writebookmark}\ 钩子。因此,不应禁用 \xpackage{hyperref}\ 宏包的书签。
%
% \xpackage{bookmark}\ 宏包使用 \xpackage{hyperref}\ 宏包的 \cs{pdfstringdef},且不提供替换(replacement)。
%
% \xpackage{hyperref}\ 宏包的一些选项也能在 \xpackage{bookmark}\ 宏包中实现(implemented):
% \begin{quote}
% \begin{tabular}{|l@{}|l@{}|}
%   \hline
%   \xpackage{hyperref}\ 宏包的选项\  &\ \xpackage{bookmark}\ 宏包的选项\ \ \\ \hline
%   \xoption{bookmarksdepth} &\ \xoption{depth}\\ \hline
%   \xoption{bookmarksopen} & \ \xoption{open}\\ \hline
%   \xoption{bookmarksopenlevel}\ \ \  &\ \xoption{openlevel}\\ \hline
%   \xoption{bookmarksnumbered} \ \ \ &\ \xoption{numbered}\\ \hline
% \end{tabular}
% \end{quote}
%
% 还可以使用以下命令:
% \begin{quote}
%   \cs{pdfbookmark}\\
%   \cs{currentpdfbookmark}\\
%   \cs{subpdfbookmark}\\
%   \cs{belowpdfbookmark}
% \end{quote}
%
% \subsection{在末尾添加书签}
%
% 宏包选项 \xoption{atend}\ 启用以下宏(macro):
% \begin{declcs}{BookmarkAtEnd}
%   \M{stuff}
% \end{declcs}
% \cs{BookmarkAtEnd}\ 宏将 \meta{stuff}\ 放在文档末尾。\meta{stuff}\ 表示书签命令(bookmark commands)。举例:
% \begin{quote}
%\begin{verbatim}
%\usepackage[atend]{bookmark}
%\BookmarkAtEnd{%
%  \bookmarksetup{startatroot}%
%  \bookmark[named=LastPage, level=0]{Last page}%
%}
%\end{verbatim}
% \end{quote}
%
% 或者,可以在 \cs{bookmark}\ 中给出 \xoption{startatroot}\ 选项:
% \begin{quote}
%\begin{verbatim}
%\BookmarkAtEnd{%
%  \bookmark[
%    startatroot,
%    named=LastPage,
%    level=0,
%  ]{Last page}%
%}
%\end{verbatim}
% \end{quote}
%
% \paragraph{备注(Remarks):}
% \begin{itemize}
% \item
%   \cs{BookmarkAtEnd} 隐藏了这样一个事实,即在文档末尾添加书签的方法取决于驱动程序(driver)。
%
%   为此,驱动程序 \xoption{pdftex}\ 使用 \xpackage{atveryend}\ 宏包。如果 \cs{AtEndDocument}\ 太早,
%   最后一个页面(last page)可能不会被发送出去(shipped out)。由于需要 \xext{aux}\ 文件,此驱动程序使
%   用 \cs{AfterLastShipout}。
%
%   其他驱动程序(\xoption{dvipdfm}、\xoption{xetex}、\xoption{vtex})的实现(implementation)
%   取决于 \cs{special},\cs{special}\ 在最后一页之后没有效果。在这种情况下,\xpackage{atenddvi}\ 宏包的
%   \cs{AtEndDvi}\ 有帮助。它将其参数(argument)放在文档的最后一页(last page)。至少需要运行 \hologo{LaTeX}\ 两次,
%   因为最后一页是由引用(reference)检测到的。
%
%   \xoption{dvips}\ 现在使用新的 LaTeX 钩子 \texttt{shipout/lastpage}。
% \item
%   未指定 \cs{BookmarkAtEnd}\ 参数的扩展时间(time of expansion)。这可以立即发生,也可以在文档末尾发生。
% \end{itemize}
%
% \subsection{限制/行动计划}
%
% \begin{itemize}
% \item 支持缺失动作(missing actions)(启动,\dots)。
% \item 对 \xpackage{hyperref}\ 的 \xoption{bookmarkstype}\ 选项进行了更好的设计(design)。
% \end{itemize}
%
% \section{示例(Example)}
%
%    \begin{macrocode}
%<*example>
%    \end{macrocode}
%    \begin{macrocode}
\documentclass{article}
\usepackage{xcolor}[2007/01/21]
\usepackage{hyperref}
\usepackage[
  open,
  openlevel=2,
  atend
]{bookmark}[2019/12/03]

\bookmarksetup{color=blue}

\BookmarkAtEnd{%
  \bookmarksetup{startatroot}%
  \bookmark[named=LastPage, level=0]{End/Last page}%
  \bookmark[named=FirstPage, level=1]{First page}%
}

\begin{document}
\section{First section}
\subsection{Subsection A}
\begin{figure}
  \hypertarget{fig}{}%
  A figure.
\end{figure}
\bookmark[
  rellevel=1,
  keeplevel,
  dest=fig
]{A figure}
\subsection{Subsection B}
\subsubsection{Subsubsection C}
\subsection{Umlauts: \"A\"O\"U\"a\"o\"u\ss}
\newpage
\bookmarksetup{
  bold,
  color=[rgb]{1,0,0}
}
\section{Very important section}
\bookmarksetup{
  italic,
  bold=false,
  color=blue
}
\subsection{Italic section}
\bookmarksetup{
  italic=false
}
\part{Misc}
\section{Diverse}
\subsubsection{Subsubsection, omitting subsection}
\bookmarksetup{
  startatroot
}
\section{Last section outside part}
\subsection{Subsection}
\bookmarksetup{
  color={}
}
\begingroup
  \bookmarksetup{level=0, color=green!80!black}
  \bookmark[named=FirstPage]{First page}
  \bookmark[named=LastPage]{Last page}
  \bookmark[named=PrevPage]{Previous page}
  \bookmark[named=NextPage]{Next page}
\endgroup
\bookmark[
  page=2,
  view=FitH 800
]{Page 2, FitH 800}
\bookmark[
  page=2,
  view=FitBH \calc{\paperheight-\topmargin-1in-\headheight-\headsep}
]{Page 2, FitBH top of text body}
\bookmark[
  uri={http://www.dante.de/},
  color=magenta
]{Dante homepage}
\bookmark[
  gotor={t.pdf},
  page=1,
  view={XYZ 0 1000 null},
  color=cyan!75!black
]{File t.pdf}
\bookmark[named=FirstPage]{First page}
\bookmark[rellevel=1, named=LastPage]{Last page (rellevel=1)}
\bookmark[named=PrevPage]{Previous page}
\bookmark[level=0, named=FirstPage]{First page (level=0)}
\bookmark[
  rellevel=1,
  keeplevel,
  named=LastPage
]{Last page (rellevel=1, keeplevel)}
\bookmark[named=PrevPage]{Previous page}
\end{document}
%    \end{macrocode}
%    \begin{macrocode}
%</example>
%    \end{macrocode}
%
% \StopEventually{
% }
%
% \section{实现(Implementation)}
%
% \subsection{宏包(Package)}
%
%    \begin{macrocode}
%<*package>
\NeedsTeXFormat{LaTeX2e}
\ProvidesPackage{bookmark}%
  [2020-11-06 v1.29 PDF bookmarks (HO)]%
%    \end{macrocode}
%
% \subsubsection{要求(Requirements)}
%
% \paragraph{\hologo{eTeX}.}
%
%    \begin{macro}{\BKM@CalcExpr}
%    \begin{macrocode}
\begingroup\expandafter\expandafter\expandafter\endgroup
\expandafter\ifx\csname numexpr\endcsname\relax
  \def\BKM@CalcExpr#1#2#3#4{%
    \begingroup
      \count@=#2\relax
      \advance\count@ by#3#4\relax
      \edef\x{\endgroup
        \def\noexpand#1{\the\count@}%
      }%
    \x
  }%
\else
  \def\BKM@CalcExpr#1#2#3#4{%
    \edef#1{%
      \the\numexpr#2#3#4\relax
    }%
  }%
\fi
%    \end{macrocode}
%    \end{macro}
%
% \paragraph{\hologo{pdfTeX}\ 的转义功能(escape features)}
%
%    \begin{macro}{\BKM@EscapeName}
%    \begin{macrocode}
\def\BKM@EscapeName#1{%
  \ifx#1\@empty
  \else
    \EdefEscapeName#1#1%
  \fi
}%
%    \end{macrocode}
%    \end{macro}
%    \begin{macro}{\BKM@EscapeString}
%    \begin{macrocode}
\def\BKM@EscapeString#1{%
  \ifx#1\@empty
  \else
    \EdefEscapeString#1#1%
  \fi
}%
%    \end{macrocode}
%    \end{macro}
%    \begin{macro}{\BKM@EscapeHex}
%    \begin{macrocode}
\def\BKM@EscapeHex#1{%
  \ifx#1\@empty
  \else
    \EdefEscapeHex#1#1%
  \fi
}%
%    \end{macrocode}
%    \end{macro}
%    \begin{macro}{\BKM@UnescapeHex}
%    \begin{macrocode}
\def\BKM@UnescapeHex#1{%
  \EdefUnescapeHex#1#1%
}%
%    \end{macrocode}
%    \end{macro}
%
% \paragraph{宏包(Packages)。}
%
% 不要加载由 \xpackage{hyperref}\ 加载的宏包
%    \begin{macrocode}
\RequirePackage{hyperref}[2010/06/18]
%    \end{macrocode}
%
% \subsubsection{宏包选项(Package options)}
%
%    \begin{macrocode}
\SetupKeyvalOptions{family=BKM,prefix=BKM@}
\DeclareLocalOptions{%
  atend,%
  bold,%
  color,%
  depth,%
  dest,%
  draft,%
  final,%
  gotor,%
  italic,%
  keeplevel,%
  level,%
  named,%
  numbered,%
  open,%
  openlevel,%
  page,%
  rawaction,%
  rellevel,%
  srcfile,%
  srcline,%
  startatroot,%
  uri,%
  view,%
}
%    \end{macrocode}
%    \begin{macro}{\bookmarksetup}
%    \begin{macrocode}
\newcommand*{\bookmarksetup}{\kvsetkeys{BKM}}
%    \end{macrocode}
%    \end{macro}
%    \begin{macro}{\BKM@setup}
%    \begin{macrocode}
\def\BKM@setup#1{%
  \bookmarksetup{#1}%
  \ifx\BKM@HookNext\ltx@empty
  \else
    \expandafter\bookmarksetup\expandafter{\BKM@HookNext}%
    \BKM@HookNextClear
  \fi
  \BKM@hook
  \ifBKM@keeplevel
  \else
    \xdef\BKM@currentlevel{\BKM@level}%
  \fi
}
%    \end{macrocode}
%    \end{macro}
%
%    \begin{macro}{\bookmarksetupnext}
%    \begin{macrocode}
\newcommand*{\bookmarksetupnext}[1]{%
  \ltx@GlobalAppendToMacro\BKM@HookNext{,#1}%
}
%    \end{macrocode}
%    \end{macro}
%    \begin{macro}{\BKM@setupnext}
%    \begin{macrocode}
%    \end{macrocode}
%    \end{macro}
%    \begin{macro}{\BKM@HookNextClear}
%    \begin{macrocode}
\def\BKM@HookNextClear{%
  \global\let\BKM@HookNext\ltx@empty
}
%    \end{macrocode}
%    \end{macro}
%    \begin{macro}{\BKM@HookNext}
%    \begin{macrocode}
\BKM@HookNextClear
%    \end{macrocode}
%    \end{macro}
%
%    \begin{macrocode}
\DeclareBoolOption{draft}
\DeclareComplementaryOption{final}{draft}
%    \end{macrocode}
%    \begin{macro}{\BKM@DisableOptions}
%    \begin{macrocode}
\def\BKM@DisableOptions{%
  \DisableKeyvalOption[action=warning,package=bookmark]%
      {BKM}{draft}%
  \DisableKeyvalOption[action=warning,package=bookmark]%
      {BKM}{final}%
}
%    \end{macrocode}
%    \end{macro}
%    \begin{macrocode}
\DeclareBoolOption[\ifHy@bookmarksopen true\else false\fi]{open}
%    \end{macrocode}
%    \begin{macro}{\bookmark@open}
%    \begin{macrocode}
\def\bookmark@open{%
  \ifBKM@open\ltx@one\else\ltx@zero\fi
}
%    \end{macrocode}
%    \end{macro}
%    \begin{macrocode}
\DeclareStringOption[\maxdimen]{openlevel}
%    \end{macrocode}
%    \begin{macro}{\BKM@openlevel}
%    \begin{macrocode}
\edef\BKM@openlevel{\number\@bookmarksopenlevel}
%    \end{macrocode}
%    \end{macro}
%    \begin{macrocode}
%\DeclareStringOption[\c@tocdepth]{depth}
\ltx@IfUndefined{Hy@bookmarksdepth}{%
  \def\BKM@depth{\c@tocdepth}%
}{%
  \let\BKM@depth\Hy@bookmarksdepth
}
\define@key{BKM}{depth}[]{%
  \edef\BKM@param{#1}%
  \ifx\BKM@param\@empty
    \def\BKM@depth{\c@tocdepth}%
  \else
    \ltx@IfUndefined{toclevel@\BKM@param}{%
      \@onelevel@sanitize\BKM@param
      \edef\BKM@temp{\expandafter\@car\BKM@param\@nil}%
      \ifcase 0\expandafter\ifx\BKM@temp-1\fi
              \expandafter\ifnum\expandafter`\BKM@temp>47 %
                \expandafter\ifnum\expandafter`\BKM@temp<58 %
                  1%
                \fi
              \fi
              \relax
        \PackageWarning{bookmark}{%
          Unknown document division name (\BKM@param)\MessageBreak
          for option `depth'%
        }%
      \else
        \BKM@SetDepthOrLevel\BKM@depth\BKM@param
      \fi
    }{%
      \BKM@SetDepthOrLevel\BKM@depth{%
        \csname toclevel@\BKM@param\endcsname
      }%
    }%
  \fi
}
%    \end{macrocode}
%    \begin{macro}{\bookmark@depth}
%    \begin{macrocode}
\def\bookmark@depth{\BKM@depth}
%    \end{macrocode}
%    \end{macro}
%    \begin{macro}{\BKM@SetDepthOrLevel}
%    \begin{macrocode}
\def\BKM@SetDepthOrLevel#1#2{%
  \begingroup
    \setbox\z@=\hbox{%
      \count@=#2\relax
      \expandafter
    }%
  \expandafter\endgroup
  \expandafter\def\expandafter#1\expandafter{\the\count@}%
}
%    \end{macrocode}
%    \end{macro}
%    \begin{macrocode}
\DeclareStringOption[\BKM@currentlevel]{level}[\BKM@currentlevel]
\define@key{BKM}{level}{%
  \edef\BKM@param{#1}%
  \ifx\BKM@param\BKM@MacroCurrentLevel
    \let\BKM@level\BKM@param
  \else
    \ltx@IfUndefined{toclevel@\BKM@param}{%
      \@onelevel@sanitize\BKM@param
      \edef\BKM@temp{\expandafter\@car\BKM@param\@nil}%
      \ifcase 0\expandafter\ifx\BKM@temp-1\fi
              \expandafter\ifnum\expandafter`\BKM@temp>47 %
                \expandafter\ifnum\expandafter`\BKM@temp<58 %
                  1%
                \fi
              \fi
              \relax
        \PackageWarning{bookmark}{%
          Unknown document division name (\BKM@param)\MessageBreak
          for option `level'%
        }%
      \else
        \BKM@SetDepthOrLevel\BKM@level\BKM@param
      \fi
    }{%
      \BKM@SetDepthOrLevel\BKM@level{%
        \csname toclevel@\BKM@param\endcsname
      }%
    }%
  \fi
}
%    \end{macrocode}
%    \begin{macro}{\BKM@MacroCurrentLevel}
%    \begin{macrocode}
\def\BKM@MacroCurrentLevel{\BKM@currentlevel}
%    \end{macrocode}
%    \end{macro}
%    \begin{macrocode}
\DeclareBoolOption{keeplevel}
\DeclareBoolOption{startatroot}
%    \end{macrocode}
%    \begin{macro}{\BKM@startatrootfalse}
%    \begin{macrocode}
\def\BKM@startatrootfalse{%
  \global\let\ifBKM@startatroot\iffalse
}
%    \end{macrocode}
%    \end{macro}
%    \begin{macro}{\BKM@startatroottrue}
%    \begin{macrocode}
\def\BKM@startatroottrue{%
  \global\let\ifBKM@startatroot\iftrue
}
%    \end{macrocode}
%    \end{macro}
%    \begin{macrocode}
\define@key{BKM}{rellevel}{%
  \BKM@CalcExpr\BKM@level{#1}+\BKM@currentlevel
}
%    \end{macrocode}
%    \begin{macro}{\bookmark@level}
%    \begin{macrocode}
\def\bookmark@level{\BKM@level}
%    \end{macrocode}
%    \end{macro}
%    \begin{macro}{\BKM@currentlevel}
%    \begin{macrocode}
\def\BKM@currentlevel{0}
%    \end{macrocode}
%    \end{macro}
%    Make \xpackage{bookmark}'s option \xoption{numbered} an alias
%    for \xpackage{hyperref}'s \xoption{bookmarksnumbered}.
%    \begin{macrocode}
\DeclareBoolOption[%
  \ifHy@bookmarksnumbered true\else false\fi
]{numbered}
\g@addto@macro\BKM@numberedtrue{%
  \let\ifHy@bookmarksnumbered\iftrue
}
\g@addto@macro\BKM@numberedfalse{%
  \let\ifHy@bookmarksnumbered\iffalse
}
\g@addto@macro\Hy@bookmarksnumberedtrue{%
  \let\ifBKM@numbered\iftrue
}
\g@addto@macro\Hy@bookmarksnumberedfalse{%
  \let\ifBKM@numbered\iffalse
}
%    \end{macrocode}
%    \begin{macro}{\bookmark@numbered}
%    \begin{macrocode}
\def\bookmark@numbered{%
  \ifBKM@numbered\ltx@one\else\ltx@zero\fi
}
%    \end{macrocode}
%    \end{macro}
%
% \paragraph{重定义 \xpackage{hyperref}\ 宏包的选项}
%
%    \begin{macro}{\BKM@PatchHyperrefOption}
%    \begin{macrocode}
\def\BKM@PatchHyperrefOption#1{%
  \expandafter\BKM@@PatchHyperrefOption\csname KV@Hyp@#1\endcsname%
}
%    \end{macrocode}
%    \end{macro}
%    \begin{macro}{\BKM@@PatchHyperrefOption}
%    \begin{macrocode}
\def\BKM@@PatchHyperrefOption#1{%
  \expandafter\BKM@@@PatchHyperrefOption#1{##1}\BKM@nil#1%
}
%    \end{macrocode}
%    \end{macro}
%    \begin{macro}{\BKM@@@PatchHyperrefOption}
%    \begin{macrocode}
\def\BKM@@@PatchHyperrefOption#1\BKM@nil#2#3{%
  \def#2##1{%
    #1%
    \bookmarksetup{#3={##1}}%
  }%
}
%    \end{macrocode}
%    \end{macro}
%    \begin{macrocode}
\BKM@PatchHyperrefOption{bookmarksopen}{open}
\BKM@PatchHyperrefOption{bookmarksopenlevel}{openlevel}
\BKM@PatchHyperrefOption{bookmarksdepth}{depth}
%    \end{macrocode}
%
% \paragraph{字体样式(font style)选项。}
%
%    注意:\xpackage{bitset}\ 宏是基于零的,PDF 规范(PDF specifications)以1开头。
%    \begin{macrocode}
\bitsetReset{BKM@FontStyle}%
\define@key{BKM}{italic}[true]{%
  \expandafter\ifx\csname if#1\endcsname\iftrue
    \bitsetSet{BKM@FontStyle}{0}%
  \else
    \bitsetClear{BKM@FontStyle}{0}%
  \fi
}%
\define@key{BKM}{bold}[true]{%
  \expandafter\ifx\csname if#1\endcsname\iftrue
    \bitsetSet{BKM@FontStyle}{1}%
  \else
    \bitsetClear{BKM@FontStyle}{1}%
  \fi
}%
%    \end{macrocode}
%    \begin{macro}{\bookmark@italic}
%    \begin{macrocode}
\def\bookmark@italic{%
  \ifnum\bitsetGet{BKM@FontStyle}{0}=1 \ltx@one\else\ltx@zero\fi
}
%    \end{macrocode}
%    \end{macro}
%    \begin{macro}{\bookmark@bold}
%    \begin{macrocode}
\def\bookmark@bold{%
  \ifnum\bitsetGet{BKM@FontStyle}{1}=1 \ltx@one\else\ltx@zero\fi
}
%    \end{macrocode}
%    \end{macro}
%    \begin{macro}{\BKM@PrintStyle}
%    \begin{macrocode}
\def\BKM@PrintStyle{%
  \bitsetGetDec{BKM@FontStyle}%
}%
%    \end{macrocode}
%    \end{macro}
%
% \paragraph{颜色(color)选项。}
%
%    \begin{macrocode}
\define@key{BKM}{color}{%
  \HyColor@BookmarkColor{#1}\BKM@color{bookmark}{color}%
}
%    \end{macrocode}
%    \begin{macro}{\BKM@color}
%    \begin{macrocode}
\let\BKM@color\@empty
%    \end{macrocode}
%    \end{macro}
%    \begin{macro}{\bookmark@color}
%    \begin{macrocode}
\def\bookmark@color{\BKM@color}
%    \end{macrocode}
%    \end{macro}
%
% \subsubsection{动作(action)选项}
%
%    \begin{macrocode}
\def\BKM@temp#1{%
  \DeclareStringOption{#1}%
  \expandafter\edef\csname bookmark@#1\endcsname{%
    \expandafter\noexpand\csname BKM@#1\endcsname
  }%
}
%    \end{macrocode}
%    \begin{macro}{\bookmark@dest}
%    \begin{macrocode}
\BKM@temp{dest}
%    \end{macrocode}
%    \end{macro}
%    \begin{macro}{\bookmark@named}
%    \begin{macrocode}
\BKM@temp{named}
%    \end{macrocode}
%    \end{macro}
%    \begin{macro}{\bookmark@uri}
%    \begin{macrocode}
\BKM@temp{uri}
%    \end{macrocode}
%    \end{macro}
%    \begin{macro}{\bookmark@gotor}
%    \begin{macrocode}
\BKM@temp{gotor}
%    \end{macrocode}
%    \end{macro}
%    \begin{macro}{\bookmark@rawaction}
%    \begin{macrocode}
\BKM@temp{rawaction}
%    \end{macrocode}
%    \end{macro}
%
%    \begin{macrocode}
\define@key{BKM}{page}{%
  \def\BKM@page{#1}%
  \ifx\BKM@page\@empty
  \else
    \edef\BKM@page{\number\BKM@page}%
    \ifnum\BKM@page>\z@
    \else
      \PackageError{bookmark}{Page must be positive}\@ehc
      \def\BKM@page{1}%
    \fi
  \fi
}
%    \end{macrocode}
%    \begin{macro}{\BKM@page}
%    \begin{macrocode}
\let\BKM@page\@empty
%    \end{macrocode}
%    \end{macro}
%    \begin{macro}{\bookmark@page}
%    \begin{macrocode}
\def\bookmark@page{\BKM@@page}
%    \end{macrocode}
%    \end{macro}
%
%    \begin{macrocode}
\define@key{BKM}{view}{%
  \BKM@CheckView{#1}%
}
%    \end{macrocode}
%    \begin{macro}{\BKM@view}
%    \begin{macrocode}
\let\BKM@view\@empty
%    \end{macrocode}
%    \end{macro}
%    \begin{macro}{\bookmark@view}
%    \begin{macrocode}
\def\bookmark@view{\BKM@view}
%    \end{macrocode}
%    \end{macro}
%    \begin{macro}{BKM@CheckView}
%    \begin{macrocode}
\def\BKM@CheckView#1{%
  \BKM@CheckViewType#1 \@nil
}
%    \end{macrocode}
%    \end{macro}
%    \begin{macro}{\BKM@CheckViewType}
%    \begin{macrocode}
\def\BKM@CheckViewType#1 #2\@nil{%
  \def\BKM@type{#1}%
  \@onelevel@sanitize\BKM@type
  \BKM@TestViewType{Fit}{}%
  \BKM@TestViewType{FitB}{}%
  \BKM@TestViewType{FitH}{%
    \BKM@CheckParam#2 \@nil{top}%
  }%
  \BKM@TestViewType{FitBH}{%
    \BKM@CheckParam#2 \@nil{top}%
  }%
  \BKM@TestViewType{FitV}{%
    \BKM@CheckParam#2 \@nil{bottom}%
  }%
  \BKM@TestViewType{FitBV}{%
    \BKM@CheckParam#2 \@nil{bottom}%
  }%
  \BKM@TestViewType{FitR}{%
    \BKM@CheckRect{#2}{ }%
  }%
  \BKM@TestViewType{XYZ}{%
    \BKM@CheckXYZ{#2}{ }%
  }%
  \@car{%
    \PackageError{bookmark}{%
      Unknown view type `\BKM@type',\MessageBreak
      using `FitH' instead%
    }\@ehc
    \def\BKM@view{FitH}%
  }%
  \@nil
}
%    \end{macrocode}
%    \end{macro}
%    \begin{macro}{\BKM@TestViewType}
%    \begin{macrocode}
\def\BKM@TestViewType#1{%
  \def\BKM@temp{#1}%
  \@onelevel@sanitize\BKM@temp
  \ifx\BKM@type\BKM@temp
    \let\BKM@view\BKM@temp
    \expandafter\@car
  \else
    \expandafter\@gobble
  \fi
}
%    \end{macrocode}
%    \end{macro}
%    \begin{macro}{BKM@CheckParam}
%    \begin{macrocode}
\def\BKM@CheckParam#1 #2\@nil#3{%
  \def\BKM@param{#1}%
  \ifx\BKM@param\@empty
    \PackageWarning{bookmark}{%
      Missing parameter (#3) for `\BKM@type',\MessageBreak
      using 0%
    }%
    \def\BKM@param{0}%
  \else
    \BKM@CalcParam
  \fi
  \edef\BKM@view{\BKM@view\space\BKM@param}%
}
%    \end{macrocode}
%    \end{macro}
%    \begin{macro}{BKM@CheckRect}
%    \begin{macrocode}
\def\BKM@CheckRect#1#2{%
  \BKM@@CheckRect#1#2#2#2#2\@nil
}
%    \end{macrocode}
%    \end{macro}
%    \begin{macro}{\BKM@@CheckRect}
%    \begin{macrocode}
\def\BKM@@CheckRect#1 #2 #3 #4 #5\@nil{%
  \def\BKM@temp{0}%
  \def\BKM@param{#1}%
  \ifx\BKM@param\@empty
    \def\BKM@param{0}%
    \def\BKM@temp{1}%
  \else
    \BKM@CalcParam
  \fi
  \edef\BKM@view{\BKM@view\space\BKM@param}%
  \def\BKM@param{#2}%
  \ifx\BKM@param\@empty
    \def\BKM@param{0}%
    \def\BKM@temp{1}%
  \else
    \BKM@CalcParam
  \fi
  \edef\BKM@view{\BKM@view\space\BKM@param}%
  \def\BKM@param{#3}%
  \ifx\BKM@param\@empty
    \def\BKM@param{0}%
    \def\BKM@temp{1}%
  \else
    \BKM@CalcParam
  \fi
  \edef\BKM@view{\BKM@view\space\BKM@param}%
  \def\BKM@param{#4}%
  \ifx\BKM@param\@empty
    \def\BKM@param{0}%
    \def\BKM@temp{1}%
  \else
    \BKM@CalcParam
  \fi
  \edef\BKM@view{\BKM@view\space\BKM@param}%
  \ifnum\BKM@temp>\z@
    \PackageWarning{bookmark}{Missing parameters for `\BKM@type'}%
  \fi
}
%    \end{macrocode}
%    \end{macro}
%    \begin{macro}{\BKM@CheckXYZ}
%    \begin{macrocode}
\def\BKM@CheckXYZ#1#2{%
  \BKM@@CheckXYZ#1#2#2#2\@nil
}
%    \end{macrocode}
%    \end{macro}
%    \begin{macro}{\BKM@@CheckXYZ}
%    \begin{macrocode}
\def\BKM@@CheckXYZ#1 #2 #3 #4\@nil{%
  \def\BKM@param{#1}%
  \let\BKM@temp\BKM@param
  \@onelevel@sanitize\BKM@temp
  \ifx\BKM@param\@empty
    \let\BKM@param\BKM@null
  \else
    \ifx\BKM@temp\BKM@null
    \else
      \BKM@CalcParam
    \fi
  \fi
  \edef\BKM@view{\BKM@view\space\BKM@param}%
  \def\BKM@param{#2}%
  \let\BKM@temp\BKM@param
  \@onelevel@sanitize\BKM@temp
  \ifx\BKM@param\@empty
    \let\BKM@param\BKM@null
  \else
    \ifx\BKM@temp\BKM@null
    \else
      \BKM@CalcParam
    \fi
  \fi
  \edef\BKM@view{\BKM@view\space\BKM@param}%
  \def\BKM@param{#3}%
  \ifx\BKM@param\@empty
    \let\BKM@param\BKM@null
  \fi
  \edef\BKM@view{\BKM@view\space\BKM@param}%
}
%    \end{macrocode}
%    \end{macro}
%    \begin{macro}{\BKM@null}
%    \begin{macrocode}
\def\BKM@null{null}
\@onelevel@sanitize\BKM@null
%    \end{macrocode}
%    \end{macro}
%
%    \begin{macro}{\BKM@CalcParam}
%    \begin{macrocode}
\def\BKM@CalcParam{%
  \begingroup
  \let\calc\@firstofone
  \expandafter\BKM@@CalcParam\BKM@param\@empty\@empty\@nil
}
%    \end{macrocode}
%    \end{macro}
%    \begin{macro}{\BKM@@CalcParam}
%    \begin{macrocode}
\def\BKM@@CalcParam#1#2#3\@nil{%
  \ifx\calc#1%
    \@ifundefined{calc@assign@dimen}{%
      \@ifundefined{dimexpr}{%
        \setlength{\dimen@}{#2}%
      }{%
        \setlength{\dimen@}{\dimexpr#2\relax}%
      }%
    }{%
      \setlength{\dimen@}{#2}%
    }%
    \dimen@.99626\dimen@
    \edef\BKM@param{\strip@pt\dimen@}%
    \expandafter\endgroup
    \expandafter\def\expandafter\BKM@param\expandafter{\BKM@param}%
  \else
    \endgroup
  \fi
}
%    \end{macrocode}
%    \end{macro}
%
% \subsubsection{\xoption{atend}\ 选项}
%
%    \begin{macrocode}
\DeclareBoolOption{atend}
\g@addto@macro\BKM@DisableOptions{%
  \DisableKeyvalOption[action=warning,package=bookmark]%
      {BKM}{atend}%
}
%    \end{macrocode}
%
% \subsubsection{\xoption{style}\ 选项}
%
%    \begin{macro}{\bookmarkdefinestyle}
%    \begin{macrocode}
\newcommand*{\bookmarkdefinestyle}[2]{%
  \@ifundefined{BKM@style@#1}{%
  }{%
    \PackageInfo{bookmark}{Redefining style `#1'}%
  }%
  \@namedef{BKM@style@#1}{#2}%
}
%    \end{macrocode}
%    \end{macro}
%    \begin{macrocode}
\define@key{BKM}{style}{%
  \BKM@StyleCall{#1}%
}
\newif\ifBKM@ok
%    \end{macrocode}
%    \begin{macro}{\BKM@StyleCall}
%    \begin{macrocode}
\def\BKM@StyleCall#1{%
  \@ifundefined{BKM@style@#1}{%
    \PackageWarning{bookmark}{%
      Ignoring unknown style `#1'%
    }%
  }{%
%    \end{macrocode}
%    检查样式堆栈(style stack)。
%    \begin{macrocode}
    \BKM@oktrue
    \edef\BKM@StyleCurrent{#1}%
    \@onelevel@sanitize\BKM@StyleCurrent
    \let\BKM@StyleEntry\BKM@StyleEntryCheck
    \BKM@StyleStack
    \ifBKM@ok
      \expandafter\@firstofone
    \else
      \PackageError{bookmark}{%
        Ignoring recursive call of style `\BKM@StyleCurrent'%
      }\@ehc
      \expandafter\@gobble
    \fi
    {%
%    \end{macrocode}
%    在堆栈上推送当前样式(Push current style on stack)。
%    \begin{macrocode}
      \let\BKM@StyleEntry\relax
      \edef\BKM@StyleStack{%
        \BKM@StyleEntry{\BKM@StyleCurrent}%
        \BKM@StyleStack
      }%
%    \end{macrocode}
%   调用样式(Call style)。
%    \begin{macrocode}
      \expandafter\expandafter\expandafter\bookmarksetup
      \expandafter\expandafter\expandafter{%
        \csname BKM@style@\BKM@StyleCurrent\endcsname
      }%
%    \end{macrocode}
%    从堆栈中弹出当前样式(Pop current style from stack)。
%    \begin{macrocode}
      \BKM@StyleStackPop
    }%
  }%
}
%    \end{macrocode}
%    \end{macro}
%    \begin{macro}{\BKM@StyleStackPop}
%    \begin{macrocode}
\def\BKM@StyleStackPop{%
  \let\BKM@StyleEntry\relax
  \edef\BKM@StyleStack{%
    \expandafter\@gobbletwo\BKM@StyleStack
  }%
}
%    \end{macrocode}
%    \end{macro}
%    \begin{macro}{\BKM@StyleEntryCheck}
%    \begin{macrocode}
\def\BKM@StyleEntryCheck#1{%
  \def\BKM@temp{#1}%
  \ifx\BKM@temp\BKM@StyleCurrent
    \BKM@okfalse
  \fi
}
%    \end{macrocode}
%    \end{macro}
%    \begin{macro}{\BKM@StyleStack}
%    \begin{macrocode}
\def\BKM@StyleStack{}
%    \end{macrocode}
%    \end{macro}
%
% \subsubsection{源文件位置(source file location)选项}
%
%    \begin{macrocode}
\DeclareStringOption{srcline}
\DeclareStringOption{srcfile}
%    \end{macrocode}
%
% \subsubsection{钩子支持(Hook support)}
%
%    \begin{macro}{\BKM@hook}
%    \begin{macrocode}
\def\BKM@hook{}
%    \end{macrocode}
%    \end{macro}
%    \begin{macrocode}
\define@key{BKM}{addtohook}{%
  \ltx@LocalAppendToMacro\BKM@hook{#1}%
}
%    \end{macrocode}
%
%    \begin{macro}{bookmarkget}
%    \begin{macrocode}
\newcommand*{\bookmarkget}[1]{%
  \romannumeral0%
  \ltx@ifundefined{bookmark@#1}{%
    \ltx@space
  }{%
    \expandafter\expandafter\expandafter\ltx@space
    \csname bookmark@#1\endcsname
  }%
}
%    \end{macrocode}
%    \end{macro}
%
% \subsubsection{设置和加载驱动程序}
%
% \paragraph{检测驱动程序。}
%
%    \begin{macro}{\BKM@DefineDriverKey}
%    \begin{macrocode}
\def\BKM@DefineDriverKey#1{%
  \define@key{BKM}{#1}[]{%
    \def\BKM@driver{#1}%
  }%
  \g@addto@macro\BKM@DisableOptions{%
    \DisableKeyvalOption[action=warning,package=bookmark]%
        {BKM}{#1}%
  }%
}
%    \end{macrocode}
%    \end{macro}
%    \begin{macrocode}
\BKM@DefineDriverKey{pdftex}
\BKM@DefineDriverKey{dvips}
\BKM@DefineDriverKey{dvipdfm}
\BKM@DefineDriverKey{dvipdfmx}
\BKM@DefineDriverKey{xetex}
\BKM@DefineDriverKey{vtex}
\define@key{BKM}{dvipdfmx-outline-open}[true]{%
 \PackageWarning{bookmark}{Option 'dvipdfmx-outline-open' is obsolete
   and ignored}{}}
%    \end{macrocode}
%    \begin{macro}{\bookmark@driver}
%    \begin{macrocode}
\def\bookmark@driver{\BKM@driver}
%    \end{macrocode}
%    \end{macro}
%    \begin{macrocode}
\InputIfFileExists{bookmark.cfg}{}{}
%    \end{macrocode}
%    \begin{macro}{\BookmarkDriverDefault}
%    \begin{macrocode}
\providecommand*{\BookmarkDriverDefault}{dvips}
%    \end{macrocode}
%    \end{macro}
%    \begin{macro}{\BKM@driver}
% Lua\TeX\ 和 pdf\TeX\ 共享驱动程序。
%    \begin{macrocode}
\ifpdf
  \def\BKM@driver{pdftex}%
  \ifx\pdfoutline\@undefined
    \ifx\pdfextension\@undefined\else
      \protected\def\pdfoutline{\pdfextension outline }
    \fi
  \fi
\else
  \ifxetex
    \def\BKM@driver{dvipdfm}%
  \else
    \ifvtex
      \def\BKM@driver{vtex}%
    \else
      \edef\BKM@driver{\BookmarkDriverDefault}%
    \fi
  \fi
\fi
%    \end{macrocode}
%    \end{macro}
%
% \paragraph{过程选项(Process options)。}
%
%    \begin{macrocode}
\ProcessKeyvalOptions*
\BKM@DisableOptions
%    \end{macrocode}
%
% \paragraph{\xoption{draft}\ 选项}
%
%    \begin{macrocode}
\ifBKM@draft
  \PackageWarningNoLine{bookmark}{Draft mode on}%
  \let\bookmarksetup\ltx@gobble
  \let\BookmarkAtEnd\ltx@gobble
  \let\bookmarkdefinestyle\ltx@gobbletwo
  \let\bookmarkget\ltx@gobble
  \let\pdfbookmark\ltx@undefined
  \newcommand*{\pdfbookmark}[3][]{}%
  \let\currentpdfbookmark\ltx@gobbletwo
  \let\subpdfbookmark\ltx@gobbletwo
  \let\belowpdfbookmark\ltx@gobbletwo
  \newcommand*{\bookmark}[2][]{}%
  \renewcommand*{\Hy@writebookmark}[5]{}%
  \let\ReadBookmarks\relax
  \let\BKM@DefGotoNameAction\ltx@gobbletwo % package `hypdestopt'
  \expandafter\endinput
\fi
%    \end{macrocode}
%
% \paragraph{验证和加载驱动程序。}
%
%    \begin{macrocode}
\def\BKM@temp{dvipdfmx}%
\ifx\BKM@temp\BKM@driver
  \def\BKM@driver{dvipdfm}%
\fi
\def\BKM@temp{pdftex}%
\ifpdf
  \ifx\BKM@temp\BKM@driver
  \else
    \PackageWarningNoLine{bookmark}{%
      Wrong driver `\BKM@driver', using `pdftex' instead%
    }%
    \let\BKM@driver\BKM@temp
  \fi
\else
  \ifx\BKM@temp\BKM@driver
    \PackageError{bookmark}{%
      Wrong driver, pdfTeX is not running in PDF mode.\MessageBreak
      Package loading is aborted%
    }\@ehc
    \expandafter\expandafter\expandafter\endinput
  \fi
  \def\BKM@temp{dvipdfm}%
  \ifxetex
    \ifx\BKM@temp\BKM@driver
    \else
      \PackageWarningNoLine{bookmark}{%
        Wrong driver `\BKM@driver',\MessageBreak
        using `dvipdfm' for XeTeX instead%
      }%
      \let\BKM@driver\BKM@temp
    \fi
  \else
    \def\BKM@temp{vtex}%
    \ifvtex
      \ifx\BKM@temp\BKM@driver
      \else
        \PackageWarningNoLine{bookmark}{%
          Wrong driver `\BKM@driver',\MessageBreak
          using `vtex' for VTeX instead%
        }%
        \let\BKM@driver\BKM@temp
      \fi
    \else
      \ifx\BKM@temp\BKM@driver
        \PackageError{bookmark}{%
          Wrong driver, VTeX is not running in PDF mode.\MessageBreak
          Package loading is aborted%
        }\@ehc
        \expandafter\expandafter\expandafter\endinput
      \fi
    \fi
  \fi
\fi
\providecommand\IfFormatAtLeastTF{\@ifl@t@r\fmtversion}
\IfFormatAtLeastTF{2020/10/01}{}{\edef\BKM@driver{\BKM@driver-2019-12-03}}
\InputIfFileExists{bkm-\BKM@driver.def}{}{%
  \PackageError{bookmark}{%
    Unsupported driver `\BKM@driver'.\MessageBreak
    Package loading is aborted%
  }\@ehc
  \endinput
}
%    \end{macrocode}
%
% \subsubsection{与 \xpackage{hyperref}\ 的兼容性}
%
%    \begin{macro}{\pdfbookmark}
%    \begin{macrocode}
\let\pdfbookmark\ltx@undefined
\newcommand*{\pdfbookmark}[3][0]{%
  \bookmark[level=#1,dest={#3.#1}]{#2}%
  \hyper@anchorstart{#3.#1}\hyper@anchorend
}
%    \end{macrocode}
%    \end{macro}
%    \begin{macro}{\currentpdfbookmark}
%    \begin{macrocode}
\def\currentpdfbookmark{%
  \pdfbookmark[\BKM@currentlevel]%
}
%    \end{macrocode}
%    \end{macro}
%    \begin{macro}{\subpdfbookmark}
%    \begin{macrocode}
\def\subpdfbookmark{%
  \BKM@CalcExpr\BKM@CalcResult\BKM@currentlevel+1%
  \expandafter\pdfbookmark\expandafter[\BKM@CalcResult]%
}
%    \end{macrocode}
%    \end{macro}
%    \begin{macro}{\belowpdfbookmark}
%    \begin{macrocode}
\def\belowpdfbookmark#1#2{%
  \xdef\BKM@gtemp{\number\BKM@currentlevel}%
  \subpdfbookmark{#1}{#2}%
  \global\let\BKM@currentlevel\BKM@gtemp
}
%    \end{macrocode}
%    \end{macro}
%
%    节号(section number)、文本(text)、标签(label)、级别(level)、文件(file)
%    \begin{macro}{\Hy@writebookmark}
%    \begin{macrocode}
\def\Hy@writebookmark#1#2#3#4#5{%
  \ifnum#4>\BKM@depth\relax
  \else
    \def\BKM@type{#5}%
    \ifx\BKM@type\Hy@bookmarkstype
      \begingroup
        \ifBKM@numbered
          \let\numberline\Hy@numberline
          \let\booknumberline\Hy@numberline
          \let\partnumberline\Hy@numberline
          \let\chapternumberline\Hy@numberline
        \else
          \let\numberline\@gobble
          \let\booknumberline\@gobble
          \let\partnumberline\@gobble
          \let\chapternumberline\@gobble
        \fi
        \bookmark[level=#4,dest={\HyperDestNameFilter{#3}}]{#2}%
      \endgroup
    \fi
  \fi
}
%    \end{macrocode}
%    \end{macro}
%
%    \begin{macro}{\ReadBookmarks}
%    \begin{macrocode}
\let\ReadBookmarks\relax
%    \end{macrocode}
%    \end{macro}
%
%    \begin{macrocode}
%</package>
%    \end{macrocode}
%
% \subsection{dvipdfm 的驱动程序}
%
%    \begin{macrocode}
%<*dvipdfm>
\NeedsTeXFormat{LaTeX2e}
\ProvidesFile{bkm-dvipdfm.def}%
  [2020-11-06 v1.29 bookmark driver for dvipdfm (HO)]%
%    \end{macrocode}
%
%    \begin{macro}{\BKM@id}
%    \begin{macrocode}
\newcount\BKM@id
\BKM@id=\z@
%    \end{macrocode}
%    \end{macro}
%
%    \begin{macro}{\BKM@0}
%    \begin{macrocode}
\@namedef{BKM@0}{000}
%    \end{macrocode}
%    \end{macro}
%    \begin{macro}{\ifBKM@sw}
%    \begin{macrocode}
\newif\ifBKM@sw
%    \end{macrocode}
%    \end{macro}
%
%    \begin{macro}{\bookmark}
%    \begin{macrocode}
\newcommand*{\bookmark}[2][]{%
  \if@filesw
    \begingroup
      \def\bookmark@text{#2}%
      \BKM@setup{#1}%
      \edef\BKM@prev{\the\BKM@id}%
      \global\advance\BKM@id\@ne
      \BKM@swtrue
      \@whilesw\ifBKM@sw\fi{%
        \def\BKM@abslevel{1}%
        \ifnum\ifBKM@startatroot\z@\else\BKM@prev\fi=\z@
          \BKM@startatrootfalse
          \expandafter\xdef\csname BKM@\the\BKM@id\endcsname{%
            0{\BKM@level}\BKM@abslevel
          }%
          \BKM@swfalse
        \else
          \expandafter\expandafter\expandafter\BKM@getx
              \csname BKM@\BKM@prev\endcsname
          \ifnum\BKM@level>\BKM@x@level\relax
            \BKM@CalcExpr\BKM@abslevel\BKM@x@abslevel+1%
            \expandafter\xdef\csname BKM@\the\BKM@id\endcsname{%
              {\BKM@prev}{\BKM@level}\BKM@abslevel
            }%
            \BKM@swfalse
          \else
            \let\BKM@prev\BKM@x@parent
          \fi
        \fi
      }%
      \csname HyPsd@XeTeXBigCharstrue\endcsname
      \pdfstringdef\BKM@title{\bookmark@text}%
      \edef\BKM@FLAGS{\BKM@PrintStyle}%
      \let\BKM@action\@empty
      \ifx\BKM@gotor\@empty
        \ifx\BKM@dest\@empty
          \ifx\BKM@named\@empty
            \ifx\BKM@rawaction\@empty
              \ifx\BKM@uri\@empty
                \ifx\BKM@page\@empty
                  \PackageError{bookmark}{Missing action}\@ehc
                  \edef\BKM@action{/Dest[@page1/Fit]}%
                \else
                  \ifx\BKM@view\@empty
                    \def\BKM@view{Fit}%
                  \fi
                  \edef\BKM@action{/Dest[@page\BKM@page/\BKM@view]}%
                \fi
              \else
                \BKM@EscapeString\BKM@uri
                \edef\BKM@action{%
                  /A<<%
                    /S/URI%
                    /URI(\BKM@uri)%
                  >>%
                }%
              \fi
            \else
              \edef\BKM@action{/A<<\BKM@rawaction>>}%
            \fi
          \else
            \BKM@EscapeName\BKM@named
            \edef\BKM@action{%
              /A<</S/Named/N/\BKM@named>>%
            }%
          \fi
        \else
          \BKM@EscapeString\BKM@dest
          \edef\BKM@action{%
            /A<<%
              /S/GoTo%
              /D(\BKM@dest)%
            >>%
          }%
        \fi
      \else
        \ifx\BKM@dest\@empty
          \ifx\BKM@page\@empty
            \def\BKM@page{0}%
          \else
            \BKM@CalcExpr\BKM@page\BKM@page-1%
          \fi
          \ifx\BKM@view\@empty
            \def\BKM@view{Fit}%
          \fi
          \edef\BKM@action{/D[\BKM@page/\BKM@view]}%
        \else
          \BKM@EscapeString\BKM@dest
          \edef\BKM@action{/D(\BKM@dest)}%
        \fi
        \BKM@EscapeString\BKM@gotor
        \edef\BKM@action{%
          /A<<%
            /S/GoToR%
            /F(\BKM@gotor)%
            \BKM@action
          >>%
        }%
      \fi
      \special{pdf:%
        out
              [%
              \ifBKM@open
                \ifnum\BKM@level<%
                    \expandafter\ltx@firstofone\expandafter
                    {\number\BKM@openlevel} %
                \else
                  -%
                \fi
              \else
                -%
              \fi
              ] %
            \BKM@abslevel
        <<%
          /Title(\BKM@title)%
          \ifx\BKM@color\@empty
          \else
            /C[\BKM@color]%
          \fi
          \ifnum\BKM@FLAGS>\z@
            /F \BKM@FLAGS
          \fi
          \BKM@action
        >>%
      }%
    \endgroup
  \fi
}
%    \end{macrocode}
%    \end{macro}
%    \begin{macro}{\BKM@getx}
%    \begin{macrocode}
\def\BKM@getx#1#2#3{%
  \def\BKM@x@parent{#1}%
  \def\BKM@x@level{#2}%
  \def\BKM@x@abslevel{#3}%
}
%    \end{macrocode}
%    \end{macro}
%
%    \begin{macrocode}
%</dvipdfm>
%    \end{macrocode}
%
% \subsection{\hologo{VTeX}\ 的驱动程序}
%
%    \begin{macrocode}
%<*vtex>
\NeedsTeXFormat{LaTeX2e}
\ProvidesFile{bkm-vtex.def}%
  [2020-11-06 v1.29 bookmark driver for VTeX (HO)]%
%    \end{macrocode}
%
%    \begin{macrocode}
\ifvtexpdf
\else
  \PackageWarningNoLine{bookmark}{%
    The VTeX driver only supports PDF mode%
  }%
\fi
%    \end{macrocode}
%
%    \begin{macro}{\BKM@id}
%    \begin{macrocode}
\newcount\BKM@id
\BKM@id=\z@
%    \end{macrocode}
%    \end{macro}
%
%    \begin{macro}{\BKM@0}
%    \begin{macrocode}
\@namedef{BKM@0}{00}
%    \end{macrocode}
%    \end{macro}
%    \begin{macro}{\ifBKM@sw}
%    \begin{macrocode}
\newif\ifBKM@sw
%    \end{macrocode}
%    \end{macro}
%
%    \begin{macro}{\bookmark}
%    \begin{macrocode}
\newcommand*{\bookmark}[2][]{%
  \if@filesw
    \begingroup
      \def\bookmark@text{#2}%
      \BKM@setup{#1}%
      \edef\BKM@prev{\the\BKM@id}%
      \global\advance\BKM@id\@ne
      \BKM@swtrue
      \@whilesw\ifBKM@sw\fi{%
        \ifnum\ifBKM@startatroot\z@\else\BKM@prev\fi=\z@
          \BKM@startatrootfalse
          \def\BKM@parent{0}%
          \expandafter\xdef\csname BKM@\the\BKM@id\endcsname{%
            0{\BKM@level}%
          }%
          \BKM@swfalse
        \else
          \expandafter\expandafter\expandafter\BKM@getx
              \csname BKM@\BKM@prev\endcsname
          \ifnum\BKM@level>\BKM@x@level\relax
            \let\BKM@parent\BKM@prev
            \expandafter\xdef\csname BKM@\the\BKM@id\endcsname{%
              {\BKM@prev}{\BKM@level}%
            }%
            \BKM@swfalse
          \else
            \let\BKM@prev\BKM@x@parent
          \fi
        \fi
      }%
      \pdfstringdef\BKM@title{\bookmark@text}%
      \BKM@vtex@title
      \edef\BKM@FLAGS{\BKM@PrintStyle}%
      \let\BKM@action\@empty
      \ifx\BKM@gotor\@empty
        \ifx\BKM@dest\@empty
          \ifx\BKM@named\@empty
            \ifx\BKM@rawaction\@empty
              \ifx\BKM@uri\@empty
                \ifx\BKM@page\@empty
                  \PackageError{bookmark}{Missing action}\@ehc
                  \def\BKM@action{!1}%
                \else
                  \edef\BKM@action{!\BKM@page}%
                \fi
              \else
                \BKM@EscapeString\BKM@uri
                \edef\BKM@action{%
                  <u=%
                    /S/URI%
                    /URI(\BKM@uri)%
                  >%
                }%
              \fi
            \else
              \edef\BKM@action{<u=\BKM@rawaction>}%
            \fi
          \else
            \BKM@EscapeName\BKM@named
            \edef\BKM@action{%
              <u=%
                /S/Named%
                /N/\BKM@named
              >%
            }%
          \fi
        \else
          \BKM@EscapeString\BKM@dest
          \edef\BKM@action{\BKM@dest}%
        \fi
      \else
        \ifx\BKM@dest\@empty
          \ifx\BKM@page\@empty
            \def\BKM@page{1}%
          \fi
          \ifx\BKM@view\@empty
            \def\BKM@view{Fit}%
          \fi
          \edef\BKM@action{/D[\BKM@page/\BKM@view]}%
        \else
          \BKM@EscapeString\BKM@dest
          \edef\BKM@action{/D(\BKM@dest)}%
        \fi
        \BKM@EscapeString\BKM@gotor
        \edef\BKM@action{%
          <u=%
            /S/GoToR%
            /F(\BKM@gotor)%
            \BKM@action
          >>%
        }%
      \fi
      \ifx\BKM@color\@empty
        \let\BKM@RGBcolor\@empty
      \else
        \expandafter\BKM@toRGB\BKM@color\@nil
      \fi
      \special{%
        !outline \BKM@action;%
        p=\BKM@parent,%
        i=\number\BKM@id,%
        s=%
          \ifBKM@open
            \ifnum\BKM@level<\BKM@openlevel
              o%
            \else
              c%
            \fi
          \else
            c%
          \fi,%
        \ifx\BKM@RGBcolor\@empty
        \else
          c=\BKM@RGBcolor,%
        \fi
        \ifnum\BKM@FLAGS>\z@
          f=\BKM@FLAGS,%
        \fi
        t=\BKM@title
      }%
    \endgroup
  \fi
}
%    \end{macrocode}
%    \end{macro}
%    \begin{macro}{\BKM@getx}
%    \begin{macrocode}
\def\BKM@getx#1#2{%
  \def\BKM@x@parent{#1}%
  \def\BKM@x@level{#2}%
}
%    \end{macrocode}
%    \end{macro}
%    \begin{macro}{\BKM@toRGB}
%    \begin{macrocode}
\def\BKM@toRGB#1 #2 #3\@nil{%
  \let\BKM@RGBcolor\@empty
  \BKM@toRGBComponent{#1}%
  \BKM@toRGBComponent{#2}%
  \BKM@toRGBComponent{#3}%
}
%    \end{macrocode}
%    \end{macro}
%    \begin{macro}{\BKM@toRGBComponent}
%    \begin{macrocode}
\def\BKM@toRGBComponent#1{%
  \dimen@=#1pt\relax
  \ifdim\dimen@>\z@
    \ifdim\dimen@<\p@
      \dimen@=255\dimen@
      \advance\dimen@ by 32768sp\relax
      \divide\dimen@ by 65536\relax
      \dimen@ii=\dimen@
      \divide\dimen@ii by 16\relax
      \edef\BKM@RGBcolor{%
        \BKM@RGBcolor
        \BKM@toHexDigit\dimen@ii
      }%
      \dimen@ii=16\dimen@ii
      \advance\dimen@-\dimen@ii
      \edef\BKM@RGBcolor{%
        \BKM@RGBcolor
        \BKM@toHexDigit\dimen@
      }%
    \else
      \edef\BKM@RGBcolor{\BKM@RGBcolor FF}%
    \fi
  \else
    \edef\BKM@RGBcolor{\BKM@RGBcolor00}%
  \fi
}
%    \end{macrocode}
%    \end{macro}
%    \begin{macro}{\BKM@toHexDigit}
%    \begin{macrocode}
\def\BKM@toHexDigit#1{%
  \ifcase\expandafter\@firstofone\expandafter{\number#1} %
    0\or 1\or 2\or 3\or 4\or 5\or 6\or 7\or
    8\or 9\or A\or B\or C\or D\or E\or F%
  \fi
}
%    \end{macrocode}
%    \end{macro}
%    \begin{macrocode}
\begingroup
  \catcode`\|=0 %
  \catcode`\\=12 %
%    \end{macrocode}
%    \begin{macro}{\BKM@vtex@title}
%    \begin{macrocode}
  |gdef|BKM@vtex@title{%
    |@onelevel@sanitize|BKM@title
    |edef|BKM@title{|expandafter|BKM@vtex@leftparen|BKM@title\(|@nil}%
    |edef|BKM@title{|expandafter|BKM@vtex@rightparen|BKM@title\)|@nil}%
    |edef|BKM@title{|expandafter|BKM@vtex@zero|BKM@title\0|@nil}%
    |edef|BKM@title{|expandafter|BKM@vtex@one|BKM@title\1|@nil}%
    |edef|BKM@title{|expandafter|BKM@vtex@two|BKM@title\2|@nil}%
    |edef|BKM@title{|expandafter|BKM@vtex@three|BKM@title\3|@nil}%
  }%
%    \end{macrocode}
%    \end{macro}
%    \begin{macro}{\BKM@vtex@leftparen}
%    \begin{macrocode}
  |gdef|BKM@vtex@leftparen#1\(#2|@nil{%
    #1%
    |ifx||#2||%
    |else
      (%
      |ltx@ReturnAfterFi{%
        |BKM@vtex@leftparen#2|@nil
      }%
    |fi
  }%
%    \end{macrocode}
%    \end{macro}
%    \begin{macro}{\BKM@vtex@rightparen}
%    \begin{macrocode}
  |gdef|BKM@vtex@rightparen#1\)#2|@nil{%
    #1%
    |ifx||#2||%
    |else
      )%
      |ltx@ReturnAfterFi{%
        |BKM@vtex@rightparen#2|@nil
      }%
    |fi
  }%
%    \end{macrocode}
%    \end{macro}
%    \begin{macro}{\BKM@vtex@zero}
%    \begin{macrocode}
  |gdef|BKM@vtex@zero#1\0#2|@nil{%
    #1%
    |ifx||#2||%
    |else
      |noexpand|hv@pdf@char0%
      |ltx@ReturnAfterFi{%
        |BKM@vtex@zero#2|@nil
      }%
    |fi
  }%
%    \end{macrocode}
%    \end{macro}
%    \begin{macro}{\BKM@vtex@one}
%    \begin{macrocode}
  |gdef|BKM@vtex@one#1\1#2|@nil{%
    #1%
    |ifx||#2||%
    |else
      |noexpand|hv@pdf@char1%
      |ltx@ReturnAfterFi{%
        |BKM@vtex@one#2|@nil
      }%
    |fi
  }%
%    \end{macrocode}
%    \end{macro}
%    \begin{macro}{\BKM@vtex@two}
%    \begin{macrocode}
  |gdef|BKM@vtex@two#1\2#2|@nil{%
    #1%
    |ifx||#2||%
    |else
      |noexpand|hv@pdf@char2%
      |ltx@ReturnAfterFi{%
        |BKM@vtex@two#2|@nil
      }%
    |fi
  }%
%    \end{macrocode}
%    \end{macro}
%    \begin{macro}{\BKM@vtex@three}
%    \begin{macrocode}
  |gdef|BKM@vtex@three#1\3#2|@nil{%
    #1%
    |ifx||#2||%
    |else
      |noexpand|hv@pdf@char3%
      |ltx@ReturnAfterFi{%
        |BKM@vtex@three#2|@nil
      }%
    |fi
  }%
%    \end{macrocode}
%    \end{macro}
%    \begin{macrocode}
|endgroup
%    \end{macrocode}
%
%    \begin{macrocode}
%</vtex>
%    \end{macrocode}
%
% \subsection{\hologo{pdfTeX}\ 的驱动程序}
%
%    \begin{macrocode}
%<*pdftex>
\NeedsTeXFormat{LaTeX2e}
\ProvidesFile{bkm-pdftex.def}%
  [2020-11-06 v1.29 bookmark driver for pdfTeX (HO)]%
%    \end{macrocode}
%
%    \begin{macro}{\BKM@DO@entry}
%    \begin{macrocode}
\def\BKM@DO@entry#1#2{%
  \begingroup
    \kvsetkeys{BKM@DO}{#1}%
    \def\BKM@DO@title{#2}%
    \ifx\BKM@DO@srcfile\@empty
    \else
      \BKM@UnescapeHex\BKM@DO@srcfile
    \fi
    \BKM@UnescapeHex\BKM@DO@title
    \expandafter\expandafter\expandafter\BKM@getx
        \csname BKM@\BKM@DO@id\endcsname\@empty\@empty
    \let\BKM@attr\@empty
    \ifx\BKM@DO@flags\@empty
    \else
      \edef\BKM@attr{\BKM@attr/F \BKM@DO@flags}%
    \fi
    \ifx\BKM@DO@color\@empty
    \else
      \edef\BKM@attr{\BKM@attr/C[\BKM@DO@color]}%
    \fi
    \ifx\BKM@attr\@empty
    \else
      \edef\BKM@attr{attr{\BKM@attr}}%
    \fi
    \let\BKM@action\@empty
    \ifx\BKM@DO@gotor\@empty
      \ifx\BKM@DO@dest\@empty
        \ifx\BKM@DO@named\@empty
          \ifx\BKM@DO@rawaction\@empty
            \ifx\BKM@DO@uri\@empty
              \ifx\BKM@DO@page\@empty
                \PackageError{bookmark}{%
                  Missing action\BKM@SourceLocation
                }\@ehc
                \edef\BKM@action{goto page1{/Fit}}%
              \else
                \ifx\BKM@DO@view\@empty
                  \def\BKM@DO@view{Fit}%
                \fi
                \edef\BKM@action{goto page\BKM@DO@page{/\BKM@DO@view}}%
              \fi
            \else
              \BKM@UnescapeHex\BKM@DO@uri
              \BKM@EscapeString\BKM@DO@uri
              \edef\BKM@action{user{<</S/URI/URI(\BKM@DO@uri)>>}}%
            \fi
          \else
            \BKM@UnescapeHex\BKM@DO@rawaction
            \edef\BKM@action{%
              user{%
                <<%
                  \BKM@DO@rawaction
                >>%
              }%
            }%
          \fi
        \else
          \BKM@EscapeName\BKM@DO@named
          \edef\BKM@action{%
            user{<</S/Named/N/\BKM@DO@named>>}%
          }%
        \fi
      \else
        \BKM@UnescapeHex\BKM@DO@dest
        \BKM@DefGotoNameAction\BKM@action\BKM@DO@dest
      \fi
    \else
      \ifx\BKM@DO@dest\@empty
        \ifx\BKM@DO@page\@empty
          \def\BKM@DO@page{0}%
        \else
          \BKM@CalcExpr\BKM@DO@page\BKM@DO@page-1%
        \fi
        \ifx\BKM@DO@view\@empty
          \def\BKM@DO@view{Fit}%
        \fi
        \edef\BKM@action{/D[\BKM@DO@page/\BKM@DO@view]}%
      \else
        \BKM@UnescapeHex\BKM@DO@dest
        \BKM@EscapeString\BKM@DO@dest
        \edef\BKM@action{/D(\BKM@DO@dest)}%
      \fi
      \BKM@UnescapeHex\BKM@DO@gotor
      \BKM@EscapeString\BKM@DO@gotor
      \edef\BKM@action{%
        user{%
          <<%
            /S/GoToR%
            /F(\BKM@DO@gotor)%
            \BKM@action
          >>%
        }%
      }%
    \fi
    \pdfoutline\BKM@attr\BKM@action
                count\ifBKM@DO@open\else-\fi\BKM@x@childs
                {\BKM@DO@title}%
  \endgroup
}
%    \end{macrocode}
%    \end{macro}
%    \begin{macro}{\BKM@DefGotoNameAction}
%    \cs{BKM@DefGotoNameAction}\ 宏是一个用于 \xpackage{hypdestopt}\ 宏包的钩子(hook)。
%    \begin{macrocode}
\def\BKM@DefGotoNameAction#1#2{%
  \BKM@EscapeString\BKM@DO@dest
  \edef#1{goto name{#2}}%
}
%    \end{macrocode}
%    \end{macro}
%    \begin{macrocode}
%</pdftex>
%    \end{macrocode}
%
%    \begin{macrocode}
%<*pdftex|pdfmark>
%    \end{macrocode}
%    \begin{macro}{\BKM@SourceLocation}
%    \begin{macrocode}
\def\BKM@SourceLocation{%
  \ifx\BKM@DO@srcfile\@empty
    \ifx\BKM@DO@srcline\@empty
    \else
      .\MessageBreak
      Source: line \BKM@DO@srcline
    \fi
  \else
    \ifx\BKM@DO@srcline\@empty
      .\MessageBreak
      Source: file `\BKM@DO@srcfile'%
    \else
      .\MessageBreak
      Source: file `\BKM@DO@srcfile', line \BKM@DO@srcline
    \fi
  \fi
}
%    \end{macrocode}
%    \end{macro}
%    \begin{macrocode}
%</pdftex|pdfmark>
%    \end{macrocode}
%
% \subsection{具有 pdfmark 特色(specials)的驱动程序}
%
% \subsubsection{dvips 驱动程序}
%
%    \begin{macrocode}
%<*dvips>
\NeedsTeXFormat{LaTeX2e}
\ProvidesFile{bkm-dvips.def}%
  [2020-11-06 v1.29 bookmark driver for dvips (HO)]%
%    \end{macrocode}
%    \begin{macro}{\BKM@PSHeaderFile}
%    \begin{macrocode}
\def\BKM@PSHeaderFile#1{%
  \special{PSfile=#1}%
}
%    \end{macrocode}
%    \begin{macro}{\BKM@filename}
%    \begin{macrocode}
\def\BKM@filename{\jobname.out.ps}
%    \end{macrocode}
%    \end{macro}
%    \begin{macrocode}
\AddToHook{shipout/lastpage}{%
  \BKM@pdfmark@out
  \BKM@PSHeaderFile\BKM@filename
  }
%    \end{macrocode}
%    \end{macro}
%    \begin{macrocode}
%</dvips>
%    \end{macrocode}
%
% \subsubsection{公共部分(Common part)}
%
%    \begin{macrocode}
%<*pdfmark>
%    \end{macrocode}
%
%    \begin{macro}{\BKM@pdfmark@out}
%    不要在这里使用 \xpackage{rerunfilecheck}\ 宏包,因为在 \hologo{TeX}\ 运行期间不会
%    读取 \cs{BKM@filename}\ 文件。
%    \begin{macrocode}
\def\BKM@pdfmark@out{%
  \if@filesw
    \newwrite\BKM@file
    \immediate\openout\BKM@file=\BKM@filename\relax
    \BKM@write{\@percentchar!}%
    \BKM@write{/pdfmark where{pop}}%
    \BKM@write{%
      {%
        /globaldict where{pop globaldict}{userdict}ifelse%
        /pdfmark/cleartomark load put%
      }%
    }%
    \BKM@write{ifelse}%
  \else
    \let\BKM@write\@gobble
    \let\BKM@DO@entry\@gobbletwo
  \fi
}
%    \end{macrocode}
%    \end{macro}
%    \begin{macro}{\BKM@write}
%    \begin{macrocode}
\def\BKM@write#{%
  \immediate\write\BKM@file
}
%    \end{macrocode}
%    \end{macro}
%
%    \begin{macro}{\BKM@DO@entry}
%    Pdfmark 的规范(specification)说明 |/Color| 是颜色(color)的键名(key name),
%    但是 ghostscript 只将键(key)传递到 PDF 文件中,因此键名必须是 |/C|。
%    \begin{macrocode}
\def\BKM@DO@entry#1#2{%
  \begingroup
    \kvsetkeys{BKM@DO}{#1}%
    \ifx\BKM@DO@srcfile\@empty
    \else
      \BKM@UnescapeHex\BKM@DO@srcfile
    \fi
    \def\BKM@DO@title{#2}%
    \BKM@UnescapeHex\BKM@DO@title
    \expandafter\expandafter\expandafter\BKM@getx
        \csname BKM@\BKM@DO@id\endcsname\@empty\@empty
    \let\BKM@attr\@empty
    \ifx\BKM@DO@flags\@empty
    \else
      \edef\BKM@attr{\BKM@attr/F \BKM@DO@flags}%
    \fi
    \ifx\BKM@DO@color\@empty
    \else
      \edef\BKM@attr{\BKM@attr/C[\BKM@DO@color]}%
    \fi
    \let\BKM@action\@empty
    \ifx\BKM@DO@gotor\@empty
      \ifx\BKM@DO@dest\@empty
        \ifx\BKM@DO@named\@empty
          \ifx\BKM@DO@rawaction\@empty
            \ifx\BKM@DO@uri\@empty
              \ifx\BKM@DO@page\@empty
                \PackageError{bookmark}{%
                  Missing action\BKM@SourceLocation
                }\@ehc
                \edef\BKM@action{%
                  /Action/GoTo%
                  /Page 1%
                  /View[/Fit]%
                }%
              \else
                \ifx\BKM@DO@view\@empty
                  \def\BKM@DO@view{Fit}%
                \fi
                \edef\BKM@action{%
                  /Action/GoTo%
                  /Page \BKM@DO@page
                  /View[/\BKM@DO@view]%
                }%
              \fi
            \else
              \BKM@UnescapeHex\BKM@DO@uri
              \BKM@EscapeString\BKM@DO@uri
              \edef\BKM@action{%
                /Action<<%
                  /Subtype/URI%
                  /URI(\BKM@DO@uri)%
                >>%
              }%
            \fi
          \else
            \BKM@UnescapeHex\BKM@DO@rawaction
            \edef\BKM@action{%
              /Action<<%
                \BKM@DO@rawaction
              >>%
            }%
          \fi
        \else
          \BKM@EscapeName\BKM@DO@named
          \edef\BKM@action{%
            /Action<<%
              /Subtype/Named%
              /N/\BKM@DO@named
            >>%
          }%
        \fi
      \else
        \BKM@UnescapeHex\BKM@DO@dest
        \BKM@EscapeString\BKM@DO@dest
        \edef\BKM@action{%
          /Action/GoTo%
          /Dest(\BKM@DO@dest)cvn%
        }%
      \fi
    \else
      \ifx\BKM@DO@dest\@empty
        \ifx\BKM@DO@page\@empty
          \def\BKM@DO@page{1}%
        \fi
        \ifx\BKM@DO@view\@empty
          \def\BKM@DO@view{Fit}%
        \fi
        \edef\BKM@action{%
          /Page \BKM@DO@page
          /View[/\BKM@DO@view]%
        }%
      \else
        \BKM@UnescapeHex\BKM@DO@dest
        \BKM@EscapeString\BKM@DO@dest
        \edef\BKM@action{%
          /Dest(\BKM@DO@dest)cvn%
        }%
      \fi
      \BKM@UnescapeHex\BKM@DO@gotor
      \BKM@EscapeString\BKM@DO@gotor
      \edef\BKM@action{%
        /Action/GoToR%
        /File(\BKM@DO@gotor)%
        \BKM@action
      }%
    \fi
    \BKM@write{[}%
    \BKM@write{/Title(\BKM@DO@title)}%
    \ifnum\BKM@x@childs>\z@
      \BKM@write{/Count \ifBKM@DO@open\else-\fi\BKM@x@childs}%
    \fi
    \ifx\BKM@attr\@empty
    \else
      \BKM@write{\BKM@attr}%
    \fi
    \BKM@write{\BKM@action}%
    \BKM@write{/OUT pdfmark}%
  \endgroup
}
%    \end{macrocode}
%    \end{macro}
%    \begin{macrocode}
%</pdfmark>
%    \end{macrocode}
%
% \subsection{\xoption{pdftex}\ 和 \xoption{pdfmark}\ 的公共部分}
%
%    \begin{macrocode}
%<*pdftex|pdfmark>
%    \end{macrocode}
%
% \subsubsection{写入辅助文件(auxiliary file)}
%
%    \begin{macrocode}
\AddToHook{begindocument}{%
 \immediate\write\@mainaux{\string\providecommand\string\BKM@entry[2]{}}}
%    \end{macrocode}
%
%    \begin{macro}{\BKM@id}
%    \begin{macrocode}
\newcount\BKM@id
\BKM@id=\z@
%    \end{macrocode}
%    \end{macro}
%
%    \begin{macro}{\BKM@0}
%    \begin{macrocode}
\@namedef{BKM@0}{000}
%    \end{macrocode}
%    \end{macro}
%    \begin{macro}{\ifBKM@sw}
%    \begin{macrocode}
\newif\ifBKM@sw
%    \end{macrocode}
%    \end{macro}
%
%    \begin{macro}{\bookmark}
%    \begin{macrocode}
\newcommand*{\bookmark}[2][]{%
  \if@filesw
    \begingroup
      \BKM@InitSourceLocation
      \def\bookmark@text{#2}%
      \BKM@setup{#1}%
      \ifx\BKM@srcfile\@empty
      \else
        \BKM@EscapeHex\BKM@srcfile
      \fi
      \edef\BKM@prev{\the\BKM@id}%
      \global\advance\BKM@id\@ne
      \BKM@swtrue
      \@whilesw\ifBKM@sw\fi{%
        \ifnum\ifBKM@startatroot\z@\else\BKM@prev\fi=\z@
          \BKM@startatrootfalse
          \expandafter\xdef\csname BKM@\the\BKM@id\endcsname{%
            0{\BKM@level}0%
          }%
          \BKM@swfalse
        \else
          \expandafter\expandafter\expandafter\BKM@getx
              \csname BKM@\BKM@prev\endcsname
          \ifnum\BKM@level>\BKM@x@level\relax
            \expandafter\xdef\csname BKM@\the\BKM@id\endcsname{%
              {\BKM@prev}{\BKM@level}0%
            }%
            \ifnum\BKM@prev>\z@
              \BKM@CalcExpr\BKM@CalcResult\BKM@x@childs+1%
              \expandafter\xdef\csname BKM@\BKM@prev\endcsname{%
                {\BKM@x@parent}{\BKM@x@level}{\BKM@CalcResult}%
              }%
            \fi
            \BKM@swfalse
          \else
            \let\BKM@prev\BKM@x@parent
          \fi
        \fi
      }%
      \pdfstringdef\BKM@title{\bookmark@text}%
      \edef\BKM@FLAGS{\BKM@PrintStyle}%
      \csname BKM@HypDestOptHook\endcsname
      \BKM@EscapeHex\BKM@dest
      \BKM@EscapeHex\BKM@uri
      \BKM@EscapeHex\BKM@gotor
      \BKM@EscapeHex\BKM@rawaction
      \BKM@EscapeHex\BKM@title
      \immediate\write\@mainaux{%
        \string\BKM@entry{%
          id=\number\BKM@id
          \ifBKM@open
            \ifnum\BKM@level<\BKM@openlevel
              ,open%
            \fi
          \fi
          \BKM@auxentry{dest}%
          \BKM@auxentry{named}%
          \BKM@auxentry{uri}%
          \BKM@auxentry{gotor}%
          \BKM@auxentry{page}%
          \BKM@auxentry{view}%
          \BKM@auxentry{rawaction}%
          \BKM@auxentry{color}%
          \ifnum\BKM@FLAGS>\z@
            ,flags=\BKM@FLAGS
          \fi
          \BKM@auxentry{srcline}%
          \BKM@auxentry{srcfile}%
        }{\BKM@title}%
      }%
    \endgroup
  \fi
}
%    \end{macrocode}
%    \end{macro}
%    \begin{macro}{\BKM@getx}
%    \begin{macrocode}
\def\BKM@getx#1#2#3{%
  \def\BKM@x@parent{#1}%
  \def\BKM@x@level{#2}%
  \def\BKM@x@childs{#3}%
}
%    \end{macrocode}
%    \end{macro}
%    \begin{macro}{\BKM@auxentry}
%    \begin{macrocode}
\def\BKM@auxentry#1{%
  \expandafter\ifx\csname BKM@#1\endcsname\@empty
  \else
    ,#1={\csname BKM@#1\endcsname}%
  \fi
}
%    \end{macrocode}
%    \end{macro}
%
%    \begin{macro}{\BKM@InitSourceLocation}
%    \begin{macrocode}
\def\BKM@InitSourceLocation{%
  \edef\BKM@srcline{\the\inputlineno}%
  \BKM@LuaTeX@InitFile
  \ifx\BKM@srcfile\@empty
    \ltx@IfUndefined{currfilepath}{}{%
      \edef\BKM@srcfile{\currfilepath}%
    }%
  \fi
}
%    \end{macrocode}
%    \end{macro}
%    \begin{macro}{\BKM@LuaTeX@InitFile}
%    \begin{macrocode}
\ifluatex
  \ifnum\luatexversion>36 %
    \def\BKM@LuaTeX@InitFile{%
      \begingroup
        \ltx@LocToksA={}%
      \edef\x{\endgroup
        \def\noexpand\BKM@srcfile{%
          \the\expandafter\ltx@LocToksA
          \directlua{%
             if status and status.filename then %
               tex.settoks('ltx@LocToksA', status.filename)%
             end%
          }%
        }%
      }\x
    }%
  \else
    \let\BKM@LuaTeX@InitFile\relax
  \fi
\else
  \let\BKM@LuaTeX@InitFile\relax
\fi
%    \end{macrocode}
%    \end{macro}
%
% \subsubsection{读取辅助数据(auxiliary data)}
%
%    \begin{macrocode}
\SetupKeyvalOptions{family=BKM@DO,prefix=BKM@DO@}
\DeclareStringOption[0]{id}
\DeclareBoolOption{open}
\DeclareStringOption{flags}
\DeclareStringOption{color}
\DeclareStringOption{dest}
\DeclareStringOption{named}
\DeclareStringOption{uri}
\DeclareStringOption{gotor}
\DeclareStringOption{page}
\DeclareStringOption{view}
\DeclareStringOption{rawaction}
\DeclareStringOption{srcline}
\DeclareStringOption{srcfile}
%    \end{macrocode}
%
%    \begin{macrocode}
\AtBeginDocument{%
  \let\BKM@entry\BKM@DO@entry
}
%    \end{macrocode}
%
%    \begin{macrocode}
%</pdftex|pdfmark>
%    \end{macrocode}
%
% \subsection{\xoption{atend}\ 选项}
%
% \subsubsection{钩子(Hook)}
%
%    \begin{macrocode}
%<*package>
%    \end{macrocode}
%    \begin{macrocode}
\ifBKM@atend
\else
%    \end{macrocode}
%    \begin{macro}{\BookmarkAtEnd}
%    这是一个虚拟定义(dummy definition),如果没有给出 \xoption{atend}\ 选项,它将生成一个警告。
%    \begin{macrocode}
  \newcommand{\BookmarkAtEnd}[1]{%
    \PackageWarning{bookmark}{%
      Ignored, because option `atend' is missing%
    }%
  }%
%    \end{macrocode}
%    \end{macro}
%    \begin{macrocode}
  \expandafter\endinput
\fi
%    \end{macrocode}
%    \begin{macro}{\BookmarkAtEnd}
%    \begin{macrocode}
\newcommand*{\BookmarkAtEnd}{%
  \g@addto@macro\BKM@EndHook
}
%    \end{macrocode}
%    \end{macro}
%    \begin{macrocode}
\let\BKM@EndHook\@empty
%    \end{macrocode}
%    \begin{macrocode}
%</package>
%    \end{macrocode}
%
% \subsubsection{在文档末尾使用钩子的驱动程序}
%
%    驱动程序 \xoption{pdftex}\ 使用 LaTeX 钩子 \xoption{enddocument/afterlastpage}
%    (相当于以前使用的 \xpackage{atveryend}\ 的 \cs{AfterLastShipout}),因为它仍然需要 \xext{aux}\ 文件。
%    它使用 \cs{pdfoutline}\ 作为最后一页之后可以使用的书签(bookmakrs)。
%    \begin{itemize}
%    \item
%      驱动程序 \xoption{pdftex}\ 使用 \cs{pdfoutline}, \cs{pdfoutline}\ 可以在最后一页之后使用。
%    \end{itemize}
%    \begin{macrocode}
%<*pdftex>
\ifBKM@atend
  \AddToHook{enddocument/afterlastpage}{%
    \BKM@EndHook
  }%
\fi
%</pdftex>
%    \end{macrocode}
%
% \subsubsection{使用 \xoption{shipout/lastpage}\ 的驱动程序}
%
%    其他驱动程序使用 \cs{special}\ 命令实现 \cs{bookmark}。因此,最后的书签(last bookmarks)
%    必须放在最后一页(last page),而不是之后。不能使用 \cs{AtEndDocument},因为为时已晚,
%    最后一页已经输出了。因此,我们使用 LaTeX 钩子 \xoption{shipout/lastpage}。至少需要运行
%    两次 \hologo{LaTeX}。PostScript 驱动程序 \xoption{dvips}\ 使用外部 PostScript 文件作为书签。
%    为了避免与 pgf 发生冲突,文件写入(file writing)也被移到了最后一个输出页面(shipout page)。
%    \begin{macrocode}
%<*dvipdfm|vtex|pdfmark>
\ifBKM@atend
  \AddToHook{shipout/lastpage}{\BKM@EndHook}%
\fi
%</dvipdfm|vtex|pdfmark>
%    \end{macrocode}
%
% \section{安装(Installation)}
%
% \subsection{下载(Download)}
%
% \paragraph{宏包(Package)。} 在 CTAN\footnote{\CTANpkg{bookmark}}上提供此宏包:
% \begin{description}
% \item[\CTAN{macros/latex/contrib/bookmark/bookmark.dtx}] 源文件(source file)。
% \item[\CTAN{macros/latex/contrib/bookmark/bookmark.pdf}] 文档(documentation)。
% \end{description}
%
%
% \paragraph{捆绑包(Bundle)。} “bookmark”捆绑包(bundle)的所有宏包(packages)都可以在兼
% 容 TDS 的 ZIP 归档文件中找到。在那里,宏包已经被解包,文档文件(documentation files)已经生成。
% 文件(files)和目录(directories)遵循 TDS 标准。
% \begin{description}
% \item[\CTANinstall{install/macros/latex/contrib/bookmark.tds.zip}]
% \end{description}
% \emph{TDS}\ 是指标准的“用于 \TeX\ 文件的目录结构(Directory Structure)”(\CTANpkg{tds})。
% 名称中带有 \xfile{texmf}\ 的目录(directories)通常以这种方式组织。
%
% \subsection{捆绑包(Bundle)的安装}
%
% \paragraph{解压(Unpacking)。} 在您选择的 TDS 树(也称为 \xfile{texmf}\ 树)中解
% 压 \xfile{bookmark.tds.zip},例如(在 linux 中):
% \begin{quote}
%   |unzip bookmark.tds.zip -d ~/texmf|
% \end{quote}
%
% \subsection{宏包(Package)的安装}
%
% \paragraph{解压(Unpacking)。} \xfile{.dtx}\ 文件是一个自解压 \docstrip\ 归档文件(archive)。
% 这些文件是通过 \plainTeX\ 运行 \xfile{.dtx}\ 来提取的:
% \begin{quote}
%   \verb|tex bookmark.dtx|
% \end{quote}
%
% \paragraph{TDS.} 现在,不同的文件必须移动到安装 TDS 树(installation TDS tree)
% (也称为 \xfile{texmf}\ 树)中的不同目录中:
% \begin{quote}
% \def\t{^^A
% \begin{tabular}{@{}>{\ttfamily}l@{ $\rightarrow$ }>{\ttfamily}l@{}}
%   bookmark.sty & tex/latex/bookmark/bookmark.sty\\
%   bkm-dvipdfm.def & tex/latex/bookmark/bkm-dvipdfm.def\\
%   bkm-dvips.def & tex/latex/bookmark/bkm-dvips.def\\
%   bkm-pdftex.def & tex/latex/bookmark/bkm-pdftex.def\\
%   bkm-vtex.def & tex/latex/bookmark/bkm-vtex.def\\
%   bookmark.pdf & doc/latex/bookmark/bookmark.pdf\\
%   bookmark-example.tex & doc/latex/bookmark/bookmark-example.tex\\
%   bookmark.dtx & source/latex/bookmark/bookmark.dtx\\
% \end{tabular}^^A
% }^^A
% \sbox0{\t}^^A
% \ifdim\wd0>\linewidth
%   \begingroup
%     \advance\linewidth by\leftmargin
%     \advance\linewidth by\rightmargin
%   \edef\x{\endgroup
%     \def\noexpand\lw{\the\linewidth}^^A
%   }\x
%   \def\lwbox{^^A
%     \leavevmode
%     \hbox to \linewidth{^^A
%       \kern-\leftmargin\relax
%       \hss
%       \usebox0
%       \hss
%       \kern-\rightmargin\relax
%     }^^A
%   }^^A
%   \ifdim\wd0>\lw
%     \sbox0{\small\t}^^A
%     \ifdim\wd0>\linewidth
%       \ifdim\wd0>\lw
%         \sbox0{\footnotesize\t}^^A
%         \ifdim\wd0>\linewidth
%           \ifdim\wd0>\lw
%             \sbox0{\scriptsize\t}^^A
%             \ifdim\wd0>\linewidth
%               \ifdim\wd0>\lw
%                 \sbox0{\tiny\t}^^A
%                 \ifdim\wd0>\linewidth
%                   \lwbox
%                 \else
%                   \usebox0
%                 \fi
%               \else
%                 \lwbox
%               \fi
%             \else
%               \usebox0
%             \fi
%           \else
%             \lwbox
%           \fi
%         \else
%           \usebox0
%         \fi
%       \else
%         \lwbox
%       \fi
%     \else
%       \usebox0
%     \fi
%   \else
%     \lwbox
%   \fi
% \else
%   \usebox0
% \fi
% \end{quote}
% 如果你有一个 \xfile{docstrip.cfg}\ 文件,该文件能配置并启用 \docstrip\ 的 TDS 安装功能,
% 则一些文件可能已经在正确的位置了,请参阅 \docstrip\ 的文档(documentation)。
%
% \subsection{刷新文件名数据库}
%
% 如果您的 \TeX~发行版(\TeX\,Live、\mikTeX、\dots)依赖于文件名数据库(file name databases),
% 则必须刷新这些文件名数据库。例如,\TeX\,Live\ 用户运行 \verb|texhash| 或 \verb|mktexlsr|。
%
% \subsection{一些感兴趣的细节}
%
% \paragraph{用 \LaTeX\ 解压。}
% \xfile{.dtx}\ 根据格式(format)选择其操作(action):
% \begin{description}
% \item[\plainTeX:] 运行 \docstrip\ 并解压文件。
% \item[\LaTeX:] 生成文档。
% \end{description}
% 如果您坚持通过 \LaTeX\ 使用\docstrip (实际上 \docstrip\ 并不需要 \LaTeX),那么请您的意图告知自动检测程序:
% \begin{quote}
%   \verb|latex \let\install=y\input{bookmark.dtx}|
% \end{quote}
% 不要忘记根据 shell 的要求引用这个参数(argument)。
%
% \paragraph{知生成文档。}
% 您可以同时使用 \xfile{.dtx}\ 或 \xfile{.drv}\ 来生成文档。可以通过配置文件 \xfile{ltxdoc.cfg}\ 配置该进程。
% 例如,如果您希望 A4 作为纸张格式,请将下面这行写入此文件中:
% \begin{quote}
%   \verb|\PassOptionsToClass{a4paper}{article}|
% \end{quote}
% 下面是一个如何使用 pdf\LaTeX\ 生成文档的示例:
% \begin{quote}
%\begin{verbatim}
%pdflatex bookmark.dtx
%makeindex -s gind.ist bookmark.idx
%pdflatex bookmark.dtx
%makeindex -s gind.ist bookmark.idx
%pdflatex bookmark.dtx
%\end{verbatim}
% \end{quote}
%
% \begin{thebibliography}{9}
%
% \bibitem{hyperref}
%   Sebastian Rahtz, Heiko Oberdiek:
%   \textit{The \xpackage{hyperref} package};
%   2011/04/17 v6.82g;
%   \CTANpkg{hyperref}
%
% \bibitem{currfile}
%   Martin Scharrer:
%   \textit{The \xpackage{currfile} package};
%   2011/01/09 v0.4.
%   \CTANpkg{currfile}
%
% \end{thebibliography}
%
% \begin{History}
%   \begin{Version}{2007/02/19 v0.1}
%   \item
%     First experimental version.
%   \end{Version}
%   \begin{Version}{2007/02/20 v0.2}
%   \item
%     Option \xoption{startatroot} added.
%   \item
%     Dummies for \cs{pdf(un)escape...} commands added to get
%     the package basically work for non-\hologo{pdfTeX} users.
%   \end{Version}
%   \begin{Version}{2007/02/21 v0.3}
%   \item
%     Dependency from \hologo{pdfTeX} 1.30 removed by using package
%     \xpackage{pdfescape}.
%   \end{Version}
%   \begin{Version}{2007/02/22 v0.4}
%   \item
%     \xpackage{hyperref}'s \xoption{bookmarkstype} respected.
%   \end{Version}
%   \begin{Version}{2007/03/02 v0.5}
%   \item
%     Driver options \xoption{vtex} (PDF mode), \xoption{dvipsone},
%     and \xoption{textures} added.
%   \item
%     Implementation of option \xoption{depth} completed. Division names
%     are supported, see \xpackage{hyperref}'s
%     option \xoption{bookmarksdepth}.
%   \item
%     \xpackage{hyperref}'s options \xoption{bookmarksopen},
%     \xoption{bookmarksopenlevel}, and \xoption{bookmarksdepth} respected.
%   \end{Version}
%   \begin{Version}{2007/03/03 v0.6}
%   \item
%     Option \xoption{numbered} as alias for \xpackage{hyperref}'s
%     \xoption{bookmarksnumbered}.
%   \end{Version}
%   \begin{Version}{2007/03/07 v0.7}
%   \item
%     Dependency from \hologo{eTeX} removed.
%   \end{Version}
%   \begin{Version}{2007/04/09 v0.8}
%   \item
%     Option \xoption{atend} added.
%   \item
%     Option \xoption{rgbcolor} removed.
%     \verb|rgbcolor=<r> <g> <b>| can be replaced by
%     \verb|color=[rgb]{<r>,<g>,<b>}|.
%   \item
%     Support of recent cvs version (2007-03-29) of dvipdfmx
%     that extends the \cs{special} for bookmarks to specify
%     open outline entries. Option \xoption{dvipdfmx-outline-open}
%     or \cs{SpecialDvipdfmxOutlineOpen} notify the package.
%   \end{Version}
%   \begin{Version}{2007/04/25 v0.9}
%   \item
%     The syntax of \cs{special} of dvipdfmx, if feature
%     \xoption{dvipdfmx-outline-open} is enabled, has changed.
%     Now cvs version 2007-04-25 is needed.
%   \end{Version}
%   \begin{Version}{2007/05/29 v1.0}
%   \item
%     Bug fix in code for second parameter of XYZ.
%   \end{Version}
%   \begin{Version}{2007/07/13 v1.1}
%   \item
%     Fix for pdfmark with GoToR action.
%   \end{Version}
%   \begin{Version}{2007/09/25 v1.2}
%   \item
%     pdfmark driver respects \cs{nofiles}.
%   \end{Version}
%   \begin{Version}{2008/08/08 v1.3}
%   \item
%     Package \xpackage{flags} replaced by package \xpackage{bitset}.
%     Now flags are also supported without \hologo{eTeX}.
%   \item
%     Hook for package \xpackage{hypdestopt} added.
%   \end{Version}
%   \begin{Version}{2008/09/13 v1.4}
%   \item
%     Fix for bug introduced in v1.3, package \xpackage{flags} is one-based,
%     but package \xpackage{bitset} is zero-based. Thus options \xoption{bold}
%     and \xoption{italic} are wrong in v1.3. (Daniel M\"ullner)
%   \end{Version}
%   \begin{Version}{2009/08/13 v1.5}
%   \item
%     Except for driver options the other options are now local options.
%     This resolves a problem with KOMA-Script v3.00 and its option \xoption{open}.
%   \end{Version}
%   \begin{Version}{2009/12/06 v1.6}
%   \item
%     Use of package \xpackage{atveryend} for drivers \xoption{pdftex}
%     and \xoption{pdfmark}.
%   \end{Version}
%   \begin{Version}{2009/12/07 v1.7}
%   \item
%     Use of package \xpackage{atveryend} fixed.
%   \end{Version}
%   \begin{Version}{2009/12/17 v1.8}
%   \item
%     Support of \xpackage{hyperref} 2009/12/17 v6.79v for \hologo{XeTeX}.
%   \end{Version}
%   \begin{Version}{2010/03/30 v1.9}
%   \item
%     Package name in an error message fixed.
%   \end{Version}
%   \begin{Version}{2010/04/03 v1.10}
%   \item
%     Option \xoption{style} and macro \cs{bookmarkdefinestyle} added.
%   \item
%     Hook support with option \xoption{addtohook} added.
%   \item
%     \cs{bookmarkget} added.
%   \end{Version}
%   \begin{Version}{2010/04/04 v1.11}
%   \item
%     Bug fix (introduced in v1.10).
%   \end{Version}
%   \begin{Version}{2010/04/08 v1.12}
%   \item
%     Requires \xpackage{ltxcmds} 2010/04/08.
%   \end{Version}
%   \begin{Version}{2010/07/23 v1.13}
%   \item
%     Support for \xclass{memoir}'s \cs{booknumberline} added.
%   \end{Version}
%   \begin{Version}{2010/09/02 v1.14}
%   \item
%     (Local) options \xoption{draft} and \xoption{final} added.
%   \end{Version}
%   \begin{Version}{2010/09/25 v1.15}
%   \item
%     Fix for option \xoption{dvipdfmx-outline-open}.
%   \item
%     Option \xoption{dvipdfmx-outline-open} is set automatically,
%     if XeTeX $\geq$ 0.9995 is detected.
%   \end{Version}
%   \begin{Version}{2010/10/19 v1.16}
%   \item
%     Option `startatroot' now acts globally.
%   \item
%     Option `level' also accepts names the same way as option `depth'.
%   \end{Version}
%   \begin{Version}{2010/10/25 v1.17}
%   \item
%     \cs{bookmarksetupnext} added.
%   \item
%     Using \cs{kvsetkeys} of package \xpackage{kvsetkeys}, because
%     \cs{setkeys} of package \xpackage{keyval} is not reentrant.
%     This can cause problems (unknown keys) with older versions of
%     hyperref that also uses \cs{setkeys} (found by GL).
%   \end{Version}
%   \begin{Version}{2010/11/05 v1.18}
%   \item
%     Use of \cs{pdf@ifdraftmode} of package \xpackage{pdftexcmds} for
%     the default of option \xoption{draft}.
%   \end{Version}
%   \begin{Version}{2011/03/20 v1.19}
%   \item
%     Use of \cs{dimexpr} fixed, if \hologo{eTeX} is not used.
%     (Bug found by Martin M\"unch.)
%   \item
%     Fix in documentation. Also layout options work without \hologo{eTeX}.
%   \end{Version}
%   \begin{Version}{2011/04/13 v1.20}
%   \item
%     Bug fix: \cs{BKM@SetDepth} renamed to \cs{BKM@SetDepthOrLevel}.
%   \end{Version}
%   \begin{Version}{2011/04/21 v1.21}
%   \item
%     Some support for file name and line number in error messages
%     at end of document (pdfTeX and pdfmark based drivers).
%   \end{Version}
%   \begin{Version}{2011/05/13 v1.22}
%   \item
%     Change of version 2010/11/05 v1.18 reverted, because otherwise
%     draftmode disables some \xext{aux} file entries.
%   \end{Version}
%   \begin{Version}{2011/09/19 v1.23}
%   \item
%     Some \cs{renewcommand}s changed to \cs{def} to avoid trouble
%     if the commands are not defined, because hyperref stopped early.
%   \end{Version}
%   \begin{Version}{2011/12/02 v1.24}
%   \item
%     Small optimization in \cs{BKM@toHexDigit}.
%   \end{Version}
%   \begin{Version}{2016/05/16 v1.25}
%   \item
%     Documentation updates.
%   \end{Version}
%   \begin{Version}{2016/05/17 v1.26}
%   \item
%     define \cs{pdfoutline} to allow pdftex driver to be used with Lua\TeX.
%   \end{Version}
%   \begin{Version}{2019/06/04 v1.27}
%   \item
%     unknown style options are ignored (issue 67)
%   \end{Version}

%   \begin{Version}{2019/12/03 v1.28}
%   \item
%     Documentation updates.
%   \item adjust package loading (all required packages already loaded
%     by \xpackage{hyperref}).
%   \end{Version}
%   \begin{Version}{2020-11-06 v1.29}
%   \item Adapted the dvips to avoid a clash with pgf.
%         https://github.com/pgf-tikz/pgf/issues/944
%   \item All drivers now use the new LaTeX hooks
%         and so require a format 2020-10-01 or newer. The older
%         drivers are provided as frozen versions and are used if an older
%         format is detected.
%   \item Added support for destlabel option of hyperref, https://github.com/ho-tex/bookmark/issues/1
%   \item Removed the \xoption{dvipsone} and \xoption{textures} driver.
%   \item Removed the code for option \xoption{dvipdfmx-outline-open}
%     and \cs{SpecialDvipdfmxOutlineOpen}. All dvipdfmx version should now support
%     this out-of-the-box.
%   \end{Version}
% \end{History}
%
% \PrintIndex
%
% \Finale
\endinput

%        (quote the arguments according to the demands of your shell)
%
% Documentation:
%    (a) If bookmark.drv is present:
%           latex bookmark.drv
%    (b) Without bookmark.drv:
%           latex bookmark.dtx; ...
%    The class ltxdoc loads the configuration file ltxdoc.cfg
%    if available. Here you can specify further options, e.g.
%    use A4 as paper format:
%       \PassOptionsToClass{a4paper}{article}
%
%    Programm calls to get the documentation (example):
%       pdflatex bookmark.dtx
%       makeindex -s gind.ist bookmark.idx
%       pdflatex bookmark.dtx
%       makeindex -s gind.ist bookmark.idx
%       pdflatex bookmark.dtx
%
% Installation:
%    TDS:tex/latex/bookmark/bookmark.sty
%    TDS:tex/latex/bookmark/bkm-dvipdfm.def
%    TDS:tex/latex/bookmark/bkm-dvips.def
%    TDS:tex/latex/bookmark/bkm-pdftex.def
%    TDS:tex/latex/bookmark/bkm-vtex.def
%    TDS:tex/latex/bookmark/bkm-dvipdfm-2019-12-03.def
%    TDS:tex/latex/bookmark/bkm-dvips-2019-12-03.def
%    TDS:tex/latex/bookmark/bkm-pdftex-2019-12-03.def
%    TDS:tex/latex/bookmark/bkm-vtex-2019-12-03.def%
%    TDS:doc/latex/bookmark/bookmark.pdf
%    TDS:doc/latex/bookmark/bookmark-example.tex
%    TDS:source/latex/bookmark/bookmark.dtx
%    TDS:source/latex/bookmark/bookmark-frozen.dtx
%
%<*ignore>
\begingroup
  \catcode123=1 %
  \catcode125=2 %
  \def\x{LaTeX2e}%
\expandafter\endgroup
\ifcase 0\ifx\install y1\fi\expandafter
         \ifx\csname processbatchFile\endcsname\relax\else1\fi
         \ifx\fmtname\x\else 1\fi\relax
\else\csname fi\endcsname
%</ignore>
%<*install>
\input docstrip.tex
\Msg{************************************************************************}
\Msg{* Installation}
\Msg{* Package: bookmark 2020-11-06 v1.29 PDF bookmarks (HO)}
\Msg{************************************************************************}

\keepsilent
\askforoverwritefalse

\let\MetaPrefix\relax
\preamble

This is a generated file.

Project: bookmark
Version: 2020-11-06 v1.29

Copyright (C)
   2007-2011 Heiko Oberdiek
   2016-2020 Oberdiek Package Support Group

This work may be distributed and/or modified under the
conditions of the LaTeX Project Public License, either
version 1.3c of this license or (at your option) any later
version. This version of this license is in
   https://www.latex-project.org/lppl/lppl-1-3c.txt
and the latest version of this license is in
   https://www.latex-project.org/lppl.txt
and version 1.3 or later is part of all distributions of
LaTeX version 2005/12/01 or later.

This work has the LPPL maintenance status "maintained".

The Current Maintainers of this work are
Heiko Oberdiek and the Oberdiek Package Support Group
https://github.com/ho-tex/bookmark/issues


This work consists of the main source file bookmark.dtx and bookmark-frozen.dtx
and the derived files
   bookmark.sty, bookmark.pdf, bookmark.ins, bookmark.drv,
   bkm-dvipdfm.def, bkm-dvips.def, bkm-pdftex.def, bkm-vtex.def,
   bkm-dvipdfm-2019-12-03.def, bkm-dvips-2019-12-03.def,
   bkm-pdftex-2019-12-03.def, bkm-vtex-2019-12-03.def,
   bookmark-example.tex.

\endpreamble
\let\MetaPrefix\DoubleperCent

\generate{%
  \file{bookmark.ins}{\from{bookmark.dtx}{install}}%
  \file{bookmark.drv}{\from{bookmark.dtx}{driver}}%
  \usedir{tex/latex/bookmark}%
  \file{bookmark.sty}{\from{bookmark.dtx}{package}}%
  \file{bkm-dvipdfm.def}{\from{bookmark.dtx}{dvipdfm}}%
  \file{bkm-dvips.def}{\from{bookmark.dtx}{dvips,pdfmark}}%
  \file{bkm-pdftex.def}{\from{bookmark.dtx}{pdftex}}%
  \file{bkm-vtex.def}{\from{bookmark.dtx}{vtex}}%
  \usedir{doc/latex/bookmark}%
  \file{bookmark-example.tex}{\from{bookmark.dtx}{example}}%
  \file{bkm-pdftex-2019-12-03.def}{\from{bookmark-frozen.dtx}{pdftexfrozen}}%
  \file{bkm-dvips-2019-12-03.def}{\from{bookmark-frozen.dtx}{dvipsfrozen}}%
  \file{bkm-vtex-2019-12-03.def}{\from{bookmark-frozen.dtx}{vtexfrozen}}%
  \file{bkm-dvipdfm-2019-12-03.def}{\from{bookmark-frozen.dtx}{dvipdfmfrozen}}%
}

\catcode32=13\relax% active space
\let =\space%
\Msg{************************************************************************}
\Msg{*}
\Msg{* To finish the installation you have to move the following}
\Msg{* files into a directory searched by TeX:}
\Msg{*}
\Msg{*     bookmark.sty, bkm-dvipdfm.def, bkm-dvips.def,}
\Msg{*     bkm-pdftex.def, bkm-vtex.def, bkm-dvipdfm-2019-12-03.def,}
\Msg{*     bkm-dvips-2019-12-03.def, bkm-pdftex-2019-12-03.def,}
\Msg{*     and bkm-vtex-2019-12-03.def}
\Msg{*}
\Msg{* To produce the documentation run the file `bookmark.drv'}
\Msg{* through LaTeX.}
\Msg{*}
\Msg{* Happy TeXing!}
\Msg{*}
\Msg{************************************************************************}

\endbatchfile
%</install>
%<*ignore>
\fi
%</ignore>
%<*driver>
\NeedsTeXFormat{LaTeX2e}
\ProvidesFile{bookmark.drv}%
  [2020-11-06 v1.29 PDF bookmarks (HO)]%
\documentclass{ltxdoc}
\usepackage{ctex}
\usepackage{indentfirst}
\setlength{\parindent}{2em}
\usepackage{holtxdoc}[2011/11/22]
\usepackage{xcolor}
\usepackage{hyperref}
\usepackage[open,openlevel=3,atend]{bookmark}[2020/11/06] %%%打开书签,显示的深度为3级,即显示part、section、subsection。
\bookmarksetup{color=red}
\begin{document}

  \renewcommand{\contentsname}{目\quad 录}
  \renewcommand{\abstractname}{摘\quad 要}
  \renewcommand{\historyname}{历史}
  \DocInput{bookmark.dtx}%
\end{document}
%</driver>
% \fi
%
%
%
% \GetFileInfo{bookmark.drv}
%
%% \title{\xpackage{bookmark} 宏包}
% \title{\heiti {\Huge \textbf{\xpackage{bookmark}\ 宏包}}}
% \date{2020-11-06\ \ \ v1.29}
% \author{Heiko Oberdiek \thanks
% {如有问题请点击:\url{https://github.com/ho-tex/bookmark/issues}}\\[5pt]赣医一附院神经科\ \ 黄旭华\ \ \ \ 译}
%
% \maketitle
%
% \begin{abstract}
% 这个宏包为 \xpackage{hyperref}\ 宏包实现了一个新的书签(bookmark)(大纲[outline])组织。现在
% 可以设置样式(style)和颜色(color)等书签属性(bookmark properties)。其他动作类型(action types)可用
% (URI、GoToR、Named)。书签是在第一次编译运行(compile run)中生成的。\xpackage{hyperref}\
% 宏包必需运行两次。
% \end{abstract}
%
% \tableofcontents
%
% \section{文档(Documentation)}
%
% \subsection{介绍}
%
% 这个 \xpackage{bookmark}\ 宏包试图为书签(bookmarks)提供一个更现代的管理:
% \begin{itemize}
% \item 书签已经在第一次 \hologo{TeX}\ 编译运行(compile run)中生成。
% \item 可以更改书签的字体样式(font style)和颜色(color)。
% \item 可以执行比简单的 GoTo 操作(actions)更多的操作。
% \end{itemize}
%
% 与 \xpackage{hyperref} \cite{hyperref} 一样,书签(bookmarks)也是按照书签生成宏
% (bookmark generating macros)(\cs{bookmark})的顺序生成的。级别号(level number)用于
% 定义书签的树结构(tree structure)。限制没有那么严格:
% \begin{itemize}
% \item 级别值(level values)可以跳变(jump)和省略(omit)。\cs{subsubsection}\ 可以跟在
%       \cs{chapter}\ 之后。这种情况如在 \xpackage{hyperref}\ 中则产生错误,它将显示一个警告(warning)
%       并尝试修复此错误。
% \item 多个书签可能指向同一目标(destination)。在 \xpackage{hyperref}\ 中,这会完全弄乱
%       书签树(bookmark tree),因为算法假设(algorithm assumes)目标名称(destination names)
%       是键(keys)(唯一的)。
% \end{itemize}
%
% 注意,这个宏包是作为书签管理(bookmark management)的实验平台(experimentation platform)。
% 欢迎反馈。此外,在未来的版本中,接口(interfaces)也可能发生变化。
%
% \subsection{选项(Options)}
%
% 可在以下四个地方放置选项(options):
% \begin{enumerate}
% \item \cs{usepackage}|[|\meta{options}|]{bookmark}|\\
%       这是放置驱动程序选项(driver options)和 \xoption{atend}\ 选项的唯一位置。
% \item \cs{bookmarksetup}|{|\meta{options}|}|\\
%       此命令仅用于设置选项(setting options)。
% \item \cs{bookmarksetupnext}|{|\meta{options}|}|\\
%       这些选项在下一个 \cs{bookmark}\ 命令的选项之后存储(stored)和调用(called)。
% \item \cs{bookmark}|[|\meta{options}|]{|\meta{title}|}|\\
%       此命令设置书签。选项设置(option settings)仅限于此书签。
% \end{enumerate}
% 异常(Exception):加载该宏包后,无法更改驱动程序选项(Driver options)、\xoption{atend}\ 选项
% 、\xoption{draft}\slash\xoption{final}选项。
%
% \subsubsection{\xoption{draft} 和 \xoption{final}\ 选项}
%
% 如果一个\LaTeX\ 文件要被编译了多次,那么可以使用 \xoption{draft}\ 选项来禁用该宏包的书签内
% 容(bookmark stuff),这样可以节省一点时间。默认 \xoption{final}\ 选项。两个选项都是
% 布尔选项(boolean options),如果没有值,则使用值 |true|。|draft=true| 与 |final=false| 相同。
%
% 除了驱动程序选项(driver options)之外,\xpackage{bookmark}\ 宏包选项都是局部选项(local options)。
% \xoption{draft}\ 选项和 \xoption{final}\ 选项均属于文档类选项(class option)(译者注:文档类选项为全局选项),
% 因此,在 \xpackage{bookmark}\ 宏包中未能看到这两个选项。如果您想使用全局的(global) \xoption{draft}选项
% 来优化第一次 \LaTeX\ 运行(runs),可以在导言(preamble)中引入 \xpackage{ifdraft}\ 宏包并设置 \LaTeX\ 的
% \cs{PassOptionsToPackage},例如:
%\begin{quote}
%\begin{verbatim}
%\documentclass[draft]{article}
%\usepackage{ifdraft}
%\ifdraft{%
%   \PassOptionsToPackage{draft}{bookmark}%
%}{}
%\end{verbatim}
%\end{quote}
%
% \subsubsection{驱动程序选项(Driver options)}
%
% 支持的驱动程序( drivers)包括 \xoption{pdftex}、\xoption{dvips}、\xoption{dvipdfm} (\xoption{xetex})、
% \xoption{vtex}。\hologo{TeX}\ 引擎 \hologo{pdfTeX}、\hologo{XeTeX}、\hologo{VTeX}\ 能被自动检测到。
% 默认的 DVI 驱动程序是 \xoption{dvips}。这可以通过 \cs{BookmarkDriverDefault}\ 在配置
% 文件 \xfile{bookmark.cfg}\ 中进行更改,例如:
% \begin{quote}
% |\def\BookmarkDriverDefault{dvipdfm}|
% \end{quote}
% 当前版本的(current versions)驱动程序使用新的 \LaTeX\ 钩子(\LaTeX-hooks)。如果检测到比
% 2020-10-01 更旧的格式,则将以前驱动程序的冻结版本(frozen versions)作为备份(fallback)。
%
% \paragraph{用 dvipdfmx 打开书签(bookmarks)。}旧版本的宏包有一个 \xoption{dvipdfmx-outline-open}\ 选项
% 可以激活代码,而该代码可以指定一个大纲条目(outline entry)是否打开。该宏包现在假设所有使用的 dvipdfmx 版本都是
% 最新版本,足以理解该代码,因此始终激活该代码。选项本身将被忽略。
%
%
% \subsubsection{布局选项(Layout options)}
%
% \paragraph{字体(Font)选项:}
%
% \begin{description}
% \item[\xoption{bold}:] 如果受 PDF 浏览器(PDF viewer)支持,书签将以粗体字体(bold font)显示(自 PDF 1.4起)。
% \item[\xoption{italic}:] 使用斜体字体(italic font)(自 PDF 1.4起)。
% \end{description}
% \xoption{bold}(粗体) 和 \xoption{italic}(斜体)可以同时使用。而 |false| 值(value)禁用字体选项。
%
% \paragraph{颜色(Color)选项:}
%
% 彩色书签(Colored bookmarks)是 PDF 1.4 的一个特性(feature),并非所有的 PDF 浏览器(PDF viewers)都支持彩色书签。
% \begin{description}
% \item[\xoption{color}:] 这里 color(颜色)可以作为 \xpackage{color}\ 宏包或 \xpackage{xcolor}\ 宏包的
% 颜色规范(color specification)给出。空值(empty value)表示未设置颜色属性。如果未加载 \xpackage{xcolor}\ 宏包,
% 能识别的值(recognized values)只有:
%   \begin{itemize}
%   \item 空值(empty value)表示未设置颜色属性,\\
%         例如:|color={}|
%   \item 颜色模型(color model) rgb 的显式颜色规范(explicit color specification),\\
%         例如,红色(red):|color=[rgb]{1,0,0}|
%   \item 颜色模型(color model)灰(gray)的显式颜色规范(explicit color specification),\\
%         例如,深灰色(dark gray):|color=[gray]{0.25}|
%   \end{itemize}
%   请注意,如果加载了 \xpackage{color}\ 宏包,此限制(restriction)也适用。然而,如果加载了 \xpackage{xcolor}\ 宏包,
%   则可以使用所有颜色规范(color specifications)。
% \end{description}
%
% \subsubsection{动作选项(Action options)}
%
% \begin{description}
% \item[\xoption{dest}:] 目的地名称(destination name)。
% \item[\xoption{page}:] 页码(page number),第一页(first page)为 1。
% \item[\xoption{view}:] 浏览规范(view specification),示例如下:\\
%   |view={FitB}|, |view={FitH 842}|, |view={XYZ 0 100 null}|\ \  一些浏览规范参数(view specification parameters)
%   将数字(numbers)视为具有单位 bp 的参数。它们可以作为普通数字(plain numbers)或在 \cs{calc}\ 内部以
%   长度表达式(length expressions)给出。如果加载了 \xpackage{calc}\ 宏包,则支持该宏包的表达式(expressions)。否则,
%   使用 \hologo{eTeX}\ 的 \cs{dimexpr}。例如:\\
%   |view={FitH \calc{\paperheight-\topmargin-1in}}|\\
%   |view={XYZ 0 \calc{\paperheight} null}|\\
%   注意 \cs{calc}\ 不能用于 |XYZ| 的第三个参数,因为该参数是缩放值(zoom value),而不是长度(length)。

% \item[\xoption{named}:] 已命名的动作(Named action)的名称:\\
%   |FirstPage|(第一页),|LastPage|(最后一页),|NextPage|(下一页),|PrevPage|(前一页)
% \item[\xoption{gotor}:] 外部(external) PDF 文件的名称。
% \item[\xoption{uri}:] URI 规范(URI specification)。
% \item[\xoption{rawaction}:] 原始动作规范(raw action specification)。由于这些规范取决于驱动程序(driver),因此不应使用此选项。
% \end{description}
% 通过分析指定的选项来选择书签的适当动作。动作由不同的选项集(sets of options)区分:
% \begin{quote}
 \begin{tabular}{|@{}r|l@{}|}
%   \hline
%   \ \textbf{动作(Action)}\  & \ \textbf{选项(Options)}\ \\ \hline
%   \ \textsf{GoTo}\  &\  \xoption{dest}\ \\ \hline
%   \ \textsf{GoTo}\  & \ \xoption{page} + \xoption{view}\ \\ \hline
%   \ \textsf{GoToR}\  & \ \xoption{gotor} + \xoption{dest}\ \\ \hline
%   \ \textsf{GoToR}\  & \ \xoption{gotor} + \xoption{page} + \xoption{view}\ \ \ \\ \hline
%   \ \textsf{Named}\  &\  \xoption{named}\ \\ \hline
%   \ \textsf{URI}\  & \ \xoption{uri}\ \\ \hline
% \end{tabular}
% \end{quote}
%
% \paragraph{缺少动作(Missing actions)。}
% 如果动作缺少 \xpackage{bookmark}\ 宏包,则抛出错误消息(error message)。根据驱动程序(driver)
% (\xoption{pdftex}、\xoption{dvips}\ 和好友[friends]),宏包在文档末尾很晚才检测到它。
% 自 2011/04/21 v1.21 版本以后,该宏包尝试打印 \cs{bookmark}\ 的相应出现的行号(line number)和文件名(file name)。
% 然而,\hologo{TeX}\ 确实提供了行号,但不幸的是,文件名是一个秘密(secret)。但该宏包有如下获取文件名的方法:
% \begin{itemize}
% \item 如果 \hologo{LuaTeX} (独立于 DVI 或 PDF 模式)正在运行,则自动使用其 |status.filename|。
% \item 宏包的 \cs{currfile} \cite{currfile}\ 重新定义了 \hologo{LaTeX}\ 的内部结构,以跟踪文件名(file name)。
% 如果加载了该宏包,那么它的 \cs{currfilepath}\ 将被检测到并由 \xpackage{bookmark}\ 自动使用。
% \item 可以通过 \cs{bookmarksetup}\ 或 \cs{bookmark}\ 中的 \xoption{scrfile}\ 选项手动设置(set manually)文件名。
% 但是要小心,手动设置会禁用以前的文件名检测方法。错误的(wrong)或丢失的(missed)文件名设置(file name setting)可能会在错误消息中
% 为您提供错误的源位置(source location)。
% \end{itemize}
%
% \subsubsection{级别选项(Level options)}
%
% 书签条目(bookmark entries)的顺序由 \cs{bookmark}\ 命令的的出现顺序(appearance order)定义。
% 树结构(tree structure)由书签节点(bookmark nodes)的属性 \xoption{level}(级别)构建。
% \xoption{level}\ 的值是整数(integers)。如果书签条目级别的值高于前一个节点,则该条目将成为
% 前一个节点的子(child)节点。差值的绝对值并不重要。
%
% \xpackage{bookmark}\ 宏包能记住全局属性(global property)“current level(当前级别)”中上
% 一个书签条目(previous bookmark entry)的级别。
%
% 级别系统的(level system)行为(behaviour)可以通过以下选项进行配置:
% \begin{description}
% \item[\xoption{level}:]
%    设置级别(level),请参阅上面的说明。如果给出的选项 \xoption{level}\ 没有值,那么将恢复默
%    认行为,即将“当前级别(current level)”用作级别值(level value)。自 2010/10/19 v1.16 版本以来,
%    如果宏 \cs{toclevel@part}、\cs{toclevel@section}\ 被定义过(通过 \xpackage{hyperref}\ 宏包完成,
%    请参阅它的 \xoption{bookmarkdepth}\ 选项),则 \xpackage{bookmark}\ 宏包还支持 |part|、|section| 等名称。
%
% \item[\xoption{rellevel}:]
%    设置相对于前一级别的(previous level)级别。正值表示书签条目成为前一个书签条目的子条目。
% \item[\xoption{keeplevel}:]
%    使用由\xoption{level}\ 或 \xoption{rellevel}\ 设置的级别,但不要更改全局属性“current level(当前级别)”。
%    可以通过设置为 |false| 来禁用该选项。
% \item[\xoption{startatroot}:]
%    此时,书签树(bookmark tree)再次从顶层(top level)开始。下一个书签条目不会作为上一个条目的子条目进行排序。
%    示例场景:文档使用 part。但是,最后几章(last chapters)不应放在最后一部分(last part)下面:
%    \begin{quote}
%\begin{verbatim}
%\documentclass{book}
%[...]
%\begin{document}
%  \part{第一部分}
%    \chapter{第一部分的第1章}
%    [...]
%  \part{第二部分(Second part)}
%    \chapter{第二部分的第1章}
%    [...]
%  \bookmarksetup{startatroot}
%  \chapter{Index}% 不属于第二部分
%\end{document}
%\end{verbatim}
%    \end{quote}
% \end{description}
%
% \subsubsection{样式定义(Style definitions)}
%
% 样式(style)是一组选项设置(option settings)。它可以由宏 \cs{bookmarkdefinestyle}\ 定义,
% 并由它的 \xoption{style}\ 选项使用。
% \begin{declcs}{bookmarkdefinestyle} \M{name} \M{key value list}
% \end{declcs}
% 选项设置(option settings)的 \meta{key value list}(键值列表)被指定为样式名(style \meta{name})。
%
% \begin{description}
% \item[\xoption{style}:]
%   \xoption{style}\ 选项的值是以前定义的样式的名称(name)。现在执行其选项设置(option settings)。
%   选项可以包括 \xoption{style}\ 选项。通过递归调用相同样式的无限递归(endless recursion)被阻止并抛出一个错误。
% \end{description}
%
% \subsubsection{钩子支持(Hook support)}
%
% 处理宏\cs{bookmark}\ 的可选选项(optional options)后,就会调用钩子(hook)。
% \begin{description}
% \item[\xoption{addtohook}:]
%   代码(code)作为该选项的值添加到钩子中。
% \end{description}
%
% \begin{declcs}{bookmarkget} \M{option}
% \end{declcs}
% \cs{bookmarkget}\ 宏提取 \meta{option}\ 选项的最新选项设置(latest option setting)的值。
% 对于布尔选项(boolean option),如果启用布尔选项,则返回 1,否则结果为零。结果数字(resulting numbers)
% 可以直接用于 \cs{ifnum}\ 或 \cs{ifcase}。如果您想要数字 \texttt{0}\ 和 \texttt{1},
% 请在 \cs{bookmarkget}\ 前面加上 \cs{number}\ 作为前缀。\cs{bookmarkget}\ 宏是可展开的(expandable)。
% 如果选项不受支持,则返回空字符串(empty string)。受支持的布尔选项有:
% \begin{quote}
%   \xoption{bold}、
%   \xoption{italic}、
%   \xoption{open}
% \end{quote}
% 其他受支持的选项有:
% \begin{quote}
%   \xoption{depth}、
%   \xoption{dest}、
%   \xoption{color}、
%   \xoption{gotor}、
%   \xoption{level}、
%   \xoption{named}、
%   \xoption{openlevel}、
%   \xoption{page}、
%   \xoption{rawaction}、
%   \xoption{uri}、
%   \xoption{view}、
% \end{quote}
% 另外,以下键(key)是可用的:
% \begin{quote}
%   \xoption{text}
% \end{quote}
% 它返回大纲条目(outline entry)的文本(text)。
%
% \paragraph{选项设置(Option setting)。}
% 在钩子(hook)内部可以使用 \cs{bookmarksetup}\ 设置选项。
%
% \subsection{与 \xpackage{hyperref}\ 的兼容性}
%
% \xpackage{bookmark}\ 宏包自动禁用 \xpackage{hyperref}\ 宏包的书签(bookmarks)。但是,
% \xpackage{bookmark}\ 宏包使用了 \xpackage{hyperref}\ 宏包的一些代码。例如,
% \xpackage{bookmark}\ 宏包重新定义了 \xpackage{hyperref}\ 宏包在 \cs{addcontentsline}\ 命令
% 和其他命令中插入的\cs{Hy@writebookmark}\ 钩子。因此,不应禁用 \xpackage{hyperref}\ 宏包的书签。
%
% \xpackage{bookmark}\ 宏包使用 \xpackage{hyperref}\ 宏包的 \cs{pdfstringdef},且不提供替换(replacement)。
%
% \xpackage{hyperref}\ 宏包的一些选项也能在 \xpackage{bookmark}\ 宏包中实现(implemented):
% \begin{quote}
% \begin{tabular}{|l@{}|l@{}|}
%   \hline
%   \xpackage{hyperref}\ 宏包的选项\  &\ \xpackage{bookmark}\ 宏包的选项\ \ \\ \hline
%   \xoption{bookmarksdepth} &\ \xoption{depth}\\ \hline
%   \xoption{bookmarksopen} & \ \xoption{open}\\ \hline
%   \xoption{bookmarksopenlevel}\ \ \  &\ \xoption{openlevel}\\ \hline
%   \xoption{bookmarksnumbered} \ \ \ &\ \xoption{numbered}\\ \hline
% \end{tabular}
% \end{quote}
%
% 还可以使用以下命令:
% \begin{quote}
%   \cs{pdfbookmark}\\
%   \cs{currentpdfbookmark}\\
%   \cs{subpdfbookmark}\\
%   \cs{belowpdfbookmark}
% \end{quote}
%
% \subsection{在末尾添加书签}
%
% 宏包选项 \xoption{atend}\ 启用以下宏(macro):
% \begin{declcs}{BookmarkAtEnd}
%   \M{stuff}
% \end{declcs}
% \cs{BookmarkAtEnd}\ 宏将 \meta{stuff}\ 放在文档末尾。\meta{stuff}\ 表示书签命令(bookmark commands)。举例:
% \begin{quote}
%\begin{verbatim}
%\usepackage[atend]{bookmark}
%\BookmarkAtEnd{%
%  \bookmarksetup{startatroot}%
%  \bookmark[named=LastPage, level=0]{Last page}%
%}
%\end{verbatim}
% \end{quote}
%
% 或者,可以在 \cs{bookmark}\ 中给出 \xoption{startatroot}\ 选项:
% \begin{quote}
%\begin{verbatim}
%\BookmarkAtEnd{%
%  \bookmark[
%    startatroot,
%    named=LastPage,
%    level=0,
%  ]{Last page}%
%}
%\end{verbatim}
% \end{quote}
%
% \paragraph{备注(Remarks):}
% \begin{itemize}
% \item
%   \cs{BookmarkAtEnd} 隐藏了这样一个事实,即在文档末尾添加书签的方法取决于驱动程序(driver)。
%
%   为此,驱动程序 \xoption{pdftex}\ 使用 \xpackage{atveryend}\ 宏包。如果 \cs{AtEndDocument}\ 太早,
%   最后一个页面(last page)可能不会被发送出去(shipped out)。由于需要 \xext{aux}\ 文件,此驱动程序使
%   用 \cs{AfterLastShipout}。
%
%   其他驱动程序(\xoption{dvipdfm}、\xoption{xetex}、\xoption{vtex})的实现(implementation)
%   取决于 \cs{special},\cs{special}\ 在最后一页之后没有效果。在这种情况下,\xpackage{atenddvi}\ 宏包的
%   \cs{AtEndDvi}\ 有帮助。它将其参数(argument)放在文档的最后一页(last page)。至少需要运行 \hologo{LaTeX}\ 两次,
%   因为最后一页是由引用(reference)检测到的。
%
%   \xoption{dvips}\ 现在使用新的 LaTeX 钩子 \texttt{shipout/lastpage}。
% \item
%   未指定 \cs{BookmarkAtEnd}\ 参数的扩展时间(time of expansion)。这可以立即发生,也可以在文档末尾发生。
% \end{itemize}
%
% \subsection{限制/行动计划}
%
% \begin{itemize}
% \item 支持缺失动作(missing actions)(启动,\dots)。
% \item 对 \xpackage{hyperref}\ 的 \xoption{bookmarkstype}\ 选项进行了更好的设计(design)。
% \end{itemize}
%
% \section{示例(Example)}
%
%    \begin{macrocode}
%<*example>
%    \end{macrocode}
%    \begin{macrocode}
\documentclass{article}
\usepackage{xcolor}[2007/01/21]
\usepackage{hyperref}
\usepackage[
  open,
  openlevel=2,
  atend
]{bookmark}[2019/12/03]

\bookmarksetup{color=blue}

\BookmarkAtEnd{%
  \bookmarksetup{startatroot}%
  \bookmark[named=LastPage, level=0]{End/Last page}%
  \bookmark[named=FirstPage, level=1]{First page}%
}

\begin{document}
\section{First section}
\subsection{Subsection A}
\begin{figure}
  \hypertarget{fig}{}%
  A figure.
\end{figure}
\bookmark[
  rellevel=1,
  keeplevel,
  dest=fig
]{A figure}
\subsection{Subsection B}
\subsubsection{Subsubsection C}
\subsection{Umlauts: \"A\"O\"U\"a\"o\"u\ss}
\newpage
\bookmarksetup{
  bold,
  color=[rgb]{1,0,0}
}
\section{Very important section}
\bookmarksetup{
  italic,
  bold=false,
  color=blue
}
\subsection{Italic section}
\bookmarksetup{
  italic=false
}
\part{Misc}
\section{Diverse}
\subsubsection{Subsubsection, omitting subsection}
\bookmarksetup{
  startatroot
}
\section{Last section outside part}
\subsection{Subsection}
\bookmarksetup{
  color={}
}
\begingroup
  \bookmarksetup{level=0, color=green!80!black}
  \bookmark[named=FirstPage]{First page}
  \bookmark[named=LastPage]{Last page}
  \bookmark[named=PrevPage]{Previous page}
  \bookmark[named=NextPage]{Next page}
\endgroup
\bookmark[
  page=2,
  view=FitH 800
]{Page 2, FitH 800}
\bookmark[
  page=2,
  view=FitBH \calc{\paperheight-\topmargin-1in-\headheight-\headsep}
]{Page 2, FitBH top of text body}
\bookmark[
  uri={http://www.dante.de/},
  color=magenta
]{Dante homepage}
\bookmark[
  gotor={t.pdf},
  page=1,
  view={XYZ 0 1000 null},
  color=cyan!75!black
]{File t.pdf}
\bookmark[named=FirstPage]{First page}
\bookmark[rellevel=1, named=LastPage]{Last page (rellevel=1)}
\bookmark[named=PrevPage]{Previous page}
\bookmark[level=0, named=FirstPage]{First page (level=0)}
\bookmark[
  rellevel=1,
  keeplevel,
  named=LastPage
]{Last page (rellevel=1, keeplevel)}
\bookmark[named=PrevPage]{Previous page}
\end{document}
%    \end{macrocode}
%    \begin{macrocode}
%</example>
%    \end{macrocode}
%
% \StopEventually{
% }
%
% \section{实现(Implementation)}
%
% \subsection{宏包(Package)}
%
%    \begin{macrocode}
%<*package>
\NeedsTeXFormat{LaTeX2e}
\ProvidesPackage{bookmark}%
  [2020-11-06 v1.29 PDF bookmarks (HO)]%
%    \end{macrocode}
%
% \subsubsection{要求(Requirements)}
%
% \paragraph{\hologo{eTeX}.}
%
%    \begin{macro}{\BKM@CalcExpr}
%    \begin{macrocode}
\begingroup\expandafter\expandafter\expandafter\endgroup
\expandafter\ifx\csname numexpr\endcsname\relax
  \def\BKM@CalcExpr#1#2#3#4{%
    \begingroup
      \count@=#2\relax
      \advance\count@ by#3#4\relax
      \edef\x{\endgroup
        \def\noexpand#1{\the\count@}%
      }%
    \x
  }%
\else
  \def\BKM@CalcExpr#1#2#3#4{%
    \edef#1{%
      \the\numexpr#2#3#4\relax
    }%
  }%
\fi
%    \end{macrocode}
%    \end{macro}
%
% \paragraph{\hologo{pdfTeX}\ 的转义功能(escape features)}
%
%    \begin{macro}{\BKM@EscapeName}
%    \begin{macrocode}
\def\BKM@EscapeName#1{%
  \ifx#1\@empty
  \else
    \EdefEscapeName#1#1%
  \fi
}%
%    \end{macrocode}
%    \end{macro}
%    \begin{macro}{\BKM@EscapeString}
%    \begin{macrocode}
\def\BKM@EscapeString#1{%
  \ifx#1\@empty
  \else
    \EdefEscapeString#1#1%
  \fi
}%
%    \end{macrocode}
%    \end{macro}
%    \begin{macro}{\BKM@EscapeHex}
%    \begin{macrocode}
\def\BKM@EscapeHex#1{%
  \ifx#1\@empty
  \else
    \EdefEscapeHex#1#1%
  \fi
}%
%    \end{macrocode}
%    \end{macro}
%    \begin{macro}{\BKM@UnescapeHex}
%    \begin{macrocode}
\def\BKM@UnescapeHex#1{%
  \EdefUnescapeHex#1#1%
}%
%    \end{macrocode}
%    \end{macro}
%
% \paragraph{宏包(Packages)。}
%
% 不要加载由 \xpackage{hyperref}\ 加载的宏包
%    \begin{macrocode}
\RequirePackage{hyperref}[2010/06/18]
%    \end{macrocode}
%
% \subsubsection{宏包选项(Package options)}
%
%    \begin{macrocode}
\SetupKeyvalOptions{family=BKM,prefix=BKM@}
\DeclareLocalOptions{%
  atend,%
  bold,%
  color,%
  depth,%
  dest,%
  draft,%
  final,%
  gotor,%
  italic,%
  keeplevel,%
  level,%
  named,%
  numbered,%
  open,%
  openlevel,%
  page,%
  rawaction,%
  rellevel,%
  srcfile,%
  srcline,%
  startatroot,%
  uri,%
  view,%
}
%    \end{macrocode}
%    \begin{macro}{\bookmarksetup}
%    \begin{macrocode}
\newcommand*{\bookmarksetup}{\kvsetkeys{BKM}}
%    \end{macrocode}
%    \end{macro}
%    \begin{macro}{\BKM@setup}
%    \begin{macrocode}
\def\BKM@setup#1{%
  \bookmarksetup{#1}%
  \ifx\BKM@HookNext\ltx@empty
  \else
    \expandafter\bookmarksetup\expandafter{\BKM@HookNext}%
    \BKM@HookNextClear
  \fi
  \BKM@hook
  \ifBKM@keeplevel
  \else
    \xdef\BKM@currentlevel{\BKM@level}%
  \fi
}
%    \end{macrocode}
%    \end{macro}
%
%    \begin{macro}{\bookmarksetupnext}
%    \begin{macrocode}
\newcommand*{\bookmarksetupnext}[1]{%
  \ltx@GlobalAppendToMacro\BKM@HookNext{,#1}%
}
%    \end{macrocode}
%    \end{macro}
%    \begin{macro}{\BKM@setupnext}
%    \begin{macrocode}
%    \end{macrocode}
%    \end{macro}
%    \begin{macro}{\BKM@HookNextClear}
%    \begin{macrocode}
\def\BKM@HookNextClear{%
  \global\let\BKM@HookNext\ltx@empty
}
%    \end{macrocode}
%    \end{macro}
%    \begin{macro}{\BKM@HookNext}
%    \begin{macrocode}
\BKM@HookNextClear
%    \end{macrocode}
%    \end{macro}
%
%    \begin{macrocode}
\DeclareBoolOption{draft}
\DeclareComplementaryOption{final}{draft}
%    \end{macrocode}
%    \begin{macro}{\BKM@DisableOptions}
%    \begin{macrocode}
\def\BKM@DisableOptions{%
  \DisableKeyvalOption[action=warning,package=bookmark]%
      {BKM}{draft}%
  \DisableKeyvalOption[action=warning,package=bookmark]%
      {BKM}{final}%
}
%    \end{macrocode}
%    \end{macro}
%    \begin{macrocode}
\DeclareBoolOption[\ifHy@bookmarksopen true\else false\fi]{open}
%    \end{macrocode}
%    \begin{macro}{\bookmark@open}
%    \begin{macrocode}
\def\bookmark@open{%
  \ifBKM@open\ltx@one\else\ltx@zero\fi
}
%    \end{macrocode}
%    \end{macro}
%    \begin{macrocode}
\DeclareStringOption[\maxdimen]{openlevel}
%    \end{macrocode}
%    \begin{macro}{\BKM@openlevel}
%    \begin{macrocode}
\edef\BKM@openlevel{\number\@bookmarksopenlevel}
%    \end{macrocode}
%    \end{macro}
%    \begin{macrocode}
%\DeclareStringOption[\c@tocdepth]{depth}
\ltx@IfUndefined{Hy@bookmarksdepth}{%
  \def\BKM@depth{\c@tocdepth}%
}{%
  \let\BKM@depth\Hy@bookmarksdepth
}
\define@key{BKM}{depth}[]{%
  \edef\BKM@param{#1}%
  \ifx\BKM@param\@empty
    \def\BKM@depth{\c@tocdepth}%
  \else
    \ltx@IfUndefined{toclevel@\BKM@param}{%
      \@onelevel@sanitize\BKM@param
      \edef\BKM@temp{\expandafter\@car\BKM@param\@nil}%
      \ifcase 0\expandafter\ifx\BKM@temp-1\fi
              \expandafter\ifnum\expandafter`\BKM@temp>47 %
                \expandafter\ifnum\expandafter`\BKM@temp<58 %
                  1%
                \fi
              \fi
              \relax
        \PackageWarning{bookmark}{%
          Unknown document division name (\BKM@param)\MessageBreak
          for option `depth'%
        }%
      \else
        \BKM@SetDepthOrLevel\BKM@depth\BKM@param
      \fi
    }{%
      \BKM@SetDepthOrLevel\BKM@depth{%
        \csname toclevel@\BKM@param\endcsname
      }%
    }%
  \fi
}
%    \end{macrocode}
%    \begin{macro}{\bookmark@depth}
%    \begin{macrocode}
\def\bookmark@depth{\BKM@depth}
%    \end{macrocode}
%    \end{macro}
%    \begin{macro}{\BKM@SetDepthOrLevel}
%    \begin{macrocode}
\def\BKM@SetDepthOrLevel#1#2{%
  \begingroup
    \setbox\z@=\hbox{%
      \count@=#2\relax
      \expandafter
    }%
  \expandafter\endgroup
  \expandafter\def\expandafter#1\expandafter{\the\count@}%
}
%    \end{macrocode}
%    \end{macro}
%    \begin{macrocode}
\DeclareStringOption[\BKM@currentlevel]{level}[\BKM@currentlevel]
\define@key{BKM}{level}{%
  \edef\BKM@param{#1}%
  \ifx\BKM@param\BKM@MacroCurrentLevel
    \let\BKM@level\BKM@param
  \else
    \ltx@IfUndefined{toclevel@\BKM@param}{%
      \@onelevel@sanitize\BKM@param
      \edef\BKM@temp{\expandafter\@car\BKM@param\@nil}%
      \ifcase 0\expandafter\ifx\BKM@temp-1\fi
              \expandafter\ifnum\expandafter`\BKM@temp>47 %
                \expandafter\ifnum\expandafter`\BKM@temp<58 %
                  1%
                \fi
              \fi
              \relax
        \PackageWarning{bookmark}{%
          Unknown document division name (\BKM@param)\MessageBreak
          for option `level'%
        }%
      \else
        \BKM@SetDepthOrLevel\BKM@level\BKM@param
      \fi
    }{%
      \BKM@SetDepthOrLevel\BKM@level{%
        \csname toclevel@\BKM@param\endcsname
      }%
    }%
  \fi
}
%    \end{macrocode}
%    \begin{macro}{\BKM@MacroCurrentLevel}
%    \begin{macrocode}
\def\BKM@MacroCurrentLevel{\BKM@currentlevel}
%    \end{macrocode}
%    \end{macro}
%    \begin{macrocode}
\DeclareBoolOption{keeplevel}
\DeclareBoolOption{startatroot}
%    \end{macrocode}
%    \begin{macro}{\BKM@startatrootfalse}
%    \begin{macrocode}
\def\BKM@startatrootfalse{%
  \global\let\ifBKM@startatroot\iffalse
}
%    \end{macrocode}
%    \end{macro}
%    \begin{macro}{\BKM@startatroottrue}
%    \begin{macrocode}
\def\BKM@startatroottrue{%
  \global\let\ifBKM@startatroot\iftrue
}
%    \end{macrocode}
%    \end{macro}
%    \begin{macrocode}
\define@key{BKM}{rellevel}{%
  \BKM@CalcExpr\BKM@level{#1}+\BKM@currentlevel
}
%    \end{macrocode}
%    \begin{macro}{\bookmark@level}
%    \begin{macrocode}
\def\bookmark@level{\BKM@level}
%    \end{macrocode}
%    \end{macro}
%    \begin{macro}{\BKM@currentlevel}
%    \begin{macrocode}
\def\BKM@currentlevel{0}
%    \end{macrocode}
%    \end{macro}
%    Make \xpackage{bookmark}'s option \xoption{numbered} an alias
%    for \xpackage{hyperref}'s \xoption{bookmarksnumbered}.
%    \begin{macrocode}
\DeclareBoolOption[%
  \ifHy@bookmarksnumbered true\else false\fi
]{numbered}
\g@addto@macro\BKM@numberedtrue{%
  \let\ifHy@bookmarksnumbered\iftrue
}
\g@addto@macro\BKM@numberedfalse{%
  \let\ifHy@bookmarksnumbered\iffalse
}
\g@addto@macro\Hy@bookmarksnumberedtrue{%
  \let\ifBKM@numbered\iftrue
}
\g@addto@macro\Hy@bookmarksnumberedfalse{%
  \let\ifBKM@numbered\iffalse
}
%    \end{macrocode}
%    \begin{macro}{\bookmark@numbered}
%    \begin{macrocode}
\def\bookmark@numbered{%
  \ifBKM@numbered\ltx@one\else\ltx@zero\fi
}
%    \end{macrocode}
%    \end{macro}
%
% \paragraph{重定义 \xpackage{hyperref}\ 宏包的选项}
%
%    \begin{macro}{\BKM@PatchHyperrefOption}
%    \begin{macrocode}
\def\BKM@PatchHyperrefOption#1{%
  \expandafter\BKM@@PatchHyperrefOption\csname KV@Hyp@#1\endcsname%
}
%    \end{macrocode}
%    \end{macro}
%    \begin{macro}{\BKM@@PatchHyperrefOption}
%    \begin{macrocode}
\def\BKM@@PatchHyperrefOption#1{%
  \expandafter\BKM@@@PatchHyperrefOption#1{##1}\BKM@nil#1%
}
%    \end{macrocode}
%    \end{macro}
%    \begin{macro}{\BKM@@@PatchHyperrefOption}
%    \begin{macrocode}
\def\BKM@@@PatchHyperrefOption#1\BKM@nil#2#3{%
  \def#2##1{%
    #1%
    \bookmarksetup{#3={##1}}%
  }%
}
%    \end{macrocode}
%    \end{macro}
%    \begin{macrocode}
\BKM@PatchHyperrefOption{bookmarksopen}{open}
\BKM@PatchHyperrefOption{bookmarksopenlevel}{openlevel}
\BKM@PatchHyperrefOption{bookmarksdepth}{depth}
%    \end{macrocode}
%
% \paragraph{字体样式(font style)选项。}
%
%    注意:\xpackage{bitset}\ 宏是基于零的,PDF 规范(PDF specifications)以1开头。
%    \begin{macrocode}
\bitsetReset{BKM@FontStyle}%
\define@key{BKM}{italic}[true]{%
  \expandafter\ifx\csname if#1\endcsname\iftrue
    \bitsetSet{BKM@FontStyle}{0}%
  \else
    \bitsetClear{BKM@FontStyle}{0}%
  \fi
}%
\define@key{BKM}{bold}[true]{%
  \expandafter\ifx\csname if#1\endcsname\iftrue
    \bitsetSet{BKM@FontStyle}{1}%
  \else
    \bitsetClear{BKM@FontStyle}{1}%
  \fi
}%
%    \end{macrocode}
%    \begin{macro}{\bookmark@italic}
%    \begin{macrocode}
\def\bookmark@italic{%
  \ifnum\bitsetGet{BKM@FontStyle}{0}=1 \ltx@one\else\ltx@zero\fi
}
%    \end{macrocode}
%    \end{macro}
%    \begin{macro}{\bookmark@bold}
%    \begin{macrocode}
\def\bookmark@bold{%
  \ifnum\bitsetGet{BKM@FontStyle}{1}=1 \ltx@one\else\ltx@zero\fi
}
%    \end{macrocode}
%    \end{macro}
%    \begin{macro}{\BKM@PrintStyle}
%    \begin{macrocode}
\def\BKM@PrintStyle{%
  \bitsetGetDec{BKM@FontStyle}%
}%
%    \end{macrocode}
%    \end{macro}
%
% \paragraph{颜色(color)选项。}
%
%    \begin{macrocode}
\define@key{BKM}{color}{%
  \HyColor@BookmarkColor{#1}\BKM@color{bookmark}{color}%
}
%    \end{macrocode}
%    \begin{macro}{\BKM@color}
%    \begin{macrocode}
\let\BKM@color\@empty
%    \end{macrocode}
%    \end{macro}
%    \begin{macro}{\bookmark@color}
%    \begin{macrocode}
\def\bookmark@color{\BKM@color}
%    \end{macrocode}
%    \end{macro}
%
% \subsubsection{动作(action)选项}
%
%    \begin{macrocode}
\def\BKM@temp#1{%
  \DeclareStringOption{#1}%
  \expandafter\edef\csname bookmark@#1\endcsname{%
    \expandafter\noexpand\csname BKM@#1\endcsname
  }%
}
%    \end{macrocode}
%    \begin{macro}{\bookmark@dest}
%    \begin{macrocode}
\BKM@temp{dest}
%    \end{macrocode}
%    \end{macro}
%    \begin{macro}{\bookmark@named}
%    \begin{macrocode}
\BKM@temp{named}
%    \end{macrocode}
%    \end{macro}
%    \begin{macro}{\bookmark@uri}
%    \begin{macrocode}
\BKM@temp{uri}
%    \end{macrocode}
%    \end{macro}
%    \begin{macro}{\bookmark@gotor}
%    \begin{macrocode}
\BKM@temp{gotor}
%    \end{macrocode}
%    \end{macro}
%    \begin{macro}{\bookmark@rawaction}
%    \begin{macrocode}
\BKM@temp{rawaction}
%    \end{macrocode}
%    \end{macro}
%
%    \begin{macrocode}
\define@key{BKM}{page}{%
  \def\BKM@page{#1}%
  \ifx\BKM@page\@empty
  \else
    \edef\BKM@page{\number\BKM@page}%
    \ifnum\BKM@page>\z@
    \else
      \PackageError{bookmark}{Page must be positive}\@ehc
      \def\BKM@page{1}%
    \fi
  \fi
}
%    \end{macrocode}
%    \begin{macro}{\BKM@page}
%    \begin{macrocode}
\let\BKM@page\@empty
%    \end{macrocode}
%    \end{macro}
%    \begin{macro}{\bookmark@page}
%    \begin{macrocode}
\def\bookmark@page{\BKM@@page}
%    \end{macrocode}
%    \end{macro}
%
%    \begin{macrocode}
\define@key{BKM}{view}{%
  \BKM@CheckView{#1}%
}
%    \end{macrocode}
%    \begin{macro}{\BKM@view}
%    \begin{macrocode}
\let\BKM@view\@empty
%    \end{macrocode}
%    \end{macro}
%    \begin{macro}{\bookmark@view}
%    \begin{macrocode}
\def\bookmark@view{\BKM@view}
%    \end{macrocode}
%    \end{macro}
%    \begin{macro}{BKM@CheckView}
%    \begin{macrocode}
\def\BKM@CheckView#1{%
  \BKM@CheckViewType#1 \@nil
}
%    \end{macrocode}
%    \end{macro}
%    \begin{macro}{\BKM@CheckViewType}
%    \begin{macrocode}
\def\BKM@CheckViewType#1 #2\@nil{%
  \def\BKM@type{#1}%
  \@onelevel@sanitize\BKM@type
  \BKM@TestViewType{Fit}{}%
  \BKM@TestViewType{FitB}{}%
  \BKM@TestViewType{FitH}{%
    \BKM@CheckParam#2 \@nil{top}%
  }%
  \BKM@TestViewType{FitBH}{%
    \BKM@CheckParam#2 \@nil{top}%
  }%
  \BKM@TestViewType{FitV}{%
    \BKM@CheckParam#2 \@nil{bottom}%
  }%
  \BKM@TestViewType{FitBV}{%
    \BKM@CheckParam#2 \@nil{bottom}%
  }%
  \BKM@TestViewType{FitR}{%
    \BKM@CheckRect{#2}{ }%
  }%
  \BKM@TestViewType{XYZ}{%
    \BKM@CheckXYZ{#2}{ }%
  }%
  \@car{%
    \PackageError{bookmark}{%
      Unknown view type `\BKM@type',\MessageBreak
      using `FitH' instead%
    }\@ehc
    \def\BKM@view{FitH}%
  }%
  \@nil
}
%    \end{macrocode}
%    \end{macro}
%    \begin{macro}{\BKM@TestViewType}
%    \begin{macrocode}
\def\BKM@TestViewType#1{%
  \def\BKM@temp{#1}%
  \@onelevel@sanitize\BKM@temp
  \ifx\BKM@type\BKM@temp
    \let\BKM@view\BKM@temp
    \expandafter\@car
  \else
    \expandafter\@gobble
  \fi
}
%    \end{macrocode}
%    \end{macro}
%    \begin{macro}{BKM@CheckParam}
%    \begin{macrocode}
\def\BKM@CheckParam#1 #2\@nil#3{%
  \def\BKM@param{#1}%
  \ifx\BKM@param\@empty
    \PackageWarning{bookmark}{%
      Missing parameter (#3) for `\BKM@type',\MessageBreak
      using 0%
    }%
    \def\BKM@param{0}%
  \else
    \BKM@CalcParam
  \fi
  \edef\BKM@view{\BKM@view\space\BKM@param}%
}
%    \end{macrocode}
%    \end{macro}
%    \begin{macro}{BKM@CheckRect}
%    \begin{macrocode}
\def\BKM@CheckRect#1#2{%
  \BKM@@CheckRect#1#2#2#2#2\@nil
}
%    \end{macrocode}
%    \end{macro}
%    \begin{macro}{\BKM@@CheckRect}
%    \begin{macrocode}
\def\BKM@@CheckRect#1 #2 #3 #4 #5\@nil{%
  \def\BKM@temp{0}%
  \def\BKM@param{#1}%
  \ifx\BKM@param\@empty
    \def\BKM@param{0}%
    \def\BKM@temp{1}%
  \else
    \BKM@CalcParam
  \fi
  \edef\BKM@view{\BKM@view\space\BKM@param}%
  \def\BKM@param{#2}%
  \ifx\BKM@param\@empty
    \def\BKM@param{0}%
    \def\BKM@temp{1}%
  \else
    \BKM@CalcParam
  \fi
  \edef\BKM@view{\BKM@view\space\BKM@param}%
  \def\BKM@param{#3}%
  \ifx\BKM@param\@empty
    \def\BKM@param{0}%
    \def\BKM@temp{1}%
  \else
    \BKM@CalcParam
  \fi
  \edef\BKM@view{\BKM@view\space\BKM@param}%
  \def\BKM@param{#4}%
  \ifx\BKM@param\@empty
    \def\BKM@param{0}%
    \def\BKM@temp{1}%
  \else
    \BKM@CalcParam
  \fi
  \edef\BKM@view{\BKM@view\space\BKM@param}%
  \ifnum\BKM@temp>\z@
    \PackageWarning{bookmark}{Missing parameters for `\BKM@type'}%
  \fi
}
%    \end{macrocode}
%    \end{macro}
%    \begin{macro}{\BKM@CheckXYZ}
%    \begin{macrocode}
\def\BKM@CheckXYZ#1#2{%
  \BKM@@CheckXYZ#1#2#2#2\@nil
}
%    \end{macrocode}
%    \end{macro}
%    \begin{macro}{\BKM@@CheckXYZ}
%    \begin{macrocode}
\def\BKM@@CheckXYZ#1 #2 #3 #4\@nil{%
  \def\BKM@param{#1}%
  \let\BKM@temp\BKM@param
  \@onelevel@sanitize\BKM@temp
  \ifx\BKM@param\@empty
    \let\BKM@param\BKM@null
  \else
    \ifx\BKM@temp\BKM@null
    \else
      \BKM@CalcParam
    \fi
  \fi
  \edef\BKM@view{\BKM@view\space\BKM@param}%
  \def\BKM@param{#2}%
  \let\BKM@temp\BKM@param
  \@onelevel@sanitize\BKM@temp
  \ifx\BKM@param\@empty
    \let\BKM@param\BKM@null
  \else
    \ifx\BKM@temp\BKM@null
    \else
      \BKM@CalcParam
    \fi
  \fi
  \edef\BKM@view{\BKM@view\space\BKM@param}%
  \def\BKM@param{#3}%
  \ifx\BKM@param\@empty
    \let\BKM@param\BKM@null
  \fi
  \edef\BKM@view{\BKM@view\space\BKM@param}%
}
%    \end{macrocode}
%    \end{macro}
%    \begin{macro}{\BKM@null}
%    \begin{macrocode}
\def\BKM@null{null}
\@onelevel@sanitize\BKM@null
%    \end{macrocode}
%    \end{macro}
%
%    \begin{macro}{\BKM@CalcParam}
%    \begin{macrocode}
\def\BKM@CalcParam{%
  \begingroup
  \let\calc\@firstofone
  \expandafter\BKM@@CalcParam\BKM@param\@empty\@empty\@nil
}
%    \end{macrocode}
%    \end{macro}
%    \begin{macro}{\BKM@@CalcParam}
%    \begin{macrocode}
\def\BKM@@CalcParam#1#2#3\@nil{%
  \ifx\calc#1%
    \@ifundefined{calc@assign@dimen}{%
      \@ifundefined{dimexpr}{%
        \setlength{\dimen@}{#2}%
      }{%
        \setlength{\dimen@}{\dimexpr#2\relax}%
      }%
    }{%
      \setlength{\dimen@}{#2}%
    }%
    \dimen@.99626\dimen@
    \edef\BKM@param{\strip@pt\dimen@}%
    \expandafter\endgroup
    \expandafter\def\expandafter\BKM@param\expandafter{\BKM@param}%
  \else
    \endgroup
  \fi
}
%    \end{macrocode}
%    \end{macro}
%
% \subsubsection{\xoption{atend}\ 选项}
%
%    \begin{macrocode}
\DeclareBoolOption{atend}
\g@addto@macro\BKM@DisableOptions{%
  \DisableKeyvalOption[action=warning,package=bookmark]%
      {BKM}{atend}%
}
%    \end{macrocode}
%
% \subsubsection{\xoption{style}\ 选项}
%
%    \begin{macro}{\bookmarkdefinestyle}
%    \begin{macrocode}
\newcommand*{\bookmarkdefinestyle}[2]{%
  \@ifundefined{BKM@style@#1}{%
  }{%
    \PackageInfo{bookmark}{Redefining style `#1'}%
  }%
  \@namedef{BKM@style@#1}{#2}%
}
%    \end{macrocode}
%    \end{macro}
%    \begin{macrocode}
\define@key{BKM}{style}{%
  \BKM@StyleCall{#1}%
}
\newif\ifBKM@ok
%    \end{macrocode}
%    \begin{macro}{\BKM@StyleCall}
%    \begin{macrocode}
\def\BKM@StyleCall#1{%
  \@ifundefined{BKM@style@#1}{%
    \PackageWarning{bookmark}{%
      Ignoring unknown style `#1'%
    }%
  }{%
%    \end{macrocode}
%    检查样式堆栈(style stack)。
%    \begin{macrocode}
    \BKM@oktrue
    \edef\BKM@StyleCurrent{#1}%
    \@onelevel@sanitize\BKM@StyleCurrent
    \let\BKM@StyleEntry\BKM@StyleEntryCheck
    \BKM@StyleStack
    \ifBKM@ok
      \expandafter\@firstofone
    \else
      \PackageError{bookmark}{%
        Ignoring recursive call of style `\BKM@StyleCurrent'%
      }\@ehc
      \expandafter\@gobble
    \fi
    {%
%    \end{macrocode}
%    在堆栈上推送当前样式(Push current style on stack)。
%    \begin{macrocode}
      \let\BKM@StyleEntry\relax
      \edef\BKM@StyleStack{%
        \BKM@StyleEntry{\BKM@StyleCurrent}%
        \BKM@StyleStack
      }%
%    \end{macrocode}
%   调用样式(Call style)。
%    \begin{macrocode}
      \expandafter\expandafter\expandafter\bookmarksetup
      \expandafter\expandafter\expandafter{%
        \csname BKM@style@\BKM@StyleCurrent\endcsname
      }%
%    \end{macrocode}
%    从堆栈中弹出当前样式(Pop current style from stack)。
%    \begin{macrocode}
      \BKM@StyleStackPop
    }%
  }%
}
%    \end{macrocode}
%    \end{macro}
%    \begin{macro}{\BKM@StyleStackPop}
%    \begin{macrocode}
\def\BKM@StyleStackPop{%
  \let\BKM@StyleEntry\relax
  \edef\BKM@StyleStack{%
    \expandafter\@gobbletwo\BKM@StyleStack
  }%
}
%    \end{macrocode}
%    \end{macro}
%    \begin{macro}{\BKM@StyleEntryCheck}
%    \begin{macrocode}
\def\BKM@StyleEntryCheck#1{%
  \def\BKM@temp{#1}%
  \ifx\BKM@temp\BKM@StyleCurrent
    \BKM@okfalse
  \fi
}
%    \end{macrocode}
%    \end{macro}
%    \begin{macro}{\BKM@StyleStack}
%    \begin{macrocode}
\def\BKM@StyleStack{}
%    \end{macrocode}
%    \end{macro}
%
% \subsubsection{源文件位置(source file location)选项}
%
%    \begin{macrocode}
\DeclareStringOption{srcline}
\DeclareStringOption{srcfile}
%    \end{macrocode}
%
% \subsubsection{钩子支持(Hook support)}
%
%    \begin{macro}{\BKM@hook}
%    \begin{macrocode}
\def\BKM@hook{}
%    \end{macrocode}
%    \end{macro}
%    \begin{macrocode}
\define@key{BKM}{addtohook}{%
  \ltx@LocalAppendToMacro\BKM@hook{#1}%
}
%    \end{macrocode}
%
%    \begin{macro}{bookmarkget}
%    \begin{macrocode}
\newcommand*{\bookmarkget}[1]{%
  \romannumeral0%
  \ltx@ifundefined{bookmark@#1}{%
    \ltx@space
  }{%
    \expandafter\expandafter\expandafter\ltx@space
    \csname bookmark@#1\endcsname
  }%
}
%    \end{macrocode}
%    \end{macro}
%
% \subsubsection{设置和加载驱动程序}
%
% \paragraph{检测驱动程序。}
%
%    \begin{macro}{\BKM@DefineDriverKey}
%    \begin{macrocode}
\def\BKM@DefineDriverKey#1{%
  \define@key{BKM}{#1}[]{%
    \def\BKM@driver{#1}%
  }%
  \g@addto@macro\BKM@DisableOptions{%
    \DisableKeyvalOption[action=warning,package=bookmark]%
        {BKM}{#1}%
  }%
}
%    \end{macrocode}
%    \end{macro}
%    \begin{macrocode}
\BKM@DefineDriverKey{pdftex}
\BKM@DefineDriverKey{dvips}
\BKM@DefineDriverKey{dvipdfm}
\BKM@DefineDriverKey{dvipdfmx}
\BKM@DefineDriverKey{xetex}
\BKM@DefineDriverKey{vtex}
\define@key{BKM}{dvipdfmx-outline-open}[true]{%
 \PackageWarning{bookmark}{Option 'dvipdfmx-outline-open' is obsolete
   and ignored}{}}
%    \end{macrocode}
%    \begin{macro}{\bookmark@driver}
%    \begin{macrocode}
\def\bookmark@driver{\BKM@driver}
%    \end{macrocode}
%    \end{macro}
%    \begin{macrocode}
\InputIfFileExists{bookmark.cfg}{}{}
%    \end{macrocode}
%    \begin{macro}{\BookmarkDriverDefault}
%    \begin{macrocode}
\providecommand*{\BookmarkDriverDefault}{dvips}
%    \end{macrocode}
%    \end{macro}
%    \begin{macro}{\BKM@driver}
% Lua\TeX\ 和 pdf\TeX\ 共享驱动程序。
%    \begin{macrocode}
\ifpdf
  \def\BKM@driver{pdftex}%
  \ifx\pdfoutline\@undefined
    \ifx\pdfextension\@undefined\else
      \protected\def\pdfoutline{\pdfextension outline }
    \fi
  \fi
\else
  \ifxetex
    \def\BKM@driver{dvipdfm}%
  \else
    \ifvtex
      \def\BKM@driver{vtex}%
    \else
      \edef\BKM@driver{\BookmarkDriverDefault}%
    \fi
  \fi
\fi
%    \end{macrocode}
%    \end{macro}
%
% \paragraph{过程选项(Process options)。}
%
%    \begin{macrocode}
\ProcessKeyvalOptions*
\BKM@DisableOptions
%    \end{macrocode}
%
% \paragraph{\xoption{draft}\ 选项}
%
%    \begin{macrocode}
\ifBKM@draft
  \PackageWarningNoLine{bookmark}{Draft mode on}%
  \let\bookmarksetup\ltx@gobble
  \let\BookmarkAtEnd\ltx@gobble
  \let\bookmarkdefinestyle\ltx@gobbletwo
  \let\bookmarkget\ltx@gobble
  \let\pdfbookmark\ltx@undefined
  \newcommand*{\pdfbookmark}[3][]{}%
  \let\currentpdfbookmark\ltx@gobbletwo
  \let\subpdfbookmark\ltx@gobbletwo
  \let\belowpdfbookmark\ltx@gobbletwo
  \newcommand*{\bookmark}[2][]{}%
  \renewcommand*{\Hy@writebookmark}[5]{}%
  \let\ReadBookmarks\relax
  \let\BKM@DefGotoNameAction\ltx@gobbletwo % package `hypdestopt'
  \expandafter\endinput
\fi
%    \end{macrocode}
%
% \paragraph{验证和加载驱动程序。}
%
%    \begin{macrocode}
\def\BKM@temp{dvipdfmx}%
\ifx\BKM@temp\BKM@driver
  \def\BKM@driver{dvipdfm}%
\fi
\def\BKM@temp{pdftex}%
\ifpdf
  \ifx\BKM@temp\BKM@driver
  \else
    \PackageWarningNoLine{bookmark}{%
      Wrong driver `\BKM@driver', using `pdftex' instead%
    }%
    \let\BKM@driver\BKM@temp
  \fi
\else
  \ifx\BKM@temp\BKM@driver
    \PackageError{bookmark}{%
      Wrong driver, pdfTeX is not running in PDF mode.\MessageBreak
      Package loading is aborted%
    }\@ehc
    \expandafter\expandafter\expandafter\endinput
  \fi
  \def\BKM@temp{dvipdfm}%
  \ifxetex
    \ifx\BKM@temp\BKM@driver
    \else
      \PackageWarningNoLine{bookmark}{%
        Wrong driver `\BKM@driver',\MessageBreak
        using `dvipdfm' for XeTeX instead%
      }%
      \let\BKM@driver\BKM@temp
    \fi
  \else
    \def\BKM@temp{vtex}%
    \ifvtex
      \ifx\BKM@temp\BKM@driver
      \else
        \PackageWarningNoLine{bookmark}{%
          Wrong driver `\BKM@driver',\MessageBreak
          using `vtex' for VTeX instead%
        }%
        \let\BKM@driver\BKM@temp
      \fi
    \else
      \ifx\BKM@temp\BKM@driver
        \PackageError{bookmark}{%
          Wrong driver, VTeX is not running in PDF mode.\MessageBreak
          Package loading is aborted%
        }\@ehc
        \expandafter\expandafter\expandafter\endinput
      \fi
    \fi
  \fi
\fi
\providecommand\IfFormatAtLeastTF{\@ifl@t@r\fmtversion}
\IfFormatAtLeastTF{2020/10/01}{}{\edef\BKM@driver{\BKM@driver-2019-12-03}}
\InputIfFileExists{bkm-\BKM@driver.def}{}{%
  \PackageError{bookmark}{%
    Unsupported driver `\BKM@driver'.\MessageBreak
    Package loading is aborted%
  }\@ehc
  \endinput
}
%    \end{macrocode}
%
% \subsubsection{与 \xpackage{hyperref}\ 的兼容性}
%
%    \begin{macro}{\pdfbookmark}
%    \begin{macrocode}
\let\pdfbookmark\ltx@undefined
\newcommand*{\pdfbookmark}[3][0]{%
  \bookmark[level=#1,dest={#3.#1}]{#2}%
  \hyper@anchorstart{#3.#1}\hyper@anchorend
}
%    \end{macrocode}
%    \end{macro}
%    \begin{macro}{\currentpdfbookmark}
%    \begin{macrocode}
\def\currentpdfbookmark{%
  \pdfbookmark[\BKM@currentlevel]%
}
%    \end{macrocode}
%    \end{macro}
%    \begin{macro}{\subpdfbookmark}
%    \begin{macrocode}
\def\subpdfbookmark{%
  \BKM@CalcExpr\BKM@CalcResult\BKM@currentlevel+1%
  \expandafter\pdfbookmark\expandafter[\BKM@CalcResult]%
}
%    \end{macrocode}
%    \end{macro}
%    \begin{macro}{\belowpdfbookmark}
%    \begin{macrocode}
\def\belowpdfbookmark#1#2{%
  \xdef\BKM@gtemp{\number\BKM@currentlevel}%
  \subpdfbookmark{#1}{#2}%
  \global\let\BKM@currentlevel\BKM@gtemp
}
%    \end{macrocode}
%    \end{macro}
%
%    节号(section number)、文本(text)、标签(label)、级别(level)、文件(file)
%    \begin{macro}{\Hy@writebookmark}
%    \begin{macrocode}
\def\Hy@writebookmark#1#2#3#4#5{%
  \ifnum#4>\BKM@depth\relax
  \else
    \def\BKM@type{#5}%
    \ifx\BKM@type\Hy@bookmarkstype
      \begingroup
        \ifBKM@numbered
          \let\numberline\Hy@numberline
          \let\booknumberline\Hy@numberline
          \let\partnumberline\Hy@numberline
          \let\chapternumberline\Hy@numberline
        \else
          \let\numberline\@gobble
          \let\booknumberline\@gobble
          \let\partnumberline\@gobble
          \let\chapternumberline\@gobble
        \fi
        \bookmark[level=#4,dest={\HyperDestNameFilter{#3}}]{#2}%
      \endgroup
    \fi
  \fi
}
%    \end{macrocode}
%    \end{macro}
%
%    \begin{macro}{\ReadBookmarks}
%    \begin{macrocode}
\let\ReadBookmarks\relax
%    \end{macrocode}
%    \end{macro}
%
%    \begin{macrocode}
%</package>
%    \end{macrocode}
%
% \subsection{dvipdfm 的驱动程序}
%
%    \begin{macrocode}
%<*dvipdfm>
\NeedsTeXFormat{LaTeX2e}
\ProvidesFile{bkm-dvipdfm.def}%
  [2020-11-06 v1.29 bookmark driver for dvipdfm (HO)]%
%    \end{macrocode}
%
%    \begin{macro}{\BKM@id}
%    \begin{macrocode}
\newcount\BKM@id
\BKM@id=\z@
%    \end{macrocode}
%    \end{macro}
%
%    \begin{macro}{\BKM@0}
%    \begin{macrocode}
\@namedef{BKM@0}{000}
%    \end{macrocode}
%    \end{macro}
%    \begin{macro}{\ifBKM@sw}
%    \begin{macrocode}
\newif\ifBKM@sw
%    \end{macrocode}
%    \end{macro}
%
%    \begin{macro}{\bookmark}
%    \begin{macrocode}
\newcommand*{\bookmark}[2][]{%
  \if@filesw
    \begingroup
      \def\bookmark@text{#2}%
      \BKM@setup{#1}%
      \edef\BKM@prev{\the\BKM@id}%
      \global\advance\BKM@id\@ne
      \BKM@swtrue
      \@whilesw\ifBKM@sw\fi{%
        \def\BKM@abslevel{1}%
        \ifnum\ifBKM@startatroot\z@\else\BKM@prev\fi=\z@
          \BKM@startatrootfalse
          \expandafter\xdef\csname BKM@\the\BKM@id\endcsname{%
            0{\BKM@level}\BKM@abslevel
          }%
          \BKM@swfalse
        \else
          \expandafter\expandafter\expandafter\BKM@getx
              \csname BKM@\BKM@prev\endcsname
          \ifnum\BKM@level>\BKM@x@level\relax
            \BKM@CalcExpr\BKM@abslevel\BKM@x@abslevel+1%
            \expandafter\xdef\csname BKM@\the\BKM@id\endcsname{%
              {\BKM@prev}{\BKM@level}\BKM@abslevel
            }%
            \BKM@swfalse
          \else
            \let\BKM@prev\BKM@x@parent
          \fi
        \fi
      }%
      \csname HyPsd@XeTeXBigCharstrue\endcsname
      \pdfstringdef\BKM@title{\bookmark@text}%
      \edef\BKM@FLAGS{\BKM@PrintStyle}%
      \let\BKM@action\@empty
      \ifx\BKM@gotor\@empty
        \ifx\BKM@dest\@empty
          \ifx\BKM@named\@empty
            \ifx\BKM@rawaction\@empty
              \ifx\BKM@uri\@empty
                \ifx\BKM@page\@empty
                  \PackageError{bookmark}{Missing action}\@ehc
                  \edef\BKM@action{/Dest[@page1/Fit]}%
                \else
                  \ifx\BKM@view\@empty
                    \def\BKM@view{Fit}%
                  \fi
                  \edef\BKM@action{/Dest[@page\BKM@page/\BKM@view]}%
                \fi
              \else
                \BKM@EscapeString\BKM@uri
                \edef\BKM@action{%
                  /A<<%
                    /S/URI%
                    /URI(\BKM@uri)%
                  >>%
                }%
              \fi
            \else
              \edef\BKM@action{/A<<\BKM@rawaction>>}%
            \fi
          \else
            \BKM@EscapeName\BKM@named
            \edef\BKM@action{%
              /A<</S/Named/N/\BKM@named>>%
            }%
          \fi
        \else
          \BKM@EscapeString\BKM@dest
          \edef\BKM@action{%
            /A<<%
              /S/GoTo%
              /D(\BKM@dest)%
            >>%
          }%
        \fi
      \else
        \ifx\BKM@dest\@empty
          \ifx\BKM@page\@empty
            \def\BKM@page{0}%
          \else
            \BKM@CalcExpr\BKM@page\BKM@page-1%
          \fi
          \ifx\BKM@view\@empty
            \def\BKM@view{Fit}%
          \fi
          \edef\BKM@action{/D[\BKM@page/\BKM@view]}%
        \else
          \BKM@EscapeString\BKM@dest
          \edef\BKM@action{/D(\BKM@dest)}%
        \fi
        \BKM@EscapeString\BKM@gotor
        \edef\BKM@action{%
          /A<<%
            /S/GoToR%
            /F(\BKM@gotor)%
            \BKM@action
          >>%
        }%
      \fi
      \special{pdf:%
        out
              [%
              \ifBKM@open
                \ifnum\BKM@level<%
                    \expandafter\ltx@firstofone\expandafter
                    {\number\BKM@openlevel} %
                \else
                  -%
                \fi
              \else
                -%
              \fi
              ] %
            \BKM@abslevel
        <<%
          /Title(\BKM@title)%
          \ifx\BKM@color\@empty
          \else
            /C[\BKM@color]%
          \fi
          \ifnum\BKM@FLAGS>\z@
            /F \BKM@FLAGS
          \fi
          \BKM@action
        >>%
      }%
    \endgroup
  \fi
}
%    \end{macrocode}
%    \end{macro}
%    \begin{macro}{\BKM@getx}
%    \begin{macrocode}
\def\BKM@getx#1#2#3{%
  \def\BKM@x@parent{#1}%
  \def\BKM@x@level{#2}%
  \def\BKM@x@abslevel{#3}%
}
%    \end{macrocode}
%    \end{macro}
%
%    \begin{macrocode}
%</dvipdfm>
%    \end{macrocode}
%
% \subsection{\hologo{VTeX}\ 的驱动程序}
%
%    \begin{macrocode}
%<*vtex>
\NeedsTeXFormat{LaTeX2e}
\ProvidesFile{bkm-vtex.def}%
  [2020-11-06 v1.29 bookmark driver for VTeX (HO)]%
%    \end{macrocode}
%
%    \begin{macrocode}
\ifvtexpdf
\else
  \PackageWarningNoLine{bookmark}{%
    The VTeX driver only supports PDF mode%
  }%
\fi
%    \end{macrocode}
%
%    \begin{macro}{\BKM@id}
%    \begin{macrocode}
\newcount\BKM@id
\BKM@id=\z@
%    \end{macrocode}
%    \end{macro}
%
%    \begin{macro}{\BKM@0}
%    \begin{macrocode}
\@namedef{BKM@0}{00}
%    \end{macrocode}
%    \end{macro}
%    \begin{macro}{\ifBKM@sw}
%    \begin{macrocode}
\newif\ifBKM@sw
%    \end{macrocode}
%    \end{macro}
%
%    \begin{macro}{\bookmark}
%    \begin{macrocode}
\newcommand*{\bookmark}[2][]{%
  \if@filesw
    \begingroup
      \def\bookmark@text{#2}%
      \BKM@setup{#1}%
      \edef\BKM@prev{\the\BKM@id}%
      \global\advance\BKM@id\@ne
      \BKM@swtrue
      \@whilesw\ifBKM@sw\fi{%
        \ifnum\ifBKM@startatroot\z@\else\BKM@prev\fi=\z@
          \BKM@startatrootfalse
          \def\BKM@parent{0}%
          \expandafter\xdef\csname BKM@\the\BKM@id\endcsname{%
            0{\BKM@level}%
          }%
          \BKM@swfalse
        \else
          \expandafter\expandafter\expandafter\BKM@getx
              \csname BKM@\BKM@prev\endcsname
          \ifnum\BKM@level>\BKM@x@level\relax
            \let\BKM@parent\BKM@prev
            \expandafter\xdef\csname BKM@\the\BKM@id\endcsname{%
              {\BKM@prev}{\BKM@level}%
            }%
            \BKM@swfalse
          \else
            \let\BKM@prev\BKM@x@parent
          \fi
        \fi
      }%
      \pdfstringdef\BKM@title{\bookmark@text}%
      \BKM@vtex@title
      \edef\BKM@FLAGS{\BKM@PrintStyle}%
      \let\BKM@action\@empty
      \ifx\BKM@gotor\@empty
        \ifx\BKM@dest\@empty
          \ifx\BKM@named\@empty
            \ifx\BKM@rawaction\@empty
              \ifx\BKM@uri\@empty
                \ifx\BKM@page\@empty
                  \PackageError{bookmark}{Missing action}\@ehc
                  \def\BKM@action{!1}%
                \else
                  \edef\BKM@action{!\BKM@page}%
                \fi
              \else
                \BKM@EscapeString\BKM@uri
                \edef\BKM@action{%
                  <u=%
                    /S/URI%
                    /URI(\BKM@uri)%
                  >%
                }%
              \fi
            \else
              \edef\BKM@action{<u=\BKM@rawaction>}%
            \fi
          \else
            \BKM@EscapeName\BKM@named
            \edef\BKM@action{%
              <u=%
                /S/Named%
                /N/\BKM@named
              >%
            }%
          \fi
        \else
          \BKM@EscapeString\BKM@dest
          \edef\BKM@action{\BKM@dest}%
        \fi
      \else
        \ifx\BKM@dest\@empty
          \ifx\BKM@page\@empty
            \def\BKM@page{1}%
          \fi
          \ifx\BKM@view\@empty
            \def\BKM@view{Fit}%
          \fi
          \edef\BKM@action{/D[\BKM@page/\BKM@view]}%
        \else
          \BKM@EscapeString\BKM@dest
          \edef\BKM@action{/D(\BKM@dest)}%
        \fi
        \BKM@EscapeString\BKM@gotor
        \edef\BKM@action{%
          <u=%
            /S/GoToR%
            /F(\BKM@gotor)%
            \BKM@action
          >>%
        }%
      \fi
      \ifx\BKM@color\@empty
        \let\BKM@RGBcolor\@empty
      \else
        \expandafter\BKM@toRGB\BKM@color\@nil
      \fi
      \special{%
        !outline \BKM@action;%
        p=\BKM@parent,%
        i=\number\BKM@id,%
        s=%
          \ifBKM@open
            \ifnum\BKM@level<\BKM@openlevel
              o%
            \else
              c%
            \fi
          \else
            c%
          \fi,%
        \ifx\BKM@RGBcolor\@empty
        \else
          c=\BKM@RGBcolor,%
        \fi
        \ifnum\BKM@FLAGS>\z@
          f=\BKM@FLAGS,%
        \fi
        t=\BKM@title
      }%
    \endgroup
  \fi
}
%    \end{macrocode}
%    \end{macro}
%    \begin{macro}{\BKM@getx}
%    \begin{macrocode}
\def\BKM@getx#1#2{%
  \def\BKM@x@parent{#1}%
  \def\BKM@x@level{#2}%
}
%    \end{macrocode}
%    \end{macro}
%    \begin{macro}{\BKM@toRGB}
%    \begin{macrocode}
\def\BKM@toRGB#1 #2 #3\@nil{%
  \let\BKM@RGBcolor\@empty
  \BKM@toRGBComponent{#1}%
  \BKM@toRGBComponent{#2}%
  \BKM@toRGBComponent{#3}%
}
%    \end{macrocode}
%    \end{macro}
%    \begin{macro}{\BKM@toRGBComponent}
%    \begin{macrocode}
\def\BKM@toRGBComponent#1{%
  \dimen@=#1pt\relax
  \ifdim\dimen@>\z@
    \ifdim\dimen@<\p@
      \dimen@=255\dimen@
      \advance\dimen@ by 32768sp\relax
      \divide\dimen@ by 65536\relax
      \dimen@ii=\dimen@
      \divide\dimen@ii by 16\relax
      \edef\BKM@RGBcolor{%
        \BKM@RGBcolor
        \BKM@toHexDigit\dimen@ii
      }%
      \dimen@ii=16\dimen@ii
      \advance\dimen@-\dimen@ii
      \edef\BKM@RGBcolor{%
        \BKM@RGBcolor
        \BKM@toHexDigit\dimen@
      }%
    \else
      \edef\BKM@RGBcolor{\BKM@RGBcolor FF}%
    \fi
  \else
    \edef\BKM@RGBcolor{\BKM@RGBcolor00}%
  \fi
}
%    \end{macrocode}
%    \end{macro}
%    \begin{macro}{\BKM@toHexDigit}
%    \begin{macrocode}
\def\BKM@toHexDigit#1{%
  \ifcase\expandafter\@firstofone\expandafter{\number#1} %
    0\or 1\or 2\or 3\or 4\or 5\or 6\or 7\or
    8\or 9\or A\or B\or C\or D\or E\or F%
  \fi
}
%    \end{macrocode}
%    \end{macro}
%    \begin{macrocode}
\begingroup
  \catcode`\|=0 %
  \catcode`\\=12 %
%    \end{macrocode}
%    \begin{macro}{\BKM@vtex@title}
%    \begin{macrocode}
  |gdef|BKM@vtex@title{%
    |@onelevel@sanitize|BKM@title
    |edef|BKM@title{|expandafter|BKM@vtex@leftparen|BKM@title\(|@nil}%
    |edef|BKM@title{|expandafter|BKM@vtex@rightparen|BKM@title\)|@nil}%
    |edef|BKM@title{|expandafter|BKM@vtex@zero|BKM@title\0|@nil}%
    |edef|BKM@title{|expandafter|BKM@vtex@one|BKM@title\1|@nil}%
    |edef|BKM@title{|expandafter|BKM@vtex@two|BKM@title\2|@nil}%
    |edef|BKM@title{|expandafter|BKM@vtex@three|BKM@title\3|@nil}%
  }%
%    \end{macrocode}
%    \end{macro}
%    \begin{macro}{\BKM@vtex@leftparen}
%    \begin{macrocode}
  |gdef|BKM@vtex@leftparen#1\(#2|@nil{%
    #1%
    |ifx||#2||%
    |else
      (%
      |ltx@ReturnAfterFi{%
        |BKM@vtex@leftparen#2|@nil
      }%
    |fi
  }%
%    \end{macrocode}
%    \end{macro}
%    \begin{macro}{\BKM@vtex@rightparen}
%    \begin{macrocode}
  |gdef|BKM@vtex@rightparen#1\)#2|@nil{%
    #1%
    |ifx||#2||%
    |else
      )%
      |ltx@ReturnAfterFi{%
        |BKM@vtex@rightparen#2|@nil
      }%
    |fi
  }%
%    \end{macrocode}
%    \end{macro}
%    \begin{macro}{\BKM@vtex@zero}
%    \begin{macrocode}
  |gdef|BKM@vtex@zero#1\0#2|@nil{%
    #1%
    |ifx||#2||%
    |else
      |noexpand|hv@pdf@char0%
      |ltx@ReturnAfterFi{%
        |BKM@vtex@zero#2|@nil
      }%
    |fi
  }%
%    \end{macrocode}
%    \end{macro}
%    \begin{macro}{\BKM@vtex@one}
%    \begin{macrocode}
  |gdef|BKM@vtex@one#1\1#2|@nil{%
    #1%
    |ifx||#2||%
    |else
      |noexpand|hv@pdf@char1%
      |ltx@ReturnAfterFi{%
        |BKM@vtex@one#2|@nil
      }%
    |fi
  }%
%    \end{macrocode}
%    \end{macro}
%    \begin{macro}{\BKM@vtex@two}
%    \begin{macrocode}
  |gdef|BKM@vtex@two#1\2#2|@nil{%
    #1%
    |ifx||#2||%
    |else
      |noexpand|hv@pdf@char2%
      |ltx@ReturnAfterFi{%
        |BKM@vtex@two#2|@nil
      }%
    |fi
  }%
%    \end{macrocode}
%    \end{macro}
%    \begin{macro}{\BKM@vtex@three}
%    \begin{macrocode}
  |gdef|BKM@vtex@three#1\3#2|@nil{%
    #1%
    |ifx||#2||%
    |else
      |noexpand|hv@pdf@char3%
      |ltx@ReturnAfterFi{%
        |BKM@vtex@three#2|@nil
      }%
    |fi
  }%
%    \end{macrocode}
%    \end{macro}
%    \begin{macrocode}
|endgroup
%    \end{macrocode}
%
%    \begin{macrocode}
%</vtex>
%    \end{macrocode}
%
% \subsection{\hologo{pdfTeX}\ 的驱动程序}
%
%    \begin{macrocode}
%<*pdftex>
\NeedsTeXFormat{LaTeX2e}
\ProvidesFile{bkm-pdftex.def}%
  [2020-11-06 v1.29 bookmark driver for pdfTeX (HO)]%
%    \end{macrocode}
%
%    \begin{macro}{\BKM@DO@entry}
%    \begin{macrocode}
\def\BKM@DO@entry#1#2{%
  \begingroup
    \kvsetkeys{BKM@DO}{#1}%
    \def\BKM@DO@title{#2}%
    \ifx\BKM@DO@srcfile\@empty
    \else
      \BKM@UnescapeHex\BKM@DO@srcfile
    \fi
    \BKM@UnescapeHex\BKM@DO@title
    \expandafter\expandafter\expandafter\BKM@getx
        \csname BKM@\BKM@DO@id\endcsname\@empty\@empty
    \let\BKM@attr\@empty
    \ifx\BKM@DO@flags\@empty
    \else
      \edef\BKM@attr{\BKM@attr/F \BKM@DO@flags}%
    \fi
    \ifx\BKM@DO@color\@empty
    \else
      \edef\BKM@attr{\BKM@attr/C[\BKM@DO@color]}%
    \fi
    \ifx\BKM@attr\@empty
    \else
      \edef\BKM@attr{attr{\BKM@attr}}%
    \fi
    \let\BKM@action\@empty
    \ifx\BKM@DO@gotor\@empty
      \ifx\BKM@DO@dest\@empty
        \ifx\BKM@DO@named\@empty
          \ifx\BKM@DO@rawaction\@empty
            \ifx\BKM@DO@uri\@empty
              \ifx\BKM@DO@page\@empty
                \PackageError{bookmark}{%
                  Missing action\BKM@SourceLocation
                }\@ehc
                \edef\BKM@action{goto page1{/Fit}}%
              \else
                \ifx\BKM@DO@view\@empty
                  \def\BKM@DO@view{Fit}%
                \fi
                \edef\BKM@action{goto page\BKM@DO@page{/\BKM@DO@view}}%
              \fi
            \else
              \BKM@UnescapeHex\BKM@DO@uri
              \BKM@EscapeString\BKM@DO@uri
              \edef\BKM@action{user{<</S/URI/URI(\BKM@DO@uri)>>}}%
            \fi
          \else
            \BKM@UnescapeHex\BKM@DO@rawaction
            \edef\BKM@action{%
              user{%
                <<%
                  \BKM@DO@rawaction
                >>%
              }%
            }%
          \fi
        \else
          \BKM@EscapeName\BKM@DO@named
          \edef\BKM@action{%
            user{<</S/Named/N/\BKM@DO@named>>}%
          }%
        \fi
      \else
        \BKM@UnescapeHex\BKM@DO@dest
        \BKM@DefGotoNameAction\BKM@action\BKM@DO@dest
      \fi
    \else
      \ifx\BKM@DO@dest\@empty
        \ifx\BKM@DO@page\@empty
          \def\BKM@DO@page{0}%
        \else
          \BKM@CalcExpr\BKM@DO@page\BKM@DO@page-1%
        \fi
        \ifx\BKM@DO@view\@empty
          \def\BKM@DO@view{Fit}%
        \fi
        \edef\BKM@action{/D[\BKM@DO@page/\BKM@DO@view]}%
      \else
        \BKM@UnescapeHex\BKM@DO@dest
        \BKM@EscapeString\BKM@DO@dest
        \edef\BKM@action{/D(\BKM@DO@dest)}%
      \fi
      \BKM@UnescapeHex\BKM@DO@gotor
      \BKM@EscapeString\BKM@DO@gotor
      \edef\BKM@action{%
        user{%
          <<%
            /S/GoToR%
            /F(\BKM@DO@gotor)%
            \BKM@action
          >>%
        }%
      }%
    \fi
    \pdfoutline\BKM@attr\BKM@action
                count\ifBKM@DO@open\else-\fi\BKM@x@childs
                {\BKM@DO@title}%
  \endgroup
}
%    \end{macrocode}
%    \end{macro}
%    \begin{macro}{\BKM@DefGotoNameAction}
%    \cs{BKM@DefGotoNameAction}\ 宏是一个用于 \xpackage{hypdestopt}\ 宏包的钩子(hook)。
%    \begin{macrocode}
\def\BKM@DefGotoNameAction#1#2{%
  \BKM@EscapeString\BKM@DO@dest
  \edef#1{goto name{#2}}%
}
%    \end{macrocode}
%    \end{macro}
%    \begin{macrocode}
%</pdftex>
%    \end{macrocode}
%
%    \begin{macrocode}
%<*pdftex|pdfmark>
%    \end{macrocode}
%    \begin{macro}{\BKM@SourceLocation}
%    \begin{macrocode}
\def\BKM@SourceLocation{%
  \ifx\BKM@DO@srcfile\@empty
    \ifx\BKM@DO@srcline\@empty
    \else
      .\MessageBreak
      Source: line \BKM@DO@srcline
    \fi
  \else
    \ifx\BKM@DO@srcline\@empty
      .\MessageBreak
      Source: file `\BKM@DO@srcfile'%
    \else
      .\MessageBreak
      Source: file `\BKM@DO@srcfile', line \BKM@DO@srcline
    \fi
  \fi
}
%    \end{macrocode}
%    \end{macro}
%    \begin{macrocode}
%</pdftex|pdfmark>
%    \end{macrocode}
%
% \subsection{具有 pdfmark 特色(specials)的驱动程序}
%
% \subsubsection{dvips 驱动程序}
%
%    \begin{macrocode}
%<*dvips>
\NeedsTeXFormat{LaTeX2e}
\ProvidesFile{bkm-dvips.def}%
  [2020-11-06 v1.29 bookmark driver for dvips (HO)]%
%    \end{macrocode}
%    \begin{macro}{\BKM@PSHeaderFile}
%    \begin{macrocode}
\def\BKM@PSHeaderFile#1{%
  \special{PSfile=#1}%
}
%    \end{macrocode}
%    \begin{macro}{\BKM@filename}
%    \begin{macrocode}
\def\BKM@filename{\jobname.out.ps}
%    \end{macrocode}
%    \end{macro}
%    \begin{macrocode}
\AddToHook{shipout/lastpage}{%
  \BKM@pdfmark@out
  \BKM@PSHeaderFile\BKM@filename
  }
%    \end{macrocode}
%    \end{macro}
%    \begin{macrocode}
%</dvips>
%    \end{macrocode}
%
% \subsubsection{公共部分(Common part)}
%
%    \begin{macrocode}
%<*pdfmark>
%    \end{macrocode}
%
%    \begin{macro}{\BKM@pdfmark@out}
%    不要在这里使用 \xpackage{rerunfilecheck}\ 宏包,因为在 \hologo{TeX}\ 运行期间不会
%    读取 \cs{BKM@filename}\ 文件。
%    \begin{macrocode}
\def\BKM@pdfmark@out{%
  \if@filesw
    \newwrite\BKM@file
    \immediate\openout\BKM@file=\BKM@filename\relax
    \BKM@write{\@percentchar!}%
    \BKM@write{/pdfmark where{pop}}%
    \BKM@write{%
      {%
        /globaldict where{pop globaldict}{userdict}ifelse%
        /pdfmark/cleartomark load put%
      }%
    }%
    \BKM@write{ifelse}%
  \else
    \let\BKM@write\@gobble
    \let\BKM@DO@entry\@gobbletwo
  \fi
}
%    \end{macrocode}
%    \end{macro}
%    \begin{macro}{\BKM@write}
%    \begin{macrocode}
\def\BKM@write#{%
  \immediate\write\BKM@file
}
%    \end{macrocode}
%    \end{macro}
%
%    \begin{macro}{\BKM@DO@entry}
%    Pdfmark 的规范(specification)说明 |/Color| 是颜色(color)的键名(key name),
%    但是 ghostscript 只将键(key)传递到 PDF 文件中,因此键名必须是 |/C|。
%    \begin{macrocode}
\def\BKM@DO@entry#1#2{%
  \begingroup
    \kvsetkeys{BKM@DO}{#1}%
    \ifx\BKM@DO@srcfile\@empty
    \else
      \BKM@UnescapeHex\BKM@DO@srcfile
    \fi
    \def\BKM@DO@title{#2}%
    \BKM@UnescapeHex\BKM@DO@title
    \expandafter\expandafter\expandafter\BKM@getx
        \csname BKM@\BKM@DO@id\endcsname\@empty\@empty
    \let\BKM@attr\@empty
    \ifx\BKM@DO@flags\@empty
    \else
      \edef\BKM@attr{\BKM@attr/F \BKM@DO@flags}%
    \fi
    \ifx\BKM@DO@color\@empty
    \else
      \edef\BKM@attr{\BKM@attr/C[\BKM@DO@color]}%
    \fi
    \let\BKM@action\@empty
    \ifx\BKM@DO@gotor\@empty
      \ifx\BKM@DO@dest\@empty
        \ifx\BKM@DO@named\@empty
          \ifx\BKM@DO@rawaction\@empty
            \ifx\BKM@DO@uri\@empty
              \ifx\BKM@DO@page\@empty
                \PackageError{bookmark}{%
                  Missing action\BKM@SourceLocation
                }\@ehc
                \edef\BKM@action{%
                  /Action/GoTo%
                  /Page 1%
                  /View[/Fit]%
                }%
              \else
                \ifx\BKM@DO@view\@empty
                  \def\BKM@DO@view{Fit}%
                \fi
                \edef\BKM@action{%
                  /Action/GoTo%
                  /Page \BKM@DO@page
                  /View[/\BKM@DO@view]%
                }%
              \fi
            \else
              \BKM@UnescapeHex\BKM@DO@uri
              \BKM@EscapeString\BKM@DO@uri
              \edef\BKM@action{%
                /Action<<%
                  /Subtype/URI%
                  /URI(\BKM@DO@uri)%
                >>%
              }%
            \fi
          \else
            \BKM@UnescapeHex\BKM@DO@rawaction
            \edef\BKM@action{%
              /Action<<%
                \BKM@DO@rawaction
              >>%
            }%
          \fi
        \else
          \BKM@EscapeName\BKM@DO@named
          \edef\BKM@action{%
            /Action<<%
              /Subtype/Named%
              /N/\BKM@DO@named
            >>%
          }%
        \fi
      \else
        \BKM@UnescapeHex\BKM@DO@dest
        \BKM@EscapeString\BKM@DO@dest
        \edef\BKM@action{%
          /Action/GoTo%
          /Dest(\BKM@DO@dest)cvn%
        }%
      \fi
    \else
      \ifx\BKM@DO@dest\@empty
        \ifx\BKM@DO@page\@empty
          \def\BKM@DO@page{1}%
        \fi
        \ifx\BKM@DO@view\@empty
          \def\BKM@DO@view{Fit}%
        \fi
        \edef\BKM@action{%
          /Page \BKM@DO@page
          /View[/\BKM@DO@view]%
        }%
      \else
        \BKM@UnescapeHex\BKM@DO@dest
        \BKM@EscapeString\BKM@DO@dest
        \edef\BKM@action{%
          /Dest(\BKM@DO@dest)cvn%
        }%
      \fi
      \BKM@UnescapeHex\BKM@DO@gotor
      \BKM@EscapeString\BKM@DO@gotor
      \edef\BKM@action{%
        /Action/GoToR%
        /File(\BKM@DO@gotor)%
        \BKM@action
      }%
    \fi
    \BKM@write{[}%
    \BKM@write{/Title(\BKM@DO@title)}%
    \ifnum\BKM@x@childs>\z@
      \BKM@write{/Count \ifBKM@DO@open\else-\fi\BKM@x@childs}%
    \fi
    \ifx\BKM@attr\@empty
    \else
      \BKM@write{\BKM@attr}%
    \fi
    \BKM@write{\BKM@action}%
    \BKM@write{/OUT pdfmark}%
  \endgroup
}
%    \end{macrocode}
%    \end{macro}
%    \begin{macrocode}
%</pdfmark>
%    \end{macrocode}
%
% \subsection{\xoption{pdftex}\ 和 \xoption{pdfmark}\ 的公共部分}
%
%    \begin{macrocode}
%<*pdftex|pdfmark>
%    \end{macrocode}
%
% \subsubsection{写入辅助文件(auxiliary file)}
%
%    \begin{macrocode}
\AddToHook{begindocument}{%
 \immediate\write\@mainaux{\string\providecommand\string\BKM@entry[2]{}}}
%    \end{macrocode}
%
%    \begin{macro}{\BKM@id}
%    \begin{macrocode}
\newcount\BKM@id
\BKM@id=\z@
%    \end{macrocode}
%    \end{macro}
%
%    \begin{macro}{\BKM@0}
%    \begin{macrocode}
\@namedef{BKM@0}{000}
%    \end{macrocode}
%    \end{macro}
%    \begin{macro}{\ifBKM@sw}
%    \begin{macrocode}
\newif\ifBKM@sw
%    \end{macrocode}
%    \end{macro}
%
%    \begin{macro}{\bookmark}
%    \begin{macrocode}
\newcommand*{\bookmark}[2][]{%
  \if@filesw
    \begingroup
      \BKM@InitSourceLocation
      \def\bookmark@text{#2}%
      \BKM@setup{#1}%
      \ifx\BKM@srcfile\@empty
      \else
        \BKM@EscapeHex\BKM@srcfile
      \fi
      \edef\BKM@prev{\the\BKM@id}%
      \global\advance\BKM@id\@ne
      \BKM@swtrue
      \@whilesw\ifBKM@sw\fi{%
        \ifnum\ifBKM@startatroot\z@\else\BKM@prev\fi=\z@
          \BKM@startatrootfalse
          \expandafter\xdef\csname BKM@\the\BKM@id\endcsname{%
            0{\BKM@level}0%
          }%
          \BKM@swfalse
        \else
          \expandafter\expandafter\expandafter\BKM@getx
              \csname BKM@\BKM@prev\endcsname
          \ifnum\BKM@level>\BKM@x@level\relax
            \expandafter\xdef\csname BKM@\the\BKM@id\endcsname{%
              {\BKM@prev}{\BKM@level}0%
            }%
            \ifnum\BKM@prev>\z@
              \BKM@CalcExpr\BKM@CalcResult\BKM@x@childs+1%
              \expandafter\xdef\csname BKM@\BKM@prev\endcsname{%
                {\BKM@x@parent}{\BKM@x@level}{\BKM@CalcResult}%
              }%
            \fi
            \BKM@swfalse
          \else
            \let\BKM@prev\BKM@x@parent
          \fi
        \fi
      }%
      \pdfstringdef\BKM@title{\bookmark@text}%
      \edef\BKM@FLAGS{\BKM@PrintStyle}%
      \csname BKM@HypDestOptHook\endcsname
      \BKM@EscapeHex\BKM@dest
      \BKM@EscapeHex\BKM@uri
      \BKM@EscapeHex\BKM@gotor
      \BKM@EscapeHex\BKM@rawaction
      \BKM@EscapeHex\BKM@title
      \immediate\write\@mainaux{%
        \string\BKM@entry{%
          id=\number\BKM@id
          \ifBKM@open
            \ifnum\BKM@level<\BKM@openlevel
              ,open%
            \fi
          \fi
          \BKM@auxentry{dest}%
          \BKM@auxentry{named}%
          \BKM@auxentry{uri}%
          \BKM@auxentry{gotor}%
          \BKM@auxentry{page}%
          \BKM@auxentry{view}%
          \BKM@auxentry{rawaction}%
          \BKM@auxentry{color}%
          \ifnum\BKM@FLAGS>\z@
            ,flags=\BKM@FLAGS
          \fi
          \BKM@auxentry{srcline}%
          \BKM@auxentry{srcfile}%
        }{\BKM@title}%
      }%
    \endgroup
  \fi
}
%    \end{macrocode}
%    \end{macro}
%    \begin{macro}{\BKM@getx}
%    \begin{macrocode}
\def\BKM@getx#1#2#3{%
  \def\BKM@x@parent{#1}%
  \def\BKM@x@level{#2}%
  \def\BKM@x@childs{#3}%
}
%    \end{macrocode}
%    \end{macro}
%    \begin{macro}{\BKM@auxentry}
%    \begin{macrocode}
\def\BKM@auxentry#1{%
  \expandafter\ifx\csname BKM@#1\endcsname\@empty
  \else
    ,#1={\csname BKM@#1\endcsname}%
  \fi
}
%    \end{macrocode}
%    \end{macro}
%
%    \begin{macro}{\BKM@InitSourceLocation}
%    \begin{macrocode}
\def\BKM@InitSourceLocation{%
  \edef\BKM@srcline{\the\inputlineno}%
  \BKM@LuaTeX@InitFile
  \ifx\BKM@srcfile\@empty
    \ltx@IfUndefined{currfilepath}{}{%
      \edef\BKM@srcfile{\currfilepath}%
    }%
  \fi
}
%    \end{macrocode}
%    \end{macro}
%    \begin{macro}{\BKM@LuaTeX@InitFile}
%    \begin{macrocode}
\ifluatex
  \ifnum\luatexversion>36 %
    \def\BKM@LuaTeX@InitFile{%
      \begingroup
        \ltx@LocToksA={}%
      \edef\x{\endgroup
        \def\noexpand\BKM@srcfile{%
          \the\expandafter\ltx@LocToksA
          \directlua{%
             if status and status.filename then %
               tex.settoks('ltx@LocToksA', status.filename)%
             end%
          }%
        }%
      }\x
    }%
  \else
    \let\BKM@LuaTeX@InitFile\relax
  \fi
\else
  \let\BKM@LuaTeX@InitFile\relax
\fi
%    \end{macrocode}
%    \end{macro}
%
% \subsubsection{读取辅助数据(auxiliary data)}
%
%    \begin{macrocode}
\SetupKeyvalOptions{family=BKM@DO,prefix=BKM@DO@}
\DeclareStringOption[0]{id}
\DeclareBoolOption{open}
\DeclareStringOption{flags}
\DeclareStringOption{color}
\DeclareStringOption{dest}
\DeclareStringOption{named}
\DeclareStringOption{uri}
\DeclareStringOption{gotor}
\DeclareStringOption{page}
\DeclareStringOption{view}
\DeclareStringOption{rawaction}
\DeclareStringOption{srcline}
\DeclareStringOption{srcfile}
%    \end{macrocode}
%
%    \begin{macrocode}
\AtBeginDocument{%
  \let\BKM@entry\BKM@DO@entry
}
%    \end{macrocode}
%
%    \begin{macrocode}
%</pdftex|pdfmark>
%    \end{macrocode}
%
% \subsection{\xoption{atend}\ 选项}
%
% \subsubsection{钩子(Hook)}
%
%    \begin{macrocode}
%<*package>
%    \end{macrocode}
%    \begin{macrocode}
\ifBKM@atend
\else
%    \end{macrocode}
%    \begin{macro}{\BookmarkAtEnd}
%    这是一个虚拟定义(dummy definition),如果没有给出 \xoption{atend}\ 选项,它将生成一个警告。
%    \begin{macrocode}
  \newcommand{\BookmarkAtEnd}[1]{%
    \PackageWarning{bookmark}{%
      Ignored, because option `atend' is missing%
    }%
  }%
%    \end{macrocode}
%    \end{macro}
%    \begin{macrocode}
  \expandafter\endinput
\fi
%    \end{macrocode}
%    \begin{macro}{\BookmarkAtEnd}
%    \begin{macrocode}
\newcommand*{\BookmarkAtEnd}{%
  \g@addto@macro\BKM@EndHook
}
%    \end{macrocode}
%    \end{macro}
%    \begin{macrocode}
\let\BKM@EndHook\@empty
%    \end{macrocode}
%    \begin{macrocode}
%</package>
%    \end{macrocode}
%
% \subsubsection{在文档末尾使用钩子的驱动程序}
%
%    驱动程序 \xoption{pdftex}\ 使用 LaTeX 钩子 \xoption{enddocument/afterlastpage}
%    (相当于以前使用的 \xpackage{atveryend}\ 的 \cs{AfterLastShipout}),因为它仍然需要 \xext{aux}\ 文件。
%    它使用 \cs{pdfoutline}\ 作为最后一页之后可以使用的书签(bookmakrs)。
%    \begin{itemize}
%    \item
%      驱动程序 \xoption{pdftex}\ 使用 \cs{pdfoutline}, \cs{pdfoutline}\ 可以在最后一页之后使用。
%    \end{itemize}
%    \begin{macrocode}
%<*pdftex>
\ifBKM@atend
  \AddToHook{enddocument/afterlastpage}{%
    \BKM@EndHook
  }%
\fi
%</pdftex>
%    \end{macrocode}
%
% \subsubsection{使用 \xoption{shipout/lastpage}\ 的驱动程序}
%
%    其他驱动程序使用 \cs{special}\ 命令实现 \cs{bookmark}。因此,最后的书签(last bookmarks)
%    必须放在最后一页(last page),而不是之后。不能使用 \cs{AtEndDocument},因为为时已晚,
%    最后一页已经输出了。因此,我们使用 LaTeX 钩子 \xoption{shipout/lastpage}。至少需要运行
%    两次 \hologo{LaTeX}。PostScript 驱动程序 \xoption{dvips}\ 使用外部 PostScript 文件作为书签。
%    为了避免与 pgf 发生冲突,文件写入(file writing)也被移到了最后一个输出页面(shipout page)。
%    \begin{macrocode}
%<*dvipdfm|vtex|pdfmark>
\ifBKM@atend
  \AddToHook{shipout/lastpage}{\BKM@EndHook}%
\fi
%</dvipdfm|vtex|pdfmark>
%    \end{macrocode}
%
% \section{安装(Installation)}
%
% \subsection{下载(Download)}
%
% \paragraph{宏包(Package)。} 在 CTAN\footnote{\CTANpkg{bookmark}}上提供此宏包:
% \begin{description}
% \item[\CTAN{macros/latex/contrib/bookmark/bookmark.dtx}] 源文件(source file)。
% \item[\CTAN{macros/latex/contrib/bookmark/bookmark.pdf}] 文档(documentation)。
% \end{description}
%
%
% \paragraph{捆绑包(Bundle)。} “bookmark”捆绑包(bundle)的所有宏包(packages)都可以在兼
% 容 TDS 的 ZIP 归档文件中找到。在那里,宏包已经被解包,文档文件(documentation files)已经生成。
% 文件(files)和目录(directories)遵循 TDS 标准。
% \begin{description}
% \item[\CTANinstall{install/macros/latex/contrib/bookmark.tds.zip}]
% \end{description}
% \emph{TDS}\ 是指标准的“用于 \TeX\ 文件的目录结构(Directory Structure)”(\CTANpkg{tds})。
% 名称中带有 \xfile{texmf}\ 的目录(directories)通常以这种方式组织。
%
% \subsection{捆绑包(Bundle)的安装}
%
% \paragraph{解压(Unpacking)。} 在您选择的 TDS 树(也称为 \xfile{texmf}\ 树)中解
% 压 \xfile{bookmark.tds.zip},例如(在 linux 中):
% \begin{quote}
%   |unzip bookmark.tds.zip -d ~/texmf|
% \end{quote}
%
% \subsection{宏包(Package)的安装}
%
% \paragraph{解压(Unpacking)。} \xfile{.dtx}\ 文件是一个自解压 \docstrip\ 归档文件(archive)。
% 这些文件是通过 \plainTeX\ 运行 \xfile{.dtx}\ 来提取的:
% \begin{quote}
%   \verb|tex bookmark.dtx|
% \end{quote}
%
% \paragraph{TDS.} 现在,不同的文件必须移动到安装 TDS 树(installation TDS tree)
% (也称为 \xfile{texmf}\ 树)中的不同目录中:
% \begin{quote}
% \def\t{^^A
% \begin{tabular}{@{}>{\ttfamily}l@{ $\rightarrow$ }>{\ttfamily}l@{}}
%   bookmark.sty & tex/latex/bookmark/bookmark.sty\\
%   bkm-dvipdfm.def & tex/latex/bookmark/bkm-dvipdfm.def\\
%   bkm-dvips.def & tex/latex/bookmark/bkm-dvips.def\\
%   bkm-pdftex.def & tex/latex/bookmark/bkm-pdftex.def\\
%   bkm-vtex.def & tex/latex/bookmark/bkm-vtex.def\\
%   bookmark.pdf & doc/latex/bookmark/bookmark.pdf\\
%   bookmark-example.tex & doc/latex/bookmark/bookmark-example.tex\\
%   bookmark.dtx & source/latex/bookmark/bookmark.dtx\\
% \end{tabular}^^A
% }^^A
% \sbox0{\t}^^A
% \ifdim\wd0>\linewidth
%   \begingroup
%     \advance\linewidth by\leftmargin
%     \advance\linewidth by\rightmargin
%   \edef\x{\endgroup
%     \def\noexpand\lw{\the\linewidth}^^A
%   }\x
%   \def\lwbox{^^A
%     \leavevmode
%     \hbox to \linewidth{^^A
%       \kern-\leftmargin\relax
%       \hss
%       \usebox0
%       \hss
%       \kern-\rightmargin\relax
%     }^^A
%   }^^A
%   \ifdim\wd0>\lw
%     \sbox0{\small\t}^^A
%     \ifdim\wd0>\linewidth
%       \ifdim\wd0>\lw
%         \sbox0{\footnotesize\t}^^A
%         \ifdim\wd0>\linewidth
%           \ifdim\wd0>\lw
%             \sbox0{\scriptsize\t}^^A
%             \ifdim\wd0>\linewidth
%               \ifdim\wd0>\lw
%                 \sbox0{\tiny\t}^^A
%                 \ifdim\wd0>\linewidth
%                   \lwbox
%                 \else
%                   \usebox0
%                 \fi
%               \else
%                 \lwbox
%               \fi
%             \else
%               \usebox0
%             \fi
%           \else
%             \lwbox
%           \fi
%         \else
%           \usebox0
%         \fi
%       \else
%         \lwbox
%       \fi
%     \else
%       \usebox0
%     \fi
%   \else
%     \lwbox
%   \fi
% \else
%   \usebox0
% \fi
% \end{quote}
% 如果你有一个 \xfile{docstrip.cfg}\ 文件,该文件能配置并启用 \docstrip\ 的 TDS 安装功能,
% 则一些文件可能已经在正确的位置了,请参阅 \docstrip\ 的文档(documentation)。
%
% \subsection{刷新文件名数据库}
%
% 如果您的 \TeX~发行版(\TeX\,Live、\mikTeX、\dots)依赖于文件名数据库(file name databases),
% 则必须刷新这些文件名数据库。例如,\TeX\,Live\ 用户运行 \verb|texhash| 或 \verb|mktexlsr|。
%
% \subsection{一些感兴趣的细节}
%
% \paragraph{用 \LaTeX\ 解压。}
% \xfile{.dtx}\ 根据格式(format)选择其操作(action):
% \begin{description}
% \item[\plainTeX:] 运行 \docstrip\ 并解压文件。
% \item[\LaTeX:] 生成文档。
% \end{description}
% 如果您坚持通过 \LaTeX\ 使用\docstrip (实际上 \docstrip\ 并不需要 \LaTeX),那么请您的意图告知自动检测程序:
% \begin{quote}
%   \verb|latex \let\install=y% \iffalse meta-comment
%
% File: bookmark.dtx
% Version: 2020-11-06 v1.29
% Info: PDF bookmarks
%
% Copyright (C)
%    2007-2011 Heiko Oberdiek
%    2016-2020 Oberdiek Package Support Group
%    https://github.com/ho-tex/bookmark/issues
%
% This work may be distributed and/or modified under the
% conditions of the LaTeX Project Public License, either
% version 1.3c of this license or (at your option) any later
% version. This version of this license is in
%    https://www.latex-project.org/lppl/lppl-1-3c.txt
% and the latest version of this license is in
%    https://www.latex-project.org/lppl.txt
% and version 1.3 or later is part of all distributions of
% LaTeX version 2005/12/01 or later.
%
% This work has the LPPL maintenance status "maintained".
%
% The Current Maintainers of this work are
% Heiko Oberdiek and the Oberdiek Package Support Group
% https://github.com/ho-tex/bookmark/issues
%
% This work consists of the main source file bookmark.dtx
% and the derived files
%    bookmark.sty, bookmark.pdf, bookmark.ins, bookmark.drv,
%    bkm-dvipdfm.def, bkm-dvips.def,
%    bkm-pdftex.def, bkm-vtex.def,
%    bkm-dvipdfm-2019-12-03.def, bkm-dvips-2019-12-03.def,
%    bkm-pdftex-2019-12-03.def, bkm-vtex-2019-12-03.def,
%    bookmark-example.tex.
%
% Distribution:
%    CTAN:macros/latex/contrib/bookmark/bookmark.dtx
%    CTAN:macros/latex/contrib/bookmark/bookmark-frozen.dtx
%    CTAN:macros/latex/contrib/bookmark/bookmark.pdf
%
% Unpacking:
%    (a) If bookmark.ins is present:
%           tex bookmark.ins
%    (b) Without bookmark.ins:
%           tex bookmark.dtx
%    (c) If you insist on using LaTeX
%           latex \let\install=y\input{bookmark.dtx}
%        (quote the arguments according to the demands of your shell)
%
% Documentation:
%    (a) If bookmark.drv is present:
%           latex bookmark.drv
%    (b) Without bookmark.drv:
%           latex bookmark.dtx; ...
%    The class ltxdoc loads the configuration file ltxdoc.cfg
%    if available. Here you can specify further options, e.g.
%    use A4 as paper format:
%       \PassOptionsToClass{a4paper}{article}
%
%    Programm calls to get the documentation (example):
%       pdflatex bookmark.dtx
%       makeindex -s gind.ist bookmark.idx
%       pdflatex bookmark.dtx
%       makeindex -s gind.ist bookmark.idx
%       pdflatex bookmark.dtx
%
% Installation:
%    TDS:tex/latex/bookmark/bookmark.sty
%    TDS:tex/latex/bookmark/bkm-dvipdfm.def
%    TDS:tex/latex/bookmark/bkm-dvips.def
%    TDS:tex/latex/bookmark/bkm-pdftex.def
%    TDS:tex/latex/bookmark/bkm-vtex.def
%    TDS:tex/latex/bookmark/bkm-dvipdfm-2019-12-03.def
%    TDS:tex/latex/bookmark/bkm-dvips-2019-12-03.def
%    TDS:tex/latex/bookmark/bkm-pdftex-2019-12-03.def
%    TDS:tex/latex/bookmark/bkm-vtex-2019-12-03.def%
%    TDS:doc/latex/bookmark/bookmark.pdf
%    TDS:doc/latex/bookmark/bookmark-example.tex
%    TDS:source/latex/bookmark/bookmark.dtx
%    TDS:source/latex/bookmark/bookmark-frozen.dtx
%
%<*ignore>
\begingroup
  \catcode123=1 %
  \catcode125=2 %
  \def\x{LaTeX2e}%
\expandafter\endgroup
\ifcase 0\ifx\install y1\fi\expandafter
         \ifx\csname processbatchFile\endcsname\relax\else1\fi
         \ifx\fmtname\x\else 1\fi\relax
\else\csname fi\endcsname
%</ignore>
%<*install>
\input docstrip.tex
\Msg{************************************************************************}
\Msg{* Installation}
\Msg{* Package: bookmark 2020-11-06 v1.29 PDF bookmarks (HO)}
\Msg{************************************************************************}

\keepsilent
\askforoverwritefalse

\let\MetaPrefix\relax
\preamble

This is a generated file.

Project: bookmark
Version: 2020-11-06 v1.29

Copyright (C)
   2007-2011 Heiko Oberdiek
   2016-2020 Oberdiek Package Support Group

This work may be distributed and/or modified under the
conditions of the LaTeX Project Public License, either
version 1.3c of this license or (at your option) any later
version. This version of this license is in
   https://www.latex-project.org/lppl/lppl-1-3c.txt
and the latest version of this license is in
   https://www.latex-project.org/lppl.txt
and version 1.3 or later is part of all distributions of
LaTeX version 2005/12/01 or later.

This work has the LPPL maintenance status "maintained".

The Current Maintainers of this work are
Heiko Oberdiek and the Oberdiek Package Support Group
https://github.com/ho-tex/bookmark/issues


This work consists of the main source file bookmark.dtx and bookmark-frozen.dtx
and the derived files
   bookmark.sty, bookmark.pdf, bookmark.ins, bookmark.drv,
   bkm-dvipdfm.def, bkm-dvips.def, bkm-pdftex.def, bkm-vtex.def,
   bkm-dvipdfm-2019-12-03.def, bkm-dvips-2019-12-03.def,
   bkm-pdftex-2019-12-03.def, bkm-vtex-2019-12-03.def,
   bookmark-example.tex.

\endpreamble
\let\MetaPrefix\DoubleperCent

\generate{%
  \file{bookmark.ins}{\from{bookmark.dtx}{install}}%
  \file{bookmark.drv}{\from{bookmark.dtx}{driver}}%
  \usedir{tex/latex/bookmark}%
  \file{bookmark.sty}{\from{bookmark.dtx}{package}}%
  \file{bkm-dvipdfm.def}{\from{bookmark.dtx}{dvipdfm}}%
  \file{bkm-dvips.def}{\from{bookmark.dtx}{dvips,pdfmark}}%
  \file{bkm-pdftex.def}{\from{bookmark.dtx}{pdftex}}%
  \file{bkm-vtex.def}{\from{bookmark.dtx}{vtex}}%
  \usedir{doc/latex/bookmark}%
  \file{bookmark-example.tex}{\from{bookmark.dtx}{example}}%
  \file{bkm-pdftex-2019-12-03.def}{\from{bookmark-frozen.dtx}{pdftexfrozen}}%
  \file{bkm-dvips-2019-12-03.def}{\from{bookmark-frozen.dtx}{dvipsfrozen}}%
  \file{bkm-vtex-2019-12-03.def}{\from{bookmark-frozen.dtx}{vtexfrozen}}%
  \file{bkm-dvipdfm-2019-12-03.def}{\from{bookmark-frozen.dtx}{dvipdfmfrozen}}%
}

\catcode32=13\relax% active space
\let =\space%
\Msg{************************************************************************}
\Msg{*}
\Msg{* To finish the installation you have to move the following}
\Msg{* files into a directory searched by TeX:}
\Msg{*}
\Msg{*     bookmark.sty, bkm-dvipdfm.def, bkm-dvips.def,}
\Msg{*     bkm-pdftex.def, bkm-vtex.def, bkm-dvipdfm-2019-12-03.def,}
\Msg{*     bkm-dvips-2019-12-03.def, bkm-pdftex-2019-12-03.def,}
\Msg{*     and bkm-vtex-2019-12-03.def}
\Msg{*}
\Msg{* To produce the documentation run the file `bookmark.drv'}
\Msg{* through LaTeX.}
\Msg{*}
\Msg{* Happy TeXing!}
\Msg{*}
\Msg{************************************************************************}

\endbatchfile
%</install>
%<*ignore>
\fi
%</ignore>
%<*driver>
\NeedsTeXFormat{LaTeX2e}
\ProvidesFile{bookmark.drv}%
  [2020-11-06 v1.29 PDF bookmarks (HO)]%
\documentclass{ltxdoc}
\usepackage{ctex}
\usepackage{indentfirst}
\setlength{\parindent}{2em}
\usepackage{holtxdoc}[2011/11/22]
\usepackage{xcolor}
\usepackage{hyperref}
\usepackage[open,openlevel=3,atend]{bookmark}[2020/11/06] %%%打开书签,显示的深度为3级,即显示part、section、subsection。
\bookmarksetup{color=red}
\begin{document}

  \renewcommand{\contentsname}{目\quad 录}
  \renewcommand{\abstractname}{摘\quad 要}
  \renewcommand{\historyname}{历史}
  \DocInput{bookmark.dtx}%
\end{document}
%</driver>
% \fi
%
%
%
% \GetFileInfo{bookmark.drv}
%
%% \title{\xpackage{bookmark} 宏包}
% \title{\heiti {\Huge \textbf{\xpackage{bookmark}\ 宏包}}}
% \date{2020-11-06\ \ \ v1.29}
% \author{Heiko Oberdiek \thanks
% {如有问题请点击:\url{https://github.com/ho-tex/bookmark/issues}}\\[5pt]赣医一附院神经科\ \ 黄旭华\ \ \ \ 译}
%
% \maketitle
%
% \begin{abstract}
% 这个宏包为 \xpackage{hyperref}\ 宏包实现了一个新的书签(bookmark)(大纲[outline])组织。现在
% 可以设置样式(style)和颜色(color)等书签属性(bookmark properties)。其他动作类型(action types)可用
% (URI、GoToR、Named)。书签是在第一次编译运行(compile run)中生成的。\xpackage{hyperref}\
% 宏包必需运行两次。
% \end{abstract}
%
% \tableofcontents
%
% \section{文档(Documentation)}
%
% \subsection{介绍}
%
% 这个 \xpackage{bookmark}\ 宏包试图为书签(bookmarks)提供一个更现代的管理:
% \begin{itemize}
% \item 书签已经在第一次 \hologo{TeX}\ 编译运行(compile run)中生成。
% \item 可以更改书签的字体样式(font style)和颜色(color)。
% \item 可以执行比简单的 GoTo 操作(actions)更多的操作。
% \end{itemize}
%
% 与 \xpackage{hyperref} \cite{hyperref} 一样,书签(bookmarks)也是按照书签生成宏
% (bookmark generating macros)(\cs{bookmark})的顺序生成的。级别号(level number)用于
% 定义书签的树结构(tree structure)。限制没有那么严格:
% \begin{itemize}
% \item 级别值(level values)可以跳变(jump)和省略(omit)。\cs{subsubsection}\ 可以跟在
%       \cs{chapter}\ 之后。这种情况如在 \xpackage{hyperref}\ 中则产生错误,它将显示一个警告(warning)
%       并尝试修复此错误。
% \item 多个书签可能指向同一目标(destination)。在 \xpackage{hyperref}\ 中,这会完全弄乱
%       书签树(bookmark tree),因为算法假设(algorithm assumes)目标名称(destination names)
%       是键(keys)(唯一的)。
% \end{itemize}
%
% 注意,这个宏包是作为书签管理(bookmark management)的实验平台(experimentation platform)。
% 欢迎反馈。此外,在未来的版本中,接口(interfaces)也可能发生变化。
%
% \subsection{选项(Options)}
%
% 可在以下四个地方放置选项(options):
% \begin{enumerate}
% \item \cs{usepackage}|[|\meta{options}|]{bookmark}|\\
%       这是放置驱动程序选项(driver options)和 \xoption{atend}\ 选项的唯一位置。
% \item \cs{bookmarksetup}|{|\meta{options}|}|\\
%       此命令仅用于设置选项(setting options)。
% \item \cs{bookmarksetupnext}|{|\meta{options}|}|\\
%       这些选项在下一个 \cs{bookmark}\ 命令的选项之后存储(stored)和调用(called)。
% \item \cs{bookmark}|[|\meta{options}|]{|\meta{title}|}|\\
%       此命令设置书签。选项设置(option settings)仅限于此书签。
% \end{enumerate}
% 异常(Exception):加载该宏包后,无法更改驱动程序选项(Driver options)、\xoption{atend}\ 选项
% 、\xoption{draft}\slash\xoption{final}选项。
%
% \subsubsection{\xoption{draft} 和 \xoption{final}\ 选项}
%
% 如果一个\LaTeX\ 文件要被编译了多次,那么可以使用 \xoption{draft}\ 选项来禁用该宏包的书签内
% 容(bookmark stuff),这样可以节省一点时间。默认 \xoption{final}\ 选项。两个选项都是
% 布尔选项(boolean options),如果没有值,则使用值 |true|。|draft=true| 与 |final=false| 相同。
%
% 除了驱动程序选项(driver options)之外,\xpackage{bookmark}\ 宏包选项都是局部选项(local options)。
% \xoption{draft}\ 选项和 \xoption{final}\ 选项均属于文档类选项(class option)(译者注:文档类选项为全局选项),
% 因此,在 \xpackage{bookmark}\ 宏包中未能看到这两个选项。如果您想使用全局的(global) \xoption{draft}选项
% 来优化第一次 \LaTeX\ 运行(runs),可以在导言(preamble)中引入 \xpackage{ifdraft}\ 宏包并设置 \LaTeX\ 的
% \cs{PassOptionsToPackage},例如:
%\begin{quote}
%\begin{verbatim}
%\documentclass[draft]{article}
%\usepackage{ifdraft}
%\ifdraft{%
%   \PassOptionsToPackage{draft}{bookmark}%
%}{}
%\end{verbatim}
%\end{quote}
%
% \subsubsection{驱动程序选项(Driver options)}
%
% 支持的驱动程序( drivers)包括 \xoption{pdftex}、\xoption{dvips}、\xoption{dvipdfm} (\xoption{xetex})、
% \xoption{vtex}。\hologo{TeX}\ 引擎 \hologo{pdfTeX}、\hologo{XeTeX}、\hologo{VTeX}\ 能被自动检测到。
% 默认的 DVI 驱动程序是 \xoption{dvips}。这可以通过 \cs{BookmarkDriverDefault}\ 在配置
% 文件 \xfile{bookmark.cfg}\ 中进行更改,例如:
% \begin{quote}
% |\def\BookmarkDriverDefault{dvipdfm}|
% \end{quote}
% 当前版本的(current versions)驱动程序使用新的 \LaTeX\ 钩子(\LaTeX-hooks)。如果检测到比
% 2020-10-01 更旧的格式,则将以前驱动程序的冻结版本(frozen versions)作为备份(fallback)。
%
% \paragraph{用 dvipdfmx 打开书签(bookmarks)。}旧版本的宏包有一个 \xoption{dvipdfmx-outline-open}\ 选项
% 可以激活代码,而该代码可以指定一个大纲条目(outline entry)是否打开。该宏包现在假设所有使用的 dvipdfmx 版本都是
% 最新版本,足以理解该代码,因此始终激活该代码。选项本身将被忽略。
%
%
% \subsubsection{布局选项(Layout options)}
%
% \paragraph{字体(Font)选项:}
%
% \begin{description}
% \item[\xoption{bold}:] 如果受 PDF 浏览器(PDF viewer)支持,书签将以粗体字体(bold font)显示(自 PDF 1.4起)。
% \item[\xoption{italic}:] 使用斜体字体(italic font)(自 PDF 1.4起)。
% \end{description}
% \xoption{bold}(粗体) 和 \xoption{italic}(斜体)可以同时使用。而 |false| 值(value)禁用字体选项。
%
% \paragraph{颜色(Color)选项:}
%
% 彩色书签(Colored bookmarks)是 PDF 1.4 的一个特性(feature),并非所有的 PDF 浏览器(PDF viewers)都支持彩色书签。
% \begin{description}
% \item[\xoption{color}:] 这里 color(颜色)可以作为 \xpackage{color}\ 宏包或 \xpackage{xcolor}\ 宏包的
% 颜色规范(color specification)给出。空值(empty value)表示未设置颜色属性。如果未加载 \xpackage{xcolor}\ 宏包,
% 能识别的值(recognized values)只有:
%   \begin{itemize}
%   \item 空值(empty value)表示未设置颜色属性,\\
%         例如:|color={}|
%   \item 颜色模型(color model) rgb 的显式颜色规范(explicit color specification),\\
%         例如,红色(red):|color=[rgb]{1,0,0}|
%   \item 颜色模型(color model)灰(gray)的显式颜色规范(explicit color specification),\\
%         例如,深灰色(dark gray):|color=[gray]{0.25}|
%   \end{itemize}
%   请注意,如果加载了 \xpackage{color}\ 宏包,此限制(restriction)也适用。然而,如果加载了 \xpackage{xcolor}\ 宏包,
%   则可以使用所有颜色规范(color specifications)。
% \end{description}
%
% \subsubsection{动作选项(Action options)}
%
% \begin{description}
% \item[\xoption{dest}:] 目的地名称(destination name)。
% \item[\xoption{page}:] 页码(page number),第一页(first page)为 1。
% \item[\xoption{view}:] 浏览规范(view specification),示例如下:\\
%   |view={FitB}|, |view={FitH 842}|, |view={XYZ 0 100 null}|\ \  一些浏览规范参数(view specification parameters)
%   将数字(numbers)视为具有单位 bp 的参数。它们可以作为普通数字(plain numbers)或在 \cs{calc}\ 内部以
%   长度表达式(length expressions)给出。如果加载了 \xpackage{calc}\ 宏包,则支持该宏包的表达式(expressions)。否则,
%   使用 \hologo{eTeX}\ 的 \cs{dimexpr}。例如:\\
%   |view={FitH \calc{\paperheight-\topmargin-1in}}|\\
%   |view={XYZ 0 \calc{\paperheight} null}|\\
%   注意 \cs{calc}\ 不能用于 |XYZ| 的第三个参数,因为该参数是缩放值(zoom value),而不是长度(length)。

% \item[\xoption{named}:] 已命名的动作(Named action)的名称:\\
%   |FirstPage|(第一页),|LastPage|(最后一页),|NextPage|(下一页),|PrevPage|(前一页)
% \item[\xoption{gotor}:] 外部(external) PDF 文件的名称。
% \item[\xoption{uri}:] URI 规范(URI specification)。
% \item[\xoption{rawaction}:] 原始动作规范(raw action specification)。由于这些规范取决于驱动程序(driver),因此不应使用此选项。
% \end{description}
% 通过分析指定的选项来选择书签的适当动作。动作由不同的选项集(sets of options)区分:
% \begin{quote}
 \begin{tabular}{|@{}r|l@{}|}
%   \hline
%   \ \textbf{动作(Action)}\  & \ \textbf{选项(Options)}\ \\ \hline
%   \ \textsf{GoTo}\  &\  \xoption{dest}\ \\ \hline
%   \ \textsf{GoTo}\  & \ \xoption{page} + \xoption{view}\ \\ \hline
%   \ \textsf{GoToR}\  & \ \xoption{gotor} + \xoption{dest}\ \\ \hline
%   \ \textsf{GoToR}\  & \ \xoption{gotor} + \xoption{page} + \xoption{view}\ \ \ \\ \hline
%   \ \textsf{Named}\  &\  \xoption{named}\ \\ \hline
%   \ \textsf{URI}\  & \ \xoption{uri}\ \\ \hline
% \end{tabular}
% \end{quote}
%
% \paragraph{缺少动作(Missing actions)。}
% 如果动作缺少 \xpackage{bookmark}\ 宏包,则抛出错误消息(error message)。根据驱动程序(driver)
% (\xoption{pdftex}、\xoption{dvips}\ 和好友[friends]),宏包在文档末尾很晚才检测到它。
% 自 2011/04/21 v1.21 版本以后,该宏包尝试打印 \cs{bookmark}\ 的相应出现的行号(line number)和文件名(file name)。
% 然而,\hologo{TeX}\ 确实提供了行号,但不幸的是,文件名是一个秘密(secret)。但该宏包有如下获取文件名的方法:
% \begin{itemize}
% \item 如果 \hologo{LuaTeX} (独立于 DVI 或 PDF 模式)正在运行,则自动使用其 |status.filename|。
% \item 宏包的 \cs{currfile} \cite{currfile}\ 重新定义了 \hologo{LaTeX}\ 的内部结构,以跟踪文件名(file name)。
% 如果加载了该宏包,那么它的 \cs{currfilepath}\ 将被检测到并由 \xpackage{bookmark}\ 自动使用。
% \item 可以通过 \cs{bookmarksetup}\ 或 \cs{bookmark}\ 中的 \xoption{scrfile}\ 选项手动设置(set manually)文件名。
% 但是要小心,手动设置会禁用以前的文件名检测方法。错误的(wrong)或丢失的(missed)文件名设置(file name setting)可能会在错误消息中
% 为您提供错误的源位置(source location)。
% \end{itemize}
%
% \subsubsection{级别选项(Level options)}
%
% 书签条目(bookmark entries)的顺序由 \cs{bookmark}\ 命令的的出现顺序(appearance order)定义。
% 树结构(tree structure)由书签节点(bookmark nodes)的属性 \xoption{level}(级别)构建。
% \xoption{level}\ 的值是整数(integers)。如果书签条目级别的值高于前一个节点,则该条目将成为
% 前一个节点的子(child)节点。差值的绝对值并不重要。
%
% \xpackage{bookmark}\ 宏包能记住全局属性(global property)“current level(当前级别)”中上
% 一个书签条目(previous bookmark entry)的级别。
%
% 级别系统的(level system)行为(behaviour)可以通过以下选项进行配置:
% \begin{description}
% \item[\xoption{level}:]
%    设置级别(level),请参阅上面的说明。如果给出的选项 \xoption{level}\ 没有值,那么将恢复默
%    认行为,即将“当前级别(current level)”用作级别值(level value)。自 2010/10/19 v1.16 版本以来,
%    如果宏 \cs{toclevel@part}、\cs{toclevel@section}\ 被定义过(通过 \xpackage{hyperref}\ 宏包完成,
%    请参阅它的 \xoption{bookmarkdepth}\ 选项),则 \xpackage{bookmark}\ 宏包还支持 |part|、|section| 等名称。
%
% \item[\xoption{rellevel}:]
%    设置相对于前一级别的(previous level)级别。正值表示书签条目成为前一个书签条目的子条目。
% \item[\xoption{keeplevel}:]
%    使用由\xoption{level}\ 或 \xoption{rellevel}\ 设置的级别,但不要更改全局属性“current level(当前级别)”。
%    可以通过设置为 |false| 来禁用该选项。
% \item[\xoption{startatroot}:]
%    此时,书签树(bookmark tree)再次从顶层(top level)开始。下一个书签条目不会作为上一个条目的子条目进行排序。
%    示例场景:文档使用 part。但是,最后几章(last chapters)不应放在最后一部分(last part)下面:
%    \begin{quote}
%\begin{verbatim}
%\documentclass{book}
%[...]
%\begin{document}
%  \part{第一部分}
%    \chapter{第一部分的第1章}
%    [...]
%  \part{第二部分(Second part)}
%    \chapter{第二部分的第1章}
%    [...]
%  \bookmarksetup{startatroot}
%  \chapter{Index}% 不属于第二部分
%\end{document}
%\end{verbatim}
%    \end{quote}
% \end{description}
%
% \subsubsection{样式定义(Style definitions)}
%
% 样式(style)是一组选项设置(option settings)。它可以由宏 \cs{bookmarkdefinestyle}\ 定义,
% 并由它的 \xoption{style}\ 选项使用。
% \begin{declcs}{bookmarkdefinestyle} \M{name} \M{key value list}
% \end{declcs}
% 选项设置(option settings)的 \meta{key value list}(键值列表)被指定为样式名(style \meta{name})。
%
% \begin{description}
% \item[\xoption{style}:]
%   \xoption{style}\ 选项的值是以前定义的样式的名称(name)。现在执行其选项设置(option settings)。
%   选项可以包括 \xoption{style}\ 选项。通过递归调用相同样式的无限递归(endless recursion)被阻止并抛出一个错误。
% \end{description}
%
% \subsubsection{钩子支持(Hook support)}
%
% 处理宏\cs{bookmark}\ 的可选选项(optional options)后,就会调用钩子(hook)。
% \begin{description}
% \item[\xoption{addtohook}:]
%   代码(code)作为该选项的值添加到钩子中。
% \end{description}
%
% \begin{declcs}{bookmarkget} \M{option}
% \end{declcs}
% \cs{bookmarkget}\ 宏提取 \meta{option}\ 选项的最新选项设置(latest option setting)的值。
% 对于布尔选项(boolean option),如果启用布尔选项,则返回 1,否则结果为零。结果数字(resulting numbers)
% 可以直接用于 \cs{ifnum}\ 或 \cs{ifcase}。如果您想要数字 \texttt{0}\ 和 \texttt{1},
% 请在 \cs{bookmarkget}\ 前面加上 \cs{number}\ 作为前缀。\cs{bookmarkget}\ 宏是可展开的(expandable)。
% 如果选项不受支持,则返回空字符串(empty string)。受支持的布尔选项有:
% \begin{quote}
%   \xoption{bold}、
%   \xoption{italic}、
%   \xoption{open}
% \end{quote}
% 其他受支持的选项有:
% \begin{quote}
%   \xoption{depth}、
%   \xoption{dest}、
%   \xoption{color}、
%   \xoption{gotor}、
%   \xoption{level}、
%   \xoption{named}、
%   \xoption{openlevel}、
%   \xoption{page}、
%   \xoption{rawaction}、
%   \xoption{uri}、
%   \xoption{view}、
% \end{quote}
% 另外,以下键(key)是可用的:
% \begin{quote}
%   \xoption{text}
% \end{quote}
% 它返回大纲条目(outline entry)的文本(text)。
%
% \paragraph{选项设置(Option setting)。}
% 在钩子(hook)内部可以使用 \cs{bookmarksetup}\ 设置选项。
%
% \subsection{与 \xpackage{hyperref}\ 的兼容性}
%
% \xpackage{bookmark}\ 宏包自动禁用 \xpackage{hyperref}\ 宏包的书签(bookmarks)。但是,
% \xpackage{bookmark}\ 宏包使用了 \xpackage{hyperref}\ 宏包的一些代码。例如,
% \xpackage{bookmark}\ 宏包重新定义了 \xpackage{hyperref}\ 宏包在 \cs{addcontentsline}\ 命令
% 和其他命令中插入的\cs{Hy@writebookmark}\ 钩子。因此,不应禁用 \xpackage{hyperref}\ 宏包的书签。
%
% \xpackage{bookmark}\ 宏包使用 \xpackage{hyperref}\ 宏包的 \cs{pdfstringdef},且不提供替换(replacement)。
%
% \xpackage{hyperref}\ 宏包的一些选项也能在 \xpackage{bookmark}\ 宏包中实现(implemented):
% \begin{quote}
% \begin{tabular}{|l@{}|l@{}|}
%   \hline
%   \xpackage{hyperref}\ 宏包的选项\  &\ \xpackage{bookmark}\ 宏包的选项\ \ \\ \hline
%   \xoption{bookmarksdepth} &\ \xoption{depth}\\ \hline
%   \xoption{bookmarksopen} & \ \xoption{open}\\ \hline
%   \xoption{bookmarksopenlevel}\ \ \  &\ \xoption{openlevel}\\ \hline
%   \xoption{bookmarksnumbered} \ \ \ &\ \xoption{numbered}\\ \hline
% \end{tabular}
% \end{quote}
%
% 还可以使用以下命令:
% \begin{quote}
%   \cs{pdfbookmark}\\
%   \cs{currentpdfbookmark}\\
%   \cs{subpdfbookmark}\\
%   \cs{belowpdfbookmark}
% \end{quote}
%
% \subsection{在末尾添加书签}
%
% 宏包选项 \xoption{atend}\ 启用以下宏(macro):
% \begin{declcs}{BookmarkAtEnd}
%   \M{stuff}
% \end{declcs}
% \cs{BookmarkAtEnd}\ 宏将 \meta{stuff}\ 放在文档末尾。\meta{stuff}\ 表示书签命令(bookmark commands)。举例:
% \begin{quote}
%\begin{verbatim}
%\usepackage[atend]{bookmark}
%\BookmarkAtEnd{%
%  \bookmarksetup{startatroot}%
%  \bookmark[named=LastPage, level=0]{Last page}%
%}
%\end{verbatim}
% \end{quote}
%
% 或者,可以在 \cs{bookmark}\ 中给出 \xoption{startatroot}\ 选项:
% \begin{quote}
%\begin{verbatim}
%\BookmarkAtEnd{%
%  \bookmark[
%    startatroot,
%    named=LastPage,
%    level=0,
%  ]{Last page}%
%}
%\end{verbatim}
% \end{quote}
%
% \paragraph{备注(Remarks):}
% \begin{itemize}
% \item
%   \cs{BookmarkAtEnd} 隐藏了这样一个事实,即在文档末尾添加书签的方法取决于驱动程序(driver)。
%
%   为此,驱动程序 \xoption{pdftex}\ 使用 \xpackage{atveryend}\ 宏包。如果 \cs{AtEndDocument}\ 太早,
%   最后一个页面(last page)可能不会被发送出去(shipped out)。由于需要 \xext{aux}\ 文件,此驱动程序使
%   用 \cs{AfterLastShipout}。
%
%   其他驱动程序(\xoption{dvipdfm}、\xoption{xetex}、\xoption{vtex})的实现(implementation)
%   取决于 \cs{special},\cs{special}\ 在最后一页之后没有效果。在这种情况下,\xpackage{atenddvi}\ 宏包的
%   \cs{AtEndDvi}\ 有帮助。它将其参数(argument)放在文档的最后一页(last page)。至少需要运行 \hologo{LaTeX}\ 两次,
%   因为最后一页是由引用(reference)检测到的。
%
%   \xoption{dvips}\ 现在使用新的 LaTeX 钩子 \texttt{shipout/lastpage}。
% \item
%   未指定 \cs{BookmarkAtEnd}\ 参数的扩展时间(time of expansion)。这可以立即发生,也可以在文档末尾发生。
% \end{itemize}
%
% \subsection{限制/行动计划}
%
% \begin{itemize}
% \item 支持缺失动作(missing actions)(启动,\dots)。
% \item 对 \xpackage{hyperref}\ 的 \xoption{bookmarkstype}\ 选项进行了更好的设计(design)。
% \end{itemize}
%
% \section{示例(Example)}
%
%    \begin{macrocode}
%<*example>
%    \end{macrocode}
%    \begin{macrocode}
\documentclass{article}
\usepackage{xcolor}[2007/01/21]
\usepackage{hyperref}
\usepackage[
  open,
  openlevel=2,
  atend
]{bookmark}[2019/12/03]

\bookmarksetup{color=blue}

\BookmarkAtEnd{%
  \bookmarksetup{startatroot}%
  \bookmark[named=LastPage, level=0]{End/Last page}%
  \bookmark[named=FirstPage, level=1]{First page}%
}

\begin{document}
\section{First section}
\subsection{Subsection A}
\begin{figure}
  \hypertarget{fig}{}%
  A figure.
\end{figure}
\bookmark[
  rellevel=1,
  keeplevel,
  dest=fig
]{A figure}
\subsection{Subsection B}
\subsubsection{Subsubsection C}
\subsection{Umlauts: \"A\"O\"U\"a\"o\"u\ss}
\newpage
\bookmarksetup{
  bold,
  color=[rgb]{1,0,0}
}
\section{Very important section}
\bookmarksetup{
  italic,
  bold=false,
  color=blue
}
\subsection{Italic section}
\bookmarksetup{
  italic=false
}
\part{Misc}
\section{Diverse}
\subsubsection{Subsubsection, omitting subsection}
\bookmarksetup{
  startatroot
}
\section{Last section outside part}
\subsection{Subsection}
\bookmarksetup{
  color={}
}
\begingroup
  \bookmarksetup{level=0, color=green!80!black}
  \bookmark[named=FirstPage]{First page}
  \bookmark[named=LastPage]{Last page}
  \bookmark[named=PrevPage]{Previous page}
  \bookmark[named=NextPage]{Next page}
\endgroup
\bookmark[
  page=2,
  view=FitH 800
]{Page 2, FitH 800}
\bookmark[
  page=2,
  view=FitBH \calc{\paperheight-\topmargin-1in-\headheight-\headsep}
]{Page 2, FitBH top of text body}
\bookmark[
  uri={http://www.dante.de/},
  color=magenta
]{Dante homepage}
\bookmark[
  gotor={t.pdf},
  page=1,
  view={XYZ 0 1000 null},
  color=cyan!75!black
]{File t.pdf}
\bookmark[named=FirstPage]{First page}
\bookmark[rellevel=1, named=LastPage]{Last page (rellevel=1)}
\bookmark[named=PrevPage]{Previous page}
\bookmark[level=0, named=FirstPage]{First page (level=0)}
\bookmark[
  rellevel=1,
  keeplevel,
  named=LastPage
]{Last page (rellevel=1, keeplevel)}
\bookmark[named=PrevPage]{Previous page}
\end{document}
%    \end{macrocode}
%    \begin{macrocode}
%</example>
%    \end{macrocode}
%
% \StopEventually{
% }
%
% \section{实现(Implementation)}
%
% \subsection{宏包(Package)}
%
%    \begin{macrocode}
%<*package>
\NeedsTeXFormat{LaTeX2e}
\ProvidesPackage{bookmark}%
  [2020-11-06 v1.29 PDF bookmarks (HO)]%
%    \end{macrocode}
%
% \subsubsection{要求(Requirements)}
%
% \paragraph{\hologo{eTeX}.}
%
%    \begin{macro}{\BKM@CalcExpr}
%    \begin{macrocode}
\begingroup\expandafter\expandafter\expandafter\endgroup
\expandafter\ifx\csname numexpr\endcsname\relax
  \def\BKM@CalcExpr#1#2#3#4{%
    \begingroup
      \count@=#2\relax
      \advance\count@ by#3#4\relax
      \edef\x{\endgroup
        \def\noexpand#1{\the\count@}%
      }%
    \x
  }%
\else
  \def\BKM@CalcExpr#1#2#3#4{%
    \edef#1{%
      \the\numexpr#2#3#4\relax
    }%
  }%
\fi
%    \end{macrocode}
%    \end{macro}
%
% \paragraph{\hologo{pdfTeX}\ 的转义功能(escape features)}
%
%    \begin{macro}{\BKM@EscapeName}
%    \begin{macrocode}
\def\BKM@EscapeName#1{%
  \ifx#1\@empty
  \else
    \EdefEscapeName#1#1%
  \fi
}%
%    \end{macrocode}
%    \end{macro}
%    \begin{macro}{\BKM@EscapeString}
%    \begin{macrocode}
\def\BKM@EscapeString#1{%
  \ifx#1\@empty
  \else
    \EdefEscapeString#1#1%
  \fi
}%
%    \end{macrocode}
%    \end{macro}
%    \begin{macro}{\BKM@EscapeHex}
%    \begin{macrocode}
\def\BKM@EscapeHex#1{%
  \ifx#1\@empty
  \else
    \EdefEscapeHex#1#1%
  \fi
}%
%    \end{macrocode}
%    \end{macro}
%    \begin{macro}{\BKM@UnescapeHex}
%    \begin{macrocode}
\def\BKM@UnescapeHex#1{%
  \EdefUnescapeHex#1#1%
}%
%    \end{macrocode}
%    \end{macro}
%
% \paragraph{宏包(Packages)。}
%
% 不要加载由 \xpackage{hyperref}\ 加载的宏包
%    \begin{macrocode}
\RequirePackage{hyperref}[2010/06/18]
%    \end{macrocode}
%
% \subsubsection{宏包选项(Package options)}
%
%    \begin{macrocode}
\SetupKeyvalOptions{family=BKM,prefix=BKM@}
\DeclareLocalOptions{%
  atend,%
  bold,%
  color,%
  depth,%
  dest,%
  draft,%
  final,%
  gotor,%
  italic,%
  keeplevel,%
  level,%
  named,%
  numbered,%
  open,%
  openlevel,%
  page,%
  rawaction,%
  rellevel,%
  srcfile,%
  srcline,%
  startatroot,%
  uri,%
  view,%
}
%    \end{macrocode}
%    \begin{macro}{\bookmarksetup}
%    \begin{macrocode}
\newcommand*{\bookmarksetup}{\kvsetkeys{BKM}}
%    \end{macrocode}
%    \end{macro}
%    \begin{macro}{\BKM@setup}
%    \begin{macrocode}
\def\BKM@setup#1{%
  \bookmarksetup{#1}%
  \ifx\BKM@HookNext\ltx@empty
  \else
    \expandafter\bookmarksetup\expandafter{\BKM@HookNext}%
    \BKM@HookNextClear
  \fi
  \BKM@hook
  \ifBKM@keeplevel
  \else
    \xdef\BKM@currentlevel{\BKM@level}%
  \fi
}
%    \end{macrocode}
%    \end{macro}
%
%    \begin{macro}{\bookmarksetupnext}
%    \begin{macrocode}
\newcommand*{\bookmarksetupnext}[1]{%
  \ltx@GlobalAppendToMacro\BKM@HookNext{,#1}%
}
%    \end{macrocode}
%    \end{macro}
%    \begin{macro}{\BKM@setupnext}
%    \begin{macrocode}
%    \end{macrocode}
%    \end{macro}
%    \begin{macro}{\BKM@HookNextClear}
%    \begin{macrocode}
\def\BKM@HookNextClear{%
  \global\let\BKM@HookNext\ltx@empty
}
%    \end{macrocode}
%    \end{macro}
%    \begin{macro}{\BKM@HookNext}
%    \begin{macrocode}
\BKM@HookNextClear
%    \end{macrocode}
%    \end{macro}
%
%    \begin{macrocode}
\DeclareBoolOption{draft}
\DeclareComplementaryOption{final}{draft}
%    \end{macrocode}
%    \begin{macro}{\BKM@DisableOptions}
%    \begin{macrocode}
\def\BKM@DisableOptions{%
  \DisableKeyvalOption[action=warning,package=bookmark]%
      {BKM}{draft}%
  \DisableKeyvalOption[action=warning,package=bookmark]%
      {BKM}{final}%
}
%    \end{macrocode}
%    \end{macro}
%    \begin{macrocode}
\DeclareBoolOption[\ifHy@bookmarksopen true\else false\fi]{open}
%    \end{macrocode}
%    \begin{macro}{\bookmark@open}
%    \begin{macrocode}
\def\bookmark@open{%
  \ifBKM@open\ltx@one\else\ltx@zero\fi
}
%    \end{macrocode}
%    \end{macro}
%    \begin{macrocode}
\DeclareStringOption[\maxdimen]{openlevel}
%    \end{macrocode}
%    \begin{macro}{\BKM@openlevel}
%    \begin{macrocode}
\edef\BKM@openlevel{\number\@bookmarksopenlevel}
%    \end{macrocode}
%    \end{macro}
%    \begin{macrocode}
%\DeclareStringOption[\c@tocdepth]{depth}
\ltx@IfUndefined{Hy@bookmarksdepth}{%
  \def\BKM@depth{\c@tocdepth}%
}{%
  \let\BKM@depth\Hy@bookmarksdepth
}
\define@key{BKM}{depth}[]{%
  \edef\BKM@param{#1}%
  \ifx\BKM@param\@empty
    \def\BKM@depth{\c@tocdepth}%
  \else
    \ltx@IfUndefined{toclevel@\BKM@param}{%
      \@onelevel@sanitize\BKM@param
      \edef\BKM@temp{\expandafter\@car\BKM@param\@nil}%
      \ifcase 0\expandafter\ifx\BKM@temp-1\fi
              \expandafter\ifnum\expandafter`\BKM@temp>47 %
                \expandafter\ifnum\expandafter`\BKM@temp<58 %
                  1%
                \fi
              \fi
              \relax
        \PackageWarning{bookmark}{%
          Unknown document division name (\BKM@param)\MessageBreak
          for option `depth'%
        }%
      \else
        \BKM@SetDepthOrLevel\BKM@depth\BKM@param
      \fi
    }{%
      \BKM@SetDepthOrLevel\BKM@depth{%
        \csname toclevel@\BKM@param\endcsname
      }%
    }%
  \fi
}
%    \end{macrocode}
%    \begin{macro}{\bookmark@depth}
%    \begin{macrocode}
\def\bookmark@depth{\BKM@depth}
%    \end{macrocode}
%    \end{macro}
%    \begin{macro}{\BKM@SetDepthOrLevel}
%    \begin{macrocode}
\def\BKM@SetDepthOrLevel#1#2{%
  \begingroup
    \setbox\z@=\hbox{%
      \count@=#2\relax
      \expandafter
    }%
  \expandafter\endgroup
  \expandafter\def\expandafter#1\expandafter{\the\count@}%
}
%    \end{macrocode}
%    \end{macro}
%    \begin{macrocode}
\DeclareStringOption[\BKM@currentlevel]{level}[\BKM@currentlevel]
\define@key{BKM}{level}{%
  \edef\BKM@param{#1}%
  \ifx\BKM@param\BKM@MacroCurrentLevel
    \let\BKM@level\BKM@param
  \else
    \ltx@IfUndefined{toclevel@\BKM@param}{%
      \@onelevel@sanitize\BKM@param
      \edef\BKM@temp{\expandafter\@car\BKM@param\@nil}%
      \ifcase 0\expandafter\ifx\BKM@temp-1\fi
              \expandafter\ifnum\expandafter`\BKM@temp>47 %
                \expandafter\ifnum\expandafter`\BKM@temp<58 %
                  1%
                \fi
              \fi
              \relax
        \PackageWarning{bookmark}{%
          Unknown document division name (\BKM@param)\MessageBreak
          for option `level'%
        }%
      \else
        \BKM@SetDepthOrLevel\BKM@level\BKM@param
      \fi
    }{%
      \BKM@SetDepthOrLevel\BKM@level{%
        \csname toclevel@\BKM@param\endcsname
      }%
    }%
  \fi
}
%    \end{macrocode}
%    \begin{macro}{\BKM@MacroCurrentLevel}
%    \begin{macrocode}
\def\BKM@MacroCurrentLevel{\BKM@currentlevel}
%    \end{macrocode}
%    \end{macro}
%    \begin{macrocode}
\DeclareBoolOption{keeplevel}
\DeclareBoolOption{startatroot}
%    \end{macrocode}
%    \begin{macro}{\BKM@startatrootfalse}
%    \begin{macrocode}
\def\BKM@startatrootfalse{%
  \global\let\ifBKM@startatroot\iffalse
}
%    \end{macrocode}
%    \end{macro}
%    \begin{macro}{\BKM@startatroottrue}
%    \begin{macrocode}
\def\BKM@startatroottrue{%
  \global\let\ifBKM@startatroot\iftrue
}
%    \end{macrocode}
%    \end{macro}
%    \begin{macrocode}
\define@key{BKM}{rellevel}{%
  \BKM@CalcExpr\BKM@level{#1}+\BKM@currentlevel
}
%    \end{macrocode}
%    \begin{macro}{\bookmark@level}
%    \begin{macrocode}
\def\bookmark@level{\BKM@level}
%    \end{macrocode}
%    \end{macro}
%    \begin{macro}{\BKM@currentlevel}
%    \begin{macrocode}
\def\BKM@currentlevel{0}
%    \end{macrocode}
%    \end{macro}
%    Make \xpackage{bookmark}'s option \xoption{numbered} an alias
%    for \xpackage{hyperref}'s \xoption{bookmarksnumbered}.
%    \begin{macrocode}
\DeclareBoolOption[%
  \ifHy@bookmarksnumbered true\else false\fi
]{numbered}
\g@addto@macro\BKM@numberedtrue{%
  \let\ifHy@bookmarksnumbered\iftrue
}
\g@addto@macro\BKM@numberedfalse{%
  \let\ifHy@bookmarksnumbered\iffalse
}
\g@addto@macro\Hy@bookmarksnumberedtrue{%
  \let\ifBKM@numbered\iftrue
}
\g@addto@macro\Hy@bookmarksnumberedfalse{%
  \let\ifBKM@numbered\iffalse
}
%    \end{macrocode}
%    \begin{macro}{\bookmark@numbered}
%    \begin{macrocode}
\def\bookmark@numbered{%
  \ifBKM@numbered\ltx@one\else\ltx@zero\fi
}
%    \end{macrocode}
%    \end{macro}
%
% \paragraph{重定义 \xpackage{hyperref}\ 宏包的选项}
%
%    \begin{macro}{\BKM@PatchHyperrefOption}
%    \begin{macrocode}
\def\BKM@PatchHyperrefOption#1{%
  \expandafter\BKM@@PatchHyperrefOption\csname KV@Hyp@#1\endcsname%
}
%    \end{macrocode}
%    \end{macro}
%    \begin{macro}{\BKM@@PatchHyperrefOption}
%    \begin{macrocode}
\def\BKM@@PatchHyperrefOption#1{%
  \expandafter\BKM@@@PatchHyperrefOption#1{##1}\BKM@nil#1%
}
%    \end{macrocode}
%    \end{macro}
%    \begin{macro}{\BKM@@@PatchHyperrefOption}
%    \begin{macrocode}
\def\BKM@@@PatchHyperrefOption#1\BKM@nil#2#3{%
  \def#2##1{%
    #1%
    \bookmarksetup{#3={##1}}%
  }%
}
%    \end{macrocode}
%    \end{macro}
%    \begin{macrocode}
\BKM@PatchHyperrefOption{bookmarksopen}{open}
\BKM@PatchHyperrefOption{bookmarksopenlevel}{openlevel}
\BKM@PatchHyperrefOption{bookmarksdepth}{depth}
%    \end{macrocode}
%
% \paragraph{字体样式(font style)选项。}
%
%    注意:\xpackage{bitset}\ 宏是基于零的,PDF 规范(PDF specifications)以1开头。
%    \begin{macrocode}
\bitsetReset{BKM@FontStyle}%
\define@key{BKM}{italic}[true]{%
  \expandafter\ifx\csname if#1\endcsname\iftrue
    \bitsetSet{BKM@FontStyle}{0}%
  \else
    \bitsetClear{BKM@FontStyle}{0}%
  \fi
}%
\define@key{BKM}{bold}[true]{%
  \expandafter\ifx\csname if#1\endcsname\iftrue
    \bitsetSet{BKM@FontStyle}{1}%
  \else
    \bitsetClear{BKM@FontStyle}{1}%
  \fi
}%
%    \end{macrocode}
%    \begin{macro}{\bookmark@italic}
%    \begin{macrocode}
\def\bookmark@italic{%
  \ifnum\bitsetGet{BKM@FontStyle}{0}=1 \ltx@one\else\ltx@zero\fi
}
%    \end{macrocode}
%    \end{macro}
%    \begin{macro}{\bookmark@bold}
%    \begin{macrocode}
\def\bookmark@bold{%
  \ifnum\bitsetGet{BKM@FontStyle}{1}=1 \ltx@one\else\ltx@zero\fi
}
%    \end{macrocode}
%    \end{macro}
%    \begin{macro}{\BKM@PrintStyle}
%    \begin{macrocode}
\def\BKM@PrintStyle{%
  \bitsetGetDec{BKM@FontStyle}%
}%
%    \end{macrocode}
%    \end{macro}
%
% \paragraph{颜色(color)选项。}
%
%    \begin{macrocode}
\define@key{BKM}{color}{%
  \HyColor@BookmarkColor{#1}\BKM@color{bookmark}{color}%
}
%    \end{macrocode}
%    \begin{macro}{\BKM@color}
%    \begin{macrocode}
\let\BKM@color\@empty
%    \end{macrocode}
%    \end{macro}
%    \begin{macro}{\bookmark@color}
%    \begin{macrocode}
\def\bookmark@color{\BKM@color}
%    \end{macrocode}
%    \end{macro}
%
% \subsubsection{动作(action)选项}
%
%    \begin{macrocode}
\def\BKM@temp#1{%
  \DeclareStringOption{#1}%
  \expandafter\edef\csname bookmark@#1\endcsname{%
    \expandafter\noexpand\csname BKM@#1\endcsname
  }%
}
%    \end{macrocode}
%    \begin{macro}{\bookmark@dest}
%    \begin{macrocode}
\BKM@temp{dest}
%    \end{macrocode}
%    \end{macro}
%    \begin{macro}{\bookmark@named}
%    \begin{macrocode}
\BKM@temp{named}
%    \end{macrocode}
%    \end{macro}
%    \begin{macro}{\bookmark@uri}
%    \begin{macrocode}
\BKM@temp{uri}
%    \end{macrocode}
%    \end{macro}
%    \begin{macro}{\bookmark@gotor}
%    \begin{macrocode}
\BKM@temp{gotor}
%    \end{macrocode}
%    \end{macro}
%    \begin{macro}{\bookmark@rawaction}
%    \begin{macrocode}
\BKM@temp{rawaction}
%    \end{macrocode}
%    \end{macro}
%
%    \begin{macrocode}
\define@key{BKM}{page}{%
  \def\BKM@page{#1}%
  \ifx\BKM@page\@empty
  \else
    \edef\BKM@page{\number\BKM@page}%
    \ifnum\BKM@page>\z@
    \else
      \PackageError{bookmark}{Page must be positive}\@ehc
      \def\BKM@page{1}%
    \fi
  \fi
}
%    \end{macrocode}
%    \begin{macro}{\BKM@page}
%    \begin{macrocode}
\let\BKM@page\@empty
%    \end{macrocode}
%    \end{macro}
%    \begin{macro}{\bookmark@page}
%    \begin{macrocode}
\def\bookmark@page{\BKM@@page}
%    \end{macrocode}
%    \end{macro}
%
%    \begin{macrocode}
\define@key{BKM}{view}{%
  \BKM@CheckView{#1}%
}
%    \end{macrocode}
%    \begin{macro}{\BKM@view}
%    \begin{macrocode}
\let\BKM@view\@empty
%    \end{macrocode}
%    \end{macro}
%    \begin{macro}{\bookmark@view}
%    \begin{macrocode}
\def\bookmark@view{\BKM@view}
%    \end{macrocode}
%    \end{macro}
%    \begin{macro}{BKM@CheckView}
%    \begin{macrocode}
\def\BKM@CheckView#1{%
  \BKM@CheckViewType#1 \@nil
}
%    \end{macrocode}
%    \end{macro}
%    \begin{macro}{\BKM@CheckViewType}
%    \begin{macrocode}
\def\BKM@CheckViewType#1 #2\@nil{%
  \def\BKM@type{#1}%
  \@onelevel@sanitize\BKM@type
  \BKM@TestViewType{Fit}{}%
  \BKM@TestViewType{FitB}{}%
  \BKM@TestViewType{FitH}{%
    \BKM@CheckParam#2 \@nil{top}%
  }%
  \BKM@TestViewType{FitBH}{%
    \BKM@CheckParam#2 \@nil{top}%
  }%
  \BKM@TestViewType{FitV}{%
    \BKM@CheckParam#2 \@nil{bottom}%
  }%
  \BKM@TestViewType{FitBV}{%
    \BKM@CheckParam#2 \@nil{bottom}%
  }%
  \BKM@TestViewType{FitR}{%
    \BKM@CheckRect{#2}{ }%
  }%
  \BKM@TestViewType{XYZ}{%
    \BKM@CheckXYZ{#2}{ }%
  }%
  \@car{%
    \PackageError{bookmark}{%
      Unknown view type `\BKM@type',\MessageBreak
      using `FitH' instead%
    }\@ehc
    \def\BKM@view{FitH}%
  }%
  \@nil
}
%    \end{macrocode}
%    \end{macro}
%    \begin{macro}{\BKM@TestViewType}
%    \begin{macrocode}
\def\BKM@TestViewType#1{%
  \def\BKM@temp{#1}%
  \@onelevel@sanitize\BKM@temp
  \ifx\BKM@type\BKM@temp
    \let\BKM@view\BKM@temp
    \expandafter\@car
  \else
    \expandafter\@gobble
  \fi
}
%    \end{macrocode}
%    \end{macro}
%    \begin{macro}{BKM@CheckParam}
%    \begin{macrocode}
\def\BKM@CheckParam#1 #2\@nil#3{%
  \def\BKM@param{#1}%
  \ifx\BKM@param\@empty
    \PackageWarning{bookmark}{%
      Missing parameter (#3) for `\BKM@type',\MessageBreak
      using 0%
    }%
    \def\BKM@param{0}%
  \else
    \BKM@CalcParam
  \fi
  \edef\BKM@view{\BKM@view\space\BKM@param}%
}
%    \end{macrocode}
%    \end{macro}
%    \begin{macro}{BKM@CheckRect}
%    \begin{macrocode}
\def\BKM@CheckRect#1#2{%
  \BKM@@CheckRect#1#2#2#2#2\@nil
}
%    \end{macrocode}
%    \end{macro}
%    \begin{macro}{\BKM@@CheckRect}
%    \begin{macrocode}
\def\BKM@@CheckRect#1 #2 #3 #4 #5\@nil{%
  \def\BKM@temp{0}%
  \def\BKM@param{#1}%
  \ifx\BKM@param\@empty
    \def\BKM@param{0}%
    \def\BKM@temp{1}%
  \else
    \BKM@CalcParam
  \fi
  \edef\BKM@view{\BKM@view\space\BKM@param}%
  \def\BKM@param{#2}%
  \ifx\BKM@param\@empty
    \def\BKM@param{0}%
    \def\BKM@temp{1}%
  \else
    \BKM@CalcParam
  \fi
  \edef\BKM@view{\BKM@view\space\BKM@param}%
  \def\BKM@param{#3}%
  \ifx\BKM@param\@empty
    \def\BKM@param{0}%
    \def\BKM@temp{1}%
  \else
    \BKM@CalcParam
  \fi
  \edef\BKM@view{\BKM@view\space\BKM@param}%
  \def\BKM@param{#4}%
  \ifx\BKM@param\@empty
    \def\BKM@param{0}%
    \def\BKM@temp{1}%
  \else
    \BKM@CalcParam
  \fi
  \edef\BKM@view{\BKM@view\space\BKM@param}%
  \ifnum\BKM@temp>\z@
    \PackageWarning{bookmark}{Missing parameters for `\BKM@type'}%
  \fi
}
%    \end{macrocode}
%    \end{macro}
%    \begin{macro}{\BKM@CheckXYZ}
%    \begin{macrocode}
\def\BKM@CheckXYZ#1#2{%
  \BKM@@CheckXYZ#1#2#2#2\@nil
}
%    \end{macrocode}
%    \end{macro}
%    \begin{macro}{\BKM@@CheckXYZ}
%    \begin{macrocode}
\def\BKM@@CheckXYZ#1 #2 #3 #4\@nil{%
  \def\BKM@param{#1}%
  \let\BKM@temp\BKM@param
  \@onelevel@sanitize\BKM@temp
  \ifx\BKM@param\@empty
    \let\BKM@param\BKM@null
  \else
    \ifx\BKM@temp\BKM@null
    \else
      \BKM@CalcParam
    \fi
  \fi
  \edef\BKM@view{\BKM@view\space\BKM@param}%
  \def\BKM@param{#2}%
  \let\BKM@temp\BKM@param
  \@onelevel@sanitize\BKM@temp
  \ifx\BKM@param\@empty
    \let\BKM@param\BKM@null
  \else
    \ifx\BKM@temp\BKM@null
    \else
      \BKM@CalcParam
    \fi
  \fi
  \edef\BKM@view{\BKM@view\space\BKM@param}%
  \def\BKM@param{#3}%
  \ifx\BKM@param\@empty
    \let\BKM@param\BKM@null
  \fi
  \edef\BKM@view{\BKM@view\space\BKM@param}%
}
%    \end{macrocode}
%    \end{macro}
%    \begin{macro}{\BKM@null}
%    \begin{macrocode}
\def\BKM@null{null}
\@onelevel@sanitize\BKM@null
%    \end{macrocode}
%    \end{macro}
%
%    \begin{macro}{\BKM@CalcParam}
%    \begin{macrocode}
\def\BKM@CalcParam{%
  \begingroup
  \let\calc\@firstofone
  \expandafter\BKM@@CalcParam\BKM@param\@empty\@empty\@nil
}
%    \end{macrocode}
%    \end{macro}
%    \begin{macro}{\BKM@@CalcParam}
%    \begin{macrocode}
\def\BKM@@CalcParam#1#2#3\@nil{%
  \ifx\calc#1%
    \@ifundefined{calc@assign@dimen}{%
      \@ifundefined{dimexpr}{%
        \setlength{\dimen@}{#2}%
      }{%
        \setlength{\dimen@}{\dimexpr#2\relax}%
      }%
    }{%
      \setlength{\dimen@}{#2}%
    }%
    \dimen@.99626\dimen@
    \edef\BKM@param{\strip@pt\dimen@}%
    \expandafter\endgroup
    \expandafter\def\expandafter\BKM@param\expandafter{\BKM@param}%
  \else
    \endgroup
  \fi
}
%    \end{macrocode}
%    \end{macro}
%
% \subsubsection{\xoption{atend}\ 选项}
%
%    \begin{macrocode}
\DeclareBoolOption{atend}
\g@addto@macro\BKM@DisableOptions{%
  \DisableKeyvalOption[action=warning,package=bookmark]%
      {BKM}{atend}%
}
%    \end{macrocode}
%
% \subsubsection{\xoption{style}\ 选项}
%
%    \begin{macro}{\bookmarkdefinestyle}
%    \begin{macrocode}
\newcommand*{\bookmarkdefinestyle}[2]{%
  \@ifundefined{BKM@style@#1}{%
  }{%
    \PackageInfo{bookmark}{Redefining style `#1'}%
  }%
  \@namedef{BKM@style@#1}{#2}%
}
%    \end{macrocode}
%    \end{macro}
%    \begin{macrocode}
\define@key{BKM}{style}{%
  \BKM@StyleCall{#1}%
}
\newif\ifBKM@ok
%    \end{macrocode}
%    \begin{macro}{\BKM@StyleCall}
%    \begin{macrocode}
\def\BKM@StyleCall#1{%
  \@ifundefined{BKM@style@#1}{%
    \PackageWarning{bookmark}{%
      Ignoring unknown style `#1'%
    }%
  }{%
%    \end{macrocode}
%    检查样式堆栈(style stack)。
%    \begin{macrocode}
    \BKM@oktrue
    \edef\BKM@StyleCurrent{#1}%
    \@onelevel@sanitize\BKM@StyleCurrent
    \let\BKM@StyleEntry\BKM@StyleEntryCheck
    \BKM@StyleStack
    \ifBKM@ok
      \expandafter\@firstofone
    \else
      \PackageError{bookmark}{%
        Ignoring recursive call of style `\BKM@StyleCurrent'%
      }\@ehc
      \expandafter\@gobble
    \fi
    {%
%    \end{macrocode}
%    在堆栈上推送当前样式(Push current style on stack)。
%    \begin{macrocode}
      \let\BKM@StyleEntry\relax
      \edef\BKM@StyleStack{%
        \BKM@StyleEntry{\BKM@StyleCurrent}%
        \BKM@StyleStack
      }%
%    \end{macrocode}
%   调用样式(Call style)。
%    \begin{macrocode}
      \expandafter\expandafter\expandafter\bookmarksetup
      \expandafter\expandafter\expandafter{%
        \csname BKM@style@\BKM@StyleCurrent\endcsname
      }%
%    \end{macrocode}
%    从堆栈中弹出当前样式(Pop current style from stack)。
%    \begin{macrocode}
      \BKM@StyleStackPop
    }%
  }%
}
%    \end{macrocode}
%    \end{macro}
%    \begin{macro}{\BKM@StyleStackPop}
%    \begin{macrocode}
\def\BKM@StyleStackPop{%
  \let\BKM@StyleEntry\relax
  \edef\BKM@StyleStack{%
    \expandafter\@gobbletwo\BKM@StyleStack
  }%
}
%    \end{macrocode}
%    \end{macro}
%    \begin{macro}{\BKM@StyleEntryCheck}
%    \begin{macrocode}
\def\BKM@StyleEntryCheck#1{%
  \def\BKM@temp{#1}%
  \ifx\BKM@temp\BKM@StyleCurrent
    \BKM@okfalse
  \fi
}
%    \end{macrocode}
%    \end{macro}
%    \begin{macro}{\BKM@StyleStack}
%    \begin{macrocode}
\def\BKM@StyleStack{}
%    \end{macrocode}
%    \end{macro}
%
% \subsubsection{源文件位置(source file location)选项}
%
%    \begin{macrocode}
\DeclareStringOption{srcline}
\DeclareStringOption{srcfile}
%    \end{macrocode}
%
% \subsubsection{钩子支持(Hook support)}
%
%    \begin{macro}{\BKM@hook}
%    \begin{macrocode}
\def\BKM@hook{}
%    \end{macrocode}
%    \end{macro}
%    \begin{macrocode}
\define@key{BKM}{addtohook}{%
  \ltx@LocalAppendToMacro\BKM@hook{#1}%
}
%    \end{macrocode}
%
%    \begin{macro}{bookmarkget}
%    \begin{macrocode}
\newcommand*{\bookmarkget}[1]{%
  \romannumeral0%
  \ltx@ifundefined{bookmark@#1}{%
    \ltx@space
  }{%
    \expandafter\expandafter\expandafter\ltx@space
    \csname bookmark@#1\endcsname
  }%
}
%    \end{macrocode}
%    \end{macro}
%
% \subsubsection{设置和加载驱动程序}
%
% \paragraph{检测驱动程序。}
%
%    \begin{macro}{\BKM@DefineDriverKey}
%    \begin{macrocode}
\def\BKM@DefineDriverKey#1{%
  \define@key{BKM}{#1}[]{%
    \def\BKM@driver{#1}%
  }%
  \g@addto@macro\BKM@DisableOptions{%
    \DisableKeyvalOption[action=warning,package=bookmark]%
        {BKM}{#1}%
  }%
}
%    \end{macrocode}
%    \end{macro}
%    \begin{macrocode}
\BKM@DefineDriverKey{pdftex}
\BKM@DefineDriverKey{dvips}
\BKM@DefineDriverKey{dvipdfm}
\BKM@DefineDriverKey{dvipdfmx}
\BKM@DefineDriverKey{xetex}
\BKM@DefineDriverKey{vtex}
\define@key{BKM}{dvipdfmx-outline-open}[true]{%
 \PackageWarning{bookmark}{Option 'dvipdfmx-outline-open' is obsolete
   and ignored}{}}
%    \end{macrocode}
%    \begin{macro}{\bookmark@driver}
%    \begin{macrocode}
\def\bookmark@driver{\BKM@driver}
%    \end{macrocode}
%    \end{macro}
%    \begin{macrocode}
\InputIfFileExists{bookmark.cfg}{}{}
%    \end{macrocode}
%    \begin{macro}{\BookmarkDriverDefault}
%    \begin{macrocode}
\providecommand*{\BookmarkDriverDefault}{dvips}
%    \end{macrocode}
%    \end{macro}
%    \begin{macro}{\BKM@driver}
% Lua\TeX\ 和 pdf\TeX\ 共享驱动程序。
%    \begin{macrocode}
\ifpdf
  \def\BKM@driver{pdftex}%
  \ifx\pdfoutline\@undefined
    \ifx\pdfextension\@undefined\else
      \protected\def\pdfoutline{\pdfextension outline }
    \fi
  \fi
\else
  \ifxetex
    \def\BKM@driver{dvipdfm}%
  \else
    \ifvtex
      \def\BKM@driver{vtex}%
    \else
      \edef\BKM@driver{\BookmarkDriverDefault}%
    \fi
  \fi
\fi
%    \end{macrocode}
%    \end{macro}
%
% \paragraph{过程选项(Process options)。}
%
%    \begin{macrocode}
\ProcessKeyvalOptions*
\BKM@DisableOptions
%    \end{macrocode}
%
% \paragraph{\xoption{draft}\ 选项}
%
%    \begin{macrocode}
\ifBKM@draft
  \PackageWarningNoLine{bookmark}{Draft mode on}%
  \let\bookmarksetup\ltx@gobble
  \let\BookmarkAtEnd\ltx@gobble
  \let\bookmarkdefinestyle\ltx@gobbletwo
  \let\bookmarkget\ltx@gobble
  \let\pdfbookmark\ltx@undefined
  \newcommand*{\pdfbookmark}[3][]{}%
  \let\currentpdfbookmark\ltx@gobbletwo
  \let\subpdfbookmark\ltx@gobbletwo
  \let\belowpdfbookmark\ltx@gobbletwo
  \newcommand*{\bookmark}[2][]{}%
  \renewcommand*{\Hy@writebookmark}[5]{}%
  \let\ReadBookmarks\relax
  \let\BKM@DefGotoNameAction\ltx@gobbletwo % package `hypdestopt'
  \expandafter\endinput
\fi
%    \end{macrocode}
%
% \paragraph{验证和加载驱动程序。}
%
%    \begin{macrocode}
\def\BKM@temp{dvipdfmx}%
\ifx\BKM@temp\BKM@driver
  \def\BKM@driver{dvipdfm}%
\fi
\def\BKM@temp{pdftex}%
\ifpdf
  \ifx\BKM@temp\BKM@driver
  \else
    \PackageWarningNoLine{bookmark}{%
      Wrong driver `\BKM@driver', using `pdftex' instead%
    }%
    \let\BKM@driver\BKM@temp
  \fi
\else
  \ifx\BKM@temp\BKM@driver
    \PackageError{bookmark}{%
      Wrong driver, pdfTeX is not running in PDF mode.\MessageBreak
      Package loading is aborted%
    }\@ehc
    \expandafter\expandafter\expandafter\endinput
  \fi
  \def\BKM@temp{dvipdfm}%
  \ifxetex
    \ifx\BKM@temp\BKM@driver
    \else
      \PackageWarningNoLine{bookmark}{%
        Wrong driver `\BKM@driver',\MessageBreak
        using `dvipdfm' for XeTeX instead%
      }%
      \let\BKM@driver\BKM@temp
    \fi
  \else
    \def\BKM@temp{vtex}%
    \ifvtex
      \ifx\BKM@temp\BKM@driver
      \else
        \PackageWarningNoLine{bookmark}{%
          Wrong driver `\BKM@driver',\MessageBreak
          using `vtex' for VTeX instead%
        }%
        \let\BKM@driver\BKM@temp
      \fi
    \else
      \ifx\BKM@temp\BKM@driver
        \PackageError{bookmark}{%
          Wrong driver, VTeX is not running in PDF mode.\MessageBreak
          Package loading is aborted%
        }\@ehc
        \expandafter\expandafter\expandafter\endinput
      \fi
    \fi
  \fi
\fi
\providecommand\IfFormatAtLeastTF{\@ifl@t@r\fmtversion}
\IfFormatAtLeastTF{2020/10/01}{}{\edef\BKM@driver{\BKM@driver-2019-12-03}}
\InputIfFileExists{bkm-\BKM@driver.def}{}{%
  \PackageError{bookmark}{%
    Unsupported driver `\BKM@driver'.\MessageBreak
    Package loading is aborted%
  }\@ehc
  \endinput
}
%    \end{macrocode}
%
% \subsubsection{与 \xpackage{hyperref}\ 的兼容性}
%
%    \begin{macro}{\pdfbookmark}
%    \begin{macrocode}
\let\pdfbookmark\ltx@undefined
\newcommand*{\pdfbookmark}[3][0]{%
  \bookmark[level=#1,dest={#3.#1}]{#2}%
  \hyper@anchorstart{#3.#1}\hyper@anchorend
}
%    \end{macrocode}
%    \end{macro}
%    \begin{macro}{\currentpdfbookmark}
%    \begin{macrocode}
\def\currentpdfbookmark{%
  \pdfbookmark[\BKM@currentlevel]%
}
%    \end{macrocode}
%    \end{macro}
%    \begin{macro}{\subpdfbookmark}
%    \begin{macrocode}
\def\subpdfbookmark{%
  \BKM@CalcExpr\BKM@CalcResult\BKM@currentlevel+1%
  \expandafter\pdfbookmark\expandafter[\BKM@CalcResult]%
}
%    \end{macrocode}
%    \end{macro}
%    \begin{macro}{\belowpdfbookmark}
%    \begin{macrocode}
\def\belowpdfbookmark#1#2{%
  \xdef\BKM@gtemp{\number\BKM@currentlevel}%
  \subpdfbookmark{#1}{#2}%
  \global\let\BKM@currentlevel\BKM@gtemp
}
%    \end{macrocode}
%    \end{macro}
%
%    节号(section number)、文本(text)、标签(label)、级别(level)、文件(file)
%    \begin{macro}{\Hy@writebookmark}
%    \begin{macrocode}
\def\Hy@writebookmark#1#2#3#4#5{%
  \ifnum#4>\BKM@depth\relax
  \else
    \def\BKM@type{#5}%
    \ifx\BKM@type\Hy@bookmarkstype
      \begingroup
        \ifBKM@numbered
          \let\numberline\Hy@numberline
          \let\booknumberline\Hy@numberline
          \let\partnumberline\Hy@numberline
          \let\chapternumberline\Hy@numberline
        \else
          \let\numberline\@gobble
          \let\booknumberline\@gobble
          \let\partnumberline\@gobble
          \let\chapternumberline\@gobble
        \fi
        \bookmark[level=#4,dest={\HyperDestNameFilter{#3}}]{#2}%
      \endgroup
    \fi
  \fi
}
%    \end{macrocode}
%    \end{macro}
%
%    \begin{macro}{\ReadBookmarks}
%    \begin{macrocode}
\let\ReadBookmarks\relax
%    \end{macrocode}
%    \end{macro}
%
%    \begin{macrocode}
%</package>
%    \end{macrocode}
%
% \subsection{dvipdfm 的驱动程序}
%
%    \begin{macrocode}
%<*dvipdfm>
\NeedsTeXFormat{LaTeX2e}
\ProvidesFile{bkm-dvipdfm.def}%
  [2020-11-06 v1.29 bookmark driver for dvipdfm (HO)]%
%    \end{macrocode}
%
%    \begin{macro}{\BKM@id}
%    \begin{macrocode}
\newcount\BKM@id
\BKM@id=\z@
%    \end{macrocode}
%    \end{macro}
%
%    \begin{macro}{\BKM@0}
%    \begin{macrocode}
\@namedef{BKM@0}{000}
%    \end{macrocode}
%    \end{macro}
%    \begin{macro}{\ifBKM@sw}
%    \begin{macrocode}
\newif\ifBKM@sw
%    \end{macrocode}
%    \end{macro}
%
%    \begin{macro}{\bookmark}
%    \begin{macrocode}
\newcommand*{\bookmark}[2][]{%
  \if@filesw
    \begingroup
      \def\bookmark@text{#2}%
      \BKM@setup{#1}%
      \edef\BKM@prev{\the\BKM@id}%
      \global\advance\BKM@id\@ne
      \BKM@swtrue
      \@whilesw\ifBKM@sw\fi{%
        \def\BKM@abslevel{1}%
        \ifnum\ifBKM@startatroot\z@\else\BKM@prev\fi=\z@
          \BKM@startatrootfalse
          \expandafter\xdef\csname BKM@\the\BKM@id\endcsname{%
            0{\BKM@level}\BKM@abslevel
          }%
          \BKM@swfalse
        \else
          \expandafter\expandafter\expandafter\BKM@getx
              \csname BKM@\BKM@prev\endcsname
          \ifnum\BKM@level>\BKM@x@level\relax
            \BKM@CalcExpr\BKM@abslevel\BKM@x@abslevel+1%
            \expandafter\xdef\csname BKM@\the\BKM@id\endcsname{%
              {\BKM@prev}{\BKM@level}\BKM@abslevel
            }%
            \BKM@swfalse
          \else
            \let\BKM@prev\BKM@x@parent
          \fi
        \fi
      }%
      \csname HyPsd@XeTeXBigCharstrue\endcsname
      \pdfstringdef\BKM@title{\bookmark@text}%
      \edef\BKM@FLAGS{\BKM@PrintStyle}%
      \let\BKM@action\@empty
      \ifx\BKM@gotor\@empty
        \ifx\BKM@dest\@empty
          \ifx\BKM@named\@empty
            \ifx\BKM@rawaction\@empty
              \ifx\BKM@uri\@empty
                \ifx\BKM@page\@empty
                  \PackageError{bookmark}{Missing action}\@ehc
                  \edef\BKM@action{/Dest[@page1/Fit]}%
                \else
                  \ifx\BKM@view\@empty
                    \def\BKM@view{Fit}%
                  \fi
                  \edef\BKM@action{/Dest[@page\BKM@page/\BKM@view]}%
                \fi
              \else
                \BKM@EscapeString\BKM@uri
                \edef\BKM@action{%
                  /A<<%
                    /S/URI%
                    /URI(\BKM@uri)%
                  >>%
                }%
              \fi
            \else
              \edef\BKM@action{/A<<\BKM@rawaction>>}%
            \fi
          \else
            \BKM@EscapeName\BKM@named
            \edef\BKM@action{%
              /A<</S/Named/N/\BKM@named>>%
            }%
          \fi
        \else
          \BKM@EscapeString\BKM@dest
          \edef\BKM@action{%
            /A<<%
              /S/GoTo%
              /D(\BKM@dest)%
            >>%
          }%
        \fi
      \else
        \ifx\BKM@dest\@empty
          \ifx\BKM@page\@empty
            \def\BKM@page{0}%
          \else
            \BKM@CalcExpr\BKM@page\BKM@page-1%
          \fi
          \ifx\BKM@view\@empty
            \def\BKM@view{Fit}%
          \fi
          \edef\BKM@action{/D[\BKM@page/\BKM@view]}%
        \else
          \BKM@EscapeString\BKM@dest
          \edef\BKM@action{/D(\BKM@dest)}%
        \fi
        \BKM@EscapeString\BKM@gotor
        \edef\BKM@action{%
          /A<<%
            /S/GoToR%
            /F(\BKM@gotor)%
            \BKM@action
          >>%
        }%
      \fi
      \special{pdf:%
        out
              [%
              \ifBKM@open
                \ifnum\BKM@level<%
                    \expandafter\ltx@firstofone\expandafter
                    {\number\BKM@openlevel} %
                \else
                  -%
                \fi
              \else
                -%
              \fi
              ] %
            \BKM@abslevel
        <<%
          /Title(\BKM@title)%
          \ifx\BKM@color\@empty
          \else
            /C[\BKM@color]%
          \fi
          \ifnum\BKM@FLAGS>\z@
            /F \BKM@FLAGS
          \fi
          \BKM@action
        >>%
      }%
    \endgroup
  \fi
}
%    \end{macrocode}
%    \end{macro}
%    \begin{macro}{\BKM@getx}
%    \begin{macrocode}
\def\BKM@getx#1#2#3{%
  \def\BKM@x@parent{#1}%
  \def\BKM@x@level{#2}%
  \def\BKM@x@abslevel{#3}%
}
%    \end{macrocode}
%    \end{macro}
%
%    \begin{macrocode}
%</dvipdfm>
%    \end{macrocode}
%
% \subsection{\hologo{VTeX}\ 的驱动程序}
%
%    \begin{macrocode}
%<*vtex>
\NeedsTeXFormat{LaTeX2e}
\ProvidesFile{bkm-vtex.def}%
  [2020-11-06 v1.29 bookmark driver for VTeX (HO)]%
%    \end{macrocode}
%
%    \begin{macrocode}
\ifvtexpdf
\else
  \PackageWarningNoLine{bookmark}{%
    The VTeX driver only supports PDF mode%
  }%
\fi
%    \end{macrocode}
%
%    \begin{macro}{\BKM@id}
%    \begin{macrocode}
\newcount\BKM@id
\BKM@id=\z@
%    \end{macrocode}
%    \end{macro}
%
%    \begin{macro}{\BKM@0}
%    \begin{macrocode}
\@namedef{BKM@0}{00}
%    \end{macrocode}
%    \end{macro}
%    \begin{macro}{\ifBKM@sw}
%    \begin{macrocode}
\newif\ifBKM@sw
%    \end{macrocode}
%    \end{macro}
%
%    \begin{macro}{\bookmark}
%    \begin{macrocode}
\newcommand*{\bookmark}[2][]{%
  \if@filesw
    \begingroup
      \def\bookmark@text{#2}%
      \BKM@setup{#1}%
      \edef\BKM@prev{\the\BKM@id}%
      \global\advance\BKM@id\@ne
      \BKM@swtrue
      \@whilesw\ifBKM@sw\fi{%
        \ifnum\ifBKM@startatroot\z@\else\BKM@prev\fi=\z@
          \BKM@startatrootfalse
          \def\BKM@parent{0}%
          \expandafter\xdef\csname BKM@\the\BKM@id\endcsname{%
            0{\BKM@level}%
          }%
          \BKM@swfalse
        \else
          \expandafter\expandafter\expandafter\BKM@getx
              \csname BKM@\BKM@prev\endcsname
          \ifnum\BKM@level>\BKM@x@level\relax
            \let\BKM@parent\BKM@prev
            \expandafter\xdef\csname BKM@\the\BKM@id\endcsname{%
              {\BKM@prev}{\BKM@level}%
            }%
            \BKM@swfalse
          \else
            \let\BKM@prev\BKM@x@parent
          \fi
        \fi
      }%
      \pdfstringdef\BKM@title{\bookmark@text}%
      \BKM@vtex@title
      \edef\BKM@FLAGS{\BKM@PrintStyle}%
      \let\BKM@action\@empty
      \ifx\BKM@gotor\@empty
        \ifx\BKM@dest\@empty
          \ifx\BKM@named\@empty
            \ifx\BKM@rawaction\@empty
              \ifx\BKM@uri\@empty
                \ifx\BKM@page\@empty
                  \PackageError{bookmark}{Missing action}\@ehc
                  \def\BKM@action{!1}%
                \else
                  \edef\BKM@action{!\BKM@page}%
                \fi
              \else
                \BKM@EscapeString\BKM@uri
                \edef\BKM@action{%
                  <u=%
                    /S/URI%
                    /URI(\BKM@uri)%
                  >%
                }%
              \fi
            \else
              \edef\BKM@action{<u=\BKM@rawaction>}%
            \fi
          \else
            \BKM@EscapeName\BKM@named
            \edef\BKM@action{%
              <u=%
                /S/Named%
                /N/\BKM@named
              >%
            }%
          \fi
        \else
          \BKM@EscapeString\BKM@dest
          \edef\BKM@action{\BKM@dest}%
        \fi
      \else
        \ifx\BKM@dest\@empty
          \ifx\BKM@page\@empty
            \def\BKM@page{1}%
          \fi
          \ifx\BKM@view\@empty
            \def\BKM@view{Fit}%
          \fi
          \edef\BKM@action{/D[\BKM@page/\BKM@view]}%
        \else
          \BKM@EscapeString\BKM@dest
          \edef\BKM@action{/D(\BKM@dest)}%
        \fi
        \BKM@EscapeString\BKM@gotor
        \edef\BKM@action{%
          <u=%
            /S/GoToR%
            /F(\BKM@gotor)%
            \BKM@action
          >>%
        }%
      \fi
      \ifx\BKM@color\@empty
        \let\BKM@RGBcolor\@empty
      \else
        \expandafter\BKM@toRGB\BKM@color\@nil
      \fi
      \special{%
        !outline \BKM@action;%
        p=\BKM@parent,%
        i=\number\BKM@id,%
        s=%
          \ifBKM@open
            \ifnum\BKM@level<\BKM@openlevel
              o%
            \else
              c%
            \fi
          \else
            c%
          \fi,%
        \ifx\BKM@RGBcolor\@empty
        \else
          c=\BKM@RGBcolor,%
        \fi
        \ifnum\BKM@FLAGS>\z@
          f=\BKM@FLAGS,%
        \fi
        t=\BKM@title
      }%
    \endgroup
  \fi
}
%    \end{macrocode}
%    \end{macro}
%    \begin{macro}{\BKM@getx}
%    \begin{macrocode}
\def\BKM@getx#1#2{%
  \def\BKM@x@parent{#1}%
  \def\BKM@x@level{#2}%
}
%    \end{macrocode}
%    \end{macro}
%    \begin{macro}{\BKM@toRGB}
%    \begin{macrocode}
\def\BKM@toRGB#1 #2 #3\@nil{%
  \let\BKM@RGBcolor\@empty
  \BKM@toRGBComponent{#1}%
  \BKM@toRGBComponent{#2}%
  \BKM@toRGBComponent{#3}%
}
%    \end{macrocode}
%    \end{macro}
%    \begin{macro}{\BKM@toRGBComponent}
%    \begin{macrocode}
\def\BKM@toRGBComponent#1{%
  \dimen@=#1pt\relax
  \ifdim\dimen@>\z@
    \ifdim\dimen@<\p@
      \dimen@=255\dimen@
      \advance\dimen@ by 32768sp\relax
      \divide\dimen@ by 65536\relax
      \dimen@ii=\dimen@
      \divide\dimen@ii by 16\relax
      \edef\BKM@RGBcolor{%
        \BKM@RGBcolor
        \BKM@toHexDigit\dimen@ii
      }%
      \dimen@ii=16\dimen@ii
      \advance\dimen@-\dimen@ii
      \edef\BKM@RGBcolor{%
        \BKM@RGBcolor
        \BKM@toHexDigit\dimen@
      }%
    \else
      \edef\BKM@RGBcolor{\BKM@RGBcolor FF}%
    \fi
  \else
    \edef\BKM@RGBcolor{\BKM@RGBcolor00}%
  \fi
}
%    \end{macrocode}
%    \end{macro}
%    \begin{macro}{\BKM@toHexDigit}
%    \begin{macrocode}
\def\BKM@toHexDigit#1{%
  \ifcase\expandafter\@firstofone\expandafter{\number#1} %
    0\or 1\or 2\or 3\or 4\or 5\or 6\or 7\or
    8\or 9\or A\or B\or C\or D\or E\or F%
  \fi
}
%    \end{macrocode}
%    \end{macro}
%    \begin{macrocode}
\begingroup
  \catcode`\|=0 %
  \catcode`\\=12 %
%    \end{macrocode}
%    \begin{macro}{\BKM@vtex@title}
%    \begin{macrocode}
  |gdef|BKM@vtex@title{%
    |@onelevel@sanitize|BKM@title
    |edef|BKM@title{|expandafter|BKM@vtex@leftparen|BKM@title\(|@nil}%
    |edef|BKM@title{|expandafter|BKM@vtex@rightparen|BKM@title\)|@nil}%
    |edef|BKM@title{|expandafter|BKM@vtex@zero|BKM@title\0|@nil}%
    |edef|BKM@title{|expandafter|BKM@vtex@one|BKM@title\1|@nil}%
    |edef|BKM@title{|expandafter|BKM@vtex@two|BKM@title\2|@nil}%
    |edef|BKM@title{|expandafter|BKM@vtex@three|BKM@title\3|@nil}%
  }%
%    \end{macrocode}
%    \end{macro}
%    \begin{macro}{\BKM@vtex@leftparen}
%    \begin{macrocode}
  |gdef|BKM@vtex@leftparen#1\(#2|@nil{%
    #1%
    |ifx||#2||%
    |else
      (%
      |ltx@ReturnAfterFi{%
        |BKM@vtex@leftparen#2|@nil
      }%
    |fi
  }%
%    \end{macrocode}
%    \end{macro}
%    \begin{macro}{\BKM@vtex@rightparen}
%    \begin{macrocode}
  |gdef|BKM@vtex@rightparen#1\)#2|@nil{%
    #1%
    |ifx||#2||%
    |else
      )%
      |ltx@ReturnAfterFi{%
        |BKM@vtex@rightparen#2|@nil
      }%
    |fi
  }%
%    \end{macrocode}
%    \end{macro}
%    \begin{macro}{\BKM@vtex@zero}
%    \begin{macrocode}
  |gdef|BKM@vtex@zero#1\0#2|@nil{%
    #1%
    |ifx||#2||%
    |else
      |noexpand|hv@pdf@char0%
      |ltx@ReturnAfterFi{%
        |BKM@vtex@zero#2|@nil
      }%
    |fi
  }%
%    \end{macrocode}
%    \end{macro}
%    \begin{macro}{\BKM@vtex@one}
%    \begin{macrocode}
  |gdef|BKM@vtex@one#1\1#2|@nil{%
    #1%
    |ifx||#2||%
    |else
      |noexpand|hv@pdf@char1%
      |ltx@ReturnAfterFi{%
        |BKM@vtex@one#2|@nil
      }%
    |fi
  }%
%    \end{macrocode}
%    \end{macro}
%    \begin{macro}{\BKM@vtex@two}
%    \begin{macrocode}
  |gdef|BKM@vtex@two#1\2#2|@nil{%
    #1%
    |ifx||#2||%
    |else
      |noexpand|hv@pdf@char2%
      |ltx@ReturnAfterFi{%
        |BKM@vtex@two#2|@nil
      }%
    |fi
  }%
%    \end{macrocode}
%    \end{macro}
%    \begin{macro}{\BKM@vtex@three}
%    \begin{macrocode}
  |gdef|BKM@vtex@three#1\3#2|@nil{%
    #1%
    |ifx||#2||%
    |else
      |noexpand|hv@pdf@char3%
      |ltx@ReturnAfterFi{%
        |BKM@vtex@three#2|@nil
      }%
    |fi
  }%
%    \end{macrocode}
%    \end{macro}
%    \begin{macrocode}
|endgroup
%    \end{macrocode}
%
%    \begin{macrocode}
%</vtex>
%    \end{macrocode}
%
% \subsection{\hologo{pdfTeX}\ 的驱动程序}
%
%    \begin{macrocode}
%<*pdftex>
\NeedsTeXFormat{LaTeX2e}
\ProvidesFile{bkm-pdftex.def}%
  [2020-11-06 v1.29 bookmark driver for pdfTeX (HO)]%
%    \end{macrocode}
%
%    \begin{macro}{\BKM@DO@entry}
%    \begin{macrocode}
\def\BKM@DO@entry#1#2{%
  \begingroup
    \kvsetkeys{BKM@DO}{#1}%
    \def\BKM@DO@title{#2}%
    \ifx\BKM@DO@srcfile\@empty
    \else
      \BKM@UnescapeHex\BKM@DO@srcfile
    \fi
    \BKM@UnescapeHex\BKM@DO@title
    \expandafter\expandafter\expandafter\BKM@getx
        \csname BKM@\BKM@DO@id\endcsname\@empty\@empty
    \let\BKM@attr\@empty
    \ifx\BKM@DO@flags\@empty
    \else
      \edef\BKM@attr{\BKM@attr/F \BKM@DO@flags}%
    \fi
    \ifx\BKM@DO@color\@empty
    \else
      \edef\BKM@attr{\BKM@attr/C[\BKM@DO@color]}%
    \fi
    \ifx\BKM@attr\@empty
    \else
      \edef\BKM@attr{attr{\BKM@attr}}%
    \fi
    \let\BKM@action\@empty
    \ifx\BKM@DO@gotor\@empty
      \ifx\BKM@DO@dest\@empty
        \ifx\BKM@DO@named\@empty
          \ifx\BKM@DO@rawaction\@empty
            \ifx\BKM@DO@uri\@empty
              \ifx\BKM@DO@page\@empty
                \PackageError{bookmark}{%
                  Missing action\BKM@SourceLocation
                }\@ehc
                \edef\BKM@action{goto page1{/Fit}}%
              \else
                \ifx\BKM@DO@view\@empty
                  \def\BKM@DO@view{Fit}%
                \fi
                \edef\BKM@action{goto page\BKM@DO@page{/\BKM@DO@view}}%
              \fi
            \else
              \BKM@UnescapeHex\BKM@DO@uri
              \BKM@EscapeString\BKM@DO@uri
              \edef\BKM@action{user{<</S/URI/URI(\BKM@DO@uri)>>}}%
            \fi
          \else
            \BKM@UnescapeHex\BKM@DO@rawaction
            \edef\BKM@action{%
              user{%
                <<%
                  \BKM@DO@rawaction
                >>%
              }%
            }%
          \fi
        \else
          \BKM@EscapeName\BKM@DO@named
          \edef\BKM@action{%
            user{<</S/Named/N/\BKM@DO@named>>}%
          }%
        \fi
      \else
        \BKM@UnescapeHex\BKM@DO@dest
        \BKM@DefGotoNameAction\BKM@action\BKM@DO@dest
      \fi
    \else
      \ifx\BKM@DO@dest\@empty
        \ifx\BKM@DO@page\@empty
          \def\BKM@DO@page{0}%
        \else
          \BKM@CalcExpr\BKM@DO@page\BKM@DO@page-1%
        \fi
        \ifx\BKM@DO@view\@empty
          \def\BKM@DO@view{Fit}%
        \fi
        \edef\BKM@action{/D[\BKM@DO@page/\BKM@DO@view]}%
      \else
        \BKM@UnescapeHex\BKM@DO@dest
        \BKM@EscapeString\BKM@DO@dest
        \edef\BKM@action{/D(\BKM@DO@dest)}%
      \fi
      \BKM@UnescapeHex\BKM@DO@gotor
      \BKM@EscapeString\BKM@DO@gotor
      \edef\BKM@action{%
        user{%
          <<%
            /S/GoToR%
            /F(\BKM@DO@gotor)%
            \BKM@action
          >>%
        }%
      }%
    \fi
    \pdfoutline\BKM@attr\BKM@action
                count\ifBKM@DO@open\else-\fi\BKM@x@childs
                {\BKM@DO@title}%
  \endgroup
}
%    \end{macrocode}
%    \end{macro}
%    \begin{macro}{\BKM@DefGotoNameAction}
%    \cs{BKM@DefGotoNameAction}\ 宏是一个用于 \xpackage{hypdestopt}\ 宏包的钩子(hook)。
%    \begin{macrocode}
\def\BKM@DefGotoNameAction#1#2{%
  \BKM@EscapeString\BKM@DO@dest
  \edef#1{goto name{#2}}%
}
%    \end{macrocode}
%    \end{macro}
%    \begin{macrocode}
%</pdftex>
%    \end{macrocode}
%
%    \begin{macrocode}
%<*pdftex|pdfmark>
%    \end{macrocode}
%    \begin{macro}{\BKM@SourceLocation}
%    \begin{macrocode}
\def\BKM@SourceLocation{%
  \ifx\BKM@DO@srcfile\@empty
    \ifx\BKM@DO@srcline\@empty
    \else
      .\MessageBreak
      Source: line \BKM@DO@srcline
    \fi
  \else
    \ifx\BKM@DO@srcline\@empty
      .\MessageBreak
      Source: file `\BKM@DO@srcfile'%
    \else
      .\MessageBreak
      Source: file `\BKM@DO@srcfile', line \BKM@DO@srcline
    \fi
  \fi
}
%    \end{macrocode}
%    \end{macro}
%    \begin{macrocode}
%</pdftex|pdfmark>
%    \end{macrocode}
%
% \subsection{具有 pdfmark 特色(specials)的驱动程序}
%
% \subsubsection{dvips 驱动程序}
%
%    \begin{macrocode}
%<*dvips>
\NeedsTeXFormat{LaTeX2e}
\ProvidesFile{bkm-dvips.def}%
  [2020-11-06 v1.29 bookmark driver for dvips (HO)]%
%    \end{macrocode}
%    \begin{macro}{\BKM@PSHeaderFile}
%    \begin{macrocode}
\def\BKM@PSHeaderFile#1{%
  \special{PSfile=#1}%
}
%    \end{macrocode}
%    \begin{macro}{\BKM@filename}
%    \begin{macrocode}
\def\BKM@filename{\jobname.out.ps}
%    \end{macrocode}
%    \end{macro}
%    \begin{macrocode}
\AddToHook{shipout/lastpage}{%
  \BKM@pdfmark@out
  \BKM@PSHeaderFile\BKM@filename
  }
%    \end{macrocode}
%    \end{macro}
%    \begin{macrocode}
%</dvips>
%    \end{macrocode}
%
% \subsubsection{公共部分(Common part)}
%
%    \begin{macrocode}
%<*pdfmark>
%    \end{macrocode}
%
%    \begin{macro}{\BKM@pdfmark@out}
%    不要在这里使用 \xpackage{rerunfilecheck}\ 宏包,因为在 \hologo{TeX}\ 运行期间不会
%    读取 \cs{BKM@filename}\ 文件。
%    \begin{macrocode}
\def\BKM@pdfmark@out{%
  \if@filesw
    \newwrite\BKM@file
    \immediate\openout\BKM@file=\BKM@filename\relax
    \BKM@write{\@percentchar!}%
    \BKM@write{/pdfmark where{pop}}%
    \BKM@write{%
      {%
        /globaldict where{pop globaldict}{userdict}ifelse%
        /pdfmark/cleartomark load put%
      }%
    }%
    \BKM@write{ifelse}%
  \else
    \let\BKM@write\@gobble
    \let\BKM@DO@entry\@gobbletwo
  \fi
}
%    \end{macrocode}
%    \end{macro}
%    \begin{macro}{\BKM@write}
%    \begin{macrocode}
\def\BKM@write#{%
  \immediate\write\BKM@file
}
%    \end{macrocode}
%    \end{macro}
%
%    \begin{macro}{\BKM@DO@entry}
%    Pdfmark 的规范(specification)说明 |/Color| 是颜色(color)的键名(key name),
%    但是 ghostscript 只将键(key)传递到 PDF 文件中,因此键名必须是 |/C|。
%    \begin{macrocode}
\def\BKM@DO@entry#1#2{%
  \begingroup
    \kvsetkeys{BKM@DO}{#1}%
    \ifx\BKM@DO@srcfile\@empty
    \else
      \BKM@UnescapeHex\BKM@DO@srcfile
    \fi
    \def\BKM@DO@title{#2}%
    \BKM@UnescapeHex\BKM@DO@title
    \expandafter\expandafter\expandafter\BKM@getx
        \csname BKM@\BKM@DO@id\endcsname\@empty\@empty
    \let\BKM@attr\@empty
    \ifx\BKM@DO@flags\@empty
    \else
      \edef\BKM@attr{\BKM@attr/F \BKM@DO@flags}%
    \fi
    \ifx\BKM@DO@color\@empty
    \else
      \edef\BKM@attr{\BKM@attr/C[\BKM@DO@color]}%
    \fi
    \let\BKM@action\@empty
    \ifx\BKM@DO@gotor\@empty
      \ifx\BKM@DO@dest\@empty
        \ifx\BKM@DO@named\@empty
          \ifx\BKM@DO@rawaction\@empty
            \ifx\BKM@DO@uri\@empty
              \ifx\BKM@DO@page\@empty
                \PackageError{bookmark}{%
                  Missing action\BKM@SourceLocation
                }\@ehc
                \edef\BKM@action{%
                  /Action/GoTo%
                  /Page 1%
                  /View[/Fit]%
                }%
              \else
                \ifx\BKM@DO@view\@empty
                  \def\BKM@DO@view{Fit}%
                \fi
                \edef\BKM@action{%
                  /Action/GoTo%
                  /Page \BKM@DO@page
                  /View[/\BKM@DO@view]%
                }%
              \fi
            \else
              \BKM@UnescapeHex\BKM@DO@uri
              \BKM@EscapeString\BKM@DO@uri
              \edef\BKM@action{%
                /Action<<%
                  /Subtype/URI%
                  /URI(\BKM@DO@uri)%
                >>%
              }%
            \fi
          \else
            \BKM@UnescapeHex\BKM@DO@rawaction
            \edef\BKM@action{%
              /Action<<%
                \BKM@DO@rawaction
              >>%
            }%
          \fi
        \else
          \BKM@EscapeName\BKM@DO@named
          \edef\BKM@action{%
            /Action<<%
              /Subtype/Named%
              /N/\BKM@DO@named
            >>%
          }%
        \fi
      \else
        \BKM@UnescapeHex\BKM@DO@dest
        \BKM@EscapeString\BKM@DO@dest
        \edef\BKM@action{%
          /Action/GoTo%
          /Dest(\BKM@DO@dest)cvn%
        }%
      \fi
    \else
      \ifx\BKM@DO@dest\@empty
        \ifx\BKM@DO@page\@empty
          \def\BKM@DO@page{1}%
        \fi
        \ifx\BKM@DO@view\@empty
          \def\BKM@DO@view{Fit}%
        \fi
        \edef\BKM@action{%
          /Page \BKM@DO@page
          /View[/\BKM@DO@view]%
        }%
      \else
        \BKM@UnescapeHex\BKM@DO@dest
        \BKM@EscapeString\BKM@DO@dest
        \edef\BKM@action{%
          /Dest(\BKM@DO@dest)cvn%
        }%
      \fi
      \BKM@UnescapeHex\BKM@DO@gotor
      \BKM@EscapeString\BKM@DO@gotor
      \edef\BKM@action{%
        /Action/GoToR%
        /File(\BKM@DO@gotor)%
        \BKM@action
      }%
    \fi
    \BKM@write{[}%
    \BKM@write{/Title(\BKM@DO@title)}%
    \ifnum\BKM@x@childs>\z@
      \BKM@write{/Count \ifBKM@DO@open\else-\fi\BKM@x@childs}%
    \fi
    \ifx\BKM@attr\@empty
    \else
      \BKM@write{\BKM@attr}%
    \fi
    \BKM@write{\BKM@action}%
    \BKM@write{/OUT pdfmark}%
  \endgroup
}
%    \end{macrocode}
%    \end{macro}
%    \begin{macrocode}
%</pdfmark>
%    \end{macrocode}
%
% \subsection{\xoption{pdftex}\ 和 \xoption{pdfmark}\ 的公共部分}
%
%    \begin{macrocode}
%<*pdftex|pdfmark>
%    \end{macrocode}
%
% \subsubsection{写入辅助文件(auxiliary file)}
%
%    \begin{macrocode}
\AddToHook{begindocument}{%
 \immediate\write\@mainaux{\string\providecommand\string\BKM@entry[2]{}}}
%    \end{macrocode}
%
%    \begin{macro}{\BKM@id}
%    \begin{macrocode}
\newcount\BKM@id
\BKM@id=\z@
%    \end{macrocode}
%    \end{macro}
%
%    \begin{macro}{\BKM@0}
%    \begin{macrocode}
\@namedef{BKM@0}{000}
%    \end{macrocode}
%    \end{macro}
%    \begin{macro}{\ifBKM@sw}
%    \begin{macrocode}
\newif\ifBKM@sw
%    \end{macrocode}
%    \end{macro}
%
%    \begin{macro}{\bookmark}
%    \begin{macrocode}
\newcommand*{\bookmark}[2][]{%
  \if@filesw
    \begingroup
      \BKM@InitSourceLocation
      \def\bookmark@text{#2}%
      \BKM@setup{#1}%
      \ifx\BKM@srcfile\@empty
      \else
        \BKM@EscapeHex\BKM@srcfile
      \fi
      \edef\BKM@prev{\the\BKM@id}%
      \global\advance\BKM@id\@ne
      \BKM@swtrue
      \@whilesw\ifBKM@sw\fi{%
        \ifnum\ifBKM@startatroot\z@\else\BKM@prev\fi=\z@
          \BKM@startatrootfalse
          \expandafter\xdef\csname BKM@\the\BKM@id\endcsname{%
            0{\BKM@level}0%
          }%
          \BKM@swfalse
        \else
          \expandafter\expandafter\expandafter\BKM@getx
              \csname BKM@\BKM@prev\endcsname
          \ifnum\BKM@level>\BKM@x@level\relax
            \expandafter\xdef\csname BKM@\the\BKM@id\endcsname{%
              {\BKM@prev}{\BKM@level}0%
            }%
            \ifnum\BKM@prev>\z@
              \BKM@CalcExpr\BKM@CalcResult\BKM@x@childs+1%
              \expandafter\xdef\csname BKM@\BKM@prev\endcsname{%
                {\BKM@x@parent}{\BKM@x@level}{\BKM@CalcResult}%
              }%
            \fi
            \BKM@swfalse
          \else
            \let\BKM@prev\BKM@x@parent
          \fi
        \fi
      }%
      \pdfstringdef\BKM@title{\bookmark@text}%
      \edef\BKM@FLAGS{\BKM@PrintStyle}%
      \csname BKM@HypDestOptHook\endcsname
      \BKM@EscapeHex\BKM@dest
      \BKM@EscapeHex\BKM@uri
      \BKM@EscapeHex\BKM@gotor
      \BKM@EscapeHex\BKM@rawaction
      \BKM@EscapeHex\BKM@title
      \immediate\write\@mainaux{%
        \string\BKM@entry{%
          id=\number\BKM@id
          \ifBKM@open
            \ifnum\BKM@level<\BKM@openlevel
              ,open%
            \fi
          \fi
          \BKM@auxentry{dest}%
          \BKM@auxentry{named}%
          \BKM@auxentry{uri}%
          \BKM@auxentry{gotor}%
          \BKM@auxentry{page}%
          \BKM@auxentry{view}%
          \BKM@auxentry{rawaction}%
          \BKM@auxentry{color}%
          \ifnum\BKM@FLAGS>\z@
            ,flags=\BKM@FLAGS
          \fi
          \BKM@auxentry{srcline}%
          \BKM@auxentry{srcfile}%
        }{\BKM@title}%
      }%
    \endgroup
  \fi
}
%    \end{macrocode}
%    \end{macro}
%    \begin{macro}{\BKM@getx}
%    \begin{macrocode}
\def\BKM@getx#1#2#3{%
  \def\BKM@x@parent{#1}%
  \def\BKM@x@level{#2}%
  \def\BKM@x@childs{#3}%
}
%    \end{macrocode}
%    \end{macro}
%    \begin{macro}{\BKM@auxentry}
%    \begin{macrocode}
\def\BKM@auxentry#1{%
  \expandafter\ifx\csname BKM@#1\endcsname\@empty
  \else
    ,#1={\csname BKM@#1\endcsname}%
  \fi
}
%    \end{macrocode}
%    \end{macro}
%
%    \begin{macro}{\BKM@InitSourceLocation}
%    \begin{macrocode}
\def\BKM@InitSourceLocation{%
  \edef\BKM@srcline{\the\inputlineno}%
  \BKM@LuaTeX@InitFile
  \ifx\BKM@srcfile\@empty
    \ltx@IfUndefined{currfilepath}{}{%
      \edef\BKM@srcfile{\currfilepath}%
    }%
  \fi
}
%    \end{macrocode}
%    \end{macro}
%    \begin{macro}{\BKM@LuaTeX@InitFile}
%    \begin{macrocode}
\ifluatex
  \ifnum\luatexversion>36 %
    \def\BKM@LuaTeX@InitFile{%
      \begingroup
        \ltx@LocToksA={}%
      \edef\x{\endgroup
        \def\noexpand\BKM@srcfile{%
          \the\expandafter\ltx@LocToksA
          \directlua{%
             if status and status.filename then %
               tex.settoks('ltx@LocToksA', status.filename)%
             end%
          }%
        }%
      }\x
    }%
  \else
    \let\BKM@LuaTeX@InitFile\relax
  \fi
\else
  \let\BKM@LuaTeX@InitFile\relax
\fi
%    \end{macrocode}
%    \end{macro}
%
% \subsubsection{读取辅助数据(auxiliary data)}
%
%    \begin{macrocode}
\SetupKeyvalOptions{family=BKM@DO,prefix=BKM@DO@}
\DeclareStringOption[0]{id}
\DeclareBoolOption{open}
\DeclareStringOption{flags}
\DeclareStringOption{color}
\DeclareStringOption{dest}
\DeclareStringOption{named}
\DeclareStringOption{uri}
\DeclareStringOption{gotor}
\DeclareStringOption{page}
\DeclareStringOption{view}
\DeclareStringOption{rawaction}
\DeclareStringOption{srcline}
\DeclareStringOption{srcfile}
%    \end{macrocode}
%
%    \begin{macrocode}
\AtBeginDocument{%
  \let\BKM@entry\BKM@DO@entry
}
%    \end{macrocode}
%
%    \begin{macrocode}
%</pdftex|pdfmark>
%    \end{macrocode}
%
% \subsection{\xoption{atend}\ 选项}
%
% \subsubsection{钩子(Hook)}
%
%    \begin{macrocode}
%<*package>
%    \end{macrocode}
%    \begin{macrocode}
\ifBKM@atend
\else
%    \end{macrocode}
%    \begin{macro}{\BookmarkAtEnd}
%    这是一个虚拟定义(dummy definition),如果没有给出 \xoption{atend}\ 选项,它将生成一个警告。
%    \begin{macrocode}
  \newcommand{\BookmarkAtEnd}[1]{%
    \PackageWarning{bookmark}{%
      Ignored, because option `atend' is missing%
    }%
  }%
%    \end{macrocode}
%    \end{macro}
%    \begin{macrocode}
  \expandafter\endinput
\fi
%    \end{macrocode}
%    \begin{macro}{\BookmarkAtEnd}
%    \begin{macrocode}
\newcommand*{\BookmarkAtEnd}{%
  \g@addto@macro\BKM@EndHook
}
%    \end{macrocode}
%    \end{macro}
%    \begin{macrocode}
\let\BKM@EndHook\@empty
%    \end{macrocode}
%    \begin{macrocode}
%</package>
%    \end{macrocode}
%
% \subsubsection{在文档末尾使用钩子的驱动程序}
%
%    驱动程序 \xoption{pdftex}\ 使用 LaTeX 钩子 \xoption{enddocument/afterlastpage}
%    (相当于以前使用的 \xpackage{atveryend}\ 的 \cs{AfterLastShipout}),因为它仍然需要 \xext{aux}\ 文件。
%    它使用 \cs{pdfoutline}\ 作为最后一页之后可以使用的书签(bookmakrs)。
%    \begin{itemize}
%    \item
%      驱动程序 \xoption{pdftex}\ 使用 \cs{pdfoutline}, \cs{pdfoutline}\ 可以在最后一页之后使用。
%    \end{itemize}
%    \begin{macrocode}
%<*pdftex>
\ifBKM@atend
  \AddToHook{enddocument/afterlastpage}{%
    \BKM@EndHook
  }%
\fi
%</pdftex>
%    \end{macrocode}
%
% \subsubsection{使用 \xoption{shipout/lastpage}\ 的驱动程序}
%
%    其他驱动程序使用 \cs{special}\ 命令实现 \cs{bookmark}。因此,最后的书签(last bookmarks)
%    必须放在最后一页(last page),而不是之后。不能使用 \cs{AtEndDocument},因为为时已晚,
%    最后一页已经输出了。因此,我们使用 LaTeX 钩子 \xoption{shipout/lastpage}。至少需要运行
%    两次 \hologo{LaTeX}。PostScript 驱动程序 \xoption{dvips}\ 使用外部 PostScript 文件作为书签。
%    为了避免与 pgf 发生冲突,文件写入(file writing)也被移到了最后一个输出页面(shipout page)。
%    \begin{macrocode}
%<*dvipdfm|vtex|pdfmark>
\ifBKM@atend
  \AddToHook{shipout/lastpage}{\BKM@EndHook}%
\fi
%</dvipdfm|vtex|pdfmark>
%    \end{macrocode}
%
% \section{安装(Installation)}
%
% \subsection{下载(Download)}
%
% \paragraph{宏包(Package)。} 在 CTAN\footnote{\CTANpkg{bookmark}}上提供此宏包:
% \begin{description}
% \item[\CTAN{macros/latex/contrib/bookmark/bookmark.dtx}] 源文件(source file)。
% \item[\CTAN{macros/latex/contrib/bookmark/bookmark.pdf}] 文档(documentation)。
% \end{description}
%
%
% \paragraph{捆绑包(Bundle)。} “bookmark”捆绑包(bundle)的所有宏包(packages)都可以在兼
% 容 TDS 的 ZIP 归档文件中找到。在那里,宏包已经被解包,文档文件(documentation files)已经生成。
% 文件(files)和目录(directories)遵循 TDS 标准。
% \begin{description}
% \item[\CTANinstall{install/macros/latex/contrib/bookmark.tds.zip}]
% \end{description}
% \emph{TDS}\ 是指标准的“用于 \TeX\ 文件的目录结构(Directory Structure)”(\CTANpkg{tds})。
% 名称中带有 \xfile{texmf}\ 的目录(directories)通常以这种方式组织。
%
% \subsection{捆绑包(Bundle)的安装}
%
% \paragraph{解压(Unpacking)。} 在您选择的 TDS 树(也称为 \xfile{texmf}\ 树)中解
% 压 \xfile{bookmark.tds.zip},例如(在 linux 中):
% \begin{quote}
%   |unzip bookmark.tds.zip -d ~/texmf|
% \end{quote}
%
% \subsection{宏包(Package)的安装}
%
% \paragraph{解压(Unpacking)。} \xfile{.dtx}\ 文件是一个自解压 \docstrip\ 归档文件(archive)。
% 这些文件是通过 \plainTeX\ 运行 \xfile{.dtx}\ 来提取的:
% \begin{quote}
%   \verb|tex bookmark.dtx|
% \end{quote}
%
% \paragraph{TDS.} 现在,不同的文件必须移动到安装 TDS 树(installation TDS tree)
% (也称为 \xfile{texmf}\ 树)中的不同目录中:
% \begin{quote}
% \def\t{^^A
% \begin{tabular}{@{}>{\ttfamily}l@{ $\rightarrow$ }>{\ttfamily}l@{}}
%   bookmark.sty & tex/latex/bookmark/bookmark.sty\\
%   bkm-dvipdfm.def & tex/latex/bookmark/bkm-dvipdfm.def\\
%   bkm-dvips.def & tex/latex/bookmark/bkm-dvips.def\\
%   bkm-pdftex.def & tex/latex/bookmark/bkm-pdftex.def\\
%   bkm-vtex.def & tex/latex/bookmark/bkm-vtex.def\\
%   bookmark.pdf & doc/latex/bookmark/bookmark.pdf\\
%   bookmark-example.tex & doc/latex/bookmark/bookmark-example.tex\\
%   bookmark.dtx & source/latex/bookmark/bookmark.dtx\\
% \end{tabular}^^A
% }^^A
% \sbox0{\t}^^A
% \ifdim\wd0>\linewidth
%   \begingroup
%     \advance\linewidth by\leftmargin
%     \advance\linewidth by\rightmargin
%   \edef\x{\endgroup
%     \def\noexpand\lw{\the\linewidth}^^A
%   }\x
%   \def\lwbox{^^A
%     \leavevmode
%     \hbox to \linewidth{^^A
%       \kern-\leftmargin\relax
%       \hss
%       \usebox0
%       \hss
%       \kern-\rightmargin\relax
%     }^^A
%   }^^A
%   \ifdim\wd0>\lw
%     \sbox0{\small\t}^^A
%     \ifdim\wd0>\linewidth
%       \ifdim\wd0>\lw
%         \sbox0{\footnotesize\t}^^A
%         \ifdim\wd0>\linewidth
%           \ifdim\wd0>\lw
%             \sbox0{\scriptsize\t}^^A
%             \ifdim\wd0>\linewidth
%               \ifdim\wd0>\lw
%                 \sbox0{\tiny\t}^^A
%                 \ifdim\wd0>\linewidth
%                   \lwbox
%                 \else
%                   \usebox0
%                 \fi
%               \else
%                 \lwbox
%               \fi
%             \else
%               \usebox0
%             \fi
%           \else
%             \lwbox
%           \fi
%         \else
%           \usebox0
%         \fi
%       \else
%         \lwbox
%       \fi
%     \else
%       \usebox0
%     \fi
%   \else
%     \lwbox
%   \fi
% \else
%   \usebox0
% \fi
% \end{quote}
% 如果你有一个 \xfile{docstrip.cfg}\ 文件,该文件能配置并启用 \docstrip\ 的 TDS 安装功能,
% 则一些文件可能已经在正确的位置了,请参阅 \docstrip\ 的文档(documentation)。
%
% \subsection{刷新文件名数据库}
%
% 如果您的 \TeX~发行版(\TeX\,Live、\mikTeX、\dots)依赖于文件名数据库(file name databases),
% 则必须刷新这些文件名数据库。例如,\TeX\,Live\ 用户运行 \verb|texhash| 或 \verb|mktexlsr|。
%
% \subsection{一些感兴趣的细节}
%
% \paragraph{用 \LaTeX\ 解压。}
% \xfile{.dtx}\ 根据格式(format)选择其操作(action):
% \begin{description}
% \item[\plainTeX:] 运行 \docstrip\ 并解压文件。
% \item[\LaTeX:] 生成文档。
% \end{description}
% 如果您坚持通过 \LaTeX\ 使用\docstrip (实际上 \docstrip\ 并不需要 \LaTeX),那么请您的意图告知自动检测程序:
% \begin{quote}
%   \verb|latex \let\install=y\input{bookmark.dtx}|
% \end{quote}
% 不要忘记根据 shell 的要求引用这个参数(argument)。
%
% \paragraph{知生成文档。}
% 您可以同时使用 \xfile{.dtx}\ 或 \xfile{.drv}\ 来生成文档。可以通过配置文件 \xfile{ltxdoc.cfg}\ 配置该进程。
% 例如,如果您希望 A4 作为纸张格式,请将下面这行写入此文件中:
% \begin{quote}
%   \verb|\PassOptionsToClass{a4paper}{article}|
% \end{quote}
% 下面是一个如何使用 pdf\LaTeX\ 生成文档的示例:
% \begin{quote}
%\begin{verbatim}
%pdflatex bookmark.dtx
%makeindex -s gind.ist bookmark.idx
%pdflatex bookmark.dtx
%makeindex -s gind.ist bookmark.idx
%pdflatex bookmark.dtx
%\end{verbatim}
% \end{quote}
%
% \begin{thebibliography}{9}
%
% \bibitem{hyperref}
%   Sebastian Rahtz, Heiko Oberdiek:
%   \textit{The \xpackage{hyperref} package};
%   2011/04/17 v6.82g;
%   \CTANpkg{hyperref}
%
% \bibitem{currfile}
%   Martin Scharrer:
%   \textit{The \xpackage{currfile} package};
%   2011/01/09 v0.4.
%   \CTANpkg{currfile}
%
% \end{thebibliography}
%
% \begin{History}
%   \begin{Version}{2007/02/19 v0.1}
%   \item
%     First experimental version.
%   \end{Version}
%   \begin{Version}{2007/02/20 v0.2}
%   \item
%     Option \xoption{startatroot} added.
%   \item
%     Dummies for \cs{pdf(un)escape...} commands added to get
%     the package basically work for non-\hologo{pdfTeX} users.
%   \end{Version}
%   \begin{Version}{2007/02/21 v0.3}
%   \item
%     Dependency from \hologo{pdfTeX} 1.30 removed by using package
%     \xpackage{pdfescape}.
%   \end{Version}
%   \begin{Version}{2007/02/22 v0.4}
%   \item
%     \xpackage{hyperref}'s \xoption{bookmarkstype} respected.
%   \end{Version}
%   \begin{Version}{2007/03/02 v0.5}
%   \item
%     Driver options \xoption{vtex} (PDF mode), \xoption{dvipsone},
%     and \xoption{textures} added.
%   \item
%     Implementation of option \xoption{depth} completed. Division names
%     are supported, see \xpackage{hyperref}'s
%     option \xoption{bookmarksdepth}.
%   \item
%     \xpackage{hyperref}'s options \xoption{bookmarksopen},
%     \xoption{bookmarksopenlevel}, and \xoption{bookmarksdepth} respected.
%   \end{Version}
%   \begin{Version}{2007/03/03 v0.6}
%   \item
%     Option \xoption{numbered} as alias for \xpackage{hyperref}'s
%     \xoption{bookmarksnumbered}.
%   \end{Version}
%   \begin{Version}{2007/03/07 v0.7}
%   \item
%     Dependency from \hologo{eTeX} removed.
%   \end{Version}
%   \begin{Version}{2007/04/09 v0.8}
%   \item
%     Option \xoption{atend} added.
%   \item
%     Option \xoption{rgbcolor} removed.
%     \verb|rgbcolor=<r> <g> <b>| can be replaced by
%     \verb|color=[rgb]{<r>,<g>,<b>}|.
%   \item
%     Support of recent cvs version (2007-03-29) of dvipdfmx
%     that extends the \cs{special} for bookmarks to specify
%     open outline entries. Option \xoption{dvipdfmx-outline-open}
%     or \cs{SpecialDvipdfmxOutlineOpen} notify the package.
%   \end{Version}
%   \begin{Version}{2007/04/25 v0.9}
%   \item
%     The syntax of \cs{special} of dvipdfmx, if feature
%     \xoption{dvipdfmx-outline-open} is enabled, has changed.
%     Now cvs version 2007-04-25 is needed.
%   \end{Version}
%   \begin{Version}{2007/05/29 v1.0}
%   \item
%     Bug fix in code for second parameter of XYZ.
%   \end{Version}
%   \begin{Version}{2007/07/13 v1.1}
%   \item
%     Fix for pdfmark with GoToR action.
%   \end{Version}
%   \begin{Version}{2007/09/25 v1.2}
%   \item
%     pdfmark driver respects \cs{nofiles}.
%   \end{Version}
%   \begin{Version}{2008/08/08 v1.3}
%   \item
%     Package \xpackage{flags} replaced by package \xpackage{bitset}.
%     Now flags are also supported without \hologo{eTeX}.
%   \item
%     Hook for package \xpackage{hypdestopt} added.
%   \end{Version}
%   \begin{Version}{2008/09/13 v1.4}
%   \item
%     Fix for bug introduced in v1.3, package \xpackage{flags} is one-based,
%     but package \xpackage{bitset} is zero-based. Thus options \xoption{bold}
%     and \xoption{italic} are wrong in v1.3. (Daniel M\"ullner)
%   \end{Version}
%   \begin{Version}{2009/08/13 v1.5}
%   \item
%     Except for driver options the other options are now local options.
%     This resolves a problem with KOMA-Script v3.00 and its option \xoption{open}.
%   \end{Version}
%   \begin{Version}{2009/12/06 v1.6}
%   \item
%     Use of package \xpackage{atveryend} for drivers \xoption{pdftex}
%     and \xoption{pdfmark}.
%   \end{Version}
%   \begin{Version}{2009/12/07 v1.7}
%   \item
%     Use of package \xpackage{atveryend} fixed.
%   \end{Version}
%   \begin{Version}{2009/12/17 v1.8}
%   \item
%     Support of \xpackage{hyperref} 2009/12/17 v6.79v for \hologo{XeTeX}.
%   \end{Version}
%   \begin{Version}{2010/03/30 v1.9}
%   \item
%     Package name in an error message fixed.
%   \end{Version}
%   \begin{Version}{2010/04/03 v1.10}
%   \item
%     Option \xoption{style} and macro \cs{bookmarkdefinestyle} added.
%   \item
%     Hook support with option \xoption{addtohook} added.
%   \item
%     \cs{bookmarkget} added.
%   \end{Version}
%   \begin{Version}{2010/04/04 v1.11}
%   \item
%     Bug fix (introduced in v1.10).
%   \end{Version}
%   \begin{Version}{2010/04/08 v1.12}
%   \item
%     Requires \xpackage{ltxcmds} 2010/04/08.
%   \end{Version}
%   \begin{Version}{2010/07/23 v1.13}
%   \item
%     Support for \xclass{memoir}'s \cs{booknumberline} added.
%   \end{Version}
%   \begin{Version}{2010/09/02 v1.14}
%   \item
%     (Local) options \xoption{draft} and \xoption{final} added.
%   \end{Version}
%   \begin{Version}{2010/09/25 v1.15}
%   \item
%     Fix for option \xoption{dvipdfmx-outline-open}.
%   \item
%     Option \xoption{dvipdfmx-outline-open} is set automatically,
%     if XeTeX $\geq$ 0.9995 is detected.
%   \end{Version}
%   \begin{Version}{2010/10/19 v1.16}
%   \item
%     Option `startatroot' now acts globally.
%   \item
%     Option `level' also accepts names the same way as option `depth'.
%   \end{Version}
%   \begin{Version}{2010/10/25 v1.17}
%   \item
%     \cs{bookmarksetupnext} added.
%   \item
%     Using \cs{kvsetkeys} of package \xpackage{kvsetkeys}, because
%     \cs{setkeys} of package \xpackage{keyval} is not reentrant.
%     This can cause problems (unknown keys) with older versions of
%     hyperref that also uses \cs{setkeys} (found by GL).
%   \end{Version}
%   \begin{Version}{2010/11/05 v1.18}
%   \item
%     Use of \cs{pdf@ifdraftmode} of package \xpackage{pdftexcmds} for
%     the default of option \xoption{draft}.
%   \end{Version}
%   \begin{Version}{2011/03/20 v1.19}
%   \item
%     Use of \cs{dimexpr} fixed, if \hologo{eTeX} is not used.
%     (Bug found by Martin M\"unch.)
%   \item
%     Fix in documentation. Also layout options work without \hologo{eTeX}.
%   \end{Version}
%   \begin{Version}{2011/04/13 v1.20}
%   \item
%     Bug fix: \cs{BKM@SetDepth} renamed to \cs{BKM@SetDepthOrLevel}.
%   \end{Version}
%   \begin{Version}{2011/04/21 v1.21}
%   \item
%     Some support for file name and line number in error messages
%     at end of document (pdfTeX and pdfmark based drivers).
%   \end{Version}
%   \begin{Version}{2011/05/13 v1.22}
%   \item
%     Change of version 2010/11/05 v1.18 reverted, because otherwise
%     draftmode disables some \xext{aux} file entries.
%   \end{Version}
%   \begin{Version}{2011/09/19 v1.23}
%   \item
%     Some \cs{renewcommand}s changed to \cs{def} to avoid trouble
%     if the commands are not defined, because hyperref stopped early.
%   \end{Version}
%   \begin{Version}{2011/12/02 v1.24}
%   \item
%     Small optimization in \cs{BKM@toHexDigit}.
%   \end{Version}
%   \begin{Version}{2016/05/16 v1.25}
%   \item
%     Documentation updates.
%   \end{Version}
%   \begin{Version}{2016/05/17 v1.26}
%   \item
%     define \cs{pdfoutline} to allow pdftex driver to be used with Lua\TeX.
%   \end{Version}
%   \begin{Version}{2019/06/04 v1.27}
%   \item
%     unknown style options are ignored (issue 67)
%   \end{Version}

%   \begin{Version}{2019/12/03 v1.28}
%   \item
%     Documentation updates.
%   \item adjust package loading (all required packages already loaded
%     by \xpackage{hyperref}).
%   \end{Version}
%   \begin{Version}{2020-11-06 v1.29}
%   \item Adapted the dvips to avoid a clash with pgf.
%         https://github.com/pgf-tikz/pgf/issues/944
%   \item All drivers now use the new LaTeX hooks
%         and so require a format 2020-10-01 or newer. The older
%         drivers are provided as frozen versions and are used if an older
%         format is detected.
%   \item Added support for destlabel option of hyperref, https://github.com/ho-tex/bookmark/issues/1
%   \item Removed the \xoption{dvipsone} and \xoption{textures} driver.
%   \item Removed the code for option \xoption{dvipdfmx-outline-open}
%     and \cs{SpecialDvipdfmxOutlineOpen}. All dvipdfmx version should now support
%     this out-of-the-box.
%   \end{Version}
% \end{History}
%
% \PrintIndex
%
% \Finale
\endinput
|
% \end{quote}
% 不要忘记根据 shell 的要求引用这个参数(argument)。
%
% \paragraph{知生成文档。}
% 您可以同时使用 \xfile{.dtx}\ 或 \xfile{.drv}\ 来生成文档。可以通过配置文件 \xfile{ltxdoc.cfg}\ 配置该进程。
% 例如,如果您希望 A4 作为纸张格式,请将下面这行写入此文件中:
% \begin{quote}
%   \verb|\PassOptionsToClass{a4paper}{article}|
% \end{quote}
% 下面是一个如何使用 pdf\LaTeX\ 生成文档的示例:
% \begin{quote}
%\begin{verbatim}
%pdflatex bookmark.dtx
%makeindex -s gind.ist bookmark.idx
%pdflatex bookmark.dtx
%makeindex -s gind.ist bookmark.idx
%pdflatex bookmark.dtx
%\end{verbatim}
% \end{quote}
%
% \begin{thebibliography}{9}
%
% \bibitem{hyperref}
%   Sebastian Rahtz, Heiko Oberdiek:
%   \textit{The \xpackage{hyperref} package};
%   2011/04/17 v6.82g;
%   \CTANpkg{hyperref}
%
% \bibitem{currfile}
%   Martin Scharrer:
%   \textit{The \xpackage{currfile} package};
%   2011/01/09 v0.4.
%   \CTANpkg{currfile}
%
% \end{thebibliography}
%
% \begin{History}
%   \begin{Version}{2007/02/19 v0.1}
%   \item
%     First experimental version.
%   \end{Version}
%   \begin{Version}{2007/02/20 v0.2}
%   \item
%     Option \xoption{startatroot} added.
%   \item
%     Dummies for \cs{pdf(un)escape...} commands added to get
%     the package basically work for non-\hologo{pdfTeX} users.
%   \end{Version}
%   \begin{Version}{2007/02/21 v0.3}
%   \item
%     Dependency from \hologo{pdfTeX} 1.30 removed by using package
%     \xpackage{pdfescape}.
%   \end{Version}
%   \begin{Version}{2007/02/22 v0.4}
%   \item
%     \xpackage{hyperref}'s \xoption{bookmarkstype} respected.
%   \end{Version}
%   \begin{Version}{2007/03/02 v0.5}
%   \item
%     Driver options \xoption{vtex} (PDF mode), \xoption{dvipsone},
%     and \xoption{textures} added.
%   \item
%     Implementation of option \xoption{depth} completed. Division names
%     are supported, see \xpackage{hyperref}'s
%     option \xoption{bookmarksdepth}.
%   \item
%     \xpackage{hyperref}'s options \xoption{bookmarksopen},
%     \xoption{bookmarksopenlevel}, and \xoption{bookmarksdepth} respected.
%   \end{Version}
%   \begin{Version}{2007/03/03 v0.6}
%   \item
%     Option \xoption{numbered} as alias for \xpackage{hyperref}'s
%     \xoption{bookmarksnumbered}.
%   \end{Version}
%   \begin{Version}{2007/03/07 v0.7}
%   \item
%     Dependency from \hologo{eTeX} removed.
%   \end{Version}
%   \begin{Version}{2007/04/09 v0.8}
%   \item
%     Option \xoption{atend} added.
%   \item
%     Option \xoption{rgbcolor} removed.
%     \verb|rgbcolor=<r> <g> <b>| can be replaced by
%     \verb|color=[rgb]{<r>,<g>,<b>}|.
%   \item
%     Support of recent cvs version (2007-03-29) of dvipdfmx
%     that extends the \cs{special} for bookmarks to specify
%     open outline entries. Option \xoption{dvipdfmx-outline-open}
%     or \cs{SpecialDvipdfmxOutlineOpen} notify the package.
%   \end{Version}
%   \begin{Version}{2007/04/25 v0.9}
%   \item
%     The syntax of \cs{special} of dvipdfmx, if feature
%     \xoption{dvipdfmx-outline-open} is enabled, has changed.
%     Now cvs version 2007-04-25 is needed.
%   \end{Version}
%   \begin{Version}{2007/05/29 v1.0}
%   \item
%     Bug fix in code for second parameter of XYZ.
%   \end{Version}
%   \begin{Version}{2007/07/13 v1.1}
%   \item
%     Fix for pdfmark with GoToR action.
%   \end{Version}
%   \begin{Version}{2007/09/25 v1.2}
%   \item
%     pdfmark driver respects \cs{nofiles}.
%   \end{Version}
%   \begin{Version}{2008/08/08 v1.3}
%   \item
%     Package \xpackage{flags} replaced by package \xpackage{bitset}.
%     Now flags are also supported without \hologo{eTeX}.
%   \item
%     Hook for package \xpackage{hypdestopt} added.
%   \end{Version}
%   \begin{Version}{2008/09/13 v1.4}
%   \item
%     Fix for bug introduced in v1.3, package \xpackage{flags} is one-based,
%     but package \xpackage{bitset} is zero-based. Thus options \xoption{bold}
%     and \xoption{italic} are wrong in v1.3. (Daniel M\"ullner)
%   \end{Version}
%   \begin{Version}{2009/08/13 v1.5}
%   \item
%     Except for driver options the other options are now local options.
%     This resolves a problem with KOMA-Script v3.00 and its option \xoption{open}.
%   \end{Version}
%   \begin{Version}{2009/12/06 v1.6}
%   \item
%     Use of package \xpackage{atveryend} for drivers \xoption{pdftex}
%     and \xoption{pdfmark}.
%   \end{Version}
%   \begin{Version}{2009/12/07 v1.7}
%   \item
%     Use of package \xpackage{atveryend} fixed.
%   \end{Version}
%   \begin{Version}{2009/12/17 v1.8}
%   \item
%     Support of \xpackage{hyperref} 2009/12/17 v6.79v for \hologo{XeTeX}.
%   \end{Version}
%   \begin{Version}{2010/03/30 v1.9}
%   \item
%     Package name in an error message fixed.
%   \end{Version}
%   \begin{Version}{2010/04/03 v1.10}
%   \item
%     Option \xoption{style} and macro \cs{bookmarkdefinestyle} added.
%   \item
%     Hook support with option \xoption{addtohook} added.
%   \item
%     \cs{bookmarkget} added.
%   \end{Version}
%   \begin{Version}{2010/04/04 v1.11}
%   \item
%     Bug fix (introduced in v1.10).
%   \end{Version}
%   \begin{Version}{2010/04/08 v1.12}
%   \item
%     Requires \xpackage{ltxcmds} 2010/04/08.
%   \end{Version}
%   \begin{Version}{2010/07/23 v1.13}
%   \item
%     Support for \xclass{memoir}'s \cs{booknumberline} added.
%   \end{Version}
%   \begin{Version}{2010/09/02 v1.14}
%   \item
%     (Local) options \xoption{draft} and \xoption{final} added.
%   \end{Version}
%   \begin{Version}{2010/09/25 v1.15}
%   \item
%     Fix for option \xoption{dvipdfmx-outline-open}.
%   \item
%     Option \xoption{dvipdfmx-outline-open} is set automatically,
%     if XeTeX $\geq$ 0.9995 is detected.
%   \end{Version}
%   \begin{Version}{2010/10/19 v1.16}
%   \item
%     Option `startatroot' now acts globally.
%   \item
%     Option `level' also accepts names the same way as option `depth'.
%   \end{Version}
%   \begin{Version}{2010/10/25 v1.17}
%   \item
%     \cs{bookmarksetupnext} added.
%   \item
%     Using \cs{kvsetkeys} of package \xpackage{kvsetkeys}, because
%     \cs{setkeys} of package \xpackage{keyval} is not reentrant.
%     This can cause problems (unknown keys) with older versions of
%     hyperref that also uses \cs{setkeys} (found by GL).
%   \end{Version}
%   \begin{Version}{2010/11/05 v1.18}
%   \item
%     Use of \cs{pdf@ifdraftmode} of package \xpackage{pdftexcmds} for
%     the default of option \xoption{draft}.
%   \end{Version}
%   \begin{Version}{2011/03/20 v1.19}
%   \item
%     Use of \cs{dimexpr} fixed, if \hologo{eTeX} is not used.
%     (Bug found by Martin M\"unch.)
%   \item
%     Fix in documentation. Also layout options work without \hologo{eTeX}.
%   \end{Version}
%   \begin{Version}{2011/04/13 v1.20}
%   \item
%     Bug fix: \cs{BKM@SetDepth} renamed to \cs{BKM@SetDepthOrLevel}.
%   \end{Version}
%   \begin{Version}{2011/04/21 v1.21}
%   \item
%     Some support for file name and line number in error messages
%     at end of document (pdfTeX and pdfmark based drivers).
%   \end{Version}
%   \begin{Version}{2011/05/13 v1.22}
%   \item
%     Change of version 2010/11/05 v1.18 reverted, because otherwise
%     draftmode disables some \xext{aux} file entries.
%   \end{Version}
%   \begin{Version}{2011/09/19 v1.23}
%   \item
%     Some \cs{renewcommand}s changed to \cs{def} to avoid trouble
%     if the commands are not defined, because hyperref stopped early.
%   \end{Version}
%   \begin{Version}{2011/12/02 v1.24}
%   \item
%     Small optimization in \cs{BKM@toHexDigit}.
%   \end{Version}
%   \begin{Version}{2016/05/16 v1.25}
%   \item
%     Documentation updates.
%   \end{Version}
%   \begin{Version}{2016/05/17 v1.26}
%   \item
%     define \cs{pdfoutline} to allow pdftex driver to be used with Lua\TeX.
%   \end{Version}
%   \begin{Version}{2019/06/04 v1.27}
%   \item
%     unknown style options are ignored (issue 67)
%   \end{Version}

%   \begin{Version}{2019/12/03 v1.28}
%   \item
%     Documentation updates.
%   \item adjust package loading (all required packages already loaded
%     by \xpackage{hyperref}).
%   \end{Version}
%   \begin{Version}{2020-11-06 v1.29}
%   \item Adapted the dvips to avoid a clash with pgf.
%         https://github.com/pgf-tikz/pgf/issues/944
%   \item All drivers now use the new LaTeX hooks
%         and so require a format 2020-10-01 or newer. The older
%         drivers are provided as frozen versions and are used if an older
%         format is detected.
%   \item Added support for destlabel option of hyperref, https://github.com/ho-tex/bookmark/issues/1
%   \item Removed the \xoption{dvipsone} and \xoption{textures} driver.
%   \item Removed the code for option \xoption{dvipdfmx-outline-open}
%     and \cs{SpecialDvipdfmxOutlineOpen}. All dvipdfmx version should now support
%     this out-of-the-box.
%   \end{Version}
% \end{History}
%
% \PrintIndex
%
% \Finale
\endinput

%        (quote the arguments according to the demands of your shell)
%
% Documentation:
%    (a) If bookmark.drv is present:
%           latex bookmark.drv
%    (b) Without bookmark.drv:
%           latex bookmark.dtx; ...
%    The class ltxdoc loads the configuration file ltxdoc.cfg
%    if available. Here you can specify further options, e.g.
%    use A4 as paper format:
%       \PassOptionsToClass{a4paper}{article}
%
%    Programm calls to get the documentation (example):
%       pdflatex bookmark.dtx
%       makeindex -s gind.ist bookmark.idx
%       pdflatex bookmark.dtx
%       makeindex -s gind.ist bookmark.idx
%       pdflatex bookmark.dtx
%
% Installation:
%    TDS:tex/latex/bookmark/bookmark.sty
%    TDS:tex/latex/bookmark/bkm-dvipdfm.def
%    TDS:tex/latex/bookmark/bkm-dvips.def
%    TDS:tex/latex/bookmark/bkm-pdftex.def
%    TDS:tex/latex/bookmark/bkm-vtex.def
%    TDS:tex/latex/bookmark/bkm-dvipdfm-2019-12-03.def
%    TDS:tex/latex/bookmark/bkm-dvips-2019-12-03.def
%    TDS:tex/latex/bookmark/bkm-pdftex-2019-12-03.def
%    TDS:tex/latex/bookmark/bkm-vtex-2019-12-03.def%
%    TDS:doc/latex/bookmark/bookmark.pdf
%    TDS:doc/latex/bookmark/bookmark-example.tex
%    TDS:source/latex/bookmark/bookmark.dtx
%    TDS:source/latex/bookmark/bookmark-frozen.dtx
%
%<*ignore>
\begingroup
  \catcode123=1 %
  \catcode125=2 %
  \def\x{LaTeX2e}%
\expandafter\endgroup
\ifcase 0\ifx\install y1\fi\expandafter
         \ifx\csname processbatchFile\endcsname\relax\else1\fi
         \ifx\fmtname\x\else 1\fi\relax
\else\csname fi\endcsname
%</ignore>
%<*install>
\input docstrip.tex
\Msg{************************************************************************}
\Msg{* Installation}
\Msg{* Package: bookmark 2020-11-06 v1.29 PDF bookmarks (HO)}
\Msg{************************************************************************}

\keepsilent
\askforoverwritefalse

\let\MetaPrefix\relax
\preamble

This is a generated file.

Project: bookmark
Version: 2020-11-06 v1.29

Copyright (C)
   2007-2011 Heiko Oberdiek
   2016-2020 Oberdiek Package Support Group

This work may be distributed and/or modified under the
conditions of the LaTeX Project Public License, either
version 1.3c of this license or (at your option) any later
version. This version of this license is in
   https://www.latex-project.org/lppl/lppl-1-3c.txt
and the latest version of this license is in
   https://www.latex-project.org/lppl.txt
and version 1.3 or later is part of all distributions of
LaTeX version 2005/12/01 or later.

This work has the LPPL maintenance status "maintained".

The Current Maintainers of this work are
Heiko Oberdiek and the Oberdiek Package Support Group
https://github.com/ho-tex/bookmark/issues


This work consists of the main source file bookmark.dtx and bookmark-frozen.dtx
and the derived files
   bookmark.sty, bookmark.pdf, bookmark.ins, bookmark.drv,
   bkm-dvipdfm.def, bkm-dvips.def, bkm-pdftex.def, bkm-vtex.def,
   bkm-dvipdfm-2019-12-03.def, bkm-dvips-2019-12-03.def,
   bkm-pdftex-2019-12-03.def, bkm-vtex-2019-12-03.def,
   bookmark-example.tex.

\endpreamble
\let\MetaPrefix\DoubleperCent

\generate{%
  \file{bookmark.ins}{\from{bookmark.dtx}{install}}%
  \file{bookmark.drv}{\from{bookmark.dtx}{driver}}%
  \usedir{tex/latex/bookmark}%
  \file{bookmark.sty}{\from{bookmark.dtx}{package}}%
  \file{bkm-dvipdfm.def}{\from{bookmark.dtx}{dvipdfm}}%
  \file{bkm-dvips.def}{\from{bookmark.dtx}{dvips,pdfmark}}%
  \file{bkm-pdftex.def}{\from{bookmark.dtx}{pdftex}}%
  \file{bkm-vtex.def}{\from{bookmark.dtx}{vtex}}%
  \usedir{doc/latex/bookmark}%
  \file{bookmark-example.tex}{\from{bookmark.dtx}{example}}%
  \file{bkm-pdftex-2019-12-03.def}{\from{bookmark-frozen.dtx}{pdftexfrozen}}%
  \file{bkm-dvips-2019-12-03.def}{\from{bookmark-frozen.dtx}{dvipsfrozen}}%
  \file{bkm-vtex-2019-12-03.def}{\from{bookmark-frozen.dtx}{vtexfrozen}}%
  \file{bkm-dvipdfm-2019-12-03.def}{\from{bookmark-frozen.dtx}{dvipdfmfrozen}}%
}

\catcode32=13\relax% active space
\let =\space%
\Msg{************************************************************************}
\Msg{*}
\Msg{* To finish the installation you have to move the following}
\Msg{* files into a directory searched by TeX:}
\Msg{*}
\Msg{*     bookmark.sty, bkm-dvipdfm.def, bkm-dvips.def,}
\Msg{*     bkm-pdftex.def, bkm-vtex.def, bkm-dvipdfm-2019-12-03.def,}
\Msg{*     bkm-dvips-2019-12-03.def, bkm-pdftex-2019-12-03.def,}
\Msg{*     and bkm-vtex-2019-12-03.def}
\Msg{*}
\Msg{* To produce the documentation run the file `bookmark.drv'}
\Msg{* through LaTeX.}
\Msg{*}
\Msg{* Happy TeXing!}
\Msg{*}
\Msg{************************************************************************}

\endbatchfile
%</install>
%<*ignore>
\fi
%</ignore>
%<*driver>
\NeedsTeXFormat{LaTeX2e}
\ProvidesFile{bookmark.drv}%
  [2020-11-06 v1.29 PDF bookmarks (HO)]%
\documentclass{ltxdoc}
\usepackage{ctex}
\usepackage{indentfirst}
\setlength{\parindent}{2em}
\usepackage{holtxdoc}[2011/11/22]
\usepackage{xcolor}
\usepackage{hyperref}
\usepackage[open,openlevel=3,atend]{bookmark}[2020/11/06] %%%打开书签,显示的深度为3级,即显示part、section、subsection。
\bookmarksetup{color=red}
\begin{document}

  \renewcommand{\contentsname}{目\quad 录}
  \renewcommand{\abstractname}{摘\quad 要}
  \renewcommand{\historyname}{历史}
  \DocInput{bookmark.dtx}%
\end{document}
%</driver>
% \fi
%
%
%
% \GetFileInfo{bookmark.drv}
%
%% \title{\xpackage{bookmark} 宏包}
% \title{\heiti {\Huge \textbf{\xpackage{bookmark}\ 宏包}}}
% \date{2020-11-06\ \ \ v1.29}
% \author{Heiko Oberdiek \thanks
% {如有问题请点击:\url{https://github.com/ho-tex/bookmark/issues}}\\[5pt]赣医一附院神经科\ \ 黄旭华\ \ \ \ 译}
%
% \maketitle
%
% \begin{abstract}
% 这个宏包为 \xpackage{hyperref}\ 宏包实现了一个新的书签(bookmark)(大纲[outline])组织。现在
% 可以设置样式(style)和颜色(color)等书签属性(bookmark properties)。其他动作类型(action types)可用
% (URI、GoToR、Named)。书签是在第一次编译运行(compile run)中生成的。\xpackage{hyperref}\
% 宏包必需运行两次。
% \end{abstract}
%
% \tableofcontents
%
% \section{文档(Documentation)}
%
% \subsection{介绍}
%
% 这个 \xpackage{bookmark}\ 宏包试图为书签(bookmarks)提供一个更现代的管理:
% \begin{itemize}
% \item 书签已经在第一次 \hologo{TeX}\ 编译运行(compile run)中生成。
% \item 可以更改书签的字体样式(font style)和颜色(color)。
% \item 可以执行比简单的 GoTo 操作(actions)更多的操作。
% \end{itemize}
%
% 与 \xpackage{hyperref} \cite{hyperref} 一样,书签(bookmarks)也是按照书签生成宏
% (bookmark generating macros)(\cs{bookmark})的顺序生成的。级别号(level number)用于
% 定义书签的树结构(tree structure)。限制没有那么严格:
% \begin{itemize}
% \item 级别值(level values)可以跳变(jump)和省略(omit)。\cs{subsubsection}\ 可以跟在
%       \cs{chapter}\ 之后。这种情况如在 \xpackage{hyperref}\ 中则产生错误,它将显示一个警告(warning)
%       并尝试修复此错误。
% \item 多个书签可能指向同一目标(destination)。在 \xpackage{hyperref}\ 中,这会完全弄乱
%       书签树(bookmark tree),因为算法假设(algorithm assumes)目标名称(destination names)
%       是键(keys)(唯一的)。
% \end{itemize}
%
% 注意,这个宏包是作为书签管理(bookmark management)的实验平台(experimentation platform)。
% 欢迎反馈。此外,在未来的版本中,接口(interfaces)也可能发生变化。
%
% \subsection{选项(Options)}
%
% 可在以下四个地方放置选项(options):
% \begin{enumerate}
% \item \cs{usepackage}|[|\meta{options}|]{bookmark}|\\
%       这是放置驱动程序选项(driver options)和 \xoption{atend}\ 选项的唯一位置。
% \item \cs{bookmarksetup}|{|\meta{options}|}|\\
%       此命令仅用于设置选项(setting options)。
% \item \cs{bookmarksetupnext}|{|\meta{options}|}|\\
%       这些选项在下一个 \cs{bookmark}\ 命令的选项之后存储(stored)和调用(called)。
% \item \cs{bookmark}|[|\meta{options}|]{|\meta{title}|}|\\
%       此命令设置书签。选项设置(option settings)仅限于此书签。
% \end{enumerate}
% 异常(Exception):加载该宏包后,无法更改驱动程序选项(Driver options)、\xoption{atend}\ 选项
% 、\xoption{draft}\slash\xoption{final}选项。
%
% \subsubsection{\xoption{draft} 和 \xoption{final}\ 选项}
%
% 如果一个\LaTeX\ 文件要被编译了多次,那么可以使用 \xoption{draft}\ 选项来禁用该宏包的书签内
% 容(bookmark stuff),这样可以节省一点时间。默认 \xoption{final}\ 选项。两个选项都是
% 布尔选项(boolean options),如果没有值,则使用值 |true|。|draft=true| 与 |final=false| 相同。
%
% 除了驱动程序选项(driver options)之外,\xpackage{bookmark}\ 宏包选项都是局部选项(local options)。
% \xoption{draft}\ 选项和 \xoption{final}\ 选项均属于文档类选项(class option)(译者注:文档类选项为全局选项),
% 因此,在 \xpackage{bookmark}\ 宏包中未能看到这两个选项。如果您想使用全局的(global) \xoption{draft}选项
% 来优化第一次 \LaTeX\ 运行(runs),可以在导言(preamble)中引入 \xpackage{ifdraft}\ 宏包并设置 \LaTeX\ 的
% \cs{PassOptionsToPackage},例如:
%\begin{quote}
%\begin{verbatim}
%\documentclass[draft]{article}
%\usepackage{ifdraft}
%\ifdraft{%
%   \PassOptionsToPackage{draft}{bookmark}%
%}{}
%\end{verbatim}
%\end{quote}
%
% \subsubsection{驱动程序选项(Driver options)}
%
% 支持的驱动程序( drivers)包括 \xoption{pdftex}、\xoption{dvips}、\xoption{dvipdfm} (\xoption{xetex})、
% \xoption{vtex}。\hologo{TeX}\ 引擎 \hologo{pdfTeX}、\hologo{XeTeX}、\hologo{VTeX}\ 能被自动检测到。
% 默认的 DVI 驱动程序是 \xoption{dvips}。这可以通过 \cs{BookmarkDriverDefault}\ 在配置
% 文件 \xfile{bookmark.cfg}\ 中进行更改,例如:
% \begin{quote}
% |\def\BookmarkDriverDefault{dvipdfm}|
% \end{quote}
% 当前版本的(current versions)驱动程序使用新的 \LaTeX\ 钩子(\LaTeX-hooks)。如果检测到比
% 2020-10-01 更旧的格式,则将以前驱动程序的冻结版本(frozen versions)作为备份(fallback)。
%
% \paragraph{用 dvipdfmx 打开书签(bookmarks)。}旧版本的宏包有一个 \xoption{dvipdfmx-outline-open}\ 选项
% 可以激活代码,而该代码可以指定一个大纲条目(outline entry)是否打开。该宏包现在假设所有使用的 dvipdfmx 版本都是
% 最新版本,足以理解该代码,因此始终激活该代码。选项本身将被忽略。
%
%
% \subsubsection{布局选项(Layout options)}
%
% \paragraph{字体(Font)选项:}
%
% \begin{description}
% \item[\xoption{bold}:] 如果受 PDF 浏览器(PDF viewer)支持,书签将以粗体字体(bold font)显示(自 PDF 1.4起)。
% \item[\xoption{italic}:] 使用斜体字体(italic font)(自 PDF 1.4起)。
% \end{description}
% \xoption{bold}(粗体) 和 \xoption{italic}(斜体)可以同时使用。而 |false| 值(value)禁用字体选项。
%
% \paragraph{颜色(Color)选项:}
%
% 彩色书签(Colored bookmarks)是 PDF 1.4 的一个特性(feature),并非所有的 PDF 浏览器(PDF viewers)都支持彩色书签。
% \begin{description}
% \item[\xoption{color}:] 这里 color(颜色)可以作为 \xpackage{color}\ 宏包或 \xpackage{xcolor}\ 宏包的
% 颜色规范(color specification)给出。空值(empty value)表示未设置颜色属性。如果未加载 \xpackage{xcolor}\ 宏包,
% 能识别的值(recognized values)只有:
%   \begin{itemize}
%   \item 空值(empty value)表示未设置颜色属性,\\
%         例如:|color={}|
%   \item 颜色模型(color model) rgb 的显式颜色规范(explicit color specification),\\
%         例如,红色(red):|color=[rgb]{1,0,0}|
%   \item 颜色模型(color model)灰(gray)的显式颜色规范(explicit color specification),\\
%         例如,深灰色(dark gray):|color=[gray]{0.25}|
%   \end{itemize}
%   请注意,如果加载了 \xpackage{color}\ 宏包,此限制(restriction)也适用。然而,如果加载了 \xpackage{xcolor}\ 宏包,
%   则可以使用所有颜色规范(color specifications)。
% \end{description}
%
% \subsubsection{动作选项(Action options)}
%
% \begin{description}
% \item[\xoption{dest}:] 目的地名称(destination name)。
% \item[\xoption{page}:] 页码(page number),第一页(first page)为 1。
% \item[\xoption{view}:] 浏览规范(view specification),示例如下:\\
%   |view={FitB}|, |view={FitH 842}|, |view={XYZ 0 100 null}|\ \  一些浏览规范参数(view specification parameters)
%   将数字(numbers)视为具有单位 bp 的参数。它们可以作为普通数字(plain numbers)或在 \cs{calc}\ 内部以
%   长度表达式(length expressions)给出。如果加载了 \xpackage{calc}\ 宏包,则支持该宏包的表达式(expressions)。否则,
%   使用 \hologo{eTeX}\ 的 \cs{dimexpr}。例如:\\
%   |view={FitH \calc{\paperheight-\topmargin-1in}}|\\
%   |view={XYZ 0 \calc{\paperheight} null}|\\
%   注意 \cs{calc}\ 不能用于 |XYZ| 的第三个参数,因为该参数是缩放值(zoom value),而不是长度(length)。

% \item[\xoption{named}:] 已命名的动作(Named action)的名称:\\
%   |FirstPage|(第一页),|LastPage|(最后一页),|NextPage|(下一页),|PrevPage|(前一页)
% \item[\xoption{gotor}:] 外部(external) PDF 文件的名称。
% \item[\xoption{uri}:] URI 规范(URI specification)。
% \item[\xoption{rawaction}:] 原始动作规范(raw action specification)。由于这些规范取决于驱动程序(driver),因此不应使用此选项。
% \end{description}
% 通过分析指定的选项来选择书签的适当动作。动作由不同的选项集(sets of options)区分:
% \begin{quote}
 \begin{tabular}{|@{}r|l@{}|}
%   \hline
%   \ \textbf{动作(Action)}\  & \ \textbf{选项(Options)}\ \\ \hline
%   \ \textsf{GoTo}\  &\  \xoption{dest}\ \\ \hline
%   \ \textsf{GoTo}\  & \ \xoption{page} + \xoption{view}\ \\ \hline
%   \ \textsf{GoToR}\  & \ \xoption{gotor} + \xoption{dest}\ \\ \hline
%   \ \textsf{GoToR}\  & \ \xoption{gotor} + \xoption{page} + \xoption{view}\ \ \ \\ \hline
%   \ \textsf{Named}\  &\  \xoption{named}\ \\ \hline
%   \ \textsf{URI}\  & \ \xoption{uri}\ \\ \hline
% \end{tabular}
% \end{quote}
%
% \paragraph{缺少动作(Missing actions)。}
% 如果动作缺少 \xpackage{bookmark}\ 宏包,则抛出错误消息(error message)。根据驱动程序(driver)
% (\xoption{pdftex}、\xoption{dvips}\ 和好友[friends]),宏包在文档末尾很晚才检测到它。
% 自 2011/04/21 v1.21 版本以后,该宏包尝试打印 \cs{bookmark}\ 的相应出现的行号(line number)和文件名(file name)。
% 然而,\hologo{TeX}\ 确实提供了行号,但不幸的是,文件名是一个秘密(secret)。但该宏包有如下获取文件名的方法:
% \begin{itemize}
% \item 如果 \hologo{LuaTeX} (独立于 DVI 或 PDF 模式)正在运行,则自动使用其 |status.filename|。
% \item 宏包的 \cs{currfile} \cite{currfile}\ 重新定义了 \hologo{LaTeX}\ 的内部结构,以跟踪文件名(file name)。
% 如果加载了该宏包,那么它的 \cs{currfilepath}\ 将被检测到并由 \xpackage{bookmark}\ 自动使用。
% \item 可以通过 \cs{bookmarksetup}\ 或 \cs{bookmark}\ 中的 \xoption{scrfile}\ 选项手动设置(set manually)文件名。
% 但是要小心,手动设置会禁用以前的文件名检测方法。错误的(wrong)或丢失的(missed)文件名设置(file name setting)可能会在错误消息中
% 为您提供错误的源位置(source location)。
% \end{itemize}
%
% \subsubsection{级别选项(Level options)}
%
% 书签条目(bookmark entries)的顺序由 \cs{bookmark}\ 命令的的出现顺序(appearance order)定义。
% 树结构(tree structure)由书签节点(bookmark nodes)的属性 \xoption{level}(级别)构建。
% \xoption{level}\ 的值是整数(integers)。如果书签条目级别的值高于前一个节点,则该条目将成为
% 前一个节点的子(child)节点。差值的绝对值并不重要。
%
% \xpackage{bookmark}\ 宏包能记住全局属性(global property)“current level(当前级别)”中上
% 一个书签条目(previous bookmark entry)的级别。
%
% 级别系统的(level system)行为(behaviour)可以通过以下选项进行配置:
% \begin{description}
% \item[\xoption{level}:]
%    设置级别(level),请参阅上面的说明。如果给出的选项 \xoption{level}\ 没有值,那么将恢复默
%    认行为,即将“当前级别(current level)”用作级别值(level value)。自 2010/10/19 v1.16 版本以来,
%    如果宏 \cs{toclevel@part}、\cs{toclevel@section}\ 被定义过(通过 \xpackage{hyperref}\ 宏包完成,
%    请参阅它的 \xoption{bookmarkdepth}\ 选项),则 \xpackage{bookmark}\ 宏包还支持 |part|、|section| 等名称。
%
% \item[\xoption{rellevel}:]
%    设置相对于前一级别的(previous level)级别。正值表示书签条目成为前一个书签条目的子条目。
% \item[\xoption{keeplevel}:]
%    使用由\xoption{level}\ 或 \xoption{rellevel}\ 设置的级别,但不要更改全局属性“current level(当前级别)”。
%    可以通过设置为 |false| 来禁用该选项。
% \item[\xoption{startatroot}:]
%    此时,书签树(bookmark tree)再次从顶层(top level)开始。下一个书签条目不会作为上一个条目的子条目进行排序。
%    示例场景:文档使用 part。但是,最后几章(last chapters)不应放在最后一部分(last part)下面:
%    \begin{quote}
%\begin{verbatim}
%\documentclass{book}
%[...]
%\begin{document}
%  \part{第一部分}
%    \chapter{第一部分的第1章}
%    [...]
%  \part{第二部分(Second part)}
%    \chapter{第二部分的第1章}
%    [...]
%  \bookmarksetup{startatroot}
%  \chapter{Index}% 不属于第二部分
%\end{document}
%\end{verbatim}
%    \end{quote}
% \end{description}
%
% \subsubsection{样式定义(Style definitions)}
%
% 样式(style)是一组选项设置(option settings)。它可以由宏 \cs{bookmarkdefinestyle}\ 定义,
% 并由它的 \xoption{style}\ 选项使用。
% \begin{declcs}{bookmarkdefinestyle} \M{name} \M{key value list}
% \end{declcs}
% 选项设置(option settings)的 \meta{key value list}(键值列表)被指定为样式名(style \meta{name})。
%
% \begin{description}
% \item[\xoption{style}:]
%   \xoption{style}\ 选项的值是以前定义的样式的名称(name)。现在执行其选项设置(option settings)。
%   选项可以包括 \xoption{style}\ 选项。通过递归调用相同样式的无限递归(endless recursion)被阻止并抛出一个错误。
% \end{description}
%
% \subsubsection{钩子支持(Hook support)}
%
% 处理宏\cs{bookmark}\ 的可选选项(optional options)后,就会调用钩子(hook)。
% \begin{description}
% \item[\xoption{addtohook}:]
%   代码(code)作为该选项的值添加到钩子中。
% \end{description}
%
% \begin{declcs}{bookmarkget} \M{option}
% \end{declcs}
% \cs{bookmarkget}\ 宏提取 \meta{option}\ 选项的最新选项设置(latest option setting)的值。
% 对于布尔选项(boolean option),如果启用布尔选项,则返回 1,否则结果为零。结果数字(resulting numbers)
% 可以直接用于 \cs{ifnum}\ 或 \cs{ifcase}。如果您想要数字 \texttt{0}\ 和 \texttt{1},
% 请在 \cs{bookmarkget}\ 前面加上 \cs{number}\ 作为前缀。\cs{bookmarkget}\ 宏是可展开的(expandable)。
% 如果选项不受支持,则返回空字符串(empty string)。受支持的布尔选项有:
% \begin{quote}
%   \xoption{bold}、
%   \xoption{italic}、
%   \xoption{open}
% \end{quote}
% 其他受支持的选项有:
% \begin{quote}
%   \xoption{depth}、
%   \xoption{dest}、
%   \xoption{color}、
%   \xoption{gotor}、
%   \xoption{level}、
%   \xoption{named}、
%   \xoption{openlevel}、
%   \xoption{page}、
%   \xoption{rawaction}、
%   \xoption{uri}、
%   \xoption{view}、
% \end{quote}
% 另外,以下键(key)是可用的:
% \begin{quote}
%   \xoption{text}
% \end{quote}
% 它返回大纲条目(outline entry)的文本(text)。
%
% \paragraph{选项设置(Option setting)。}
% 在钩子(hook)内部可以使用 \cs{bookmarksetup}\ 设置选项。
%
% \subsection{与 \xpackage{hyperref}\ 的兼容性}
%
% \xpackage{bookmark}\ 宏包自动禁用 \xpackage{hyperref}\ 宏包的书签(bookmarks)。但是,
% \xpackage{bookmark}\ 宏包使用了 \xpackage{hyperref}\ 宏包的一些代码。例如,
% \xpackage{bookmark}\ 宏包重新定义了 \xpackage{hyperref}\ 宏包在 \cs{addcontentsline}\ 命令
% 和其他命令中插入的\cs{Hy@writebookmark}\ 钩子。因此,不应禁用 \xpackage{hyperref}\ 宏包的书签。
%
% \xpackage{bookmark}\ 宏包使用 \xpackage{hyperref}\ 宏包的 \cs{pdfstringdef},且不提供替换(replacement)。
%
% \xpackage{hyperref}\ 宏包的一些选项也能在 \xpackage{bookmark}\ 宏包中实现(implemented):
% \begin{quote}
% \begin{tabular}{|l@{}|l@{}|}
%   \hline
%   \xpackage{hyperref}\ 宏包的选项\  &\ \xpackage{bookmark}\ 宏包的选项\ \ \\ \hline
%   \xoption{bookmarksdepth} &\ \xoption{depth}\\ \hline
%   \xoption{bookmarksopen} & \ \xoption{open}\\ \hline
%   \xoption{bookmarksopenlevel}\ \ \  &\ \xoption{openlevel}\\ \hline
%   \xoption{bookmarksnumbered} \ \ \ &\ \xoption{numbered}\\ \hline
% \end{tabular}
% \end{quote}
%
% 还可以使用以下命令:
% \begin{quote}
%   \cs{pdfbookmark}\\
%   \cs{currentpdfbookmark}\\
%   \cs{subpdfbookmark}\\
%   \cs{belowpdfbookmark}
% \end{quote}
%
% \subsection{在末尾添加书签}
%
% 宏包选项 \xoption{atend}\ 启用以下宏(macro):
% \begin{declcs}{BookmarkAtEnd}
%   \M{stuff}
% \end{declcs}
% \cs{BookmarkAtEnd}\ 宏将 \meta{stuff}\ 放在文档末尾。\meta{stuff}\ 表示书签命令(bookmark commands)。举例:
% \begin{quote}
%\begin{verbatim}
%\usepackage[atend]{bookmark}
%\BookmarkAtEnd{%
%  \bookmarksetup{startatroot}%
%  \bookmark[named=LastPage, level=0]{Last page}%
%}
%\end{verbatim}
% \end{quote}
%
% 或者,可以在 \cs{bookmark}\ 中给出 \xoption{startatroot}\ 选项:
% \begin{quote}
%\begin{verbatim}
%\BookmarkAtEnd{%
%  \bookmark[
%    startatroot,
%    named=LastPage,
%    level=0,
%  ]{Last page}%
%}
%\end{verbatim}
% \end{quote}
%
% \paragraph{备注(Remarks):}
% \begin{itemize}
% \item
%   \cs{BookmarkAtEnd} 隐藏了这样一个事实,即在文档末尾添加书签的方法取决于驱动程序(driver)。
%
%   为此,驱动程序 \xoption{pdftex}\ 使用 \xpackage{atveryend}\ 宏包。如果 \cs{AtEndDocument}\ 太早,
%   最后一个页面(last page)可能不会被发送出去(shipped out)。由于需要 \xext{aux}\ 文件,此驱动程序使
%   用 \cs{AfterLastShipout}。
%
%   其他驱动程序(\xoption{dvipdfm}、\xoption{xetex}、\xoption{vtex})的实现(implementation)
%   取决于 \cs{special},\cs{special}\ 在最后一页之后没有效果。在这种情况下,\xpackage{atenddvi}\ 宏包的
%   \cs{AtEndDvi}\ 有帮助。它将其参数(argument)放在文档的最后一页(last page)。至少需要运行 \hologo{LaTeX}\ 两次,
%   因为最后一页是由引用(reference)检测到的。
%
%   \xoption{dvips}\ 现在使用新的 LaTeX 钩子 \texttt{shipout/lastpage}。
% \item
%   未指定 \cs{BookmarkAtEnd}\ 参数的扩展时间(time of expansion)。这可以立即发生,也可以在文档末尾发生。
% \end{itemize}
%
% \subsection{限制/行动计划}
%
% \begin{itemize}
% \item 支持缺失动作(missing actions)(启动,\dots)。
% \item 对 \xpackage{hyperref}\ 的 \xoption{bookmarkstype}\ 选项进行了更好的设计(design)。
% \end{itemize}
%
% \section{示例(Example)}
%
%    \begin{macrocode}
%<*example>
%    \end{macrocode}
%    \begin{macrocode}
\documentclass{article}
\usepackage{xcolor}[2007/01/21]
\usepackage{hyperref}
\usepackage[
  open,
  openlevel=2,
  atend
]{bookmark}[2019/12/03]

\bookmarksetup{color=blue}

\BookmarkAtEnd{%
  \bookmarksetup{startatroot}%
  \bookmark[named=LastPage, level=0]{End/Last page}%
  \bookmark[named=FirstPage, level=1]{First page}%
}

\begin{document}
\section{First section}
\subsection{Subsection A}
\begin{figure}
  \hypertarget{fig}{}%
  A figure.
\end{figure}
\bookmark[
  rellevel=1,
  keeplevel,
  dest=fig
]{A figure}
\subsection{Subsection B}
\subsubsection{Subsubsection C}
\subsection{Umlauts: \"A\"O\"U\"a\"o\"u\ss}
\newpage
\bookmarksetup{
  bold,
  color=[rgb]{1,0,0}
}
\section{Very important section}
\bookmarksetup{
  italic,
  bold=false,
  color=blue
}
\subsection{Italic section}
\bookmarksetup{
  italic=false
}
\part{Misc}
\section{Diverse}
\subsubsection{Subsubsection, omitting subsection}
\bookmarksetup{
  startatroot
}
\section{Last section outside part}
\subsection{Subsection}
\bookmarksetup{
  color={}
}
\begingroup
  \bookmarksetup{level=0, color=green!80!black}
  \bookmark[named=FirstPage]{First page}
  \bookmark[named=LastPage]{Last page}
  \bookmark[named=PrevPage]{Previous page}
  \bookmark[named=NextPage]{Next page}
\endgroup
\bookmark[
  page=2,
  view=FitH 800
]{Page 2, FitH 800}
\bookmark[
  page=2,
  view=FitBH \calc{\paperheight-\topmargin-1in-\headheight-\headsep}
]{Page 2, FitBH top of text body}
\bookmark[
  uri={http://www.dante.de/},
  color=magenta
]{Dante homepage}
\bookmark[
  gotor={t.pdf},
  page=1,
  view={XYZ 0 1000 null},
  color=cyan!75!black
]{File t.pdf}
\bookmark[named=FirstPage]{First page}
\bookmark[rellevel=1, named=LastPage]{Last page (rellevel=1)}
\bookmark[named=PrevPage]{Previous page}
\bookmark[level=0, named=FirstPage]{First page (level=0)}
\bookmark[
  rellevel=1,
  keeplevel,
  named=LastPage
]{Last page (rellevel=1, keeplevel)}
\bookmark[named=PrevPage]{Previous page}
\end{document}
%    \end{macrocode}
%    \begin{macrocode}
%</example>
%    \end{macrocode}
%
% \StopEventually{
% }
%
% \section{实现(Implementation)}
%
% \subsection{宏包(Package)}
%
%    \begin{macrocode}
%<*package>
\NeedsTeXFormat{LaTeX2e}
\ProvidesPackage{bookmark}%
  [2020-11-06 v1.29 PDF bookmarks (HO)]%
%    \end{macrocode}
%
% \subsubsection{要求(Requirements)}
%
% \paragraph{\hologo{eTeX}.}
%
%    \begin{macro}{\BKM@CalcExpr}
%    \begin{macrocode}
\begingroup\expandafter\expandafter\expandafter\endgroup
\expandafter\ifx\csname numexpr\endcsname\relax
  \def\BKM@CalcExpr#1#2#3#4{%
    \begingroup
      \count@=#2\relax
      \advance\count@ by#3#4\relax
      \edef\x{\endgroup
        \def\noexpand#1{\the\count@}%
      }%
    \x
  }%
\else
  \def\BKM@CalcExpr#1#2#3#4{%
    \edef#1{%
      \the\numexpr#2#3#4\relax
    }%
  }%
\fi
%    \end{macrocode}
%    \end{macro}
%
% \paragraph{\hologo{pdfTeX}\ 的转义功能(escape features)}
%
%    \begin{macro}{\BKM@EscapeName}
%    \begin{macrocode}
\def\BKM@EscapeName#1{%
  \ifx#1\@empty
  \else
    \EdefEscapeName#1#1%
  \fi
}%
%    \end{macrocode}
%    \end{macro}
%    \begin{macro}{\BKM@EscapeString}
%    \begin{macrocode}
\def\BKM@EscapeString#1{%
  \ifx#1\@empty
  \else
    \EdefEscapeString#1#1%
  \fi
}%
%    \end{macrocode}
%    \end{macro}
%    \begin{macro}{\BKM@EscapeHex}
%    \begin{macrocode}
\def\BKM@EscapeHex#1{%
  \ifx#1\@empty
  \else
    \EdefEscapeHex#1#1%
  \fi
}%
%    \end{macrocode}
%    \end{macro}
%    \begin{macro}{\BKM@UnescapeHex}
%    \begin{macrocode}
\def\BKM@UnescapeHex#1{%
  \EdefUnescapeHex#1#1%
}%
%    \end{macrocode}
%    \end{macro}
%
% \paragraph{宏包(Packages)。}
%
% 不要加载由 \xpackage{hyperref}\ 加载的宏包
%    \begin{macrocode}
\RequirePackage{hyperref}[2010/06/18]
%    \end{macrocode}
%
% \subsubsection{宏包选项(Package options)}
%
%    \begin{macrocode}
\SetupKeyvalOptions{family=BKM,prefix=BKM@}
\DeclareLocalOptions{%
  atend,%
  bold,%
  color,%
  depth,%
  dest,%
  draft,%
  final,%
  gotor,%
  italic,%
  keeplevel,%
  level,%
  named,%
  numbered,%
  open,%
  openlevel,%
  page,%
  rawaction,%
  rellevel,%
  srcfile,%
  srcline,%
  startatroot,%
  uri,%
  view,%
}
%    \end{macrocode}
%    \begin{macro}{\bookmarksetup}
%    \begin{macrocode}
\newcommand*{\bookmarksetup}{\kvsetkeys{BKM}}
%    \end{macrocode}
%    \end{macro}
%    \begin{macro}{\BKM@setup}
%    \begin{macrocode}
\def\BKM@setup#1{%
  \bookmarksetup{#1}%
  \ifx\BKM@HookNext\ltx@empty
  \else
    \expandafter\bookmarksetup\expandafter{\BKM@HookNext}%
    \BKM@HookNextClear
  \fi
  \BKM@hook
  \ifBKM@keeplevel
  \else
    \xdef\BKM@currentlevel{\BKM@level}%
  \fi
}
%    \end{macrocode}
%    \end{macro}
%
%    \begin{macro}{\bookmarksetupnext}
%    \begin{macrocode}
\newcommand*{\bookmarksetupnext}[1]{%
  \ltx@GlobalAppendToMacro\BKM@HookNext{,#1}%
}
%    \end{macrocode}
%    \end{macro}
%    \begin{macro}{\BKM@setupnext}
%    \begin{macrocode}
%    \end{macrocode}
%    \end{macro}
%    \begin{macro}{\BKM@HookNextClear}
%    \begin{macrocode}
\def\BKM@HookNextClear{%
  \global\let\BKM@HookNext\ltx@empty
}
%    \end{macrocode}
%    \end{macro}
%    \begin{macro}{\BKM@HookNext}
%    \begin{macrocode}
\BKM@HookNextClear
%    \end{macrocode}
%    \end{macro}
%
%    \begin{macrocode}
\DeclareBoolOption{draft}
\DeclareComplementaryOption{final}{draft}
%    \end{macrocode}
%    \begin{macro}{\BKM@DisableOptions}
%    \begin{macrocode}
\def\BKM@DisableOptions{%
  \DisableKeyvalOption[action=warning,package=bookmark]%
      {BKM}{draft}%
  \DisableKeyvalOption[action=warning,package=bookmark]%
      {BKM}{final}%
}
%    \end{macrocode}
%    \end{macro}
%    \begin{macrocode}
\DeclareBoolOption[\ifHy@bookmarksopen true\else false\fi]{open}
%    \end{macrocode}
%    \begin{macro}{\bookmark@open}
%    \begin{macrocode}
\def\bookmark@open{%
  \ifBKM@open\ltx@one\else\ltx@zero\fi
}
%    \end{macrocode}
%    \end{macro}
%    \begin{macrocode}
\DeclareStringOption[\maxdimen]{openlevel}
%    \end{macrocode}
%    \begin{macro}{\BKM@openlevel}
%    \begin{macrocode}
\edef\BKM@openlevel{\number\@bookmarksopenlevel}
%    \end{macrocode}
%    \end{macro}
%    \begin{macrocode}
%\DeclareStringOption[\c@tocdepth]{depth}
\ltx@IfUndefined{Hy@bookmarksdepth}{%
  \def\BKM@depth{\c@tocdepth}%
}{%
  \let\BKM@depth\Hy@bookmarksdepth
}
\define@key{BKM}{depth}[]{%
  \edef\BKM@param{#1}%
  \ifx\BKM@param\@empty
    \def\BKM@depth{\c@tocdepth}%
  \else
    \ltx@IfUndefined{toclevel@\BKM@param}{%
      \@onelevel@sanitize\BKM@param
      \edef\BKM@temp{\expandafter\@car\BKM@param\@nil}%
      \ifcase 0\expandafter\ifx\BKM@temp-1\fi
              \expandafter\ifnum\expandafter`\BKM@temp>47 %
                \expandafter\ifnum\expandafter`\BKM@temp<58 %
                  1%
                \fi
              \fi
              \relax
        \PackageWarning{bookmark}{%
          Unknown document division name (\BKM@param)\MessageBreak
          for option `depth'%
        }%
      \else
        \BKM@SetDepthOrLevel\BKM@depth\BKM@param
      \fi
    }{%
      \BKM@SetDepthOrLevel\BKM@depth{%
        \csname toclevel@\BKM@param\endcsname
      }%
    }%
  \fi
}
%    \end{macrocode}
%    \begin{macro}{\bookmark@depth}
%    \begin{macrocode}
\def\bookmark@depth{\BKM@depth}
%    \end{macrocode}
%    \end{macro}
%    \begin{macro}{\BKM@SetDepthOrLevel}
%    \begin{macrocode}
\def\BKM@SetDepthOrLevel#1#2{%
  \begingroup
    \setbox\z@=\hbox{%
      \count@=#2\relax
      \expandafter
    }%
  \expandafter\endgroup
  \expandafter\def\expandafter#1\expandafter{\the\count@}%
}
%    \end{macrocode}
%    \end{macro}
%    \begin{macrocode}
\DeclareStringOption[\BKM@currentlevel]{level}[\BKM@currentlevel]
\define@key{BKM}{level}{%
  \edef\BKM@param{#1}%
  \ifx\BKM@param\BKM@MacroCurrentLevel
    \let\BKM@level\BKM@param
  \else
    \ltx@IfUndefined{toclevel@\BKM@param}{%
      \@onelevel@sanitize\BKM@param
      \edef\BKM@temp{\expandafter\@car\BKM@param\@nil}%
      \ifcase 0\expandafter\ifx\BKM@temp-1\fi
              \expandafter\ifnum\expandafter`\BKM@temp>47 %
                \expandafter\ifnum\expandafter`\BKM@temp<58 %
                  1%
                \fi
              \fi
              \relax
        \PackageWarning{bookmark}{%
          Unknown document division name (\BKM@param)\MessageBreak
          for option `level'%
        }%
      \else
        \BKM@SetDepthOrLevel\BKM@level\BKM@param
      \fi
    }{%
      \BKM@SetDepthOrLevel\BKM@level{%
        \csname toclevel@\BKM@param\endcsname
      }%
    }%
  \fi
}
%    \end{macrocode}
%    \begin{macro}{\BKM@MacroCurrentLevel}
%    \begin{macrocode}
\def\BKM@MacroCurrentLevel{\BKM@currentlevel}
%    \end{macrocode}
%    \end{macro}
%    \begin{macrocode}
\DeclareBoolOption{keeplevel}
\DeclareBoolOption{startatroot}
%    \end{macrocode}
%    \begin{macro}{\BKM@startatrootfalse}
%    \begin{macrocode}
\def\BKM@startatrootfalse{%
  \global\let\ifBKM@startatroot\iffalse
}
%    \end{macrocode}
%    \end{macro}
%    \begin{macro}{\BKM@startatroottrue}
%    \begin{macrocode}
\def\BKM@startatroottrue{%
  \global\let\ifBKM@startatroot\iftrue
}
%    \end{macrocode}
%    \end{macro}
%    \begin{macrocode}
\define@key{BKM}{rellevel}{%
  \BKM@CalcExpr\BKM@level{#1}+\BKM@currentlevel
}
%    \end{macrocode}
%    \begin{macro}{\bookmark@level}
%    \begin{macrocode}
\def\bookmark@level{\BKM@level}
%    \end{macrocode}
%    \end{macro}
%    \begin{macro}{\BKM@currentlevel}
%    \begin{macrocode}
\def\BKM@currentlevel{0}
%    \end{macrocode}
%    \end{macro}
%    Make \xpackage{bookmark}'s option \xoption{numbered} an alias
%    for \xpackage{hyperref}'s \xoption{bookmarksnumbered}.
%    \begin{macrocode}
\DeclareBoolOption[%
  \ifHy@bookmarksnumbered true\else false\fi
]{numbered}
\g@addto@macro\BKM@numberedtrue{%
  \let\ifHy@bookmarksnumbered\iftrue
}
\g@addto@macro\BKM@numberedfalse{%
  \let\ifHy@bookmarksnumbered\iffalse
}
\g@addto@macro\Hy@bookmarksnumberedtrue{%
  \let\ifBKM@numbered\iftrue
}
\g@addto@macro\Hy@bookmarksnumberedfalse{%
  \let\ifBKM@numbered\iffalse
}
%    \end{macrocode}
%    \begin{macro}{\bookmark@numbered}
%    \begin{macrocode}
\def\bookmark@numbered{%
  \ifBKM@numbered\ltx@one\else\ltx@zero\fi
}
%    \end{macrocode}
%    \end{macro}
%
% \paragraph{重定义 \xpackage{hyperref}\ 宏包的选项}
%
%    \begin{macro}{\BKM@PatchHyperrefOption}
%    \begin{macrocode}
\def\BKM@PatchHyperrefOption#1{%
  \expandafter\BKM@@PatchHyperrefOption\csname KV@Hyp@#1\endcsname%
}
%    \end{macrocode}
%    \end{macro}
%    \begin{macro}{\BKM@@PatchHyperrefOption}
%    \begin{macrocode}
\def\BKM@@PatchHyperrefOption#1{%
  \expandafter\BKM@@@PatchHyperrefOption#1{##1}\BKM@nil#1%
}
%    \end{macrocode}
%    \end{macro}
%    \begin{macro}{\BKM@@@PatchHyperrefOption}
%    \begin{macrocode}
\def\BKM@@@PatchHyperrefOption#1\BKM@nil#2#3{%
  \def#2##1{%
    #1%
    \bookmarksetup{#3={##1}}%
  }%
}
%    \end{macrocode}
%    \end{macro}
%    \begin{macrocode}
\BKM@PatchHyperrefOption{bookmarksopen}{open}
\BKM@PatchHyperrefOption{bookmarksopenlevel}{openlevel}
\BKM@PatchHyperrefOption{bookmarksdepth}{depth}
%    \end{macrocode}
%
% \paragraph{字体样式(font style)选项。}
%
%    注意:\xpackage{bitset}\ 宏是基于零的,PDF 规范(PDF specifications)以1开头。
%    \begin{macrocode}
\bitsetReset{BKM@FontStyle}%
\define@key{BKM}{italic}[true]{%
  \expandafter\ifx\csname if#1\endcsname\iftrue
    \bitsetSet{BKM@FontStyle}{0}%
  \else
    \bitsetClear{BKM@FontStyle}{0}%
  \fi
}%
\define@key{BKM}{bold}[true]{%
  \expandafter\ifx\csname if#1\endcsname\iftrue
    \bitsetSet{BKM@FontStyle}{1}%
  \else
    \bitsetClear{BKM@FontStyle}{1}%
  \fi
}%
%    \end{macrocode}
%    \begin{macro}{\bookmark@italic}
%    \begin{macrocode}
\def\bookmark@italic{%
  \ifnum\bitsetGet{BKM@FontStyle}{0}=1 \ltx@one\else\ltx@zero\fi
}
%    \end{macrocode}
%    \end{macro}
%    \begin{macro}{\bookmark@bold}
%    \begin{macrocode}
\def\bookmark@bold{%
  \ifnum\bitsetGet{BKM@FontStyle}{1}=1 \ltx@one\else\ltx@zero\fi
}
%    \end{macrocode}
%    \end{macro}
%    \begin{macro}{\BKM@PrintStyle}
%    \begin{macrocode}
\def\BKM@PrintStyle{%
  \bitsetGetDec{BKM@FontStyle}%
}%
%    \end{macrocode}
%    \end{macro}
%
% \paragraph{颜色(color)选项。}
%
%    \begin{macrocode}
\define@key{BKM}{color}{%
  \HyColor@BookmarkColor{#1}\BKM@color{bookmark}{color}%
}
%    \end{macrocode}
%    \begin{macro}{\BKM@color}
%    \begin{macrocode}
\let\BKM@color\@empty
%    \end{macrocode}
%    \end{macro}
%    \begin{macro}{\bookmark@color}
%    \begin{macrocode}
\def\bookmark@color{\BKM@color}
%    \end{macrocode}
%    \end{macro}
%
% \subsubsection{动作(action)选项}
%
%    \begin{macrocode}
\def\BKM@temp#1{%
  \DeclareStringOption{#1}%
  \expandafter\edef\csname bookmark@#1\endcsname{%
    \expandafter\noexpand\csname BKM@#1\endcsname
  }%
}
%    \end{macrocode}
%    \begin{macro}{\bookmark@dest}
%    \begin{macrocode}
\BKM@temp{dest}
%    \end{macrocode}
%    \end{macro}
%    \begin{macro}{\bookmark@named}
%    \begin{macrocode}
\BKM@temp{named}
%    \end{macrocode}
%    \end{macro}
%    \begin{macro}{\bookmark@uri}
%    \begin{macrocode}
\BKM@temp{uri}
%    \end{macrocode}
%    \end{macro}
%    \begin{macro}{\bookmark@gotor}
%    \begin{macrocode}
\BKM@temp{gotor}
%    \end{macrocode}
%    \end{macro}
%    \begin{macro}{\bookmark@rawaction}
%    \begin{macrocode}
\BKM@temp{rawaction}
%    \end{macrocode}
%    \end{macro}
%
%    \begin{macrocode}
\define@key{BKM}{page}{%
  \def\BKM@page{#1}%
  \ifx\BKM@page\@empty
  \else
    \edef\BKM@page{\number\BKM@page}%
    \ifnum\BKM@page>\z@
    \else
      \PackageError{bookmark}{Page must be positive}\@ehc
      \def\BKM@page{1}%
    \fi
  \fi
}
%    \end{macrocode}
%    \begin{macro}{\BKM@page}
%    \begin{macrocode}
\let\BKM@page\@empty
%    \end{macrocode}
%    \end{macro}
%    \begin{macro}{\bookmark@page}
%    \begin{macrocode}
\def\bookmark@page{\BKM@@page}
%    \end{macrocode}
%    \end{macro}
%
%    \begin{macrocode}
\define@key{BKM}{view}{%
  \BKM@CheckView{#1}%
}
%    \end{macrocode}
%    \begin{macro}{\BKM@view}
%    \begin{macrocode}
\let\BKM@view\@empty
%    \end{macrocode}
%    \end{macro}
%    \begin{macro}{\bookmark@view}
%    \begin{macrocode}
\def\bookmark@view{\BKM@view}
%    \end{macrocode}
%    \end{macro}
%    \begin{macro}{BKM@CheckView}
%    \begin{macrocode}
\def\BKM@CheckView#1{%
  \BKM@CheckViewType#1 \@nil
}
%    \end{macrocode}
%    \end{macro}
%    \begin{macro}{\BKM@CheckViewType}
%    \begin{macrocode}
\def\BKM@CheckViewType#1 #2\@nil{%
  \def\BKM@type{#1}%
  \@onelevel@sanitize\BKM@type
  \BKM@TestViewType{Fit}{}%
  \BKM@TestViewType{FitB}{}%
  \BKM@TestViewType{FitH}{%
    \BKM@CheckParam#2 \@nil{top}%
  }%
  \BKM@TestViewType{FitBH}{%
    \BKM@CheckParam#2 \@nil{top}%
  }%
  \BKM@TestViewType{FitV}{%
    \BKM@CheckParam#2 \@nil{bottom}%
  }%
  \BKM@TestViewType{FitBV}{%
    \BKM@CheckParam#2 \@nil{bottom}%
  }%
  \BKM@TestViewType{FitR}{%
    \BKM@CheckRect{#2}{ }%
  }%
  \BKM@TestViewType{XYZ}{%
    \BKM@CheckXYZ{#2}{ }%
  }%
  \@car{%
    \PackageError{bookmark}{%
      Unknown view type `\BKM@type',\MessageBreak
      using `FitH' instead%
    }\@ehc
    \def\BKM@view{FitH}%
  }%
  \@nil
}
%    \end{macrocode}
%    \end{macro}
%    \begin{macro}{\BKM@TestViewType}
%    \begin{macrocode}
\def\BKM@TestViewType#1{%
  \def\BKM@temp{#1}%
  \@onelevel@sanitize\BKM@temp
  \ifx\BKM@type\BKM@temp
    \let\BKM@view\BKM@temp
    \expandafter\@car
  \else
    \expandafter\@gobble
  \fi
}
%    \end{macrocode}
%    \end{macro}
%    \begin{macro}{BKM@CheckParam}
%    \begin{macrocode}
\def\BKM@CheckParam#1 #2\@nil#3{%
  \def\BKM@param{#1}%
  \ifx\BKM@param\@empty
    \PackageWarning{bookmark}{%
      Missing parameter (#3) for `\BKM@type',\MessageBreak
      using 0%
    }%
    \def\BKM@param{0}%
  \else
    \BKM@CalcParam
  \fi
  \edef\BKM@view{\BKM@view\space\BKM@param}%
}
%    \end{macrocode}
%    \end{macro}
%    \begin{macro}{BKM@CheckRect}
%    \begin{macrocode}
\def\BKM@CheckRect#1#2{%
  \BKM@@CheckRect#1#2#2#2#2\@nil
}
%    \end{macrocode}
%    \end{macro}
%    \begin{macro}{\BKM@@CheckRect}
%    \begin{macrocode}
\def\BKM@@CheckRect#1 #2 #3 #4 #5\@nil{%
  \def\BKM@temp{0}%
  \def\BKM@param{#1}%
  \ifx\BKM@param\@empty
    \def\BKM@param{0}%
    \def\BKM@temp{1}%
  \else
    \BKM@CalcParam
  \fi
  \edef\BKM@view{\BKM@view\space\BKM@param}%
  \def\BKM@param{#2}%
  \ifx\BKM@param\@empty
    \def\BKM@param{0}%
    \def\BKM@temp{1}%
  \else
    \BKM@CalcParam
  \fi
  \edef\BKM@view{\BKM@view\space\BKM@param}%
  \def\BKM@param{#3}%
  \ifx\BKM@param\@empty
    \def\BKM@param{0}%
    \def\BKM@temp{1}%
  \else
    \BKM@CalcParam
  \fi
  \edef\BKM@view{\BKM@view\space\BKM@param}%
  \def\BKM@param{#4}%
  \ifx\BKM@param\@empty
    \def\BKM@param{0}%
    \def\BKM@temp{1}%
  \else
    \BKM@CalcParam
  \fi
  \edef\BKM@view{\BKM@view\space\BKM@param}%
  \ifnum\BKM@temp>\z@
    \PackageWarning{bookmark}{Missing parameters for `\BKM@type'}%
  \fi
}
%    \end{macrocode}
%    \end{macro}
%    \begin{macro}{\BKM@CheckXYZ}
%    \begin{macrocode}
\def\BKM@CheckXYZ#1#2{%
  \BKM@@CheckXYZ#1#2#2#2\@nil
}
%    \end{macrocode}
%    \end{macro}
%    \begin{macro}{\BKM@@CheckXYZ}
%    \begin{macrocode}
\def\BKM@@CheckXYZ#1 #2 #3 #4\@nil{%
  \def\BKM@param{#1}%
  \let\BKM@temp\BKM@param
  \@onelevel@sanitize\BKM@temp
  \ifx\BKM@param\@empty
    \let\BKM@param\BKM@null
  \else
    \ifx\BKM@temp\BKM@null
    \else
      \BKM@CalcParam
    \fi
  \fi
  \edef\BKM@view{\BKM@view\space\BKM@param}%
  \def\BKM@param{#2}%
  \let\BKM@temp\BKM@param
  \@onelevel@sanitize\BKM@temp
  \ifx\BKM@param\@empty
    \let\BKM@param\BKM@null
  \else
    \ifx\BKM@temp\BKM@null
    \else
      \BKM@CalcParam
    \fi
  \fi
  \edef\BKM@view{\BKM@view\space\BKM@param}%
  \def\BKM@param{#3}%
  \ifx\BKM@param\@empty
    \let\BKM@param\BKM@null
  \fi
  \edef\BKM@view{\BKM@view\space\BKM@param}%
}
%    \end{macrocode}
%    \end{macro}
%    \begin{macro}{\BKM@null}
%    \begin{macrocode}
\def\BKM@null{null}
\@onelevel@sanitize\BKM@null
%    \end{macrocode}
%    \end{macro}
%
%    \begin{macro}{\BKM@CalcParam}
%    \begin{macrocode}
\def\BKM@CalcParam{%
  \begingroup
  \let\calc\@firstofone
  \expandafter\BKM@@CalcParam\BKM@param\@empty\@empty\@nil
}
%    \end{macrocode}
%    \end{macro}
%    \begin{macro}{\BKM@@CalcParam}
%    \begin{macrocode}
\def\BKM@@CalcParam#1#2#3\@nil{%
  \ifx\calc#1%
    \@ifundefined{calc@assign@dimen}{%
      \@ifundefined{dimexpr}{%
        \setlength{\dimen@}{#2}%
      }{%
        \setlength{\dimen@}{\dimexpr#2\relax}%
      }%
    }{%
      \setlength{\dimen@}{#2}%
    }%
    \dimen@.99626\dimen@
    \edef\BKM@param{\strip@pt\dimen@}%
    \expandafter\endgroup
    \expandafter\def\expandafter\BKM@param\expandafter{\BKM@param}%
  \else
    \endgroup
  \fi
}
%    \end{macrocode}
%    \end{macro}
%
% \subsubsection{\xoption{atend}\ 选项}
%
%    \begin{macrocode}
\DeclareBoolOption{atend}
\g@addto@macro\BKM@DisableOptions{%
  \DisableKeyvalOption[action=warning,package=bookmark]%
      {BKM}{atend}%
}
%    \end{macrocode}
%
% \subsubsection{\xoption{style}\ 选项}
%
%    \begin{macro}{\bookmarkdefinestyle}
%    \begin{macrocode}
\newcommand*{\bookmarkdefinestyle}[2]{%
  \@ifundefined{BKM@style@#1}{%
  }{%
    \PackageInfo{bookmark}{Redefining style `#1'}%
  }%
  \@namedef{BKM@style@#1}{#2}%
}
%    \end{macrocode}
%    \end{macro}
%    \begin{macrocode}
\define@key{BKM}{style}{%
  \BKM@StyleCall{#1}%
}
\newif\ifBKM@ok
%    \end{macrocode}
%    \begin{macro}{\BKM@StyleCall}
%    \begin{macrocode}
\def\BKM@StyleCall#1{%
  \@ifundefined{BKM@style@#1}{%
    \PackageWarning{bookmark}{%
      Ignoring unknown style `#1'%
    }%
  }{%
%    \end{macrocode}
%    检查样式堆栈(style stack)。
%    \begin{macrocode}
    \BKM@oktrue
    \edef\BKM@StyleCurrent{#1}%
    \@onelevel@sanitize\BKM@StyleCurrent
    \let\BKM@StyleEntry\BKM@StyleEntryCheck
    \BKM@StyleStack
    \ifBKM@ok
      \expandafter\@firstofone
    \else
      \PackageError{bookmark}{%
        Ignoring recursive call of style `\BKM@StyleCurrent'%
      }\@ehc
      \expandafter\@gobble
    \fi
    {%
%    \end{macrocode}
%    在堆栈上推送当前样式(Push current style on stack)。
%    \begin{macrocode}
      \let\BKM@StyleEntry\relax
      \edef\BKM@StyleStack{%
        \BKM@StyleEntry{\BKM@StyleCurrent}%
        \BKM@StyleStack
      }%
%    \end{macrocode}
%   调用样式(Call style)。
%    \begin{macrocode}
      \expandafter\expandafter\expandafter\bookmarksetup
      \expandafter\expandafter\expandafter{%
        \csname BKM@style@\BKM@StyleCurrent\endcsname
      }%
%    \end{macrocode}
%    从堆栈中弹出当前样式(Pop current style from stack)。
%    \begin{macrocode}
      \BKM@StyleStackPop
    }%
  }%
}
%    \end{macrocode}
%    \end{macro}
%    \begin{macro}{\BKM@StyleStackPop}
%    \begin{macrocode}
\def\BKM@StyleStackPop{%
  \let\BKM@StyleEntry\relax
  \edef\BKM@StyleStack{%
    \expandafter\@gobbletwo\BKM@StyleStack
  }%
}
%    \end{macrocode}
%    \end{macro}
%    \begin{macro}{\BKM@StyleEntryCheck}
%    \begin{macrocode}
\def\BKM@StyleEntryCheck#1{%
  \def\BKM@temp{#1}%
  \ifx\BKM@temp\BKM@StyleCurrent
    \BKM@okfalse
  \fi
}
%    \end{macrocode}
%    \end{macro}
%    \begin{macro}{\BKM@StyleStack}
%    \begin{macrocode}
\def\BKM@StyleStack{}
%    \end{macrocode}
%    \end{macro}
%
% \subsubsection{源文件位置(source file location)选项}
%
%    \begin{macrocode}
\DeclareStringOption{srcline}
\DeclareStringOption{srcfile}
%    \end{macrocode}
%
% \subsubsection{钩子支持(Hook support)}
%
%    \begin{macro}{\BKM@hook}
%    \begin{macrocode}
\def\BKM@hook{}
%    \end{macrocode}
%    \end{macro}
%    \begin{macrocode}
\define@key{BKM}{addtohook}{%
  \ltx@LocalAppendToMacro\BKM@hook{#1}%
}
%    \end{macrocode}
%
%    \begin{macro}{bookmarkget}
%    \begin{macrocode}
\newcommand*{\bookmarkget}[1]{%
  \romannumeral0%
  \ltx@ifundefined{bookmark@#1}{%
    \ltx@space
  }{%
    \expandafter\expandafter\expandafter\ltx@space
    \csname bookmark@#1\endcsname
  }%
}
%    \end{macrocode}
%    \end{macro}
%
% \subsubsection{设置和加载驱动程序}
%
% \paragraph{检测驱动程序。}
%
%    \begin{macro}{\BKM@DefineDriverKey}
%    \begin{macrocode}
\def\BKM@DefineDriverKey#1{%
  \define@key{BKM}{#1}[]{%
    \def\BKM@driver{#1}%
  }%
  \g@addto@macro\BKM@DisableOptions{%
    \DisableKeyvalOption[action=warning,package=bookmark]%
        {BKM}{#1}%
  }%
}
%    \end{macrocode}
%    \end{macro}
%    \begin{macrocode}
\BKM@DefineDriverKey{pdftex}
\BKM@DefineDriverKey{dvips}
\BKM@DefineDriverKey{dvipdfm}
\BKM@DefineDriverKey{dvipdfmx}
\BKM@DefineDriverKey{xetex}
\BKM@DefineDriverKey{vtex}
\define@key{BKM}{dvipdfmx-outline-open}[true]{%
 \PackageWarning{bookmark}{Option 'dvipdfmx-outline-open' is obsolete
   and ignored}{}}
%    \end{macrocode}
%    \begin{macro}{\bookmark@driver}
%    \begin{macrocode}
\def\bookmark@driver{\BKM@driver}
%    \end{macrocode}
%    \end{macro}
%    \begin{macrocode}
\InputIfFileExists{bookmark.cfg}{}{}
%    \end{macrocode}
%    \begin{macro}{\BookmarkDriverDefault}
%    \begin{macrocode}
\providecommand*{\BookmarkDriverDefault}{dvips}
%    \end{macrocode}
%    \end{macro}
%    \begin{macro}{\BKM@driver}
% Lua\TeX\ 和 pdf\TeX\ 共享驱动程序。
%    \begin{macrocode}
\ifpdf
  \def\BKM@driver{pdftex}%
  \ifx\pdfoutline\@undefined
    \ifx\pdfextension\@undefined\else
      \protected\def\pdfoutline{\pdfextension outline }
    \fi
  \fi
\else
  \ifxetex
    \def\BKM@driver{dvipdfm}%
  \else
    \ifvtex
      \def\BKM@driver{vtex}%
    \else
      \edef\BKM@driver{\BookmarkDriverDefault}%
    \fi
  \fi
\fi
%    \end{macrocode}
%    \end{macro}
%
% \paragraph{过程选项(Process options)。}
%
%    \begin{macrocode}
\ProcessKeyvalOptions*
\BKM@DisableOptions
%    \end{macrocode}
%
% \paragraph{\xoption{draft}\ 选项}
%
%    \begin{macrocode}
\ifBKM@draft
  \PackageWarningNoLine{bookmark}{Draft mode on}%
  \let\bookmarksetup\ltx@gobble
  \let\BookmarkAtEnd\ltx@gobble
  \let\bookmarkdefinestyle\ltx@gobbletwo
  \let\bookmarkget\ltx@gobble
  \let\pdfbookmark\ltx@undefined
  \newcommand*{\pdfbookmark}[3][]{}%
  \let\currentpdfbookmark\ltx@gobbletwo
  \let\subpdfbookmark\ltx@gobbletwo
  \let\belowpdfbookmark\ltx@gobbletwo
  \newcommand*{\bookmark}[2][]{}%
  \renewcommand*{\Hy@writebookmark}[5]{}%
  \let\ReadBookmarks\relax
  \let\BKM@DefGotoNameAction\ltx@gobbletwo % package `hypdestopt'
  \expandafter\endinput
\fi
%    \end{macrocode}
%
% \paragraph{验证和加载驱动程序。}
%
%    \begin{macrocode}
\def\BKM@temp{dvipdfmx}%
\ifx\BKM@temp\BKM@driver
  \def\BKM@driver{dvipdfm}%
\fi
\def\BKM@temp{pdftex}%
\ifpdf
  \ifx\BKM@temp\BKM@driver
  \else
    \PackageWarningNoLine{bookmark}{%
      Wrong driver `\BKM@driver', using `pdftex' instead%
    }%
    \let\BKM@driver\BKM@temp
  \fi
\else
  \ifx\BKM@temp\BKM@driver
    \PackageError{bookmark}{%
      Wrong driver, pdfTeX is not running in PDF mode.\MessageBreak
      Package loading is aborted%
    }\@ehc
    \expandafter\expandafter\expandafter\endinput
  \fi
  \def\BKM@temp{dvipdfm}%
  \ifxetex
    \ifx\BKM@temp\BKM@driver
    \else
      \PackageWarningNoLine{bookmark}{%
        Wrong driver `\BKM@driver',\MessageBreak
        using `dvipdfm' for XeTeX instead%
      }%
      \let\BKM@driver\BKM@temp
    \fi
  \else
    \def\BKM@temp{vtex}%
    \ifvtex
      \ifx\BKM@temp\BKM@driver
      \else
        \PackageWarningNoLine{bookmark}{%
          Wrong driver `\BKM@driver',\MessageBreak
          using `vtex' for VTeX instead%
        }%
        \let\BKM@driver\BKM@temp
      \fi
    \else
      \ifx\BKM@temp\BKM@driver
        \PackageError{bookmark}{%
          Wrong driver, VTeX is not running in PDF mode.\MessageBreak
          Package loading is aborted%
        }\@ehc
        \expandafter\expandafter\expandafter\endinput
      \fi
    \fi
  \fi
\fi
\providecommand\IfFormatAtLeastTF{\@ifl@t@r\fmtversion}
\IfFormatAtLeastTF{2020/10/01}{}{\edef\BKM@driver{\BKM@driver-2019-12-03}}
\InputIfFileExists{bkm-\BKM@driver.def}{}{%
  \PackageError{bookmark}{%
    Unsupported driver `\BKM@driver'.\MessageBreak
    Package loading is aborted%
  }\@ehc
  \endinput
}
%    \end{macrocode}
%
% \subsubsection{与 \xpackage{hyperref}\ 的兼容性}
%
%    \begin{macro}{\pdfbookmark}
%    \begin{macrocode}
\let\pdfbookmark\ltx@undefined
\newcommand*{\pdfbookmark}[3][0]{%
  \bookmark[level=#1,dest={#3.#1}]{#2}%
  \hyper@anchorstart{#3.#1}\hyper@anchorend
}
%    \end{macrocode}
%    \end{macro}
%    \begin{macro}{\currentpdfbookmark}
%    \begin{macrocode}
\def\currentpdfbookmark{%
  \pdfbookmark[\BKM@currentlevel]%
}
%    \end{macrocode}
%    \end{macro}
%    \begin{macro}{\subpdfbookmark}
%    \begin{macrocode}
\def\subpdfbookmark{%
  \BKM@CalcExpr\BKM@CalcResult\BKM@currentlevel+1%
  \expandafter\pdfbookmark\expandafter[\BKM@CalcResult]%
}
%    \end{macrocode}
%    \end{macro}
%    \begin{macro}{\belowpdfbookmark}
%    \begin{macrocode}
\def\belowpdfbookmark#1#2{%
  \xdef\BKM@gtemp{\number\BKM@currentlevel}%
  \subpdfbookmark{#1}{#2}%
  \global\let\BKM@currentlevel\BKM@gtemp
}
%    \end{macrocode}
%    \end{macro}
%
%    节号(section number)、文本(text)、标签(label)、级别(level)、文件(file)
%    \begin{macro}{\Hy@writebookmark}
%    \begin{macrocode}
\def\Hy@writebookmark#1#2#3#4#5{%
  \ifnum#4>\BKM@depth\relax
  \else
    \def\BKM@type{#5}%
    \ifx\BKM@type\Hy@bookmarkstype
      \begingroup
        \ifBKM@numbered
          \let\numberline\Hy@numberline
          \let\booknumberline\Hy@numberline
          \let\partnumberline\Hy@numberline
          \let\chapternumberline\Hy@numberline
        \else
          \let\numberline\@gobble
          \let\booknumberline\@gobble
          \let\partnumberline\@gobble
          \let\chapternumberline\@gobble
        \fi
        \bookmark[level=#4,dest={\HyperDestNameFilter{#3}}]{#2}%
      \endgroup
    \fi
  \fi
}
%    \end{macrocode}
%    \end{macro}
%
%    \begin{macro}{\ReadBookmarks}
%    \begin{macrocode}
\let\ReadBookmarks\relax
%    \end{macrocode}
%    \end{macro}
%
%    \begin{macrocode}
%</package>
%    \end{macrocode}
%
% \subsection{dvipdfm 的驱动程序}
%
%    \begin{macrocode}
%<*dvipdfm>
\NeedsTeXFormat{LaTeX2e}
\ProvidesFile{bkm-dvipdfm.def}%
  [2020-11-06 v1.29 bookmark driver for dvipdfm (HO)]%
%    \end{macrocode}
%
%    \begin{macro}{\BKM@id}
%    \begin{macrocode}
\newcount\BKM@id
\BKM@id=\z@
%    \end{macrocode}
%    \end{macro}
%
%    \begin{macro}{\BKM@0}
%    \begin{macrocode}
\@namedef{BKM@0}{000}
%    \end{macrocode}
%    \end{macro}
%    \begin{macro}{\ifBKM@sw}
%    \begin{macrocode}
\newif\ifBKM@sw
%    \end{macrocode}
%    \end{macro}
%
%    \begin{macro}{\bookmark}
%    \begin{macrocode}
\newcommand*{\bookmark}[2][]{%
  \if@filesw
    \begingroup
      \def\bookmark@text{#2}%
      \BKM@setup{#1}%
      \edef\BKM@prev{\the\BKM@id}%
      \global\advance\BKM@id\@ne
      \BKM@swtrue
      \@whilesw\ifBKM@sw\fi{%
        \def\BKM@abslevel{1}%
        \ifnum\ifBKM@startatroot\z@\else\BKM@prev\fi=\z@
          \BKM@startatrootfalse
          \expandafter\xdef\csname BKM@\the\BKM@id\endcsname{%
            0{\BKM@level}\BKM@abslevel
          }%
          \BKM@swfalse
        \else
          \expandafter\expandafter\expandafter\BKM@getx
              \csname BKM@\BKM@prev\endcsname
          \ifnum\BKM@level>\BKM@x@level\relax
            \BKM@CalcExpr\BKM@abslevel\BKM@x@abslevel+1%
            \expandafter\xdef\csname BKM@\the\BKM@id\endcsname{%
              {\BKM@prev}{\BKM@level}\BKM@abslevel
            }%
            \BKM@swfalse
          \else
            \let\BKM@prev\BKM@x@parent
          \fi
        \fi
      }%
      \csname HyPsd@XeTeXBigCharstrue\endcsname
      \pdfstringdef\BKM@title{\bookmark@text}%
      \edef\BKM@FLAGS{\BKM@PrintStyle}%
      \let\BKM@action\@empty
      \ifx\BKM@gotor\@empty
        \ifx\BKM@dest\@empty
          \ifx\BKM@named\@empty
            \ifx\BKM@rawaction\@empty
              \ifx\BKM@uri\@empty
                \ifx\BKM@page\@empty
                  \PackageError{bookmark}{Missing action}\@ehc
                  \edef\BKM@action{/Dest[@page1/Fit]}%
                \else
                  \ifx\BKM@view\@empty
                    \def\BKM@view{Fit}%
                  \fi
                  \edef\BKM@action{/Dest[@page\BKM@page/\BKM@view]}%
                \fi
              \else
                \BKM@EscapeString\BKM@uri
                \edef\BKM@action{%
                  /A<<%
                    /S/URI%
                    /URI(\BKM@uri)%
                  >>%
                }%
              \fi
            \else
              \edef\BKM@action{/A<<\BKM@rawaction>>}%
            \fi
          \else
            \BKM@EscapeName\BKM@named
            \edef\BKM@action{%
              /A<</S/Named/N/\BKM@named>>%
            }%
          \fi
        \else
          \BKM@EscapeString\BKM@dest
          \edef\BKM@action{%
            /A<<%
              /S/GoTo%
              /D(\BKM@dest)%
            >>%
          }%
        \fi
      \else
        \ifx\BKM@dest\@empty
          \ifx\BKM@page\@empty
            \def\BKM@page{0}%
          \else
            \BKM@CalcExpr\BKM@page\BKM@page-1%
          \fi
          \ifx\BKM@view\@empty
            \def\BKM@view{Fit}%
          \fi
          \edef\BKM@action{/D[\BKM@page/\BKM@view]}%
        \else
          \BKM@EscapeString\BKM@dest
          \edef\BKM@action{/D(\BKM@dest)}%
        \fi
        \BKM@EscapeString\BKM@gotor
        \edef\BKM@action{%
          /A<<%
            /S/GoToR%
            /F(\BKM@gotor)%
            \BKM@action
          >>%
        }%
      \fi
      \special{pdf:%
        out
              [%
              \ifBKM@open
                \ifnum\BKM@level<%
                    \expandafter\ltx@firstofone\expandafter
                    {\number\BKM@openlevel} %
                \else
                  -%
                \fi
              \else
                -%
              \fi
              ] %
            \BKM@abslevel
        <<%
          /Title(\BKM@title)%
          \ifx\BKM@color\@empty
          \else
            /C[\BKM@color]%
          \fi
          \ifnum\BKM@FLAGS>\z@
            /F \BKM@FLAGS
          \fi
          \BKM@action
        >>%
      }%
    \endgroup
  \fi
}
%    \end{macrocode}
%    \end{macro}
%    \begin{macro}{\BKM@getx}
%    \begin{macrocode}
\def\BKM@getx#1#2#3{%
  \def\BKM@x@parent{#1}%
  \def\BKM@x@level{#2}%
  \def\BKM@x@abslevel{#3}%
}
%    \end{macrocode}
%    \end{macro}
%
%    \begin{macrocode}
%</dvipdfm>
%    \end{macrocode}
%
% \subsection{\hologo{VTeX}\ 的驱动程序}
%
%    \begin{macrocode}
%<*vtex>
\NeedsTeXFormat{LaTeX2e}
\ProvidesFile{bkm-vtex.def}%
  [2020-11-06 v1.29 bookmark driver for VTeX (HO)]%
%    \end{macrocode}
%
%    \begin{macrocode}
\ifvtexpdf
\else
  \PackageWarningNoLine{bookmark}{%
    The VTeX driver only supports PDF mode%
  }%
\fi
%    \end{macrocode}
%
%    \begin{macro}{\BKM@id}
%    \begin{macrocode}
\newcount\BKM@id
\BKM@id=\z@
%    \end{macrocode}
%    \end{macro}
%
%    \begin{macro}{\BKM@0}
%    \begin{macrocode}
\@namedef{BKM@0}{00}
%    \end{macrocode}
%    \end{macro}
%    \begin{macro}{\ifBKM@sw}
%    \begin{macrocode}
\newif\ifBKM@sw
%    \end{macrocode}
%    \end{macro}
%
%    \begin{macro}{\bookmark}
%    \begin{macrocode}
\newcommand*{\bookmark}[2][]{%
  \if@filesw
    \begingroup
      \def\bookmark@text{#2}%
      \BKM@setup{#1}%
      \edef\BKM@prev{\the\BKM@id}%
      \global\advance\BKM@id\@ne
      \BKM@swtrue
      \@whilesw\ifBKM@sw\fi{%
        \ifnum\ifBKM@startatroot\z@\else\BKM@prev\fi=\z@
          \BKM@startatrootfalse
          \def\BKM@parent{0}%
          \expandafter\xdef\csname BKM@\the\BKM@id\endcsname{%
            0{\BKM@level}%
          }%
          \BKM@swfalse
        \else
          \expandafter\expandafter\expandafter\BKM@getx
              \csname BKM@\BKM@prev\endcsname
          \ifnum\BKM@level>\BKM@x@level\relax
            \let\BKM@parent\BKM@prev
            \expandafter\xdef\csname BKM@\the\BKM@id\endcsname{%
              {\BKM@prev}{\BKM@level}%
            }%
            \BKM@swfalse
          \else
            \let\BKM@prev\BKM@x@parent
          \fi
        \fi
      }%
      \pdfstringdef\BKM@title{\bookmark@text}%
      \BKM@vtex@title
      \edef\BKM@FLAGS{\BKM@PrintStyle}%
      \let\BKM@action\@empty
      \ifx\BKM@gotor\@empty
        \ifx\BKM@dest\@empty
          \ifx\BKM@named\@empty
            \ifx\BKM@rawaction\@empty
              \ifx\BKM@uri\@empty
                \ifx\BKM@page\@empty
                  \PackageError{bookmark}{Missing action}\@ehc
                  \def\BKM@action{!1}%
                \else
                  \edef\BKM@action{!\BKM@page}%
                \fi
              \else
                \BKM@EscapeString\BKM@uri
                \edef\BKM@action{%
                  <u=%
                    /S/URI%
                    /URI(\BKM@uri)%
                  >%
                }%
              \fi
            \else
              \edef\BKM@action{<u=\BKM@rawaction>}%
            \fi
          \else
            \BKM@EscapeName\BKM@named
            \edef\BKM@action{%
              <u=%
                /S/Named%
                /N/\BKM@named
              >%
            }%
          \fi
        \else
          \BKM@EscapeString\BKM@dest
          \edef\BKM@action{\BKM@dest}%
        \fi
      \else
        \ifx\BKM@dest\@empty
          \ifx\BKM@page\@empty
            \def\BKM@page{1}%
          \fi
          \ifx\BKM@view\@empty
            \def\BKM@view{Fit}%
          \fi
          \edef\BKM@action{/D[\BKM@page/\BKM@view]}%
        \else
          \BKM@EscapeString\BKM@dest
          \edef\BKM@action{/D(\BKM@dest)}%
        \fi
        \BKM@EscapeString\BKM@gotor
        \edef\BKM@action{%
          <u=%
            /S/GoToR%
            /F(\BKM@gotor)%
            \BKM@action
          >>%
        }%
      \fi
      \ifx\BKM@color\@empty
        \let\BKM@RGBcolor\@empty
      \else
        \expandafter\BKM@toRGB\BKM@color\@nil
      \fi
      \special{%
        !outline \BKM@action;%
        p=\BKM@parent,%
        i=\number\BKM@id,%
        s=%
          \ifBKM@open
            \ifnum\BKM@level<\BKM@openlevel
              o%
            \else
              c%
            \fi
          \else
            c%
          \fi,%
        \ifx\BKM@RGBcolor\@empty
        \else
          c=\BKM@RGBcolor,%
        \fi
        \ifnum\BKM@FLAGS>\z@
          f=\BKM@FLAGS,%
        \fi
        t=\BKM@title
      }%
    \endgroup
  \fi
}
%    \end{macrocode}
%    \end{macro}
%    \begin{macro}{\BKM@getx}
%    \begin{macrocode}
\def\BKM@getx#1#2{%
  \def\BKM@x@parent{#1}%
  \def\BKM@x@level{#2}%
}
%    \end{macrocode}
%    \end{macro}
%    \begin{macro}{\BKM@toRGB}
%    \begin{macrocode}
\def\BKM@toRGB#1 #2 #3\@nil{%
  \let\BKM@RGBcolor\@empty
  \BKM@toRGBComponent{#1}%
  \BKM@toRGBComponent{#2}%
  \BKM@toRGBComponent{#3}%
}
%    \end{macrocode}
%    \end{macro}
%    \begin{macro}{\BKM@toRGBComponent}
%    \begin{macrocode}
\def\BKM@toRGBComponent#1{%
  \dimen@=#1pt\relax
  \ifdim\dimen@>\z@
    \ifdim\dimen@<\p@
      \dimen@=255\dimen@
      \advance\dimen@ by 32768sp\relax
      \divide\dimen@ by 65536\relax
      \dimen@ii=\dimen@
      \divide\dimen@ii by 16\relax
      \edef\BKM@RGBcolor{%
        \BKM@RGBcolor
        \BKM@toHexDigit\dimen@ii
      }%
      \dimen@ii=16\dimen@ii
      \advance\dimen@-\dimen@ii
      \edef\BKM@RGBcolor{%
        \BKM@RGBcolor
        \BKM@toHexDigit\dimen@
      }%
    \else
      \edef\BKM@RGBcolor{\BKM@RGBcolor FF}%
    \fi
  \else
    \edef\BKM@RGBcolor{\BKM@RGBcolor00}%
  \fi
}
%    \end{macrocode}
%    \end{macro}
%    \begin{macro}{\BKM@toHexDigit}
%    \begin{macrocode}
\def\BKM@toHexDigit#1{%
  \ifcase\expandafter\@firstofone\expandafter{\number#1} %
    0\or 1\or 2\or 3\or 4\or 5\or 6\or 7\or
    8\or 9\or A\or B\or C\or D\or E\or F%
  \fi
}
%    \end{macrocode}
%    \end{macro}
%    \begin{macrocode}
\begingroup
  \catcode`\|=0 %
  \catcode`\\=12 %
%    \end{macrocode}
%    \begin{macro}{\BKM@vtex@title}
%    \begin{macrocode}
  |gdef|BKM@vtex@title{%
    |@onelevel@sanitize|BKM@title
    |edef|BKM@title{|expandafter|BKM@vtex@leftparen|BKM@title\(|@nil}%
    |edef|BKM@title{|expandafter|BKM@vtex@rightparen|BKM@title\)|@nil}%
    |edef|BKM@title{|expandafter|BKM@vtex@zero|BKM@title\0|@nil}%
    |edef|BKM@title{|expandafter|BKM@vtex@one|BKM@title\1|@nil}%
    |edef|BKM@title{|expandafter|BKM@vtex@two|BKM@title\2|@nil}%
    |edef|BKM@title{|expandafter|BKM@vtex@three|BKM@title\3|@nil}%
  }%
%    \end{macrocode}
%    \end{macro}
%    \begin{macro}{\BKM@vtex@leftparen}
%    \begin{macrocode}
  |gdef|BKM@vtex@leftparen#1\(#2|@nil{%
    #1%
    |ifx||#2||%
    |else
      (%
      |ltx@ReturnAfterFi{%
        |BKM@vtex@leftparen#2|@nil
      }%
    |fi
  }%
%    \end{macrocode}
%    \end{macro}
%    \begin{macro}{\BKM@vtex@rightparen}
%    \begin{macrocode}
  |gdef|BKM@vtex@rightparen#1\)#2|@nil{%
    #1%
    |ifx||#2||%
    |else
      )%
      |ltx@ReturnAfterFi{%
        |BKM@vtex@rightparen#2|@nil
      }%
    |fi
  }%
%    \end{macrocode}
%    \end{macro}
%    \begin{macro}{\BKM@vtex@zero}
%    \begin{macrocode}
  |gdef|BKM@vtex@zero#1\0#2|@nil{%
    #1%
    |ifx||#2||%
    |else
      |noexpand|hv@pdf@char0%
      |ltx@ReturnAfterFi{%
        |BKM@vtex@zero#2|@nil
      }%
    |fi
  }%
%    \end{macrocode}
%    \end{macro}
%    \begin{macro}{\BKM@vtex@one}
%    \begin{macrocode}
  |gdef|BKM@vtex@one#1\1#2|@nil{%
    #1%
    |ifx||#2||%
    |else
      |noexpand|hv@pdf@char1%
      |ltx@ReturnAfterFi{%
        |BKM@vtex@one#2|@nil
      }%
    |fi
  }%
%    \end{macrocode}
%    \end{macro}
%    \begin{macro}{\BKM@vtex@two}
%    \begin{macrocode}
  |gdef|BKM@vtex@two#1\2#2|@nil{%
    #1%
    |ifx||#2||%
    |else
      |noexpand|hv@pdf@char2%
      |ltx@ReturnAfterFi{%
        |BKM@vtex@two#2|@nil
      }%
    |fi
  }%
%    \end{macrocode}
%    \end{macro}
%    \begin{macro}{\BKM@vtex@three}
%    \begin{macrocode}
  |gdef|BKM@vtex@three#1\3#2|@nil{%
    #1%
    |ifx||#2||%
    |else
      |noexpand|hv@pdf@char3%
      |ltx@ReturnAfterFi{%
        |BKM@vtex@three#2|@nil
      }%
    |fi
  }%
%    \end{macrocode}
%    \end{macro}
%    \begin{macrocode}
|endgroup
%    \end{macrocode}
%
%    \begin{macrocode}
%</vtex>
%    \end{macrocode}
%
% \subsection{\hologo{pdfTeX}\ 的驱动程序}
%
%    \begin{macrocode}
%<*pdftex>
\NeedsTeXFormat{LaTeX2e}
\ProvidesFile{bkm-pdftex.def}%
  [2020-11-06 v1.29 bookmark driver for pdfTeX (HO)]%
%    \end{macrocode}
%
%    \begin{macro}{\BKM@DO@entry}
%    \begin{macrocode}
\def\BKM@DO@entry#1#2{%
  \begingroup
    \kvsetkeys{BKM@DO}{#1}%
    \def\BKM@DO@title{#2}%
    \ifx\BKM@DO@srcfile\@empty
    \else
      \BKM@UnescapeHex\BKM@DO@srcfile
    \fi
    \BKM@UnescapeHex\BKM@DO@title
    \expandafter\expandafter\expandafter\BKM@getx
        \csname BKM@\BKM@DO@id\endcsname\@empty\@empty
    \let\BKM@attr\@empty
    \ifx\BKM@DO@flags\@empty
    \else
      \edef\BKM@attr{\BKM@attr/F \BKM@DO@flags}%
    \fi
    \ifx\BKM@DO@color\@empty
    \else
      \edef\BKM@attr{\BKM@attr/C[\BKM@DO@color]}%
    \fi
    \ifx\BKM@attr\@empty
    \else
      \edef\BKM@attr{attr{\BKM@attr}}%
    \fi
    \let\BKM@action\@empty
    \ifx\BKM@DO@gotor\@empty
      \ifx\BKM@DO@dest\@empty
        \ifx\BKM@DO@named\@empty
          \ifx\BKM@DO@rawaction\@empty
            \ifx\BKM@DO@uri\@empty
              \ifx\BKM@DO@page\@empty
                \PackageError{bookmark}{%
                  Missing action\BKM@SourceLocation
                }\@ehc
                \edef\BKM@action{goto page1{/Fit}}%
              \else
                \ifx\BKM@DO@view\@empty
                  \def\BKM@DO@view{Fit}%
                \fi
                \edef\BKM@action{goto page\BKM@DO@page{/\BKM@DO@view}}%
              \fi
            \else
              \BKM@UnescapeHex\BKM@DO@uri
              \BKM@EscapeString\BKM@DO@uri
              \edef\BKM@action{user{<</S/URI/URI(\BKM@DO@uri)>>}}%
            \fi
          \else
            \BKM@UnescapeHex\BKM@DO@rawaction
            \edef\BKM@action{%
              user{%
                <<%
                  \BKM@DO@rawaction
                >>%
              }%
            }%
          \fi
        \else
          \BKM@EscapeName\BKM@DO@named
          \edef\BKM@action{%
            user{<</S/Named/N/\BKM@DO@named>>}%
          }%
        \fi
      \else
        \BKM@UnescapeHex\BKM@DO@dest
        \BKM@DefGotoNameAction\BKM@action\BKM@DO@dest
      \fi
    \else
      \ifx\BKM@DO@dest\@empty
        \ifx\BKM@DO@page\@empty
          \def\BKM@DO@page{0}%
        \else
          \BKM@CalcExpr\BKM@DO@page\BKM@DO@page-1%
        \fi
        \ifx\BKM@DO@view\@empty
          \def\BKM@DO@view{Fit}%
        \fi
        \edef\BKM@action{/D[\BKM@DO@page/\BKM@DO@view]}%
      \else
        \BKM@UnescapeHex\BKM@DO@dest
        \BKM@EscapeString\BKM@DO@dest
        \edef\BKM@action{/D(\BKM@DO@dest)}%
      \fi
      \BKM@UnescapeHex\BKM@DO@gotor
      \BKM@EscapeString\BKM@DO@gotor
      \edef\BKM@action{%
        user{%
          <<%
            /S/GoToR%
            /F(\BKM@DO@gotor)%
            \BKM@action
          >>%
        }%
      }%
    \fi
    \pdfoutline\BKM@attr\BKM@action
                count\ifBKM@DO@open\else-\fi\BKM@x@childs
                {\BKM@DO@title}%
  \endgroup
}
%    \end{macrocode}
%    \end{macro}
%    \begin{macro}{\BKM@DefGotoNameAction}
%    \cs{BKM@DefGotoNameAction}\ 宏是一个用于 \xpackage{hypdestopt}\ 宏包的钩子(hook)。
%    \begin{macrocode}
\def\BKM@DefGotoNameAction#1#2{%
  \BKM@EscapeString\BKM@DO@dest
  \edef#1{goto name{#2}}%
}
%    \end{macrocode}
%    \end{macro}
%    \begin{macrocode}
%</pdftex>
%    \end{macrocode}
%
%    \begin{macrocode}
%<*pdftex|pdfmark>
%    \end{macrocode}
%    \begin{macro}{\BKM@SourceLocation}
%    \begin{macrocode}
\def\BKM@SourceLocation{%
  \ifx\BKM@DO@srcfile\@empty
    \ifx\BKM@DO@srcline\@empty
    \else
      .\MessageBreak
      Source: line \BKM@DO@srcline
    \fi
  \else
    \ifx\BKM@DO@srcline\@empty
      .\MessageBreak
      Source: file `\BKM@DO@srcfile'%
    \else
      .\MessageBreak
      Source: file `\BKM@DO@srcfile', line \BKM@DO@srcline
    \fi
  \fi
}
%    \end{macrocode}
%    \end{macro}
%    \begin{macrocode}
%</pdftex|pdfmark>
%    \end{macrocode}
%
% \subsection{具有 pdfmark 特色(specials)的驱动程序}
%
% \subsubsection{dvips 驱动程序}
%
%    \begin{macrocode}
%<*dvips>
\NeedsTeXFormat{LaTeX2e}
\ProvidesFile{bkm-dvips.def}%
  [2020-11-06 v1.29 bookmark driver for dvips (HO)]%
%    \end{macrocode}
%    \begin{macro}{\BKM@PSHeaderFile}
%    \begin{macrocode}
\def\BKM@PSHeaderFile#1{%
  \special{PSfile=#1}%
}
%    \end{macrocode}
%    \begin{macro}{\BKM@filename}
%    \begin{macrocode}
\def\BKM@filename{\jobname.out.ps}
%    \end{macrocode}
%    \end{macro}
%    \begin{macrocode}
\AddToHook{shipout/lastpage}{%
  \BKM@pdfmark@out
  \BKM@PSHeaderFile\BKM@filename
  }
%    \end{macrocode}
%    \end{macro}
%    \begin{macrocode}
%</dvips>
%    \end{macrocode}
%
% \subsubsection{公共部分(Common part)}
%
%    \begin{macrocode}
%<*pdfmark>
%    \end{macrocode}
%
%    \begin{macro}{\BKM@pdfmark@out}
%    不要在这里使用 \xpackage{rerunfilecheck}\ 宏包,因为在 \hologo{TeX}\ 运行期间不会
%    读取 \cs{BKM@filename}\ 文件。
%    \begin{macrocode}
\def\BKM@pdfmark@out{%
  \if@filesw
    \newwrite\BKM@file
    \immediate\openout\BKM@file=\BKM@filename\relax
    \BKM@write{\@percentchar!}%
    \BKM@write{/pdfmark where{pop}}%
    \BKM@write{%
      {%
        /globaldict where{pop globaldict}{userdict}ifelse%
        /pdfmark/cleartomark load put%
      }%
    }%
    \BKM@write{ifelse}%
  \else
    \let\BKM@write\@gobble
    \let\BKM@DO@entry\@gobbletwo
  \fi
}
%    \end{macrocode}
%    \end{macro}
%    \begin{macro}{\BKM@write}
%    \begin{macrocode}
\def\BKM@write#{%
  \immediate\write\BKM@file
}
%    \end{macrocode}
%    \end{macro}
%
%    \begin{macro}{\BKM@DO@entry}
%    Pdfmark 的规范(specification)说明 |/Color| 是颜色(color)的键名(key name),
%    但是 ghostscript 只将键(key)传递到 PDF 文件中,因此键名必须是 |/C|。
%    \begin{macrocode}
\def\BKM@DO@entry#1#2{%
  \begingroup
    \kvsetkeys{BKM@DO}{#1}%
    \ifx\BKM@DO@srcfile\@empty
    \else
      \BKM@UnescapeHex\BKM@DO@srcfile
    \fi
    \def\BKM@DO@title{#2}%
    \BKM@UnescapeHex\BKM@DO@title
    \expandafter\expandafter\expandafter\BKM@getx
        \csname BKM@\BKM@DO@id\endcsname\@empty\@empty
    \let\BKM@attr\@empty
    \ifx\BKM@DO@flags\@empty
    \else
      \edef\BKM@attr{\BKM@attr/F \BKM@DO@flags}%
    \fi
    \ifx\BKM@DO@color\@empty
    \else
      \edef\BKM@attr{\BKM@attr/C[\BKM@DO@color]}%
    \fi
    \let\BKM@action\@empty
    \ifx\BKM@DO@gotor\@empty
      \ifx\BKM@DO@dest\@empty
        \ifx\BKM@DO@named\@empty
          \ifx\BKM@DO@rawaction\@empty
            \ifx\BKM@DO@uri\@empty
              \ifx\BKM@DO@page\@empty
                \PackageError{bookmark}{%
                  Missing action\BKM@SourceLocation
                }\@ehc
                \edef\BKM@action{%
                  /Action/GoTo%
                  /Page 1%
                  /View[/Fit]%
                }%
              \else
                \ifx\BKM@DO@view\@empty
                  \def\BKM@DO@view{Fit}%
                \fi
                \edef\BKM@action{%
                  /Action/GoTo%
                  /Page \BKM@DO@page
                  /View[/\BKM@DO@view]%
                }%
              \fi
            \else
              \BKM@UnescapeHex\BKM@DO@uri
              \BKM@EscapeString\BKM@DO@uri
              \edef\BKM@action{%
                /Action<<%
                  /Subtype/URI%
                  /URI(\BKM@DO@uri)%
                >>%
              }%
            \fi
          \else
            \BKM@UnescapeHex\BKM@DO@rawaction
            \edef\BKM@action{%
              /Action<<%
                \BKM@DO@rawaction
              >>%
            }%
          \fi
        \else
          \BKM@EscapeName\BKM@DO@named
          \edef\BKM@action{%
            /Action<<%
              /Subtype/Named%
              /N/\BKM@DO@named
            >>%
          }%
        \fi
      \else
        \BKM@UnescapeHex\BKM@DO@dest
        \BKM@EscapeString\BKM@DO@dest
        \edef\BKM@action{%
          /Action/GoTo%
          /Dest(\BKM@DO@dest)cvn%
        }%
      \fi
    \else
      \ifx\BKM@DO@dest\@empty
        \ifx\BKM@DO@page\@empty
          \def\BKM@DO@page{1}%
        \fi
        \ifx\BKM@DO@view\@empty
          \def\BKM@DO@view{Fit}%
        \fi
        \edef\BKM@action{%
          /Page \BKM@DO@page
          /View[/\BKM@DO@view]%
        }%
      \else
        \BKM@UnescapeHex\BKM@DO@dest
        \BKM@EscapeString\BKM@DO@dest
        \edef\BKM@action{%
          /Dest(\BKM@DO@dest)cvn%
        }%
      \fi
      \BKM@UnescapeHex\BKM@DO@gotor
      \BKM@EscapeString\BKM@DO@gotor
      \edef\BKM@action{%
        /Action/GoToR%
        /File(\BKM@DO@gotor)%
        \BKM@action
      }%
    \fi
    \BKM@write{[}%
    \BKM@write{/Title(\BKM@DO@title)}%
    \ifnum\BKM@x@childs>\z@
      \BKM@write{/Count \ifBKM@DO@open\else-\fi\BKM@x@childs}%
    \fi
    \ifx\BKM@attr\@empty
    \else
      \BKM@write{\BKM@attr}%
    \fi
    \BKM@write{\BKM@action}%
    \BKM@write{/OUT pdfmark}%
  \endgroup
}
%    \end{macrocode}
%    \end{macro}
%    \begin{macrocode}
%</pdfmark>
%    \end{macrocode}
%
% \subsection{\xoption{pdftex}\ 和 \xoption{pdfmark}\ 的公共部分}
%
%    \begin{macrocode}
%<*pdftex|pdfmark>
%    \end{macrocode}
%
% \subsubsection{写入辅助文件(auxiliary file)}
%
%    \begin{macrocode}
\AddToHook{begindocument}{%
 \immediate\write\@mainaux{\string\providecommand\string\BKM@entry[2]{}}}
%    \end{macrocode}
%
%    \begin{macro}{\BKM@id}
%    \begin{macrocode}
\newcount\BKM@id
\BKM@id=\z@
%    \end{macrocode}
%    \end{macro}
%
%    \begin{macro}{\BKM@0}
%    \begin{macrocode}
\@namedef{BKM@0}{000}
%    \end{macrocode}
%    \end{macro}
%    \begin{macro}{\ifBKM@sw}
%    \begin{macrocode}
\newif\ifBKM@sw
%    \end{macrocode}
%    \end{macro}
%
%    \begin{macro}{\bookmark}
%    \begin{macrocode}
\newcommand*{\bookmark}[2][]{%
  \if@filesw
    \begingroup
      \BKM@InitSourceLocation
      \def\bookmark@text{#2}%
      \BKM@setup{#1}%
      \ifx\BKM@srcfile\@empty
      \else
        \BKM@EscapeHex\BKM@srcfile
      \fi
      \edef\BKM@prev{\the\BKM@id}%
      \global\advance\BKM@id\@ne
      \BKM@swtrue
      \@whilesw\ifBKM@sw\fi{%
        \ifnum\ifBKM@startatroot\z@\else\BKM@prev\fi=\z@
          \BKM@startatrootfalse
          \expandafter\xdef\csname BKM@\the\BKM@id\endcsname{%
            0{\BKM@level}0%
          }%
          \BKM@swfalse
        \else
          \expandafter\expandafter\expandafter\BKM@getx
              \csname BKM@\BKM@prev\endcsname
          \ifnum\BKM@level>\BKM@x@level\relax
            \expandafter\xdef\csname BKM@\the\BKM@id\endcsname{%
              {\BKM@prev}{\BKM@level}0%
            }%
            \ifnum\BKM@prev>\z@
              \BKM@CalcExpr\BKM@CalcResult\BKM@x@childs+1%
              \expandafter\xdef\csname BKM@\BKM@prev\endcsname{%
                {\BKM@x@parent}{\BKM@x@level}{\BKM@CalcResult}%
              }%
            \fi
            \BKM@swfalse
          \else
            \let\BKM@prev\BKM@x@parent
          \fi
        \fi
      }%
      \pdfstringdef\BKM@title{\bookmark@text}%
      \edef\BKM@FLAGS{\BKM@PrintStyle}%
      \csname BKM@HypDestOptHook\endcsname
      \BKM@EscapeHex\BKM@dest
      \BKM@EscapeHex\BKM@uri
      \BKM@EscapeHex\BKM@gotor
      \BKM@EscapeHex\BKM@rawaction
      \BKM@EscapeHex\BKM@title
      \immediate\write\@mainaux{%
        \string\BKM@entry{%
          id=\number\BKM@id
          \ifBKM@open
            \ifnum\BKM@level<\BKM@openlevel
              ,open%
            \fi
          \fi
          \BKM@auxentry{dest}%
          \BKM@auxentry{named}%
          \BKM@auxentry{uri}%
          \BKM@auxentry{gotor}%
          \BKM@auxentry{page}%
          \BKM@auxentry{view}%
          \BKM@auxentry{rawaction}%
          \BKM@auxentry{color}%
          \ifnum\BKM@FLAGS>\z@
            ,flags=\BKM@FLAGS
          \fi
          \BKM@auxentry{srcline}%
          \BKM@auxentry{srcfile}%
        }{\BKM@title}%
      }%
    \endgroup
  \fi
}
%    \end{macrocode}
%    \end{macro}
%    \begin{macro}{\BKM@getx}
%    \begin{macrocode}
\def\BKM@getx#1#2#3{%
  \def\BKM@x@parent{#1}%
  \def\BKM@x@level{#2}%
  \def\BKM@x@childs{#3}%
}
%    \end{macrocode}
%    \end{macro}
%    \begin{macro}{\BKM@auxentry}
%    \begin{macrocode}
\def\BKM@auxentry#1{%
  \expandafter\ifx\csname BKM@#1\endcsname\@empty
  \else
    ,#1={\csname BKM@#1\endcsname}%
  \fi
}
%    \end{macrocode}
%    \end{macro}
%
%    \begin{macro}{\BKM@InitSourceLocation}
%    \begin{macrocode}
\def\BKM@InitSourceLocation{%
  \edef\BKM@srcline{\the\inputlineno}%
  \BKM@LuaTeX@InitFile
  \ifx\BKM@srcfile\@empty
    \ltx@IfUndefined{currfilepath}{}{%
      \edef\BKM@srcfile{\currfilepath}%
    }%
  \fi
}
%    \end{macrocode}
%    \end{macro}
%    \begin{macro}{\BKM@LuaTeX@InitFile}
%    \begin{macrocode}
\ifluatex
  \ifnum\luatexversion>36 %
    \def\BKM@LuaTeX@InitFile{%
      \begingroup
        \ltx@LocToksA={}%
      \edef\x{\endgroup
        \def\noexpand\BKM@srcfile{%
          \the\expandafter\ltx@LocToksA
          \directlua{%
             if status and status.filename then %
               tex.settoks('ltx@LocToksA', status.filename)%
             end%
          }%
        }%
      }\x
    }%
  \else
    \let\BKM@LuaTeX@InitFile\relax
  \fi
\else
  \let\BKM@LuaTeX@InitFile\relax
\fi
%    \end{macrocode}
%    \end{macro}
%
% \subsubsection{读取辅助数据(auxiliary data)}
%
%    \begin{macrocode}
\SetupKeyvalOptions{family=BKM@DO,prefix=BKM@DO@}
\DeclareStringOption[0]{id}
\DeclareBoolOption{open}
\DeclareStringOption{flags}
\DeclareStringOption{color}
\DeclareStringOption{dest}
\DeclareStringOption{named}
\DeclareStringOption{uri}
\DeclareStringOption{gotor}
\DeclareStringOption{page}
\DeclareStringOption{view}
\DeclareStringOption{rawaction}
\DeclareStringOption{srcline}
\DeclareStringOption{srcfile}
%    \end{macrocode}
%
%    \begin{macrocode}
\AtBeginDocument{%
  \let\BKM@entry\BKM@DO@entry
}
%    \end{macrocode}
%
%    \begin{macrocode}
%</pdftex|pdfmark>
%    \end{macrocode}
%
% \subsection{\xoption{atend}\ 选项}
%
% \subsubsection{钩子(Hook)}
%
%    \begin{macrocode}
%<*package>
%    \end{macrocode}
%    \begin{macrocode}
\ifBKM@atend
\else
%    \end{macrocode}
%    \begin{macro}{\BookmarkAtEnd}
%    这是一个虚拟定义(dummy definition),如果没有给出 \xoption{atend}\ 选项,它将生成一个警告。
%    \begin{macrocode}
  \newcommand{\BookmarkAtEnd}[1]{%
    \PackageWarning{bookmark}{%
      Ignored, because option `atend' is missing%
    }%
  }%
%    \end{macrocode}
%    \end{macro}
%    \begin{macrocode}
  \expandafter\endinput
\fi
%    \end{macrocode}
%    \begin{macro}{\BookmarkAtEnd}
%    \begin{macrocode}
\newcommand*{\BookmarkAtEnd}{%
  \g@addto@macro\BKM@EndHook
}
%    \end{macrocode}
%    \end{macro}
%    \begin{macrocode}
\let\BKM@EndHook\@empty
%    \end{macrocode}
%    \begin{macrocode}
%</package>
%    \end{macrocode}
%
% \subsubsection{在文档末尾使用钩子的驱动程序}
%
%    驱动程序 \xoption{pdftex}\ 使用 LaTeX 钩子 \xoption{enddocument/afterlastpage}
%    (相当于以前使用的 \xpackage{atveryend}\ 的 \cs{AfterLastShipout}),因为它仍然需要 \xext{aux}\ 文件。
%    它使用 \cs{pdfoutline}\ 作为最后一页之后可以使用的书签(bookmakrs)。
%    \begin{itemize}
%    \item
%      驱动程序 \xoption{pdftex}\ 使用 \cs{pdfoutline}, \cs{pdfoutline}\ 可以在最后一页之后使用。
%    \end{itemize}
%    \begin{macrocode}
%<*pdftex>
\ifBKM@atend
  \AddToHook{enddocument/afterlastpage}{%
    \BKM@EndHook
  }%
\fi
%</pdftex>
%    \end{macrocode}
%
% \subsubsection{使用 \xoption{shipout/lastpage}\ 的驱动程序}
%
%    其他驱动程序使用 \cs{special}\ 命令实现 \cs{bookmark}。因此,最后的书签(last bookmarks)
%    必须放在最后一页(last page),而不是之后。不能使用 \cs{AtEndDocument},因为为时已晚,
%    最后一页已经输出了。因此,我们使用 LaTeX 钩子 \xoption{shipout/lastpage}。至少需要运行
%    两次 \hologo{LaTeX}。PostScript 驱动程序 \xoption{dvips}\ 使用外部 PostScript 文件作为书签。
%    为了避免与 pgf 发生冲突,文件写入(file writing)也被移到了最后一个输出页面(shipout page)。
%    \begin{macrocode}
%<*dvipdfm|vtex|pdfmark>
\ifBKM@atend
  \AddToHook{shipout/lastpage}{\BKM@EndHook}%
\fi
%</dvipdfm|vtex|pdfmark>
%    \end{macrocode}
%
% \section{安装(Installation)}
%
% \subsection{下载(Download)}
%
% \paragraph{宏包(Package)。} 在 CTAN\footnote{\CTANpkg{bookmark}}上提供此宏包:
% \begin{description}
% \item[\CTAN{macros/latex/contrib/bookmark/bookmark.dtx}] 源文件(source file)。
% \item[\CTAN{macros/latex/contrib/bookmark/bookmark.pdf}] 文档(documentation)。
% \end{description}
%
%
% \paragraph{捆绑包(Bundle)。} “bookmark”捆绑包(bundle)的所有宏包(packages)都可以在兼
% 容 TDS 的 ZIP 归档文件中找到。在那里,宏包已经被解包,文档文件(documentation files)已经生成。
% 文件(files)和目录(directories)遵循 TDS 标准。
% \begin{description}
% \item[\CTANinstall{install/macros/latex/contrib/bookmark.tds.zip}]
% \end{description}
% \emph{TDS}\ 是指标准的“用于 \TeX\ 文件的目录结构(Directory Structure)”(\CTANpkg{tds})。
% 名称中带有 \xfile{texmf}\ 的目录(directories)通常以这种方式组织。
%
% \subsection{捆绑包(Bundle)的安装}
%
% \paragraph{解压(Unpacking)。} 在您选择的 TDS 树(也称为 \xfile{texmf}\ 树)中解
% 压 \xfile{bookmark.tds.zip},例如(在 linux 中):
% \begin{quote}
%   |unzip bookmark.tds.zip -d ~/texmf|
% \end{quote}
%
% \subsection{宏包(Package)的安装}
%
% \paragraph{解压(Unpacking)。} \xfile{.dtx}\ 文件是一个自解压 \docstrip\ 归档文件(archive)。
% 这些文件是通过 \plainTeX\ 运行 \xfile{.dtx}\ 来提取的:
% \begin{quote}
%   \verb|tex bookmark.dtx|
% \end{quote}
%
% \paragraph{TDS.} 现在,不同的文件必须移动到安装 TDS 树(installation TDS tree)
% (也称为 \xfile{texmf}\ 树)中的不同目录中:
% \begin{quote}
% \def\t{^^A
% \begin{tabular}{@{}>{\ttfamily}l@{ $\rightarrow$ }>{\ttfamily}l@{}}
%   bookmark.sty & tex/latex/bookmark/bookmark.sty\\
%   bkm-dvipdfm.def & tex/latex/bookmark/bkm-dvipdfm.def\\
%   bkm-dvips.def & tex/latex/bookmark/bkm-dvips.def\\
%   bkm-pdftex.def & tex/latex/bookmark/bkm-pdftex.def\\
%   bkm-vtex.def & tex/latex/bookmark/bkm-vtex.def\\
%   bookmark.pdf & doc/latex/bookmark/bookmark.pdf\\
%   bookmark-example.tex & doc/latex/bookmark/bookmark-example.tex\\
%   bookmark.dtx & source/latex/bookmark/bookmark.dtx\\
% \end{tabular}^^A
% }^^A
% \sbox0{\t}^^A
% \ifdim\wd0>\linewidth
%   \begingroup
%     \advance\linewidth by\leftmargin
%     \advance\linewidth by\rightmargin
%   \edef\x{\endgroup
%     \def\noexpand\lw{\the\linewidth}^^A
%   }\x
%   \def\lwbox{^^A
%     \leavevmode
%     \hbox to \linewidth{^^A
%       \kern-\leftmargin\relax
%       \hss
%       \usebox0
%       \hss
%       \kern-\rightmargin\relax
%     }^^A
%   }^^A
%   \ifdim\wd0>\lw
%     \sbox0{\small\t}^^A
%     \ifdim\wd0>\linewidth
%       \ifdim\wd0>\lw
%         \sbox0{\footnotesize\t}^^A
%         \ifdim\wd0>\linewidth
%           \ifdim\wd0>\lw
%             \sbox0{\scriptsize\t}^^A
%             \ifdim\wd0>\linewidth
%               \ifdim\wd0>\lw
%                 \sbox0{\tiny\t}^^A
%                 \ifdim\wd0>\linewidth
%                   \lwbox
%                 \else
%                   \usebox0
%                 \fi
%               \else
%                 \lwbox
%               \fi
%             \else
%               \usebox0
%             \fi
%           \else
%             \lwbox
%           \fi
%         \else
%           \usebox0
%         \fi
%       \else
%         \lwbox
%       \fi
%     \else
%       \usebox0
%     \fi
%   \else
%     \lwbox
%   \fi
% \else
%   \usebox0
% \fi
% \end{quote}
% 如果你有一个 \xfile{docstrip.cfg}\ 文件,该文件能配置并启用 \docstrip\ 的 TDS 安装功能,
% 则一些文件可能已经在正确的位置了,请参阅 \docstrip\ 的文档(documentation)。
%
% \subsection{刷新文件名数据库}
%
% 如果您的 \TeX~发行版(\TeX\,Live、\mikTeX、\dots)依赖于文件名数据库(file name databases),
% 则必须刷新这些文件名数据库。例如,\TeX\,Live\ 用户运行 \verb|texhash| 或 \verb|mktexlsr|。
%
% \subsection{一些感兴趣的细节}
%
% \paragraph{用 \LaTeX\ 解压。}
% \xfile{.dtx}\ 根据格式(format)选择其操作(action):
% \begin{description}
% \item[\plainTeX:] 运行 \docstrip\ 并解压文件。
% \item[\LaTeX:] 生成文档。
% \end{description}
% 如果您坚持通过 \LaTeX\ 使用\docstrip (实际上 \docstrip\ 并不需要 \LaTeX),那么请您的意图告知自动检测程序:
% \begin{quote}
%   \verb|latex \let\install=y% \iffalse meta-comment
%
% File: bookmark.dtx
% Version: 2020-11-06 v1.29
% Info: PDF bookmarks
%
% Copyright (C)
%    2007-2011 Heiko Oberdiek
%    2016-2020 Oberdiek Package Support Group
%    https://github.com/ho-tex/bookmark/issues
%
% This work may be distributed and/or modified under the
% conditions of the LaTeX Project Public License, either
% version 1.3c of this license or (at your option) any later
% version. This version of this license is in
%    https://www.latex-project.org/lppl/lppl-1-3c.txt
% and the latest version of this license is in
%    https://www.latex-project.org/lppl.txt
% and version 1.3 or later is part of all distributions of
% LaTeX version 2005/12/01 or later.
%
% This work has the LPPL maintenance status "maintained".
%
% The Current Maintainers of this work are
% Heiko Oberdiek and the Oberdiek Package Support Group
% https://github.com/ho-tex/bookmark/issues
%
% This work consists of the main source file bookmark.dtx
% and the derived files
%    bookmark.sty, bookmark.pdf, bookmark.ins, bookmark.drv,
%    bkm-dvipdfm.def, bkm-dvips.def,
%    bkm-pdftex.def, bkm-vtex.def,
%    bkm-dvipdfm-2019-12-03.def, bkm-dvips-2019-12-03.def,
%    bkm-pdftex-2019-12-03.def, bkm-vtex-2019-12-03.def,
%    bookmark-example.tex.
%
% Distribution:
%    CTAN:macros/latex/contrib/bookmark/bookmark.dtx
%    CTAN:macros/latex/contrib/bookmark/bookmark-frozen.dtx
%    CTAN:macros/latex/contrib/bookmark/bookmark.pdf
%
% Unpacking:
%    (a) If bookmark.ins is present:
%           tex bookmark.ins
%    (b) Without bookmark.ins:
%           tex bookmark.dtx
%    (c) If you insist on using LaTeX
%           latex \let\install=y% \iffalse meta-comment
%
% File: bookmark.dtx
% Version: 2020-11-06 v1.29
% Info: PDF bookmarks
%
% Copyright (C)
%    2007-2011 Heiko Oberdiek
%    2016-2020 Oberdiek Package Support Group
%    https://github.com/ho-tex/bookmark/issues
%
% This work may be distributed and/or modified under the
% conditions of the LaTeX Project Public License, either
% version 1.3c of this license or (at your option) any later
% version. This version of this license is in
%    https://www.latex-project.org/lppl/lppl-1-3c.txt
% and the latest version of this license is in
%    https://www.latex-project.org/lppl.txt
% and version 1.3 or later is part of all distributions of
% LaTeX version 2005/12/01 or later.
%
% This work has the LPPL maintenance status "maintained".
%
% The Current Maintainers of this work are
% Heiko Oberdiek and the Oberdiek Package Support Group
% https://github.com/ho-tex/bookmark/issues
%
% This work consists of the main source file bookmark.dtx
% and the derived files
%    bookmark.sty, bookmark.pdf, bookmark.ins, bookmark.drv,
%    bkm-dvipdfm.def, bkm-dvips.def,
%    bkm-pdftex.def, bkm-vtex.def,
%    bkm-dvipdfm-2019-12-03.def, bkm-dvips-2019-12-03.def,
%    bkm-pdftex-2019-12-03.def, bkm-vtex-2019-12-03.def,
%    bookmark-example.tex.
%
% Distribution:
%    CTAN:macros/latex/contrib/bookmark/bookmark.dtx
%    CTAN:macros/latex/contrib/bookmark/bookmark-frozen.dtx
%    CTAN:macros/latex/contrib/bookmark/bookmark.pdf
%
% Unpacking:
%    (a) If bookmark.ins is present:
%           tex bookmark.ins
%    (b) Without bookmark.ins:
%           tex bookmark.dtx
%    (c) If you insist on using LaTeX
%           latex \let\install=y\input{bookmark.dtx}
%        (quote the arguments according to the demands of your shell)
%
% Documentation:
%    (a) If bookmark.drv is present:
%           latex bookmark.drv
%    (b) Without bookmark.drv:
%           latex bookmark.dtx; ...
%    The class ltxdoc loads the configuration file ltxdoc.cfg
%    if available. Here you can specify further options, e.g.
%    use A4 as paper format:
%       \PassOptionsToClass{a4paper}{article}
%
%    Programm calls to get the documentation (example):
%       pdflatex bookmark.dtx
%       makeindex -s gind.ist bookmark.idx
%       pdflatex bookmark.dtx
%       makeindex -s gind.ist bookmark.idx
%       pdflatex bookmark.dtx
%
% Installation:
%    TDS:tex/latex/bookmark/bookmark.sty
%    TDS:tex/latex/bookmark/bkm-dvipdfm.def
%    TDS:tex/latex/bookmark/bkm-dvips.def
%    TDS:tex/latex/bookmark/bkm-pdftex.def
%    TDS:tex/latex/bookmark/bkm-vtex.def
%    TDS:tex/latex/bookmark/bkm-dvipdfm-2019-12-03.def
%    TDS:tex/latex/bookmark/bkm-dvips-2019-12-03.def
%    TDS:tex/latex/bookmark/bkm-pdftex-2019-12-03.def
%    TDS:tex/latex/bookmark/bkm-vtex-2019-12-03.def%
%    TDS:doc/latex/bookmark/bookmark.pdf
%    TDS:doc/latex/bookmark/bookmark-example.tex
%    TDS:source/latex/bookmark/bookmark.dtx
%    TDS:source/latex/bookmark/bookmark-frozen.dtx
%
%<*ignore>
\begingroup
  \catcode123=1 %
  \catcode125=2 %
  \def\x{LaTeX2e}%
\expandafter\endgroup
\ifcase 0\ifx\install y1\fi\expandafter
         \ifx\csname processbatchFile\endcsname\relax\else1\fi
         \ifx\fmtname\x\else 1\fi\relax
\else\csname fi\endcsname
%</ignore>
%<*install>
\input docstrip.tex
\Msg{************************************************************************}
\Msg{* Installation}
\Msg{* Package: bookmark 2020-11-06 v1.29 PDF bookmarks (HO)}
\Msg{************************************************************************}

\keepsilent
\askforoverwritefalse

\let\MetaPrefix\relax
\preamble

This is a generated file.

Project: bookmark
Version: 2020-11-06 v1.29

Copyright (C)
   2007-2011 Heiko Oberdiek
   2016-2020 Oberdiek Package Support Group

This work may be distributed and/or modified under the
conditions of the LaTeX Project Public License, either
version 1.3c of this license or (at your option) any later
version. This version of this license is in
   https://www.latex-project.org/lppl/lppl-1-3c.txt
and the latest version of this license is in
   https://www.latex-project.org/lppl.txt
and version 1.3 or later is part of all distributions of
LaTeX version 2005/12/01 or later.

This work has the LPPL maintenance status "maintained".

The Current Maintainers of this work are
Heiko Oberdiek and the Oberdiek Package Support Group
https://github.com/ho-tex/bookmark/issues


This work consists of the main source file bookmark.dtx and bookmark-frozen.dtx
and the derived files
   bookmark.sty, bookmark.pdf, bookmark.ins, bookmark.drv,
   bkm-dvipdfm.def, bkm-dvips.def, bkm-pdftex.def, bkm-vtex.def,
   bkm-dvipdfm-2019-12-03.def, bkm-dvips-2019-12-03.def,
   bkm-pdftex-2019-12-03.def, bkm-vtex-2019-12-03.def,
   bookmark-example.tex.

\endpreamble
\let\MetaPrefix\DoubleperCent

\generate{%
  \file{bookmark.ins}{\from{bookmark.dtx}{install}}%
  \file{bookmark.drv}{\from{bookmark.dtx}{driver}}%
  \usedir{tex/latex/bookmark}%
  \file{bookmark.sty}{\from{bookmark.dtx}{package}}%
  \file{bkm-dvipdfm.def}{\from{bookmark.dtx}{dvipdfm}}%
  \file{bkm-dvips.def}{\from{bookmark.dtx}{dvips,pdfmark}}%
  \file{bkm-pdftex.def}{\from{bookmark.dtx}{pdftex}}%
  \file{bkm-vtex.def}{\from{bookmark.dtx}{vtex}}%
  \usedir{doc/latex/bookmark}%
  \file{bookmark-example.tex}{\from{bookmark.dtx}{example}}%
  \file{bkm-pdftex-2019-12-03.def}{\from{bookmark-frozen.dtx}{pdftexfrozen}}%
  \file{bkm-dvips-2019-12-03.def}{\from{bookmark-frozen.dtx}{dvipsfrozen}}%
  \file{bkm-vtex-2019-12-03.def}{\from{bookmark-frozen.dtx}{vtexfrozen}}%
  \file{bkm-dvipdfm-2019-12-03.def}{\from{bookmark-frozen.dtx}{dvipdfmfrozen}}%
}

\catcode32=13\relax% active space
\let =\space%
\Msg{************************************************************************}
\Msg{*}
\Msg{* To finish the installation you have to move the following}
\Msg{* files into a directory searched by TeX:}
\Msg{*}
\Msg{*     bookmark.sty, bkm-dvipdfm.def, bkm-dvips.def,}
\Msg{*     bkm-pdftex.def, bkm-vtex.def, bkm-dvipdfm-2019-12-03.def,}
\Msg{*     bkm-dvips-2019-12-03.def, bkm-pdftex-2019-12-03.def,}
\Msg{*     and bkm-vtex-2019-12-03.def}
\Msg{*}
\Msg{* To produce the documentation run the file `bookmark.drv'}
\Msg{* through LaTeX.}
\Msg{*}
\Msg{* Happy TeXing!}
\Msg{*}
\Msg{************************************************************************}

\endbatchfile
%</install>
%<*ignore>
\fi
%</ignore>
%<*driver>
\NeedsTeXFormat{LaTeX2e}
\ProvidesFile{bookmark.drv}%
  [2020-11-06 v1.29 PDF bookmarks (HO)]%
\documentclass{ltxdoc}
\usepackage{ctex}
\usepackage{indentfirst}
\setlength{\parindent}{2em}
\usepackage{holtxdoc}[2011/11/22]
\usepackage{xcolor}
\usepackage{hyperref}
\usepackage[open,openlevel=3,atend]{bookmark}[2020/11/06] %%%打开书签,显示的深度为3级,即显示part、section、subsection。
\bookmarksetup{color=red}
\begin{document}

  \renewcommand{\contentsname}{目\quad 录}
  \renewcommand{\abstractname}{摘\quad 要}
  \renewcommand{\historyname}{历史}
  \DocInput{bookmark.dtx}%
\end{document}
%</driver>
% \fi
%
%
%
% \GetFileInfo{bookmark.drv}
%
%% \title{\xpackage{bookmark} 宏包}
% \title{\heiti {\Huge \textbf{\xpackage{bookmark}\ 宏包}}}
% \date{2020-11-06\ \ \ v1.29}
% \author{Heiko Oberdiek \thanks
% {如有问题请点击:\url{https://github.com/ho-tex/bookmark/issues}}\\[5pt]赣医一附院神经科\ \ 黄旭华\ \ \ \ 译}
%
% \maketitle
%
% \begin{abstract}
% 这个宏包为 \xpackage{hyperref}\ 宏包实现了一个新的书签(bookmark)(大纲[outline])组织。现在
% 可以设置样式(style)和颜色(color)等书签属性(bookmark properties)。其他动作类型(action types)可用
% (URI、GoToR、Named)。书签是在第一次编译运行(compile run)中生成的。\xpackage{hyperref}\
% 宏包必需运行两次。
% \end{abstract}
%
% \tableofcontents
%
% \section{文档(Documentation)}
%
% \subsection{介绍}
%
% 这个 \xpackage{bookmark}\ 宏包试图为书签(bookmarks)提供一个更现代的管理:
% \begin{itemize}
% \item 书签已经在第一次 \hologo{TeX}\ 编译运行(compile run)中生成。
% \item 可以更改书签的字体样式(font style)和颜色(color)。
% \item 可以执行比简单的 GoTo 操作(actions)更多的操作。
% \end{itemize}
%
% 与 \xpackage{hyperref} \cite{hyperref} 一样,书签(bookmarks)也是按照书签生成宏
% (bookmark generating macros)(\cs{bookmark})的顺序生成的。级别号(level number)用于
% 定义书签的树结构(tree structure)。限制没有那么严格:
% \begin{itemize}
% \item 级别值(level values)可以跳变(jump)和省略(omit)。\cs{subsubsection}\ 可以跟在
%       \cs{chapter}\ 之后。这种情况如在 \xpackage{hyperref}\ 中则产生错误,它将显示一个警告(warning)
%       并尝试修复此错误。
% \item 多个书签可能指向同一目标(destination)。在 \xpackage{hyperref}\ 中,这会完全弄乱
%       书签树(bookmark tree),因为算法假设(algorithm assumes)目标名称(destination names)
%       是键(keys)(唯一的)。
% \end{itemize}
%
% 注意,这个宏包是作为书签管理(bookmark management)的实验平台(experimentation platform)。
% 欢迎反馈。此外,在未来的版本中,接口(interfaces)也可能发生变化。
%
% \subsection{选项(Options)}
%
% 可在以下四个地方放置选项(options):
% \begin{enumerate}
% \item \cs{usepackage}|[|\meta{options}|]{bookmark}|\\
%       这是放置驱动程序选项(driver options)和 \xoption{atend}\ 选项的唯一位置。
% \item \cs{bookmarksetup}|{|\meta{options}|}|\\
%       此命令仅用于设置选项(setting options)。
% \item \cs{bookmarksetupnext}|{|\meta{options}|}|\\
%       这些选项在下一个 \cs{bookmark}\ 命令的选项之后存储(stored)和调用(called)。
% \item \cs{bookmark}|[|\meta{options}|]{|\meta{title}|}|\\
%       此命令设置书签。选项设置(option settings)仅限于此书签。
% \end{enumerate}
% 异常(Exception):加载该宏包后,无法更改驱动程序选项(Driver options)、\xoption{atend}\ 选项
% 、\xoption{draft}\slash\xoption{final}选项。
%
% \subsubsection{\xoption{draft} 和 \xoption{final}\ 选项}
%
% 如果一个\LaTeX\ 文件要被编译了多次,那么可以使用 \xoption{draft}\ 选项来禁用该宏包的书签内
% 容(bookmark stuff),这样可以节省一点时间。默认 \xoption{final}\ 选项。两个选项都是
% 布尔选项(boolean options),如果没有值,则使用值 |true|。|draft=true| 与 |final=false| 相同。
%
% 除了驱动程序选项(driver options)之外,\xpackage{bookmark}\ 宏包选项都是局部选项(local options)。
% \xoption{draft}\ 选项和 \xoption{final}\ 选项均属于文档类选项(class option)(译者注:文档类选项为全局选项),
% 因此,在 \xpackage{bookmark}\ 宏包中未能看到这两个选项。如果您想使用全局的(global) \xoption{draft}选项
% 来优化第一次 \LaTeX\ 运行(runs),可以在导言(preamble)中引入 \xpackage{ifdraft}\ 宏包并设置 \LaTeX\ 的
% \cs{PassOptionsToPackage},例如:
%\begin{quote}
%\begin{verbatim}
%\documentclass[draft]{article}
%\usepackage{ifdraft}
%\ifdraft{%
%   \PassOptionsToPackage{draft}{bookmark}%
%}{}
%\end{verbatim}
%\end{quote}
%
% \subsubsection{驱动程序选项(Driver options)}
%
% 支持的驱动程序( drivers)包括 \xoption{pdftex}、\xoption{dvips}、\xoption{dvipdfm} (\xoption{xetex})、
% \xoption{vtex}。\hologo{TeX}\ 引擎 \hologo{pdfTeX}、\hologo{XeTeX}、\hologo{VTeX}\ 能被自动检测到。
% 默认的 DVI 驱动程序是 \xoption{dvips}。这可以通过 \cs{BookmarkDriverDefault}\ 在配置
% 文件 \xfile{bookmark.cfg}\ 中进行更改,例如:
% \begin{quote}
% |\def\BookmarkDriverDefault{dvipdfm}|
% \end{quote}
% 当前版本的(current versions)驱动程序使用新的 \LaTeX\ 钩子(\LaTeX-hooks)。如果检测到比
% 2020-10-01 更旧的格式,则将以前驱动程序的冻结版本(frozen versions)作为备份(fallback)。
%
% \paragraph{用 dvipdfmx 打开书签(bookmarks)。}旧版本的宏包有一个 \xoption{dvipdfmx-outline-open}\ 选项
% 可以激活代码,而该代码可以指定一个大纲条目(outline entry)是否打开。该宏包现在假设所有使用的 dvipdfmx 版本都是
% 最新版本,足以理解该代码,因此始终激活该代码。选项本身将被忽略。
%
%
% \subsubsection{布局选项(Layout options)}
%
% \paragraph{字体(Font)选项:}
%
% \begin{description}
% \item[\xoption{bold}:] 如果受 PDF 浏览器(PDF viewer)支持,书签将以粗体字体(bold font)显示(自 PDF 1.4起)。
% \item[\xoption{italic}:] 使用斜体字体(italic font)(自 PDF 1.4起)。
% \end{description}
% \xoption{bold}(粗体) 和 \xoption{italic}(斜体)可以同时使用。而 |false| 值(value)禁用字体选项。
%
% \paragraph{颜色(Color)选项:}
%
% 彩色书签(Colored bookmarks)是 PDF 1.4 的一个特性(feature),并非所有的 PDF 浏览器(PDF viewers)都支持彩色书签。
% \begin{description}
% \item[\xoption{color}:] 这里 color(颜色)可以作为 \xpackage{color}\ 宏包或 \xpackage{xcolor}\ 宏包的
% 颜色规范(color specification)给出。空值(empty value)表示未设置颜色属性。如果未加载 \xpackage{xcolor}\ 宏包,
% 能识别的值(recognized values)只有:
%   \begin{itemize}
%   \item 空值(empty value)表示未设置颜色属性,\\
%         例如:|color={}|
%   \item 颜色模型(color model) rgb 的显式颜色规范(explicit color specification),\\
%         例如,红色(red):|color=[rgb]{1,0,0}|
%   \item 颜色模型(color model)灰(gray)的显式颜色规范(explicit color specification),\\
%         例如,深灰色(dark gray):|color=[gray]{0.25}|
%   \end{itemize}
%   请注意,如果加载了 \xpackage{color}\ 宏包,此限制(restriction)也适用。然而,如果加载了 \xpackage{xcolor}\ 宏包,
%   则可以使用所有颜色规范(color specifications)。
% \end{description}
%
% \subsubsection{动作选项(Action options)}
%
% \begin{description}
% \item[\xoption{dest}:] 目的地名称(destination name)。
% \item[\xoption{page}:] 页码(page number),第一页(first page)为 1。
% \item[\xoption{view}:] 浏览规范(view specification),示例如下:\\
%   |view={FitB}|, |view={FitH 842}|, |view={XYZ 0 100 null}|\ \  一些浏览规范参数(view specification parameters)
%   将数字(numbers)视为具有单位 bp 的参数。它们可以作为普通数字(plain numbers)或在 \cs{calc}\ 内部以
%   长度表达式(length expressions)给出。如果加载了 \xpackage{calc}\ 宏包,则支持该宏包的表达式(expressions)。否则,
%   使用 \hologo{eTeX}\ 的 \cs{dimexpr}。例如:\\
%   |view={FitH \calc{\paperheight-\topmargin-1in}}|\\
%   |view={XYZ 0 \calc{\paperheight} null}|\\
%   注意 \cs{calc}\ 不能用于 |XYZ| 的第三个参数,因为该参数是缩放值(zoom value),而不是长度(length)。

% \item[\xoption{named}:] 已命名的动作(Named action)的名称:\\
%   |FirstPage|(第一页),|LastPage|(最后一页),|NextPage|(下一页),|PrevPage|(前一页)
% \item[\xoption{gotor}:] 外部(external) PDF 文件的名称。
% \item[\xoption{uri}:] URI 规范(URI specification)。
% \item[\xoption{rawaction}:] 原始动作规范(raw action specification)。由于这些规范取决于驱动程序(driver),因此不应使用此选项。
% \end{description}
% 通过分析指定的选项来选择书签的适当动作。动作由不同的选项集(sets of options)区分:
% \begin{quote}
 \begin{tabular}{|@{}r|l@{}|}
%   \hline
%   \ \textbf{动作(Action)}\  & \ \textbf{选项(Options)}\ \\ \hline
%   \ \textsf{GoTo}\  &\  \xoption{dest}\ \\ \hline
%   \ \textsf{GoTo}\  & \ \xoption{page} + \xoption{view}\ \\ \hline
%   \ \textsf{GoToR}\  & \ \xoption{gotor} + \xoption{dest}\ \\ \hline
%   \ \textsf{GoToR}\  & \ \xoption{gotor} + \xoption{page} + \xoption{view}\ \ \ \\ \hline
%   \ \textsf{Named}\  &\  \xoption{named}\ \\ \hline
%   \ \textsf{URI}\  & \ \xoption{uri}\ \\ \hline
% \end{tabular}
% \end{quote}
%
% \paragraph{缺少动作(Missing actions)。}
% 如果动作缺少 \xpackage{bookmark}\ 宏包,则抛出错误消息(error message)。根据驱动程序(driver)
% (\xoption{pdftex}、\xoption{dvips}\ 和好友[friends]),宏包在文档末尾很晚才检测到它。
% 自 2011/04/21 v1.21 版本以后,该宏包尝试打印 \cs{bookmark}\ 的相应出现的行号(line number)和文件名(file name)。
% 然而,\hologo{TeX}\ 确实提供了行号,但不幸的是,文件名是一个秘密(secret)。但该宏包有如下获取文件名的方法:
% \begin{itemize}
% \item 如果 \hologo{LuaTeX} (独立于 DVI 或 PDF 模式)正在运行,则自动使用其 |status.filename|。
% \item 宏包的 \cs{currfile} \cite{currfile}\ 重新定义了 \hologo{LaTeX}\ 的内部结构,以跟踪文件名(file name)。
% 如果加载了该宏包,那么它的 \cs{currfilepath}\ 将被检测到并由 \xpackage{bookmark}\ 自动使用。
% \item 可以通过 \cs{bookmarksetup}\ 或 \cs{bookmark}\ 中的 \xoption{scrfile}\ 选项手动设置(set manually)文件名。
% 但是要小心,手动设置会禁用以前的文件名检测方法。错误的(wrong)或丢失的(missed)文件名设置(file name setting)可能会在错误消息中
% 为您提供错误的源位置(source location)。
% \end{itemize}
%
% \subsubsection{级别选项(Level options)}
%
% 书签条目(bookmark entries)的顺序由 \cs{bookmark}\ 命令的的出现顺序(appearance order)定义。
% 树结构(tree structure)由书签节点(bookmark nodes)的属性 \xoption{level}(级别)构建。
% \xoption{level}\ 的值是整数(integers)。如果书签条目级别的值高于前一个节点,则该条目将成为
% 前一个节点的子(child)节点。差值的绝对值并不重要。
%
% \xpackage{bookmark}\ 宏包能记住全局属性(global property)“current level(当前级别)”中上
% 一个书签条目(previous bookmark entry)的级别。
%
% 级别系统的(level system)行为(behaviour)可以通过以下选项进行配置:
% \begin{description}
% \item[\xoption{level}:]
%    设置级别(level),请参阅上面的说明。如果给出的选项 \xoption{level}\ 没有值,那么将恢复默
%    认行为,即将“当前级别(current level)”用作级别值(level value)。自 2010/10/19 v1.16 版本以来,
%    如果宏 \cs{toclevel@part}、\cs{toclevel@section}\ 被定义过(通过 \xpackage{hyperref}\ 宏包完成,
%    请参阅它的 \xoption{bookmarkdepth}\ 选项),则 \xpackage{bookmark}\ 宏包还支持 |part|、|section| 等名称。
%
% \item[\xoption{rellevel}:]
%    设置相对于前一级别的(previous level)级别。正值表示书签条目成为前一个书签条目的子条目。
% \item[\xoption{keeplevel}:]
%    使用由\xoption{level}\ 或 \xoption{rellevel}\ 设置的级别,但不要更改全局属性“current level(当前级别)”。
%    可以通过设置为 |false| 来禁用该选项。
% \item[\xoption{startatroot}:]
%    此时,书签树(bookmark tree)再次从顶层(top level)开始。下一个书签条目不会作为上一个条目的子条目进行排序。
%    示例场景:文档使用 part。但是,最后几章(last chapters)不应放在最后一部分(last part)下面:
%    \begin{quote}
%\begin{verbatim}
%\documentclass{book}
%[...]
%\begin{document}
%  \part{第一部分}
%    \chapter{第一部分的第1章}
%    [...]
%  \part{第二部分(Second part)}
%    \chapter{第二部分的第1章}
%    [...]
%  \bookmarksetup{startatroot}
%  \chapter{Index}% 不属于第二部分
%\end{document}
%\end{verbatim}
%    \end{quote}
% \end{description}
%
% \subsubsection{样式定义(Style definitions)}
%
% 样式(style)是一组选项设置(option settings)。它可以由宏 \cs{bookmarkdefinestyle}\ 定义,
% 并由它的 \xoption{style}\ 选项使用。
% \begin{declcs}{bookmarkdefinestyle} \M{name} \M{key value list}
% \end{declcs}
% 选项设置(option settings)的 \meta{key value list}(键值列表)被指定为样式名(style \meta{name})。
%
% \begin{description}
% \item[\xoption{style}:]
%   \xoption{style}\ 选项的值是以前定义的样式的名称(name)。现在执行其选项设置(option settings)。
%   选项可以包括 \xoption{style}\ 选项。通过递归调用相同样式的无限递归(endless recursion)被阻止并抛出一个错误。
% \end{description}
%
% \subsubsection{钩子支持(Hook support)}
%
% 处理宏\cs{bookmark}\ 的可选选项(optional options)后,就会调用钩子(hook)。
% \begin{description}
% \item[\xoption{addtohook}:]
%   代码(code)作为该选项的值添加到钩子中。
% \end{description}
%
% \begin{declcs}{bookmarkget} \M{option}
% \end{declcs}
% \cs{bookmarkget}\ 宏提取 \meta{option}\ 选项的最新选项设置(latest option setting)的值。
% 对于布尔选项(boolean option),如果启用布尔选项,则返回 1,否则结果为零。结果数字(resulting numbers)
% 可以直接用于 \cs{ifnum}\ 或 \cs{ifcase}。如果您想要数字 \texttt{0}\ 和 \texttt{1},
% 请在 \cs{bookmarkget}\ 前面加上 \cs{number}\ 作为前缀。\cs{bookmarkget}\ 宏是可展开的(expandable)。
% 如果选项不受支持,则返回空字符串(empty string)。受支持的布尔选项有:
% \begin{quote}
%   \xoption{bold}、
%   \xoption{italic}、
%   \xoption{open}
% \end{quote}
% 其他受支持的选项有:
% \begin{quote}
%   \xoption{depth}、
%   \xoption{dest}、
%   \xoption{color}、
%   \xoption{gotor}、
%   \xoption{level}、
%   \xoption{named}、
%   \xoption{openlevel}、
%   \xoption{page}、
%   \xoption{rawaction}、
%   \xoption{uri}、
%   \xoption{view}、
% \end{quote}
% 另外,以下键(key)是可用的:
% \begin{quote}
%   \xoption{text}
% \end{quote}
% 它返回大纲条目(outline entry)的文本(text)。
%
% \paragraph{选项设置(Option setting)。}
% 在钩子(hook)内部可以使用 \cs{bookmarksetup}\ 设置选项。
%
% \subsection{与 \xpackage{hyperref}\ 的兼容性}
%
% \xpackage{bookmark}\ 宏包自动禁用 \xpackage{hyperref}\ 宏包的书签(bookmarks)。但是,
% \xpackage{bookmark}\ 宏包使用了 \xpackage{hyperref}\ 宏包的一些代码。例如,
% \xpackage{bookmark}\ 宏包重新定义了 \xpackage{hyperref}\ 宏包在 \cs{addcontentsline}\ 命令
% 和其他命令中插入的\cs{Hy@writebookmark}\ 钩子。因此,不应禁用 \xpackage{hyperref}\ 宏包的书签。
%
% \xpackage{bookmark}\ 宏包使用 \xpackage{hyperref}\ 宏包的 \cs{pdfstringdef},且不提供替换(replacement)。
%
% \xpackage{hyperref}\ 宏包的一些选项也能在 \xpackage{bookmark}\ 宏包中实现(implemented):
% \begin{quote}
% \begin{tabular}{|l@{}|l@{}|}
%   \hline
%   \xpackage{hyperref}\ 宏包的选项\  &\ \xpackage{bookmark}\ 宏包的选项\ \ \\ \hline
%   \xoption{bookmarksdepth} &\ \xoption{depth}\\ \hline
%   \xoption{bookmarksopen} & \ \xoption{open}\\ \hline
%   \xoption{bookmarksopenlevel}\ \ \  &\ \xoption{openlevel}\\ \hline
%   \xoption{bookmarksnumbered} \ \ \ &\ \xoption{numbered}\\ \hline
% \end{tabular}
% \end{quote}
%
% 还可以使用以下命令:
% \begin{quote}
%   \cs{pdfbookmark}\\
%   \cs{currentpdfbookmark}\\
%   \cs{subpdfbookmark}\\
%   \cs{belowpdfbookmark}
% \end{quote}
%
% \subsection{在末尾添加书签}
%
% 宏包选项 \xoption{atend}\ 启用以下宏(macro):
% \begin{declcs}{BookmarkAtEnd}
%   \M{stuff}
% \end{declcs}
% \cs{BookmarkAtEnd}\ 宏将 \meta{stuff}\ 放在文档末尾。\meta{stuff}\ 表示书签命令(bookmark commands)。举例:
% \begin{quote}
%\begin{verbatim}
%\usepackage[atend]{bookmark}
%\BookmarkAtEnd{%
%  \bookmarksetup{startatroot}%
%  \bookmark[named=LastPage, level=0]{Last page}%
%}
%\end{verbatim}
% \end{quote}
%
% 或者,可以在 \cs{bookmark}\ 中给出 \xoption{startatroot}\ 选项:
% \begin{quote}
%\begin{verbatim}
%\BookmarkAtEnd{%
%  \bookmark[
%    startatroot,
%    named=LastPage,
%    level=0,
%  ]{Last page}%
%}
%\end{verbatim}
% \end{quote}
%
% \paragraph{备注(Remarks):}
% \begin{itemize}
% \item
%   \cs{BookmarkAtEnd} 隐藏了这样一个事实,即在文档末尾添加书签的方法取决于驱动程序(driver)。
%
%   为此,驱动程序 \xoption{pdftex}\ 使用 \xpackage{atveryend}\ 宏包。如果 \cs{AtEndDocument}\ 太早,
%   最后一个页面(last page)可能不会被发送出去(shipped out)。由于需要 \xext{aux}\ 文件,此驱动程序使
%   用 \cs{AfterLastShipout}。
%
%   其他驱动程序(\xoption{dvipdfm}、\xoption{xetex}、\xoption{vtex})的实现(implementation)
%   取决于 \cs{special},\cs{special}\ 在最后一页之后没有效果。在这种情况下,\xpackage{atenddvi}\ 宏包的
%   \cs{AtEndDvi}\ 有帮助。它将其参数(argument)放在文档的最后一页(last page)。至少需要运行 \hologo{LaTeX}\ 两次,
%   因为最后一页是由引用(reference)检测到的。
%
%   \xoption{dvips}\ 现在使用新的 LaTeX 钩子 \texttt{shipout/lastpage}。
% \item
%   未指定 \cs{BookmarkAtEnd}\ 参数的扩展时间(time of expansion)。这可以立即发生,也可以在文档末尾发生。
% \end{itemize}
%
% \subsection{限制/行动计划}
%
% \begin{itemize}
% \item 支持缺失动作(missing actions)(启动,\dots)。
% \item 对 \xpackage{hyperref}\ 的 \xoption{bookmarkstype}\ 选项进行了更好的设计(design)。
% \end{itemize}
%
% \section{示例(Example)}
%
%    \begin{macrocode}
%<*example>
%    \end{macrocode}
%    \begin{macrocode}
\documentclass{article}
\usepackage{xcolor}[2007/01/21]
\usepackage{hyperref}
\usepackage[
  open,
  openlevel=2,
  atend
]{bookmark}[2019/12/03]

\bookmarksetup{color=blue}

\BookmarkAtEnd{%
  \bookmarksetup{startatroot}%
  \bookmark[named=LastPage, level=0]{End/Last page}%
  \bookmark[named=FirstPage, level=1]{First page}%
}

\begin{document}
\section{First section}
\subsection{Subsection A}
\begin{figure}
  \hypertarget{fig}{}%
  A figure.
\end{figure}
\bookmark[
  rellevel=1,
  keeplevel,
  dest=fig
]{A figure}
\subsection{Subsection B}
\subsubsection{Subsubsection C}
\subsection{Umlauts: \"A\"O\"U\"a\"o\"u\ss}
\newpage
\bookmarksetup{
  bold,
  color=[rgb]{1,0,0}
}
\section{Very important section}
\bookmarksetup{
  italic,
  bold=false,
  color=blue
}
\subsection{Italic section}
\bookmarksetup{
  italic=false
}
\part{Misc}
\section{Diverse}
\subsubsection{Subsubsection, omitting subsection}
\bookmarksetup{
  startatroot
}
\section{Last section outside part}
\subsection{Subsection}
\bookmarksetup{
  color={}
}
\begingroup
  \bookmarksetup{level=0, color=green!80!black}
  \bookmark[named=FirstPage]{First page}
  \bookmark[named=LastPage]{Last page}
  \bookmark[named=PrevPage]{Previous page}
  \bookmark[named=NextPage]{Next page}
\endgroup
\bookmark[
  page=2,
  view=FitH 800
]{Page 2, FitH 800}
\bookmark[
  page=2,
  view=FitBH \calc{\paperheight-\topmargin-1in-\headheight-\headsep}
]{Page 2, FitBH top of text body}
\bookmark[
  uri={http://www.dante.de/},
  color=magenta
]{Dante homepage}
\bookmark[
  gotor={t.pdf},
  page=1,
  view={XYZ 0 1000 null},
  color=cyan!75!black
]{File t.pdf}
\bookmark[named=FirstPage]{First page}
\bookmark[rellevel=1, named=LastPage]{Last page (rellevel=1)}
\bookmark[named=PrevPage]{Previous page}
\bookmark[level=0, named=FirstPage]{First page (level=0)}
\bookmark[
  rellevel=1,
  keeplevel,
  named=LastPage
]{Last page (rellevel=1, keeplevel)}
\bookmark[named=PrevPage]{Previous page}
\end{document}
%    \end{macrocode}
%    \begin{macrocode}
%</example>
%    \end{macrocode}
%
% \StopEventually{
% }
%
% \section{实现(Implementation)}
%
% \subsection{宏包(Package)}
%
%    \begin{macrocode}
%<*package>
\NeedsTeXFormat{LaTeX2e}
\ProvidesPackage{bookmark}%
  [2020-11-06 v1.29 PDF bookmarks (HO)]%
%    \end{macrocode}
%
% \subsubsection{要求(Requirements)}
%
% \paragraph{\hologo{eTeX}.}
%
%    \begin{macro}{\BKM@CalcExpr}
%    \begin{macrocode}
\begingroup\expandafter\expandafter\expandafter\endgroup
\expandafter\ifx\csname numexpr\endcsname\relax
  \def\BKM@CalcExpr#1#2#3#4{%
    \begingroup
      \count@=#2\relax
      \advance\count@ by#3#4\relax
      \edef\x{\endgroup
        \def\noexpand#1{\the\count@}%
      }%
    \x
  }%
\else
  \def\BKM@CalcExpr#1#2#3#4{%
    \edef#1{%
      \the\numexpr#2#3#4\relax
    }%
  }%
\fi
%    \end{macrocode}
%    \end{macro}
%
% \paragraph{\hologo{pdfTeX}\ 的转义功能(escape features)}
%
%    \begin{macro}{\BKM@EscapeName}
%    \begin{macrocode}
\def\BKM@EscapeName#1{%
  \ifx#1\@empty
  \else
    \EdefEscapeName#1#1%
  \fi
}%
%    \end{macrocode}
%    \end{macro}
%    \begin{macro}{\BKM@EscapeString}
%    \begin{macrocode}
\def\BKM@EscapeString#1{%
  \ifx#1\@empty
  \else
    \EdefEscapeString#1#1%
  \fi
}%
%    \end{macrocode}
%    \end{macro}
%    \begin{macro}{\BKM@EscapeHex}
%    \begin{macrocode}
\def\BKM@EscapeHex#1{%
  \ifx#1\@empty
  \else
    \EdefEscapeHex#1#1%
  \fi
}%
%    \end{macrocode}
%    \end{macro}
%    \begin{macro}{\BKM@UnescapeHex}
%    \begin{macrocode}
\def\BKM@UnescapeHex#1{%
  \EdefUnescapeHex#1#1%
}%
%    \end{macrocode}
%    \end{macro}
%
% \paragraph{宏包(Packages)。}
%
% 不要加载由 \xpackage{hyperref}\ 加载的宏包
%    \begin{macrocode}
\RequirePackage{hyperref}[2010/06/18]
%    \end{macrocode}
%
% \subsubsection{宏包选项(Package options)}
%
%    \begin{macrocode}
\SetupKeyvalOptions{family=BKM,prefix=BKM@}
\DeclareLocalOptions{%
  atend,%
  bold,%
  color,%
  depth,%
  dest,%
  draft,%
  final,%
  gotor,%
  italic,%
  keeplevel,%
  level,%
  named,%
  numbered,%
  open,%
  openlevel,%
  page,%
  rawaction,%
  rellevel,%
  srcfile,%
  srcline,%
  startatroot,%
  uri,%
  view,%
}
%    \end{macrocode}
%    \begin{macro}{\bookmarksetup}
%    \begin{macrocode}
\newcommand*{\bookmarksetup}{\kvsetkeys{BKM}}
%    \end{macrocode}
%    \end{macro}
%    \begin{macro}{\BKM@setup}
%    \begin{macrocode}
\def\BKM@setup#1{%
  \bookmarksetup{#1}%
  \ifx\BKM@HookNext\ltx@empty
  \else
    \expandafter\bookmarksetup\expandafter{\BKM@HookNext}%
    \BKM@HookNextClear
  \fi
  \BKM@hook
  \ifBKM@keeplevel
  \else
    \xdef\BKM@currentlevel{\BKM@level}%
  \fi
}
%    \end{macrocode}
%    \end{macro}
%
%    \begin{macro}{\bookmarksetupnext}
%    \begin{macrocode}
\newcommand*{\bookmarksetupnext}[1]{%
  \ltx@GlobalAppendToMacro\BKM@HookNext{,#1}%
}
%    \end{macrocode}
%    \end{macro}
%    \begin{macro}{\BKM@setupnext}
%    \begin{macrocode}
%    \end{macrocode}
%    \end{macro}
%    \begin{macro}{\BKM@HookNextClear}
%    \begin{macrocode}
\def\BKM@HookNextClear{%
  \global\let\BKM@HookNext\ltx@empty
}
%    \end{macrocode}
%    \end{macro}
%    \begin{macro}{\BKM@HookNext}
%    \begin{macrocode}
\BKM@HookNextClear
%    \end{macrocode}
%    \end{macro}
%
%    \begin{macrocode}
\DeclareBoolOption{draft}
\DeclareComplementaryOption{final}{draft}
%    \end{macrocode}
%    \begin{macro}{\BKM@DisableOptions}
%    \begin{macrocode}
\def\BKM@DisableOptions{%
  \DisableKeyvalOption[action=warning,package=bookmark]%
      {BKM}{draft}%
  \DisableKeyvalOption[action=warning,package=bookmark]%
      {BKM}{final}%
}
%    \end{macrocode}
%    \end{macro}
%    \begin{macrocode}
\DeclareBoolOption[\ifHy@bookmarksopen true\else false\fi]{open}
%    \end{macrocode}
%    \begin{macro}{\bookmark@open}
%    \begin{macrocode}
\def\bookmark@open{%
  \ifBKM@open\ltx@one\else\ltx@zero\fi
}
%    \end{macrocode}
%    \end{macro}
%    \begin{macrocode}
\DeclareStringOption[\maxdimen]{openlevel}
%    \end{macrocode}
%    \begin{macro}{\BKM@openlevel}
%    \begin{macrocode}
\edef\BKM@openlevel{\number\@bookmarksopenlevel}
%    \end{macrocode}
%    \end{macro}
%    \begin{macrocode}
%\DeclareStringOption[\c@tocdepth]{depth}
\ltx@IfUndefined{Hy@bookmarksdepth}{%
  \def\BKM@depth{\c@tocdepth}%
}{%
  \let\BKM@depth\Hy@bookmarksdepth
}
\define@key{BKM}{depth}[]{%
  \edef\BKM@param{#1}%
  \ifx\BKM@param\@empty
    \def\BKM@depth{\c@tocdepth}%
  \else
    \ltx@IfUndefined{toclevel@\BKM@param}{%
      \@onelevel@sanitize\BKM@param
      \edef\BKM@temp{\expandafter\@car\BKM@param\@nil}%
      \ifcase 0\expandafter\ifx\BKM@temp-1\fi
              \expandafter\ifnum\expandafter`\BKM@temp>47 %
                \expandafter\ifnum\expandafter`\BKM@temp<58 %
                  1%
                \fi
              \fi
              \relax
        \PackageWarning{bookmark}{%
          Unknown document division name (\BKM@param)\MessageBreak
          for option `depth'%
        }%
      \else
        \BKM@SetDepthOrLevel\BKM@depth\BKM@param
      \fi
    }{%
      \BKM@SetDepthOrLevel\BKM@depth{%
        \csname toclevel@\BKM@param\endcsname
      }%
    }%
  \fi
}
%    \end{macrocode}
%    \begin{macro}{\bookmark@depth}
%    \begin{macrocode}
\def\bookmark@depth{\BKM@depth}
%    \end{macrocode}
%    \end{macro}
%    \begin{macro}{\BKM@SetDepthOrLevel}
%    \begin{macrocode}
\def\BKM@SetDepthOrLevel#1#2{%
  \begingroup
    \setbox\z@=\hbox{%
      \count@=#2\relax
      \expandafter
    }%
  \expandafter\endgroup
  \expandafter\def\expandafter#1\expandafter{\the\count@}%
}
%    \end{macrocode}
%    \end{macro}
%    \begin{macrocode}
\DeclareStringOption[\BKM@currentlevel]{level}[\BKM@currentlevel]
\define@key{BKM}{level}{%
  \edef\BKM@param{#1}%
  \ifx\BKM@param\BKM@MacroCurrentLevel
    \let\BKM@level\BKM@param
  \else
    \ltx@IfUndefined{toclevel@\BKM@param}{%
      \@onelevel@sanitize\BKM@param
      \edef\BKM@temp{\expandafter\@car\BKM@param\@nil}%
      \ifcase 0\expandafter\ifx\BKM@temp-1\fi
              \expandafter\ifnum\expandafter`\BKM@temp>47 %
                \expandafter\ifnum\expandafter`\BKM@temp<58 %
                  1%
                \fi
              \fi
              \relax
        \PackageWarning{bookmark}{%
          Unknown document division name (\BKM@param)\MessageBreak
          for option `level'%
        }%
      \else
        \BKM@SetDepthOrLevel\BKM@level\BKM@param
      \fi
    }{%
      \BKM@SetDepthOrLevel\BKM@level{%
        \csname toclevel@\BKM@param\endcsname
      }%
    }%
  \fi
}
%    \end{macrocode}
%    \begin{macro}{\BKM@MacroCurrentLevel}
%    \begin{macrocode}
\def\BKM@MacroCurrentLevel{\BKM@currentlevel}
%    \end{macrocode}
%    \end{macro}
%    \begin{macrocode}
\DeclareBoolOption{keeplevel}
\DeclareBoolOption{startatroot}
%    \end{macrocode}
%    \begin{macro}{\BKM@startatrootfalse}
%    \begin{macrocode}
\def\BKM@startatrootfalse{%
  \global\let\ifBKM@startatroot\iffalse
}
%    \end{macrocode}
%    \end{macro}
%    \begin{macro}{\BKM@startatroottrue}
%    \begin{macrocode}
\def\BKM@startatroottrue{%
  \global\let\ifBKM@startatroot\iftrue
}
%    \end{macrocode}
%    \end{macro}
%    \begin{macrocode}
\define@key{BKM}{rellevel}{%
  \BKM@CalcExpr\BKM@level{#1}+\BKM@currentlevel
}
%    \end{macrocode}
%    \begin{macro}{\bookmark@level}
%    \begin{macrocode}
\def\bookmark@level{\BKM@level}
%    \end{macrocode}
%    \end{macro}
%    \begin{macro}{\BKM@currentlevel}
%    \begin{macrocode}
\def\BKM@currentlevel{0}
%    \end{macrocode}
%    \end{macro}
%    Make \xpackage{bookmark}'s option \xoption{numbered} an alias
%    for \xpackage{hyperref}'s \xoption{bookmarksnumbered}.
%    \begin{macrocode}
\DeclareBoolOption[%
  \ifHy@bookmarksnumbered true\else false\fi
]{numbered}
\g@addto@macro\BKM@numberedtrue{%
  \let\ifHy@bookmarksnumbered\iftrue
}
\g@addto@macro\BKM@numberedfalse{%
  \let\ifHy@bookmarksnumbered\iffalse
}
\g@addto@macro\Hy@bookmarksnumberedtrue{%
  \let\ifBKM@numbered\iftrue
}
\g@addto@macro\Hy@bookmarksnumberedfalse{%
  \let\ifBKM@numbered\iffalse
}
%    \end{macrocode}
%    \begin{macro}{\bookmark@numbered}
%    \begin{macrocode}
\def\bookmark@numbered{%
  \ifBKM@numbered\ltx@one\else\ltx@zero\fi
}
%    \end{macrocode}
%    \end{macro}
%
% \paragraph{重定义 \xpackage{hyperref}\ 宏包的选项}
%
%    \begin{macro}{\BKM@PatchHyperrefOption}
%    \begin{macrocode}
\def\BKM@PatchHyperrefOption#1{%
  \expandafter\BKM@@PatchHyperrefOption\csname KV@Hyp@#1\endcsname%
}
%    \end{macrocode}
%    \end{macro}
%    \begin{macro}{\BKM@@PatchHyperrefOption}
%    \begin{macrocode}
\def\BKM@@PatchHyperrefOption#1{%
  \expandafter\BKM@@@PatchHyperrefOption#1{##1}\BKM@nil#1%
}
%    \end{macrocode}
%    \end{macro}
%    \begin{macro}{\BKM@@@PatchHyperrefOption}
%    \begin{macrocode}
\def\BKM@@@PatchHyperrefOption#1\BKM@nil#2#3{%
  \def#2##1{%
    #1%
    \bookmarksetup{#3={##1}}%
  }%
}
%    \end{macrocode}
%    \end{macro}
%    \begin{macrocode}
\BKM@PatchHyperrefOption{bookmarksopen}{open}
\BKM@PatchHyperrefOption{bookmarksopenlevel}{openlevel}
\BKM@PatchHyperrefOption{bookmarksdepth}{depth}
%    \end{macrocode}
%
% \paragraph{字体样式(font style)选项。}
%
%    注意:\xpackage{bitset}\ 宏是基于零的,PDF 规范(PDF specifications)以1开头。
%    \begin{macrocode}
\bitsetReset{BKM@FontStyle}%
\define@key{BKM}{italic}[true]{%
  \expandafter\ifx\csname if#1\endcsname\iftrue
    \bitsetSet{BKM@FontStyle}{0}%
  \else
    \bitsetClear{BKM@FontStyle}{0}%
  \fi
}%
\define@key{BKM}{bold}[true]{%
  \expandafter\ifx\csname if#1\endcsname\iftrue
    \bitsetSet{BKM@FontStyle}{1}%
  \else
    \bitsetClear{BKM@FontStyle}{1}%
  \fi
}%
%    \end{macrocode}
%    \begin{macro}{\bookmark@italic}
%    \begin{macrocode}
\def\bookmark@italic{%
  \ifnum\bitsetGet{BKM@FontStyle}{0}=1 \ltx@one\else\ltx@zero\fi
}
%    \end{macrocode}
%    \end{macro}
%    \begin{macro}{\bookmark@bold}
%    \begin{macrocode}
\def\bookmark@bold{%
  \ifnum\bitsetGet{BKM@FontStyle}{1}=1 \ltx@one\else\ltx@zero\fi
}
%    \end{macrocode}
%    \end{macro}
%    \begin{macro}{\BKM@PrintStyle}
%    \begin{macrocode}
\def\BKM@PrintStyle{%
  \bitsetGetDec{BKM@FontStyle}%
}%
%    \end{macrocode}
%    \end{macro}
%
% \paragraph{颜色(color)选项。}
%
%    \begin{macrocode}
\define@key{BKM}{color}{%
  \HyColor@BookmarkColor{#1}\BKM@color{bookmark}{color}%
}
%    \end{macrocode}
%    \begin{macro}{\BKM@color}
%    \begin{macrocode}
\let\BKM@color\@empty
%    \end{macrocode}
%    \end{macro}
%    \begin{macro}{\bookmark@color}
%    \begin{macrocode}
\def\bookmark@color{\BKM@color}
%    \end{macrocode}
%    \end{macro}
%
% \subsubsection{动作(action)选项}
%
%    \begin{macrocode}
\def\BKM@temp#1{%
  \DeclareStringOption{#1}%
  \expandafter\edef\csname bookmark@#1\endcsname{%
    \expandafter\noexpand\csname BKM@#1\endcsname
  }%
}
%    \end{macrocode}
%    \begin{macro}{\bookmark@dest}
%    \begin{macrocode}
\BKM@temp{dest}
%    \end{macrocode}
%    \end{macro}
%    \begin{macro}{\bookmark@named}
%    \begin{macrocode}
\BKM@temp{named}
%    \end{macrocode}
%    \end{macro}
%    \begin{macro}{\bookmark@uri}
%    \begin{macrocode}
\BKM@temp{uri}
%    \end{macrocode}
%    \end{macro}
%    \begin{macro}{\bookmark@gotor}
%    \begin{macrocode}
\BKM@temp{gotor}
%    \end{macrocode}
%    \end{macro}
%    \begin{macro}{\bookmark@rawaction}
%    \begin{macrocode}
\BKM@temp{rawaction}
%    \end{macrocode}
%    \end{macro}
%
%    \begin{macrocode}
\define@key{BKM}{page}{%
  \def\BKM@page{#1}%
  \ifx\BKM@page\@empty
  \else
    \edef\BKM@page{\number\BKM@page}%
    \ifnum\BKM@page>\z@
    \else
      \PackageError{bookmark}{Page must be positive}\@ehc
      \def\BKM@page{1}%
    \fi
  \fi
}
%    \end{macrocode}
%    \begin{macro}{\BKM@page}
%    \begin{macrocode}
\let\BKM@page\@empty
%    \end{macrocode}
%    \end{macro}
%    \begin{macro}{\bookmark@page}
%    \begin{macrocode}
\def\bookmark@page{\BKM@@page}
%    \end{macrocode}
%    \end{macro}
%
%    \begin{macrocode}
\define@key{BKM}{view}{%
  \BKM@CheckView{#1}%
}
%    \end{macrocode}
%    \begin{macro}{\BKM@view}
%    \begin{macrocode}
\let\BKM@view\@empty
%    \end{macrocode}
%    \end{macro}
%    \begin{macro}{\bookmark@view}
%    \begin{macrocode}
\def\bookmark@view{\BKM@view}
%    \end{macrocode}
%    \end{macro}
%    \begin{macro}{BKM@CheckView}
%    \begin{macrocode}
\def\BKM@CheckView#1{%
  \BKM@CheckViewType#1 \@nil
}
%    \end{macrocode}
%    \end{macro}
%    \begin{macro}{\BKM@CheckViewType}
%    \begin{macrocode}
\def\BKM@CheckViewType#1 #2\@nil{%
  \def\BKM@type{#1}%
  \@onelevel@sanitize\BKM@type
  \BKM@TestViewType{Fit}{}%
  \BKM@TestViewType{FitB}{}%
  \BKM@TestViewType{FitH}{%
    \BKM@CheckParam#2 \@nil{top}%
  }%
  \BKM@TestViewType{FitBH}{%
    \BKM@CheckParam#2 \@nil{top}%
  }%
  \BKM@TestViewType{FitV}{%
    \BKM@CheckParam#2 \@nil{bottom}%
  }%
  \BKM@TestViewType{FitBV}{%
    \BKM@CheckParam#2 \@nil{bottom}%
  }%
  \BKM@TestViewType{FitR}{%
    \BKM@CheckRect{#2}{ }%
  }%
  \BKM@TestViewType{XYZ}{%
    \BKM@CheckXYZ{#2}{ }%
  }%
  \@car{%
    \PackageError{bookmark}{%
      Unknown view type `\BKM@type',\MessageBreak
      using `FitH' instead%
    }\@ehc
    \def\BKM@view{FitH}%
  }%
  \@nil
}
%    \end{macrocode}
%    \end{macro}
%    \begin{macro}{\BKM@TestViewType}
%    \begin{macrocode}
\def\BKM@TestViewType#1{%
  \def\BKM@temp{#1}%
  \@onelevel@sanitize\BKM@temp
  \ifx\BKM@type\BKM@temp
    \let\BKM@view\BKM@temp
    \expandafter\@car
  \else
    \expandafter\@gobble
  \fi
}
%    \end{macrocode}
%    \end{macro}
%    \begin{macro}{BKM@CheckParam}
%    \begin{macrocode}
\def\BKM@CheckParam#1 #2\@nil#3{%
  \def\BKM@param{#1}%
  \ifx\BKM@param\@empty
    \PackageWarning{bookmark}{%
      Missing parameter (#3) for `\BKM@type',\MessageBreak
      using 0%
    }%
    \def\BKM@param{0}%
  \else
    \BKM@CalcParam
  \fi
  \edef\BKM@view{\BKM@view\space\BKM@param}%
}
%    \end{macrocode}
%    \end{macro}
%    \begin{macro}{BKM@CheckRect}
%    \begin{macrocode}
\def\BKM@CheckRect#1#2{%
  \BKM@@CheckRect#1#2#2#2#2\@nil
}
%    \end{macrocode}
%    \end{macro}
%    \begin{macro}{\BKM@@CheckRect}
%    \begin{macrocode}
\def\BKM@@CheckRect#1 #2 #3 #4 #5\@nil{%
  \def\BKM@temp{0}%
  \def\BKM@param{#1}%
  \ifx\BKM@param\@empty
    \def\BKM@param{0}%
    \def\BKM@temp{1}%
  \else
    \BKM@CalcParam
  \fi
  \edef\BKM@view{\BKM@view\space\BKM@param}%
  \def\BKM@param{#2}%
  \ifx\BKM@param\@empty
    \def\BKM@param{0}%
    \def\BKM@temp{1}%
  \else
    \BKM@CalcParam
  \fi
  \edef\BKM@view{\BKM@view\space\BKM@param}%
  \def\BKM@param{#3}%
  \ifx\BKM@param\@empty
    \def\BKM@param{0}%
    \def\BKM@temp{1}%
  \else
    \BKM@CalcParam
  \fi
  \edef\BKM@view{\BKM@view\space\BKM@param}%
  \def\BKM@param{#4}%
  \ifx\BKM@param\@empty
    \def\BKM@param{0}%
    \def\BKM@temp{1}%
  \else
    \BKM@CalcParam
  \fi
  \edef\BKM@view{\BKM@view\space\BKM@param}%
  \ifnum\BKM@temp>\z@
    \PackageWarning{bookmark}{Missing parameters for `\BKM@type'}%
  \fi
}
%    \end{macrocode}
%    \end{macro}
%    \begin{macro}{\BKM@CheckXYZ}
%    \begin{macrocode}
\def\BKM@CheckXYZ#1#2{%
  \BKM@@CheckXYZ#1#2#2#2\@nil
}
%    \end{macrocode}
%    \end{macro}
%    \begin{macro}{\BKM@@CheckXYZ}
%    \begin{macrocode}
\def\BKM@@CheckXYZ#1 #2 #3 #4\@nil{%
  \def\BKM@param{#1}%
  \let\BKM@temp\BKM@param
  \@onelevel@sanitize\BKM@temp
  \ifx\BKM@param\@empty
    \let\BKM@param\BKM@null
  \else
    \ifx\BKM@temp\BKM@null
    \else
      \BKM@CalcParam
    \fi
  \fi
  \edef\BKM@view{\BKM@view\space\BKM@param}%
  \def\BKM@param{#2}%
  \let\BKM@temp\BKM@param
  \@onelevel@sanitize\BKM@temp
  \ifx\BKM@param\@empty
    \let\BKM@param\BKM@null
  \else
    \ifx\BKM@temp\BKM@null
    \else
      \BKM@CalcParam
    \fi
  \fi
  \edef\BKM@view{\BKM@view\space\BKM@param}%
  \def\BKM@param{#3}%
  \ifx\BKM@param\@empty
    \let\BKM@param\BKM@null
  \fi
  \edef\BKM@view{\BKM@view\space\BKM@param}%
}
%    \end{macrocode}
%    \end{macro}
%    \begin{macro}{\BKM@null}
%    \begin{macrocode}
\def\BKM@null{null}
\@onelevel@sanitize\BKM@null
%    \end{macrocode}
%    \end{macro}
%
%    \begin{macro}{\BKM@CalcParam}
%    \begin{macrocode}
\def\BKM@CalcParam{%
  \begingroup
  \let\calc\@firstofone
  \expandafter\BKM@@CalcParam\BKM@param\@empty\@empty\@nil
}
%    \end{macrocode}
%    \end{macro}
%    \begin{macro}{\BKM@@CalcParam}
%    \begin{macrocode}
\def\BKM@@CalcParam#1#2#3\@nil{%
  \ifx\calc#1%
    \@ifundefined{calc@assign@dimen}{%
      \@ifundefined{dimexpr}{%
        \setlength{\dimen@}{#2}%
      }{%
        \setlength{\dimen@}{\dimexpr#2\relax}%
      }%
    }{%
      \setlength{\dimen@}{#2}%
    }%
    \dimen@.99626\dimen@
    \edef\BKM@param{\strip@pt\dimen@}%
    \expandafter\endgroup
    \expandafter\def\expandafter\BKM@param\expandafter{\BKM@param}%
  \else
    \endgroup
  \fi
}
%    \end{macrocode}
%    \end{macro}
%
% \subsubsection{\xoption{atend}\ 选项}
%
%    \begin{macrocode}
\DeclareBoolOption{atend}
\g@addto@macro\BKM@DisableOptions{%
  \DisableKeyvalOption[action=warning,package=bookmark]%
      {BKM}{atend}%
}
%    \end{macrocode}
%
% \subsubsection{\xoption{style}\ 选项}
%
%    \begin{macro}{\bookmarkdefinestyle}
%    \begin{macrocode}
\newcommand*{\bookmarkdefinestyle}[2]{%
  \@ifundefined{BKM@style@#1}{%
  }{%
    \PackageInfo{bookmark}{Redefining style `#1'}%
  }%
  \@namedef{BKM@style@#1}{#2}%
}
%    \end{macrocode}
%    \end{macro}
%    \begin{macrocode}
\define@key{BKM}{style}{%
  \BKM@StyleCall{#1}%
}
\newif\ifBKM@ok
%    \end{macrocode}
%    \begin{macro}{\BKM@StyleCall}
%    \begin{macrocode}
\def\BKM@StyleCall#1{%
  \@ifundefined{BKM@style@#1}{%
    \PackageWarning{bookmark}{%
      Ignoring unknown style `#1'%
    }%
  }{%
%    \end{macrocode}
%    检查样式堆栈(style stack)。
%    \begin{macrocode}
    \BKM@oktrue
    \edef\BKM@StyleCurrent{#1}%
    \@onelevel@sanitize\BKM@StyleCurrent
    \let\BKM@StyleEntry\BKM@StyleEntryCheck
    \BKM@StyleStack
    \ifBKM@ok
      \expandafter\@firstofone
    \else
      \PackageError{bookmark}{%
        Ignoring recursive call of style `\BKM@StyleCurrent'%
      }\@ehc
      \expandafter\@gobble
    \fi
    {%
%    \end{macrocode}
%    在堆栈上推送当前样式(Push current style on stack)。
%    \begin{macrocode}
      \let\BKM@StyleEntry\relax
      \edef\BKM@StyleStack{%
        \BKM@StyleEntry{\BKM@StyleCurrent}%
        \BKM@StyleStack
      }%
%    \end{macrocode}
%   调用样式(Call style)。
%    \begin{macrocode}
      \expandafter\expandafter\expandafter\bookmarksetup
      \expandafter\expandafter\expandafter{%
        \csname BKM@style@\BKM@StyleCurrent\endcsname
      }%
%    \end{macrocode}
%    从堆栈中弹出当前样式(Pop current style from stack)。
%    \begin{macrocode}
      \BKM@StyleStackPop
    }%
  }%
}
%    \end{macrocode}
%    \end{macro}
%    \begin{macro}{\BKM@StyleStackPop}
%    \begin{macrocode}
\def\BKM@StyleStackPop{%
  \let\BKM@StyleEntry\relax
  \edef\BKM@StyleStack{%
    \expandafter\@gobbletwo\BKM@StyleStack
  }%
}
%    \end{macrocode}
%    \end{macro}
%    \begin{macro}{\BKM@StyleEntryCheck}
%    \begin{macrocode}
\def\BKM@StyleEntryCheck#1{%
  \def\BKM@temp{#1}%
  \ifx\BKM@temp\BKM@StyleCurrent
    \BKM@okfalse
  \fi
}
%    \end{macrocode}
%    \end{macro}
%    \begin{macro}{\BKM@StyleStack}
%    \begin{macrocode}
\def\BKM@StyleStack{}
%    \end{macrocode}
%    \end{macro}
%
% \subsubsection{源文件位置(source file location)选项}
%
%    \begin{macrocode}
\DeclareStringOption{srcline}
\DeclareStringOption{srcfile}
%    \end{macrocode}
%
% \subsubsection{钩子支持(Hook support)}
%
%    \begin{macro}{\BKM@hook}
%    \begin{macrocode}
\def\BKM@hook{}
%    \end{macrocode}
%    \end{macro}
%    \begin{macrocode}
\define@key{BKM}{addtohook}{%
  \ltx@LocalAppendToMacro\BKM@hook{#1}%
}
%    \end{macrocode}
%
%    \begin{macro}{bookmarkget}
%    \begin{macrocode}
\newcommand*{\bookmarkget}[1]{%
  \romannumeral0%
  \ltx@ifundefined{bookmark@#1}{%
    \ltx@space
  }{%
    \expandafter\expandafter\expandafter\ltx@space
    \csname bookmark@#1\endcsname
  }%
}
%    \end{macrocode}
%    \end{macro}
%
% \subsubsection{设置和加载驱动程序}
%
% \paragraph{检测驱动程序。}
%
%    \begin{macro}{\BKM@DefineDriverKey}
%    \begin{macrocode}
\def\BKM@DefineDriverKey#1{%
  \define@key{BKM}{#1}[]{%
    \def\BKM@driver{#1}%
  }%
  \g@addto@macro\BKM@DisableOptions{%
    \DisableKeyvalOption[action=warning,package=bookmark]%
        {BKM}{#1}%
  }%
}
%    \end{macrocode}
%    \end{macro}
%    \begin{macrocode}
\BKM@DefineDriverKey{pdftex}
\BKM@DefineDriverKey{dvips}
\BKM@DefineDriverKey{dvipdfm}
\BKM@DefineDriverKey{dvipdfmx}
\BKM@DefineDriverKey{xetex}
\BKM@DefineDriverKey{vtex}
\define@key{BKM}{dvipdfmx-outline-open}[true]{%
 \PackageWarning{bookmark}{Option 'dvipdfmx-outline-open' is obsolete
   and ignored}{}}
%    \end{macrocode}
%    \begin{macro}{\bookmark@driver}
%    \begin{macrocode}
\def\bookmark@driver{\BKM@driver}
%    \end{macrocode}
%    \end{macro}
%    \begin{macrocode}
\InputIfFileExists{bookmark.cfg}{}{}
%    \end{macrocode}
%    \begin{macro}{\BookmarkDriverDefault}
%    \begin{macrocode}
\providecommand*{\BookmarkDriverDefault}{dvips}
%    \end{macrocode}
%    \end{macro}
%    \begin{macro}{\BKM@driver}
% Lua\TeX\ 和 pdf\TeX\ 共享驱动程序。
%    \begin{macrocode}
\ifpdf
  \def\BKM@driver{pdftex}%
  \ifx\pdfoutline\@undefined
    \ifx\pdfextension\@undefined\else
      \protected\def\pdfoutline{\pdfextension outline }
    \fi
  \fi
\else
  \ifxetex
    \def\BKM@driver{dvipdfm}%
  \else
    \ifvtex
      \def\BKM@driver{vtex}%
    \else
      \edef\BKM@driver{\BookmarkDriverDefault}%
    \fi
  \fi
\fi
%    \end{macrocode}
%    \end{macro}
%
% \paragraph{过程选项(Process options)。}
%
%    \begin{macrocode}
\ProcessKeyvalOptions*
\BKM@DisableOptions
%    \end{macrocode}
%
% \paragraph{\xoption{draft}\ 选项}
%
%    \begin{macrocode}
\ifBKM@draft
  \PackageWarningNoLine{bookmark}{Draft mode on}%
  \let\bookmarksetup\ltx@gobble
  \let\BookmarkAtEnd\ltx@gobble
  \let\bookmarkdefinestyle\ltx@gobbletwo
  \let\bookmarkget\ltx@gobble
  \let\pdfbookmark\ltx@undefined
  \newcommand*{\pdfbookmark}[3][]{}%
  \let\currentpdfbookmark\ltx@gobbletwo
  \let\subpdfbookmark\ltx@gobbletwo
  \let\belowpdfbookmark\ltx@gobbletwo
  \newcommand*{\bookmark}[2][]{}%
  \renewcommand*{\Hy@writebookmark}[5]{}%
  \let\ReadBookmarks\relax
  \let\BKM@DefGotoNameAction\ltx@gobbletwo % package `hypdestopt'
  \expandafter\endinput
\fi
%    \end{macrocode}
%
% \paragraph{验证和加载驱动程序。}
%
%    \begin{macrocode}
\def\BKM@temp{dvipdfmx}%
\ifx\BKM@temp\BKM@driver
  \def\BKM@driver{dvipdfm}%
\fi
\def\BKM@temp{pdftex}%
\ifpdf
  \ifx\BKM@temp\BKM@driver
  \else
    \PackageWarningNoLine{bookmark}{%
      Wrong driver `\BKM@driver', using `pdftex' instead%
    }%
    \let\BKM@driver\BKM@temp
  \fi
\else
  \ifx\BKM@temp\BKM@driver
    \PackageError{bookmark}{%
      Wrong driver, pdfTeX is not running in PDF mode.\MessageBreak
      Package loading is aborted%
    }\@ehc
    \expandafter\expandafter\expandafter\endinput
  \fi
  \def\BKM@temp{dvipdfm}%
  \ifxetex
    \ifx\BKM@temp\BKM@driver
    \else
      \PackageWarningNoLine{bookmark}{%
        Wrong driver `\BKM@driver',\MessageBreak
        using `dvipdfm' for XeTeX instead%
      }%
      \let\BKM@driver\BKM@temp
    \fi
  \else
    \def\BKM@temp{vtex}%
    \ifvtex
      \ifx\BKM@temp\BKM@driver
      \else
        \PackageWarningNoLine{bookmark}{%
          Wrong driver `\BKM@driver',\MessageBreak
          using `vtex' for VTeX instead%
        }%
        \let\BKM@driver\BKM@temp
      \fi
    \else
      \ifx\BKM@temp\BKM@driver
        \PackageError{bookmark}{%
          Wrong driver, VTeX is not running in PDF mode.\MessageBreak
          Package loading is aborted%
        }\@ehc
        \expandafter\expandafter\expandafter\endinput
      \fi
    \fi
  \fi
\fi
\providecommand\IfFormatAtLeastTF{\@ifl@t@r\fmtversion}
\IfFormatAtLeastTF{2020/10/01}{}{\edef\BKM@driver{\BKM@driver-2019-12-03}}
\InputIfFileExists{bkm-\BKM@driver.def}{}{%
  \PackageError{bookmark}{%
    Unsupported driver `\BKM@driver'.\MessageBreak
    Package loading is aborted%
  }\@ehc
  \endinput
}
%    \end{macrocode}
%
% \subsubsection{与 \xpackage{hyperref}\ 的兼容性}
%
%    \begin{macro}{\pdfbookmark}
%    \begin{macrocode}
\let\pdfbookmark\ltx@undefined
\newcommand*{\pdfbookmark}[3][0]{%
  \bookmark[level=#1,dest={#3.#1}]{#2}%
  \hyper@anchorstart{#3.#1}\hyper@anchorend
}
%    \end{macrocode}
%    \end{macro}
%    \begin{macro}{\currentpdfbookmark}
%    \begin{macrocode}
\def\currentpdfbookmark{%
  \pdfbookmark[\BKM@currentlevel]%
}
%    \end{macrocode}
%    \end{macro}
%    \begin{macro}{\subpdfbookmark}
%    \begin{macrocode}
\def\subpdfbookmark{%
  \BKM@CalcExpr\BKM@CalcResult\BKM@currentlevel+1%
  \expandafter\pdfbookmark\expandafter[\BKM@CalcResult]%
}
%    \end{macrocode}
%    \end{macro}
%    \begin{macro}{\belowpdfbookmark}
%    \begin{macrocode}
\def\belowpdfbookmark#1#2{%
  \xdef\BKM@gtemp{\number\BKM@currentlevel}%
  \subpdfbookmark{#1}{#2}%
  \global\let\BKM@currentlevel\BKM@gtemp
}
%    \end{macrocode}
%    \end{macro}
%
%    节号(section number)、文本(text)、标签(label)、级别(level)、文件(file)
%    \begin{macro}{\Hy@writebookmark}
%    \begin{macrocode}
\def\Hy@writebookmark#1#2#3#4#5{%
  \ifnum#4>\BKM@depth\relax
  \else
    \def\BKM@type{#5}%
    \ifx\BKM@type\Hy@bookmarkstype
      \begingroup
        \ifBKM@numbered
          \let\numberline\Hy@numberline
          \let\booknumberline\Hy@numberline
          \let\partnumberline\Hy@numberline
          \let\chapternumberline\Hy@numberline
        \else
          \let\numberline\@gobble
          \let\booknumberline\@gobble
          \let\partnumberline\@gobble
          \let\chapternumberline\@gobble
        \fi
        \bookmark[level=#4,dest={\HyperDestNameFilter{#3}}]{#2}%
      \endgroup
    \fi
  \fi
}
%    \end{macrocode}
%    \end{macro}
%
%    \begin{macro}{\ReadBookmarks}
%    \begin{macrocode}
\let\ReadBookmarks\relax
%    \end{macrocode}
%    \end{macro}
%
%    \begin{macrocode}
%</package>
%    \end{macrocode}
%
% \subsection{dvipdfm 的驱动程序}
%
%    \begin{macrocode}
%<*dvipdfm>
\NeedsTeXFormat{LaTeX2e}
\ProvidesFile{bkm-dvipdfm.def}%
  [2020-11-06 v1.29 bookmark driver for dvipdfm (HO)]%
%    \end{macrocode}
%
%    \begin{macro}{\BKM@id}
%    \begin{macrocode}
\newcount\BKM@id
\BKM@id=\z@
%    \end{macrocode}
%    \end{macro}
%
%    \begin{macro}{\BKM@0}
%    \begin{macrocode}
\@namedef{BKM@0}{000}
%    \end{macrocode}
%    \end{macro}
%    \begin{macro}{\ifBKM@sw}
%    \begin{macrocode}
\newif\ifBKM@sw
%    \end{macrocode}
%    \end{macro}
%
%    \begin{macro}{\bookmark}
%    \begin{macrocode}
\newcommand*{\bookmark}[2][]{%
  \if@filesw
    \begingroup
      \def\bookmark@text{#2}%
      \BKM@setup{#1}%
      \edef\BKM@prev{\the\BKM@id}%
      \global\advance\BKM@id\@ne
      \BKM@swtrue
      \@whilesw\ifBKM@sw\fi{%
        \def\BKM@abslevel{1}%
        \ifnum\ifBKM@startatroot\z@\else\BKM@prev\fi=\z@
          \BKM@startatrootfalse
          \expandafter\xdef\csname BKM@\the\BKM@id\endcsname{%
            0{\BKM@level}\BKM@abslevel
          }%
          \BKM@swfalse
        \else
          \expandafter\expandafter\expandafter\BKM@getx
              \csname BKM@\BKM@prev\endcsname
          \ifnum\BKM@level>\BKM@x@level\relax
            \BKM@CalcExpr\BKM@abslevel\BKM@x@abslevel+1%
            \expandafter\xdef\csname BKM@\the\BKM@id\endcsname{%
              {\BKM@prev}{\BKM@level}\BKM@abslevel
            }%
            \BKM@swfalse
          \else
            \let\BKM@prev\BKM@x@parent
          \fi
        \fi
      }%
      \csname HyPsd@XeTeXBigCharstrue\endcsname
      \pdfstringdef\BKM@title{\bookmark@text}%
      \edef\BKM@FLAGS{\BKM@PrintStyle}%
      \let\BKM@action\@empty
      \ifx\BKM@gotor\@empty
        \ifx\BKM@dest\@empty
          \ifx\BKM@named\@empty
            \ifx\BKM@rawaction\@empty
              \ifx\BKM@uri\@empty
                \ifx\BKM@page\@empty
                  \PackageError{bookmark}{Missing action}\@ehc
                  \edef\BKM@action{/Dest[@page1/Fit]}%
                \else
                  \ifx\BKM@view\@empty
                    \def\BKM@view{Fit}%
                  \fi
                  \edef\BKM@action{/Dest[@page\BKM@page/\BKM@view]}%
                \fi
              \else
                \BKM@EscapeString\BKM@uri
                \edef\BKM@action{%
                  /A<<%
                    /S/URI%
                    /URI(\BKM@uri)%
                  >>%
                }%
              \fi
            \else
              \edef\BKM@action{/A<<\BKM@rawaction>>}%
            \fi
          \else
            \BKM@EscapeName\BKM@named
            \edef\BKM@action{%
              /A<</S/Named/N/\BKM@named>>%
            }%
          \fi
        \else
          \BKM@EscapeString\BKM@dest
          \edef\BKM@action{%
            /A<<%
              /S/GoTo%
              /D(\BKM@dest)%
            >>%
          }%
        \fi
      \else
        \ifx\BKM@dest\@empty
          \ifx\BKM@page\@empty
            \def\BKM@page{0}%
          \else
            \BKM@CalcExpr\BKM@page\BKM@page-1%
          \fi
          \ifx\BKM@view\@empty
            \def\BKM@view{Fit}%
          \fi
          \edef\BKM@action{/D[\BKM@page/\BKM@view]}%
        \else
          \BKM@EscapeString\BKM@dest
          \edef\BKM@action{/D(\BKM@dest)}%
        \fi
        \BKM@EscapeString\BKM@gotor
        \edef\BKM@action{%
          /A<<%
            /S/GoToR%
            /F(\BKM@gotor)%
            \BKM@action
          >>%
        }%
      \fi
      \special{pdf:%
        out
              [%
              \ifBKM@open
                \ifnum\BKM@level<%
                    \expandafter\ltx@firstofone\expandafter
                    {\number\BKM@openlevel} %
                \else
                  -%
                \fi
              \else
                -%
              \fi
              ] %
            \BKM@abslevel
        <<%
          /Title(\BKM@title)%
          \ifx\BKM@color\@empty
          \else
            /C[\BKM@color]%
          \fi
          \ifnum\BKM@FLAGS>\z@
            /F \BKM@FLAGS
          \fi
          \BKM@action
        >>%
      }%
    \endgroup
  \fi
}
%    \end{macrocode}
%    \end{macro}
%    \begin{macro}{\BKM@getx}
%    \begin{macrocode}
\def\BKM@getx#1#2#3{%
  \def\BKM@x@parent{#1}%
  \def\BKM@x@level{#2}%
  \def\BKM@x@abslevel{#3}%
}
%    \end{macrocode}
%    \end{macro}
%
%    \begin{macrocode}
%</dvipdfm>
%    \end{macrocode}
%
% \subsection{\hologo{VTeX}\ 的驱动程序}
%
%    \begin{macrocode}
%<*vtex>
\NeedsTeXFormat{LaTeX2e}
\ProvidesFile{bkm-vtex.def}%
  [2020-11-06 v1.29 bookmark driver for VTeX (HO)]%
%    \end{macrocode}
%
%    \begin{macrocode}
\ifvtexpdf
\else
  \PackageWarningNoLine{bookmark}{%
    The VTeX driver only supports PDF mode%
  }%
\fi
%    \end{macrocode}
%
%    \begin{macro}{\BKM@id}
%    \begin{macrocode}
\newcount\BKM@id
\BKM@id=\z@
%    \end{macrocode}
%    \end{macro}
%
%    \begin{macro}{\BKM@0}
%    \begin{macrocode}
\@namedef{BKM@0}{00}
%    \end{macrocode}
%    \end{macro}
%    \begin{macro}{\ifBKM@sw}
%    \begin{macrocode}
\newif\ifBKM@sw
%    \end{macrocode}
%    \end{macro}
%
%    \begin{macro}{\bookmark}
%    \begin{macrocode}
\newcommand*{\bookmark}[2][]{%
  \if@filesw
    \begingroup
      \def\bookmark@text{#2}%
      \BKM@setup{#1}%
      \edef\BKM@prev{\the\BKM@id}%
      \global\advance\BKM@id\@ne
      \BKM@swtrue
      \@whilesw\ifBKM@sw\fi{%
        \ifnum\ifBKM@startatroot\z@\else\BKM@prev\fi=\z@
          \BKM@startatrootfalse
          \def\BKM@parent{0}%
          \expandafter\xdef\csname BKM@\the\BKM@id\endcsname{%
            0{\BKM@level}%
          }%
          \BKM@swfalse
        \else
          \expandafter\expandafter\expandafter\BKM@getx
              \csname BKM@\BKM@prev\endcsname
          \ifnum\BKM@level>\BKM@x@level\relax
            \let\BKM@parent\BKM@prev
            \expandafter\xdef\csname BKM@\the\BKM@id\endcsname{%
              {\BKM@prev}{\BKM@level}%
            }%
            \BKM@swfalse
          \else
            \let\BKM@prev\BKM@x@parent
          \fi
        \fi
      }%
      \pdfstringdef\BKM@title{\bookmark@text}%
      \BKM@vtex@title
      \edef\BKM@FLAGS{\BKM@PrintStyle}%
      \let\BKM@action\@empty
      \ifx\BKM@gotor\@empty
        \ifx\BKM@dest\@empty
          \ifx\BKM@named\@empty
            \ifx\BKM@rawaction\@empty
              \ifx\BKM@uri\@empty
                \ifx\BKM@page\@empty
                  \PackageError{bookmark}{Missing action}\@ehc
                  \def\BKM@action{!1}%
                \else
                  \edef\BKM@action{!\BKM@page}%
                \fi
              \else
                \BKM@EscapeString\BKM@uri
                \edef\BKM@action{%
                  <u=%
                    /S/URI%
                    /URI(\BKM@uri)%
                  >%
                }%
              \fi
            \else
              \edef\BKM@action{<u=\BKM@rawaction>}%
            \fi
          \else
            \BKM@EscapeName\BKM@named
            \edef\BKM@action{%
              <u=%
                /S/Named%
                /N/\BKM@named
              >%
            }%
          \fi
        \else
          \BKM@EscapeString\BKM@dest
          \edef\BKM@action{\BKM@dest}%
        \fi
      \else
        \ifx\BKM@dest\@empty
          \ifx\BKM@page\@empty
            \def\BKM@page{1}%
          \fi
          \ifx\BKM@view\@empty
            \def\BKM@view{Fit}%
          \fi
          \edef\BKM@action{/D[\BKM@page/\BKM@view]}%
        \else
          \BKM@EscapeString\BKM@dest
          \edef\BKM@action{/D(\BKM@dest)}%
        \fi
        \BKM@EscapeString\BKM@gotor
        \edef\BKM@action{%
          <u=%
            /S/GoToR%
            /F(\BKM@gotor)%
            \BKM@action
          >>%
        }%
      \fi
      \ifx\BKM@color\@empty
        \let\BKM@RGBcolor\@empty
      \else
        \expandafter\BKM@toRGB\BKM@color\@nil
      \fi
      \special{%
        !outline \BKM@action;%
        p=\BKM@parent,%
        i=\number\BKM@id,%
        s=%
          \ifBKM@open
            \ifnum\BKM@level<\BKM@openlevel
              o%
            \else
              c%
            \fi
          \else
            c%
          \fi,%
        \ifx\BKM@RGBcolor\@empty
        \else
          c=\BKM@RGBcolor,%
        \fi
        \ifnum\BKM@FLAGS>\z@
          f=\BKM@FLAGS,%
        \fi
        t=\BKM@title
      }%
    \endgroup
  \fi
}
%    \end{macrocode}
%    \end{macro}
%    \begin{macro}{\BKM@getx}
%    \begin{macrocode}
\def\BKM@getx#1#2{%
  \def\BKM@x@parent{#1}%
  \def\BKM@x@level{#2}%
}
%    \end{macrocode}
%    \end{macro}
%    \begin{macro}{\BKM@toRGB}
%    \begin{macrocode}
\def\BKM@toRGB#1 #2 #3\@nil{%
  \let\BKM@RGBcolor\@empty
  \BKM@toRGBComponent{#1}%
  \BKM@toRGBComponent{#2}%
  \BKM@toRGBComponent{#3}%
}
%    \end{macrocode}
%    \end{macro}
%    \begin{macro}{\BKM@toRGBComponent}
%    \begin{macrocode}
\def\BKM@toRGBComponent#1{%
  \dimen@=#1pt\relax
  \ifdim\dimen@>\z@
    \ifdim\dimen@<\p@
      \dimen@=255\dimen@
      \advance\dimen@ by 32768sp\relax
      \divide\dimen@ by 65536\relax
      \dimen@ii=\dimen@
      \divide\dimen@ii by 16\relax
      \edef\BKM@RGBcolor{%
        \BKM@RGBcolor
        \BKM@toHexDigit\dimen@ii
      }%
      \dimen@ii=16\dimen@ii
      \advance\dimen@-\dimen@ii
      \edef\BKM@RGBcolor{%
        \BKM@RGBcolor
        \BKM@toHexDigit\dimen@
      }%
    \else
      \edef\BKM@RGBcolor{\BKM@RGBcolor FF}%
    \fi
  \else
    \edef\BKM@RGBcolor{\BKM@RGBcolor00}%
  \fi
}
%    \end{macrocode}
%    \end{macro}
%    \begin{macro}{\BKM@toHexDigit}
%    \begin{macrocode}
\def\BKM@toHexDigit#1{%
  \ifcase\expandafter\@firstofone\expandafter{\number#1} %
    0\or 1\or 2\or 3\or 4\or 5\or 6\or 7\or
    8\or 9\or A\or B\or C\or D\or E\or F%
  \fi
}
%    \end{macrocode}
%    \end{macro}
%    \begin{macrocode}
\begingroup
  \catcode`\|=0 %
  \catcode`\\=12 %
%    \end{macrocode}
%    \begin{macro}{\BKM@vtex@title}
%    \begin{macrocode}
  |gdef|BKM@vtex@title{%
    |@onelevel@sanitize|BKM@title
    |edef|BKM@title{|expandafter|BKM@vtex@leftparen|BKM@title\(|@nil}%
    |edef|BKM@title{|expandafter|BKM@vtex@rightparen|BKM@title\)|@nil}%
    |edef|BKM@title{|expandafter|BKM@vtex@zero|BKM@title\0|@nil}%
    |edef|BKM@title{|expandafter|BKM@vtex@one|BKM@title\1|@nil}%
    |edef|BKM@title{|expandafter|BKM@vtex@two|BKM@title\2|@nil}%
    |edef|BKM@title{|expandafter|BKM@vtex@three|BKM@title\3|@nil}%
  }%
%    \end{macrocode}
%    \end{macro}
%    \begin{macro}{\BKM@vtex@leftparen}
%    \begin{macrocode}
  |gdef|BKM@vtex@leftparen#1\(#2|@nil{%
    #1%
    |ifx||#2||%
    |else
      (%
      |ltx@ReturnAfterFi{%
        |BKM@vtex@leftparen#2|@nil
      }%
    |fi
  }%
%    \end{macrocode}
%    \end{macro}
%    \begin{macro}{\BKM@vtex@rightparen}
%    \begin{macrocode}
  |gdef|BKM@vtex@rightparen#1\)#2|@nil{%
    #1%
    |ifx||#2||%
    |else
      )%
      |ltx@ReturnAfterFi{%
        |BKM@vtex@rightparen#2|@nil
      }%
    |fi
  }%
%    \end{macrocode}
%    \end{macro}
%    \begin{macro}{\BKM@vtex@zero}
%    \begin{macrocode}
  |gdef|BKM@vtex@zero#1\0#2|@nil{%
    #1%
    |ifx||#2||%
    |else
      |noexpand|hv@pdf@char0%
      |ltx@ReturnAfterFi{%
        |BKM@vtex@zero#2|@nil
      }%
    |fi
  }%
%    \end{macrocode}
%    \end{macro}
%    \begin{macro}{\BKM@vtex@one}
%    \begin{macrocode}
  |gdef|BKM@vtex@one#1\1#2|@nil{%
    #1%
    |ifx||#2||%
    |else
      |noexpand|hv@pdf@char1%
      |ltx@ReturnAfterFi{%
        |BKM@vtex@one#2|@nil
      }%
    |fi
  }%
%    \end{macrocode}
%    \end{macro}
%    \begin{macro}{\BKM@vtex@two}
%    \begin{macrocode}
  |gdef|BKM@vtex@two#1\2#2|@nil{%
    #1%
    |ifx||#2||%
    |else
      |noexpand|hv@pdf@char2%
      |ltx@ReturnAfterFi{%
        |BKM@vtex@two#2|@nil
      }%
    |fi
  }%
%    \end{macrocode}
%    \end{macro}
%    \begin{macro}{\BKM@vtex@three}
%    \begin{macrocode}
  |gdef|BKM@vtex@three#1\3#2|@nil{%
    #1%
    |ifx||#2||%
    |else
      |noexpand|hv@pdf@char3%
      |ltx@ReturnAfterFi{%
        |BKM@vtex@three#2|@nil
      }%
    |fi
  }%
%    \end{macrocode}
%    \end{macro}
%    \begin{macrocode}
|endgroup
%    \end{macrocode}
%
%    \begin{macrocode}
%</vtex>
%    \end{macrocode}
%
% \subsection{\hologo{pdfTeX}\ 的驱动程序}
%
%    \begin{macrocode}
%<*pdftex>
\NeedsTeXFormat{LaTeX2e}
\ProvidesFile{bkm-pdftex.def}%
  [2020-11-06 v1.29 bookmark driver for pdfTeX (HO)]%
%    \end{macrocode}
%
%    \begin{macro}{\BKM@DO@entry}
%    \begin{macrocode}
\def\BKM@DO@entry#1#2{%
  \begingroup
    \kvsetkeys{BKM@DO}{#1}%
    \def\BKM@DO@title{#2}%
    \ifx\BKM@DO@srcfile\@empty
    \else
      \BKM@UnescapeHex\BKM@DO@srcfile
    \fi
    \BKM@UnescapeHex\BKM@DO@title
    \expandafter\expandafter\expandafter\BKM@getx
        \csname BKM@\BKM@DO@id\endcsname\@empty\@empty
    \let\BKM@attr\@empty
    \ifx\BKM@DO@flags\@empty
    \else
      \edef\BKM@attr{\BKM@attr/F \BKM@DO@flags}%
    \fi
    \ifx\BKM@DO@color\@empty
    \else
      \edef\BKM@attr{\BKM@attr/C[\BKM@DO@color]}%
    \fi
    \ifx\BKM@attr\@empty
    \else
      \edef\BKM@attr{attr{\BKM@attr}}%
    \fi
    \let\BKM@action\@empty
    \ifx\BKM@DO@gotor\@empty
      \ifx\BKM@DO@dest\@empty
        \ifx\BKM@DO@named\@empty
          \ifx\BKM@DO@rawaction\@empty
            \ifx\BKM@DO@uri\@empty
              \ifx\BKM@DO@page\@empty
                \PackageError{bookmark}{%
                  Missing action\BKM@SourceLocation
                }\@ehc
                \edef\BKM@action{goto page1{/Fit}}%
              \else
                \ifx\BKM@DO@view\@empty
                  \def\BKM@DO@view{Fit}%
                \fi
                \edef\BKM@action{goto page\BKM@DO@page{/\BKM@DO@view}}%
              \fi
            \else
              \BKM@UnescapeHex\BKM@DO@uri
              \BKM@EscapeString\BKM@DO@uri
              \edef\BKM@action{user{<</S/URI/URI(\BKM@DO@uri)>>}}%
            \fi
          \else
            \BKM@UnescapeHex\BKM@DO@rawaction
            \edef\BKM@action{%
              user{%
                <<%
                  \BKM@DO@rawaction
                >>%
              }%
            }%
          \fi
        \else
          \BKM@EscapeName\BKM@DO@named
          \edef\BKM@action{%
            user{<</S/Named/N/\BKM@DO@named>>}%
          }%
        \fi
      \else
        \BKM@UnescapeHex\BKM@DO@dest
        \BKM@DefGotoNameAction\BKM@action\BKM@DO@dest
      \fi
    \else
      \ifx\BKM@DO@dest\@empty
        \ifx\BKM@DO@page\@empty
          \def\BKM@DO@page{0}%
        \else
          \BKM@CalcExpr\BKM@DO@page\BKM@DO@page-1%
        \fi
        \ifx\BKM@DO@view\@empty
          \def\BKM@DO@view{Fit}%
        \fi
        \edef\BKM@action{/D[\BKM@DO@page/\BKM@DO@view]}%
      \else
        \BKM@UnescapeHex\BKM@DO@dest
        \BKM@EscapeString\BKM@DO@dest
        \edef\BKM@action{/D(\BKM@DO@dest)}%
      \fi
      \BKM@UnescapeHex\BKM@DO@gotor
      \BKM@EscapeString\BKM@DO@gotor
      \edef\BKM@action{%
        user{%
          <<%
            /S/GoToR%
            /F(\BKM@DO@gotor)%
            \BKM@action
          >>%
        }%
      }%
    \fi
    \pdfoutline\BKM@attr\BKM@action
                count\ifBKM@DO@open\else-\fi\BKM@x@childs
                {\BKM@DO@title}%
  \endgroup
}
%    \end{macrocode}
%    \end{macro}
%    \begin{macro}{\BKM@DefGotoNameAction}
%    \cs{BKM@DefGotoNameAction}\ 宏是一个用于 \xpackage{hypdestopt}\ 宏包的钩子(hook)。
%    \begin{macrocode}
\def\BKM@DefGotoNameAction#1#2{%
  \BKM@EscapeString\BKM@DO@dest
  \edef#1{goto name{#2}}%
}
%    \end{macrocode}
%    \end{macro}
%    \begin{macrocode}
%</pdftex>
%    \end{macrocode}
%
%    \begin{macrocode}
%<*pdftex|pdfmark>
%    \end{macrocode}
%    \begin{macro}{\BKM@SourceLocation}
%    \begin{macrocode}
\def\BKM@SourceLocation{%
  \ifx\BKM@DO@srcfile\@empty
    \ifx\BKM@DO@srcline\@empty
    \else
      .\MessageBreak
      Source: line \BKM@DO@srcline
    \fi
  \else
    \ifx\BKM@DO@srcline\@empty
      .\MessageBreak
      Source: file `\BKM@DO@srcfile'%
    \else
      .\MessageBreak
      Source: file `\BKM@DO@srcfile', line \BKM@DO@srcline
    \fi
  \fi
}
%    \end{macrocode}
%    \end{macro}
%    \begin{macrocode}
%</pdftex|pdfmark>
%    \end{macrocode}
%
% \subsection{具有 pdfmark 特色(specials)的驱动程序}
%
% \subsubsection{dvips 驱动程序}
%
%    \begin{macrocode}
%<*dvips>
\NeedsTeXFormat{LaTeX2e}
\ProvidesFile{bkm-dvips.def}%
  [2020-11-06 v1.29 bookmark driver for dvips (HO)]%
%    \end{macrocode}
%    \begin{macro}{\BKM@PSHeaderFile}
%    \begin{macrocode}
\def\BKM@PSHeaderFile#1{%
  \special{PSfile=#1}%
}
%    \end{macrocode}
%    \begin{macro}{\BKM@filename}
%    \begin{macrocode}
\def\BKM@filename{\jobname.out.ps}
%    \end{macrocode}
%    \end{macro}
%    \begin{macrocode}
\AddToHook{shipout/lastpage}{%
  \BKM@pdfmark@out
  \BKM@PSHeaderFile\BKM@filename
  }
%    \end{macrocode}
%    \end{macro}
%    \begin{macrocode}
%</dvips>
%    \end{macrocode}
%
% \subsubsection{公共部分(Common part)}
%
%    \begin{macrocode}
%<*pdfmark>
%    \end{macrocode}
%
%    \begin{macro}{\BKM@pdfmark@out}
%    不要在这里使用 \xpackage{rerunfilecheck}\ 宏包,因为在 \hologo{TeX}\ 运行期间不会
%    读取 \cs{BKM@filename}\ 文件。
%    \begin{macrocode}
\def\BKM@pdfmark@out{%
  \if@filesw
    \newwrite\BKM@file
    \immediate\openout\BKM@file=\BKM@filename\relax
    \BKM@write{\@percentchar!}%
    \BKM@write{/pdfmark where{pop}}%
    \BKM@write{%
      {%
        /globaldict where{pop globaldict}{userdict}ifelse%
        /pdfmark/cleartomark load put%
      }%
    }%
    \BKM@write{ifelse}%
  \else
    \let\BKM@write\@gobble
    \let\BKM@DO@entry\@gobbletwo
  \fi
}
%    \end{macrocode}
%    \end{macro}
%    \begin{macro}{\BKM@write}
%    \begin{macrocode}
\def\BKM@write#{%
  \immediate\write\BKM@file
}
%    \end{macrocode}
%    \end{macro}
%
%    \begin{macro}{\BKM@DO@entry}
%    Pdfmark 的规范(specification)说明 |/Color| 是颜色(color)的键名(key name),
%    但是 ghostscript 只将键(key)传递到 PDF 文件中,因此键名必须是 |/C|。
%    \begin{macrocode}
\def\BKM@DO@entry#1#2{%
  \begingroup
    \kvsetkeys{BKM@DO}{#1}%
    \ifx\BKM@DO@srcfile\@empty
    \else
      \BKM@UnescapeHex\BKM@DO@srcfile
    \fi
    \def\BKM@DO@title{#2}%
    \BKM@UnescapeHex\BKM@DO@title
    \expandafter\expandafter\expandafter\BKM@getx
        \csname BKM@\BKM@DO@id\endcsname\@empty\@empty
    \let\BKM@attr\@empty
    \ifx\BKM@DO@flags\@empty
    \else
      \edef\BKM@attr{\BKM@attr/F \BKM@DO@flags}%
    \fi
    \ifx\BKM@DO@color\@empty
    \else
      \edef\BKM@attr{\BKM@attr/C[\BKM@DO@color]}%
    \fi
    \let\BKM@action\@empty
    \ifx\BKM@DO@gotor\@empty
      \ifx\BKM@DO@dest\@empty
        \ifx\BKM@DO@named\@empty
          \ifx\BKM@DO@rawaction\@empty
            \ifx\BKM@DO@uri\@empty
              \ifx\BKM@DO@page\@empty
                \PackageError{bookmark}{%
                  Missing action\BKM@SourceLocation
                }\@ehc
                \edef\BKM@action{%
                  /Action/GoTo%
                  /Page 1%
                  /View[/Fit]%
                }%
              \else
                \ifx\BKM@DO@view\@empty
                  \def\BKM@DO@view{Fit}%
                \fi
                \edef\BKM@action{%
                  /Action/GoTo%
                  /Page \BKM@DO@page
                  /View[/\BKM@DO@view]%
                }%
              \fi
            \else
              \BKM@UnescapeHex\BKM@DO@uri
              \BKM@EscapeString\BKM@DO@uri
              \edef\BKM@action{%
                /Action<<%
                  /Subtype/URI%
                  /URI(\BKM@DO@uri)%
                >>%
              }%
            \fi
          \else
            \BKM@UnescapeHex\BKM@DO@rawaction
            \edef\BKM@action{%
              /Action<<%
                \BKM@DO@rawaction
              >>%
            }%
          \fi
        \else
          \BKM@EscapeName\BKM@DO@named
          \edef\BKM@action{%
            /Action<<%
              /Subtype/Named%
              /N/\BKM@DO@named
            >>%
          }%
        \fi
      \else
        \BKM@UnescapeHex\BKM@DO@dest
        \BKM@EscapeString\BKM@DO@dest
        \edef\BKM@action{%
          /Action/GoTo%
          /Dest(\BKM@DO@dest)cvn%
        }%
      \fi
    \else
      \ifx\BKM@DO@dest\@empty
        \ifx\BKM@DO@page\@empty
          \def\BKM@DO@page{1}%
        \fi
        \ifx\BKM@DO@view\@empty
          \def\BKM@DO@view{Fit}%
        \fi
        \edef\BKM@action{%
          /Page \BKM@DO@page
          /View[/\BKM@DO@view]%
        }%
      \else
        \BKM@UnescapeHex\BKM@DO@dest
        \BKM@EscapeString\BKM@DO@dest
        \edef\BKM@action{%
          /Dest(\BKM@DO@dest)cvn%
        }%
      \fi
      \BKM@UnescapeHex\BKM@DO@gotor
      \BKM@EscapeString\BKM@DO@gotor
      \edef\BKM@action{%
        /Action/GoToR%
        /File(\BKM@DO@gotor)%
        \BKM@action
      }%
    \fi
    \BKM@write{[}%
    \BKM@write{/Title(\BKM@DO@title)}%
    \ifnum\BKM@x@childs>\z@
      \BKM@write{/Count \ifBKM@DO@open\else-\fi\BKM@x@childs}%
    \fi
    \ifx\BKM@attr\@empty
    \else
      \BKM@write{\BKM@attr}%
    \fi
    \BKM@write{\BKM@action}%
    \BKM@write{/OUT pdfmark}%
  \endgroup
}
%    \end{macrocode}
%    \end{macro}
%    \begin{macrocode}
%</pdfmark>
%    \end{macrocode}
%
% \subsection{\xoption{pdftex}\ 和 \xoption{pdfmark}\ 的公共部分}
%
%    \begin{macrocode}
%<*pdftex|pdfmark>
%    \end{macrocode}
%
% \subsubsection{写入辅助文件(auxiliary file)}
%
%    \begin{macrocode}
\AddToHook{begindocument}{%
 \immediate\write\@mainaux{\string\providecommand\string\BKM@entry[2]{}}}
%    \end{macrocode}
%
%    \begin{macro}{\BKM@id}
%    \begin{macrocode}
\newcount\BKM@id
\BKM@id=\z@
%    \end{macrocode}
%    \end{macro}
%
%    \begin{macro}{\BKM@0}
%    \begin{macrocode}
\@namedef{BKM@0}{000}
%    \end{macrocode}
%    \end{macro}
%    \begin{macro}{\ifBKM@sw}
%    \begin{macrocode}
\newif\ifBKM@sw
%    \end{macrocode}
%    \end{macro}
%
%    \begin{macro}{\bookmark}
%    \begin{macrocode}
\newcommand*{\bookmark}[2][]{%
  \if@filesw
    \begingroup
      \BKM@InitSourceLocation
      \def\bookmark@text{#2}%
      \BKM@setup{#1}%
      \ifx\BKM@srcfile\@empty
      \else
        \BKM@EscapeHex\BKM@srcfile
      \fi
      \edef\BKM@prev{\the\BKM@id}%
      \global\advance\BKM@id\@ne
      \BKM@swtrue
      \@whilesw\ifBKM@sw\fi{%
        \ifnum\ifBKM@startatroot\z@\else\BKM@prev\fi=\z@
          \BKM@startatrootfalse
          \expandafter\xdef\csname BKM@\the\BKM@id\endcsname{%
            0{\BKM@level}0%
          }%
          \BKM@swfalse
        \else
          \expandafter\expandafter\expandafter\BKM@getx
              \csname BKM@\BKM@prev\endcsname
          \ifnum\BKM@level>\BKM@x@level\relax
            \expandafter\xdef\csname BKM@\the\BKM@id\endcsname{%
              {\BKM@prev}{\BKM@level}0%
            }%
            \ifnum\BKM@prev>\z@
              \BKM@CalcExpr\BKM@CalcResult\BKM@x@childs+1%
              \expandafter\xdef\csname BKM@\BKM@prev\endcsname{%
                {\BKM@x@parent}{\BKM@x@level}{\BKM@CalcResult}%
              }%
            \fi
            \BKM@swfalse
          \else
            \let\BKM@prev\BKM@x@parent
          \fi
        \fi
      }%
      \pdfstringdef\BKM@title{\bookmark@text}%
      \edef\BKM@FLAGS{\BKM@PrintStyle}%
      \csname BKM@HypDestOptHook\endcsname
      \BKM@EscapeHex\BKM@dest
      \BKM@EscapeHex\BKM@uri
      \BKM@EscapeHex\BKM@gotor
      \BKM@EscapeHex\BKM@rawaction
      \BKM@EscapeHex\BKM@title
      \immediate\write\@mainaux{%
        \string\BKM@entry{%
          id=\number\BKM@id
          \ifBKM@open
            \ifnum\BKM@level<\BKM@openlevel
              ,open%
            \fi
          \fi
          \BKM@auxentry{dest}%
          \BKM@auxentry{named}%
          \BKM@auxentry{uri}%
          \BKM@auxentry{gotor}%
          \BKM@auxentry{page}%
          \BKM@auxentry{view}%
          \BKM@auxentry{rawaction}%
          \BKM@auxentry{color}%
          \ifnum\BKM@FLAGS>\z@
            ,flags=\BKM@FLAGS
          \fi
          \BKM@auxentry{srcline}%
          \BKM@auxentry{srcfile}%
        }{\BKM@title}%
      }%
    \endgroup
  \fi
}
%    \end{macrocode}
%    \end{macro}
%    \begin{macro}{\BKM@getx}
%    \begin{macrocode}
\def\BKM@getx#1#2#3{%
  \def\BKM@x@parent{#1}%
  \def\BKM@x@level{#2}%
  \def\BKM@x@childs{#3}%
}
%    \end{macrocode}
%    \end{macro}
%    \begin{macro}{\BKM@auxentry}
%    \begin{macrocode}
\def\BKM@auxentry#1{%
  \expandafter\ifx\csname BKM@#1\endcsname\@empty
  \else
    ,#1={\csname BKM@#1\endcsname}%
  \fi
}
%    \end{macrocode}
%    \end{macro}
%
%    \begin{macro}{\BKM@InitSourceLocation}
%    \begin{macrocode}
\def\BKM@InitSourceLocation{%
  \edef\BKM@srcline{\the\inputlineno}%
  \BKM@LuaTeX@InitFile
  \ifx\BKM@srcfile\@empty
    \ltx@IfUndefined{currfilepath}{}{%
      \edef\BKM@srcfile{\currfilepath}%
    }%
  \fi
}
%    \end{macrocode}
%    \end{macro}
%    \begin{macro}{\BKM@LuaTeX@InitFile}
%    \begin{macrocode}
\ifluatex
  \ifnum\luatexversion>36 %
    \def\BKM@LuaTeX@InitFile{%
      \begingroup
        \ltx@LocToksA={}%
      \edef\x{\endgroup
        \def\noexpand\BKM@srcfile{%
          \the\expandafter\ltx@LocToksA
          \directlua{%
             if status and status.filename then %
               tex.settoks('ltx@LocToksA', status.filename)%
             end%
          }%
        }%
      }\x
    }%
  \else
    \let\BKM@LuaTeX@InitFile\relax
  \fi
\else
  \let\BKM@LuaTeX@InitFile\relax
\fi
%    \end{macrocode}
%    \end{macro}
%
% \subsubsection{读取辅助数据(auxiliary data)}
%
%    \begin{macrocode}
\SetupKeyvalOptions{family=BKM@DO,prefix=BKM@DO@}
\DeclareStringOption[0]{id}
\DeclareBoolOption{open}
\DeclareStringOption{flags}
\DeclareStringOption{color}
\DeclareStringOption{dest}
\DeclareStringOption{named}
\DeclareStringOption{uri}
\DeclareStringOption{gotor}
\DeclareStringOption{page}
\DeclareStringOption{view}
\DeclareStringOption{rawaction}
\DeclareStringOption{srcline}
\DeclareStringOption{srcfile}
%    \end{macrocode}
%
%    \begin{macrocode}
\AtBeginDocument{%
  \let\BKM@entry\BKM@DO@entry
}
%    \end{macrocode}
%
%    \begin{macrocode}
%</pdftex|pdfmark>
%    \end{macrocode}
%
% \subsection{\xoption{atend}\ 选项}
%
% \subsubsection{钩子(Hook)}
%
%    \begin{macrocode}
%<*package>
%    \end{macrocode}
%    \begin{macrocode}
\ifBKM@atend
\else
%    \end{macrocode}
%    \begin{macro}{\BookmarkAtEnd}
%    这是一个虚拟定义(dummy definition),如果没有给出 \xoption{atend}\ 选项,它将生成一个警告。
%    \begin{macrocode}
  \newcommand{\BookmarkAtEnd}[1]{%
    \PackageWarning{bookmark}{%
      Ignored, because option `atend' is missing%
    }%
  }%
%    \end{macrocode}
%    \end{macro}
%    \begin{macrocode}
  \expandafter\endinput
\fi
%    \end{macrocode}
%    \begin{macro}{\BookmarkAtEnd}
%    \begin{macrocode}
\newcommand*{\BookmarkAtEnd}{%
  \g@addto@macro\BKM@EndHook
}
%    \end{macrocode}
%    \end{macro}
%    \begin{macrocode}
\let\BKM@EndHook\@empty
%    \end{macrocode}
%    \begin{macrocode}
%</package>
%    \end{macrocode}
%
% \subsubsection{在文档末尾使用钩子的驱动程序}
%
%    驱动程序 \xoption{pdftex}\ 使用 LaTeX 钩子 \xoption{enddocument/afterlastpage}
%    (相当于以前使用的 \xpackage{atveryend}\ 的 \cs{AfterLastShipout}),因为它仍然需要 \xext{aux}\ 文件。
%    它使用 \cs{pdfoutline}\ 作为最后一页之后可以使用的书签(bookmakrs)。
%    \begin{itemize}
%    \item
%      驱动程序 \xoption{pdftex}\ 使用 \cs{pdfoutline}, \cs{pdfoutline}\ 可以在最后一页之后使用。
%    \end{itemize}
%    \begin{macrocode}
%<*pdftex>
\ifBKM@atend
  \AddToHook{enddocument/afterlastpage}{%
    \BKM@EndHook
  }%
\fi
%</pdftex>
%    \end{macrocode}
%
% \subsubsection{使用 \xoption{shipout/lastpage}\ 的驱动程序}
%
%    其他驱动程序使用 \cs{special}\ 命令实现 \cs{bookmark}。因此,最后的书签(last bookmarks)
%    必须放在最后一页(last page),而不是之后。不能使用 \cs{AtEndDocument},因为为时已晚,
%    最后一页已经输出了。因此,我们使用 LaTeX 钩子 \xoption{shipout/lastpage}。至少需要运行
%    两次 \hologo{LaTeX}。PostScript 驱动程序 \xoption{dvips}\ 使用外部 PostScript 文件作为书签。
%    为了避免与 pgf 发生冲突,文件写入(file writing)也被移到了最后一个输出页面(shipout page)。
%    \begin{macrocode}
%<*dvipdfm|vtex|pdfmark>
\ifBKM@atend
  \AddToHook{shipout/lastpage}{\BKM@EndHook}%
\fi
%</dvipdfm|vtex|pdfmark>
%    \end{macrocode}
%
% \section{安装(Installation)}
%
% \subsection{下载(Download)}
%
% \paragraph{宏包(Package)。} 在 CTAN\footnote{\CTANpkg{bookmark}}上提供此宏包:
% \begin{description}
% \item[\CTAN{macros/latex/contrib/bookmark/bookmark.dtx}] 源文件(source file)。
% \item[\CTAN{macros/latex/contrib/bookmark/bookmark.pdf}] 文档(documentation)。
% \end{description}
%
%
% \paragraph{捆绑包(Bundle)。} “bookmark”捆绑包(bundle)的所有宏包(packages)都可以在兼
% 容 TDS 的 ZIP 归档文件中找到。在那里,宏包已经被解包,文档文件(documentation files)已经生成。
% 文件(files)和目录(directories)遵循 TDS 标准。
% \begin{description}
% \item[\CTANinstall{install/macros/latex/contrib/bookmark.tds.zip}]
% \end{description}
% \emph{TDS}\ 是指标准的“用于 \TeX\ 文件的目录结构(Directory Structure)”(\CTANpkg{tds})。
% 名称中带有 \xfile{texmf}\ 的目录(directories)通常以这种方式组织。
%
% \subsection{捆绑包(Bundle)的安装}
%
% \paragraph{解压(Unpacking)。} 在您选择的 TDS 树(也称为 \xfile{texmf}\ 树)中解
% 压 \xfile{bookmark.tds.zip},例如(在 linux 中):
% \begin{quote}
%   |unzip bookmark.tds.zip -d ~/texmf|
% \end{quote}
%
% \subsection{宏包(Package)的安装}
%
% \paragraph{解压(Unpacking)。} \xfile{.dtx}\ 文件是一个自解压 \docstrip\ 归档文件(archive)。
% 这些文件是通过 \plainTeX\ 运行 \xfile{.dtx}\ 来提取的:
% \begin{quote}
%   \verb|tex bookmark.dtx|
% \end{quote}
%
% \paragraph{TDS.} 现在,不同的文件必须移动到安装 TDS 树(installation TDS tree)
% (也称为 \xfile{texmf}\ 树)中的不同目录中:
% \begin{quote}
% \def\t{^^A
% \begin{tabular}{@{}>{\ttfamily}l@{ $\rightarrow$ }>{\ttfamily}l@{}}
%   bookmark.sty & tex/latex/bookmark/bookmark.sty\\
%   bkm-dvipdfm.def & tex/latex/bookmark/bkm-dvipdfm.def\\
%   bkm-dvips.def & tex/latex/bookmark/bkm-dvips.def\\
%   bkm-pdftex.def & tex/latex/bookmark/bkm-pdftex.def\\
%   bkm-vtex.def & tex/latex/bookmark/bkm-vtex.def\\
%   bookmark.pdf & doc/latex/bookmark/bookmark.pdf\\
%   bookmark-example.tex & doc/latex/bookmark/bookmark-example.tex\\
%   bookmark.dtx & source/latex/bookmark/bookmark.dtx\\
% \end{tabular}^^A
% }^^A
% \sbox0{\t}^^A
% \ifdim\wd0>\linewidth
%   \begingroup
%     \advance\linewidth by\leftmargin
%     \advance\linewidth by\rightmargin
%   \edef\x{\endgroup
%     \def\noexpand\lw{\the\linewidth}^^A
%   }\x
%   \def\lwbox{^^A
%     \leavevmode
%     \hbox to \linewidth{^^A
%       \kern-\leftmargin\relax
%       \hss
%       \usebox0
%       \hss
%       \kern-\rightmargin\relax
%     }^^A
%   }^^A
%   \ifdim\wd0>\lw
%     \sbox0{\small\t}^^A
%     \ifdim\wd0>\linewidth
%       \ifdim\wd0>\lw
%         \sbox0{\footnotesize\t}^^A
%         \ifdim\wd0>\linewidth
%           \ifdim\wd0>\lw
%             \sbox0{\scriptsize\t}^^A
%             \ifdim\wd0>\linewidth
%               \ifdim\wd0>\lw
%                 \sbox0{\tiny\t}^^A
%                 \ifdim\wd0>\linewidth
%                   \lwbox
%                 \else
%                   \usebox0
%                 \fi
%               \else
%                 \lwbox
%               \fi
%             \else
%               \usebox0
%             \fi
%           \else
%             \lwbox
%           \fi
%         \else
%           \usebox0
%         \fi
%       \else
%         \lwbox
%       \fi
%     \else
%       \usebox0
%     \fi
%   \else
%     \lwbox
%   \fi
% \else
%   \usebox0
% \fi
% \end{quote}
% 如果你有一个 \xfile{docstrip.cfg}\ 文件,该文件能配置并启用 \docstrip\ 的 TDS 安装功能,
% 则一些文件可能已经在正确的位置了,请参阅 \docstrip\ 的文档(documentation)。
%
% \subsection{刷新文件名数据库}
%
% 如果您的 \TeX~发行版(\TeX\,Live、\mikTeX、\dots)依赖于文件名数据库(file name databases),
% 则必须刷新这些文件名数据库。例如,\TeX\,Live\ 用户运行 \verb|texhash| 或 \verb|mktexlsr|。
%
% \subsection{一些感兴趣的细节}
%
% \paragraph{用 \LaTeX\ 解压。}
% \xfile{.dtx}\ 根据格式(format)选择其操作(action):
% \begin{description}
% \item[\plainTeX:] 运行 \docstrip\ 并解压文件。
% \item[\LaTeX:] 生成文档。
% \end{description}
% 如果您坚持通过 \LaTeX\ 使用\docstrip (实际上 \docstrip\ 并不需要 \LaTeX),那么请您的意图告知自动检测程序:
% \begin{quote}
%   \verb|latex \let\install=y\input{bookmark.dtx}|
% \end{quote}
% 不要忘记根据 shell 的要求引用这个参数(argument)。
%
% \paragraph{知生成文档。}
% 您可以同时使用 \xfile{.dtx}\ 或 \xfile{.drv}\ 来生成文档。可以通过配置文件 \xfile{ltxdoc.cfg}\ 配置该进程。
% 例如,如果您希望 A4 作为纸张格式,请将下面这行写入此文件中:
% \begin{quote}
%   \verb|\PassOptionsToClass{a4paper}{article}|
% \end{quote}
% 下面是一个如何使用 pdf\LaTeX\ 生成文档的示例:
% \begin{quote}
%\begin{verbatim}
%pdflatex bookmark.dtx
%makeindex -s gind.ist bookmark.idx
%pdflatex bookmark.dtx
%makeindex -s gind.ist bookmark.idx
%pdflatex bookmark.dtx
%\end{verbatim}
% \end{quote}
%
% \begin{thebibliography}{9}
%
% \bibitem{hyperref}
%   Sebastian Rahtz, Heiko Oberdiek:
%   \textit{The \xpackage{hyperref} package};
%   2011/04/17 v6.82g;
%   \CTANpkg{hyperref}
%
% \bibitem{currfile}
%   Martin Scharrer:
%   \textit{The \xpackage{currfile} package};
%   2011/01/09 v0.4.
%   \CTANpkg{currfile}
%
% \end{thebibliography}
%
% \begin{History}
%   \begin{Version}{2007/02/19 v0.1}
%   \item
%     First experimental version.
%   \end{Version}
%   \begin{Version}{2007/02/20 v0.2}
%   \item
%     Option \xoption{startatroot} added.
%   \item
%     Dummies for \cs{pdf(un)escape...} commands added to get
%     the package basically work for non-\hologo{pdfTeX} users.
%   \end{Version}
%   \begin{Version}{2007/02/21 v0.3}
%   \item
%     Dependency from \hologo{pdfTeX} 1.30 removed by using package
%     \xpackage{pdfescape}.
%   \end{Version}
%   \begin{Version}{2007/02/22 v0.4}
%   \item
%     \xpackage{hyperref}'s \xoption{bookmarkstype} respected.
%   \end{Version}
%   \begin{Version}{2007/03/02 v0.5}
%   \item
%     Driver options \xoption{vtex} (PDF mode), \xoption{dvipsone},
%     and \xoption{textures} added.
%   \item
%     Implementation of option \xoption{depth} completed. Division names
%     are supported, see \xpackage{hyperref}'s
%     option \xoption{bookmarksdepth}.
%   \item
%     \xpackage{hyperref}'s options \xoption{bookmarksopen},
%     \xoption{bookmarksopenlevel}, and \xoption{bookmarksdepth} respected.
%   \end{Version}
%   \begin{Version}{2007/03/03 v0.6}
%   \item
%     Option \xoption{numbered} as alias for \xpackage{hyperref}'s
%     \xoption{bookmarksnumbered}.
%   \end{Version}
%   \begin{Version}{2007/03/07 v0.7}
%   \item
%     Dependency from \hologo{eTeX} removed.
%   \end{Version}
%   \begin{Version}{2007/04/09 v0.8}
%   \item
%     Option \xoption{atend} added.
%   \item
%     Option \xoption{rgbcolor} removed.
%     \verb|rgbcolor=<r> <g> <b>| can be replaced by
%     \verb|color=[rgb]{<r>,<g>,<b>}|.
%   \item
%     Support of recent cvs version (2007-03-29) of dvipdfmx
%     that extends the \cs{special} for bookmarks to specify
%     open outline entries. Option \xoption{dvipdfmx-outline-open}
%     or \cs{SpecialDvipdfmxOutlineOpen} notify the package.
%   \end{Version}
%   \begin{Version}{2007/04/25 v0.9}
%   \item
%     The syntax of \cs{special} of dvipdfmx, if feature
%     \xoption{dvipdfmx-outline-open} is enabled, has changed.
%     Now cvs version 2007-04-25 is needed.
%   \end{Version}
%   \begin{Version}{2007/05/29 v1.0}
%   \item
%     Bug fix in code for second parameter of XYZ.
%   \end{Version}
%   \begin{Version}{2007/07/13 v1.1}
%   \item
%     Fix for pdfmark with GoToR action.
%   \end{Version}
%   \begin{Version}{2007/09/25 v1.2}
%   \item
%     pdfmark driver respects \cs{nofiles}.
%   \end{Version}
%   \begin{Version}{2008/08/08 v1.3}
%   \item
%     Package \xpackage{flags} replaced by package \xpackage{bitset}.
%     Now flags are also supported without \hologo{eTeX}.
%   \item
%     Hook for package \xpackage{hypdestopt} added.
%   \end{Version}
%   \begin{Version}{2008/09/13 v1.4}
%   \item
%     Fix for bug introduced in v1.3, package \xpackage{flags} is one-based,
%     but package \xpackage{bitset} is zero-based. Thus options \xoption{bold}
%     and \xoption{italic} are wrong in v1.3. (Daniel M\"ullner)
%   \end{Version}
%   \begin{Version}{2009/08/13 v1.5}
%   \item
%     Except for driver options the other options are now local options.
%     This resolves a problem with KOMA-Script v3.00 and its option \xoption{open}.
%   \end{Version}
%   \begin{Version}{2009/12/06 v1.6}
%   \item
%     Use of package \xpackage{atveryend} for drivers \xoption{pdftex}
%     and \xoption{pdfmark}.
%   \end{Version}
%   \begin{Version}{2009/12/07 v1.7}
%   \item
%     Use of package \xpackage{atveryend} fixed.
%   \end{Version}
%   \begin{Version}{2009/12/17 v1.8}
%   \item
%     Support of \xpackage{hyperref} 2009/12/17 v6.79v for \hologo{XeTeX}.
%   \end{Version}
%   \begin{Version}{2010/03/30 v1.9}
%   \item
%     Package name in an error message fixed.
%   \end{Version}
%   \begin{Version}{2010/04/03 v1.10}
%   \item
%     Option \xoption{style} and macro \cs{bookmarkdefinestyle} added.
%   \item
%     Hook support with option \xoption{addtohook} added.
%   \item
%     \cs{bookmarkget} added.
%   \end{Version}
%   \begin{Version}{2010/04/04 v1.11}
%   \item
%     Bug fix (introduced in v1.10).
%   \end{Version}
%   \begin{Version}{2010/04/08 v1.12}
%   \item
%     Requires \xpackage{ltxcmds} 2010/04/08.
%   \end{Version}
%   \begin{Version}{2010/07/23 v1.13}
%   \item
%     Support for \xclass{memoir}'s \cs{booknumberline} added.
%   \end{Version}
%   \begin{Version}{2010/09/02 v1.14}
%   \item
%     (Local) options \xoption{draft} and \xoption{final} added.
%   \end{Version}
%   \begin{Version}{2010/09/25 v1.15}
%   \item
%     Fix for option \xoption{dvipdfmx-outline-open}.
%   \item
%     Option \xoption{dvipdfmx-outline-open} is set automatically,
%     if XeTeX $\geq$ 0.9995 is detected.
%   \end{Version}
%   \begin{Version}{2010/10/19 v1.16}
%   \item
%     Option `startatroot' now acts globally.
%   \item
%     Option `level' also accepts names the same way as option `depth'.
%   \end{Version}
%   \begin{Version}{2010/10/25 v1.17}
%   \item
%     \cs{bookmarksetupnext} added.
%   \item
%     Using \cs{kvsetkeys} of package \xpackage{kvsetkeys}, because
%     \cs{setkeys} of package \xpackage{keyval} is not reentrant.
%     This can cause problems (unknown keys) with older versions of
%     hyperref that also uses \cs{setkeys} (found by GL).
%   \end{Version}
%   \begin{Version}{2010/11/05 v1.18}
%   \item
%     Use of \cs{pdf@ifdraftmode} of package \xpackage{pdftexcmds} for
%     the default of option \xoption{draft}.
%   \end{Version}
%   \begin{Version}{2011/03/20 v1.19}
%   \item
%     Use of \cs{dimexpr} fixed, if \hologo{eTeX} is not used.
%     (Bug found by Martin M\"unch.)
%   \item
%     Fix in documentation. Also layout options work without \hologo{eTeX}.
%   \end{Version}
%   \begin{Version}{2011/04/13 v1.20}
%   \item
%     Bug fix: \cs{BKM@SetDepth} renamed to \cs{BKM@SetDepthOrLevel}.
%   \end{Version}
%   \begin{Version}{2011/04/21 v1.21}
%   \item
%     Some support for file name and line number in error messages
%     at end of document (pdfTeX and pdfmark based drivers).
%   \end{Version}
%   \begin{Version}{2011/05/13 v1.22}
%   \item
%     Change of version 2010/11/05 v1.18 reverted, because otherwise
%     draftmode disables some \xext{aux} file entries.
%   \end{Version}
%   \begin{Version}{2011/09/19 v1.23}
%   \item
%     Some \cs{renewcommand}s changed to \cs{def} to avoid trouble
%     if the commands are not defined, because hyperref stopped early.
%   \end{Version}
%   \begin{Version}{2011/12/02 v1.24}
%   \item
%     Small optimization in \cs{BKM@toHexDigit}.
%   \end{Version}
%   \begin{Version}{2016/05/16 v1.25}
%   \item
%     Documentation updates.
%   \end{Version}
%   \begin{Version}{2016/05/17 v1.26}
%   \item
%     define \cs{pdfoutline} to allow pdftex driver to be used with Lua\TeX.
%   \end{Version}
%   \begin{Version}{2019/06/04 v1.27}
%   \item
%     unknown style options are ignored (issue 67)
%   \end{Version}

%   \begin{Version}{2019/12/03 v1.28}
%   \item
%     Documentation updates.
%   \item adjust package loading (all required packages already loaded
%     by \xpackage{hyperref}).
%   \end{Version}
%   \begin{Version}{2020-11-06 v1.29}
%   \item Adapted the dvips to avoid a clash with pgf.
%         https://github.com/pgf-tikz/pgf/issues/944
%   \item All drivers now use the new LaTeX hooks
%         and so require a format 2020-10-01 or newer. The older
%         drivers are provided as frozen versions and are used if an older
%         format is detected.
%   \item Added support for destlabel option of hyperref, https://github.com/ho-tex/bookmark/issues/1
%   \item Removed the \xoption{dvipsone} and \xoption{textures} driver.
%   \item Removed the code for option \xoption{dvipdfmx-outline-open}
%     and \cs{SpecialDvipdfmxOutlineOpen}. All dvipdfmx version should now support
%     this out-of-the-box.
%   \end{Version}
% \end{History}
%
% \PrintIndex
%
% \Finale
\endinput

%        (quote the arguments according to the demands of your shell)
%
% Documentation:
%    (a) If bookmark.drv is present:
%           latex bookmark.drv
%    (b) Without bookmark.drv:
%           latex bookmark.dtx; ...
%    The class ltxdoc loads the configuration file ltxdoc.cfg
%    if available. Here you can specify further options, e.g.
%    use A4 as paper format:
%       \PassOptionsToClass{a4paper}{article}
%
%    Programm calls to get the documentation (example):
%       pdflatex bookmark.dtx
%       makeindex -s gind.ist bookmark.idx
%       pdflatex bookmark.dtx
%       makeindex -s gind.ist bookmark.idx
%       pdflatex bookmark.dtx
%
% Installation:
%    TDS:tex/latex/bookmark/bookmark.sty
%    TDS:tex/latex/bookmark/bkm-dvipdfm.def
%    TDS:tex/latex/bookmark/bkm-dvips.def
%    TDS:tex/latex/bookmark/bkm-pdftex.def
%    TDS:tex/latex/bookmark/bkm-vtex.def
%    TDS:tex/latex/bookmark/bkm-dvipdfm-2019-12-03.def
%    TDS:tex/latex/bookmark/bkm-dvips-2019-12-03.def
%    TDS:tex/latex/bookmark/bkm-pdftex-2019-12-03.def
%    TDS:tex/latex/bookmark/bkm-vtex-2019-12-03.def%
%    TDS:doc/latex/bookmark/bookmark.pdf
%    TDS:doc/latex/bookmark/bookmark-example.tex
%    TDS:source/latex/bookmark/bookmark.dtx
%    TDS:source/latex/bookmark/bookmark-frozen.dtx
%
%<*ignore>
\begingroup
  \catcode123=1 %
  \catcode125=2 %
  \def\x{LaTeX2e}%
\expandafter\endgroup
\ifcase 0\ifx\install y1\fi\expandafter
         \ifx\csname processbatchFile\endcsname\relax\else1\fi
         \ifx\fmtname\x\else 1\fi\relax
\else\csname fi\endcsname
%</ignore>
%<*install>
\input docstrip.tex
\Msg{************************************************************************}
\Msg{* Installation}
\Msg{* Package: bookmark 2020-11-06 v1.29 PDF bookmarks (HO)}
\Msg{************************************************************************}

\keepsilent
\askforoverwritefalse

\let\MetaPrefix\relax
\preamble

This is a generated file.

Project: bookmark
Version: 2020-11-06 v1.29

Copyright (C)
   2007-2011 Heiko Oberdiek
   2016-2020 Oberdiek Package Support Group

This work may be distributed and/or modified under the
conditions of the LaTeX Project Public License, either
version 1.3c of this license or (at your option) any later
version. This version of this license is in
   https://www.latex-project.org/lppl/lppl-1-3c.txt
and the latest version of this license is in
   https://www.latex-project.org/lppl.txt
and version 1.3 or later is part of all distributions of
LaTeX version 2005/12/01 or later.

This work has the LPPL maintenance status "maintained".

The Current Maintainers of this work are
Heiko Oberdiek and the Oberdiek Package Support Group
https://github.com/ho-tex/bookmark/issues


This work consists of the main source file bookmark.dtx and bookmark-frozen.dtx
and the derived files
   bookmark.sty, bookmark.pdf, bookmark.ins, bookmark.drv,
   bkm-dvipdfm.def, bkm-dvips.def, bkm-pdftex.def, bkm-vtex.def,
   bkm-dvipdfm-2019-12-03.def, bkm-dvips-2019-12-03.def,
   bkm-pdftex-2019-12-03.def, bkm-vtex-2019-12-03.def,
   bookmark-example.tex.

\endpreamble
\let\MetaPrefix\DoubleperCent

\generate{%
  \file{bookmark.ins}{\from{bookmark.dtx}{install}}%
  \file{bookmark.drv}{\from{bookmark.dtx}{driver}}%
  \usedir{tex/latex/bookmark}%
  \file{bookmark.sty}{\from{bookmark.dtx}{package}}%
  \file{bkm-dvipdfm.def}{\from{bookmark.dtx}{dvipdfm}}%
  \file{bkm-dvips.def}{\from{bookmark.dtx}{dvips,pdfmark}}%
  \file{bkm-pdftex.def}{\from{bookmark.dtx}{pdftex}}%
  \file{bkm-vtex.def}{\from{bookmark.dtx}{vtex}}%
  \usedir{doc/latex/bookmark}%
  \file{bookmark-example.tex}{\from{bookmark.dtx}{example}}%
  \file{bkm-pdftex-2019-12-03.def}{\from{bookmark-frozen.dtx}{pdftexfrozen}}%
  \file{bkm-dvips-2019-12-03.def}{\from{bookmark-frozen.dtx}{dvipsfrozen}}%
  \file{bkm-vtex-2019-12-03.def}{\from{bookmark-frozen.dtx}{vtexfrozen}}%
  \file{bkm-dvipdfm-2019-12-03.def}{\from{bookmark-frozen.dtx}{dvipdfmfrozen}}%
}

\catcode32=13\relax% active space
\let =\space%
\Msg{************************************************************************}
\Msg{*}
\Msg{* To finish the installation you have to move the following}
\Msg{* files into a directory searched by TeX:}
\Msg{*}
\Msg{*     bookmark.sty, bkm-dvipdfm.def, bkm-dvips.def,}
\Msg{*     bkm-pdftex.def, bkm-vtex.def, bkm-dvipdfm-2019-12-03.def,}
\Msg{*     bkm-dvips-2019-12-03.def, bkm-pdftex-2019-12-03.def,}
\Msg{*     and bkm-vtex-2019-12-03.def}
\Msg{*}
\Msg{* To produce the documentation run the file `bookmark.drv'}
\Msg{* through LaTeX.}
\Msg{*}
\Msg{* Happy TeXing!}
\Msg{*}
\Msg{************************************************************************}

\endbatchfile
%</install>
%<*ignore>
\fi
%</ignore>
%<*driver>
\NeedsTeXFormat{LaTeX2e}
\ProvidesFile{bookmark.drv}%
  [2020-11-06 v1.29 PDF bookmarks (HO)]%
\documentclass{ltxdoc}
\usepackage{ctex}
\usepackage{indentfirst}
\setlength{\parindent}{2em}
\usepackage{holtxdoc}[2011/11/22]
\usepackage{xcolor}
\usepackage{hyperref}
\usepackage[open,openlevel=3,atend]{bookmark}[2020/11/06] %%%打开书签,显示的深度为3级,即显示part、section、subsection。
\bookmarksetup{color=red}
\begin{document}

  \renewcommand{\contentsname}{目\quad 录}
  \renewcommand{\abstractname}{摘\quad 要}
  \renewcommand{\historyname}{历史}
  \DocInput{bookmark.dtx}%
\end{document}
%</driver>
% \fi
%
%
%
% \GetFileInfo{bookmark.drv}
%
%% \title{\xpackage{bookmark} 宏包}
% \title{\heiti {\Huge \textbf{\xpackage{bookmark}\ 宏包}}}
% \date{2020-11-06\ \ \ v1.29}
% \author{Heiko Oberdiek \thanks
% {如有问题请点击:\url{https://github.com/ho-tex/bookmark/issues}}\\[5pt]赣医一附院神经科\ \ 黄旭华\ \ \ \ 译}
%
% \maketitle
%
% \begin{abstract}
% 这个宏包为 \xpackage{hyperref}\ 宏包实现了一个新的书签(bookmark)(大纲[outline])组织。现在
% 可以设置样式(style)和颜色(color)等书签属性(bookmark properties)。其他动作类型(action types)可用
% (URI、GoToR、Named)。书签是在第一次编译运行(compile run)中生成的。\xpackage{hyperref}\
% 宏包必需运行两次。
% \end{abstract}
%
% \tableofcontents
%
% \section{文档(Documentation)}
%
% \subsection{介绍}
%
% 这个 \xpackage{bookmark}\ 宏包试图为书签(bookmarks)提供一个更现代的管理:
% \begin{itemize}
% \item 书签已经在第一次 \hologo{TeX}\ 编译运行(compile run)中生成。
% \item 可以更改书签的字体样式(font style)和颜色(color)。
% \item 可以执行比简单的 GoTo 操作(actions)更多的操作。
% \end{itemize}
%
% 与 \xpackage{hyperref} \cite{hyperref} 一样,书签(bookmarks)也是按照书签生成宏
% (bookmark generating macros)(\cs{bookmark})的顺序生成的。级别号(level number)用于
% 定义书签的树结构(tree structure)。限制没有那么严格:
% \begin{itemize}
% \item 级别值(level values)可以跳变(jump)和省略(omit)。\cs{subsubsection}\ 可以跟在
%       \cs{chapter}\ 之后。这种情况如在 \xpackage{hyperref}\ 中则产生错误,它将显示一个警告(warning)
%       并尝试修复此错误。
% \item 多个书签可能指向同一目标(destination)。在 \xpackage{hyperref}\ 中,这会完全弄乱
%       书签树(bookmark tree),因为算法假设(algorithm assumes)目标名称(destination names)
%       是键(keys)(唯一的)。
% \end{itemize}
%
% 注意,这个宏包是作为书签管理(bookmark management)的实验平台(experimentation platform)。
% 欢迎反馈。此外,在未来的版本中,接口(interfaces)也可能发生变化。
%
% \subsection{选项(Options)}
%
% 可在以下四个地方放置选项(options):
% \begin{enumerate}
% \item \cs{usepackage}|[|\meta{options}|]{bookmark}|\\
%       这是放置驱动程序选项(driver options)和 \xoption{atend}\ 选项的唯一位置。
% \item \cs{bookmarksetup}|{|\meta{options}|}|\\
%       此命令仅用于设置选项(setting options)。
% \item \cs{bookmarksetupnext}|{|\meta{options}|}|\\
%       这些选项在下一个 \cs{bookmark}\ 命令的选项之后存储(stored)和调用(called)。
% \item \cs{bookmark}|[|\meta{options}|]{|\meta{title}|}|\\
%       此命令设置书签。选项设置(option settings)仅限于此书签。
% \end{enumerate}
% 异常(Exception):加载该宏包后,无法更改驱动程序选项(Driver options)、\xoption{atend}\ 选项
% 、\xoption{draft}\slash\xoption{final}选项。
%
% \subsubsection{\xoption{draft} 和 \xoption{final}\ 选项}
%
% 如果一个\LaTeX\ 文件要被编译了多次,那么可以使用 \xoption{draft}\ 选项来禁用该宏包的书签内
% 容(bookmark stuff),这样可以节省一点时间。默认 \xoption{final}\ 选项。两个选项都是
% 布尔选项(boolean options),如果没有值,则使用值 |true|。|draft=true| 与 |final=false| 相同。
%
% 除了驱动程序选项(driver options)之外,\xpackage{bookmark}\ 宏包选项都是局部选项(local options)。
% \xoption{draft}\ 选项和 \xoption{final}\ 选项均属于文档类选项(class option)(译者注:文档类选项为全局选项),
% 因此,在 \xpackage{bookmark}\ 宏包中未能看到这两个选项。如果您想使用全局的(global) \xoption{draft}选项
% 来优化第一次 \LaTeX\ 运行(runs),可以在导言(preamble)中引入 \xpackage{ifdraft}\ 宏包并设置 \LaTeX\ 的
% \cs{PassOptionsToPackage},例如:
%\begin{quote}
%\begin{verbatim}
%\documentclass[draft]{article}
%\usepackage{ifdraft}
%\ifdraft{%
%   \PassOptionsToPackage{draft}{bookmark}%
%}{}
%\end{verbatim}
%\end{quote}
%
% \subsubsection{驱动程序选项(Driver options)}
%
% 支持的驱动程序( drivers)包括 \xoption{pdftex}、\xoption{dvips}、\xoption{dvipdfm} (\xoption{xetex})、
% \xoption{vtex}。\hologo{TeX}\ 引擎 \hologo{pdfTeX}、\hologo{XeTeX}、\hologo{VTeX}\ 能被自动检测到。
% 默认的 DVI 驱动程序是 \xoption{dvips}。这可以通过 \cs{BookmarkDriverDefault}\ 在配置
% 文件 \xfile{bookmark.cfg}\ 中进行更改,例如:
% \begin{quote}
% |\def\BookmarkDriverDefault{dvipdfm}|
% \end{quote}
% 当前版本的(current versions)驱动程序使用新的 \LaTeX\ 钩子(\LaTeX-hooks)。如果检测到比
% 2020-10-01 更旧的格式,则将以前驱动程序的冻结版本(frozen versions)作为备份(fallback)。
%
% \paragraph{用 dvipdfmx 打开书签(bookmarks)。}旧版本的宏包有一个 \xoption{dvipdfmx-outline-open}\ 选项
% 可以激活代码,而该代码可以指定一个大纲条目(outline entry)是否打开。该宏包现在假设所有使用的 dvipdfmx 版本都是
% 最新版本,足以理解该代码,因此始终激活该代码。选项本身将被忽略。
%
%
% \subsubsection{布局选项(Layout options)}
%
% \paragraph{字体(Font)选项:}
%
% \begin{description}
% \item[\xoption{bold}:] 如果受 PDF 浏览器(PDF viewer)支持,书签将以粗体字体(bold font)显示(自 PDF 1.4起)。
% \item[\xoption{italic}:] 使用斜体字体(italic font)(自 PDF 1.4起)。
% \end{description}
% \xoption{bold}(粗体) 和 \xoption{italic}(斜体)可以同时使用。而 |false| 值(value)禁用字体选项。
%
% \paragraph{颜色(Color)选项:}
%
% 彩色书签(Colored bookmarks)是 PDF 1.4 的一个特性(feature),并非所有的 PDF 浏览器(PDF viewers)都支持彩色书签。
% \begin{description}
% \item[\xoption{color}:] 这里 color(颜色)可以作为 \xpackage{color}\ 宏包或 \xpackage{xcolor}\ 宏包的
% 颜色规范(color specification)给出。空值(empty value)表示未设置颜色属性。如果未加载 \xpackage{xcolor}\ 宏包,
% 能识别的值(recognized values)只有:
%   \begin{itemize}
%   \item 空值(empty value)表示未设置颜色属性,\\
%         例如:|color={}|
%   \item 颜色模型(color model) rgb 的显式颜色规范(explicit color specification),\\
%         例如,红色(red):|color=[rgb]{1,0,0}|
%   \item 颜色模型(color model)灰(gray)的显式颜色规范(explicit color specification),\\
%         例如,深灰色(dark gray):|color=[gray]{0.25}|
%   \end{itemize}
%   请注意,如果加载了 \xpackage{color}\ 宏包,此限制(restriction)也适用。然而,如果加载了 \xpackage{xcolor}\ 宏包,
%   则可以使用所有颜色规范(color specifications)。
% \end{description}
%
% \subsubsection{动作选项(Action options)}
%
% \begin{description}
% \item[\xoption{dest}:] 目的地名称(destination name)。
% \item[\xoption{page}:] 页码(page number),第一页(first page)为 1。
% \item[\xoption{view}:] 浏览规范(view specification),示例如下:\\
%   |view={FitB}|, |view={FitH 842}|, |view={XYZ 0 100 null}|\ \  一些浏览规范参数(view specification parameters)
%   将数字(numbers)视为具有单位 bp 的参数。它们可以作为普通数字(plain numbers)或在 \cs{calc}\ 内部以
%   长度表达式(length expressions)给出。如果加载了 \xpackage{calc}\ 宏包,则支持该宏包的表达式(expressions)。否则,
%   使用 \hologo{eTeX}\ 的 \cs{dimexpr}。例如:\\
%   |view={FitH \calc{\paperheight-\topmargin-1in}}|\\
%   |view={XYZ 0 \calc{\paperheight} null}|\\
%   注意 \cs{calc}\ 不能用于 |XYZ| 的第三个参数,因为该参数是缩放值(zoom value),而不是长度(length)。

% \item[\xoption{named}:] 已命名的动作(Named action)的名称:\\
%   |FirstPage|(第一页),|LastPage|(最后一页),|NextPage|(下一页),|PrevPage|(前一页)
% \item[\xoption{gotor}:] 外部(external) PDF 文件的名称。
% \item[\xoption{uri}:] URI 规范(URI specification)。
% \item[\xoption{rawaction}:] 原始动作规范(raw action specification)。由于这些规范取决于驱动程序(driver),因此不应使用此选项。
% \end{description}
% 通过分析指定的选项来选择书签的适当动作。动作由不同的选项集(sets of options)区分:
% \begin{quote}
 \begin{tabular}{|@{}r|l@{}|}
%   \hline
%   \ \textbf{动作(Action)}\  & \ \textbf{选项(Options)}\ \\ \hline
%   \ \textsf{GoTo}\  &\  \xoption{dest}\ \\ \hline
%   \ \textsf{GoTo}\  & \ \xoption{page} + \xoption{view}\ \\ \hline
%   \ \textsf{GoToR}\  & \ \xoption{gotor} + \xoption{dest}\ \\ \hline
%   \ \textsf{GoToR}\  & \ \xoption{gotor} + \xoption{page} + \xoption{view}\ \ \ \\ \hline
%   \ \textsf{Named}\  &\  \xoption{named}\ \\ \hline
%   \ \textsf{URI}\  & \ \xoption{uri}\ \\ \hline
% \end{tabular}
% \end{quote}
%
% \paragraph{缺少动作(Missing actions)。}
% 如果动作缺少 \xpackage{bookmark}\ 宏包,则抛出错误消息(error message)。根据驱动程序(driver)
% (\xoption{pdftex}、\xoption{dvips}\ 和好友[friends]),宏包在文档末尾很晚才检测到它。
% 自 2011/04/21 v1.21 版本以后,该宏包尝试打印 \cs{bookmark}\ 的相应出现的行号(line number)和文件名(file name)。
% 然而,\hologo{TeX}\ 确实提供了行号,但不幸的是,文件名是一个秘密(secret)。但该宏包有如下获取文件名的方法:
% \begin{itemize}
% \item 如果 \hologo{LuaTeX} (独立于 DVI 或 PDF 模式)正在运行,则自动使用其 |status.filename|。
% \item 宏包的 \cs{currfile} \cite{currfile}\ 重新定义了 \hologo{LaTeX}\ 的内部结构,以跟踪文件名(file name)。
% 如果加载了该宏包,那么它的 \cs{currfilepath}\ 将被检测到并由 \xpackage{bookmark}\ 自动使用。
% \item 可以通过 \cs{bookmarksetup}\ 或 \cs{bookmark}\ 中的 \xoption{scrfile}\ 选项手动设置(set manually)文件名。
% 但是要小心,手动设置会禁用以前的文件名检测方法。错误的(wrong)或丢失的(missed)文件名设置(file name setting)可能会在错误消息中
% 为您提供错误的源位置(source location)。
% \end{itemize}
%
% \subsubsection{级别选项(Level options)}
%
% 书签条目(bookmark entries)的顺序由 \cs{bookmark}\ 命令的的出现顺序(appearance order)定义。
% 树结构(tree structure)由书签节点(bookmark nodes)的属性 \xoption{level}(级别)构建。
% \xoption{level}\ 的值是整数(integers)。如果书签条目级别的值高于前一个节点,则该条目将成为
% 前一个节点的子(child)节点。差值的绝对值并不重要。
%
% \xpackage{bookmark}\ 宏包能记住全局属性(global property)“current level(当前级别)”中上
% 一个书签条目(previous bookmark entry)的级别。
%
% 级别系统的(level system)行为(behaviour)可以通过以下选项进行配置:
% \begin{description}
% \item[\xoption{level}:]
%    设置级别(level),请参阅上面的说明。如果给出的选项 \xoption{level}\ 没有值,那么将恢复默
%    认行为,即将“当前级别(current level)”用作级别值(level value)。自 2010/10/19 v1.16 版本以来,
%    如果宏 \cs{toclevel@part}、\cs{toclevel@section}\ 被定义过(通过 \xpackage{hyperref}\ 宏包完成,
%    请参阅它的 \xoption{bookmarkdepth}\ 选项),则 \xpackage{bookmark}\ 宏包还支持 |part|、|section| 等名称。
%
% \item[\xoption{rellevel}:]
%    设置相对于前一级别的(previous level)级别。正值表示书签条目成为前一个书签条目的子条目。
% \item[\xoption{keeplevel}:]
%    使用由\xoption{level}\ 或 \xoption{rellevel}\ 设置的级别,但不要更改全局属性“current level(当前级别)”。
%    可以通过设置为 |false| 来禁用该选项。
% \item[\xoption{startatroot}:]
%    此时,书签树(bookmark tree)再次从顶层(top level)开始。下一个书签条目不会作为上一个条目的子条目进行排序。
%    示例场景:文档使用 part。但是,最后几章(last chapters)不应放在最后一部分(last part)下面:
%    \begin{quote}
%\begin{verbatim}
%\documentclass{book}
%[...]
%\begin{document}
%  \part{第一部分}
%    \chapter{第一部分的第1章}
%    [...]
%  \part{第二部分(Second part)}
%    \chapter{第二部分的第1章}
%    [...]
%  \bookmarksetup{startatroot}
%  \chapter{Index}% 不属于第二部分
%\end{document}
%\end{verbatim}
%    \end{quote}
% \end{description}
%
% \subsubsection{样式定义(Style definitions)}
%
% 样式(style)是一组选项设置(option settings)。它可以由宏 \cs{bookmarkdefinestyle}\ 定义,
% 并由它的 \xoption{style}\ 选项使用。
% \begin{declcs}{bookmarkdefinestyle} \M{name} \M{key value list}
% \end{declcs}
% 选项设置(option settings)的 \meta{key value list}(键值列表)被指定为样式名(style \meta{name})。
%
% \begin{description}
% \item[\xoption{style}:]
%   \xoption{style}\ 选项的值是以前定义的样式的名称(name)。现在执行其选项设置(option settings)。
%   选项可以包括 \xoption{style}\ 选项。通过递归调用相同样式的无限递归(endless recursion)被阻止并抛出一个错误。
% \end{description}
%
% \subsubsection{钩子支持(Hook support)}
%
% 处理宏\cs{bookmark}\ 的可选选项(optional options)后,就会调用钩子(hook)。
% \begin{description}
% \item[\xoption{addtohook}:]
%   代码(code)作为该选项的值添加到钩子中。
% \end{description}
%
% \begin{declcs}{bookmarkget} \M{option}
% \end{declcs}
% \cs{bookmarkget}\ 宏提取 \meta{option}\ 选项的最新选项设置(latest option setting)的值。
% 对于布尔选项(boolean option),如果启用布尔选项,则返回 1,否则结果为零。结果数字(resulting numbers)
% 可以直接用于 \cs{ifnum}\ 或 \cs{ifcase}。如果您想要数字 \texttt{0}\ 和 \texttt{1},
% 请在 \cs{bookmarkget}\ 前面加上 \cs{number}\ 作为前缀。\cs{bookmarkget}\ 宏是可展开的(expandable)。
% 如果选项不受支持,则返回空字符串(empty string)。受支持的布尔选项有:
% \begin{quote}
%   \xoption{bold}、
%   \xoption{italic}、
%   \xoption{open}
% \end{quote}
% 其他受支持的选项有:
% \begin{quote}
%   \xoption{depth}、
%   \xoption{dest}、
%   \xoption{color}、
%   \xoption{gotor}、
%   \xoption{level}、
%   \xoption{named}、
%   \xoption{openlevel}、
%   \xoption{page}、
%   \xoption{rawaction}、
%   \xoption{uri}、
%   \xoption{view}、
% \end{quote}
% 另外,以下键(key)是可用的:
% \begin{quote}
%   \xoption{text}
% \end{quote}
% 它返回大纲条目(outline entry)的文本(text)。
%
% \paragraph{选项设置(Option setting)。}
% 在钩子(hook)内部可以使用 \cs{bookmarksetup}\ 设置选项。
%
% \subsection{与 \xpackage{hyperref}\ 的兼容性}
%
% \xpackage{bookmark}\ 宏包自动禁用 \xpackage{hyperref}\ 宏包的书签(bookmarks)。但是,
% \xpackage{bookmark}\ 宏包使用了 \xpackage{hyperref}\ 宏包的一些代码。例如,
% \xpackage{bookmark}\ 宏包重新定义了 \xpackage{hyperref}\ 宏包在 \cs{addcontentsline}\ 命令
% 和其他命令中插入的\cs{Hy@writebookmark}\ 钩子。因此,不应禁用 \xpackage{hyperref}\ 宏包的书签。
%
% \xpackage{bookmark}\ 宏包使用 \xpackage{hyperref}\ 宏包的 \cs{pdfstringdef},且不提供替换(replacement)。
%
% \xpackage{hyperref}\ 宏包的一些选项也能在 \xpackage{bookmark}\ 宏包中实现(implemented):
% \begin{quote}
% \begin{tabular}{|l@{}|l@{}|}
%   \hline
%   \xpackage{hyperref}\ 宏包的选项\  &\ \xpackage{bookmark}\ 宏包的选项\ \ \\ \hline
%   \xoption{bookmarksdepth} &\ \xoption{depth}\\ \hline
%   \xoption{bookmarksopen} & \ \xoption{open}\\ \hline
%   \xoption{bookmarksopenlevel}\ \ \  &\ \xoption{openlevel}\\ \hline
%   \xoption{bookmarksnumbered} \ \ \ &\ \xoption{numbered}\\ \hline
% \end{tabular}
% \end{quote}
%
% 还可以使用以下命令:
% \begin{quote}
%   \cs{pdfbookmark}\\
%   \cs{currentpdfbookmark}\\
%   \cs{subpdfbookmark}\\
%   \cs{belowpdfbookmark}
% \end{quote}
%
% \subsection{在末尾添加书签}
%
% 宏包选项 \xoption{atend}\ 启用以下宏(macro):
% \begin{declcs}{BookmarkAtEnd}
%   \M{stuff}
% \end{declcs}
% \cs{BookmarkAtEnd}\ 宏将 \meta{stuff}\ 放在文档末尾。\meta{stuff}\ 表示书签命令(bookmark commands)。举例:
% \begin{quote}
%\begin{verbatim}
%\usepackage[atend]{bookmark}
%\BookmarkAtEnd{%
%  \bookmarksetup{startatroot}%
%  \bookmark[named=LastPage, level=0]{Last page}%
%}
%\end{verbatim}
% \end{quote}
%
% 或者,可以在 \cs{bookmark}\ 中给出 \xoption{startatroot}\ 选项:
% \begin{quote}
%\begin{verbatim}
%\BookmarkAtEnd{%
%  \bookmark[
%    startatroot,
%    named=LastPage,
%    level=0,
%  ]{Last page}%
%}
%\end{verbatim}
% \end{quote}
%
% \paragraph{备注(Remarks):}
% \begin{itemize}
% \item
%   \cs{BookmarkAtEnd} 隐藏了这样一个事实,即在文档末尾添加书签的方法取决于驱动程序(driver)。
%
%   为此,驱动程序 \xoption{pdftex}\ 使用 \xpackage{atveryend}\ 宏包。如果 \cs{AtEndDocument}\ 太早,
%   最后一个页面(last page)可能不会被发送出去(shipped out)。由于需要 \xext{aux}\ 文件,此驱动程序使
%   用 \cs{AfterLastShipout}。
%
%   其他驱动程序(\xoption{dvipdfm}、\xoption{xetex}、\xoption{vtex})的实现(implementation)
%   取决于 \cs{special},\cs{special}\ 在最后一页之后没有效果。在这种情况下,\xpackage{atenddvi}\ 宏包的
%   \cs{AtEndDvi}\ 有帮助。它将其参数(argument)放在文档的最后一页(last page)。至少需要运行 \hologo{LaTeX}\ 两次,
%   因为最后一页是由引用(reference)检测到的。
%
%   \xoption{dvips}\ 现在使用新的 LaTeX 钩子 \texttt{shipout/lastpage}。
% \item
%   未指定 \cs{BookmarkAtEnd}\ 参数的扩展时间(time of expansion)。这可以立即发生,也可以在文档末尾发生。
% \end{itemize}
%
% \subsection{限制/行动计划}
%
% \begin{itemize}
% \item 支持缺失动作(missing actions)(启动,\dots)。
% \item 对 \xpackage{hyperref}\ 的 \xoption{bookmarkstype}\ 选项进行了更好的设计(design)。
% \end{itemize}
%
% \section{示例(Example)}
%
%    \begin{macrocode}
%<*example>
%    \end{macrocode}
%    \begin{macrocode}
\documentclass{article}
\usepackage{xcolor}[2007/01/21]
\usepackage{hyperref}
\usepackage[
  open,
  openlevel=2,
  atend
]{bookmark}[2019/12/03]

\bookmarksetup{color=blue}

\BookmarkAtEnd{%
  \bookmarksetup{startatroot}%
  \bookmark[named=LastPage, level=0]{End/Last page}%
  \bookmark[named=FirstPage, level=1]{First page}%
}

\begin{document}
\section{First section}
\subsection{Subsection A}
\begin{figure}
  \hypertarget{fig}{}%
  A figure.
\end{figure}
\bookmark[
  rellevel=1,
  keeplevel,
  dest=fig
]{A figure}
\subsection{Subsection B}
\subsubsection{Subsubsection C}
\subsection{Umlauts: \"A\"O\"U\"a\"o\"u\ss}
\newpage
\bookmarksetup{
  bold,
  color=[rgb]{1,0,0}
}
\section{Very important section}
\bookmarksetup{
  italic,
  bold=false,
  color=blue
}
\subsection{Italic section}
\bookmarksetup{
  italic=false
}
\part{Misc}
\section{Diverse}
\subsubsection{Subsubsection, omitting subsection}
\bookmarksetup{
  startatroot
}
\section{Last section outside part}
\subsection{Subsection}
\bookmarksetup{
  color={}
}
\begingroup
  \bookmarksetup{level=0, color=green!80!black}
  \bookmark[named=FirstPage]{First page}
  \bookmark[named=LastPage]{Last page}
  \bookmark[named=PrevPage]{Previous page}
  \bookmark[named=NextPage]{Next page}
\endgroup
\bookmark[
  page=2,
  view=FitH 800
]{Page 2, FitH 800}
\bookmark[
  page=2,
  view=FitBH \calc{\paperheight-\topmargin-1in-\headheight-\headsep}
]{Page 2, FitBH top of text body}
\bookmark[
  uri={http://www.dante.de/},
  color=magenta
]{Dante homepage}
\bookmark[
  gotor={t.pdf},
  page=1,
  view={XYZ 0 1000 null},
  color=cyan!75!black
]{File t.pdf}
\bookmark[named=FirstPage]{First page}
\bookmark[rellevel=1, named=LastPage]{Last page (rellevel=1)}
\bookmark[named=PrevPage]{Previous page}
\bookmark[level=0, named=FirstPage]{First page (level=0)}
\bookmark[
  rellevel=1,
  keeplevel,
  named=LastPage
]{Last page (rellevel=1, keeplevel)}
\bookmark[named=PrevPage]{Previous page}
\end{document}
%    \end{macrocode}
%    \begin{macrocode}
%</example>
%    \end{macrocode}
%
% \StopEventually{
% }
%
% \section{实现(Implementation)}
%
% \subsection{宏包(Package)}
%
%    \begin{macrocode}
%<*package>
\NeedsTeXFormat{LaTeX2e}
\ProvidesPackage{bookmark}%
  [2020-11-06 v1.29 PDF bookmarks (HO)]%
%    \end{macrocode}
%
% \subsubsection{要求(Requirements)}
%
% \paragraph{\hologo{eTeX}.}
%
%    \begin{macro}{\BKM@CalcExpr}
%    \begin{macrocode}
\begingroup\expandafter\expandafter\expandafter\endgroup
\expandafter\ifx\csname numexpr\endcsname\relax
  \def\BKM@CalcExpr#1#2#3#4{%
    \begingroup
      \count@=#2\relax
      \advance\count@ by#3#4\relax
      \edef\x{\endgroup
        \def\noexpand#1{\the\count@}%
      }%
    \x
  }%
\else
  \def\BKM@CalcExpr#1#2#3#4{%
    \edef#1{%
      \the\numexpr#2#3#4\relax
    }%
  }%
\fi
%    \end{macrocode}
%    \end{macro}
%
% \paragraph{\hologo{pdfTeX}\ 的转义功能(escape features)}
%
%    \begin{macro}{\BKM@EscapeName}
%    \begin{macrocode}
\def\BKM@EscapeName#1{%
  \ifx#1\@empty
  \else
    \EdefEscapeName#1#1%
  \fi
}%
%    \end{macrocode}
%    \end{macro}
%    \begin{macro}{\BKM@EscapeString}
%    \begin{macrocode}
\def\BKM@EscapeString#1{%
  \ifx#1\@empty
  \else
    \EdefEscapeString#1#1%
  \fi
}%
%    \end{macrocode}
%    \end{macro}
%    \begin{macro}{\BKM@EscapeHex}
%    \begin{macrocode}
\def\BKM@EscapeHex#1{%
  \ifx#1\@empty
  \else
    \EdefEscapeHex#1#1%
  \fi
}%
%    \end{macrocode}
%    \end{macro}
%    \begin{macro}{\BKM@UnescapeHex}
%    \begin{macrocode}
\def\BKM@UnescapeHex#1{%
  \EdefUnescapeHex#1#1%
}%
%    \end{macrocode}
%    \end{macro}
%
% \paragraph{宏包(Packages)。}
%
% 不要加载由 \xpackage{hyperref}\ 加载的宏包
%    \begin{macrocode}
\RequirePackage{hyperref}[2010/06/18]
%    \end{macrocode}
%
% \subsubsection{宏包选项(Package options)}
%
%    \begin{macrocode}
\SetupKeyvalOptions{family=BKM,prefix=BKM@}
\DeclareLocalOptions{%
  atend,%
  bold,%
  color,%
  depth,%
  dest,%
  draft,%
  final,%
  gotor,%
  italic,%
  keeplevel,%
  level,%
  named,%
  numbered,%
  open,%
  openlevel,%
  page,%
  rawaction,%
  rellevel,%
  srcfile,%
  srcline,%
  startatroot,%
  uri,%
  view,%
}
%    \end{macrocode}
%    \begin{macro}{\bookmarksetup}
%    \begin{macrocode}
\newcommand*{\bookmarksetup}{\kvsetkeys{BKM}}
%    \end{macrocode}
%    \end{macro}
%    \begin{macro}{\BKM@setup}
%    \begin{macrocode}
\def\BKM@setup#1{%
  \bookmarksetup{#1}%
  \ifx\BKM@HookNext\ltx@empty
  \else
    \expandafter\bookmarksetup\expandafter{\BKM@HookNext}%
    \BKM@HookNextClear
  \fi
  \BKM@hook
  \ifBKM@keeplevel
  \else
    \xdef\BKM@currentlevel{\BKM@level}%
  \fi
}
%    \end{macrocode}
%    \end{macro}
%
%    \begin{macro}{\bookmarksetupnext}
%    \begin{macrocode}
\newcommand*{\bookmarksetupnext}[1]{%
  \ltx@GlobalAppendToMacro\BKM@HookNext{,#1}%
}
%    \end{macrocode}
%    \end{macro}
%    \begin{macro}{\BKM@setupnext}
%    \begin{macrocode}
%    \end{macrocode}
%    \end{macro}
%    \begin{macro}{\BKM@HookNextClear}
%    \begin{macrocode}
\def\BKM@HookNextClear{%
  \global\let\BKM@HookNext\ltx@empty
}
%    \end{macrocode}
%    \end{macro}
%    \begin{macro}{\BKM@HookNext}
%    \begin{macrocode}
\BKM@HookNextClear
%    \end{macrocode}
%    \end{macro}
%
%    \begin{macrocode}
\DeclareBoolOption{draft}
\DeclareComplementaryOption{final}{draft}
%    \end{macrocode}
%    \begin{macro}{\BKM@DisableOptions}
%    \begin{macrocode}
\def\BKM@DisableOptions{%
  \DisableKeyvalOption[action=warning,package=bookmark]%
      {BKM}{draft}%
  \DisableKeyvalOption[action=warning,package=bookmark]%
      {BKM}{final}%
}
%    \end{macrocode}
%    \end{macro}
%    \begin{macrocode}
\DeclareBoolOption[\ifHy@bookmarksopen true\else false\fi]{open}
%    \end{macrocode}
%    \begin{macro}{\bookmark@open}
%    \begin{macrocode}
\def\bookmark@open{%
  \ifBKM@open\ltx@one\else\ltx@zero\fi
}
%    \end{macrocode}
%    \end{macro}
%    \begin{macrocode}
\DeclareStringOption[\maxdimen]{openlevel}
%    \end{macrocode}
%    \begin{macro}{\BKM@openlevel}
%    \begin{macrocode}
\edef\BKM@openlevel{\number\@bookmarksopenlevel}
%    \end{macrocode}
%    \end{macro}
%    \begin{macrocode}
%\DeclareStringOption[\c@tocdepth]{depth}
\ltx@IfUndefined{Hy@bookmarksdepth}{%
  \def\BKM@depth{\c@tocdepth}%
}{%
  \let\BKM@depth\Hy@bookmarksdepth
}
\define@key{BKM}{depth}[]{%
  \edef\BKM@param{#1}%
  \ifx\BKM@param\@empty
    \def\BKM@depth{\c@tocdepth}%
  \else
    \ltx@IfUndefined{toclevel@\BKM@param}{%
      \@onelevel@sanitize\BKM@param
      \edef\BKM@temp{\expandafter\@car\BKM@param\@nil}%
      \ifcase 0\expandafter\ifx\BKM@temp-1\fi
              \expandafter\ifnum\expandafter`\BKM@temp>47 %
                \expandafter\ifnum\expandafter`\BKM@temp<58 %
                  1%
                \fi
              \fi
              \relax
        \PackageWarning{bookmark}{%
          Unknown document division name (\BKM@param)\MessageBreak
          for option `depth'%
        }%
      \else
        \BKM@SetDepthOrLevel\BKM@depth\BKM@param
      \fi
    }{%
      \BKM@SetDepthOrLevel\BKM@depth{%
        \csname toclevel@\BKM@param\endcsname
      }%
    }%
  \fi
}
%    \end{macrocode}
%    \begin{macro}{\bookmark@depth}
%    \begin{macrocode}
\def\bookmark@depth{\BKM@depth}
%    \end{macrocode}
%    \end{macro}
%    \begin{macro}{\BKM@SetDepthOrLevel}
%    \begin{macrocode}
\def\BKM@SetDepthOrLevel#1#2{%
  \begingroup
    \setbox\z@=\hbox{%
      \count@=#2\relax
      \expandafter
    }%
  \expandafter\endgroup
  \expandafter\def\expandafter#1\expandafter{\the\count@}%
}
%    \end{macrocode}
%    \end{macro}
%    \begin{macrocode}
\DeclareStringOption[\BKM@currentlevel]{level}[\BKM@currentlevel]
\define@key{BKM}{level}{%
  \edef\BKM@param{#1}%
  \ifx\BKM@param\BKM@MacroCurrentLevel
    \let\BKM@level\BKM@param
  \else
    \ltx@IfUndefined{toclevel@\BKM@param}{%
      \@onelevel@sanitize\BKM@param
      \edef\BKM@temp{\expandafter\@car\BKM@param\@nil}%
      \ifcase 0\expandafter\ifx\BKM@temp-1\fi
              \expandafter\ifnum\expandafter`\BKM@temp>47 %
                \expandafter\ifnum\expandafter`\BKM@temp<58 %
                  1%
                \fi
              \fi
              \relax
        \PackageWarning{bookmark}{%
          Unknown document division name (\BKM@param)\MessageBreak
          for option `level'%
        }%
      \else
        \BKM@SetDepthOrLevel\BKM@level\BKM@param
      \fi
    }{%
      \BKM@SetDepthOrLevel\BKM@level{%
        \csname toclevel@\BKM@param\endcsname
      }%
    }%
  \fi
}
%    \end{macrocode}
%    \begin{macro}{\BKM@MacroCurrentLevel}
%    \begin{macrocode}
\def\BKM@MacroCurrentLevel{\BKM@currentlevel}
%    \end{macrocode}
%    \end{macro}
%    \begin{macrocode}
\DeclareBoolOption{keeplevel}
\DeclareBoolOption{startatroot}
%    \end{macrocode}
%    \begin{macro}{\BKM@startatrootfalse}
%    \begin{macrocode}
\def\BKM@startatrootfalse{%
  \global\let\ifBKM@startatroot\iffalse
}
%    \end{macrocode}
%    \end{macro}
%    \begin{macro}{\BKM@startatroottrue}
%    \begin{macrocode}
\def\BKM@startatroottrue{%
  \global\let\ifBKM@startatroot\iftrue
}
%    \end{macrocode}
%    \end{macro}
%    \begin{macrocode}
\define@key{BKM}{rellevel}{%
  \BKM@CalcExpr\BKM@level{#1}+\BKM@currentlevel
}
%    \end{macrocode}
%    \begin{macro}{\bookmark@level}
%    \begin{macrocode}
\def\bookmark@level{\BKM@level}
%    \end{macrocode}
%    \end{macro}
%    \begin{macro}{\BKM@currentlevel}
%    \begin{macrocode}
\def\BKM@currentlevel{0}
%    \end{macrocode}
%    \end{macro}
%    Make \xpackage{bookmark}'s option \xoption{numbered} an alias
%    for \xpackage{hyperref}'s \xoption{bookmarksnumbered}.
%    \begin{macrocode}
\DeclareBoolOption[%
  \ifHy@bookmarksnumbered true\else false\fi
]{numbered}
\g@addto@macro\BKM@numberedtrue{%
  \let\ifHy@bookmarksnumbered\iftrue
}
\g@addto@macro\BKM@numberedfalse{%
  \let\ifHy@bookmarksnumbered\iffalse
}
\g@addto@macro\Hy@bookmarksnumberedtrue{%
  \let\ifBKM@numbered\iftrue
}
\g@addto@macro\Hy@bookmarksnumberedfalse{%
  \let\ifBKM@numbered\iffalse
}
%    \end{macrocode}
%    \begin{macro}{\bookmark@numbered}
%    \begin{macrocode}
\def\bookmark@numbered{%
  \ifBKM@numbered\ltx@one\else\ltx@zero\fi
}
%    \end{macrocode}
%    \end{macro}
%
% \paragraph{重定义 \xpackage{hyperref}\ 宏包的选项}
%
%    \begin{macro}{\BKM@PatchHyperrefOption}
%    \begin{macrocode}
\def\BKM@PatchHyperrefOption#1{%
  \expandafter\BKM@@PatchHyperrefOption\csname KV@Hyp@#1\endcsname%
}
%    \end{macrocode}
%    \end{macro}
%    \begin{macro}{\BKM@@PatchHyperrefOption}
%    \begin{macrocode}
\def\BKM@@PatchHyperrefOption#1{%
  \expandafter\BKM@@@PatchHyperrefOption#1{##1}\BKM@nil#1%
}
%    \end{macrocode}
%    \end{macro}
%    \begin{macro}{\BKM@@@PatchHyperrefOption}
%    \begin{macrocode}
\def\BKM@@@PatchHyperrefOption#1\BKM@nil#2#3{%
  \def#2##1{%
    #1%
    \bookmarksetup{#3={##1}}%
  }%
}
%    \end{macrocode}
%    \end{macro}
%    \begin{macrocode}
\BKM@PatchHyperrefOption{bookmarksopen}{open}
\BKM@PatchHyperrefOption{bookmarksopenlevel}{openlevel}
\BKM@PatchHyperrefOption{bookmarksdepth}{depth}
%    \end{macrocode}
%
% \paragraph{字体样式(font style)选项。}
%
%    注意:\xpackage{bitset}\ 宏是基于零的,PDF 规范(PDF specifications)以1开头。
%    \begin{macrocode}
\bitsetReset{BKM@FontStyle}%
\define@key{BKM}{italic}[true]{%
  \expandafter\ifx\csname if#1\endcsname\iftrue
    \bitsetSet{BKM@FontStyle}{0}%
  \else
    \bitsetClear{BKM@FontStyle}{0}%
  \fi
}%
\define@key{BKM}{bold}[true]{%
  \expandafter\ifx\csname if#1\endcsname\iftrue
    \bitsetSet{BKM@FontStyle}{1}%
  \else
    \bitsetClear{BKM@FontStyle}{1}%
  \fi
}%
%    \end{macrocode}
%    \begin{macro}{\bookmark@italic}
%    \begin{macrocode}
\def\bookmark@italic{%
  \ifnum\bitsetGet{BKM@FontStyle}{0}=1 \ltx@one\else\ltx@zero\fi
}
%    \end{macrocode}
%    \end{macro}
%    \begin{macro}{\bookmark@bold}
%    \begin{macrocode}
\def\bookmark@bold{%
  \ifnum\bitsetGet{BKM@FontStyle}{1}=1 \ltx@one\else\ltx@zero\fi
}
%    \end{macrocode}
%    \end{macro}
%    \begin{macro}{\BKM@PrintStyle}
%    \begin{macrocode}
\def\BKM@PrintStyle{%
  \bitsetGetDec{BKM@FontStyle}%
}%
%    \end{macrocode}
%    \end{macro}
%
% \paragraph{颜色(color)选项。}
%
%    \begin{macrocode}
\define@key{BKM}{color}{%
  \HyColor@BookmarkColor{#1}\BKM@color{bookmark}{color}%
}
%    \end{macrocode}
%    \begin{macro}{\BKM@color}
%    \begin{macrocode}
\let\BKM@color\@empty
%    \end{macrocode}
%    \end{macro}
%    \begin{macro}{\bookmark@color}
%    \begin{macrocode}
\def\bookmark@color{\BKM@color}
%    \end{macrocode}
%    \end{macro}
%
% \subsubsection{动作(action)选项}
%
%    \begin{macrocode}
\def\BKM@temp#1{%
  \DeclareStringOption{#1}%
  \expandafter\edef\csname bookmark@#1\endcsname{%
    \expandafter\noexpand\csname BKM@#1\endcsname
  }%
}
%    \end{macrocode}
%    \begin{macro}{\bookmark@dest}
%    \begin{macrocode}
\BKM@temp{dest}
%    \end{macrocode}
%    \end{macro}
%    \begin{macro}{\bookmark@named}
%    \begin{macrocode}
\BKM@temp{named}
%    \end{macrocode}
%    \end{macro}
%    \begin{macro}{\bookmark@uri}
%    \begin{macrocode}
\BKM@temp{uri}
%    \end{macrocode}
%    \end{macro}
%    \begin{macro}{\bookmark@gotor}
%    \begin{macrocode}
\BKM@temp{gotor}
%    \end{macrocode}
%    \end{macro}
%    \begin{macro}{\bookmark@rawaction}
%    \begin{macrocode}
\BKM@temp{rawaction}
%    \end{macrocode}
%    \end{macro}
%
%    \begin{macrocode}
\define@key{BKM}{page}{%
  \def\BKM@page{#1}%
  \ifx\BKM@page\@empty
  \else
    \edef\BKM@page{\number\BKM@page}%
    \ifnum\BKM@page>\z@
    \else
      \PackageError{bookmark}{Page must be positive}\@ehc
      \def\BKM@page{1}%
    \fi
  \fi
}
%    \end{macrocode}
%    \begin{macro}{\BKM@page}
%    \begin{macrocode}
\let\BKM@page\@empty
%    \end{macrocode}
%    \end{macro}
%    \begin{macro}{\bookmark@page}
%    \begin{macrocode}
\def\bookmark@page{\BKM@@page}
%    \end{macrocode}
%    \end{macro}
%
%    \begin{macrocode}
\define@key{BKM}{view}{%
  \BKM@CheckView{#1}%
}
%    \end{macrocode}
%    \begin{macro}{\BKM@view}
%    \begin{macrocode}
\let\BKM@view\@empty
%    \end{macrocode}
%    \end{macro}
%    \begin{macro}{\bookmark@view}
%    \begin{macrocode}
\def\bookmark@view{\BKM@view}
%    \end{macrocode}
%    \end{macro}
%    \begin{macro}{BKM@CheckView}
%    \begin{macrocode}
\def\BKM@CheckView#1{%
  \BKM@CheckViewType#1 \@nil
}
%    \end{macrocode}
%    \end{macro}
%    \begin{macro}{\BKM@CheckViewType}
%    \begin{macrocode}
\def\BKM@CheckViewType#1 #2\@nil{%
  \def\BKM@type{#1}%
  \@onelevel@sanitize\BKM@type
  \BKM@TestViewType{Fit}{}%
  \BKM@TestViewType{FitB}{}%
  \BKM@TestViewType{FitH}{%
    \BKM@CheckParam#2 \@nil{top}%
  }%
  \BKM@TestViewType{FitBH}{%
    \BKM@CheckParam#2 \@nil{top}%
  }%
  \BKM@TestViewType{FitV}{%
    \BKM@CheckParam#2 \@nil{bottom}%
  }%
  \BKM@TestViewType{FitBV}{%
    \BKM@CheckParam#2 \@nil{bottom}%
  }%
  \BKM@TestViewType{FitR}{%
    \BKM@CheckRect{#2}{ }%
  }%
  \BKM@TestViewType{XYZ}{%
    \BKM@CheckXYZ{#2}{ }%
  }%
  \@car{%
    \PackageError{bookmark}{%
      Unknown view type `\BKM@type',\MessageBreak
      using `FitH' instead%
    }\@ehc
    \def\BKM@view{FitH}%
  }%
  \@nil
}
%    \end{macrocode}
%    \end{macro}
%    \begin{macro}{\BKM@TestViewType}
%    \begin{macrocode}
\def\BKM@TestViewType#1{%
  \def\BKM@temp{#1}%
  \@onelevel@sanitize\BKM@temp
  \ifx\BKM@type\BKM@temp
    \let\BKM@view\BKM@temp
    \expandafter\@car
  \else
    \expandafter\@gobble
  \fi
}
%    \end{macrocode}
%    \end{macro}
%    \begin{macro}{BKM@CheckParam}
%    \begin{macrocode}
\def\BKM@CheckParam#1 #2\@nil#3{%
  \def\BKM@param{#1}%
  \ifx\BKM@param\@empty
    \PackageWarning{bookmark}{%
      Missing parameter (#3) for `\BKM@type',\MessageBreak
      using 0%
    }%
    \def\BKM@param{0}%
  \else
    \BKM@CalcParam
  \fi
  \edef\BKM@view{\BKM@view\space\BKM@param}%
}
%    \end{macrocode}
%    \end{macro}
%    \begin{macro}{BKM@CheckRect}
%    \begin{macrocode}
\def\BKM@CheckRect#1#2{%
  \BKM@@CheckRect#1#2#2#2#2\@nil
}
%    \end{macrocode}
%    \end{macro}
%    \begin{macro}{\BKM@@CheckRect}
%    \begin{macrocode}
\def\BKM@@CheckRect#1 #2 #3 #4 #5\@nil{%
  \def\BKM@temp{0}%
  \def\BKM@param{#1}%
  \ifx\BKM@param\@empty
    \def\BKM@param{0}%
    \def\BKM@temp{1}%
  \else
    \BKM@CalcParam
  \fi
  \edef\BKM@view{\BKM@view\space\BKM@param}%
  \def\BKM@param{#2}%
  \ifx\BKM@param\@empty
    \def\BKM@param{0}%
    \def\BKM@temp{1}%
  \else
    \BKM@CalcParam
  \fi
  \edef\BKM@view{\BKM@view\space\BKM@param}%
  \def\BKM@param{#3}%
  \ifx\BKM@param\@empty
    \def\BKM@param{0}%
    \def\BKM@temp{1}%
  \else
    \BKM@CalcParam
  \fi
  \edef\BKM@view{\BKM@view\space\BKM@param}%
  \def\BKM@param{#4}%
  \ifx\BKM@param\@empty
    \def\BKM@param{0}%
    \def\BKM@temp{1}%
  \else
    \BKM@CalcParam
  \fi
  \edef\BKM@view{\BKM@view\space\BKM@param}%
  \ifnum\BKM@temp>\z@
    \PackageWarning{bookmark}{Missing parameters for `\BKM@type'}%
  \fi
}
%    \end{macrocode}
%    \end{macro}
%    \begin{macro}{\BKM@CheckXYZ}
%    \begin{macrocode}
\def\BKM@CheckXYZ#1#2{%
  \BKM@@CheckXYZ#1#2#2#2\@nil
}
%    \end{macrocode}
%    \end{macro}
%    \begin{macro}{\BKM@@CheckXYZ}
%    \begin{macrocode}
\def\BKM@@CheckXYZ#1 #2 #3 #4\@nil{%
  \def\BKM@param{#1}%
  \let\BKM@temp\BKM@param
  \@onelevel@sanitize\BKM@temp
  \ifx\BKM@param\@empty
    \let\BKM@param\BKM@null
  \else
    \ifx\BKM@temp\BKM@null
    \else
      \BKM@CalcParam
    \fi
  \fi
  \edef\BKM@view{\BKM@view\space\BKM@param}%
  \def\BKM@param{#2}%
  \let\BKM@temp\BKM@param
  \@onelevel@sanitize\BKM@temp
  \ifx\BKM@param\@empty
    \let\BKM@param\BKM@null
  \else
    \ifx\BKM@temp\BKM@null
    \else
      \BKM@CalcParam
    \fi
  \fi
  \edef\BKM@view{\BKM@view\space\BKM@param}%
  \def\BKM@param{#3}%
  \ifx\BKM@param\@empty
    \let\BKM@param\BKM@null
  \fi
  \edef\BKM@view{\BKM@view\space\BKM@param}%
}
%    \end{macrocode}
%    \end{macro}
%    \begin{macro}{\BKM@null}
%    \begin{macrocode}
\def\BKM@null{null}
\@onelevel@sanitize\BKM@null
%    \end{macrocode}
%    \end{macro}
%
%    \begin{macro}{\BKM@CalcParam}
%    \begin{macrocode}
\def\BKM@CalcParam{%
  \begingroup
  \let\calc\@firstofone
  \expandafter\BKM@@CalcParam\BKM@param\@empty\@empty\@nil
}
%    \end{macrocode}
%    \end{macro}
%    \begin{macro}{\BKM@@CalcParam}
%    \begin{macrocode}
\def\BKM@@CalcParam#1#2#3\@nil{%
  \ifx\calc#1%
    \@ifundefined{calc@assign@dimen}{%
      \@ifundefined{dimexpr}{%
        \setlength{\dimen@}{#2}%
      }{%
        \setlength{\dimen@}{\dimexpr#2\relax}%
      }%
    }{%
      \setlength{\dimen@}{#2}%
    }%
    \dimen@.99626\dimen@
    \edef\BKM@param{\strip@pt\dimen@}%
    \expandafter\endgroup
    \expandafter\def\expandafter\BKM@param\expandafter{\BKM@param}%
  \else
    \endgroup
  \fi
}
%    \end{macrocode}
%    \end{macro}
%
% \subsubsection{\xoption{atend}\ 选项}
%
%    \begin{macrocode}
\DeclareBoolOption{atend}
\g@addto@macro\BKM@DisableOptions{%
  \DisableKeyvalOption[action=warning,package=bookmark]%
      {BKM}{atend}%
}
%    \end{macrocode}
%
% \subsubsection{\xoption{style}\ 选项}
%
%    \begin{macro}{\bookmarkdefinestyle}
%    \begin{macrocode}
\newcommand*{\bookmarkdefinestyle}[2]{%
  \@ifundefined{BKM@style@#1}{%
  }{%
    \PackageInfo{bookmark}{Redefining style `#1'}%
  }%
  \@namedef{BKM@style@#1}{#2}%
}
%    \end{macrocode}
%    \end{macro}
%    \begin{macrocode}
\define@key{BKM}{style}{%
  \BKM@StyleCall{#1}%
}
\newif\ifBKM@ok
%    \end{macrocode}
%    \begin{macro}{\BKM@StyleCall}
%    \begin{macrocode}
\def\BKM@StyleCall#1{%
  \@ifundefined{BKM@style@#1}{%
    \PackageWarning{bookmark}{%
      Ignoring unknown style `#1'%
    }%
  }{%
%    \end{macrocode}
%    检查样式堆栈(style stack)。
%    \begin{macrocode}
    \BKM@oktrue
    \edef\BKM@StyleCurrent{#1}%
    \@onelevel@sanitize\BKM@StyleCurrent
    \let\BKM@StyleEntry\BKM@StyleEntryCheck
    \BKM@StyleStack
    \ifBKM@ok
      \expandafter\@firstofone
    \else
      \PackageError{bookmark}{%
        Ignoring recursive call of style `\BKM@StyleCurrent'%
      }\@ehc
      \expandafter\@gobble
    \fi
    {%
%    \end{macrocode}
%    在堆栈上推送当前样式(Push current style on stack)。
%    \begin{macrocode}
      \let\BKM@StyleEntry\relax
      \edef\BKM@StyleStack{%
        \BKM@StyleEntry{\BKM@StyleCurrent}%
        \BKM@StyleStack
      }%
%    \end{macrocode}
%   调用样式(Call style)。
%    \begin{macrocode}
      \expandafter\expandafter\expandafter\bookmarksetup
      \expandafter\expandafter\expandafter{%
        \csname BKM@style@\BKM@StyleCurrent\endcsname
      }%
%    \end{macrocode}
%    从堆栈中弹出当前样式(Pop current style from stack)。
%    \begin{macrocode}
      \BKM@StyleStackPop
    }%
  }%
}
%    \end{macrocode}
%    \end{macro}
%    \begin{macro}{\BKM@StyleStackPop}
%    \begin{macrocode}
\def\BKM@StyleStackPop{%
  \let\BKM@StyleEntry\relax
  \edef\BKM@StyleStack{%
    \expandafter\@gobbletwo\BKM@StyleStack
  }%
}
%    \end{macrocode}
%    \end{macro}
%    \begin{macro}{\BKM@StyleEntryCheck}
%    \begin{macrocode}
\def\BKM@StyleEntryCheck#1{%
  \def\BKM@temp{#1}%
  \ifx\BKM@temp\BKM@StyleCurrent
    \BKM@okfalse
  \fi
}
%    \end{macrocode}
%    \end{macro}
%    \begin{macro}{\BKM@StyleStack}
%    \begin{macrocode}
\def\BKM@StyleStack{}
%    \end{macrocode}
%    \end{macro}
%
% \subsubsection{源文件位置(source file location)选项}
%
%    \begin{macrocode}
\DeclareStringOption{srcline}
\DeclareStringOption{srcfile}
%    \end{macrocode}
%
% \subsubsection{钩子支持(Hook support)}
%
%    \begin{macro}{\BKM@hook}
%    \begin{macrocode}
\def\BKM@hook{}
%    \end{macrocode}
%    \end{macro}
%    \begin{macrocode}
\define@key{BKM}{addtohook}{%
  \ltx@LocalAppendToMacro\BKM@hook{#1}%
}
%    \end{macrocode}
%
%    \begin{macro}{bookmarkget}
%    \begin{macrocode}
\newcommand*{\bookmarkget}[1]{%
  \romannumeral0%
  \ltx@ifundefined{bookmark@#1}{%
    \ltx@space
  }{%
    \expandafter\expandafter\expandafter\ltx@space
    \csname bookmark@#1\endcsname
  }%
}
%    \end{macrocode}
%    \end{macro}
%
% \subsubsection{设置和加载驱动程序}
%
% \paragraph{检测驱动程序。}
%
%    \begin{macro}{\BKM@DefineDriverKey}
%    \begin{macrocode}
\def\BKM@DefineDriverKey#1{%
  \define@key{BKM}{#1}[]{%
    \def\BKM@driver{#1}%
  }%
  \g@addto@macro\BKM@DisableOptions{%
    \DisableKeyvalOption[action=warning,package=bookmark]%
        {BKM}{#1}%
  }%
}
%    \end{macrocode}
%    \end{macro}
%    \begin{macrocode}
\BKM@DefineDriverKey{pdftex}
\BKM@DefineDriverKey{dvips}
\BKM@DefineDriverKey{dvipdfm}
\BKM@DefineDriverKey{dvipdfmx}
\BKM@DefineDriverKey{xetex}
\BKM@DefineDriverKey{vtex}
\define@key{BKM}{dvipdfmx-outline-open}[true]{%
 \PackageWarning{bookmark}{Option 'dvipdfmx-outline-open' is obsolete
   and ignored}{}}
%    \end{macrocode}
%    \begin{macro}{\bookmark@driver}
%    \begin{macrocode}
\def\bookmark@driver{\BKM@driver}
%    \end{macrocode}
%    \end{macro}
%    \begin{macrocode}
\InputIfFileExists{bookmark.cfg}{}{}
%    \end{macrocode}
%    \begin{macro}{\BookmarkDriverDefault}
%    \begin{macrocode}
\providecommand*{\BookmarkDriverDefault}{dvips}
%    \end{macrocode}
%    \end{macro}
%    \begin{macro}{\BKM@driver}
% Lua\TeX\ 和 pdf\TeX\ 共享驱动程序。
%    \begin{macrocode}
\ifpdf
  \def\BKM@driver{pdftex}%
  \ifx\pdfoutline\@undefined
    \ifx\pdfextension\@undefined\else
      \protected\def\pdfoutline{\pdfextension outline }
    \fi
  \fi
\else
  \ifxetex
    \def\BKM@driver{dvipdfm}%
  \else
    \ifvtex
      \def\BKM@driver{vtex}%
    \else
      \edef\BKM@driver{\BookmarkDriverDefault}%
    \fi
  \fi
\fi
%    \end{macrocode}
%    \end{macro}
%
% \paragraph{过程选项(Process options)。}
%
%    \begin{macrocode}
\ProcessKeyvalOptions*
\BKM@DisableOptions
%    \end{macrocode}
%
% \paragraph{\xoption{draft}\ 选项}
%
%    \begin{macrocode}
\ifBKM@draft
  \PackageWarningNoLine{bookmark}{Draft mode on}%
  \let\bookmarksetup\ltx@gobble
  \let\BookmarkAtEnd\ltx@gobble
  \let\bookmarkdefinestyle\ltx@gobbletwo
  \let\bookmarkget\ltx@gobble
  \let\pdfbookmark\ltx@undefined
  \newcommand*{\pdfbookmark}[3][]{}%
  \let\currentpdfbookmark\ltx@gobbletwo
  \let\subpdfbookmark\ltx@gobbletwo
  \let\belowpdfbookmark\ltx@gobbletwo
  \newcommand*{\bookmark}[2][]{}%
  \renewcommand*{\Hy@writebookmark}[5]{}%
  \let\ReadBookmarks\relax
  \let\BKM@DefGotoNameAction\ltx@gobbletwo % package `hypdestopt'
  \expandafter\endinput
\fi
%    \end{macrocode}
%
% \paragraph{验证和加载驱动程序。}
%
%    \begin{macrocode}
\def\BKM@temp{dvipdfmx}%
\ifx\BKM@temp\BKM@driver
  \def\BKM@driver{dvipdfm}%
\fi
\def\BKM@temp{pdftex}%
\ifpdf
  \ifx\BKM@temp\BKM@driver
  \else
    \PackageWarningNoLine{bookmark}{%
      Wrong driver `\BKM@driver', using `pdftex' instead%
    }%
    \let\BKM@driver\BKM@temp
  \fi
\else
  \ifx\BKM@temp\BKM@driver
    \PackageError{bookmark}{%
      Wrong driver, pdfTeX is not running in PDF mode.\MessageBreak
      Package loading is aborted%
    }\@ehc
    \expandafter\expandafter\expandafter\endinput
  \fi
  \def\BKM@temp{dvipdfm}%
  \ifxetex
    \ifx\BKM@temp\BKM@driver
    \else
      \PackageWarningNoLine{bookmark}{%
        Wrong driver `\BKM@driver',\MessageBreak
        using `dvipdfm' for XeTeX instead%
      }%
      \let\BKM@driver\BKM@temp
    \fi
  \else
    \def\BKM@temp{vtex}%
    \ifvtex
      \ifx\BKM@temp\BKM@driver
      \else
        \PackageWarningNoLine{bookmark}{%
          Wrong driver `\BKM@driver',\MessageBreak
          using `vtex' for VTeX instead%
        }%
        \let\BKM@driver\BKM@temp
      \fi
    \else
      \ifx\BKM@temp\BKM@driver
        \PackageError{bookmark}{%
          Wrong driver, VTeX is not running in PDF mode.\MessageBreak
          Package loading is aborted%
        }\@ehc
        \expandafter\expandafter\expandafter\endinput
      \fi
    \fi
  \fi
\fi
\providecommand\IfFormatAtLeastTF{\@ifl@t@r\fmtversion}
\IfFormatAtLeastTF{2020/10/01}{}{\edef\BKM@driver{\BKM@driver-2019-12-03}}
\InputIfFileExists{bkm-\BKM@driver.def}{}{%
  \PackageError{bookmark}{%
    Unsupported driver `\BKM@driver'.\MessageBreak
    Package loading is aborted%
  }\@ehc
  \endinput
}
%    \end{macrocode}
%
% \subsubsection{与 \xpackage{hyperref}\ 的兼容性}
%
%    \begin{macro}{\pdfbookmark}
%    \begin{macrocode}
\let\pdfbookmark\ltx@undefined
\newcommand*{\pdfbookmark}[3][0]{%
  \bookmark[level=#1,dest={#3.#1}]{#2}%
  \hyper@anchorstart{#3.#1}\hyper@anchorend
}
%    \end{macrocode}
%    \end{macro}
%    \begin{macro}{\currentpdfbookmark}
%    \begin{macrocode}
\def\currentpdfbookmark{%
  \pdfbookmark[\BKM@currentlevel]%
}
%    \end{macrocode}
%    \end{macro}
%    \begin{macro}{\subpdfbookmark}
%    \begin{macrocode}
\def\subpdfbookmark{%
  \BKM@CalcExpr\BKM@CalcResult\BKM@currentlevel+1%
  \expandafter\pdfbookmark\expandafter[\BKM@CalcResult]%
}
%    \end{macrocode}
%    \end{macro}
%    \begin{macro}{\belowpdfbookmark}
%    \begin{macrocode}
\def\belowpdfbookmark#1#2{%
  \xdef\BKM@gtemp{\number\BKM@currentlevel}%
  \subpdfbookmark{#1}{#2}%
  \global\let\BKM@currentlevel\BKM@gtemp
}
%    \end{macrocode}
%    \end{macro}
%
%    节号(section number)、文本(text)、标签(label)、级别(level)、文件(file)
%    \begin{macro}{\Hy@writebookmark}
%    \begin{macrocode}
\def\Hy@writebookmark#1#2#3#4#5{%
  \ifnum#4>\BKM@depth\relax
  \else
    \def\BKM@type{#5}%
    \ifx\BKM@type\Hy@bookmarkstype
      \begingroup
        \ifBKM@numbered
          \let\numberline\Hy@numberline
          \let\booknumberline\Hy@numberline
          \let\partnumberline\Hy@numberline
          \let\chapternumberline\Hy@numberline
        \else
          \let\numberline\@gobble
          \let\booknumberline\@gobble
          \let\partnumberline\@gobble
          \let\chapternumberline\@gobble
        \fi
        \bookmark[level=#4,dest={\HyperDestNameFilter{#3}}]{#2}%
      \endgroup
    \fi
  \fi
}
%    \end{macrocode}
%    \end{macro}
%
%    \begin{macro}{\ReadBookmarks}
%    \begin{macrocode}
\let\ReadBookmarks\relax
%    \end{macrocode}
%    \end{macro}
%
%    \begin{macrocode}
%</package>
%    \end{macrocode}
%
% \subsection{dvipdfm 的驱动程序}
%
%    \begin{macrocode}
%<*dvipdfm>
\NeedsTeXFormat{LaTeX2e}
\ProvidesFile{bkm-dvipdfm.def}%
  [2020-11-06 v1.29 bookmark driver for dvipdfm (HO)]%
%    \end{macrocode}
%
%    \begin{macro}{\BKM@id}
%    \begin{macrocode}
\newcount\BKM@id
\BKM@id=\z@
%    \end{macrocode}
%    \end{macro}
%
%    \begin{macro}{\BKM@0}
%    \begin{macrocode}
\@namedef{BKM@0}{000}
%    \end{macrocode}
%    \end{macro}
%    \begin{macro}{\ifBKM@sw}
%    \begin{macrocode}
\newif\ifBKM@sw
%    \end{macrocode}
%    \end{macro}
%
%    \begin{macro}{\bookmark}
%    \begin{macrocode}
\newcommand*{\bookmark}[2][]{%
  \if@filesw
    \begingroup
      \def\bookmark@text{#2}%
      \BKM@setup{#1}%
      \edef\BKM@prev{\the\BKM@id}%
      \global\advance\BKM@id\@ne
      \BKM@swtrue
      \@whilesw\ifBKM@sw\fi{%
        \def\BKM@abslevel{1}%
        \ifnum\ifBKM@startatroot\z@\else\BKM@prev\fi=\z@
          \BKM@startatrootfalse
          \expandafter\xdef\csname BKM@\the\BKM@id\endcsname{%
            0{\BKM@level}\BKM@abslevel
          }%
          \BKM@swfalse
        \else
          \expandafter\expandafter\expandafter\BKM@getx
              \csname BKM@\BKM@prev\endcsname
          \ifnum\BKM@level>\BKM@x@level\relax
            \BKM@CalcExpr\BKM@abslevel\BKM@x@abslevel+1%
            \expandafter\xdef\csname BKM@\the\BKM@id\endcsname{%
              {\BKM@prev}{\BKM@level}\BKM@abslevel
            }%
            \BKM@swfalse
          \else
            \let\BKM@prev\BKM@x@parent
          \fi
        \fi
      }%
      \csname HyPsd@XeTeXBigCharstrue\endcsname
      \pdfstringdef\BKM@title{\bookmark@text}%
      \edef\BKM@FLAGS{\BKM@PrintStyle}%
      \let\BKM@action\@empty
      \ifx\BKM@gotor\@empty
        \ifx\BKM@dest\@empty
          \ifx\BKM@named\@empty
            \ifx\BKM@rawaction\@empty
              \ifx\BKM@uri\@empty
                \ifx\BKM@page\@empty
                  \PackageError{bookmark}{Missing action}\@ehc
                  \edef\BKM@action{/Dest[@page1/Fit]}%
                \else
                  \ifx\BKM@view\@empty
                    \def\BKM@view{Fit}%
                  \fi
                  \edef\BKM@action{/Dest[@page\BKM@page/\BKM@view]}%
                \fi
              \else
                \BKM@EscapeString\BKM@uri
                \edef\BKM@action{%
                  /A<<%
                    /S/URI%
                    /URI(\BKM@uri)%
                  >>%
                }%
              \fi
            \else
              \edef\BKM@action{/A<<\BKM@rawaction>>}%
            \fi
          \else
            \BKM@EscapeName\BKM@named
            \edef\BKM@action{%
              /A<</S/Named/N/\BKM@named>>%
            }%
          \fi
        \else
          \BKM@EscapeString\BKM@dest
          \edef\BKM@action{%
            /A<<%
              /S/GoTo%
              /D(\BKM@dest)%
            >>%
          }%
        \fi
      \else
        \ifx\BKM@dest\@empty
          \ifx\BKM@page\@empty
            \def\BKM@page{0}%
          \else
            \BKM@CalcExpr\BKM@page\BKM@page-1%
          \fi
          \ifx\BKM@view\@empty
            \def\BKM@view{Fit}%
          \fi
          \edef\BKM@action{/D[\BKM@page/\BKM@view]}%
        \else
          \BKM@EscapeString\BKM@dest
          \edef\BKM@action{/D(\BKM@dest)}%
        \fi
        \BKM@EscapeString\BKM@gotor
        \edef\BKM@action{%
          /A<<%
            /S/GoToR%
            /F(\BKM@gotor)%
            \BKM@action
          >>%
        }%
      \fi
      \special{pdf:%
        out
              [%
              \ifBKM@open
                \ifnum\BKM@level<%
                    \expandafter\ltx@firstofone\expandafter
                    {\number\BKM@openlevel} %
                \else
                  -%
                \fi
              \else
                -%
              \fi
              ] %
            \BKM@abslevel
        <<%
          /Title(\BKM@title)%
          \ifx\BKM@color\@empty
          \else
            /C[\BKM@color]%
          \fi
          \ifnum\BKM@FLAGS>\z@
            /F \BKM@FLAGS
          \fi
          \BKM@action
        >>%
      }%
    \endgroup
  \fi
}
%    \end{macrocode}
%    \end{macro}
%    \begin{macro}{\BKM@getx}
%    \begin{macrocode}
\def\BKM@getx#1#2#3{%
  \def\BKM@x@parent{#1}%
  \def\BKM@x@level{#2}%
  \def\BKM@x@abslevel{#3}%
}
%    \end{macrocode}
%    \end{macro}
%
%    \begin{macrocode}
%</dvipdfm>
%    \end{macrocode}
%
% \subsection{\hologo{VTeX}\ 的驱动程序}
%
%    \begin{macrocode}
%<*vtex>
\NeedsTeXFormat{LaTeX2e}
\ProvidesFile{bkm-vtex.def}%
  [2020-11-06 v1.29 bookmark driver for VTeX (HO)]%
%    \end{macrocode}
%
%    \begin{macrocode}
\ifvtexpdf
\else
  \PackageWarningNoLine{bookmark}{%
    The VTeX driver only supports PDF mode%
  }%
\fi
%    \end{macrocode}
%
%    \begin{macro}{\BKM@id}
%    \begin{macrocode}
\newcount\BKM@id
\BKM@id=\z@
%    \end{macrocode}
%    \end{macro}
%
%    \begin{macro}{\BKM@0}
%    \begin{macrocode}
\@namedef{BKM@0}{00}
%    \end{macrocode}
%    \end{macro}
%    \begin{macro}{\ifBKM@sw}
%    \begin{macrocode}
\newif\ifBKM@sw
%    \end{macrocode}
%    \end{macro}
%
%    \begin{macro}{\bookmark}
%    \begin{macrocode}
\newcommand*{\bookmark}[2][]{%
  \if@filesw
    \begingroup
      \def\bookmark@text{#2}%
      \BKM@setup{#1}%
      \edef\BKM@prev{\the\BKM@id}%
      \global\advance\BKM@id\@ne
      \BKM@swtrue
      \@whilesw\ifBKM@sw\fi{%
        \ifnum\ifBKM@startatroot\z@\else\BKM@prev\fi=\z@
          \BKM@startatrootfalse
          \def\BKM@parent{0}%
          \expandafter\xdef\csname BKM@\the\BKM@id\endcsname{%
            0{\BKM@level}%
          }%
          \BKM@swfalse
        \else
          \expandafter\expandafter\expandafter\BKM@getx
              \csname BKM@\BKM@prev\endcsname
          \ifnum\BKM@level>\BKM@x@level\relax
            \let\BKM@parent\BKM@prev
            \expandafter\xdef\csname BKM@\the\BKM@id\endcsname{%
              {\BKM@prev}{\BKM@level}%
            }%
            \BKM@swfalse
          \else
            \let\BKM@prev\BKM@x@parent
          \fi
        \fi
      }%
      \pdfstringdef\BKM@title{\bookmark@text}%
      \BKM@vtex@title
      \edef\BKM@FLAGS{\BKM@PrintStyle}%
      \let\BKM@action\@empty
      \ifx\BKM@gotor\@empty
        \ifx\BKM@dest\@empty
          \ifx\BKM@named\@empty
            \ifx\BKM@rawaction\@empty
              \ifx\BKM@uri\@empty
                \ifx\BKM@page\@empty
                  \PackageError{bookmark}{Missing action}\@ehc
                  \def\BKM@action{!1}%
                \else
                  \edef\BKM@action{!\BKM@page}%
                \fi
              \else
                \BKM@EscapeString\BKM@uri
                \edef\BKM@action{%
                  <u=%
                    /S/URI%
                    /URI(\BKM@uri)%
                  >%
                }%
              \fi
            \else
              \edef\BKM@action{<u=\BKM@rawaction>}%
            \fi
          \else
            \BKM@EscapeName\BKM@named
            \edef\BKM@action{%
              <u=%
                /S/Named%
                /N/\BKM@named
              >%
            }%
          \fi
        \else
          \BKM@EscapeString\BKM@dest
          \edef\BKM@action{\BKM@dest}%
        \fi
      \else
        \ifx\BKM@dest\@empty
          \ifx\BKM@page\@empty
            \def\BKM@page{1}%
          \fi
          \ifx\BKM@view\@empty
            \def\BKM@view{Fit}%
          \fi
          \edef\BKM@action{/D[\BKM@page/\BKM@view]}%
        \else
          \BKM@EscapeString\BKM@dest
          \edef\BKM@action{/D(\BKM@dest)}%
        \fi
        \BKM@EscapeString\BKM@gotor
        \edef\BKM@action{%
          <u=%
            /S/GoToR%
            /F(\BKM@gotor)%
            \BKM@action
          >>%
        }%
      \fi
      \ifx\BKM@color\@empty
        \let\BKM@RGBcolor\@empty
      \else
        \expandafter\BKM@toRGB\BKM@color\@nil
      \fi
      \special{%
        !outline \BKM@action;%
        p=\BKM@parent,%
        i=\number\BKM@id,%
        s=%
          \ifBKM@open
            \ifnum\BKM@level<\BKM@openlevel
              o%
            \else
              c%
            \fi
          \else
            c%
          \fi,%
        \ifx\BKM@RGBcolor\@empty
        \else
          c=\BKM@RGBcolor,%
        \fi
        \ifnum\BKM@FLAGS>\z@
          f=\BKM@FLAGS,%
        \fi
        t=\BKM@title
      }%
    \endgroup
  \fi
}
%    \end{macrocode}
%    \end{macro}
%    \begin{macro}{\BKM@getx}
%    \begin{macrocode}
\def\BKM@getx#1#2{%
  \def\BKM@x@parent{#1}%
  \def\BKM@x@level{#2}%
}
%    \end{macrocode}
%    \end{macro}
%    \begin{macro}{\BKM@toRGB}
%    \begin{macrocode}
\def\BKM@toRGB#1 #2 #3\@nil{%
  \let\BKM@RGBcolor\@empty
  \BKM@toRGBComponent{#1}%
  \BKM@toRGBComponent{#2}%
  \BKM@toRGBComponent{#3}%
}
%    \end{macrocode}
%    \end{macro}
%    \begin{macro}{\BKM@toRGBComponent}
%    \begin{macrocode}
\def\BKM@toRGBComponent#1{%
  \dimen@=#1pt\relax
  \ifdim\dimen@>\z@
    \ifdim\dimen@<\p@
      \dimen@=255\dimen@
      \advance\dimen@ by 32768sp\relax
      \divide\dimen@ by 65536\relax
      \dimen@ii=\dimen@
      \divide\dimen@ii by 16\relax
      \edef\BKM@RGBcolor{%
        \BKM@RGBcolor
        \BKM@toHexDigit\dimen@ii
      }%
      \dimen@ii=16\dimen@ii
      \advance\dimen@-\dimen@ii
      \edef\BKM@RGBcolor{%
        \BKM@RGBcolor
        \BKM@toHexDigit\dimen@
      }%
    \else
      \edef\BKM@RGBcolor{\BKM@RGBcolor FF}%
    \fi
  \else
    \edef\BKM@RGBcolor{\BKM@RGBcolor00}%
  \fi
}
%    \end{macrocode}
%    \end{macro}
%    \begin{macro}{\BKM@toHexDigit}
%    \begin{macrocode}
\def\BKM@toHexDigit#1{%
  \ifcase\expandafter\@firstofone\expandafter{\number#1} %
    0\or 1\or 2\or 3\or 4\or 5\or 6\or 7\or
    8\or 9\or A\or B\or C\or D\or E\or F%
  \fi
}
%    \end{macrocode}
%    \end{macro}
%    \begin{macrocode}
\begingroup
  \catcode`\|=0 %
  \catcode`\\=12 %
%    \end{macrocode}
%    \begin{macro}{\BKM@vtex@title}
%    \begin{macrocode}
  |gdef|BKM@vtex@title{%
    |@onelevel@sanitize|BKM@title
    |edef|BKM@title{|expandafter|BKM@vtex@leftparen|BKM@title\(|@nil}%
    |edef|BKM@title{|expandafter|BKM@vtex@rightparen|BKM@title\)|@nil}%
    |edef|BKM@title{|expandafter|BKM@vtex@zero|BKM@title\0|@nil}%
    |edef|BKM@title{|expandafter|BKM@vtex@one|BKM@title\1|@nil}%
    |edef|BKM@title{|expandafter|BKM@vtex@two|BKM@title\2|@nil}%
    |edef|BKM@title{|expandafter|BKM@vtex@three|BKM@title\3|@nil}%
  }%
%    \end{macrocode}
%    \end{macro}
%    \begin{macro}{\BKM@vtex@leftparen}
%    \begin{macrocode}
  |gdef|BKM@vtex@leftparen#1\(#2|@nil{%
    #1%
    |ifx||#2||%
    |else
      (%
      |ltx@ReturnAfterFi{%
        |BKM@vtex@leftparen#2|@nil
      }%
    |fi
  }%
%    \end{macrocode}
%    \end{macro}
%    \begin{macro}{\BKM@vtex@rightparen}
%    \begin{macrocode}
  |gdef|BKM@vtex@rightparen#1\)#2|@nil{%
    #1%
    |ifx||#2||%
    |else
      )%
      |ltx@ReturnAfterFi{%
        |BKM@vtex@rightparen#2|@nil
      }%
    |fi
  }%
%    \end{macrocode}
%    \end{macro}
%    \begin{macro}{\BKM@vtex@zero}
%    \begin{macrocode}
  |gdef|BKM@vtex@zero#1\0#2|@nil{%
    #1%
    |ifx||#2||%
    |else
      |noexpand|hv@pdf@char0%
      |ltx@ReturnAfterFi{%
        |BKM@vtex@zero#2|@nil
      }%
    |fi
  }%
%    \end{macrocode}
%    \end{macro}
%    \begin{macro}{\BKM@vtex@one}
%    \begin{macrocode}
  |gdef|BKM@vtex@one#1\1#2|@nil{%
    #1%
    |ifx||#2||%
    |else
      |noexpand|hv@pdf@char1%
      |ltx@ReturnAfterFi{%
        |BKM@vtex@one#2|@nil
      }%
    |fi
  }%
%    \end{macrocode}
%    \end{macro}
%    \begin{macro}{\BKM@vtex@two}
%    \begin{macrocode}
  |gdef|BKM@vtex@two#1\2#2|@nil{%
    #1%
    |ifx||#2||%
    |else
      |noexpand|hv@pdf@char2%
      |ltx@ReturnAfterFi{%
        |BKM@vtex@two#2|@nil
      }%
    |fi
  }%
%    \end{macrocode}
%    \end{macro}
%    \begin{macro}{\BKM@vtex@three}
%    \begin{macrocode}
  |gdef|BKM@vtex@three#1\3#2|@nil{%
    #1%
    |ifx||#2||%
    |else
      |noexpand|hv@pdf@char3%
      |ltx@ReturnAfterFi{%
        |BKM@vtex@three#2|@nil
      }%
    |fi
  }%
%    \end{macrocode}
%    \end{macro}
%    \begin{macrocode}
|endgroup
%    \end{macrocode}
%
%    \begin{macrocode}
%</vtex>
%    \end{macrocode}
%
% \subsection{\hologo{pdfTeX}\ 的驱动程序}
%
%    \begin{macrocode}
%<*pdftex>
\NeedsTeXFormat{LaTeX2e}
\ProvidesFile{bkm-pdftex.def}%
  [2020-11-06 v1.29 bookmark driver for pdfTeX (HO)]%
%    \end{macrocode}
%
%    \begin{macro}{\BKM@DO@entry}
%    \begin{macrocode}
\def\BKM@DO@entry#1#2{%
  \begingroup
    \kvsetkeys{BKM@DO}{#1}%
    \def\BKM@DO@title{#2}%
    \ifx\BKM@DO@srcfile\@empty
    \else
      \BKM@UnescapeHex\BKM@DO@srcfile
    \fi
    \BKM@UnescapeHex\BKM@DO@title
    \expandafter\expandafter\expandafter\BKM@getx
        \csname BKM@\BKM@DO@id\endcsname\@empty\@empty
    \let\BKM@attr\@empty
    \ifx\BKM@DO@flags\@empty
    \else
      \edef\BKM@attr{\BKM@attr/F \BKM@DO@flags}%
    \fi
    \ifx\BKM@DO@color\@empty
    \else
      \edef\BKM@attr{\BKM@attr/C[\BKM@DO@color]}%
    \fi
    \ifx\BKM@attr\@empty
    \else
      \edef\BKM@attr{attr{\BKM@attr}}%
    \fi
    \let\BKM@action\@empty
    \ifx\BKM@DO@gotor\@empty
      \ifx\BKM@DO@dest\@empty
        \ifx\BKM@DO@named\@empty
          \ifx\BKM@DO@rawaction\@empty
            \ifx\BKM@DO@uri\@empty
              \ifx\BKM@DO@page\@empty
                \PackageError{bookmark}{%
                  Missing action\BKM@SourceLocation
                }\@ehc
                \edef\BKM@action{goto page1{/Fit}}%
              \else
                \ifx\BKM@DO@view\@empty
                  \def\BKM@DO@view{Fit}%
                \fi
                \edef\BKM@action{goto page\BKM@DO@page{/\BKM@DO@view}}%
              \fi
            \else
              \BKM@UnescapeHex\BKM@DO@uri
              \BKM@EscapeString\BKM@DO@uri
              \edef\BKM@action{user{<</S/URI/URI(\BKM@DO@uri)>>}}%
            \fi
          \else
            \BKM@UnescapeHex\BKM@DO@rawaction
            \edef\BKM@action{%
              user{%
                <<%
                  \BKM@DO@rawaction
                >>%
              }%
            }%
          \fi
        \else
          \BKM@EscapeName\BKM@DO@named
          \edef\BKM@action{%
            user{<</S/Named/N/\BKM@DO@named>>}%
          }%
        \fi
      \else
        \BKM@UnescapeHex\BKM@DO@dest
        \BKM@DefGotoNameAction\BKM@action\BKM@DO@dest
      \fi
    \else
      \ifx\BKM@DO@dest\@empty
        \ifx\BKM@DO@page\@empty
          \def\BKM@DO@page{0}%
        \else
          \BKM@CalcExpr\BKM@DO@page\BKM@DO@page-1%
        \fi
        \ifx\BKM@DO@view\@empty
          \def\BKM@DO@view{Fit}%
        \fi
        \edef\BKM@action{/D[\BKM@DO@page/\BKM@DO@view]}%
      \else
        \BKM@UnescapeHex\BKM@DO@dest
        \BKM@EscapeString\BKM@DO@dest
        \edef\BKM@action{/D(\BKM@DO@dest)}%
      \fi
      \BKM@UnescapeHex\BKM@DO@gotor
      \BKM@EscapeString\BKM@DO@gotor
      \edef\BKM@action{%
        user{%
          <<%
            /S/GoToR%
            /F(\BKM@DO@gotor)%
            \BKM@action
          >>%
        }%
      }%
    \fi
    \pdfoutline\BKM@attr\BKM@action
                count\ifBKM@DO@open\else-\fi\BKM@x@childs
                {\BKM@DO@title}%
  \endgroup
}
%    \end{macrocode}
%    \end{macro}
%    \begin{macro}{\BKM@DefGotoNameAction}
%    \cs{BKM@DefGotoNameAction}\ 宏是一个用于 \xpackage{hypdestopt}\ 宏包的钩子(hook)。
%    \begin{macrocode}
\def\BKM@DefGotoNameAction#1#2{%
  \BKM@EscapeString\BKM@DO@dest
  \edef#1{goto name{#2}}%
}
%    \end{macrocode}
%    \end{macro}
%    \begin{macrocode}
%</pdftex>
%    \end{macrocode}
%
%    \begin{macrocode}
%<*pdftex|pdfmark>
%    \end{macrocode}
%    \begin{macro}{\BKM@SourceLocation}
%    \begin{macrocode}
\def\BKM@SourceLocation{%
  \ifx\BKM@DO@srcfile\@empty
    \ifx\BKM@DO@srcline\@empty
    \else
      .\MessageBreak
      Source: line \BKM@DO@srcline
    \fi
  \else
    \ifx\BKM@DO@srcline\@empty
      .\MessageBreak
      Source: file `\BKM@DO@srcfile'%
    \else
      .\MessageBreak
      Source: file `\BKM@DO@srcfile', line \BKM@DO@srcline
    \fi
  \fi
}
%    \end{macrocode}
%    \end{macro}
%    \begin{macrocode}
%</pdftex|pdfmark>
%    \end{macrocode}
%
% \subsection{具有 pdfmark 特色(specials)的驱动程序}
%
% \subsubsection{dvips 驱动程序}
%
%    \begin{macrocode}
%<*dvips>
\NeedsTeXFormat{LaTeX2e}
\ProvidesFile{bkm-dvips.def}%
  [2020-11-06 v1.29 bookmark driver for dvips (HO)]%
%    \end{macrocode}
%    \begin{macro}{\BKM@PSHeaderFile}
%    \begin{macrocode}
\def\BKM@PSHeaderFile#1{%
  \special{PSfile=#1}%
}
%    \end{macrocode}
%    \begin{macro}{\BKM@filename}
%    \begin{macrocode}
\def\BKM@filename{\jobname.out.ps}
%    \end{macrocode}
%    \end{macro}
%    \begin{macrocode}
\AddToHook{shipout/lastpage}{%
  \BKM@pdfmark@out
  \BKM@PSHeaderFile\BKM@filename
  }
%    \end{macrocode}
%    \end{macro}
%    \begin{macrocode}
%</dvips>
%    \end{macrocode}
%
% \subsubsection{公共部分(Common part)}
%
%    \begin{macrocode}
%<*pdfmark>
%    \end{macrocode}
%
%    \begin{macro}{\BKM@pdfmark@out}
%    不要在这里使用 \xpackage{rerunfilecheck}\ 宏包,因为在 \hologo{TeX}\ 运行期间不会
%    读取 \cs{BKM@filename}\ 文件。
%    \begin{macrocode}
\def\BKM@pdfmark@out{%
  \if@filesw
    \newwrite\BKM@file
    \immediate\openout\BKM@file=\BKM@filename\relax
    \BKM@write{\@percentchar!}%
    \BKM@write{/pdfmark where{pop}}%
    \BKM@write{%
      {%
        /globaldict where{pop globaldict}{userdict}ifelse%
        /pdfmark/cleartomark load put%
      }%
    }%
    \BKM@write{ifelse}%
  \else
    \let\BKM@write\@gobble
    \let\BKM@DO@entry\@gobbletwo
  \fi
}
%    \end{macrocode}
%    \end{macro}
%    \begin{macro}{\BKM@write}
%    \begin{macrocode}
\def\BKM@write#{%
  \immediate\write\BKM@file
}
%    \end{macrocode}
%    \end{macro}
%
%    \begin{macro}{\BKM@DO@entry}
%    Pdfmark 的规范(specification)说明 |/Color| 是颜色(color)的键名(key name),
%    但是 ghostscript 只将键(key)传递到 PDF 文件中,因此键名必须是 |/C|。
%    \begin{macrocode}
\def\BKM@DO@entry#1#2{%
  \begingroup
    \kvsetkeys{BKM@DO}{#1}%
    \ifx\BKM@DO@srcfile\@empty
    \else
      \BKM@UnescapeHex\BKM@DO@srcfile
    \fi
    \def\BKM@DO@title{#2}%
    \BKM@UnescapeHex\BKM@DO@title
    \expandafter\expandafter\expandafter\BKM@getx
        \csname BKM@\BKM@DO@id\endcsname\@empty\@empty
    \let\BKM@attr\@empty
    \ifx\BKM@DO@flags\@empty
    \else
      \edef\BKM@attr{\BKM@attr/F \BKM@DO@flags}%
    \fi
    \ifx\BKM@DO@color\@empty
    \else
      \edef\BKM@attr{\BKM@attr/C[\BKM@DO@color]}%
    \fi
    \let\BKM@action\@empty
    \ifx\BKM@DO@gotor\@empty
      \ifx\BKM@DO@dest\@empty
        \ifx\BKM@DO@named\@empty
          \ifx\BKM@DO@rawaction\@empty
            \ifx\BKM@DO@uri\@empty
              \ifx\BKM@DO@page\@empty
                \PackageError{bookmark}{%
                  Missing action\BKM@SourceLocation
                }\@ehc
                \edef\BKM@action{%
                  /Action/GoTo%
                  /Page 1%
                  /View[/Fit]%
                }%
              \else
                \ifx\BKM@DO@view\@empty
                  \def\BKM@DO@view{Fit}%
                \fi
                \edef\BKM@action{%
                  /Action/GoTo%
                  /Page \BKM@DO@page
                  /View[/\BKM@DO@view]%
                }%
              \fi
            \else
              \BKM@UnescapeHex\BKM@DO@uri
              \BKM@EscapeString\BKM@DO@uri
              \edef\BKM@action{%
                /Action<<%
                  /Subtype/URI%
                  /URI(\BKM@DO@uri)%
                >>%
              }%
            \fi
          \else
            \BKM@UnescapeHex\BKM@DO@rawaction
            \edef\BKM@action{%
              /Action<<%
                \BKM@DO@rawaction
              >>%
            }%
          \fi
        \else
          \BKM@EscapeName\BKM@DO@named
          \edef\BKM@action{%
            /Action<<%
              /Subtype/Named%
              /N/\BKM@DO@named
            >>%
          }%
        \fi
      \else
        \BKM@UnescapeHex\BKM@DO@dest
        \BKM@EscapeString\BKM@DO@dest
        \edef\BKM@action{%
          /Action/GoTo%
          /Dest(\BKM@DO@dest)cvn%
        }%
      \fi
    \else
      \ifx\BKM@DO@dest\@empty
        \ifx\BKM@DO@page\@empty
          \def\BKM@DO@page{1}%
        \fi
        \ifx\BKM@DO@view\@empty
          \def\BKM@DO@view{Fit}%
        \fi
        \edef\BKM@action{%
          /Page \BKM@DO@page
          /View[/\BKM@DO@view]%
        }%
      \else
        \BKM@UnescapeHex\BKM@DO@dest
        \BKM@EscapeString\BKM@DO@dest
        \edef\BKM@action{%
          /Dest(\BKM@DO@dest)cvn%
        }%
      \fi
      \BKM@UnescapeHex\BKM@DO@gotor
      \BKM@EscapeString\BKM@DO@gotor
      \edef\BKM@action{%
        /Action/GoToR%
        /File(\BKM@DO@gotor)%
        \BKM@action
      }%
    \fi
    \BKM@write{[}%
    \BKM@write{/Title(\BKM@DO@title)}%
    \ifnum\BKM@x@childs>\z@
      \BKM@write{/Count \ifBKM@DO@open\else-\fi\BKM@x@childs}%
    \fi
    \ifx\BKM@attr\@empty
    \else
      \BKM@write{\BKM@attr}%
    \fi
    \BKM@write{\BKM@action}%
    \BKM@write{/OUT pdfmark}%
  \endgroup
}
%    \end{macrocode}
%    \end{macro}
%    \begin{macrocode}
%</pdfmark>
%    \end{macrocode}
%
% \subsection{\xoption{pdftex}\ 和 \xoption{pdfmark}\ 的公共部分}
%
%    \begin{macrocode}
%<*pdftex|pdfmark>
%    \end{macrocode}
%
% \subsubsection{写入辅助文件(auxiliary file)}
%
%    \begin{macrocode}
\AddToHook{begindocument}{%
 \immediate\write\@mainaux{\string\providecommand\string\BKM@entry[2]{}}}
%    \end{macrocode}
%
%    \begin{macro}{\BKM@id}
%    \begin{macrocode}
\newcount\BKM@id
\BKM@id=\z@
%    \end{macrocode}
%    \end{macro}
%
%    \begin{macro}{\BKM@0}
%    \begin{macrocode}
\@namedef{BKM@0}{000}
%    \end{macrocode}
%    \end{macro}
%    \begin{macro}{\ifBKM@sw}
%    \begin{macrocode}
\newif\ifBKM@sw
%    \end{macrocode}
%    \end{macro}
%
%    \begin{macro}{\bookmark}
%    \begin{macrocode}
\newcommand*{\bookmark}[2][]{%
  \if@filesw
    \begingroup
      \BKM@InitSourceLocation
      \def\bookmark@text{#2}%
      \BKM@setup{#1}%
      \ifx\BKM@srcfile\@empty
      \else
        \BKM@EscapeHex\BKM@srcfile
      \fi
      \edef\BKM@prev{\the\BKM@id}%
      \global\advance\BKM@id\@ne
      \BKM@swtrue
      \@whilesw\ifBKM@sw\fi{%
        \ifnum\ifBKM@startatroot\z@\else\BKM@prev\fi=\z@
          \BKM@startatrootfalse
          \expandafter\xdef\csname BKM@\the\BKM@id\endcsname{%
            0{\BKM@level}0%
          }%
          \BKM@swfalse
        \else
          \expandafter\expandafter\expandafter\BKM@getx
              \csname BKM@\BKM@prev\endcsname
          \ifnum\BKM@level>\BKM@x@level\relax
            \expandafter\xdef\csname BKM@\the\BKM@id\endcsname{%
              {\BKM@prev}{\BKM@level}0%
            }%
            \ifnum\BKM@prev>\z@
              \BKM@CalcExpr\BKM@CalcResult\BKM@x@childs+1%
              \expandafter\xdef\csname BKM@\BKM@prev\endcsname{%
                {\BKM@x@parent}{\BKM@x@level}{\BKM@CalcResult}%
              }%
            \fi
            \BKM@swfalse
          \else
            \let\BKM@prev\BKM@x@parent
          \fi
        \fi
      }%
      \pdfstringdef\BKM@title{\bookmark@text}%
      \edef\BKM@FLAGS{\BKM@PrintStyle}%
      \csname BKM@HypDestOptHook\endcsname
      \BKM@EscapeHex\BKM@dest
      \BKM@EscapeHex\BKM@uri
      \BKM@EscapeHex\BKM@gotor
      \BKM@EscapeHex\BKM@rawaction
      \BKM@EscapeHex\BKM@title
      \immediate\write\@mainaux{%
        \string\BKM@entry{%
          id=\number\BKM@id
          \ifBKM@open
            \ifnum\BKM@level<\BKM@openlevel
              ,open%
            \fi
          \fi
          \BKM@auxentry{dest}%
          \BKM@auxentry{named}%
          \BKM@auxentry{uri}%
          \BKM@auxentry{gotor}%
          \BKM@auxentry{page}%
          \BKM@auxentry{view}%
          \BKM@auxentry{rawaction}%
          \BKM@auxentry{color}%
          \ifnum\BKM@FLAGS>\z@
            ,flags=\BKM@FLAGS
          \fi
          \BKM@auxentry{srcline}%
          \BKM@auxentry{srcfile}%
        }{\BKM@title}%
      }%
    \endgroup
  \fi
}
%    \end{macrocode}
%    \end{macro}
%    \begin{macro}{\BKM@getx}
%    \begin{macrocode}
\def\BKM@getx#1#2#3{%
  \def\BKM@x@parent{#1}%
  \def\BKM@x@level{#2}%
  \def\BKM@x@childs{#3}%
}
%    \end{macrocode}
%    \end{macro}
%    \begin{macro}{\BKM@auxentry}
%    \begin{macrocode}
\def\BKM@auxentry#1{%
  \expandafter\ifx\csname BKM@#1\endcsname\@empty
  \else
    ,#1={\csname BKM@#1\endcsname}%
  \fi
}
%    \end{macrocode}
%    \end{macro}
%
%    \begin{macro}{\BKM@InitSourceLocation}
%    \begin{macrocode}
\def\BKM@InitSourceLocation{%
  \edef\BKM@srcline{\the\inputlineno}%
  \BKM@LuaTeX@InitFile
  \ifx\BKM@srcfile\@empty
    \ltx@IfUndefined{currfilepath}{}{%
      \edef\BKM@srcfile{\currfilepath}%
    }%
  \fi
}
%    \end{macrocode}
%    \end{macro}
%    \begin{macro}{\BKM@LuaTeX@InitFile}
%    \begin{macrocode}
\ifluatex
  \ifnum\luatexversion>36 %
    \def\BKM@LuaTeX@InitFile{%
      \begingroup
        \ltx@LocToksA={}%
      \edef\x{\endgroup
        \def\noexpand\BKM@srcfile{%
          \the\expandafter\ltx@LocToksA
          \directlua{%
             if status and status.filename then %
               tex.settoks('ltx@LocToksA', status.filename)%
             end%
          }%
        }%
      }\x
    }%
  \else
    \let\BKM@LuaTeX@InitFile\relax
  \fi
\else
  \let\BKM@LuaTeX@InitFile\relax
\fi
%    \end{macrocode}
%    \end{macro}
%
% \subsubsection{读取辅助数据(auxiliary data)}
%
%    \begin{macrocode}
\SetupKeyvalOptions{family=BKM@DO,prefix=BKM@DO@}
\DeclareStringOption[0]{id}
\DeclareBoolOption{open}
\DeclareStringOption{flags}
\DeclareStringOption{color}
\DeclareStringOption{dest}
\DeclareStringOption{named}
\DeclareStringOption{uri}
\DeclareStringOption{gotor}
\DeclareStringOption{page}
\DeclareStringOption{view}
\DeclareStringOption{rawaction}
\DeclareStringOption{srcline}
\DeclareStringOption{srcfile}
%    \end{macrocode}
%
%    \begin{macrocode}
\AtBeginDocument{%
  \let\BKM@entry\BKM@DO@entry
}
%    \end{macrocode}
%
%    \begin{macrocode}
%</pdftex|pdfmark>
%    \end{macrocode}
%
% \subsection{\xoption{atend}\ 选项}
%
% \subsubsection{钩子(Hook)}
%
%    \begin{macrocode}
%<*package>
%    \end{macrocode}
%    \begin{macrocode}
\ifBKM@atend
\else
%    \end{macrocode}
%    \begin{macro}{\BookmarkAtEnd}
%    这是一个虚拟定义(dummy definition),如果没有给出 \xoption{atend}\ 选项,它将生成一个警告。
%    \begin{macrocode}
  \newcommand{\BookmarkAtEnd}[1]{%
    \PackageWarning{bookmark}{%
      Ignored, because option `atend' is missing%
    }%
  }%
%    \end{macrocode}
%    \end{macro}
%    \begin{macrocode}
  \expandafter\endinput
\fi
%    \end{macrocode}
%    \begin{macro}{\BookmarkAtEnd}
%    \begin{macrocode}
\newcommand*{\BookmarkAtEnd}{%
  \g@addto@macro\BKM@EndHook
}
%    \end{macrocode}
%    \end{macro}
%    \begin{macrocode}
\let\BKM@EndHook\@empty
%    \end{macrocode}
%    \begin{macrocode}
%</package>
%    \end{macrocode}
%
% \subsubsection{在文档末尾使用钩子的驱动程序}
%
%    驱动程序 \xoption{pdftex}\ 使用 LaTeX 钩子 \xoption{enddocument/afterlastpage}
%    (相当于以前使用的 \xpackage{atveryend}\ 的 \cs{AfterLastShipout}),因为它仍然需要 \xext{aux}\ 文件。
%    它使用 \cs{pdfoutline}\ 作为最后一页之后可以使用的书签(bookmakrs)。
%    \begin{itemize}
%    \item
%      驱动程序 \xoption{pdftex}\ 使用 \cs{pdfoutline}, \cs{pdfoutline}\ 可以在最后一页之后使用。
%    \end{itemize}
%    \begin{macrocode}
%<*pdftex>
\ifBKM@atend
  \AddToHook{enddocument/afterlastpage}{%
    \BKM@EndHook
  }%
\fi
%</pdftex>
%    \end{macrocode}
%
% \subsubsection{使用 \xoption{shipout/lastpage}\ 的驱动程序}
%
%    其他驱动程序使用 \cs{special}\ 命令实现 \cs{bookmark}。因此,最后的书签(last bookmarks)
%    必须放在最后一页(last page),而不是之后。不能使用 \cs{AtEndDocument},因为为时已晚,
%    最后一页已经输出了。因此,我们使用 LaTeX 钩子 \xoption{shipout/lastpage}。至少需要运行
%    两次 \hologo{LaTeX}。PostScript 驱动程序 \xoption{dvips}\ 使用外部 PostScript 文件作为书签。
%    为了避免与 pgf 发生冲突,文件写入(file writing)也被移到了最后一个输出页面(shipout page)。
%    \begin{macrocode}
%<*dvipdfm|vtex|pdfmark>
\ifBKM@atend
  \AddToHook{shipout/lastpage}{\BKM@EndHook}%
\fi
%</dvipdfm|vtex|pdfmark>
%    \end{macrocode}
%
% \section{安装(Installation)}
%
% \subsection{下载(Download)}
%
% \paragraph{宏包(Package)。} 在 CTAN\footnote{\CTANpkg{bookmark}}上提供此宏包:
% \begin{description}
% \item[\CTAN{macros/latex/contrib/bookmark/bookmark.dtx}] 源文件(source file)。
% \item[\CTAN{macros/latex/contrib/bookmark/bookmark.pdf}] 文档(documentation)。
% \end{description}
%
%
% \paragraph{捆绑包(Bundle)。} “bookmark”捆绑包(bundle)的所有宏包(packages)都可以在兼
% 容 TDS 的 ZIP 归档文件中找到。在那里,宏包已经被解包,文档文件(documentation files)已经生成。
% 文件(files)和目录(directories)遵循 TDS 标准。
% \begin{description}
% \item[\CTANinstall{install/macros/latex/contrib/bookmark.tds.zip}]
% \end{description}
% \emph{TDS}\ 是指标准的“用于 \TeX\ 文件的目录结构(Directory Structure)”(\CTANpkg{tds})。
% 名称中带有 \xfile{texmf}\ 的目录(directories)通常以这种方式组织。
%
% \subsection{捆绑包(Bundle)的安装}
%
% \paragraph{解压(Unpacking)。} 在您选择的 TDS 树(也称为 \xfile{texmf}\ 树)中解
% 压 \xfile{bookmark.tds.zip},例如(在 linux 中):
% \begin{quote}
%   |unzip bookmark.tds.zip -d ~/texmf|
% \end{quote}
%
% \subsection{宏包(Package)的安装}
%
% \paragraph{解压(Unpacking)。} \xfile{.dtx}\ 文件是一个自解压 \docstrip\ 归档文件(archive)。
% 这些文件是通过 \plainTeX\ 运行 \xfile{.dtx}\ 来提取的:
% \begin{quote}
%   \verb|tex bookmark.dtx|
% \end{quote}
%
% \paragraph{TDS.} 现在,不同的文件必须移动到安装 TDS 树(installation TDS tree)
% (也称为 \xfile{texmf}\ 树)中的不同目录中:
% \begin{quote}
% \def\t{^^A
% \begin{tabular}{@{}>{\ttfamily}l@{ $\rightarrow$ }>{\ttfamily}l@{}}
%   bookmark.sty & tex/latex/bookmark/bookmark.sty\\
%   bkm-dvipdfm.def & tex/latex/bookmark/bkm-dvipdfm.def\\
%   bkm-dvips.def & tex/latex/bookmark/bkm-dvips.def\\
%   bkm-pdftex.def & tex/latex/bookmark/bkm-pdftex.def\\
%   bkm-vtex.def & tex/latex/bookmark/bkm-vtex.def\\
%   bookmark.pdf & doc/latex/bookmark/bookmark.pdf\\
%   bookmark-example.tex & doc/latex/bookmark/bookmark-example.tex\\
%   bookmark.dtx & source/latex/bookmark/bookmark.dtx\\
% \end{tabular}^^A
% }^^A
% \sbox0{\t}^^A
% \ifdim\wd0>\linewidth
%   \begingroup
%     \advance\linewidth by\leftmargin
%     \advance\linewidth by\rightmargin
%   \edef\x{\endgroup
%     \def\noexpand\lw{\the\linewidth}^^A
%   }\x
%   \def\lwbox{^^A
%     \leavevmode
%     \hbox to \linewidth{^^A
%       \kern-\leftmargin\relax
%       \hss
%       \usebox0
%       \hss
%       \kern-\rightmargin\relax
%     }^^A
%   }^^A
%   \ifdim\wd0>\lw
%     \sbox0{\small\t}^^A
%     \ifdim\wd0>\linewidth
%       \ifdim\wd0>\lw
%         \sbox0{\footnotesize\t}^^A
%         \ifdim\wd0>\linewidth
%           \ifdim\wd0>\lw
%             \sbox0{\scriptsize\t}^^A
%             \ifdim\wd0>\linewidth
%               \ifdim\wd0>\lw
%                 \sbox0{\tiny\t}^^A
%                 \ifdim\wd0>\linewidth
%                   \lwbox
%                 \else
%                   \usebox0
%                 \fi
%               \else
%                 \lwbox
%               \fi
%             \else
%               \usebox0
%             \fi
%           \else
%             \lwbox
%           \fi
%         \else
%           \usebox0
%         \fi
%       \else
%         \lwbox
%       \fi
%     \else
%       \usebox0
%     \fi
%   \else
%     \lwbox
%   \fi
% \else
%   \usebox0
% \fi
% \end{quote}
% 如果你有一个 \xfile{docstrip.cfg}\ 文件,该文件能配置并启用 \docstrip\ 的 TDS 安装功能,
% 则一些文件可能已经在正确的位置了,请参阅 \docstrip\ 的文档(documentation)。
%
% \subsection{刷新文件名数据库}
%
% 如果您的 \TeX~发行版(\TeX\,Live、\mikTeX、\dots)依赖于文件名数据库(file name databases),
% 则必须刷新这些文件名数据库。例如,\TeX\,Live\ 用户运行 \verb|texhash| 或 \verb|mktexlsr|。
%
% \subsection{一些感兴趣的细节}
%
% \paragraph{用 \LaTeX\ 解压。}
% \xfile{.dtx}\ 根据格式(format)选择其操作(action):
% \begin{description}
% \item[\plainTeX:] 运行 \docstrip\ 并解压文件。
% \item[\LaTeX:] 生成文档。
% \end{description}
% 如果您坚持通过 \LaTeX\ 使用\docstrip (实际上 \docstrip\ 并不需要 \LaTeX),那么请您的意图告知自动检测程序:
% \begin{quote}
%   \verb|latex \let\install=y% \iffalse meta-comment
%
% File: bookmark.dtx
% Version: 2020-11-06 v1.29
% Info: PDF bookmarks
%
% Copyright (C)
%    2007-2011 Heiko Oberdiek
%    2016-2020 Oberdiek Package Support Group
%    https://github.com/ho-tex/bookmark/issues
%
% This work may be distributed and/or modified under the
% conditions of the LaTeX Project Public License, either
% version 1.3c of this license or (at your option) any later
% version. This version of this license is in
%    https://www.latex-project.org/lppl/lppl-1-3c.txt
% and the latest version of this license is in
%    https://www.latex-project.org/lppl.txt
% and version 1.3 or later is part of all distributions of
% LaTeX version 2005/12/01 or later.
%
% This work has the LPPL maintenance status "maintained".
%
% The Current Maintainers of this work are
% Heiko Oberdiek and the Oberdiek Package Support Group
% https://github.com/ho-tex/bookmark/issues
%
% This work consists of the main source file bookmark.dtx
% and the derived files
%    bookmark.sty, bookmark.pdf, bookmark.ins, bookmark.drv,
%    bkm-dvipdfm.def, bkm-dvips.def,
%    bkm-pdftex.def, bkm-vtex.def,
%    bkm-dvipdfm-2019-12-03.def, bkm-dvips-2019-12-03.def,
%    bkm-pdftex-2019-12-03.def, bkm-vtex-2019-12-03.def,
%    bookmark-example.tex.
%
% Distribution:
%    CTAN:macros/latex/contrib/bookmark/bookmark.dtx
%    CTAN:macros/latex/contrib/bookmark/bookmark-frozen.dtx
%    CTAN:macros/latex/contrib/bookmark/bookmark.pdf
%
% Unpacking:
%    (a) If bookmark.ins is present:
%           tex bookmark.ins
%    (b) Without bookmark.ins:
%           tex bookmark.dtx
%    (c) If you insist on using LaTeX
%           latex \let\install=y\input{bookmark.dtx}
%        (quote the arguments according to the demands of your shell)
%
% Documentation:
%    (a) If bookmark.drv is present:
%           latex bookmark.drv
%    (b) Without bookmark.drv:
%           latex bookmark.dtx; ...
%    The class ltxdoc loads the configuration file ltxdoc.cfg
%    if available. Here you can specify further options, e.g.
%    use A4 as paper format:
%       \PassOptionsToClass{a4paper}{article}
%
%    Programm calls to get the documentation (example):
%       pdflatex bookmark.dtx
%       makeindex -s gind.ist bookmark.idx
%       pdflatex bookmark.dtx
%       makeindex -s gind.ist bookmark.idx
%       pdflatex bookmark.dtx
%
% Installation:
%    TDS:tex/latex/bookmark/bookmark.sty
%    TDS:tex/latex/bookmark/bkm-dvipdfm.def
%    TDS:tex/latex/bookmark/bkm-dvips.def
%    TDS:tex/latex/bookmark/bkm-pdftex.def
%    TDS:tex/latex/bookmark/bkm-vtex.def
%    TDS:tex/latex/bookmark/bkm-dvipdfm-2019-12-03.def
%    TDS:tex/latex/bookmark/bkm-dvips-2019-12-03.def
%    TDS:tex/latex/bookmark/bkm-pdftex-2019-12-03.def
%    TDS:tex/latex/bookmark/bkm-vtex-2019-12-03.def%
%    TDS:doc/latex/bookmark/bookmark.pdf
%    TDS:doc/latex/bookmark/bookmark-example.tex
%    TDS:source/latex/bookmark/bookmark.dtx
%    TDS:source/latex/bookmark/bookmark-frozen.dtx
%
%<*ignore>
\begingroup
  \catcode123=1 %
  \catcode125=2 %
  \def\x{LaTeX2e}%
\expandafter\endgroup
\ifcase 0\ifx\install y1\fi\expandafter
         \ifx\csname processbatchFile\endcsname\relax\else1\fi
         \ifx\fmtname\x\else 1\fi\relax
\else\csname fi\endcsname
%</ignore>
%<*install>
\input docstrip.tex
\Msg{************************************************************************}
\Msg{* Installation}
\Msg{* Package: bookmark 2020-11-06 v1.29 PDF bookmarks (HO)}
\Msg{************************************************************************}

\keepsilent
\askforoverwritefalse

\let\MetaPrefix\relax
\preamble

This is a generated file.

Project: bookmark
Version: 2020-11-06 v1.29

Copyright (C)
   2007-2011 Heiko Oberdiek
   2016-2020 Oberdiek Package Support Group

This work may be distributed and/or modified under the
conditions of the LaTeX Project Public License, either
version 1.3c of this license or (at your option) any later
version. This version of this license is in
   https://www.latex-project.org/lppl/lppl-1-3c.txt
and the latest version of this license is in
   https://www.latex-project.org/lppl.txt
and version 1.3 or later is part of all distributions of
LaTeX version 2005/12/01 or later.

This work has the LPPL maintenance status "maintained".

The Current Maintainers of this work are
Heiko Oberdiek and the Oberdiek Package Support Group
https://github.com/ho-tex/bookmark/issues


This work consists of the main source file bookmark.dtx and bookmark-frozen.dtx
and the derived files
   bookmark.sty, bookmark.pdf, bookmark.ins, bookmark.drv,
   bkm-dvipdfm.def, bkm-dvips.def, bkm-pdftex.def, bkm-vtex.def,
   bkm-dvipdfm-2019-12-03.def, bkm-dvips-2019-12-03.def,
   bkm-pdftex-2019-12-03.def, bkm-vtex-2019-12-03.def,
   bookmark-example.tex.

\endpreamble
\let\MetaPrefix\DoubleperCent

\generate{%
  \file{bookmark.ins}{\from{bookmark.dtx}{install}}%
  \file{bookmark.drv}{\from{bookmark.dtx}{driver}}%
  \usedir{tex/latex/bookmark}%
  \file{bookmark.sty}{\from{bookmark.dtx}{package}}%
  \file{bkm-dvipdfm.def}{\from{bookmark.dtx}{dvipdfm}}%
  \file{bkm-dvips.def}{\from{bookmark.dtx}{dvips,pdfmark}}%
  \file{bkm-pdftex.def}{\from{bookmark.dtx}{pdftex}}%
  \file{bkm-vtex.def}{\from{bookmark.dtx}{vtex}}%
  \usedir{doc/latex/bookmark}%
  \file{bookmark-example.tex}{\from{bookmark.dtx}{example}}%
  \file{bkm-pdftex-2019-12-03.def}{\from{bookmark-frozen.dtx}{pdftexfrozen}}%
  \file{bkm-dvips-2019-12-03.def}{\from{bookmark-frozen.dtx}{dvipsfrozen}}%
  \file{bkm-vtex-2019-12-03.def}{\from{bookmark-frozen.dtx}{vtexfrozen}}%
  \file{bkm-dvipdfm-2019-12-03.def}{\from{bookmark-frozen.dtx}{dvipdfmfrozen}}%
}

\catcode32=13\relax% active space
\let =\space%
\Msg{************************************************************************}
\Msg{*}
\Msg{* To finish the installation you have to move the following}
\Msg{* files into a directory searched by TeX:}
\Msg{*}
\Msg{*     bookmark.sty, bkm-dvipdfm.def, bkm-dvips.def,}
\Msg{*     bkm-pdftex.def, bkm-vtex.def, bkm-dvipdfm-2019-12-03.def,}
\Msg{*     bkm-dvips-2019-12-03.def, bkm-pdftex-2019-12-03.def,}
\Msg{*     and bkm-vtex-2019-12-03.def}
\Msg{*}
\Msg{* To produce the documentation run the file `bookmark.drv'}
\Msg{* through LaTeX.}
\Msg{*}
\Msg{* Happy TeXing!}
\Msg{*}
\Msg{************************************************************************}

\endbatchfile
%</install>
%<*ignore>
\fi
%</ignore>
%<*driver>
\NeedsTeXFormat{LaTeX2e}
\ProvidesFile{bookmark.drv}%
  [2020-11-06 v1.29 PDF bookmarks (HO)]%
\documentclass{ltxdoc}
\usepackage{ctex}
\usepackage{indentfirst}
\setlength{\parindent}{2em}
\usepackage{holtxdoc}[2011/11/22]
\usepackage{xcolor}
\usepackage{hyperref}
\usepackage[open,openlevel=3,atend]{bookmark}[2020/11/06] %%%打开书签,显示的深度为3级,即显示part、section、subsection。
\bookmarksetup{color=red}
\begin{document}

  \renewcommand{\contentsname}{目\quad 录}
  \renewcommand{\abstractname}{摘\quad 要}
  \renewcommand{\historyname}{历史}
  \DocInput{bookmark.dtx}%
\end{document}
%</driver>
% \fi
%
%
%
% \GetFileInfo{bookmark.drv}
%
%% \title{\xpackage{bookmark} 宏包}
% \title{\heiti {\Huge \textbf{\xpackage{bookmark}\ 宏包}}}
% \date{2020-11-06\ \ \ v1.29}
% \author{Heiko Oberdiek \thanks
% {如有问题请点击:\url{https://github.com/ho-tex/bookmark/issues}}\\[5pt]赣医一附院神经科\ \ 黄旭华\ \ \ \ 译}
%
% \maketitle
%
% \begin{abstract}
% 这个宏包为 \xpackage{hyperref}\ 宏包实现了一个新的书签(bookmark)(大纲[outline])组织。现在
% 可以设置样式(style)和颜色(color)等书签属性(bookmark properties)。其他动作类型(action types)可用
% (URI、GoToR、Named)。书签是在第一次编译运行(compile run)中生成的。\xpackage{hyperref}\
% 宏包必需运行两次。
% \end{abstract}
%
% \tableofcontents
%
% \section{文档(Documentation)}
%
% \subsection{介绍}
%
% 这个 \xpackage{bookmark}\ 宏包试图为书签(bookmarks)提供一个更现代的管理:
% \begin{itemize}
% \item 书签已经在第一次 \hologo{TeX}\ 编译运行(compile run)中生成。
% \item 可以更改书签的字体样式(font style)和颜色(color)。
% \item 可以执行比简单的 GoTo 操作(actions)更多的操作。
% \end{itemize}
%
% 与 \xpackage{hyperref} \cite{hyperref} 一样,书签(bookmarks)也是按照书签生成宏
% (bookmark generating macros)(\cs{bookmark})的顺序生成的。级别号(level number)用于
% 定义书签的树结构(tree structure)。限制没有那么严格:
% \begin{itemize}
% \item 级别值(level values)可以跳变(jump)和省略(omit)。\cs{subsubsection}\ 可以跟在
%       \cs{chapter}\ 之后。这种情况如在 \xpackage{hyperref}\ 中则产生错误,它将显示一个警告(warning)
%       并尝试修复此错误。
% \item 多个书签可能指向同一目标(destination)。在 \xpackage{hyperref}\ 中,这会完全弄乱
%       书签树(bookmark tree),因为算法假设(algorithm assumes)目标名称(destination names)
%       是键(keys)(唯一的)。
% \end{itemize}
%
% 注意,这个宏包是作为书签管理(bookmark management)的实验平台(experimentation platform)。
% 欢迎反馈。此外,在未来的版本中,接口(interfaces)也可能发生变化。
%
% \subsection{选项(Options)}
%
% 可在以下四个地方放置选项(options):
% \begin{enumerate}
% \item \cs{usepackage}|[|\meta{options}|]{bookmark}|\\
%       这是放置驱动程序选项(driver options)和 \xoption{atend}\ 选项的唯一位置。
% \item \cs{bookmarksetup}|{|\meta{options}|}|\\
%       此命令仅用于设置选项(setting options)。
% \item \cs{bookmarksetupnext}|{|\meta{options}|}|\\
%       这些选项在下一个 \cs{bookmark}\ 命令的选项之后存储(stored)和调用(called)。
% \item \cs{bookmark}|[|\meta{options}|]{|\meta{title}|}|\\
%       此命令设置书签。选项设置(option settings)仅限于此书签。
% \end{enumerate}
% 异常(Exception):加载该宏包后,无法更改驱动程序选项(Driver options)、\xoption{atend}\ 选项
% 、\xoption{draft}\slash\xoption{final}选项。
%
% \subsubsection{\xoption{draft} 和 \xoption{final}\ 选项}
%
% 如果一个\LaTeX\ 文件要被编译了多次,那么可以使用 \xoption{draft}\ 选项来禁用该宏包的书签内
% 容(bookmark stuff),这样可以节省一点时间。默认 \xoption{final}\ 选项。两个选项都是
% 布尔选项(boolean options),如果没有值,则使用值 |true|。|draft=true| 与 |final=false| 相同。
%
% 除了驱动程序选项(driver options)之外,\xpackage{bookmark}\ 宏包选项都是局部选项(local options)。
% \xoption{draft}\ 选项和 \xoption{final}\ 选项均属于文档类选项(class option)(译者注:文档类选项为全局选项),
% 因此,在 \xpackage{bookmark}\ 宏包中未能看到这两个选项。如果您想使用全局的(global) \xoption{draft}选项
% 来优化第一次 \LaTeX\ 运行(runs),可以在导言(preamble)中引入 \xpackage{ifdraft}\ 宏包并设置 \LaTeX\ 的
% \cs{PassOptionsToPackage},例如:
%\begin{quote}
%\begin{verbatim}
%\documentclass[draft]{article}
%\usepackage{ifdraft}
%\ifdraft{%
%   \PassOptionsToPackage{draft}{bookmark}%
%}{}
%\end{verbatim}
%\end{quote}
%
% \subsubsection{驱动程序选项(Driver options)}
%
% 支持的驱动程序( drivers)包括 \xoption{pdftex}、\xoption{dvips}、\xoption{dvipdfm} (\xoption{xetex})、
% \xoption{vtex}。\hologo{TeX}\ 引擎 \hologo{pdfTeX}、\hologo{XeTeX}、\hologo{VTeX}\ 能被自动检测到。
% 默认的 DVI 驱动程序是 \xoption{dvips}。这可以通过 \cs{BookmarkDriverDefault}\ 在配置
% 文件 \xfile{bookmark.cfg}\ 中进行更改,例如:
% \begin{quote}
% |\def\BookmarkDriverDefault{dvipdfm}|
% \end{quote}
% 当前版本的(current versions)驱动程序使用新的 \LaTeX\ 钩子(\LaTeX-hooks)。如果检测到比
% 2020-10-01 更旧的格式,则将以前驱动程序的冻结版本(frozen versions)作为备份(fallback)。
%
% \paragraph{用 dvipdfmx 打开书签(bookmarks)。}旧版本的宏包有一个 \xoption{dvipdfmx-outline-open}\ 选项
% 可以激活代码,而该代码可以指定一个大纲条目(outline entry)是否打开。该宏包现在假设所有使用的 dvipdfmx 版本都是
% 最新版本,足以理解该代码,因此始终激活该代码。选项本身将被忽略。
%
%
% \subsubsection{布局选项(Layout options)}
%
% \paragraph{字体(Font)选项:}
%
% \begin{description}
% \item[\xoption{bold}:] 如果受 PDF 浏览器(PDF viewer)支持,书签将以粗体字体(bold font)显示(自 PDF 1.4起)。
% \item[\xoption{italic}:] 使用斜体字体(italic font)(自 PDF 1.4起)。
% \end{description}
% \xoption{bold}(粗体) 和 \xoption{italic}(斜体)可以同时使用。而 |false| 值(value)禁用字体选项。
%
% \paragraph{颜色(Color)选项:}
%
% 彩色书签(Colored bookmarks)是 PDF 1.4 的一个特性(feature),并非所有的 PDF 浏览器(PDF viewers)都支持彩色书签。
% \begin{description}
% \item[\xoption{color}:] 这里 color(颜色)可以作为 \xpackage{color}\ 宏包或 \xpackage{xcolor}\ 宏包的
% 颜色规范(color specification)给出。空值(empty value)表示未设置颜色属性。如果未加载 \xpackage{xcolor}\ 宏包,
% 能识别的值(recognized values)只有:
%   \begin{itemize}
%   \item 空值(empty value)表示未设置颜色属性,\\
%         例如:|color={}|
%   \item 颜色模型(color model) rgb 的显式颜色规范(explicit color specification),\\
%         例如,红色(red):|color=[rgb]{1,0,0}|
%   \item 颜色模型(color model)灰(gray)的显式颜色规范(explicit color specification),\\
%         例如,深灰色(dark gray):|color=[gray]{0.25}|
%   \end{itemize}
%   请注意,如果加载了 \xpackage{color}\ 宏包,此限制(restriction)也适用。然而,如果加载了 \xpackage{xcolor}\ 宏包,
%   则可以使用所有颜色规范(color specifications)。
% \end{description}
%
% \subsubsection{动作选项(Action options)}
%
% \begin{description}
% \item[\xoption{dest}:] 目的地名称(destination name)。
% \item[\xoption{page}:] 页码(page number),第一页(first page)为 1。
% \item[\xoption{view}:] 浏览规范(view specification),示例如下:\\
%   |view={FitB}|, |view={FitH 842}|, |view={XYZ 0 100 null}|\ \  一些浏览规范参数(view specification parameters)
%   将数字(numbers)视为具有单位 bp 的参数。它们可以作为普通数字(plain numbers)或在 \cs{calc}\ 内部以
%   长度表达式(length expressions)给出。如果加载了 \xpackage{calc}\ 宏包,则支持该宏包的表达式(expressions)。否则,
%   使用 \hologo{eTeX}\ 的 \cs{dimexpr}。例如:\\
%   |view={FitH \calc{\paperheight-\topmargin-1in}}|\\
%   |view={XYZ 0 \calc{\paperheight} null}|\\
%   注意 \cs{calc}\ 不能用于 |XYZ| 的第三个参数,因为该参数是缩放值(zoom value),而不是长度(length)。

% \item[\xoption{named}:] 已命名的动作(Named action)的名称:\\
%   |FirstPage|(第一页),|LastPage|(最后一页),|NextPage|(下一页),|PrevPage|(前一页)
% \item[\xoption{gotor}:] 外部(external) PDF 文件的名称。
% \item[\xoption{uri}:] URI 规范(URI specification)。
% \item[\xoption{rawaction}:] 原始动作规范(raw action specification)。由于这些规范取决于驱动程序(driver),因此不应使用此选项。
% \end{description}
% 通过分析指定的选项来选择书签的适当动作。动作由不同的选项集(sets of options)区分:
% \begin{quote}
 \begin{tabular}{|@{}r|l@{}|}
%   \hline
%   \ \textbf{动作(Action)}\  & \ \textbf{选项(Options)}\ \\ \hline
%   \ \textsf{GoTo}\  &\  \xoption{dest}\ \\ \hline
%   \ \textsf{GoTo}\  & \ \xoption{page} + \xoption{view}\ \\ \hline
%   \ \textsf{GoToR}\  & \ \xoption{gotor} + \xoption{dest}\ \\ \hline
%   \ \textsf{GoToR}\  & \ \xoption{gotor} + \xoption{page} + \xoption{view}\ \ \ \\ \hline
%   \ \textsf{Named}\  &\  \xoption{named}\ \\ \hline
%   \ \textsf{URI}\  & \ \xoption{uri}\ \\ \hline
% \end{tabular}
% \end{quote}
%
% \paragraph{缺少动作(Missing actions)。}
% 如果动作缺少 \xpackage{bookmark}\ 宏包,则抛出错误消息(error message)。根据驱动程序(driver)
% (\xoption{pdftex}、\xoption{dvips}\ 和好友[friends]),宏包在文档末尾很晚才检测到它。
% 自 2011/04/21 v1.21 版本以后,该宏包尝试打印 \cs{bookmark}\ 的相应出现的行号(line number)和文件名(file name)。
% 然而,\hologo{TeX}\ 确实提供了行号,但不幸的是,文件名是一个秘密(secret)。但该宏包有如下获取文件名的方法:
% \begin{itemize}
% \item 如果 \hologo{LuaTeX} (独立于 DVI 或 PDF 模式)正在运行,则自动使用其 |status.filename|。
% \item 宏包的 \cs{currfile} \cite{currfile}\ 重新定义了 \hologo{LaTeX}\ 的内部结构,以跟踪文件名(file name)。
% 如果加载了该宏包,那么它的 \cs{currfilepath}\ 将被检测到并由 \xpackage{bookmark}\ 自动使用。
% \item 可以通过 \cs{bookmarksetup}\ 或 \cs{bookmark}\ 中的 \xoption{scrfile}\ 选项手动设置(set manually)文件名。
% 但是要小心,手动设置会禁用以前的文件名检测方法。错误的(wrong)或丢失的(missed)文件名设置(file name setting)可能会在错误消息中
% 为您提供错误的源位置(source location)。
% \end{itemize}
%
% \subsubsection{级别选项(Level options)}
%
% 书签条目(bookmark entries)的顺序由 \cs{bookmark}\ 命令的的出现顺序(appearance order)定义。
% 树结构(tree structure)由书签节点(bookmark nodes)的属性 \xoption{level}(级别)构建。
% \xoption{level}\ 的值是整数(integers)。如果书签条目级别的值高于前一个节点,则该条目将成为
% 前一个节点的子(child)节点。差值的绝对值并不重要。
%
% \xpackage{bookmark}\ 宏包能记住全局属性(global property)“current level(当前级别)”中上
% 一个书签条目(previous bookmark entry)的级别。
%
% 级别系统的(level system)行为(behaviour)可以通过以下选项进行配置:
% \begin{description}
% \item[\xoption{level}:]
%    设置级别(level),请参阅上面的说明。如果给出的选项 \xoption{level}\ 没有值,那么将恢复默
%    认行为,即将“当前级别(current level)”用作级别值(level value)。自 2010/10/19 v1.16 版本以来,
%    如果宏 \cs{toclevel@part}、\cs{toclevel@section}\ 被定义过(通过 \xpackage{hyperref}\ 宏包完成,
%    请参阅它的 \xoption{bookmarkdepth}\ 选项),则 \xpackage{bookmark}\ 宏包还支持 |part|、|section| 等名称。
%
% \item[\xoption{rellevel}:]
%    设置相对于前一级别的(previous level)级别。正值表示书签条目成为前一个书签条目的子条目。
% \item[\xoption{keeplevel}:]
%    使用由\xoption{level}\ 或 \xoption{rellevel}\ 设置的级别,但不要更改全局属性“current level(当前级别)”。
%    可以通过设置为 |false| 来禁用该选项。
% \item[\xoption{startatroot}:]
%    此时,书签树(bookmark tree)再次从顶层(top level)开始。下一个书签条目不会作为上一个条目的子条目进行排序。
%    示例场景:文档使用 part。但是,最后几章(last chapters)不应放在最后一部分(last part)下面:
%    \begin{quote}
%\begin{verbatim}
%\documentclass{book}
%[...]
%\begin{document}
%  \part{第一部分}
%    \chapter{第一部分的第1章}
%    [...]
%  \part{第二部分(Second part)}
%    \chapter{第二部分的第1章}
%    [...]
%  \bookmarksetup{startatroot}
%  \chapter{Index}% 不属于第二部分
%\end{document}
%\end{verbatim}
%    \end{quote}
% \end{description}
%
% \subsubsection{样式定义(Style definitions)}
%
% 样式(style)是一组选项设置(option settings)。它可以由宏 \cs{bookmarkdefinestyle}\ 定义,
% 并由它的 \xoption{style}\ 选项使用。
% \begin{declcs}{bookmarkdefinestyle} \M{name} \M{key value list}
% \end{declcs}
% 选项设置(option settings)的 \meta{key value list}(键值列表)被指定为样式名(style \meta{name})。
%
% \begin{description}
% \item[\xoption{style}:]
%   \xoption{style}\ 选项的值是以前定义的样式的名称(name)。现在执行其选项设置(option settings)。
%   选项可以包括 \xoption{style}\ 选项。通过递归调用相同样式的无限递归(endless recursion)被阻止并抛出一个错误。
% \end{description}
%
% \subsubsection{钩子支持(Hook support)}
%
% 处理宏\cs{bookmark}\ 的可选选项(optional options)后,就会调用钩子(hook)。
% \begin{description}
% \item[\xoption{addtohook}:]
%   代码(code)作为该选项的值添加到钩子中。
% \end{description}
%
% \begin{declcs}{bookmarkget} \M{option}
% \end{declcs}
% \cs{bookmarkget}\ 宏提取 \meta{option}\ 选项的最新选项设置(latest option setting)的值。
% 对于布尔选项(boolean option),如果启用布尔选项,则返回 1,否则结果为零。结果数字(resulting numbers)
% 可以直接用于 \cs{ifnum}\ 或 \cs{ifcase}。如果您想要数字 \texttt{0}\ 和 \texttt{1},
% 请在 \cs{bookmarkget}\ 前面加上 \cs{number}\ 作为前缀。\cs{bookmarkget}\ 宏是可展开的(expandable)。
% 如果选项不受支持,则返回空字符串(empty string)。受支持的布尔选项有:
% \begin{quote}
%   \xoption{bold}、
%   \xoption{italic}、
%   \xoption{open}
% \end{quote}
% 其他受支持的选项有:
% \begin{quote}
%   \xoption{depth}、
%   \xoption{dest}、
%   \xoption{color}、
%   \xoption{gotor}、
%   \xoption{level}、
%   \xoption{named}、
%   \xoption{openlevel}、
%   \xoption{page}、
%   \xoption{rawaction}、
%   \xoption{uri}、
%   \xoption{view}、
% \end{quote}
% 另外,以下键(key)是可用的:
% \begin{quote}
%   \xoption{text}
% \end{quote}
% 它返回大纲条目(outline entry)的文本(text)。
%
% \paragraph{选项设置(Option setting)。}
% 在钩子(hook)内部可以使用 \cs{bookmarksetup}\ 设置选项。
%
% \subsection{与 \xpackage{hyperref}\ 的兼容性}
%
% \xpackage{bookmark}\ 宏包自动禁用 \xpackage{hyperref}\ 宏包的书签(bookmarks)。但是,
% \xpackage{bookmark}\ 宏包使用了 \xpackage{hyperref}\ 宏包的一些代码。例如,
% \xpackage{bookmark}\ 宏包重新定义了 \xpackage{hyperref}\ 宏包在 \cs{addcontentsline}\ 命令
% 和其他命令中插入的\cs{Hy@writebookmark}\ 钩子。因此,不应禁用 \xpackage{hyperref}\ 宏包的书签。
%
% \xpackage{bookmark}\ 宏包使用 \xpackage{hyperref}\ 宏包的 \cs{pdfstringdef},且不提供替换(replacement)。
%
% \xpackage{hyperref}\ 宏包的一些选项也能在 \xpackage{bookmark}\ 宏包中实现(implemented):
% \begin{quote}
% \begin{tabular}{|l@{}|l@{}|}
%   \hline
%   \xpackage{hyperref}\ 宏包的选项\  &\ \xpackage{bookmark}\ 宏包的选项\ \ \\ \hline
%   \xoption{bookmarksdepth} &\ \xoption{depth}\\ \hline
%   \xoption{bookmarksopen} & \ \xoption{open}\\ \hline
%   \xoption{bookmarksopenlevel}\ \ \  &\ \xoption{openlevel}\\ \hline
%   \xoption{bookmarksnumbered} \ \ \ &\ \xoption{numbered}\\ \hline
% \end{tabular}
% \end{quote}
%
% 还可以使用以下命令:
% \begin{quote}
%   \cs{pdfbookmark}\\
%   \cs{currentpdfbookmark}\\
%   \cs{subpdfbookmark}\\
%   \cs{belowpdfbookmark}
% \end{quote}
%
% \subsection{在末尾添加书签}
%
% 宏包选项 \xoption{atend}\ 启用以下宏(macro):
% \begin{declcs}{BookmarkAtEnd}
%   \M{stuff}
% \end{declcs}
% \cs{BookmarkAtEnd}\ 宏将 \meta{stuff}\ 放在文档末尾。\meta{stuff}\ 表示书签命令(bookmark commands)。举例:
% \begin{quote}
%\begin{verbatim}
%\usepackage[atend]{bookmark}
%\BookmarkAtEnd{%
%  \bookmarksetup{startatroot}%
%  \bookmark[named=LastPage, level=0]{Last page}%
%}
%\end{verbatim}
% \end{quote}
%
% 或者,可以在 \cs{bookmark}\ 中给出 \xoption{startatroot}\ 选项:
% \begin{quote}
%\begin{verbatim}
%\BookmarkAtEnd{%
%  \bookmark[
%    startatroot,
%    named=LastPage,
%    level=0,
%  ]{Last page}%
%}
%\end{verbatim}
% \end{quote}
%
% \paragraph{备注(Remarks):}
% \begin{itemize}
% \item
%   \cs{BookmarkAtEnd} 隐藏了这样一个事实,即在文档末尾添加书签的方法取决于驱动程序(driver)。
%
%   为此,驱动程序 \xoption{pdftex}\ 使用 \xpackage{atveryend}\ 宏包。如果 \cs{AtEndDocument}\ 太早,
%   最后一个页面(last page)可能不会被发送出去(shipped out)。由于需要 \xext{aux}\ 文件,此驱动程序使
%   用 \cs{AfterLastShipout}。
%
%   其他驱动程序(\xoption{dvipdfm}、\xoption{xetex}、\xoption{vtex})的实现(implementation)
%   取决于 \cs{special},\cs{special}\ 在最后一页之后没有效果。在这种情况下,\xpackage{atenddvi}\ 宏包的
%   \cs{AtEndDvi}\ 有帮助。它将其参数(argument)放在文档的最后一页(last page)。至少需要运行 \hologo{LaTeX}\ 两次,
%   因为最后一页是由引用(reference)检测到的。
%
%   \xoption{dvips}\ 现在使用新的 LaTeX 钩子 \texttt{shipout/lastpage}。
% \item
%   未指定 \cs{BookmarkAtEnd}\ 参数的扩展时间(time of expansion)。这可以立即发生,也可以在文档末尾发生。
% \end{itemize}
%
% \subsection{限制/行动计划}
%
% \begin{itemize}
% \item 支持缺失动作(missing actions)(启动,\dots)。
% \item 对 \xpackage{hyperref}\ 的 \xoption{bookmarkstype}\ 选项进行了更好的设计(design)。
% \end{itemize}
%
% \section{示例(Example)}
%
%    \begin{macrocode}
%<*example>
%    \end{macrocode}
%    \begin{macrocode}
\documentclass{article}
\usepackage{xcolor}[2007/01/21]
\usepackage{hyperref}
\usepackage[
  open,
  openlevel=2,
  atend
]{bookmark}[2019/12/03]

\bookmarksetup{color=blue}

\BookmarkAtEnd{%
  \bookmarksetup{startatroot}%
  \bookmark[named=LastPage, level=0]{End/Last page}%
  \bookmark[named=FirstPage, level=1]{First page}%
}

\begin{document}
\section{First section}
\subsection{Subsection A}
\begin{figure}
  \hypertarget{fig}{}%
  A figure.
\end{figure}
\bookmark[
  rellevel=1,
  keeplevel,
  dest=fig
]{A figure}
\subsection{Subsection B}
\subsubsection{Subsubsection C}
\subsection{Umlauts: \"A\"O\"U\"a\"o\"u\ss}
\newpage
\bookmarksetup{
  bold,
  color=[rgb]{1,0,0}
}
\section{Very important section}
\bookmarksetup{
  italic,
  bold=false,
  color=blue
}
\subsection{Italic section}
\bookmarksetup{
  italic=false
}
\part{Misc}
\section{Diverse}
\subsubsection{Subsubsection, omitting subsection}
\bookmarksetup{
  startatroot
}
\section{Last section outside part}
\subsection{Subsection}
\bookmarksetup{
  color={}
}
\begingroup
  \bookmarksetup{level=0, color=green!80!black}
  \bookmark[named=FirstPage]{First page}
  \bookmark[named=LastPage]{Last page}
  \bookmark[named=PrevPage]{Previous page}
  \bookmark[named=NextPage]{Next page}
\endgroup
\bookmark[
  page=2,
  view=FitH 800
]{Page 2, FitH 800}
\bookmark[
  page=2,
  view=FitBH \calc{\paperheight-\topmargin-1in-\headheight-\headsep}
]{Page 2, FitBH top of text body}
\bookmark[
  uri={http://www.dante.de/},
  color=magenta
]{Dante homepage}
\bookmark[
  gotor={t.pdf},
  page=1,
  view={XYZ 0 1000 null},
  color=cyan!75!black
]{File t.pdf}
\bookmark[named=FirstPage]{First page}
\bookmark[rellevel=1, named=LastPage]{Last page (rellevel=1)}
\bookmark[named=PrevPage]{Previous page}
\bookmark[level=0, named=FirstPage]{First page (level=0)}
\bookmark[
  rellevel=1,
  keeplevel,
  named=LastPage
]{Last page (rellevel=1, keeplevel)}
\bookmark[named=PrevPage]{Previous page}
\end{document}
%    \end{macrocode}
%    \begin{macrocode}
%</example>
%    \end{macrocode}
%
% \StopEventually{
% }
%
% \section{实现(Implementation)}
%
% \subsection{宏包(Package)}
%
%    \begin{macrocode}
%<*package>
\NeedsTeXFormat{LaTeX2e}
\ProvidesPackage{bookmark}%
  [2020-11-06 v1.29 PDF bookmarks (HO)]%
%    \end{macrocode}
%
% \subsubsection{要求(Requirements)}
%
% \paragraph{\hologo{eTeX}.}
%
%    \begin{macro}{\BKM@CalcExpr}
%    \begin{macrocode}
\begingroup\expandafter\expandafter\expandafter\endgroup
\expandafter\ifx\csname numexpr\endcsname\relax
  \def\BKM@CalcExpr#1#2#3#4{%
    \begingroup
      \count@=#2\relax
      \advance\count@ by#3#4\relax
      \edef\x{\endgroup
        \def\noexpand#1{\the\count@}%
      }%
    \x
  }%
\else
  \def\BKM@CalcExpr#1#2#3#4{%
    \edef#1{%
      \the\numexpr#2#3#4\relax
    }%
  }%
\fi
%    \end{macrocode}
%    \end{macro}
%
% \paragraph{\hologo{pdfTeX}\ 的转义功能(escape features)}
%
%    \begin{macro}{\BKM@EscapeName}
%    \begin{macrocode}
\def\BKM@EscapeName#1{%
  \ifx#1\@empty
  \else
    \EdefEscapeName#1#1%
  \fi
}%
%    \end{macrocode}
%    \end{macro}
%    \begin{macro}{\BKM@EscapeString}
%    \begin{macrocode}
\def\BKM@EscapeString#1{%
  \ifx#1\@empty
  \else
    \EdefEscapeString#1#1%
  \fi
}%
%    \end{macrocode}
%    \end{macro}
%    \begin{macro}{\BKM@EscapeHex}
%    \begin{macrocode}
\def\BKM@EscapeHex#1{%
  \ifx#1\@empty
  \else
    \EdefEscapeHex#1#1%
  \fi
}%
%    \end{macrocode}
%    \end{macro}
%    \begin{macro}{\BKM@UnescapeHex}
%    \begin{macrocode}
\def\BKM@UnescapeHex#1{%
  \EdefUnescapeHex#1#1%
}%
%    \end{macrocode}
%    \end{macro}
%
% \paragraph{宏包(Packages)。}
%
% 不要加载由 \xpackage{hyperref}\ 加载的宏包
%    \begin{macrocode}
\RequirePackage{hyperref}[2010/06/18]
%    \end{macrocode}
%
% \subsubsection{宏包选项(Package options)}
%
%    \begin{macrocode}
\SetupKeyvalOptions{family=BKM,prefix=BKM@}
\DeclareLocalOptions{%
  atend,%
  bold,%
  color,%
  depth,%
  dest,%
  draft,%
  final,%
  gotor,%
  italic,%
  keeplevel,%
  level,%
  named,%
  numbered,%
  open,%
  openlevel,%
  page,%
  rawaction,%
  rellevel,%
  srcfile,%
  srcline,%
  startatroot,%
  uri,%
  view,%
}
%    \end{macrocode}
%    \begin{macro}{\bookmarksetup}
%    \begin{macrocode}
\newcommand*{\bookmarksetup}{\kvsetkeys{BKM}}
%    \end{macrocode}
%    \end{macro}
%    \begin{macro}{\BKM@setup}
%    \begin{macrocode}
\def\BKM@setup#1{%
  \bookmarksetup{#1}%
  \ifx\BKM@HookNext\ltx@empty
  \else
    \expandafter\bookmarksetup\expandafter{\BKM@HookNext}%
    \BKM@HookNextClear
  \fi
  \BKM@hook
  \ifBKM@keeplevel
  \else
    \xdef\BKM@currentlevel{\BKM@level}%
  \fi
}
%    \end{macrocode}
%    \end{macro}
%
%    \begin{macro}{\bookmarksetupnext}
%    \begin{macrocode}
\newcommand*{\bookmarksetupnext}[1]{%
  \ltx@GlobalAppendToMacro\BKM@HookNext{,#1}%
}
%    \end{macrocode}
%    \end{macro}
%    \begin{macro}{\BKM@setupnext}
%    \begin{macrocode}
%    \end{macrocode}
%    \end{macro}
%    \begin{macro}{\BKM@HookNextClear}
%    \begin{macrocode}
\def\BKM@HookNextClear{%
  \global\let\BKM@HookNext\ltx@empty
}
%    \end{macrocode}
%    \end{macro}
%    \begin{macro}{\BKM@HookNext}
%    \begin{macrocode}
\BKM@HookNextClear
%    \end{macrocode}
%    \end{macro}
%
%    \begin{macrocode}
\DeclareBoolOption{draft}
\DeclareComplementaryOption{final}{draft}
%    \end{macrocode}
%    \begin{macro}{\BKM@DisableOptions}
%    \begin{macrocode}
\def\BKM@DisableOptions{%
  \DisableKeyvalOption[action=warning,package=bookmark]%
      {BKM}{draft}%
  \DisableKeyvalOption[action=warning,package=bookmark]%
      {BKM}{final}%
}
%    \end{macrocode}
%    \end{macro}
%    \begin{macrocode}
\DeclareBoolOption[\ifHy@bookmarksopen true\else false\fi]{open}
%    \end{macrocode}
%    \begin{macro}{\bookmark@open}
%    \begin{macrocode}
\def\bookmark@open{%
  \ifBKM@open\ltx@one\else\ltx@zero\fi
}
%    \end{macrocode}
%    \end{macro}
%    \begin{macrocode}
\DeclareStringOption[\maxdimen]{openlevel}
%    \end{macrocode}
%    \begin{macro}{\BKM@openlevel}
%    \begin{macrocode}
\edef\BKM@openlevel{\number\@bookmarksopenlevel}
%    \end{macrocode}
%    \end{macro}
%    \begin{macrocode}
%\DeclareStringOption[\c@tocdepth]{depth}
\ltx@IfUndefined{Hy@bookmarksdepth}{%
  \def\BKM@depth{\c@tocdepth}%
}{%
  \let\BKM@depth\Hy@bookmarksdepth
}
\define@key{BKM}{depth}[]{%
  \edef\BKM@param{#1}%
  \ifx\BKM@param\@empty
    \def\BKM@depth{\c@tocdepth}%
  \else
    \ltx@IfUndefined{toclevel@\BKM@param}{%
      \@onelevel@sanitize\BKM@param
      \edef\BKM@temp{\expandafter\@car\BKM@param\@nil}%
      \ifcase 0\expandafter\ifx\BKM@temp-1\fi
              \expandafter\ifnum\expandafter`\BKM@temp>47 %
                \expandafter\ifnum\expandafter`\BKM@temp<58 %
                  1%
                \fi
              \fi
              \relax
        \PackageWarning{bookmark}{%
          Unknown document division name (\BKM@param)\MessageBreak
          for option `depth'%
        }%
      \else
        \BKM@SetDepthOrLevel\BKM@depth\BKM@param
      \fi
    }{%
      \BKM@SetDepthOrLevel\BKM@depth{%
        \csname toclevel@\BKM@param\endcsname
      }%
    }%
  \fi
}
%    \end{macrocode}
%    \begin{macro}{\bookmark@depth}
%    \begin{macrocode}
\def\bookmark@depth{\BKM@depth}
%    \end{macrocode}
%    \end{macro}
%    \begin{macro}{\BKM@SetDepthOrLevel}
%    \begin{macrocode}
\def\BKM@SetDepthOrLevel#1#2{%
  \begingroup
    \setbox\z@=\hbox{%
      \count@=#2\relax
      \expandafter
    }%
  \expandafter\endgroup
  \expandafter\def\expandafter#1\expandafter{\the\count@}%
}
%    \end{macrocode}
%    \end{macro}
%    \begin{macrocode}
\DeclareStringOption[\BKM@currentlevel]{level}[\BKM@currentlevel]
\define@key{BKM}{level}{%
  \edef\BKM@param{#1}%
  \ifx\BKM@param\BKM@MacroCurrentLevel
    \let\BKM@level\BKM@param
  \else
    \ltx@IfUndefined{toclevel@\BKM@param}{%
      \@onelevel@sanitize\BKM@param
      \edef\BKM@temp{\expandafter\@car\BKM@param\@nil}%
      \ifcase 0\expandafter\ifx\BKM@temp-1\fi
              \expandafter\ifnum\expandafter`\BKM@temp>47 %
                \expandafter\ifnum\expandafter`\BKM@temp<58 %
                  1%
                \fi
              \fi
              \relax
        \PackageWarning{bookmark}{%
          Unknown document division name (\BKM@param)\MessageBreak
          for option `level'%
        }%
      \else
        \BKM@SetDepthOrLevel\BKM@level\BKM@param
      \fi
    }{%
      \BKM@SetDepthOrLevel\BKM@level{%
        \csname toclevel@\BKM@param\endcsname
      }%
    }%
  \fi
}
%    \end{macrocode}
%    \begin{macro}{\BKM@MacroCurrentLevel}
%    \begin{macrocode}
\def\BKM@MacroCurrentLevel{\BKM@currentlevel}
%    \end{macrocode}
%    \end{macro}
%    \begin{macrocode}
\DeclareBoolOption{keeplevel}
\DeclareBoolOption{startatroot}
%    \end{macrocode}
%    \begin{macro}{\BKM@startatrootfalse}
%    \begin{macrocode}
\def\BKM@startatrootfalse{%
  \global\let\ifBKM@startatroot\iffalse
}
%    \end{macrocode}
%    \end{macro}
%    \begin{macro}{\BKM@startatroottrue}
%    \begin{macrocode}
\def\BKM@startatroottrue{%
  \global\let\ifBKM@startatroot\iftrue
}
%    \end{macrocode}
%    \end{macro}
%    \begin{macrocode}
\define@key{BKM}{rellevel}{%
  \BKM@CalcExpr\BKM@level{#1}+\BKM@currentlevel
}
%    \end{macrocode}
%    \begin{macro}{\bookmark@level}
%    \begin{macrocode}
\def\bookmark@level{\BKM@level}
%    \end{macrocode}
%    \end{macro}
%    \begin{macro}{\BKM@currentlevel}
%    \begin{macrocode}
\def\BKM@currentlevel{0}
%    \end{macrocode}
%    \end{macro}
%    Make \xpackage{bookmark}'s option \xoption{numbered} an alias
%    for \xpackage{hyperref}'s \xoption{bookmarksnumbered}.
%    \begin{macrocode}
\DeclareBoolOption[%
  \ifHy@bookmarksnumbered true\else false\fi
]{numbered}
\g@addto@macro\BKM@numberedtrue{%
  \let\ifHy@bookmarksnumbered\iftrue
}
\g@addto@macro\BKM@numberedfalse{%
  \let\ifHy@bookmarksnumbered\iffalse
}
\g@addto@macro\Hy@bookmarksnumberedtrue{%
  \let\ifBKM@numbered\iftrue
}
\g@addto@macro\Hy@bookmarksnumberedfalse{%
  \let\ifBKM@numbered\iffalse
}
%    \end{macrocode}
%    \begin{macro}{\bookmark@numbered}
%    \begin{macrocode}
\def\bookmark@numbered{%
  \ifBKM@numbered\ltx@one\else\ltx@zero\fi
}
%    \end{macrocode}
%    \end{macro}
%
% \paragraph{重定义 \xpackage{hyperref}\ 宏包的选项}
%
%    \begin{macro}{\BKM@PatchHyperrefOption}
%    \begin{macrocode}
\def\BKM@PatchHyperrefOption#1{%
  \expandafter\BKM@@PatchHyperrefOption\csname KV@Hyp@#1\endcsname%
}
%    \end{macrocode}
%    \end{macro}
%    \begin{macro}{\BKM@@PatchHyperrefOption}
%    \begin{macrocode}
\def\BKM@@PatchHyperrefOption#1{%
  \expandafter\BKM@@@PatchHyperrefOption#1{##1}\BKM@nil#1%
}
%    \end{macrocode}
%    \end{macro}
%    \begin{macro}{\BKM@@@PatchHyperrefOption}
%    \begin{macrocode}
\def\BKM@@@PatchHyperrefOption#1\BKM@nil#2#3{%
  \def#2##1{%
    #1%
    \bookmarksetup{#3={##1}}%
  }%
}
%    \end{macrocode}
%    \end{macro}
%    \begin{macrocode}
\BKM@PatchHyperrefOption{bookmarksopen}{open}
\BKM@PatchHyperrefOption{bookmarksopenlevel}{openlevel}
\BKM@PatchHyperrefOption{bookmarksdepth}{depth}
%    \end{macrocode}
%
% \paragraph{字体样式(font style)选项。}
%
%    注意:\xpackage{bitset}\ 宏是基于零的,PDF 规范(PDF specifications)以1开头。
%    \begin{macrocode}
\bitsetReset{BKM@FontStyle}%
\define@key{BKM}{italic}[true]{%
  \expandafter\ifx\csname if#1\endcsname\iftrue
    \bitsetSet{BKM@FontStyle}{0}%
  \else
    \bitsetClear{BKM@FontStyle}{0}%
  \fi
}%
\define@key{BKM}{bold}[true]{%
  \expandafter\ifx\csname if#1\endcsname\iftrue
    \bitsetSet{BKM@FontStyle}{1}%
  \else
    \bitsetClear{BKM@FontStyle}{1}%
  \fi
}%
%    \end{macrocode}
%    \begin{macro}{\bookmark@italic}
%    \begin{macrocode}
\def\bookmark@italic{%
  \ifnum\bitsetGet{BKM@FontStyle}{0}=1 \ltx@one\else\ltx@zero\fi
}
%    \end{macrocode}
%    \end{macro}
%    \begin{macro}{\bookmark@bold}
%    \begin{macrocode}
\def\bookmark@bold{%
  \ifnum\bitsetGet{BKM@FontStyle}{1}=1 \ltx@one\else\ltx@zero\fi
}
%    \end{macrocode}
%    \end{macro}
%    \begin{macro}{\BKM@PrintStyle}
%    \begin{macrocode}
\def\BKM@PrintStyle{%
  \bitsetGetDec{BKM@FontStyle}%
}%
%    \end{macrocode}
%    \end{macro}
%
% \paragraph{颜色(color)选项。}
%
%    \begin{macrocode}
\define@key{BKM}{color}{%
  \HyColor@BookmarkColor{#1}\BKM@color{bookmark}{color}%
}
%    \end{macrocode}
%    \begin{macro}{\BKM@color}
%    \begin{macrocode}
\let\BKM@color\@empty
%    \end{macrocode}
%    \end{macro}
%    \begin{macro}{\bookmark@color}
%    \begin{macrocode}
\def\bookmark@color{\BKM@color}
%    \end{macrocode}
%    \end{macro}
%
% \subsubsection{动作(action)选项}
%
%    \begin{macrocode}
\def\BKM@temp#1{%
  \DeclareStringOption{#1}%
  \expandafter\edef\csname bookmark@#1\endcsname{%
    \expandafter\noexpand\csname BKM@#1\endcsname
  }%
}
%    \end{macrocode}
%    \begin{macro}{\bookmark@dest}
%    \begin{macrocode}
\BKM@temp{dest}
%    \end{macrocode}
%    \end{macro}
%    \begin{macro}{\bookmark@named}
%    \begin{macrocode}
\BKM@temp{named}
%    \end{macrocode}
%    \end{macro}
%    \begin{macro}{\bookmark@uri}
%    \begin{macrocode}
\BKM@temp{uri}
%    \end{macrocode}
%    \end{macro}
%    \begin{macro}{\bookmark@gotor}
%    \begin{macrocode}
\BKM@temp{gotor}
%    \end{macrocode}
%    \end{macro}
%    \begin{macro}{\bookmark@rawaction}
%    \begin{macrocode}
\BKM@temp{rawaction}
%    \end{macrocode}
%    \end{macro}
%
%    \begin{macrocode}
\define@key{BKM}{page}{%
  \def\BKM@page{#1}%
  \ifx\BKM@page\@empty
  \else
    \edef\BKM@page{\number\BKM@page}%
    \ifnum\BKM@page>\z@
    \else
      \PackageError{bookmark}{Page must be positive}\@ehc
      \def\BKM@page{1}%
    \fi
  \fi
}
%    \end{macrocode}
%    \begin{macro}{\BKM@page}
%    \begin{macrocode}
\let\BKM@page\@empty
%    \end{macrocode}
%    \end{macro}
%    \begin{macro}{\bookmark@page}
%    \begin{macrocode}
\def\bookmark@page{\BKM@@page}
%    \end{macrocode}
%    \end{macro}
%
%    \begin{macrocode}
\define@key{BKM}{view}{%
  \BKM@CheckView{#1}%
}
%    \end{macrocode}
%    \begin{macro}{\BKM@view}
%    \begin{macrocode}
\let\BKM@view\@empty
%    \end{macrocode}
%    \end{macro}
%    \begin{macro}{\bookmark@view}
%    \begin{macrocode}
\def\bookmark@view{\BKM@view}
%    \end{macrocode}
%    \end{macro}
%    \begin{macro}{BKM@CheckView}
%    \begin{macrocode}
\def\BKM@CheckView#1{%
  \BKM@CheckViewType#1 \@nil
}
%    \end{macrocode}
%    \end{macro}
%    \begin{macro}{\BKM@CheckViewType}
%    \begin{macrocode}
\def\BKM@CheckViewType#1 #2\@nil{%
  \def\BKM@type{#1}%
  \@onelevel@sanitize\BKM@type
  \BKM@TestViewType{Fit}{}%
  \BKM@TestViewType{FitB}{}%
  \BKM@TestViewType{FitH}{%
    \BKM@CheckParam#2 \@nil{top}%
  }%
  \BKM@TestViewType{FitBH}{%
    \BKM@CheckParam#2 \@nil{top}%
  }%
  \BKM@TestViewType{FitV}{%
    \BKM@CheckParam#2 \@nil{bottom}%
  }%
  \BKM@TestViewType{FitBV}{%
    \BKM@CheckParam#2 \@nil{bottom}%
  }%
  \BKM@TestViewType{FitR}{%
    \BKM@CheckRect{#2}{ }%
  }%
  \BKM@TestViewType{XYZ}{%
    \BKM@CheckXYZ{#2}{ }%
  }%
  \@car{%
    \PackageError{bookmark}{%
      Unknown view type `\BKM@type',\MessageBreak
      using `FitH' instead%
    }\@ehc
    \def\BKM@view{FitH}%
  }%
  \@nil
}
%    \end{macrocode}
%    \end{macro}
%    \begin{macro}{\BKM@TestViewType}
%    \begin{macrocode}
\def\BKM@TestViewType#1{%
  \def\BKM@temp{#1}%
  \@onelevel@sanitize\BKM@temp
  \ifx\BKM@type\BKM@temp
    \let\BKM@view\BKM@temp
    \expandafter\@car
  \else
    \expandafter\@gobble
  \fi
}
%    \end{macrocode}
%    \end{macro}
%    \begin{macro}{BKM@CheckParam}
%    \begin{macrocode}
\def\BKM@CheckParam#1 #2\@nil#3{%
  \def\BKM@param{#1}%
  \ifx\BKM@param\@empty
    \PackageWarning{bookmark}{%
      Missing parameter (#3) for `\BKM@type',\MessageBreak
      using 0%
    }%
    \def\BKM@param{0}%
  \else
    \BKM@CalcParam
  \fi
  \edef\BKM@view{\BKM@view\space\BKM@param}%
}
%    \end{macrocode}
%    \end{macro}
%    \begin{macro}{BKM@CheckRect}
%    \begin{macrocode}
\def\BKM@CheckRect#1#2{%
  \BKM@@CheckRect#1#2#2#2#2\@nil
}
%    \end{macrocode}
%    \end{macro}
%    \begin{macro}{\BKM@@CheckRect}
%    \begin{macrocode}
\def\BKM@@CheckRect#1 #2 #3 #4 #5\@nil{%
  \def\BKM@temp{0}%
  \def\BKM@param{#1}%
  \ifx\BKM@param\@empty
    \def\BKM@param{0}%
    \def\BKM@temp{1}%
  \else
    \BKM@CalcParam
  \fi
  \edef\BKM@view{\BKM@view\space\BKM@param}%
  \def\BKM@param{#2}%
  \ifx\BKM@param\@empty
    \def\BKM@param{0}%
    \def\BKM@temp{1}%
  \else
    \BKM@CalcParam
  \fi
  \edef\BKM@view{\BKM@view\space\BKM@param}%
  \def\BKM@param{#3}%
  \ifx\BKM@param\@empty
    \def\BKM@param{0}%
    \def\BKM@temp{1}%
  \else
    \BKM@CalcParam
  \fi
  \edef\BKM@view{\BKM@view\space\BKM@param}%
  \def\BKM@param{#4}%
  \ifx\BKM@param\@empty
    \def\BKM@param{0}%
    \def\BKM@temp{1}%
  \else
    \BKM@CalcParam
  \fi
  \edef\BKM@view{\BKM@view\space\BKM@param}%
  \ifnum\BKM@temp>\z@
    \PackageWarning{bookmark}{Missing parameters for `\BKM@type'}%
  \fi
}
%    \end{macrocode}
%    \end{macro}
%    \begin{macro}{\BKM@CheckXYZ}
%    \begin{macrocode}
\def\BKM@CheckXYZ#1#2{%
  \BKM@@CheckXYZ#1#2#2#2\@nil
}
%    \end{macrocode}
%    \end{macro}
%    \begin{macro}{\BKM@@CheckXYZ}
%    \begin{macrocode}
\def\BKM@@CheckXYZ#1 #2 #3 #4\@nil{%
  \def\BKM@param{#1}%
  \let\BKM@temp\BKM@param
  \@onelevel@sanitize\BKM@temp
  \ifx\BKM@param\@empty
    \let\BKM@param\BKM@null
  \else
    \ifx\BKM@temp\BKM@null
    \else
      \BKM@CalcParam
    \fi
  \fi
  \edef\BKM@view{\BKM@view\space\BKM@param}%
  \def\BKM@param{#2}%
  \let\BKM@temp\BKM@param
  \@onelevel@sanitize\BKM@temp
  \ifx\BKM@param\@empty
    \let\BKM@param\BKM@null
  \else
    \ifx\BKM@temp\BKM@null
    \else
      \BKM@CalcParam
    \fi
  \fi
  \edef\BKM@view{\BKM@view\space\BKM@param}%
  \def\BKM@param{#3}%
  \ifx\BKM@param\@empty
    \let\BKM@param\BKM@null
  \fi
  \edef\BKM@view{\BKM@view\space\BKM@param}%
}
%    \end{macrocode}
%    \end{macro}
%    \begin{macro}{\BKM@null}
%    \begin{macrocode}
\def\BKM@null{null}
\@onelevel@sanitize\BKM@null
%    \end{macrocode}
%    \end{macro}
%
%    \begin{macro}{\BKM@CalcParam}
%    \begin{macrocode}
\def\BKM@CalcParam{%
  \begingroup
  \let\calc\@firstofone
  \expandafter\BKM@@CalcParam\BKM@param\@empty\@empty\@nil
}
%    \end{macrocode}
%    \end{macro}
%    \begin{macro}{\BKM@@CalcParam}
%    \begin{macrocode}
\def\BKM@@CalcParam#1#2#3\@nil{%
  \ifx\calc#1%
    \@ifundefined{calc@assign@dimen}{%
      \@ifundefined{dimexpr}{%
        \setlength{\dimen@}{#2}%
      }{%
        \setlength{\dimen@}{\dimexpr#2\relax}%
      }%
    }{%
      \setlength{\dimen@}{#2}%
    }%
    \dimen@.99626\dimen@
    \edef\BKM@param{\strip@pt\dimen@}%
    \expandafter\endgroup
    \expandafter\def\expandafter\BKM@param\expandafter{\BKM@param}%
  \else
    \endgroup
  \fi
}
%    \end{macrocode}
%    \end{macro}
%
% \subsubsection{\xoption{atend}\ 选项}
%
%    \begin{macrocode}
\DeclareBoolOption{atend}
\g@addto@macro\BKM@DisableOptions{%
  \DisableKeyvalOption[action=warning,package=bookmark]%
      {BKM}{atend}%
}
%    \end{macrocode}
%
% \subsubsection{\xoption{style}\ 选项}
%
%    \begin{macro}{\bookmarkdefinestyle}
%    \begin{macrocode}
\newcommand*{\bookmarkdefinestyle}[2]{%
  \@ifundefined{BKM@style@#1}{%
  }{%
    \PackageInfo{bookmark}{Redefining style `#1'}%
  }%
  \@namedef{BKM@style@#1}{#2}%
}
%    \end{macrocode}
%    \end{macro}
%    \begin{macrocode}
\define@key{BKM}{style}{%
  \BKM@StyleCall{#1}%
}
\newif\ifBKM@ok
%    \end{macrocode}
%    \begin{macro}{\BKM@StyleCall}
%    \begin{macrocode}
\def\BKM@StyleCall#1{%
  \@ifundefined{BKM@style@#1}{%
    \PackageWarning{bookmark}{%
      Ignoring unknown style `#1'%
    }%
  }{%
%    \end{macrocode}
%    检查样式堆栈(style stack)。
%    \begin{macrocode}
    \BKM@oktrue
    \edef\BKM@StyleCurrent{#1}%
    \@onelevel@sanitize\BKM@StyleCurrent
    \let\BKM@StyleEntry\BKM@StyleEntryCheck
    \BKM@StyleStack
    \ifBKM@ok
      \expandafter\@firstofone
    \else
      \PackageError{bookmark}{%
        Ignoring recursive call of style `\BKM@StyleCurrent'%
      }\@ehc
      \expandafter\@gobble
    \fi
    {%
%    \end{macrocode}
%    在堆栈上推送当前样式(Push current style on stack)。
%    \begin{macrocode}
      \let\BKM@StyleEntry\relax
      \edef\BKM@StyleStack{%
        \BKM@StyleEntry{\BKM@StyleCurrent}%
        \BKM@StyleStack
      }%
%    \end{macrocode}
%   调用样式(Call style)。
%    \begin{macrocode}
      \expandafter\expandafter\expandafter\bookmarksetup
      \expandafter\expandafter\expandafter{%
        \csname BKM@style@\BKM@StyleCurrent\endcsname
      }%
%    \end{macrocode}
%    从堆栈中弹出当前样式(Pop current style from stack)。
%    \begin{macrocode}
      \BKM@StyleStackPop
    }%
  }%
}
%    \end{macrocode}
%    \end{macro}
%    \begin{macro}{\BKM@StyleStackPop}
%    \begin{macrocode}
\def\BKM@StyleStackPop{%
  \let\BKM@StyleEntry\relax
  \edef\BKM@StyleStack{%
    \expandafter\@gobbletwo\BKM@StyleStack
  }%
}
%    \end{macrocode}
%    \end{macro}
%    \begin{macro}{\BKM@StyleEntryCheck}
%    \begin{macrocode}
\def\BKM@StyleEntryCheck#1{%
  \def\BKM@temp{#1}%
  \ifx\BKM@temp\BKM@StyleCurrent
    \BKM@okfalse
  \fi
}
%    \end{macrocode}
%    \end{macro}
%    \begin{macro}{\BKM@StyleStack}
%    \begin{macrocode}
\def\BKM@StyleStack{}
%    \end{macrocode}
%    \end{macro}
%
% \subsubsection{源文件位置(source file location)选项}
%
%    \begin{macrocode}
\DeclareStringOption{srcline}
\DeclareStringOption{srcfile}
%    \end{macrocode}
%
% \subsubsection{钩子支持(Hook support)}
%
%    \begin{macro}{\BKM@hook}
%    \begin{macrocode}
\def\BKM@hook{}
%    \end{macrocode}
%    \end{macro}
%    \begin{macrocode}
\define@key{BKM}{addtohook}{%
  \ltx@LocalAppendToMacro\BKM@hook{#1}%
}
%    \end{macrocode}
%
%    \begin{macro}{bookmarkget}
%    \begin{macrocode}
\newcommand*{\bookmarkget}[1]{%
  \romannumeral0%
  \ltx@ifundefined{bookmark@#1}{%
    \ltx@space
  }{%
    \expandafter\expandafter\expandafter\ltx@space
    \csname bookmark@#1\endcsname
  }%
}
%    \end{macrocode}
%    \end{macro}
%
% \subsubsection{设置和加载驱动程序}
%
% \paragraph{检测驱动程序。}
%
%    \begin{macro}{\BKM@DefineDriverKey}
%    \begin{macrocode}
\def\BKM@DefineDriverKey#1{%
  \define@key{BKM}{#1}[]{%
    \def\BKM@driver{#1}%
  }%
  \g@addto@macro\BKM@DisableOptions{%
    \DisableKeyvalOption[action=warning,package=bookmark]%
        {BKM}{#1}%
  }%
}
%    \end{macrocode}
%    \end{macro}
%    \begin{macrocode}
\BKM@DefineDriverKey{pdftex}
\BKM@DefineDriverKey{dvips}
\BKM@DefineDriverKey{dvipdfm}
\BKM@DefineDriverKey{dvipdfmx}
\BKM@DefineDriverKey{xetex}
\BKM@DefineDriverKey{vtex}
\define@key{BKM}{dvipdfmx-outline-open}[true]{%
 \PackageWarning{bookmark}{Option 'dvipdfmx-outline-open' is obsolete
   and ignored}{}}
%    \end{macrocode}
%    \begin{macro}{\bookmark@driver}
%    \begin{macrocode}
\def\bookmark@driver{\BKM@driver}
%    \end{macrocode}
%    \end{macro}
%    \begin{macrocode}
\InputIfFileExists{bookmark.cfg}{}{}
%    \end{macrocode}
%    \begin{macro}{\BookmarkDriverDefault}
%    \begin{macrocode}
\providecommand*{\BookmarkDriverDefault}{dvips}
%    \end{macrocode}
%    \end{macro}
%    \begin{macro}{\BKM@driver}
% Lua\TeX\ 和 pdf\TeX\ 共享驱动程序。
%    \begin{macrocode}
\ifpdf
  \def\BKM@driver{pdftex}%
  \ifx\pdfoutline\@undefined
    \ifx\pdfextension\@undefined\else
      \protected\def\pdfoutline{\pdfextension outline }
    \fi
  \fi
\else
  \ifxetex
    \def\BKM@driver{dvipdfm}%
  \else
    \ifvtex
      \def\BKM@driver{vtex}%
    \else
      \edef\BKM@driver{\BookmarkDriverDefault}%
    \fi
  \fi
\fi
%    \end{macrocode}
%    \end{macro}
%
% \paragraph{过程选项(Process options)。}
%
%    \begin{macrocode}
\ProcessKeyvalOptions*
\BKM@DisableOptions
%    \end{macrocode}
%
% \paragraph{\xoption{draft}\ 选项}
%
%    \begin{macrocode}
\ifBKM@draft
  \PackageWarningNoLine{bookmark}{Draft mode on}%
  \let\bookmarksetup\ltx@gobble
  \let\BookmarkAtEnd\ltx@gobble
  \let\bookmarkdefinestyle\ltx@gobbletwo
  \let\bookmarkget\ltx@gobble
  \let\pdfbookmark\ltx@undefined
  \newcommand*{\pdfbookmark}[3][]{}%
  \let\currentpdfbookmark\ltx@gobbletwo
  \let\subpdfbookmark\ltx@gobbletwo
  \let\belowpdfbookmark\ltx@gobbletwo
  \newcommand*{\bookmark}[2][]{}%
  \renewcommand*{\Hy@writebookmark}[5]{}%
  \let\ReadBookmarks\relax
  \let\BKM@DefGotoNameAction\ltx@gobbletwo % package `hypdestopt'
  \expandafter\endinput
\fi
%    \end{macrocode}
%
% \paragraph{验证和加载驱动程序。}
%
%    \begin{macrocode}
\def\BKM@temp{dvipdfmx}%
\ifx\BKM@temp\BKM@driver
  \def\BKM@driver{dvipdfm}%
\fi
\def\BKM@temp{pdftex}%
\ifpdf
  \ifx\BKM@temp\BKM@driver
  \else
    \PackageWarningNoLine{bookmark}{%
      Wrong driver `\BKM@driver', using `pdftex' instead%
    }%
    \let\BKM@driver\BKM@temp
  \fi
\else
  \ifx\BKM@temp\BKM@driver
    \PackageError{bookmark}{%
      Wrong driver, pdfTeX is not running in PDF mode.\MessageBreak
      Package loading is aborted%
    }\@ehc
    \expandafter\expandafter\expandafter\endinput
  \fi
  \def\BKM@temp{dvipdfm}%
  \ifxetex
    \ifx\BKM@temp\BKM@driver
    \else
      \PackageWarningNoLine{bookmark}{%
        Wrong driver `\BKM@driver',\MessageBreak
        using `dvipdfm' for XeTeX instead%
      }%
      \let\BKM@driver\BKM@temp
    \fi
  \else
    \def\BKM@temp{vtex}%
    \ifvtex
      \ifx\BKM@temp\BKM@driver
      \else
        \PackageWarningNoLine{bookmark}{%
          Wrong driver `\BKM@driver',\MessageBreak
          using `vtex' for VTeX instead%
        }%
        \let\BKM@driver\BKM@temp
      \fi
    \else
      \ifx\BKM@temp\BKM@driver
        \PackageError{bookmark}{%
          Wrong driver, VTeX is not running in PDF mode.\MessageBreak
          Package loading is aborted%
        }\@ehc
        \expandafter\expandafter\expandafter\endinput
      \fi
    \fi
  \fi
\fi
\providecommand\IfFormatAtLeastTF{\@ifl@t@r\fmtversion}
\IfFormatAtLeastTF{2020/10/01}{}{\edef\BKM@driver{\BKM@driver-2019-12-03}}
\InputIfFileExists{bkm-\BKM@driver.def}{}{%
  \PackageError{bookmark}{%
    Unsupported driver `\BKM@driver'.\MessageBreak
    Package loading is aborted%
  }\@ehc
  \endinput
}
%    \end{macrocode}
%
% \subsubsection{与 \xpackage{hyperref}\ 的兼容性}
%
%    \begin{macro}{\pdfbookmark}
%    \begin{macrocode}
\let\pdfbookmark\ltx@undefined
\newcommand*{\pdfbookmark}[3][0]{%
  \bookmark[level=#1,dest={#3.#1}]{#2}%
  \hyper@anchorstart{#3.#1}\hyper@anchorend
}
%    \end{macrocode}
%    \end{macro}
%    \begin{macro}{\currentpdfbookmark}
%    \begin{macrocode}
\def\currentpdfbookmark{%
  \pdfbookmark[\BKM@currentlevel]%
}
%    \end{macrocode}
%    \end{macro}
%    \begin{macro}{\subpdfbookmark}
%    \begin{macrocode}
\def\subpdfbookmark{%
  \BKM@CalcExpr\BKM@CalcResult\BKM@currentlevel+1%
  \expandafter\pdfbookmark\expandafter[\BKM@CalcResult]%
}
%    \end{macrocode}
%    \end{macro}
%    \begin{macro}{\belowpdfbookmark}
%    \begin{macrocode}
\def\belowpdfbookmark#1#2{%
  \xdef\BKM@gtemp{\number\BKM@currentlevel}%
  \subpdfbookmark{#1}{#2}%
  \global\let\BKM@currentlevel\BKM@gtemp
}
%    \end{macrocode}
%    \end{macro}
%
%    节号(section number)、文本(text)、标签(label)、级别(level)、文件(file)
%    \begin{macro}{\Hy@writebookmark}
%    \begin{macrocode}
\def\Hy@writebookmark#1#2#3#4#5{%
  \ifnum#4>\BKM@depth\relax
  \else
    \def\BKM@type{#5}%
    \ifx\BKM@type\Hy@bookmarkstype
      \begingroup
        \ifBKM@numbered
          \let\numberline\Hy@numberline
          \let\booknumberline\Hy@numberline
          \let\partnumberline\Hy@numberline
          \let\chapternumberline\Hy@numberline
        \else
          \let\numberline\@gobble
          \let\booknumberline\@gobble
          \let\partnumberline\@gobble
          \let\chapternumberline\@gobble
        \fi
        \bookmark[level=#4,dest={\HyperDestNameFilter{#3}}]{#2}%
      \endgroup
    \fi
  \fi
}
%    \end{macrocode}
%    \end{macro}
%
%    \begin{macro}{\ReadBookmarks}
%    \begin{macrocode}
\let\ReadBookmarks\relax
%    \end{macrocode}
%    \end{macro}
%
%    \begin{macrocode}
%</package>
%    \end{macrocode}
%
% \subsection{dvipdfm 的驱动程序}
%
%    \begin{macrocode}
%<*dvipdfm>
\NeedsTeXFormat{LaTeX2e}
\ProvidesFile{bkm-dvipdfm.def}%
  [2020-11-06 v1.29 bookmark driver for dvipdfm (HO)]%
%    \end{macrocode}
%
%    \begin{macro}{\BKM@id}
%    \begin{macrocode}
\newcount\BKM@id
\BKM@id=\z@
%    \end{macrocode}
%    \end{macro}
%
%    \begin{macro}{\BKM@0}
%    \begin{macrocode}
\@namedef{BKM@0}{000}
%    \end{macrocode}
%    \end{macro}
%    \begin{macro}{\ifBKM@sw}
%    \begin{macrocode}
\newif\ifBKM@sw
%    \end{macrocode}
%    \end{macro}
%
%    \begin{macro}{\bookmark}
%    \begin{macrocode}
\newcommand*{\bookmark}[2][]{%
  \if@filesw
    \begingroup
      \def\bookmark@text{#2}%
      \BKM@setup{#1}%
      \edef\BKM@prev{\the\BKM@id}%
      \global\advance\BKM@id\@ne
      \BKM@swtrue
      \@whilesw\ifBKM@sw\fi{%
        \def\BKM@abslevel{1}%
        \ifnum\ifBKM@startatroot\z@\else\BKM@prev\fi=\z@
          \BKM@startatrootfalse
          \expandafter\xdef\csname BKM@\the\BKM@id\endcsname{%
            0{\BKM@level}\BKM@abslevel
          }%
          \BKM@swfalse
        \else
          \expandafter\expandafter\expandafter\BKM@getx
              \csname BKM@\BKM@prev\endcsname
          \ifnum\BKM@level>\BKM@x@level\relax
            \BKM@CalcExpr\BKM@abslevel\BKM@x@abslevel+1%
            \expandafter\xdef\csname BKM@\the\BKM@id\endcsname{%
              {\BKM@prev}{\BKM@level}\BKM@abslevel
            }%
            \BKM@swfalse
          \else
            \let\BKM@prev\BKM@x@parent
          \fi
        \fi
      }%
      \csname HyPsd@XeTeXBigCharstrue\endcsname
      \pdfstringdef\BKM@title{\bookmark@text}%
      \edef\BKM@FLAGS{\BKM@PrintStyle}%
      \let\BKM@action\@empty
      \ifx\BKM@gotor\@empty
        \ifx\BKM@dest\@empty
          \ifx\BKM@named\@empty
            \ifx\BKM@rawaction\@empty
              \ifx\BKM@uri\@empty
                \ifx\BKM@page\@empty
                  \PackageError{bookmark}{Missing action}\@ehc
                  \edef\BKM@action{/Dest[@page1/Fit]}%
                \else
                  \ifx\BKM@view\@empty
                    \def\BKM@view{Fit}%
                  \fi
                  \edef\BKM@action{/Dest[@page\BKM@page/\BKM@view]}%
                \fi
              \else
                \BKM@EscapeString\BKM@uri
                \edef\BKM@action{%
                  /A<<%
                    /S/URI%
                    /URI(\BKM@uri)%
                  >>%
                }%
              \fi
            \else
              \edef\BKM@action{/A<<\BKM@rawaction>>}%
            \fi
          \else
            \BKM@EscapeName\BKM@named
            \edef\BKM@action{%
              /A<</S/Named/N/\BKM@named>>%
            }%
          \fi
        \else
          \BKM@EscapeString\BKM@dest
          \edef\BKM@action{%
            /A<<%
              /S/GoTo%
              /D(\BKM@dest)%
            >>%
          }%
        \fi
      \else
        \ifx\BKM@dest\@empty
          \ifx\BKM@page\@empty
            \def\BKM@page{0}%
          \else
            \BKM@CalcExpr\BKM@page\BKM@page-1%
          \fi
          \ifx\BKM@view\@empty
            \def\BKM@view{Fit}%
          \fi
          \edef\BKM@action{/D[\BKM@page/\BKM@view]}%
        \else
          \BKM@EscapeString\BKM@dest
          \edef\BKM@action{/D(\BKM@dest)}%
        \fi
        \BKM@EscapeString\BKM@gotor
        \edef\BKM@action{%
          /A<<%
            /S/GoToR%
            /F(\BKM@gotor)%
            \BKM@action
          >>%
        }%
      \fi
      \special{pdf:%
        out
              [%
              \ifBKM@open
                \ifnum\BKM@level<%
                    \expandafter\ltx@firstofone\expandafter
                    {\number\BKM@openlevel} %
                \else
                  -%
                \fi
              \else
                -%
              \fi
              ] %
            \BKM@abslevel
        <<%
          /Title(\BKM@title)%
          \ifx\BKM@color\@empty
          \else
            /C[\BKM@color]%
          \fi
          \ifnum\BKM@FLAGS>\z@
            /F \BKM@FLAGS
          \fi
          \BKM@action
        >>%
      }%
    \endgroup
  \fi
}
%    \end{macrocode}
%    \end{macro}
%    \begin{macro}{\BKM@getx}
%    \begin{macrocode}
\def\BKM@getx#1#2#3{%
  \def\BKM@x@parent{#1}%
  \def\BKM@x@level{#2}%
  \def\BKM@x@abslevel{#3}%
}
%    \end{macrocode}
%    \end{macro}
%
%    \begin{macrocode}
%</dvipdfm>
%    \end{macrocode}
%
% \subsection{\hologo{VTeX}\ 的驱动程序}
%
%    \begin{macrocode}
%<*vtex>
\NeedsTeXFormat{LaTeX2e}
\ProvidesFile{bkm-vtex.def}%
  [2020-11-06 v1.29 bookmark driver for VTeX (HO)]%
%    \end{macrocode}
%
%    \begin{macrocode}
\ifvtexpdf
\else
  \PackageWarningNoLine{bookmark}{%
    The VTeX driver only supports PDF mode%
  }%
\fi
%    \end{macrocode}
%
%    \begin{macro}{\BKM@id}
%    \begin{macrocode}
\newcount\BKM@id
\BKM@id=\z@
%    \end{macrocode}
%    \end{macro}
%
%    \begin{macro}{\BKM@0}
%    \begin{macrocode}
\@namedef{BKM@0}{00}
%    \end{macrocode}
%    \end{macro}
%    \begin{macro}{\ifBKM@sw}
%    \begin{macrocode}
\newif\ifBKM@sw
%    \end{macrocode}
%    \end{macro}
%
%    \begin{macro}{\bookmark}
%    \begin{macrocode}
\newcommand*{\bookmark}[2][]{%
  \if@filesw
    \begingroup
      \def\bookmark@text{#2}%
      \BKM@setup{#1}%
      \edef\BKM@prev{\the\BKM@id}%
      \global\advance\BKM@id\@ne
      \BKM@swtrue
      \@whilesw\ifBKM@sw\fi{%
        \ifnum\ifBKM@startatroot\z@\else\BKM@prev\fi=\z@
          \BKM@startatrootfalse
          \def\BKM@parent{0}%
          \expandafter\xdef\csname BKM@\the\BKM@id\endcsname{%
            0{\BKM@level}%
          }%
          \BKM@swfalse
        \else
          \expandafter\expandafter\expandafter\BKM@getx
              \csname BKM@\BKM@prev\endcsname
          \ifnum\BKM@level>\BKM@x@level\relax
            \let\BKM@parent\BKM@prev
            \expandafter\xdef\csname BKM@\the\BKM@id\endcsname{%
              {\BKM@prev}{\BKM@level}%
            }%
            \BKM@swfalse
          \else
            \let\BKM@prev\BKM@x@parent
          \fi
        \fi
      }%
      \pdfstringdef\BKM@title{\bookmark@text}%
      \BKM@vtex@title
      \edef\BKM@FLAGS{\BKM@PrintStyle}%
      \let\BKM@action\@empty
      \ifx\BKM@gotor\@empty
        \ifx\BKM@dest\@empty
          \ifx\BKM@named\@empty
            \ifx\BKM@rawaction\@empty
              \ifx\BKM@uri\@empty
                \ifx\BKM@page\@empty
                  \PackageError{bookmark}{Missing action}\@ehc
                  \def\BKM@action{!1}%
                \else
                  \edef\BKM@action{!\BKM@page}%
                \fi
              \else
                \BKM@EscapeString\BKM@uri
                \edef\BKM@action{%
                  <u=%
                    /S/URI%
                    /URI(\BKM@uri)%
                  >%
                }%
              \fi
            \else
              \edef\BKM@action{<u=\BKM@rawaction>}%
            \fi
          \else
            \BKM@EscapeName\BKM@named
            \edef\BKM@action{%
              <u=%
                /S/Named%
                /N/\BKM@named
              >%
            }%
          \fi
        \else
          \BKM@EscapeString\BKM@dest
          \edef\BKM@action{\BKM@dest}%
        \fi
      \else
        \ifx\BKM@dest\@empty
          \ifx\BKM@page\@empty
            \def\BKM@page{1}%
          \fi
          \ifx\BKM@view\@empty
            \def\BKM@view{Fit}%
          \fi
          \edef\BKM@action{/D[\BKM@page/\BKM@view]}%
        \else
          \BKM@EscapeString\BKM@dest
          \edef\BKM@action{/D(\BKM@dest)}%
        \fi
        \BKM@EscapeString\BKM@gotor
        \edef\BKM@action{%
          <u=%
            /S/GoToR%
            /F(\BKM@gotor)%
            \BKM@action
          >>%
        }%
      \fi
      \ifx\BKM@color\@empty
        \let\BKM@RGBcolor\@empty
      \else
        \expandafter\BKM@toRGB\BKM@color\@nil
      \fi
      \special{%
        !outline \BKM@action;%
        p=\BKM@parent,%
        i=\number\BKM@id,%
        s=%
          \ifBKM@open
            \ifnum\BKM@level<\BKM@openlevel
              o%
            \else
              c%
            \fi
          \else
            c%
          \fi,%
        \ifx\BKM@RGBcolor\@empty
        \else
          c=\BKM@RGBcolor,%
        \fi
        \ifnum\BKM@FLAGS>\z@
          f=\BKM@FLAGS,%
        \fi
        t=\BKM@title
      }%
    \endgroup
  \fi
}
%    \end{macrocode}
%    \end{macro}
%    \begin{macro}{\BKM@getx}
%    \begin{macrocode}
\def\BKM@getx#1#2{%
  \def\BKM@x@parent{#1}%
  \def\BKM@x@level{#2}%
}
%    \end{macrocode}
%    \end{macro}
%    \begin{macro}{\BKM@toRGB}
%    \begin{macrocode}
\def\BKM@toRGB#1 #2 #3\@nil{%
  \let\BKM@RGBcolor\@empty
  \BKM@toRGBComponent{#1}%
  \BKM@toRGBComponent{#2}%
  \BKM@toRGBComponent{#3}%
}
%    \end{macrocode}
%    \end{macro}
%    \begin{macro}{\BKM@toRGBComponent}
%    \begin{macrocode}
\def\BKM@toRGBComponent#1{%
  \dimen@=#1pt\relax
  \ifdim\dimen@>\z@
    \ifdim\dimen@<\p@
      \dimen@=255\dimen@
      \advance\dimen@ by 32768sp\relax
      \divide\dimen@ by 65536\relax
      \dimen@ii=\dimen@
      \divide\dimen@ii by 16\relax
      \edef\BKM@RGBcolor{%
        \BKM@RGBcolor
        \BKM@toHexDigit\dimen@ii
      }%
      \dimen@ii=16\dimen@ii
      \advance\dimen@-\dimen@ii
      \edef\BKM@RGBcolor{%
        \BKM@RGBcolor
        \BKM@toHexDigit\dimen@
      }%
    \else
      \edef\BKM@RGBcolor{\BKM@RGBcolor FF}%
    \fi
  \else
    \edef\BKM@RGBcolor{\BKM@RGBcolor00}%
  \fi
}
%    \end{macrocode}
%    \end{macro}
%    \begin{macro}{\BKM@toHexDigit}
%    \begin{macrocode}
\def\BKM@toHexDigit#1{%
  \ifcase\expandafter\@firstofone\expandafter{\number#1} %
    0\or 1\or 2\or 3\or 4\or 5\or 6\or 7\or
    8\or 9\or A\or B\or C\or D\or E\or F%
  \fi
}
%    \end{macrocode}
%    \end{macro}
%    \begin{macrocode}
\begingroup
  \catcode`\|=0 %
  \catcode`\\=12 %
%    \end{macrocode}
%    \begin{macro}{\BKM@vtex@title}
%    \begin{macrocode}
  |gdef|BKM@vtex@title{%
    |@onelevel@sanitize|BKM@title
    |edef|BKM@title{|expandafter|BKM@vtex@leftparen|BKM@title\(|@nil}%
    |edef|BKM@title{|expandafter|BKM@vtex@rightparen|BKM@title\)|@nil}%
    |edef|BKM@title{|expandafter|BKM@vtex@zero|BKM@title\0|@nil}%
    |edef|BKM@title{|expandafter|BKM@vtex@one|BKM@title\1|@nil}%
    |edef|BKM@title{|expandafter|BKM@vtex@two|BKM@title\2|@nil}%
    |edef|BKM@title{|expandafter|BKM@vtex@three|BKM@title\3|@nil}%
  }%
%    \end{macrocode}
%    \end{macro}
%    \begin{macro}{\BKM@vtex@leftparen}
%    \begin{macrocode}
  |gdef|BKM@vtex@leftparen#1\(#2|@nil{%
    #1%
    |ifx||#2||%
    |else
      (%
      |ltx@ReturnAfterFi{%
        |BKM@vtex@leftparen#2|@nil
      }%
    |fi
  }%
%    \end{macrocode}
%    \end{macro}
%    \begin{macro}{\BKM@vtex@rightparen}
%    \begin{macrocode}
  |gdef|BKM@vtex@rightparen#1\)#2|@nil{%
    #1%
    |ifx||#2||%
    |else
      )%
      |ltx@ReturnAfterFi{%
        |BKM@vtex@rightparen#2|@nil
      }%
    |fi
  }%
%    \end{macrocode}
%    \end{macro}
%    \begin{macro}{\BKM@vtex@zero}
%    \begin{macrocode}
  |gdef|BKM@vtex@zero#1\0#2|@nil{%
    #1%
    |ifx||#2||%
    |else
      |noexpand|hv@pdf@char0%
      |ltx@ReturnAfterFi{%
        |BKM@vtex@zero#2|@nil
      }%
    |fi
  }%
%    \end{macrocode}
%    \end{macro}
%    \begin{macro}{\BKM@vtex@one}
%    \begin{macrocode}
  |gdef|BKM@vtex@one#1\1#2|@nil{%
    #1%
    |ifx||#2||%
    |else
      |noexpand|hv@pdf@char1%
      |ltx@ReturnAfterFi{%
        |BKM@vtex@one#2|@nil
      }%
    |fi
  }%
%    \end{macrocode}
%    \end{macro}
%    \begin{macro}{\BKM@vtex@two}
%    \begin{macrocode}
  |gdef|BKM@vtex@two#1\2#2|@nil{%
    #1%
    |ifx||#2||%
    |else
      |noexpand|hv@pdf@char2%
      |ltx@ReturnAfterFi{%
        |BKM@vtex@two#2|@nil
      }%
    |fi
  }%
%    \end{macrocode}
%    \end{macro}
%    \begin{macro}{\BKM@vtex@three}
%    \begin{macrocode}
  |gdef|BKM@vtex@three#1\3#2|@nil{%
    #1%
    |ifx||#2||%
    |else
      |noexpand|hv@pdf@char3%
      |ltx@ReturnAfterFi{%
        |BKM@vtex@three#2|@nil
      }%
    |fi
  }%
%    \end{macrocode}
%    \end{macro}
%    \begin{macrocode}
|endgroup
%    \end{macrocode}
%
%    \begin{macrocode}
%</vtex>
%    \end{macrocode}
%
% \subsection{\hologo{pdfTeX}\ 的驱动程序}
%
%    \begin{macrocode}
%<*pdftex>
\NeedsTeXFormat{LaTeX2e}
\ProvidesFile{bkm-pdftex.def}%
  [2020-11-06 v1.29 bookmark driver for pdfTeX (HO)]%
%    \end{macrocode}
%
%    \begin{macro}{\BKM@DO@entry}
%    \begin{macrocode}
\def\BKM@DO@entry#1#2{%
  \begingroup
    \kvsetkeys{BKM@DO}{#1}%
    \def\BKM@DO@title{#2}%
    \ifx\BKM@DO@srcfile\@empty
    \else
      \BKM@UnescapeHex\BKM@DO@srcfile
    \fi
    \BKM@UnescapeHex\BKM@DO@title
    \expandafter\expandafter\expandafter\BKM@getx
        \csname BKM@\BKM@DO@id\endcsname\@empty\@empty
    \let\BKM@attr\@empty
    \ifx\BKM@DO@flags\@empty
    \else
      \edef\BKM@attr{\BKM@attr/F \BKM@DO@flags}%
    \fi
    \ifx\BKM@DO@color\@empty
    \else
      \edef\BKM@attr{\BKM@attr/C[\BKM@DO@color]}%
    \fi
    \ifx\BKM@attr\@empty
    \else
      \edef\BKM@attr{attr{\BKM@attr}}%
    \fi
    \let\BKM@action\@empty
    \ifx\BKM@DO@gotor\@empty
      \ifx\BKM@DO@dest\@empty
        \ifx\BKM@DO@named\@empty
          \ifx\BKM@DO@rawaction\@empty
            \ifx\BKM@DO@uri\@empty
              \ifx\BKM@DO@page\@empty
                \PackageError{bookmark}{%
                  Missing action\BKM@SourceLocation
                }\@ehc
                \edef\BKM@action{goto page1{/Fit}}%
              \else
                \ifx\BKM@DO@view\@empty
                  \def\BKM@DO@view{Fit}%
                \fi
                \edef\BKM@action{goto page\BKM@DO@page{/\BKM@DO@view}}%
              \fi
            \else
              \BKM@UnescapeHex\BKM@DO@uri
              \BKM@EscapeString\BKM@DO@uri
              \edef\BKM@action{user{<</S/URI/URI(\BKM@DO@uri)>>}}%
            \fi
          \else
            \BKM@UnescapeHex\BKM@DO@rawaction
            \edef\BKM@action{%
              user{%
                <<%
                  \BKM@DO@rawaction
                >>%
              }%
            }%
          \fi
        \else
          \BKM@EscapeName\BKM@DO@named
          \edef\BKM@action{%
            user{<</S/Named/N/\BKM@DO@named>>}%
          }%
        \fi
      \else
        \BKM@UnescapeHex\BKM@DO@dest
        \BKM@DefGotoNameAction\BKM@action\BKM@DO@dest
      \fi
    \else
      \ifx\BKM@DO@dest\@empty
        \ifx\BKM@DO@page\@empty
          \def\BKM@DO@page{0}%
        \else
          \BKM@CalcExpr\BKM@DO@page\BKM@DO@page-1%
        \fi
        \ifx\BKM@DO@view\@empty
          \def\BKM@DO@view{Fit}%
        \fi
        \edef\BKM@action{/D[\BKM@DO@page/\BKM@DO@view]}%
      \else
        \BKM@UnescapeHex\BKM@DO@dest
        \BKM@EscapeString\BKM@DO@dest
        \edef\BKM@action{/D(\BKM@DO@dest)}%
      \fi
      \BKM@UnescapeHex\BKM@DO@gotor
      \BKM@EscapeString\BKM@DO@gotor
      \edef\BKM@action{%
        user{%
          <<%
            /S/GoToR%
            /F(\BKM@DO@gotor)%
            \BKM@action
          >>%
        }%
      }%
    \fi
    \pdfoutline\BKM@attr\BKM@action
                count\ifBKM@DO@open\else-\fi\BKM@x@childs
                {\BKM@DO@title}%
  \endgroup
}
%    \end{macrocode}
%    \end{macro}
%    \begin{macro}{\BKM@DefGotoNameAction}
%    \cs{BKM@DefGotoNameAction}\ 宏是一个用于 \xpackage{hypdestopt}\ 宏包的钩子(hook)。
%    \begin{macrocode}
\def\BKM@DefGotoNameAction#1#2{%
  \BKM@EscapeString\BKM@DO@dest
  \edef#1{goto name{#2}}%
}
%    \end{macrocode}
%    \end{macro}
%    \begin{macrocode}
%</pdftex>
%    \end{macrocode}
%
%    \begin{macrocode}
%<*pdftex|pdfmark>
%    \end{macrocode}
%    \begin{macro}{\BKM@SourceLocation}
%    \begin{macrocode}
\def\BKM@SourceLocation{%
  \ifx\BKM@DO@srcfile\@empty
    \ifx\BKM@DO@srcline\@empty
    \else
      .\MessageBreak
      Source: line \BKM@DO@srcline
    \fi
  \else
    \ifx\BKM@DO@srcline\@empty
      .\MessageBreak
      Source: file `\BKM@DO@srcfile'%
    \else
      .\MessageBreak
      Source: file `\BKM@DO@srcfile', line \BKM@DO@srcline
    \fi
  \fi
}
%    \end{macrocode}
%    \end{macro}
%    \begin{macrocode}
%</pdftex|pdfmark>
%    \end{macrocode}
%
% \subsection{具有 pdfmark 特色(specials)的驱动程序}
%
% \subsubsection{dvips 驱动程序}
%
%    \begin{macrocode}
%<*dvips>
\NeedsTeXFormat{LaTeX2e}
\ProvidesFile{bkm-dvips.def}%
  [2020-11-06 v1.29 bookmark driver for dvips (HO)]%
%    \end{macrocode}
%    \begin{macro}{\BKM@PSHeaderFile}
%    \begin{macrocode}
\def\BKM@PSHeaderFile#1{%
  \special{PSfile=#1}%
}
%    \end{macrocode}
%    \begin{macro}{\BKM@filename}
%    \begin{macrocode}
\def\BKM@filename{\jobname.out.ps}
%    \end{macrocode}
%    \end{macro}
%    \begin{macrocode}
\AddToHook{shipout/lastpage}{%
  \BKM@pdfmark@out
  \BKM@PSHeaderFile\BKM@filename
  }
%    \end{macrocode}
%    \end{macro}
%    \begin{macrocode}
%</dvips>
%    \end{macrocode}
%
% \subsubsection{公共部分(Common part)}
%
%    \begin{macrocode}
%<*pdfmark>
%    \end{macrocode}
%
%    \begin{macro}{\BKM@pdfmark@out}
%    不要在这里使用 \xpackage{rerunfilecheck}\ 宏包,因为在 \hologo{TeX}\ 运行期间不会
%    读取 \cs{BKM@filename}\ 文件。
%    \begin{macrocode}
\def\BKM@pdfmark@out{%
  \if@filesw
    \newwrite\BKM@file
    \immediate\openout\BKM@file=\BKM@filename\relax
    \BKM@write{\@percentchar!}%
    \BKM@write{/pdfmark where{pop}}%
    \BKM@write{%
      {%
        /globaldict where{pop globaldict}{userdict}ifelse%
        /pdfmark/cleartomark load put%
      }%
    }%
    \BKM@write{ifelse}%
  \else
    \let\BKM@write\@gobble
    \let\BKM@DO@entry\@gobbletwo
  \fi
}
%    \end{macrocode}
%    \end{macro}
%    \begin{macro}{\BKM@write}
%    \begin{macrocode}
\def\BKM@write#{%
  \immediate\write\BKM@file
}
%    \end{macrocode}
%    \end{macro}
%
%    \begin{macro}{\BKM@DO@entry}
%    Pdfmark 的规范(specification)说明 |/Color| 是颜色(color)的键名(key name),
%    但是 ghostscript 只将键(key)传递到 PDF 文件中,因此键名必须是 |/C|。
%    \begin{macrocode}
\def\BKM@DO@entry#1#2{%
  \begingroup
    \kvsetkeys{BKM@DO}{#1}%
    \ifx\BKM@DO@srcfile\@empty
    \else
      \BKM@UnescapeHex\BKM@DO@srcfile
    \fi
    \def\BKM@DO@title{#2}%
    \BKM@UnescapeHex\BKM@DO@title
    \expandafter\expandafter\expandafter\BKM@getx
        \csname BKM@\BKM@DO@id\endcsname\@empty\@empty
    \let\BKM@attr\@empty
    \ifx\BKM@DO@flags\@empty
    \else
      \edef\BKM@attr{\BKM@attr/F \BKM@DO@flags}%
    \fi
    \ifx\BKM@DO@color\@empty
    \else
      \edef\BKM@attr{\BKM@attr/C[\BKM@DO@color]}%
    \fi
    \let\BKM@action\@empty
    \ifx\BKM@DO@gotor\@empty
      \ifx\BKM@DO@dest\@empty
        \ifx\BKM@DO@named\@empty
          \ifx\BKM@DO@rawaction\@empty
            \ifx\BKM@DO@uri\@empty
              \ifx\BKM@DO@page\@empty
                \PackageError{bookmark}{%
                  Missing action\BKM@SourceLocation
                }\@ehc
                \edef\BKM@action{%
                  /Action/GoTo%
                  /Page 1%
                  /View[/Fit]%
                }%
              \else
                \ifx\BKM@DO@view\@empty
                  \def\BKM@DO@view{Fit}%
                \fi
                \edef\BKM@action{%
                  /Action/GoTo%
                  /Page \BKM@DO@page
                  /View[/\BKM@DO@view]%
                }%
              \fi
            \else
              \BKM@UnescapeHex\BKM@DO@uri
              \BKM@EscapeString\BKM@DO@uri
              \edef\BKM@action{%
                /Action<<%
                  /Subtype/URI%
                  /URI(\BKM@DO@uri)%
                >>%
              }%
            \fi
          \else
            \BKM@UnescapeHex\BKM@DO@rawaction
            \edef\BKM@action{%
              /Action<<%
                \BKM@DO@rawaction
              >>%
            }%
          \fi
        \else
          \BKM@EscapeName\BKM@DO@named
          \edef\BKM@action{%
            /Action<<%
              /Subtype/Named%
              /N/\BKM@DO@named
            >>%
          }%
        \fi
      \else
        \BKM@UnescapeHex\BKM@DO@dest
        \BKM@EscapeString\BKM@DO@dest
        \edef\BKM@action{%
          /Action/GoTo%
          /Dest(\BKM@DO@dest)cvn%
        }%
      \fi
    \else
      \ifx\BKM@DO@dest\@empty
        \ifx\BKM@DO@page\@empty
          \def\BKM@DO@page{1}%
        \fi
        \ifx\BKM@DO@view\@empty
          \def\BKM@DO@view{Fit}%
        \fi
        \edef\BKM@action{%
          /Page \BKM@DO@page
          /View[/\BKM@DO@view]%
        }%
      \else
        \BKM@UnescapeHex\BKM@DO@dest
        \BKM@EscapeString\BKM@DO@dest
        \edef\BKM@action{%
          /Dest(\BKM@DO@dest)cvn%
        }%
      \fi
      \BKM@UnescapeHex\BKM@DO@gotor
      \BKM@EscapeString\BKM@DO@gotor
      \edef\BKM@action{%
        /Action/GoToR%
        /File(\BKM@DO@gotor)%
        \BKM@action
      }%
    \fi
    \BKM@write{[}%
    \BKM@write{/Title(\BKM@DO@title)}%
    \ifnum\BKM@x@childs>\z@
      \BKM@write{/Count \ifBKM@DO@open\else-\fi\BKM@x@childs}%
    \fi
    \ifx\BKM@attr\@empty
    \else
      \BKM@write{\BKM@attr}%
    \fi
    \BKM@write{\BKM@action}%
    \BKM@write{/OUT pdfmark}%
  \endgroup
}
%    \end{macrocode}
%    \end{macro}
%    \begin{macrocode}
%</pdfmark>
%    \end{macrocode}
%
% \subsection{\xoption{pdftex}\ 和 \xoption{pdfmark}\ 的公共部分}
%
%    \begin{macrocode}
%<*pdftex|pdfmark>
%    \end{macrocode}
%
% \subsubsection{写入辅助文件(auxiliary file)}
%
%    \begin{macrocode}
\AddToHook{begindocument}{%
 \immediate\write\@mainaux{\string\providecommand\string\BKM@entry[2]{}}}
%    \end{macrocode}
%
%    \begin{macro}{\BKM@id}
%    \begin{macrocode}
\newcount\BKM@id
\BKM@id=\z@
%    \end{macrocode}
%    \end{macro}
%
%    \begin{macro}{\BKM@0}
%    \begin{macrocode}
\@namedef{BKM@0}{000}
%    \end{macrocode}
%    \end{macro}
%    \begin{macro}{\ifBKM@sw}
%    \begin{macrocode}
\newif\ifBKM@sw
%    \end{macrocode}
%    \end{macro}
%
%    \begin{macro}{\bookmark}
%    \begin{macrocode}
\newcommand*{\bookmark}[2][]{%
  \if@filesw
    \begingroup
      \BKM@InitSourceLocation
      \def\bookmark@text{#2}%
      \BKM@setup{#1}%
      \ifx\BKM@srcfile\@empty
      \else
        \BKM@EscapeHex\BKM@srcfile
      \fi
      \edef\BKM@prev{\the\BKM@id}%
      \global\advance\BKM@id\@ne
      \BKM@swtrue
      \@whilesw\ifBKM@sw\fi{%
        \ifnum\ifBKM@startatroot\z@\else\BKM@prev\fi=\z@
          \BKM@startatrootfalse
          \expandafter\xdef\csname BKM@\the\BKM@id\endcsname{%
            0{\BKM@level}0%
          }%
          \BKM@swfalse
        \else
          \expandafter\expandafter\expandafter\BKM@getx
              \csname BKM@\BKM@prev\endcsname
          \ifnum\BKM@level>\BKM@x@level\relax
            \expandafter\xdef\csname BKM@\the\BKM@id\endcsname{%
              {\BKM@prev}{\BKM@level}0%
            }%
            \ifnum\BKM@prev>\z@
              \BKM@CalcExpr\BKM@CalcResult\BKM@x@childs+1%
              \expandafter\xdef\csname BKM@\BKM@prev\endcsname{%
                {\BKM@x@parent}{\BKM@x@level}{\BKM@CalcResult}%
              }%
            \fi
            \BKM@swfalse
          \else
            \let\BKM@prev\BKM@x@parent
          \fi
        \fi
      }%
      \pdfstringdef\BKM@title{\bookmark@text}%
      \edef\BKM@FLAGS{\BKM@PrintStyle}%
      \csname BKM@HypDestOptHook\endcsname
      \BKM@EscapeHex\BKM@dest
      \BKM@EscapeHex\BKM@uri
      \BKM@EscapeHex\BKM@gotor
      \BKM@EscapeHex\BKM@rawaction
      \BKM@EscapeHex\BKM@title
      \immediate\write\@mainaux{%
        \string\BKM@entry{%
          id=\number\BKM@id
          \ifBKM@open
            \ifnum\BKM@level<\BKM@openlevel
              ,open%
            \fi
          \fi
          \BKM@auxentry{dest}%
          \BKM@auxentry{named}%
          \BKM@auxentry{uri}%
          \BKM@auxentry{gotor}%
          \BKM@auxentry{page}%
          \BKM@auxentry{view}%
          \BKM@auxentry{rawaction}%
          \BKM@auxentry{color}%
          \ifnum\BKM@FLAGS>\z@
            ,flags=\BKM@FLAGS
          \fi
          \BKM@auxentry{srcline}%
          \BKM@auxentry{srcfile}%
        }{\BKM@title}%
      }%
    \endgroup
  \fi
}
%    \end{macrocode}
%    \end{macro}
%    \begin{macro}{\BKM@getx}
%    \begin{macrocode}
\def\BKM@getx#1#2#3{%
  \def\BKM@x@parent{#1}%
  \def\BKM@x@level{#2}%
  \def\BKM@x@childs{#3}%
}
%    \end{macrocode}
%    \end{macro}
%    \begin{macro}{\BKM@auxentry}
%    \begin{macrocode}
\def\BKM@auxentry#1{%
  \expandafter\ifx\csname BKM@#1\endcsname\@empty
  \else
    ,#1={\csname BKM@#1\endcsname}%
  \fi
}
%    \end{macrocode}
%    \end{macro}
%
%    \begin{macro}{\BKM@InitSourceLocation}
%    \begin{macrocode}
\def\BKM@InitSourceLocation{%
  \edef\BKM@srcline{\the\inputlineno}%
  \BKM@LuaTeX@InitFile
  \ifx\BKM@srcfile\@empty
    \ltx@IfUndefined{currfilepath}{}{%
      \edef\BKM@srcfile{\currfilepath}%
    }%
  \fi
}
%    \end{macrocode}
%    \end{macro}
%    \begin{macro}{\BKM@LuaTeX@InitFile}
%    \begin{macrocode}
\ifluatex
  \ifnum\luatexversion>36 %
    \def\BKM@LuaTeX@InitFile{%
      \begingroup
        \ltx@LocToksA={}%
      \edef\x{\endgroup
        \def\noexpand\BKM@srcfile{%
          \the\expandafter\ltx@LocToksA
          \directlua{%
             if status and status.filename then %
               tex.settoks('ltx@LocToksA', status.filename)%
             end%
          }%
        }%
      }\x
    }%
  \else
    \let\BKM@LuaTeX@InitFile\relax
  \fi
\else
  \let\BKM@LuaTeX@InitFile\relax
\fi
%    \end{macrocode}
%    \end{macro}
%
% \subsubsection{读取辅助数据(auxiliary data)}
%
%    \begin{macrocode}
\SetupKeyvalOptions{family=BKM@DO,prefix=BKM@DO@}
\DeclareStringOption[0]{id}
\DeclareBoolOption{open}
\DeclareStringOption{flags}
\DeclareStringOption{color}
\DeclareStringOption{dest}
\DeclareStringOption{named}
\DeclareStringOption{uri}
\DeclareStringOption{gotor}
\DeclareStringOption{page}
\DeclareStringOption{view}
\DeclareStringOption{rawaction}
\DeclareStringOption{srcline}
\DeclareStringOption{srcfile}
%    \end{macrocode}
%
%    \begin{macrocode}
\AtBeginDocument{%
  \let\BKM@entry\BKM@DO@entry
}
%    \end{macrocode}
%
%    \begin{macrocode}
%</pdftex|pdfmark>
%    \end{macrocode}
%
% \subsection{\xoption{atend}\ 选项}
%
% \subsubsection{钩子(Hook)}
%
%    \begin{macrocode}
%<*package>
%    \end{macrocode}
%    \begin{macrocode}
\ifBKM@atend
\else
%    \end{macrocode}
%    \begin{macro}{\BookmarkAtEnd}
%    这是一个虚拟定义(dummy definition),如果没有给出 \xoption{atend}\ 选项,它将生成一个警告。
%    \begin{macrocode}
  \newcommand{\BookmarkAtEnd}[1]{%
    \PackageWarning{bookmark}{%
      Ignored, because option `atend' is missing%
    }%
  }%
%    \end{macrocode}
%    \end{macro}
%    \begin{macrocode}
  \expandafter\endinput
\fi
%    \end{macrocode}
%    \begin{macro}{\BookmarkAtEnd}
%    \begin{macrocode}
\newcommand*{\BookmarkAtEnd}{%
  \g@addto@macro\BKM@EndHook
}
%    \end{macrocode}
%    \end{macro}
%    \begin{macrocode}
\let\BKM@EndHook\@empty
%    \end{macrocode}
%    \begin{macrocode}
%</package>
%    \end{macrocode}
%
% \subsubsection{在文档末尾使用钩子的驱动程序}
%
%    驱动程序 \xoption{pdftex}\ 使用 LaTeX 钩子 \xoption{enddocument/afterlastpage}
%    (相当于以前使用的 \xpackage{atveryend}\ 的 \cs{AfterLastShipout}),因为它仍然需要 \xext{aux}\ 文件。
%    它使用 \cs{pdfoutline}\ 作为最后一页之后可以使用的书签(bookmakrs)。
%    \begin{itemize}
%    \item
%      驱动程序 \xoption{pdftex}\ 使用 \cs{pdfoutline}, \cs{pdfoutline}\ 可以在最后一页之后使用。
%    \end{itemize}
%    \begin{macrocode}
%<*pdftex>
\ifBKM@atend
  \AddToHook{enddocument/afterlastpage}{%
    \BKM@EndHook
  }%
\fi
%</pdftex>
%    \end{macrocode}
%
% \subsubsection{使用 \xoption{shipout/lastpage}\ 的驱动程序}
%
%    其他驱动程序使用 \cs{special}\ 命令实现 \cs{bookmark}。因此,最后的书签(last bookmarks)
%    必须放在最后一页(last page),而不是之后。不能使用 \cs{AtEndDocument},因为为时已晚,
%    最后一页已经输出了。因此,我们使用 LaTeX 钩子 \xoption{shipout/lastpage}。至少需要运行
%    两次 \hologo{LaTeX}。PostScript 驱动程序 \xoption{dvips}\ 使用外部 PostScript 文件作为书签。
%    为了避免与 pgf 发生冲突,文件写入(file writing)也被移到了最后一个输出页面(shipout page)。
%    \begin{macrocode}
%<*dvipdfm|vtex|pdfmark>
\ifBKM@atend
  \AddToHook{shipout/lastpage}{\BKM@EndHook}%
\fi
%</dvipdfm|vtex|pdfmark>
%    \end{macrocode}
%
% \section{安装(Installation)}
%
% \subsection{下载(Download)}
%
% \paragraph{宏包(Package)。} 在 CTAN\footnote{\CTANpkg{bookmark}}上提供此宏包:
% \begin{description}
% \item[\CTAN{macros/latex/contrib/bookmark/bookmark.dtx}] 源文件(source file)。
% \item[\CTAN{macros/latex/contrib/bookmark/bookmark.pdf}] 文档(documentation)。
% \end{description}
%
%
% \paragraph{捆绑包(Bundle)。} “bookmark”捆绑包(bundle)的所有宏包(packages)都可以在兼
% 容 TDS 的 ZIP 归档文件中找到。在那里,宏包已经被解包,文档文件(documentation files)已经生成。
% 文件(files)和目录(directories)遵循 TDS 标准。
% \begin{description}
% \item[\CTANinstall{install/macros/latex/contrib/bookmark.tds.zip}]
% \end{description}
% \emph{TDS}\ 是指标准的“用于 \TeX\ 文件的目录结构(Directory Structure)”(\CTANpkg{tds})。
% 名称中带有 \xfile{texmf}\ 的目录(directories)通常以这种方式组织。
%
% \subsection{捆绑包(Bundle)的安装}
%
% \paragraph{解压(Unpacking)。} 在您选择的 TDS 树(也称为 \xfile{texmf}\ 树)中解
% 压 \xfile{bookmark.tds.zip},例如(在 linux 中):
% \begin{quote}
%   |unzip bookmark.tds.zip -d ~/texmf|
% \end{quote}
%
% \subsection{宏包(Package)的安装}
%
% \paragraph{解压(Unpacking)。} \xfile{.dtx}\ 文件是一个自解压 \docstrip\ 归档文件(archive)。
% 这些文件是通过 \plainTeX\ 运行 \xfile{.dtx}\ 来提取的:
% \begin{quote}
%   \verb|tex bookmark.dtx|
% \end{quote}
%
% \paragraph{TDS.} 现在,不同的文件必须移动到安装 TDS 树(installation TDS tree)
% (也称为 \xfile{texmf}\ 树)中的不同目录中:
% \begin{quote}
% \def\t{^^A
% \begin{tabular}{@{}>{\ttfamily}l@{ $\rightarrow$ }>{\ttfamily}l@{}}
%   bookmark.sty & tex/latex/bookmark/bookmark.sty\\
%   bkm-dvipdfm.def & tex/latex/bookmark/bkm-dvipdfm.def\\
%   bkm-dvips.def & tex/latex/bookmark/bkm-dvips.def\\
%   bkm-pdftex.def & tex/latex/bookmark/bkm-pdftex.def\\
%   bkm-vtex.def & tex/latex/bookmark/bkm-vtex.def\\
%   bookmark.pdf & doc/latex/bookmark/bookmark.pdf\\
%   bookmark-example.tex & doc/latex/bookmark/bookmark-example.tex\\
%   bookmark.dtx & source/latex/bookmark/bookmark.dtx\\
% \end{tabular}^^A
% }^^A
% \sbox0{\t}^^A
% \ifdim\wd0>\linewidth
%   \begingroup
%     \advance\linewidth by\leftmargin
%     \advance\linewidth by\rightmargin
%   \edef\x{\endgroup
%     \def\noexpand\lw{\the\linewidth}^^A
%   }\x
%   \def\lwbox{^^A
%     \leavevmode
%     \hbox to \linewidth{^^A
%       \kern-\leftmargin\relax
%       \hss
%       \usebox0
%       \hss
%       \kern-\rightmargin\relax
%     }^^A
%   }^^A
%   \ifdim\wd0>\lw
%     \sbox0{\small\t}^^A
%     \ifdim\wd0>\linewidth
%       \ifdim\wd0>\lw
%         \sbox0{\footnotesize\t}^^A
%         \ifdim\wd0>\linewidth
%           \ifdim\wd0>\lw
%             \sbox0{\scriptsize\t}^^A
%             \ifdim\wd0>\linewidth
%               \ifdim\wd0>\lw
%                 \sbox0{\tiny\t}^^A
%                 \ifdim\wd0>\linewidth
%                   \lwbox
%                 \else
%                   \usebox0
%                 \fi
%               \else
%                 \lwbox
%               \fi
%             \else
%               \usebox0
%             \fi
%           \else
%             \lwbox
%           \fi
%         \else
%           \usebox0
%         \fi
%       \else
%         \lwbox
%       \fi
%     \else
%       \usebox0
%     \fi
%   \else
%     \lwbox
%   \fi
% \else
%   \usebox0
% \fi
% \end{quote}
% 如果你有一个 \xfile{docstrip.cfg}\ 文件,该文件能配置并启用 \docstrip\ 的 TDS 安装功能,
% 则一些文件可能已经在正确的位置了,请参阅 \docstrip\ 的文档(documentation)。
%
% \subsection{刷新文件名数据库}
%
% 如果您的 \TeX~发行版(\TeX\,Live、\mikTeX、\dots)依赖于文件名数据库(file name databases),
% 则必须刷新这些文件名数据库。例如,\TeX\,Live\ 用户运行 \verb|texhash| 或 \verb|mktexlsr|。
%
% \subsection{一些感兴趣的细节}
%
% \paragraph{用 \LaTeX\ 解压。}
% \xfile{.dtx}\ 根据格式(format)选择其操作(action):
% \begin{description}
% \item[\plainTeX:] 运行 \docstrip\ 并解压文件。
% \item[\LaTeX:] 生成文档。
% \end{description}
% 如果您坚持通过 \LaTeX\ 使用\docstrip (实际上 \docstrip\ 并不需要 \LaTeX),那么请您的意图告知自动检测程序:
% \begin{quote}
%   \verb|latex \let\install=y\input{bookmark.dtx}|
% \end{quote}
% 不要忘记根据 shell 的要求引用这个参数(argument)。
%
% \paragraph{知生成文档。}
% 您可以同时使用 \xfile{.dtx}\ 或 \xfile{.drv}\ 来生成文档。可以通过配置文件 \xfile{ltxdoc.cfg}\ 配置该进程。
% 例如,如果您希望 A4 作为纸张格式,请将下面这行写入此文件中:
% \begin{quote}
%   \verb|\PassOptionsToClass{a4paper}{article}|
% \end{quote}
% 下面是一个如何使用 pdf\LaTeX\ 生成文档的示例:
% \begin{quote}
%\begin{verbatim}
%pdflatex bookmark.dtx
%makeindex -s gind.ist bookmark.idx
%pdflatex bookmark.dtx
%makeindex -s gind.ist bookmark.idx
%pdflatex bookmark.dtx
%\end{verbatim}
% \end{quote}
%
% \begin{thebibliography}{9}
%
% \bibitem{hyperref}
%   Sebastian Rahtz, Heiko Oberdiek:
%   \textit{The \xpackage{hyperref} package};
%   2011/04/17 v6.82g;
%   \CTANpkg{hyperref}
%
% \bibitem{currfile}
%   Martin Scharrer:
%   \textit{The \xpackage{currfile} package};
%   2011/01/09 v0.4.
%   \CTANpkg{currfile}
%
% \end{thebibliography}
%
% \begin{History}
%   \begin{Version}{2007/02/19 v0.1}
%   \item
%     First experimental version.
%   \end{Version}
%   \begin{Version}{2007/02/20 v0.2}
%   \item
%     Option \xoption{startatroot} added.
%   \item
%     Dummies for \cs{pdf(un)escape...} commands added to get
%     the package basically work for non-\hologo{pdfTeX} users.
%   \end{Version}
%   \begin{Version}{2007/02/21 v0.3}
%   \item
%     Dependency from \hologo{pdfTeX} 1.30 removed by using package
%     \xpackage{pdfescape}.
%   \end{Version}
%   \begin{Version}{2007/02/22 v0.4}
%   \item
%     \xpackage{hyperref}'s \xoption{bookmarkstype} respected.
%   \end{Version}
%   \begin{Version}{2007/03/02 v0.5}
%   \item
%     Driver options \xoption{vtex} (PDF mode), \xoption{dvipsone},
%     and \xoption{textures} added.
%   \item
%     Implementation of option \xoption{depth} completed. Division names
%     are supported, see \xpackage{hyperref}'s
%     option \xoption{bookmarksdepth}.
%   \item
%     \xpackage{hyperref}'s options \xoption{bookmarksopen},
%     \xoption{bookmarksopenlevel}, and \xoption{bookmarksdepth} respected.
%   \end{Version}
%   \begin{Version}{2007/03/03 v0.6}
%   \item
%     Option \xoption{numbered} as alias for \xpackage{hyperref}'s
%     \xoption{bookmarksnumbered}.
%   \end{Version}
%   \begin{Version}{2007/03/07 v0.7}
%   \item
%     Dependency from \hologo{eTeX} removed.
%   \end{Version}
%   \begin{Version}{2007/04/09 v0.8}
%   \item
%     Option \xoption{atend} added.
%   \item
%     Option \xoption{rgbcolor} removed.
%     \verb|rgbcolor=<r> <g> <b>| can be replaced by
%     \verb|color=[rgb]{<r>,<g>,<b>}|.
%   \item
%     Support of recent cvs version (2007-03-29) of dvipdfmx
%     that extends the \cs{special} for bookmarks to specify
%     open outline entries. Option \xoption{dvipdfmx-outline-open}
%     or \cs{SpecialDvipdfmxOutlineOpen} notify the package.
%   \end{Version}
%   \begin{Version}{2007/04/25 v0.9}
%   \item
%     The syntax of \cs{special} of dvipdfmx, if feature
%     \xoption{dvipdfmx-outline-open} is enabled, has changed.
%     Now cvs version 2007-04-25 is needed.
%   \end{Version}
%   \begin{Version}{2007/05/29 v1.0}
%   \item
%     Bug fix in code for second parameter of XYZ.
%   \end{Version}
%   \begin{Version}{2007/07/13 v1.1}
%   \item
%     Fix for pdfmark with GoToR action.
%   \end{Version}
%   \begin{Version}{2007/09/25 v1.2}
%   \item
%     pdfmark driver respects \cs{nofiles}.
%   \end{Version}
%   \begin{Version}{2008/08/08 v1.3}
%   \item
%     Package \xpackage{flags} replaced by package \xpackage{bitset}.
%     Now flags are also supported without \hologo{eTeX}.
%   \item
%     Hook for package \xpackage{hypdestopt} added.
%   \end{Version}
%   \begin{Version}{2008/09/13 v1.4}
%   \item
%     Fix for bug introduced in v1.3, package \xpackage{flags} is one-based,
%     but package \xpackage{bitset} is zero-based. Thus options \xoption{bold}
%     and \xoption{italic} are wrong in v1.3. (Daniel M\"ullner)
%   \end{Version}
%   \begin{Version}{2009/08/13 v1.5}
%   \item
%     Except for driver options the other options are now local options.
%     This resolves a problem with KOMA-Script v3.00 and its option \xoption{open}.
%   \end{Version}
%   \begin{Version}{2009/12/06 v1.6}
%   \item
%     Use of package \xpackage{atveryend} for drivers \xoption{pdftex}
%     and \xoption{pdfmark}.
%   \end{Version}
%   \begin{Version}{2009/12/07 v1.7}
%   \item
%     Use of package \xpackage{atveryend} fixed.
%   \end{Version}
%   \begin{Version}{2009/12/17 v1.8}
%   \item
%     Support of \xpackage{hyperref} 2009/12/17 v6.79v for \hologo{XeTeX}.
%   \end{Version}
%   \begin{Version}{2010/03/30 v1.9}
%   \item
%     Package name in an error message fixed.
%   \end{Version}
%   \begin{Version}{2010/04/03 v1.10}
%   \item
%     Option \xoption{style} and macro \cs{bookmarkdefinestyle} added.
%   \item
%     Hook support with option \xoption{addtohook} added.
%   \item
%     \cs{bookmarkget} added.
%   \end{Version}
%   \begin{Version}{2010/04/04 v1.11}
%   \item
%     Bug fix (introduced in v1.10).
%   \end{Version}
%   \begin{Version}{2010/04/08 v1.12}
%   \item
%     Requires \xpackage{ltxcmds} 2010/04/08.
%   \end{Version}
%   \begin{Version}{2010/07/23 v1.13}
%   \item
%     Support for \xclass{memoir}'s \cs{booknumberline} added.
%   \end{Version}
%   \begin{Version}{2010/09/02 v1.14}
%   \item
%     (Local) options \xoption{draft} and \xoption{final} added.
%   \end{Version}
%   \begin{Version}{2010/09/25 v1.15}
%   \item
%     Fix for option \xoption{dvipdfmx-outline-open}.
%   \item
%     Option \xoption{dvipdfmx-outline-open} is set automatically,
%     if XeTeX $\geq$ 0.9995 is detected.
%   \end{Version}
%   \begin{Version}{2010/10/19 v1.16}
%   \item
%     Option `startatroot' now acts globally.
%   \item
%     Option `level' also accepts names the same way as option `depth'.
%   \end{Version}
%   \begin{Version}{2010/10/25 v1.17}
%   \item
%     \cs{bookmarksetupnext} added.
%   \item
%     Using \cs{kvsetkeys} of package \xpackage{kvsetkeys}, because
%     \cs{setkeys} of package \xpackage{keyval} is not reentrant.
%     This can cause problems (unknown keys) with older versions of
%     hyperref that also uses \cs{setkeys} (found by GL).
%   \end{Version}
%   \begin{Version}{2010/11/05 v1.18}
%   \item
%     Use of \cs{pdf@ifdraftmode} of package \xpackage{pdftexcmds} for
%     the default of option \xoption{draft}.
%   \end{Version}
%   \begin{Version}{2011/03/20 v1.19}
%   \item
%     Use of \cs{dimexpr} fixed, if \hologo{eTeX} is not used.
%     (Bug found by Martin M\"unch.)
%   \item
%     Fix in documentation. Also layout options work without \hologo{eTeX}.
%   \end{Version}
%   \begin{Version}{2011/04/13 v1.20}
%   \item
%     Bug fix: \cs{BKM@SetDepth} renamed to \cs{BKM@SetDepthOrLevel}.
%   \end{Version}
%   \begin{Version}{2011/04/21 v1.21}
%   \item
%     Some support for file name and line number in error messages
%     at end of document (pdfTeX and pdfmark based drivers).
%   \end{Version}
%   \begin{Version}{2011/05/13 v1.22}
%   \item
%     Change of version 2010/11/05 v1.18 reverted, because otherwise
%     draftmode disables some \xext{aux} file entries.
%   \end{Version}
%   \begin{Version}{2011/09/19 v1.23}
%   \item
%     Some \cs{renewcommand}s changed to \cs{def} to avoid trouble
%     if the commands are not defined, because hyperref stopped early.
%   \end{Version}
%   \begin{Version}{2011/12/02 v1.24}
%   \item
%     Small optimization in \cs{BKM@toHexDigit}.
%   \end{Version}
%   \begin{Version}{2016/05/16 v1.25}
%   \item
%     Documentation updates.
%   \end{Version}
%   \begin{Version}{2016/05/17 v1.26}
%   \item
%     define \cs{pdfoutline} to allow pdftex driver to be used with Lua\TeX.
%   \end{Version}
%   \begin{Version}{2019/06/04 v1.27}
%   \item
%     unknown style options are ignored (issue 67)
%   \end{Version}

%   \begin{Version}{2019/12/03 v1.28}
%   \item
%     Documentation updates.
%   \item adjust package loading (all required packages already loaded
%     by \xpackage{hyperref}).
%   \end{Version}
%   \begin{Version}{2020-11-06 v1.29}
%   \item Adapted the dvips to avoid a clash with pgf.
%         https://github.com/pgf-tikz/pgf/issues/944
%   \item All drivers now use the new LaTeX hooks
%         and so require a format 2020-10-01 or newer. The older
%         drivers are provided as frozen versions and are used if an older
%         format is detected.
%   \item Added support for destlabel option of hyperref, https://github.com/ho-tex/bookmark/issues/1
%   \item Removed the \xoption{dvipsone} and \xoption{textures} driver.
%   \item Removed the code for option \xoption{dvipdfmx-outline-open}
%     and \cs{SpecialDvipdfmxOutlineOpen}. All dvipdfmx version should now support
%     this out-of-the-box.
%   \end{Version}
% \end{History}
%
% \PrintIndex
%
% \Finale
\endinput
|
% \end{quote}
% 不要忘记根据 shell 的要求引用这个参数(argument)。
%
% \paragraph{知生成文档。}
% 您可以同时使用 \xfile{.dtx}\ 或 \xfile{.drv}\ 来生成文档。可以通过配置文件 \xfile{ltxdoc.cfg}\ 配置该进程。
% 例如,如果您希望 A4 作为纸张格式,请将下面这行写入此文件中:
% \begin{quote}
%   \verb|\PassOptionsToClass{a4paper}{article}|
% \end{quote}
% 下面是一个如何使用 pdf\LaTeX\ 生成文档的示例:
% \begin{quote}
%\begin{verbatim}
%pdflatex bookmark.dtx
%makeindex -s gind.ist bookmark.idx
%pdflatex bookmark.dtx
%makeindex -s gind.ist bookmark.idx
%pdflatex bookmark.dtx
%\end{verbatim}
% \end{quote}
%
% \begin{thebibliography}{9}
%
% \bibitem{hyperref}
%   Sebastian Rahtz, Heiko Oberdiek:
%   \textit{The \xpackage{hyperref} package};
%   2011/04/17 v6.82g;
%   \CTANpkg{hyperref}
%
% \bibitem{currfile}
%   Martin Scharrer:
%   \textit{The \xpackage{currfile} package};
%   2011/01/09 v0.4.
%   \CTANpkg{currfile}
%
% \end{thebibliography}
%
% \begin{History}
%   \begin{Version}{2007/02/19 v0.1}
%   \item
%     First experimental version.
%   \end{Version}
%   \begin{Version}{2007/02/20 v0.2}
%   \item
%     Option \xoption{startatroot} added.
%   \item
%     Dummies for \cs{pdf(un)escape...} commands added to get
%     the package basically work for non-\hologo{pdfTeX} users.
%   \end{Version}
%   \begin{Version}{2007/02/21 v0.3}
%   \item
%     Dependency from \hologo{pdfTeX} 1.30 removed by using package
%     \xpackage{pdfescape}.
%   \end{Version}
%   \begin{Version}{2007/02/22 v0.4}
%   \item
%     \xpackage{hyperref}'s \xoption{bookmarkstype} respected.
%   \end{Version}
%   \begin{Version}{2007/03/02 v0.5}
%   \item
%     Driver options \xoption{vtex} (PDF mode), \xoption{dvipsone},
%     and \xoption{textures} added.
%   \item
%     Implementation of option \xoption{depth} completed. Division names
%     are supported, see \xpackage{hyperref}'s
%     option \xoption{bookmarksdepth}.
%   \item
%     \xpackage{hyperref}'s options \xoption{bookmarksopen},
%     \xoption{bookmarksopenlevel}, and \xoption{bookmarksdepth} respected.
%   \end{Version}
%   \begin{Version}{2007/03/03 v0.6}
%   \item
%     Option \xoption{numbered} as alias for \xpackage{hyperref}'s
%     \xoption{bookmarksnumbered}.
%   \end{Version}
%   \begin{Version}{2007/03/07 v0.7}
%   \item
%     Dependency from \hologo{eTeX} removed.
%   \end{Version}
%   \begin{Version}{2007/04/09 v0.8}
%   \item
%     Option \xoption{atend} added.
%   \item
%     Option \xoption{rgbcolor} removed.
%     \verb|rgbcolor=<r> <g> <b>| can be replaced by
%     \verb|color=[rgb]{<r>,<g>,<b>}|.
%   \item
%     Support of recent cvs version (2007-03-29) of dvipdfmx
%     that extends the \cs{special} for bookmarks to specify
%     open outline entries. Option \xoption{dvipdfmx-outline-open}
%     or \cs{SpecialDvipdfmxOutlineOpen} notify the package.
%   \end{Version}
%   \begin{Version}{2007/04/25 v0.9}
%   \item
%     The syntax of \cs{special} of dvipdfmx, if feature
%     \xoption{dvipdfmx-outline-open} is enabled, has changed.
%     Now cvs version 2007-04-25 is needed.
%   \end{Version}
%   \begin{Version}{2007/05/29 v1.0}
%   \item
%     Bug fix in code for second parameter of XYZ.
%   \end{Version}
%   \begin{Version}{2007/07/13 v1.1}
%   \item
%     Fix for pdfmark with GoToR action.
%   \end{Version}
%   \begin{Version}{2007/09/25 v1.2}
%   \item
%     pdfmark driver respects \cs{nofiles}.
%   \end{Version}
%   \begin{Version}{2008/08/08 v1.3}
%   \item
%     Package \xpackage{flags} replaced by package \xpackage{bitset}.
%     Now flags are also supported without \hologo{eTeX}.
%   \item
%     Hook for package \xpackage{hypdestopt} added.
%   \end{Version}
%   \begin{Version}{2008/09/13 v1.4}
%   \item
%     Fix for bug introduced in v1.3, package \xpackage{flags} is one-based,
%     but package \xpackage{bitset} is zero-based. Thus options \xoption{bold}
%     and \xoption{italic} are wrong in v1.3. (Daniel M\"ullner)
%   \end{Version}
%   \begin{Version}{2009/08/13 v1.5}
%   \item
%     Except for driver options the other options are now local options.
%     This resolves a problem with KOMA-Script v3.00 and its option \xoption{open}.
%   \end{Version}
%   \begin{Version}{2009/12/06 v1.6}
%   \item
%     Use of package \xpackage{atveryend} for drivers \xoption{pdftex}
%     and \xoption{pdfmark}.
%   \end{Version}
%   \begin{Version}{2009/12/07 v1.7}
%   \item
%     Use of package \xpackage{atveryend} fixed.
%   \end{Version}
%   \begin{Version}{2009/12/17 v1.8}
%   \item
%     Support of \xpackage{hyperref} 2009/12/17 v6.79v for \hologo{XeTeX}.
%   \end{Version}
%   \begin{Version}{2010/03/30 v1.9}
%   \item
%     Package name in an error message fixed.
%   \end{Version}
%   \begin{Version}{2010/04/03 v1.10}
%   \item
%     Option \xoption{style} and macro \cs{bookmarkdefinestyle} added.
%   \item
%     Hook support with option \xoption{addtohook} added.
%   \item
%     \cs{bookmarkget} added.
%   \end{Version}
%   \begin{Version}{2010/04/04 v1.11}
%   \item
%     Bug fix (introduced in v1.10).
%   \end{Version}
%   \begin{Version}{2010/04/08 v1.12}
%   \item
%     Requires \xpackage{ltxcmds} 2010/04/08.
%   \end{Version}
%   \begin{Version}{2010/07/23 v1.13}
%   \item
%     Support for \xclass{memoir}'s \cs{booknumberline} added.
%   \end{Version}
%   \begin{Version}{2010/09/02 v1.14}
%   \item
%     (Local) options \xoption{draft} and \xoption{final} added.
%   \end{Version}
%   \begin{Version}{2010/09/25 v1.15}
%   \item
%     Fix for option \xoption{dvipdfmx-outline-open}.
%   \item
%     Option \xoption{dvipdfmx-outline-open} is set automatically,
%     if XeTeX $\geq$ 0.9995 is detected.
%   \end{Version}
%   \begin{Version}{2010/10/19 v1.16}
%   \item
%     Option `startatroot' now acts globally.
%   \item
%     Option `level' also accepts names the same way as option `depth'.
%   \end{Version}
%   \begin{Version}{2010/10/25 v1.17}
%   \item
%     \cs{bookmarksetupnext} added.
%   \item
%     Using \cs{kvsetkeys} of package \xpackage{kvsetkeys}, because
%     \cs{setkeys} of package \xpackage{keyval} is not reentrant.
%     This can cause problems (unknown keys) with older versions of
%     hyperref that also uses \cs{setkeys} (found by GL).
%   \end{Version}
%   \begin{Version}{2010/11/05 v1.18}
%   \item
%     Use of \cs{pdf@ifdraftmode} of package \xpackage{pdftexcmds} for
%     the default of option \xoption{draft}.
%   \end{Version}
%   \begin{Version}{2011/03/20 v1.19}
%   \item
%     Use of \cs{dimexpr} fixed, if \hologo{eTeX} is not used.
%     (Bug found by Martin M\"unch.)
%   \item
%     Fix in documentation. Also layout options work without \hologo{eTeX}.
%   \end{Version}
%   \begin{Version}{2011/04/13 v1.20}
%   \item
%     Bug fix: \cs{BKM@SetDepth} renamed to \cs{BKM@SetDepthOrLevel}.
%   \end{Version}
%   \begin{Version}{2011/04/21 v1.21}
%   \item
%     Some support for file name and line number in error messages
%     at end of document (pdfTeX and pdfmark based drivers).
%   \end{Version}
%   \begin{Version}{2011/05/13 v1.22}
%   \item
%     Change of version 2010/11/05 v1.18 reverted, because otherwise
%     draftmode disables some \xext{aux} file entries.
%   \end{Version}
%   \begin{Version}{2011/09/19 v1.23}
%   \item
%     Some \cs{renewcommand}s changed to \cs{def} to avoid trouble
%     if the commands are not defined, because hyperref stopped early.
%   \end{Version}
%   \begin{Version}{2011/12/02 v1.24}
%   \item
%     Small optimization in \cs{BKM@toHexDigit}.
%   \end{Version}
%   \begin{Version}{2016/05/16 v1.25}
%   \item
%     Documentation updates.
%   \end{Version}
%   \begin{Version}{2016/05/17 v1.26}
%   \item
%     define \cs{pdfoutline} to allow pdftex driver to be used with Lua\TeX.
%   \end{Version}
%   \begin{Version}{2019/06/04 v1.27}
%   \item
%     unknown style options are ignored (issue 67)
%   \end{Version}

%   \begin{Version}{2019/12/03 v1.28}
%   \item
%     Documentation updates.
%   \item adjust package loading (all required packages already loaded
%     by \xpackage{hyperref}).
%   \end{Version}
%   \begin{Version}{2020-11-06 v1.29}
%   \item Adapted the dvips to avoid a clash with pgf.
%         https://github.com/pgf-tikz/pgf/issues/944
%   \item All drivers now use the new LaTeX hooks
%         and so require a format 2020-10-01 or newer. The older
%         drivers are provided as frozen versions and are used if an older
%         format is detected.
%   \item Added support for destlabel option of hyperref, https://github.com/ho-tex/bookmark/issues/1
%   \item Removed the \xoption{dvipsone} and \xoption{textures} driver.
%   \item Removed the code for option \xoption{dvipdfmx-outline-open}
%     and \cs{SpecialDvipdfmxOutlineOpen}. All dvipdfmx version should now support
%     this out-of-the-box.
%   \end{Version}
% \end{History}
%
% \PrintIndex
%
% \Finale
\endinput
|
% \end{quote}
% 不要忘记根据 shell 的要求引用这个参数(argument)。
%
% \paragraph{知生成文档。}
% 您可以同时使用 \xfile{.dtx}\ 或 \xfile{.drv}\ 来生成文档。可以通过配置文件 \xfile{ltxdoc.cfg}\ 配置该进程。
% 例如,如果您希望 A4 作为纸张格式,请将下面这行写入此文件中:
% \begin{quote}
%   \verb|\PassOptionsToClass{a4paper}{article}|
% \end{quote}
% 下面是一个如何使用 pdf\LaTeX\ 生成文档的示例:
% \begin{quote}
%\begin{verbatim}
%pdflatex bookmark.dtx
%makeindex -s gind.ist bookmark.idx
%pdflatex bookmark.dtx
%makeindex -s gind.ist bookmark.idx
%pdflatex bookmark.dtx
%\end{verbatim}
% \end{quote}
%
% \begin{thebibliography}{9}
%
% \bibitem{hyperref}
%   Sebastian Rahtz, Heiko Oberdiek:
%   \textit{The \xpackage{hyperref} package};
%   2011/04/17 v6.82g;
%   \CTANpkg{hyperref}
%
% \bibitem{currfile}
%   Martin Scharrer:
%   \textit{The \xpackage{currfile} package};
%   2011/01/09 v0.4.
%   \CTANpkg{currfile}
%
% \end{thebibliography}
%
% \begin{History}
%   \begin{Version}{2007/02/19 v0.1}
%   \item
%     First experimental version.
%   \end{Version}
%   \begin{Version}{2007/02/20 v0.2}
%   \item
%     Option \xoption{startatroot} added.
%   \item
%     Dummies for \cs{pdf(un)escape...} commands added to get
%     the package basically work for non-\hologo{pdfTeX} users.
%   \end{Version}
%   \begin{Version}{2007/02/21 v0.3}
%   \item
%     Dependency from \hologo{pdfTeX} 1.30 removed by using package
%     \xpackage{pdfescape}.
%   \end{Version}
%   \begin{Version}{2007/02/22 v0.4}
%   \item
%     \xpackage{hyperref}'s \xoption{bookmarkstype} respected.
%   \end{Version}
%   \begin{Version}{2007/03/02 v0.5}
%   \item
%     Driver options \xoption{vtex} (PDF mode), \xoption{dvipsone},
%     and \xoption{textures} added.
%   \item
%     Implementation of option \xoption{depth} completed. Division names
%     are supported, see \xpackage{hyperref}'s
%     option \xoption{bookmarksdepth}.
%   \item
%     \xpackage{hyperref}'s options \xoption{bookmarksopen},
%     \xoption{bookmarksopenlevel}, and \xoption{bookmarksdepth} respected.
%   \end{Version}
%   \begin{Version}{2007/03/03 v0.6}
%   \item
%     Option \xoption{numbered} as alias for \xpackage{hyperref}'s
%     \xoption{bookmarksnumbered}.
%   \end{Version}
%   \begin{Version}{2007/03/07 v0.7}
%   \item
%     Dependency from \hologo{eTeX} removed.
%   \end{Version}
%   \begin{Version}{2007/04/09 v0.8}
%   \item
%     Option \xoption{atend} added.
%   \item
%     Option \xoption{rgbcolor} removed.
%     \verb|rgbcolor=<r> <g> <b>| can be replaced by
%     \verb|color=[rgb]{<r>,<g>,<b>}|.
%   \item
%     Support of recent cvs version (2007-03-29) of dvipdfmx
%     that extends the \cs{special} for bookmarks to specify
%     open outline entries. Option \xoption{dvipdfmx-outline-open}
%     or \cs{SpecialDvipdfmxOutlineOpen} notify the package.
%   \end{Version}
%   \begin{Version}{2007/04/25 v0.9}
%   \item
%     The syntax of \cs{special} of dvipdfmx, if feature
%     \xoption{dvipdfmx-outline-open} is enabled, has changed.
%     Now cvs version 2007-04-25 is needed.
%   \end{Version}
%   \begin{Version}{2007/05/29 v1.0}
%   \item
%     Bug fix in code for second parameter of XYZ.
%   \end{Version}
%   \begin{Version}{2007/07/13 v1.1}
%   \item
%     Fix for pdfmark with GoToR action.
%   \end{Version}
%   \begin{Version}{2007/09/25 v1.2}
%   \item
%     pdfmark driver respects \cs{nofiles}.
%   \end{Version}
%   \begin{Version}{2008/08/08 v1.3}
%   \item
%     Package \xpackage{flags} replaced by package \xpackage{bitset}.
%     Now flags are also supported without \hologo{eTeX}.
%   \item
%     Hook for package \xpackage{hypdestopt} added.
%   \end{Version}
%   \begin{Version}{2008/09/13 v1.4}
%   \item
%     Fix for bug introduced in v1.3, package \xpackage{flags} is one-based,
%     but package \xpackage{bitset} is zero-based. Thus options \xoption{bold}
%     and \xoption{italic} are wrong in v1.3. (Daniel M\"ullner)
%   \end{Version}
%   \begin{Version}{2009/08/13 v1.5}
%   \item
%     Except for driver options the other options are now local options.
%     This resolves a problem with KOMA-Script v3.00 and its option \xoption{open}.
%   \end{Version}
%   \begin{Version}{2009/12/06 v1.6}
%   \item
%     Use of package \xpackage{atveryend} for drivers \xoption{pdftex}
%     and \xoption{pdfmark}.
%   \end{Version}
%   \begin{Version}{2009/12/07 v1.7}
%   \item
%     Use of package \xpackage{atveryend} fixed.
%   \end{Version}
%   \begin{Version}{2009/12/17 v1.8}
%   \item
%     Support of \xpackage{hyperref} 2009/12/17 v6.79v for \hologo{XeTeX}.
%   \end{Version}
%   \begin{Version}{2010/03/30 v1.9}
%   \item
%     Package name in an error message fixed.
%   \end{Version}
%   \begin{Version}{2010/04/03 v1.10}
%   \item
%     Option \xoption{style} and macro \cs{bookmarkdefinestyle} added.
%   \item
%     Hook support with option \xoption{addtohook} added.
%   \item
%     \cs{bookmarkget} added.
%   \end{Version}
%   \begin{Version}{2010/04/04 v1.11}
%   \item
%     Bug fix (introduced in v1.10).
%   \end{Version}
%   \begin{Version}{2010/04/08 v1.12}
%   \item
%     Requires \xpackage{ltxcmds} 2010/04/08.
%   \end{Version}
%   \begin{Version}{2010/07/23 v1.13}
%   \item
%     Support for \xclass{memoir}'s \cs{booknumberline} added.
%   \end{Version}
%   \begin{Version}{2010/09/02 v1.14}
%   \item
%     (Local) options \xoption{draft} and \xoption{final} added.
%   \end{Version}
%   \begin{Version}{2010/09/25 v1.15}
%   \item
%     Fix for option \xoption{dvipdfmx-outline-open}.
%   \item
%     Option \xoption{dvipdfmx-outline-open} is set automatically,
%     if XeTeX $\geq$ 0.9995 is detected.
%   \end{Version}
%   \begin{Version}{2010/10/19 v1.16}
%   \item
%     Option `startatroot' now acts globally.
%   \item
%     Option `level' also accepts names the same way as option `depth'.
%   \end{Version}
%   \begin{Version}{2010/10/25 v1.17}
%   \item
%     \cs{bookmarksetupnext} added.
%   \item
%     Using \cs{kvsetkeys} of package \xpackage{kvsetkeys}, because
%     \cs{setkeys} of package \xpackage{keyval} is not reentrant.
%     This can cause problems (unknown keys) with older versions of
%     hyperref that also uses \cs{setkeys} (found by GL).
%   \end{Version}
%   \begin{Version}{2010/11/05 v1.18}
%   \item
%     Use of \cs{pdf@ifdraftmode} of package \xpackage{pdftexcmds} for
%     the default of option \xoption{draft}.
%   \end{Version}
%   \begin{Version}{2011/03/20 v1.19}
%   \item
%     Use of \cs{dimexpr} fixed, if \hologo{eTeX} is not used.
%     (Bug found by Martin M\"unch.)
%   \item
%     Fix in documentation. Also layout options work without \hologo{eTeX}.
%   \end{Version}
%   \begin{Version}{2011/04/13 v1.20}
%   \item
%     Bug fix: \cs{BKM@SetDepth} renamed to \cs{BKM@SetDepthOrLevel}.
%   \end{Version}
%   \begin{Version}{2011/04/21 v1.21}
%   \item
%     Some support for file name and line number in error messages
%     at end of document (pdfTeX and pdfmark based drivers).
%   \end{Version}
%   \begin{Version}{2011/05/13 v1.22}
%   \item
%     Change of version 2010/11/05 v1.18 reverted, because otherwise
%     draftmode disables some \xext{aux} file entries.
%   \end{Version}
%   \begin{Version}{2011/09/19 v1.23}
%   \item
%     Some \cs{renewcommand}s changed to \cs{def} to avoid trouble
%     if the commands are not defined, because hyperref stopped early.
%   \end{Version}
%   \begin{Version}{2011/12/02 v1.24}
%   \item
%     Small optimization in \cs{BKM@toHexDigit}.
%   \end{Version}
%   \begin{Version}{2016/05/16 v1.25}
%   \item
%     Documentation updates.
%   \end{Version}
%   \begin{Version}{2016/05/17 v1.26}
%   \item
%     define \cs{pdfoutline} to allow pdftex driver to be used with Lua\TeX.
%   \end{Version}
%   \begin{Version}{2019/06/04 v1.27}
%   \item
%     unknown style options are ignored (issue 67)
%   \end{Version}

%   \begin{Version}{2019/12/03 v1.28}
%   \item
%     Documentation updates.
%   \item adjust package loading (all required packages already loaded
%     by \xpackage{hyperref}).
%   \end{Version}
%   \begin{Version}{2020-11-06 v1.29}
%   \item Adapted the dvips to avoid a clash with pgf.
%         https://github.com/pgf-tikz/pgf/issues/944
%   \item All drivers now use the new LaTeX hooks
%         and so require a format 2020-10-01 or newer. The older
%         drivers are provided as frozen versions and are used if an older
%         format is detected.
%   \item Added support for destlabel option of hyperref, https://github.com/ho-tex/bookmark/issues/1
%   \item Removed the \xoption{dvipsone} and \xoption{textures} driver.
%   \item Removed the code for option \xoption{dvipdfmx-outline-open}
%     and \cs{SpecialDvipdfmxOutlineOpen}. All dvipdfmx version should now support
%     this out-of-the-box.
%   \end{Version}
% \end{History}
%
% \PrintIndex
%
% \Finale
\endinput
|
% \end{quote}
% 不要忘记根据 shell 的要求引用这个参数(argument)。
%
% \paragraph{知生成文档。}
% 您可以同时使用 \xfile{.dtx}\ 或 \xfile{.drv}\ 来生成文档。可以通过配置文件 \xfile{ltxdoc.cfg}\ 配置该进程。
% 例如,如果您希望 A4 作为纸张格式,请将下面这行写入此文件中:
% \begin{quote}
%   \verb|\PassOptionsToClass{a4paper}{article}|
% \end{quote}
% 下面是一个如何使用 pdf\LaTeX\ 生成文档的示例:
% \begin{quote}
%\begin{verbatim}
%pdflatex bookmark.dtx
%makeindex -s gind.ist bookmark.idx
%pdflatex bookmark.dtx
%makeindex -s gind.ist bookmark.idx
%pdflatex bookmark.dtx
%\end{verbatim}
% \end{quote}
%
% \begin{thebibliography}{9}
%
% \bibitem{hyperref}
%   Sebastian Rahtz, Heiko Oberdiek:
%   \textit{The \xpackage{hyperref} package};
%   2011/04/17 v6.82g;
%   \CTANpkg{hyperref}
%
% \bibitem{currfile}
%   Martin Scharrer:
%   \textit{The \xpackage{currfile} package};
%   2011/01/09 v0.4.
%   \CTANpkg{currfile}
%
% \end{thebibliography}
%
% \begin{History}
%   \begin{Version}{2007/02/19 v0.1}
%   \item
%     First experimental version.
%   \end{Version}
%   \begin{Version}{2007/02/20 v0.2}
%   \item
%     Option \xoption{startatroot} added.
%   \item
%     Dummies for \cs{pdf(un)escape...} commands added to get
%     the package basically work for non-\hologo{pdfTeX} users.
%   \end{Version}
%   \begin{Version}{2007/02/21 v0.3}
%   \item
%     Dependency from \hologo{pdfTeX} 1.30 removed by using package
%     \xpackage{pdfescape}.
%   \end{Version}
%   \begin{Version}{2007/02/22 v0.4}
%   \item
%     \xpackage{hyperref}'s \xoption{bookmarkstype} respected.
%   \end{Version}
%   \begin{Version}{2007/03/02 v0.5}
%   \item
%     Driver options \xoption{vtex} (PDF mode), \xoption{dvipsone},
%     and \xoption{textures} added.
%   \item
%     Implementation of option \xoption{depth} completed. Division names
%     are supported, see \xpackage{hyperref}'s
%     option \xoption{bookmarksdepth}.
%   \item
%     \xpackage{hyperref}'s options \xoption{bookmarksopen},
%     \xoption{bookmarksopenlevel}, and \xoption{bookmarksdepth} respected.
%   \end{Version}
%   \begin{Version}{2007/03/03 v0.6}
%   \item
%     Option \xoption{numbered} as alias for \xpackage{hyperref}'s
%     \xoption{bookmarksnumbered}.
%   \end{Version}
%   \begin{Version}{2007/03/07 v0.7}
%   \item
%     Dependency from \hologo{eTeX} removed.
%   \end{Version}
%   \begin{Version}{2007/04/09 v0.8}
%   \item
%     Option \xoption{atend} added.
%   \item
%     Option \xoption{rgbcolor} removed.
%     \verb|rgbcolor=<r> <g> <b>| can be replaced by
%     \verb|color=[rgb]{<r>,<g>,<b>}|.
%   \item
%     Support of recent cvs version (2007-03-29) of dvipdfmx
%     that extends the \cs{special} for bookmarks to specify
%     open outline entries. Option \xoption{dvipdfmx-outline-open}
%     or \cs{SpecialDvipdfmxOutlineOpen} notify the package.
%   \end{Version}
%   \begin{Version}{2007/04/25 v0.9}
%   \item
%     The syntax of \cs{special} of dvipdfmx, if feature
%     \xoption{dvipdfmx-outline-open} is enabled, has changed.
%     Now cvs version 2007-04-25 is needed.
%   \end{Version}
%   \begin{Version}{2007/05/29 v1.0}
%   \item
%     Bug fix in code for second parameter of XYZ.
%   \end{Version}
%   \begin{Version}{2007/07/13 v1.1}
%   \item
%     Fix for pdfmark with GoToR action.
%   \end{Version}
%   \begin{Version}{2007/09/25 v1.2}
%   \item
%     pdfmark driver respects \cs{nofiles}.
%   \end{Version}
%   \begin{Version}{2008/08/08 v1.3}
%   \item
%     Package \xpackage{flags} replaced by package \xpackage{bitset}.
%     Now flags are also supported without \hologo{eTeX}.
%   \item
%     Hook for package \xpackage{hypdestopt} added.
%   \end{Version}
%   \begin{Version}{2008/09/13 v1.4}
%   \item
%     Fix for bug introduced in v1.3, package \xpackage{flags} is one-based,
%     but package \xpackage{bitset} is zero-based. Thus options \xoption{bold}
%     and \xoption{italic} are wrong in v1.3. (Daniel M\"ullner)
%   \end{Version}
%   \begin{Version}{2009/08/13 v1.5}
%   \item
%     Except for driver options the other options are now local options.
%     This resolves a problem with KOMA-Script v3.00 and its option \xoption{open}.
%   \end{Version}
%   \begin{Version}{2009/12/06 v1.6}
%   \item
%     Use of package \xpackage{atveryend} for drivers \xoption{pdftex}
%     and \xoption{pdfmark}.
%   \end{Version}
%   \begin{Version}{2009/12/07 v1.7}
%   \item
%     Use of package \xpackage{atveryend} fixed.
%   \end{Version}
%   \begin{Version}{2009/12/17 v1.8}
%   \item
%     Support of \xpackage{hyperref} 2009/12/17 v6.79v for \hologo{XeTeX}.
%   \end{Version}
%   \begin{Version}{2010/03/30 v1.9}
%   \item
%     Package name in an error message fixed.
%   \end{Version}
%   \begin{Version}{2010/04/03 v1.10}
%   \item
%     Option \xoption{style} and macro \cs{bookmarkdefinestyle} added.
%   \item
%     Hook support with option \xoption{addtohook} added.
%   \item
%     \cs{bookmarkget} added.
%   \end{Version}
%   \begin{Version}{2010/04/04 v1.11}
%   \item
%     Bug fix (introduced in v1.10).
%   \end{Version}
%   \begin{Version}{2010/04/08 v1.12}
%   \item
%     Requires \xpackage{ltxcmds} 2010/04/08.
%   \end{Version}
%   \begin{Version}{2010/07/23 v1.13}
%   \item
%     Support for \xclass{memoir}'s \cs{booknumberline} added.
%   \end{Version}
%   \begin{Version}{2010/09/02 v1.14}
%   \item
%     (Local) options \xoption{draft} and \xoption{final} added.
%   \end{Version}
%   \begin{Version}{2010/09/25 v1.15}
%   \item
%     Fix for option \xoption{dvipdfmx-outline-open}.
%   \item
%     Option \xoption{dvipdfmx-outline-open} is set automatically,
%     if XeTeX $\geq$ 0.9995 is detected.
%   \end{Version}
%   \begin{Version}{2010/10/19 v1.16}
%   \item
%     Option `startatroot' now acts globally.
%   \item
%     Option `level' also accepts names the same way as option `depth'.
%   \end{Version}
%   \begin{Version}{2010/10/25 v1.17}
%   \item
%     \cs{bookmarksetupnext} added.
%   \item
%     Using \cs{kvsetkeys} of package \xpackage{kvsetkeys}, because
%     \cs{setkeys} of package \xpackage{keyval} is not reentrant.
%     This can cause problems (unknown keys) with older versions of
%     hyperref that also uses \cs{setkeys} (found by GL).
%   \end{Version}
%   \begin{Version}{2010/11/05 v1.18}
%   \item
%     Use of \cs{pdf@ifdraftmode} of package \xpackage{pdftexcmds} for
%     the default of option \xoption{draft}.
%   \end{Version}
%   \begin{Version}{2011/03/20 v1.19}
%   \item
%     Use of \cs{dimexpr} fixed, if \hologo{eTeX} is not used.
%     (Bug found by Martin M\"unch.)
%   \item
%     Fix in documentation. Also layout options work without \hologo{eTeX}.
%   \end{Version}
%   \begin{Version}{2011/04/13 v1.20}
%   \item
%     Bug fix: \cs{BKM@SetDepth} renamed to \cs{BKM@SetDepthOrLevel}.
%   \end{Version}
%   \begin{Version}{2011/04/21 v1.21}
%   \item
%     Some support for file name and line number in error messages
%     at end of document (pdfTeX and pdfmark based drivers).
%   \end{Version}
%   \begin{Version}{2011/05/13 v1.22}
%   \item
%     Change of version 2010/11/05 v1.18 reverted, because otherwise
%     draftmode disables some \xext{aux} file entries.
%   \end{Version}
%   \begin{Version}{2011/09/19 v1.23}
%   \item
%     Some \cs{renewcommand}s changed to \cs{def} to avoid trouble
%     if the commands are not defined, because hyperref stopped early.
%   \end{Version}
%   \begin{Version}{2011/12/02 v1.24}
%   \item
%     Small optimization in \cs{BKM@toHexDigit}.
%   \end{Version}
%   \begin{Version}{2016/05/16 v1.25}
%   \item
%     Documentation updates.
%   \end{Version}
%   \begin{Version}{2016/05/17 v1.26}
%   \item
%     define \cs{pdfoutline} to allow pdftex driver to be used with Lua\TeX.
%   \end{Version}
%   \begin{Version}{2019/06/04 v1.27}
%   \item
%     unknown style options are ignored (issue 67)
%   \end{Version}

%   \begin{Version}{2019/12/03 v1.28}
%   \item
%     Documentation updates.
%   \item adjust package loading (all required packages already loaded
%     by \xpackage{hyperref}).
%   \end{Version}
%   \begin{Version}{2020-11-06 v1.29}
%   \item Adapted the dvips to avoid a clash with pgf.
%         https://github.com/pgf-tikz/pgf/issues/944
%   \item All drivers now use the new LaTeX hooks
%         and so require a format 2020-10-01 or newer. The older
%         drivers are provided as frozen versions and are used if an older
%         format is detected.
%   \item Added support for destlabel option of hyperref, https://github.com/ho-tex/bookmark/issues/1
%   \item Removed the \xoption{dvipsone} and \xoption{textures} driver.
%   \item Removed the code for option \xoption{dvipdfmx-outline-open}
%     and \cs{SpecialDvipdfmxOutlineOpen}. All dvipdfmx version should now support
%     this out-of-the-box.
%   \end{Version}
% \end{History}
%
% \PrintIndex
%
% \Finale
\endinput
